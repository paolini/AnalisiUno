\documentclass{registro}
\RequirePackage{amsmath}
\RequirePackage{amsfonts}
\RequirePackage{amssymb}
\RequirePackage[latin1]{inputenc}

\newcommand{\R}{\mathbb R}
\newcommand{\Q}{\mathbb Q}
\newcommand{\Z}{\mathbb Z}
\newcommand{\N}{\mathbb N}
\renewcommand{\epsilon}{\varepsilon}

\DeclareMathOperator{\arctg}{arctg}

\annoAccademico{2009/2010}
\facolta{Matematica}
\insegnamento{Analisi Matematica II}
\prof{Emanuele Paolini}
\inquadramento{Analisi Matematica}

\begin{document}
%% data / n. ore / argomenti

\esercitazione{6.10.2009}{1}{
Prodotto scalare e norma canonica su $\R^n$. Disuguaglianza di
Cauchy-Schwarz. Sub-additivit\`a della norma.
}

\esercitazione{13.10.2009}{2}{
Distanza su $\R^n$, disuguaglianza triangolare. Elementi di topologia:
intorno circolare, aperto, punto interno, intorno, parte interna,
chiuso, chiusura, punto di aderenza, punto esterno, frontiera o bordo,
punto di accumulazione, punto isolato. Esercizi.
}

\esercitazione{20.10.2009}{1}{
% un ora c'era assemblea...
Come dimostrare l'esistenza di un limite in pi\`u variabili. Esercizi
sui limiti in due variabili.
}

\esercitazione{27.10.2009}{2}{
Teorema di collegamento tra limiti di funzione e limiti di
successione. Restrizione di un limite in pi\`u variabili.
Come dimostrare la non esistenza dei limiti in pi\`u variabili. Esercizi.
}

\esercitazione{3.10.2009}{2}{
Differenziabilit\`a, differenziale. Derivate parziali, derivate
direzionali, gradiente. Derivate seconde, teorema di Schwarz. Esercizi.
}

\esercitazione{10.11.2009}{2}{
Massimi e minimi. Il teorema di Bolzano-Weierstrass. Il teorema di
Weierstrass. Condizione necessaria al primo ordine per i punti di
massimo e minimo. Esercizi.
}

\esercitazione{17.11.2009}{2}{
Esercizi su massimi e minimi assoluti, punti critici, matrice hessiana.
}

\esercitazione{24.11.2009}{2}{
Punti critici con hessiano nullo.
Videoproiezione. Insiemi di livello. Il gradiente \`e ortogonale agli
insiemi di livello.
}

\esercitazione{1.12.2009}{2}{
Disegnare gli insiemi di livello delle funzioni:
\[
f(x,y) = x^4 - 2x^2 +2y^2 
\qquad 
f(x,y) = x^3 - x^2 - y^4 - y^2.
\]
}

%% il 15.12 ero ammalato

\esercitazione{18.12.2009}{1}{
Una fabbrica di vernici produce vernice rossa e blu tramite due
macchinari $A$ e $B$. La macchina $A$, se impiegata per un tempo $x$,
produce $4x$ barili di vernice rossa e $x$ di vernice blu ad un costo
pari a $x^2+x$ euro. La macchina $B$, se impiegata per un tempo $y$,
produce $x$ barili di vernice rossa e $2y$ di vernice blu, ad un costo
pari a $20y$ euro. Determinare l'impiego ottimale $x,y\ge 0$ delle macchine
$A,B$ in modo da produrre almeno $12$ barili di vernice rossa e $6$ di
vernice blu al costo minore possibile.   

Per casa.
Studiare la continuit�, derivabilit� e differenziabilit� della funzione
\[
  f(x,y) = \begin{cases}
  \frac{x^\alpha + xy^\alpha}{x^2+y^2} & \text{se $(x,y)\neq (0,0)$}\\
  0 & \text{se $(x,y)=(0,0)$}
\end{cases}
\]
al variare del parametro $\alpha\in \R$.

Determinare la natura dei punti critici della funzione
\[
  f(x,y) = 3x^2y - 2x^3-y^2.
\]
}

\esercitazione{12.1.2010}{2}{
Svolgimento prima prova scritta preliminare.
}

\esercitazione{15.1.2010}{2}{
Soluzioni della prima prova scritta preliminare.
}

\esercitazione{19.1.2010}{2}{
Convergenza puntuale e uniforme delle successioni di funzioni. 
Studiare la convergenza puntuale ed uniforme delle seguenti
successioni
\[
  f_k(x)=x^k,\qquad
  f_k(x)=x e^{-kx^2},\qquad
  f_k(x)=kxe^{-k^2x^2}.
\]

Consegna dei compitini.
}

\esercitazione{17.2.2010}{1}{
Studiare la convergenza puntuale ed uniforme delle seguenti
successioni di funzioni  
\begin{gather*}
  f_k(x) = \frac{1}{1+(x-k)^2},\qquad
  f_k(x) = \frac{(\log x) - 1}{kx^2},\\
  f_k(x) = \frac{1}{1+kx^2},\qquad
  f_k(x) = \frac{x}{1+kx^2},\qquad
  f_k(x) = \frac{x \sin(kx)}{1+kx^2}
\end{gather*}
}

\esercitazione{4.3.2010}{2}{
Passaggio al limite sotto il segno di integrale. Dimostrare che
\[
  \lim_{k\to \infty} \int_0^1 \sin^k(x)\, dx =0,
\qquad
  \lim_{k\to \infty} \int_0^1 \sqrt[k]{\sin(x)}\, dx = 1.
\]

Serie di funzioni. Studiare la convergenza puntuale, uniforme e totale della
serie di funzioni
\[
   \sum_n \left(\frac{x+1}{x^2+1}\right)^n.
\]
}

\esercitazione{11.3.2010}{2}{
Si consideri la funzione
\[
  f(x) = \sum_{n=1}^\infty \frac{1-e^{-nx^2}}{n^2}.
\]
Dimostrare che la funzione � definita per ogni $x\in \R$, che �
derivabile e calcolare $f'(\sqrt{ \log 2})$. Calcolare 
\[
  \sum_{k=1}^\infty \frac{1}{kx^k}.
\]
Serie di potenze. Raggio di convergenza. Serie derivata. Dimostrare
che la serie
\[
  \sum_{k=1}^\infty \frac{1}{kx^k}
\]
converge uniformemente sull'intervallo $[-1,0]$ (cenni).
}

\esercitazione{18.3.2010}{2}{
Equazioni differenziali a variabili separabili.
\[
  y' = -2xy^2,\qquad y'=x\sqrt{1-y^2}.
\]
Equazioni lineari del prim'ordine.
\[
\begin{cases}
 y'-2y=1,\\
 y(0)=1;
\end{cases}
\qquad
\begin{cases}
 y'+2xy=x^3,\\
 y(0)=1.
\end{cases}
\]
Equazioni di Bernoulli.
}

\esercitazione{25.3.2010}{2}{
Equazioni omogenee. Esercizi su equazioni Omogenee ed equazioni di Bernoulli.
}

\esercitazione{8.4.2010}{3}{
Seconda prova scritta preliminare.
}

\esercitazione{15.4.2010}{2}{
Equazioni differenziali esatte:
\[
  y' = \frac{y-2x}{4y^3-x}.
\]
Equazioni di Clairaut:
\[
  y= xy' - 2 \sqrt{y'} + 1
\]
Equazioni del secondo ordine autonome
\[
\begin{cases}
 y'' = (y')^3 y^2\\
 y(0) = 1 \\
 y'(0) = -3.
\end{cases}
\]
}

\esercitazione{21.4.2010}{1}{
Studio qualitativo delle soluzioni di equazioni differenziali normali
del prim'ordine. Esercizio:
\[
  y' = x \log y.
\]
Criterio: le soluzioni escono dai compatti. Asintoti
orizzontali. Asintoti verticali.
}

\esercitazione{22.4.2010}{2}{
Ancora studio qualitativo equazioni differenziali. Criterio di
confronto. Esercizi:
\[
 \begin{cases}
   y' = y (y-\arctg x)^3\\
   y(1)=\frac 1 2,
 \end{cases} \qquad 
y'=(y-1)\arctg y,\qquad
y'=\log y,\]
\[
y' = y - 4 y^3, \qquad
\begin{cases}
y'=1-x^2y^2\\
y(0)=y_0.
\end{cases}
\] 
}

\esercitazione{28.4.2010}{1}{
Forme differenziali. Forme esatte e chiuse. Integrale di linea. Esempi.
}

\esercitazione{29.4.2010}{2}{ Ancora esercizi sulle forme
  differenziali. Integrali di linea delle forme chiuse ma non
  esatte. Avvolgimento attorno alle singolarit\`a. Esempi.  }

\esercitazione{6.5.2010}{2}{
Esercizi sugli integrali multipli. Formule di riduzione. Cambio
dell'ordine di integrazione.
}

\end{document}
