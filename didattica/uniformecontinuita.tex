\documentclass[italian,a4paper]{article}
\usepackage{babel}
\usepackage{a4}
\usepackage{amssymb}
\usepackage{latexsym}
\newcommand{\R}{\mathbf{R}}
\renewcommand{\epsilon}{\varepsilon}
%%title{Uniforme continuit\`a}
\title{\bf Uniforme continuit\`a}
\author{E. Paolini}
\date{17 marzo 2003}

\newtheorem{theorem}{Teorema}
\newtheorem{proposition}[theorem]{Proposizione}
\newtheorem{example}[theorem]{Esempio}
\newtheorem{corollary}[theorem]{Corollario}
\newtheorem{lemma}[theorem]{Lemma}
\newtheorem{definition}[theorem]{Definizione}
\newtheorem{exercise}[theorem]{Esercizio}
\newenvironment{proof}
        {%
%       \begin{list}%
                {}%
                {%
                %\setlength{\rightmargin}{\leftmargin}%
                }%
%       \item%
        \noindent% 
        {\it Dimostrazione:}\\%
        }%
        {%
        \hfill$\square$%
%       \end{list}%
        }


\begin{document}
\maketitle
\section{Funzioni uniformemente continue, Lipschitziane, $\alpha$-H\"o{}lderiane}
\begin{definition}
Una funzione $f\colon A\to \R$ 
\begin{itemize}
\item
si dice \emph{continua} se
\[
\forall x \in A\ \forall \epsilon>0\ \exists\delta\ \forall y\in A\quad
\vert x-y\vert < \delta \Rightarrow \vert f(x)-f(y)\vert <\epsilon
\]
\item
si dice \emph{uniformemente continua} se
\[
	\forall \epsilon>0\ \exists \delta\ 
	\forall x,y\in A\quad
	\vert x-y\vert < \delta \Rightarrow \vert f(x)-f(y)\vert <\epsilon
\]
\item
si dice \emph{Lipschitziana} se esiste
una costante $L>0$ tale che
\[
	\forall x,y\in A\quad
	\vert f(x) - f(y)\vert \le L \vert x-y\vert
\]
\item
si dice \emph{$\alpha$-H\"o{}lderiana} (con $\alpha>0$) se esiste
una costante $C>0$ tale che
\[
	\forall x,y\in A\quad
	\vert f(x) - f(y)\vert \le C \vert x-y\vert^\alpha
\] 
\end{itemize}
\end{definition}

Chiaramente ogni funzione uniformemente continua \`e anche continua.
Si noti che $1$-H\"o{}lderiano e Lipschitziano sono sinonimi. 
Notiamo anche che se $f$ \`e $\alpha$-H\"o{}lderiana con $\alpha>1$ si ha
\[
	\frac{\vert f(x)-f(x_0)\vert}{\vert x-x_0\vert}
	\le C \vert x-x_0\vert^{\alpha-1}\to 0\quad\mathrm{per}\
	x\to x_0
\]
e quindi $f'(x_0)=0$ per ogni $x_0\in A$. Dunque se $A$ \`e un
intervallo allora $f$ \`e costante. Per questo le funzioni
$\alpha$-H\"o{}lderiane sono interessanti solo quando $\alpha < 1$.

\begin{theorem}
Se $f\colon [a,b]\to \R$ \`e $\alpha$-H\"o{}lderiana per un certo
$\alpha>0$ allora $f$ \`e $\beta$-H\"o{}lderiana per ogni $\beta\le \alpha$.
In particolare se $f$ \`e Lipschitziana allora $f$ \`e
$\alpha$-H\"o{}lderiana per ogni $\alpha<1$.
\end{theorem}
\begin{proof}
Supponiamo dunque che valga 
\[
	\vert f(x)-f(y)\vert \le C \vert x-y\vert^\alpha\quad
	\forall x,y\in [a,b].
\]
Dato $\beta\le\alpha$ poniamo $C'=C\vert b-a\vert^{\alpha-\beta}$.
Allora se $x,y\in[a,b]$ si ha
\[
	C \vert x-y\vert^\alpha \le C' \vert x-y\vert^\beta
\]
e dunque $f$ \`e $\beta$-H\"o{}lderiana.
\end{proof}

\begin{theorem}
Per verificare che una funzione $f\colon A\to \R$ \`e uniformemente
continua \`e sufficiente mostrare che esiste una funzione $G\colon \R^+ \to \R$
che
\begin{itemize}
\item $\forall x,y\in A \quad \vert f(x)-f(y)\vert \le G(\vert x-y\vert)$;
\item $\lim_{t\to 0^+} G(t)=0$.
\end{itemize}
\end{theorem}
\begin{proof}
Essendo $G(t)\to 0$ si ha 
\[
	\forall \epsilon>0\  \exists \delta\quad 
	t<\delta \Rightarrow G(t)<\epsilon
\]
dunque
\[
	\forall \epsilon>0\ \exists \delta\ \forall x,y\in A\quad
	\vert x-y \vert <\delta \Rightarrow \vert f(x)-f(y)\vert \le
	G(\vert x-y\vert)< \epsilon
\]
cio\`e $f$ \`e uniformemente continua.
\end{proof}

Se ne deduce facilmente che ogni funzione Lipschitziana o
$\alpha$-H\"o{}lderiana \`e uniformemente continua.

\begin{theorem}
Per verificare che una funzione $f\colon A\to \R$ non \`e
uniformemente continua \`e sufficiente trovare due successioni
$x_k,y_k\in A$ tali che
\[
	\lim_{k\to\infty} \vert x_k - y_k\vert =0
\]
ma
\[
	\lim_{k\to \infty} \vert f(x_k)-f(y_k)\vert > 0.
\]
\end{theorem}
\begin{proof}
Siccome $\vert f(x_k)-f(y_k)\vert$ \`e una successione positiva ma non
infinitesima possiamo trovare un $\epsilon>0$ e una sottosuccessione
indicizzata tramite $k_j$ tale che $\vert
f(x_{k_j})-f(y_{k_j})\vert \ge \epsilon$ per ogni $j$.
D'altra parte essendo $\vert x_k - y_k\vert \to 0$ si
ha
\[
	\forall \delta>0\ \exists k \quad \vert x_k- y_k\vert < \delta
\]
e quindi per ogni $\delta>0$ \`e possibile trovare $k_j$ tale che
posto $x=x_{k_j}$ e $y=y_{k_j}$ si abbia $\vert f(x)-f(y)\vert \ge
\epsilon$ mentre $\vert x - y\vert <\delta$.
Dunque $f$ verifica la negazione della propriet\`a di uniforme continuit\`a
\[
\exists \epsilon>0\ \forall \delta>0\ \exists x,y\in A
\quad \vert x-y\vert < \delta\ \mathrm{ma}\ \vert f(x)-f(y)\vert \ge \epsilon.
\]
\end{proof}

\section{Esercizi}
\begin{enumerate}
\item
Mostrare che $f(x)=x^2$ non \`e uniformemente continua.
\item
Mostrare che $f(x)=1/x$ non \`e uniformemente continua.
\item
Mostrare che $f(x)=\sin(1/x)$ non \`e uniformemente continua.
\item
Mostrare che $f(x)=\vert x \vert$ \`e Lipschitziana.
\item
Mostrare che $f(x)=\sqrt{\vert x\vert}$ \`e $1/2$-H\"o{}lderiana ma
non Lipschitziana.
\item
Trovare una funzione uniformemente continua ma non
$\alpha$-H\"o{}lderiana per alcun $\alpha>0$.
\end{enumerate}

\section{Successioni di Cauchy}

\begin{definition}
Una successione $x_k$ si dice \emph{di Cauchy} se vale la seguente
propriet\`a
\[
	\forall \epsilon>0\ \exists N\ 
	\forall k,j\ge N\quad \vert x_k-x_j\vert < \epsilon
\]
\end{definition}

\begin{theorem}
Ogni successione convergente \`e di Cauchy.
\end{theorem}
\begin{proof}
Sia $x_k\to x$ una successione convergente. Dunque vale
\[
	\forall \epsilon>0\ \exists N\ 
	\forall k\ge N \quad \vert x_k - x \vert < \epsilon
\]
e quindi presi $k,j\ge N$ si ha 
\[
	\vert x_k - x_j\vert \le \vert x_k - x\vert + \vert x-
	x_j\vert
	< 2 \epsilon
\]
cio\`e
\[
	\forall \epsilon>0\ \exists N\ 
	\forall k,j\ge N \quad \vert x_k - x_j\vert < 2\epsilon
\]
che \`e equivalente alla definizione delle successioni di Cauchy.
\end{proof}

\begin{theorem}
Ogni successione di Cauchy converge.
\end{theorem}
\begin{proof}
Data una successione di Cauchy $x_k$ definiamo
\[
	A=\{x\in\R \colon \exists N\ \forall k\ge N\quad x_k\le x\}.
\]
Chiaramente se $x\in A$ e $y\ge x$ anche $y\in A$. Dunque $A$ \`e un
intervallo illimitato superiormente.
Se $A$ fosse anche illimitato inferiormente (cio\`e $A=\R$) si avrebbe 
\[
	\forall M\  \exists N\ \forall k\ge N\quad x_k \le -M
\]
che significa $x_k\to -\infty$. Questo significa in particolare che
dato qualunque $\epsilon>0$, qualunque $N$ e qualunque $j\ge N$ 
esiste $k\ge N$ tale che
$x_k< x_j - \epsilon$ cio\`e $\vert x_k - x_j\vert > \epsilon$. Ma
questo va contro l'ipotesi che $x_k$ sia di Cauchy.

Dunque $A$ \`e inferiormente limitato e quindi ammette estremo
inferiore. Posto $a=\inf A$ si ha
\[
	\forall \epsilon>0\ \exists x\in A\quad a+\epsilon> x
\]
e quindi $a+\epsilon \in A$ per ogni $\epsilon>0$. Inoltre per ogni
$\epsilon>0$ si ha $a-\epsilon\not\in A$.
Da $a+\epsilon\in A$ ricaviamo che
\[
	\exists N_1\ \forall k\ge N_1 \quad x_k \le a+\epsilon
\]
e da $a-\epsilon \not \in A$ troviamo
\[
	\forall N\ \exists j \ge N \quad x_j \ge a-\epsilon
\]
dalla propriet\`a di Cauchy si ha invece
\[
	\exists N_2\ \forall k,j\ge N_2 \quad x_j < x_k  + \epsilon.
\]
Mettendo assieme queste propriet\`a e scegliendo $N=\max \{N_1,N_2\}$
si ottiene
\[
	\forall \epsilon>0\ \exists N\ 
	\exists j\ge N\ \forall k\ge N\quad a-\epsilon< x_j <
	x_k+\epsilon
	\le a + 2\epsilon
\]
da cui
\[
	\forall \epsilon>0\ \exists N\ \forall k\ge N
	a-2\epsilon < x_k \le a+\epsilon 
\]
che significa 
\[
	\lim_{k\to\infty} x_k =a.
\]
\end{proof}

\begin{theorem}
Se $f\colon A\to \R$ \`e uniformemente continua e $x_k$ \`e una
successione di Cauchy in $A$ allora la successione $y_k=f(x_k)$ \`e di Cauchy.
\end{theorem}
\begin{proof}
Sia $x_k$ una successione di Cauchy e sia $f$ uniformemente continua.
Dall'uniforme continuit\`a di $f$ 
\[
	\forall \epsilon>0\ \exists \delta \quad \vert x-y\vert <
	\delta
	\Rightarrow \vert f(x)-f(y)\vert <\epsilon
\]
troviamo
\[
	\forall \epsilon>0\ \exists \delta_\epsilon\quad
	\vert x_k - x_j\vert < \delta \Rightarrow \vert y_k - y_j\vert <\epsilon.
\]
Essendo $x_k$ di Cauchy si ha
\[
	\forall \epsilon>0\ \exists N\ \forall k,j\ge N\quad
	\vert x_k -x_j\vert <\delta_\epsilon
\]
da cui
\[
	\forall \epsilon>0\ \exists N\ \forall k,j\ge N\quad
	\vert y_k - y_j\vert < \epsilon
\]
cio\`e $y_k$ \`e di Cauchy.
\end{proof}

\section{Estensioni}
\begin{theorem}
Sia $f\colon \left]a,b\right]\to \R$ una funzione uniformemente continua.
Allora esiste finito il limite
\[
	\lim_{x\to a^+} f(x).
\]
\end{theorem}
\begin{proof}
Sia $x_k> a$ una qualunque successione convergente ad $a$.
Allora $x_k$ \`e una successione di Cauchy e quindi $f(x_k)$ \`e a sua
volta una successione di Cauchy che quindi converge ad un certo valore
$l$. D'altra parte il valore $l$ trovato non dipende dalla successione
scelta.
Infatti prese 
$x_k$ e $y_k$ due diverse successioni convergenti ad $a$, 
sappiamo che $f(x_k)\to l_1$ e $f(y_k)\to l_2$.
Ma allora anche la successione
\[
	z_k= \left\{\begin{array}{ll}
		x_k & \mathrm{per}\ k\ \mathrm{pari}\\
		y_k & \mathrm{per}\ k\ \mathrm{dispari}\\
		    \end{array}\right.
\]
converge ad $a$ e quindi anche $f(z_k)$ converge.
Ma i termini pari di $f(z_k)$ convergono a $l_1$ e i termini dispari
convergono a $l_2$. Dunque per l'unicit\`a del limite $l_1=l_2$.
In conclusione esiste $l\in\R$ tale che per qualunque successione
$x_k\to a^+$ si ha $f(x_k)\to l$ e quindi 
\[
	\lim_{x\to a^+} f(x)=l.
\]
\end{proof}

Notiamo dunque che una funzione uniformemente continua $f\colon
\left]a,b\right]\to \R$ pu\`o essere con continuit\`a ad una funzione
continua $\bar f\colon [a,b]\to\R$. D'altra parte la funzione $\bar f$
risulta essere anche uniformemente continua per il teorema di Cantor
(o per verifica diretta).


%\begin{theorem}
%Sia $A\subset \R$ e $f\colon A\to \R$ una funzione
%Lipschitziana. Allora \`e possibile estendere $f$ ad una funzione
%lipschitziana $\bar f\colon \R\to \R$ mantenendo la stessa costante di
%Lipschitz $L$.
%\end{theorem}
%\begin{proof}
%Sia $f$ tale che 
%\[
%	\vert f(x)-f(y)\vert \le L \vert x-y\vert \quad \forall x,y\in A.
%\]
%Definiamo $\bar f \colon \R\to\R
%\[
%\bar f(x)= \inf_{a\in A} f(a)+L\vert x-a\vert.
%\]
%Innanzitutto verifichiamo che $\bar
%\end{proof}

\section{Modulo di continuit\`a}

\begin{definition}
Data una funzione $f\colon A\to \R$, chiamiamo \emph{modulo di
continuit\`a} di $f$ la funzione $M\colon \R^+\to \R\cup\{+\infty\}$
definita da
\[
	M(t)=\sup \{ \vert f(x)-f(y)\vert\colon x,y\in A,\ \vert
	x-y\vert \le t\}
\]
\end{definition}

Dalla definizione si verifica facilmente che $M(t)$ \`e una funzione
crescente. Infatti se l'insieme di cui si calcola l'estremo superiore
diventa pi\`u grande (rispetto all'inclusione di insiemi) al crescere
di $t$ e quindi anche l'estremo superiore cresce al crescere di $t$.

\begin{theorem}
Se $f\colon A\to\R$ \`e una funzione qualunque e $M$ \`e il suo modulo
di continuit\`a allora:
\begin{itemize}
\item
$f$ \`e uniformemente continua se e solo se $M(t)\to 0$ per $t\to
0^+$;
\item
$f$ \`e lipschitziana se esiste $L>0$ tale che $M(t)\le Lt$;
\item
$f$ \`e $\alpha$-H\"o{}lderiana se esiste $C>0$ tale che $M(t)\le C t^\alpha$.
\end{itemize}
\end{theorem}
\begin{proof}
Innanzitutto si verifica direttamente dalla definizione di $M$ che si
ha la seguente equivalenza 
\begin{eqnarray*}
		&M(\delta) \le \epsilon &\\
		&\Updownarrow &\\
		&\forall x,y\in A\quad \vert x-y\vert \le \delta \Rightarrow
	\vert f(x)-f(y)\vert \le \epsilon&
\end{eqnarray*}
e dalla definizione di limite (ricordando che $M(t)$ \`e positiva e
crescente) si ha anche
\[
	\lim_{t\to 0^+} M(t) = 0
	\quad \Leftrightarrow \quad
	\forall \epsilon>0\  \exists \delta\ M(\delta)\le \epsilon.
\]
Mettendo assieme le due precedenti equivalenze si ottiene
\[
	\forall \epsilon>0\ \exists \delta\ \forall x,y\in A\quad
	\vert x-y\vert \le \delta \Rightarrow \vert f(x)-f(y)\vert \le
	\epsilon 
\]
che \`e equivalente alla definizione di uniforme continuit\`a per $f$.

Chiaramente se $M$ \`e il modulo di continuit\`a di $f$ allora
si ha
\[
	\vert f(x)-f(y)\vert \le M(\vert x-y\vert)\quad \forall x,y\in A.
\]
Dunque se $M(t)\le Lt$ si trova che $f$ \`e lipschitziana e se
$M(t)\le Lt^\alpha$ si trova che $f$ \`e $\alpha$-H\"o{}lderiana.

D'altro canto se $f$ \`e Lipschitziana si ha 
\[
	M(t)=\sup_{\vert x-y\vert \le t} \vert f(x)-f(y)\vert
	\le \sup_{\vert x-y\vert \le t} L \vert x-y\vert
	\le Lt.
\]
Risultato analogo si ottiene per le funzione $\alpha$-H\"o{}lderiane.

\end{proof}
\end{document}
