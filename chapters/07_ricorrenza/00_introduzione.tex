Nella maggior parte di questo capitolo prenderemo in considerazione le successioni $a_n$
definite \emph{per ricorrenza} o \emph{ricorsivamente} dalle condizioni:
\index{successione!ricorsiva}
\index{successione!definita per ricorrenza}
\index{ricorsione}
\begin{equation}\label{eq1}
\begin{cases}
  a_0 = \alpha,\\
  a_{n+1} = f(a_n)
\end{cases}
\end{equation}

Fissato il termine iniziale $\alpha$ e la legge di ricorrenza $f$,
c'è una unica successione che soddisfa \eqref{eq1} e i suoi termini
sono:
\begin{align*}
a_0 & =\alpha,\\
a_1 &= f(a_0)=f(\alpha),\\
a_2 &= f(a_1)=f(f(\alpha)),\\
a_3 &= f(a_2)=f(f(f(\alpha))),\\
&\ \vdots\\
a_n &= f(a_{n-1}) = f^n(\alpha),\\
&\ \vdots
\end{align*}


Il valore di $a_n$ potrebbe rappresentare lo stato di un sistema che
si evolve a partire da uno stato iniziale $a_0=\alpha$ tramite la
funzione $f$ che rappresenta il cambiamento di stato.
Il numero naturale $n$ potrebbe quindi rappresentare un passo temporale.
In tal senso \eqref{eq1} si chiama anche \emph{sistema dinamico discreto}%
\mymargin{sistema dinamico discreto}\index{sistema!dinamico!discreto}.

L'equazione $a_{n+1} = f(a_n)$ viene chiamata una \emph{equazione ricorsiva autonoma del primo ordine}.
Ci sono altre tipologie di equazioni che considereremo
solo marginalmente nei capitoli successivi.
Ad esempio quando abbiamo definito
il fattoriale: $a_n = n!$ abbiamo dato le condizioni:
\[
\begin{cases}
  a_0 = 1\\
  a_{n+1} = (n+1) \cdot a_n
\end{cases}
\]
ma l'equazione $a_{n+1} = (n+1) \cdot a_n$ è della forma $a_{n+1} =
f(n, a_n)$ e si dice essere \emph{non autonoma} perché la funzione di
ricorrenza $f$ dipende esplicitamente da $n$ oltre che dal termine
precedente $a_n$.

Si potrebbero anche considerare equazioni di ordine maggiore del
primo. Ad esempio la successione $F_n$ di \emph{Fibonacci}
(Leonardo Pisano, 1170--1242)
i cui primi termini sono riportati nella tabella~\ref{tab:Fibonacci}
\begin{table}
  \begin{center}
  0, 1, 1, 2, 3, 5, 8, 13, 21, 34, 55, 89, 144,
  233, 377, 610, 987\dots
\end{center}
\caption{I primi termini della succession di Fibonacci.
Ogni termine è la somma dei due precedenti.}
\label{tab:Fibonacci}
\end{table}
\mymargin{Fibonacci}%
\index{Fibonacci}%
\index{Fibonacci!successione di}%
\index{successione!di Fibonacci}%
soddisfa l'equazione ricorsiva:
\begin{equation}\label{eq:Fibonacci}
\begin{cases}
  F_0 = 0 \\
  F_1 = 1 \\
  F_{n+2} = F_{n+1} + F_n
\end{cases}
\end{equation}
che è una relazione del secondo ordine in quanto ogni termine può
essere definito utilizzando i valori dei \emph{due} termini precedenti.
Se l'equazione è lineare, come in questo caso, si possono trovare delle
formule esplicite per scrivere l'$n$-esimo termine della successione.
Lo faremo nella sezione~\ref{sec:ricorrenza_lineare}

Si potrebbero anche considerare i sistemi di equazioni ricorsive.
Ad esempio se $f$ fosse una funzione complessa $f\colon \CC \to \CC$,
$f(x+iy) = f_1(x,y) + i f_2(x,y)$ con $f_1,f_2 \colon \RR\times\RR\to\RR$
si potrebbe scrivere $a_n = x_n + i y_n$ con $x_n, y_n \in \RR$
e l'equazione ricorsiva $a_{n+1} = f(a_n)$
diventerebbe un sistema di due equazioni:
\[
  \begin{cases}
    x_{n+1} = f_1(x_n, y_n)\\
    y_{n+1} = f_2(x_n, y_n).
  \end{cases}
\]
Lo studio dei sistemi va oltre gli scopi di questo capitolo,
ma accenneremo solamente ad un esempio nella sezione~\ref{sec:mandelbrot}.

