\chapter{successioni}

Una \myemph{successione} in un insieme $X$ è una
funzione\footnote{%
Utilizziamo il grassetto per evidenziare il fatto
che $\vec a$ è una \emph{lista} (o vettore) di numeri.
Nella scrittura a mano non è possibile usare il grassetto, ma potremmo
sottolineare il nome della variabile scrivendo $\underline a$ invece che $\vec a$.
In alternativa si potrebbe scrivere $\stackrel{\rightarrow}a$ oppure evitare
qualunque distinzione e scrivere più semplicemente $a$.}
$\vec a\colon \NN \to X$.
L'insieme delle funzioni $\NN \to X$ viene usualmente indicato
con $X^\NN$ e potremmo dunque scrivere $\vec a \in X^\NN$. In effetti una successione $\vec a$ può essere interpretata
come una sequenza infinita di elementi di $X$:
\[
  \vec a = (a_0, a_1, a_2, \dots, a_n, \dots )
\]
dove si intende
\[
   a_n = \vec a(n).
\]
Le componenti $a_n$ si chiamano \emph{termini} della successione.
I numeri $n$ si chiamano, invece, \emph{indici}.
L'intera
successione $\vec a$ può essere indicata con $(a_n)_{n=0}^\infty$
oppure $(a_n)_n$ oppure,
più semplicemente, con $a_n$ quando sia chiaro che si intende l'intera
successione $\vec a$ e non un singolo termine della stessa.%
\footnote{In alcuni testi le successioni vengono indicate
con la notazione $\{a_n\}$ che però è fuorviante in quanto
una successione in $X$ non è semplicemente un sottoinsieme di $X$:
l'ordine in cui vengono presi gli elementi è rilevante.}

Per noi il caso più interessante sarà quello delle \emph{successioni reali}
$a_n\in \RR$ cioè il caso $X=\RR$. Considereremo però anche il caso di
\emph{successioni complesse} $a_n\in \CC$ (dunque $X=\CC$) perché questo
potrà essere utile in alcune situazioni.

Le successioni vengono usualmente
considerate nei procedimenti di approssimazione.
Spesso infatti siamo interessati a capire qual è il numero (se esiste) a cui
la successione si avvicina al crescere di $n$.

\begin{definition}[successione convergente]
\mymark{***}
Diremo che una successione $a_n\in \RR$ converge
\mymargin{convergenza}
\index{successione!convergente}
ad un numero $\ell \in \RR$
e scriveremo:
\[
  a_n \to \ell \qquad\text{(per $n\to +\infty$)}
  \mymargin{$a_n\to \ell$}
\]
se scelto comunque un errore $\eps>0$ ogni termine della successione,
da un certo punto in poi, si trova a distanza inferiore di $\eps$
dal punto limite $\ell$. Formalmente:
\[
\forall \eps>0\colon \exists N\in\NN \colon \forall n\in \NN\colon
n>N \implies \abs{a_n - \ell}
< \eps.
\]
Il valore $\ell$ verrà in tal caso chiamato \emph{limite} della successione.

La stessa definizione è valida per le successioni complesse
$a_n \in \CC$ dove il simbolo $\abs{\cdot}$ rappresenta il modulo complesso
invece che il valore assoluto. Si ottiene che $a_n\to \ell\in \CC$ se e solo se
$\abs{a_n-\ell}\to 0$ (si osservi che $\abs{a_n-\ell}$ è una successione reale).
\end{definition}

\begin{example}
La successione $a_n = \frac{n}{n+1}$ converge a $\ell=1$:
\[
  \frac{n}{n+1}\to 1 \qquad \text{per $n\to +\infty$}.
\]
\end{example}
%
Possiamo intuitivamente capire che la successione
$a_n = \frac{n}{n+1}$
tende a $1$ scrivendone i primi valori:
\begin{center}
\begin{tabular}{c|ccccccccccc}
$n$ & $0$ & $1$ & $2$ & $3$ & $4$ & $5$ & $6$ & $7$ & $8$ & $9$ & $\dots$ \\ \hline
$a_n $ \rule{0pt}{3ex} & $0$ &
$ \frac{1}{2} $ &
$ \frac{2}{3} $ &
$ \frac{3}{4} $ &
$ \frac{4}{5} $ &
$ \frac{5}{6} $ &
$ \frac{6}{7} $ &
$ \frac{7}{8} $ &
$ \frac{8}{9} $ &
$ \frac{9}{10} $ &
$ \dots $
\end{tabular}
\end{center}
Vogliamo però dare una dimostrazione formale.
%
\begin{proof}
Dato $\eps>0$ ci chiediamo quali siano gli indici $n$
per i quali risulta $\abs{a_n -1}<\eps$ e troviamo
che devono valere due disequazioni:
\[
  1- \eps < \frac{n}{n+1} < 1+\eps.
\]
Facilmente possiamo osservare che $n/(n+1)<1$ per ogni $n$, dunque
la seconda disequazione è sempre verificata. La prima disequazione
si riconduce a
\[
 n + 1 > \frac{1}{\eps}.
\]
Dunque qualunque sia $\eps>0$, scelto $N = \lceil 1 / \eps\rceil$
sappiamo che per ogni $n>N$
si ha $n+1 > n > N \ge \frac{1}{\eps}$ e quindi $1-\eps < a_n$.
Inoltre la disuguaglianza $a_n < 1 < 1+\eps$ è verificata per ogni $n$.
Abbiamo quindi verificato che vale la condizione che definisce
la convergenza.
\end{proof}

\begin{exercise}
Una successione complessa $z_n\in \CC$ potrà essere scritta
nella forma $z_n = x_n + i y_n$ dove $x_n$ e $y_n$ sono successioni
reali. Si può allora verificare che $z_n\to z$ per $z\in \CC$ se e solo se
$x_n\to x$ e $y_n\to y$ dove $z=x+ iy$ con $x,y\in \RR$.
\end{exercise}

Le successioni convergenti sono quelle che approssimano un
certo numero finito.
Vogliamo ora catturare l'idea di successioni che vanno verso i
punti all'infinito.

\begin{definition}[successione divergente]
\mymark{***}
\mymargin{divergente}
\index{successione!divergente}
Una successione $a_n\in \RR$ si dice avere limite $+\infty$
o divergere a $+\infty$
\mymargin{$a_n\to +\infty$}
\[
  a_n \to +\infty \qquad\text{(per $n\to +\infty$)}
\]
se comunque si scelga un numero reale, anche molto grande,
ogni termine della successione, da un certo punto in poi,
risulta essere maggiore di tale numero scelto. Formalmente
\[
  \forall M\in \RR\colon \exists N\in \NN \colon \forall n\in \NN\colon
  n>N \implies a_n >M.
\]
Definizione analoga si ha per il limite $-\infty$. Scriveremo
\mymargin{$a_n\to -\infty$}
\[
  a_n \to -\infty \qquad \text{(per $n\to +\infty$)}
\]
se $-a_n\to +\infty$ ovvero se
\[
  \forall M\in \RR\colon \exists N\in \NN \colon \forall n\in \NN\colon
  n>N\implies a_n < -M.
\]

Un successione reale $a_n$ si dirà essere \emph{divergente}
se $a_n\to +\infty$ o $a_n\to -\infty$.

Se invece $a_n$ è una successione complessa diremo che $a_n$ diverge e scriveremo
$a_n \to \infty$
se
\[
  \forall M\in \RR\colon \exists N \in \NN \colon \forall n\in \NN \colon
  n>N \implies \abs{a_n} > M
\]
che, è equivalente a richiedere che $\abs{a_n}\to +\infty$
(si osservi infatti che $\abs{a_n}$ è una successione reale).
\end{definition}

\begin{example}
Si ha
\[
  1000-n^2 \to -\infty.
\]
\end{example}
%
\begin{proof}
Per dimostrare che $1000-n^2\to -\infty$ sarà
necessario trovare per ogni $M\in \RR$
dei valori di $n$ per i quali si abbia $1000-n^2 < -M$.
Questo avviene se $n^2 > 1000 + M$. Visto che per ogni $n\in \NN$
si ha $n^2 \ge n$ (verificare!) sappiamo che se $n> 1000+M$ allora
anche $n^2 > 1000+M$. Dunque per ogni $M$ sarà sufficiente considerare
un numero intero $N \ge 1000 + M$
(ad esempio si potrebbe scegliere $N = \max\{0, \lceil 1000 + M\rceil\}$)
cosicché per ogni $n>N$ si avrebbe:
\[
 a_n = 1000 - n^2 \le 1000 - n < 1000 - N \le 1000 - (1000  + M) = -M
\]
come richiesto dalla definizione di limite $-\infty$.
\end{proof}

Volendo esprimere il concetto di limite in maniera uniforme
(senza dover distinguere limiti finiti e infiniti) possiamo
rendere la definizione un poco più astratta introducendo il concetto
di \emph{intorno}.
La condizione $\abs{a_n - \ell}< \eps$ può essere scritta in
modo equivalente come
$a_n \in (\ell-\eps, \ell+\eps)$. L'insieme $B_\eps(\ell) = (\ell-\eps, \ell+\eps)$
si chiama \myemph{intorno simmetrico} di raggio $\eps$ centrato in $\ell$.
Possiamo quindi considerare la famiglia
di tutti questi intorni del punto $\ell$:
\[
 \B_\ell = \{B_\eps(\ell)\colon \eps>0 \}.
\]
Questa famiglia di insiemi si chiama \myemph{base di intorni} del punto $\ell$.
La definizione
di limite finito si può dunque riscrivere così:
\begin{equation}\label{eq:034333}
  a_n\to \ell
  \qquad \iff \qquad
  \forall B \in \B_\ell\colon \exists N\in \NN\colon \forall n>N\colon a_n\in B.
\end{equation}
Il vantaggio di questa \emph{astrazione} è che ora possiamo definire
gli intorni di $+\infty$ e di $-\infty$ nel modo seguente:
\begin{align*}
  \B_{+\infty} &= \{ (M,+\infty]\colon M\in \RR\}\\
  \B_{-\infty} &= \{ [-\infty, -M)\colon M \in \RR\}
\end{align*}
e la definizione~\eqref{eq:034333} risulta valida anche nel
caso in cui $\ell$ sia infinito.

In generale se per ogni punto $x$ di un insieme $X$ specifichiamo quale
sia una base di intorni $\B_x$ di $x$, potremo definire i limiti delle successioni
a valori in $X$.
Nel caso $X=\CC$ gli intorni di un punto $z_0\in \CC$ sono gli insiemi
(palle) della forma:
\[
  B_\rho(z_0) = \{z\in \CC \colon \abs{z-z_0}<\rho\}
\]
e dunque $\B_{z} = \{B_\rho(z)\colon \rho>0\}$. Se $X=\bar \CC$ dovremo
definire anche gli intorni di $\infty$:
\[
  \B_\infty = \{ \{z\in \CC\colon \abs{z}>\rho\} \colon \rho>0\}.
\]
Con queste definizioni la condizione~\eqref{eq:034333} è valida
anche per le successioni complesse.

\begin{definition}[proprietà frequenti e definitive]
Sia $P(n)$ un predicato dipendente da un numero naturale $n\in \NN$.
Diremo che $P(n)$ vale \myemph{definitivamente} (in $n$) se
\[
  \exists N\in \NN \colon \forall n>N\colon P(n).
\]
Diremo che $P(n)$ vale \myemph{frequentemente} (in $n$) se
\[
  \forall N\in \NN \colon \exists n>N \colon P(n).
\]
\end{definition}

Chiaramente se una proprietà vale definitivamente vale anche frequentemente.
E dire che vale frequentemente è equivalente a dire che
l'insieme $\{n\in \NN\colon P(n)\}$ è infinito (cioè la proprietà vale per infiniti
valori di $n\in \NN$).

Le due proprietà sono complementari nel senso che vale la seguente
relazione:
\[
  \text{non frequentemente $P(n)$} \iff
  \text{definitivamente non $P(n)$}
\]

Se due proprietà $P(n)$ e $Q(n)$ valgono definitivamente allora anche
$P(n)\land Q(n)$ vale definitivamente. Se invece valgono entrambe
frequentemente allora anche $P(n) \lor Q(n)$ vale frequentemente.

\begin{example}
La proprietà
\[
  \text{$n$ è un numero pari}
\]
vale frequentemente in $n$. La proprietà
\[
 \text{$n! > 2^n$}
\]
vale definitivamente in $n$.
\end{example}

La definizione di limite $a_n \to \ell$ potrebbe quindi enunciarsi così:
per ogni intorno $B$ di $\ell$ si ha $a_n\in B$ definitivamente.
E la sua negazione è: esiste un intorno $B$ di $\ell$ per cui
frequentemente $a_n\not\in B$.

\begin{theorem}[proprietà di separazione di $\bar \RR$ e $\bar \CC$]
\label{th:separazione_R_e_C}
Punti distinti $\ell_1 \neq \ell_2$ in $\bar \RR$ o in $\bar \CC$
ammettono intorni disgiunti $B_1 \in \B_{\ell_1}$, $B_2\in \B_{\ell_2}$,
$B_1 \cap B_2 = \emptyset$.
\end{theorem}
\begin{proof}
Consideriamo innanzitutto $\ell_1,\ell_2\in \bar \RR$.
Se $\ell_1$ ed $\ell_2$ sono entrambi finiti, è sufficiente
considerare $\eps = \abs{\ell_2-\ell_1}/2$.
Dall'interpretazione
geometrica risulta chiaro
che gli intorni $B_1=B_\eps(\ell_1)$ e
$B_2=B_\eps(\ell_2)$
sono disgiunti. Algebricamente questo si ottiene applicando
opportunamente la disuguaglianza triangolare per il valore assoluto.

Se $\ell_1$ ed $\ell_2$ sono entrambi infiniti e sono diversi,
possiamo supporre $\ell_1=-\infty$ e $\ell_2=+\infty$. In tal caso gli
intorni $B_1= [-\infty,0)$ e $B_2=(0,+\infty]$ sono disgiunti.
Se $\ell_1$ è finito e $\ell_2 = +\infty$, basterà prendere
$B_1 = (\ell_1-1, \ell_1+1)$ ed $B_2 = (\ell_1+1,+\infty]$ per avere due
intorni disgiunti. Analogamente si procederà nel caso $\ell_2=-\infty$.

Per $\ell_1,\ell_2 \in \bar \CC$ la dimostrazione si fa in maniera del tutto analoga:
per due punti in $\CC$ si può prendere come $\eps=\abs{\ell_2-\ell_1}/2$
metà della distanza tra i due
punti: la disuguaglianza triangolare del modulo garantirà che
$B_\rho(\ell_1)\cap B_\rho(\ell_2)=\emptyset$.
 Se uno dei due punti è $\ell_2=+\infty$ e l'altro è $\ell_1\in \CC$
si potrà prendere $B_1(\ell_1)$  come intorno di raggio unitario di $\ell_2$ e
$\{z\colon \abs{z}>\abs{\ell_2}+1 \}$
come intorno del punto $\infty$. La disuguaglianza triangolare ci assicura
che questi due intorni sono disgiunti, come ovvio da una interpretazione geometrica.
\end{proof}
\begin{theorem}[unicità del limite]
\mymargin{unicità del limite}
Sia $a_n$ una successione a valori reali e siano $\ell_1,\ell_2\in \bar \RR$
tali che
\[
  a_n \to \ell_1, \qquad
  a_n \to \ell_2.
\]
Allora si ha $\ell_1=\ell_2$ (il limite, se esiste, è unico).
Lo stesso vale per le successioni a valori complessi che hanno limite
$\ell_1, \ell_2 \in \bar \CC$.
\end{theorem}
%
\begin{proof}
Questo risultato discende direttamente dalla proprietà enunciata nel
teorema~\ref{th:separazione_R_e_C}. Se una successione avesse due limiti
diversi allora potrei trovare due intorni disgiunti dei due punti limite
e la definizione di limite mi direbbe che la successione deve stare
definitivamente in ognuno dei due intorni. Questo è impossibile.
\end{proof}

Osserviamo che non è detto che un limite esista, come si vede dal seguente
esempio.

\begin{example}
Sia $a_n = (-1)^n$. Non esiste $\ell \in \bar \RR$
tale che $a_n \to \ell$.
\end{example}
\begin{proof}
La successione $a_n$ ha come valori solamente i numeri $1$ e $-1$,
infatti se $n$ è pari si ha $(-1)^n=1$ e se $n$ è dispari $(-1)^n=-1$.
Supponiamo per assurdo che la successione abbia limite $\ell$
e consideriamo l'intorno $B = (\ell - 1, \ell + 1)$.
Se $\ell\ge 0$ certamente $-1 \not\in B$ in quanto $\ell-1\ge -1$.
Se invece $\ell \le 0$ certamente $1\not\in B$ in quanto $\ell+1\le 1$.
Dunque frequentemente $a_n\not\in B$ che significa che non può
essere $a_n\to \ell$.
\end{proof}

Abbiamo dunque osservato che in generale il limite di una successione può
non esistere ma se esiste è unico.
Questo ci permette di definire l'operatore $\lim$
(limite) che associa ad ogni successione che ammette limite
il suo (unico) limite in $\bar \RR$
(o in $\bar \CC$ per le sottosuccessioni complesse).
Dunque scriveremo se $a_n \to \ell$ scriveremo
\mymargin{$\lim a_n$}
\[
  \lim a_n = \ell.
\]
Per evidenziare il fatto che nella precedente formula
$n$ compare come variabile muta potremo scrivere
in maniera più espressiva
\[
   \lim_{n\to+\infty} a_n = \ell
   \qquad\text{oppure}\qquad
   \lim_n a_n = \ell
\]

\begin{definition}[carattere di una successione]
\mymark{***}
Sia $a_n$ una successione a valori reali o complessi.
Se $a_n$ non ammette limite
si dice anche che $a_n$ è \emph{indeterminata}
\mynote{successione indeterminata}%
\index{successione!indeterminata}
(si intende che è indeterminato il suo limite!).
Abbiamo dunque le seguenti alternative
\begin{enumerate}
 \item la successione è convergente (ha limite finito);
 \item la successione è divergente (ha limite infinito);
 \item la successione è indeterminata (non ha limite).
\end{enumerate}
Determinare il \emph{carattere}
\mynote{carattere di una successione}%
\index{carattere!di una successione}%
di una successione
significa specificare a quale delle tre categorie appartiene.
\end{definition}

Si faccia attenzione al fatto che una successione $a_n\in \RR \subset \CC$
può essere interpretata sia come successione reale che come successione complessa.
Le condizioni di convergenza in $\RR$ o $\CC$ sono allora equivalenti. Ma
se $a_n \to \infty$ (in $\bar \CC$) significa che $\abs{a_n}\to +\infty$
e può capitare che $a_n$ non abbia limite né $+\infty$
né $-\infty$ in quanto potrebbe frequentemente cambiare di segno
(un esempio è $a_n = (-n)^n$).
In alcuni testi si usa il simbolo $\pm \infty$ per indicare
il limite $\infty$ nel contesto dei numeri reali. Si faccia però
attenzione che se consideriamo come possibili valore di un limite
sia $\pm\infty$ che $+\infty$ e $-\infty$ non si avrà più l'unicità
del limite in quanto una successione che tende a $+\infty$
tende anche a $\pm \infty$.

\section{criteri di convergenza}

\begin{theorem}[criteri di confronto]
\mymark{***}
Siano $a_n$, $b_n$ e $c_n$ successioni a valori reali%
\footnote{Non è possibile fare confronti tra valori complessi, visto che
sui numeri complessi non abbiamo un ordinamento}.
\mymargin{confronto tra limiti}
\begin{enumerate}
\item
Se per ogni $n\in \NN$ si ha
\[
a_n \le b_n
\]
e se entrambe le successioni ammettono limite: $a_n \to a$ e $b_n \to b$
allora
\[
a \le b.
\]

\item
Se per ogni $n$ si ha:
\[
a_n \le b_n
\]
e se $a_n\to +\infty$ allora anche $b_n \to +\infty$.
Viceversa se $a_n \le b_n$ e $b_n \to -\infty$ allora anche $a_n \to -\infty$.

\item
(teorema dei carabinieri)
\mynote{teorema dei carabinieri}
\index{teorema!dei carabinieri}
Se per ogni $n$ vale
\[
a_n \le b_n \le c_n
\]
 e se le due
successioni $a_n$ e $c_n$ hanno lo stesso limite: $a_n \to \ell$ e $c_n\to \ell$
allora anche $b_n \to \ell$.
\end{enumerate}
\end{theorem}
%
\begin{proof}
\mymark{**}
\begin{enumerate}
\item
Se per assurdo fosse $a > b$ esisterebbero degli intorni disgiunti $B_a\in \B_a$
e $B_b \in \B_b$.
Inoltre si avrebbe $B_a > B_b$ (cioè: ogni punto di $B_a$ sarebbe maggiore
di ogni punto di $B_b$) visto che $a>b$.
Ma, dalla definizione di limite,
si dovrebbe avere che definitivamente $a_n\in B_a$ e $b_n\in B_b$ il che
è assurdo se $a_n\le b_n$ e $B_a > B_b$.

\item
Se $a_n \to +\infty$ per ogni $M\in \RR$
si ha $a_n > M$ definitivamente.
Ma se $b_n>a_n$ si ha anche $b_n >M$.
Questo è vero per ogni $M\in \RR$ e quindi
si ottiene la definizione di limite $b_n \to +\infty$.
Dimostrazione analoga si ottiene nel caso $b_n \to -\infty$.

\item
Se $a_n$ e $c_n$ hanno lo stesso limite $\ell$ significa che per ogni
$B \in \B_\ell$
si ha $a_n\in B$ e $c_n\in B$ definitivamente.
Ma allora anche $b_n\in B$ definitivamente. Essendo questo
vero per ogni $B\in \B_\ell$ si ottiene la definizione di limite
$b_n \to \ell$.
\end{enumerate}
\end{proof}


\begin{corollary}[permanenza del segno]
\mymark{***}
\mynote{permanenza del segno}
\index{permanenza del segno (successioni)}
\index{teorema!della permanenza del segno (successioni)}
Sia $a_n$ una successione e $c\in \RR$.
Se per ogni $n\in \NN$ si ha $a_n \ge c$ e se $a_n$ ha limite $\ell \in \bar \RR$
allora $\ell \ge c$.
In particolare se una successione ha valori non negativi
ed ammette limite, allora
il limite è non negativo.

Viceversa, se una successione $a_n$ ha limite positivo:
$a_n \to a >0$ allora $a_n>0$ per ogni $n$ tranne
al più un numero finito di termini.
 Risultato analogo vale se $a_n \to a <0$: a parte un numero
 finito di termini, i valori della successione sono negativi.
\end{corollary}
%
\begin{proof}
\mymark{**}
Prima parte.
E' sufficiente considerare la successione costante $c_n = c$
cosicché si ha $c_n \le a_n$ per ogni $n$. Ma ovviamente
la successione $c_n$ ha limite $c$ e dunque, per confronto,
deve essere $c\le \ell$.

Seconda parte. Se $a>0$ e $a\in \RR$
per definizione di limite esiste $N$
tale che per ogni $n>N$ si ha $a_n> a/2 > 0$. Dunque solo
un numero finito di termini (quelli con indice $n\le N$)
possono essere negativi. Se $a=+\infty$ il ragionamento
si ripete a maggior ragione sapendo che esiste $N$
tale che $a_n>1$ per $n>N$.
Cambiando segno alla successione si ottiene il caso $a<0$.
\end{proof}

Si presti molta attenzione al fatto che se $a_n$ è a termini positivi
non è detto che il limite sia positivo, possiamo solo affermare
che non è negativo. Ad esempio $a_n = \frac{1}{n+1}>0$
ma $a_n \to 0$ (verificare!).
In generale se una successione ha valori in un intervallo il suo limite,
se esiste, deve essere un punto del corrispondente intervallo chiuso.

Se una successione ha limite $0$ può avere infiniti termini
positivi e infiniti termini negativi, come nel caso
della successione $a_n = 1/(-2)^n$.

\begin{theorem}[successioni che differiscono su un numero finito di termini]
\mymark{**}
\mynote{successioni che differiscono su un numero finito di termini}%
\index{successioni!che differiscono su un numero finito di termini}
Se $a_n$ e $b_n$ sono due successioni
che differiscono solamente per un numero finito di termini
(significa l'insieme $\{n\in \NN\colon a_n \neq b_n\}$ è finito)
allora $a_n$ e $b_n$ hanno lo stesso carattere e se non
sono indeterminate hanno lo stesso limite.
\end{theorem}
%
\begin{proof}
Due successioni che differiscono su un numero finito
di termini sono definitivamente uguali.
Avere limite $\ell$ significa che per ogni $B$ intorno di $\ell$
la successione sta definitivamente in $B$ e dunque la condizione
è equivalente se le successioni sono definitivamente uguali.
Significa che hanno lo stesso carattere e lo stesso limite.
\end{proof}

Il teorema precedente è molto utile perché ci permette di applicare
i criteri di confronto anche nel caso in cui le ipotesi siano violate
su un numero finito di termini, come nel seguente esempio.

\begin{example}
Sapendo che la successione $a_n = n$ tende a $+\infty$ dimostrare
che anche la successione $b_n = n^2-10$ tende a $+\infty$.
\end{example}
%
\begin{proof}
E' sufficiente osservare che $n^2-10 > n$ se $n\ge 4$ infatti
se $n\ge 4$ si ha
\[
n^2 - 10 \ge 4 n -10 = n + 3n - 10 \ge n + 12 -10 \ge n+2 > n.
\]
Dunque se consideriamo la successione $c_n$ ottenuta da $b_n$
modificando i primi quattro termini (ponendo ad esempio $c_0 = a_0$,
$c_1=a_1$, $c_2=a_2$ e $c_3=a_3$) si ottiene $c_n\ge a_n$ per ogni $n$.
Ma allora $c_n \to +\infty$ (per confronto con $a_n$) ma visto che
$b_n$ differisce da $c_n$ solo nei primi 4 termini anche $b_n\to +\infty$.
\end{proof}

Il teorema precedente garantisce inoltre
che per quanto riguarda lo studio del limite possiamo considerare
successioni che siano definite solamente da un certo indice in poi.
Ad esempio è molto frequente considerare successioni il cui primo indice sia
$n=1$ invece che $n=0$. Questo non cambia nulla per quanto riguarda il limite
della successione.

\begin{example}
La successione $a_n = 1/n$ è definita per $n\in \NN$ ma $n\neq 0$.
Ciò non toglie che possiamo studiarne il limite come qualunque altra
successione. Per evidenziare il fatto che il primo indice è $n=1$
si potrà usare la notazione $(a_n)_{n=1}^\infty$.
Per la cronaca: $1/n \to 0$.
\end{example}

Data una successione $a_n$ potremo considerare l'\emph{insieme
dei suoi valori}: $\{a_n\colon n\in \NN\}$.
Si tratta dell'immagine della funzione $n\mapsto a_n$
e a volte si chiama \myemph{supporto} della successione.
Si faccia attenzione al fatto che l'insieme dei valori
non descrive completamente la successione perché
viene persa l'informazione sull'ordine in cui vengono elencati
i termini della successione e sulla loro molteplicità (ogni valore
potrebbe essere assunto su molti indici diversi). Ad esempio
l'insieme dei valori della successione $a_n = (-1)^n$ è l'insieme $\{-1, 1\}$. Ma anche la successione
\[
 b_n = \begin{cases}
   -1 & \text{se $n\le 42$}\\
   1  & \text{altrimenti}
 \end{cases}
\]
ha lo stesso insieme dei valori. Si osservi però
che la prima successione non ammette limite mentre la seconda
è convergente (verificare!).

Ricordandoci che una successione reale $a_n$ non è altro che
una funzione $\vec a \colon \NN \to \RR$
ha senso applicare alle successioni gli operatori
$\sup$, $\inf$, $\max$ e $\min$
(si veda la definizione~\ref{def:funzione_limitata}).
Queste operazioni potrebbero essere sensibili anche ai primi termini della
successione (a differenza dell'operazione di limite) dunque potrebbe
essere necessario, per chiarezza, specificare qual è il primo indice
da cui si intende cominciare a considerare i valori. Ad esempio
se la successione $a_n$ è definita sui naturali tranne lo zero
si avrà:
\[
  \sup a_n = \sup_{n=1}^\infty a_n = \sup\{a_n \colon n \in \NN, n \ge 1\}.
\]

\begin{example}
Si consideri $a_n = \frac{1}{n+1}$ definita per $n\in \NN$.
Allora
\[
  \sup a_n = \max a_n = 1, \qquad
  \inf a_n = 0, \qquad \text{non esiste }\min a_n.
\]
\end{example}
\begin{proof}
Per ogni $n \in \NN$ si ha $n+1\ge 1$ e quindi $a_n = 1/(n+1) \le 1$.
Visto poi che $a_0 = 1$ si ottiene immediatamente che $\max a_n = 1$
e di conseguenza $\sup a_n = 1$.

Per verificare che $\inf a_n = 0$ dobbiamo verificare innanzitutto
che $0$ è minorante, e questo è vero in quanto $a_n = 1/(n+1)> 0$ essendo $n+1\ge 1 \ge 0$.
Inoltre dobbiamo verificare che per ogni $\eps >0$ esiste $n\in \NN$ tale
che $a_n < 0 + \eps = \eps$. Questo succede se $1/(n+1) < \eps$ ovvero
se $n > 1/\eps -1$ ad esempio per $n=\lceil 1/\eps\rceil$.
Abbiamo dunque verificato che $\inf a_n = 0$.
Il minimo di $a_n$ non esiste perché se esistesse dovrebbe essere uguale
all'estremo inferiore cioè dovrebbe essere $0$. Ma questo è impossibile
perché per ogni $n\in \NN$ si ha $a_n = 1/(n+1)\neq 0$.
\end{proof}

Sempre considerando la definizione~\ref{def:funzione_limitata}
una successione $a_n$ risulta essere
\emph{superiormente limitata}
\mymargin{successione limitata}%
\index{successione!superiormente limitata}%
se $\sup a_n < +\infty$
ovvero:
\[
  \exists M\in \RR\colon \forall n \in \NN\colon a_n \le M.
\]
In maniera analoga si dice che la successione $a_n$ è
\emph{inferiormente limitata}
\index{successione!inferiormente limitata}%
quando $\inf a_n > +\infty$
ed è \emph{limitata}
\index{successione!limitata}%
\index{limitato!successione}%
quando $\sup \abs{a_n}< +\infty$.

Per le successioni di numeri complessi non potremo parlare di
limitatezza superiore e inferiore in quanto sui complessi non c'è
un ordinamento.
Diremo però comunque che la successione
$a_n\in \CC$ è limitata se vale $\sup \abs{a_n} < \infty$.

\begin{theorem}[limitatezza delle successioni convergenti]
\mymark{**}
\mymargin{limitatezza delle successioni convergenti}
Sia $a_n\in\RR$ una successione.
Se $a_n$ è convergente allora $a_n$ è limitata.
Se $a_n\to +\infty$ allora $a_n$ è inferiormente limitata.
Se $a_n\to -\infty$ allora $a_n$ è superiormente limitata.

Anche nel caso complesso:
se $a_n\in \CC$ è una successione convergente allora è
limitata.
\end{theorem}
%
\begin{proof}
Sia $\ell \in \RR$ il limite di $a_n$.
Se $\ell$ è finito,
dalla definizione di limite (ponendo $\eps=1$) sappiamo che esiste $N\in \NN$
tale che per ogni $n> N$ si ha $\ell -1 < a_n < \ell+ 1$.
Prendiamo allora $M=\max\{a_0, a_1, \dots, a_N, \ell +1\}$
e $m =\min \{a_0, a_1, \dots, a_N, \ell-1\}$. Si avrà allora
che per ogni $n\in \NN$ vale
\[
  m \le a_n \le M
\]
e dunque $a_n$ è limitata.

Se $a_n \to +\infty$ allora, per definizione di limite, deve esistere un $N$
tale che per ogni $n>N$ si abbia $a_n \ge 0$ (abbiamo scelto arbitrariamente
$M=0$ nella definizione). Ma allora ponendo $K=\min\{a_0, a_1, \dots, a_N, 0\}$
si avrà che $a_n\ge K$ per ogni $n\in \NN$ dunque $a_n$ è inferiormente
limitata.

Dimostrazione analoga si fa nel caso $a_n \to -\infty$.

Se $a_n\in \CC$ e $a_n\to a\in \CC$, allora $\abs{a_n}\to \abs{a}\in \RR$.
Dunque la successione reale $\abs{a_n}$ è limitata che significa (per definizione)
che la successione complessa $a_n$ è limitata.
\end{proof}

\section{operazioni con i limiti}

\begin{theorem}[continuità sequenziale]%
\label{th:cont_sequenziale}%
\index{continuità!sequenziale}%
\index{funzione!sequenzialmente continua}%
Sia $f\colon A \subset \RR \to \RR$ una funzione e $a\in A$.
Allora $f$ è continua nel punto $a$ se e solo se per ogni successione
$a_n\in A$, $a_n\to a$ risulta
\[
   f(a_n) \to f(a).
\]
\end{theorem}
%
\begin{proof}
Supponiamo che $f$ sia continua nel punto $a$ e sia $a_n\in A$
una successione tale che
$a_n \to a$. Scriviamo di seguito la definizione di continuità nel punto $a$
e la definizione di convergenza $a_n \to a$:
\begin{align*}
  \forall \eps>0 \colon \exists \delta>0\colon
  \abs{x-a}< \delta \implies \abs{f(x)-f(a)}< \eps \\
  \forall \delta>0 \colon \exists N\in \NN \colon
  n>N \implies \abs{a_n-a} < \delta.
\end{align*}
Scegliendo nella seconda condizione lo stesso $\delta$ dato dalla prima
condizione si ottiene $n>N \implies \abs{a_n-a}< \delta \implies \abs{f(a_n)-f(a)}
<\eps$ da cui si ottiene esattamente la definizione di limite $f(a_n)\to f(a)$:
\[
  \forall \eps>0\colon \exists N\in \NN\colon
  n>N\implies \abs{f(a_n)-f(a)} < \eps.
\]

Viceversa supponiamo che per ogni $a_n\to a$, $a_n\in A$
si abbia $f(a_n)\to f(a)$ e supponiamo,
per assurdo, che $f$ non sia continua in $a$. Allora negando la
condizione di continuità in $a$ si ottiene:
\[
  \exists \eps>0 \colon \forall \delta>0 \colon \exists x\in A\colon
  \abs{x-a}<\delta \land \abs{f(x)-f(a)}\ge \eps.
\]
Fissato $\eps$ possiamo dunque porre $\delta = 1/n$ per ogni $n\in \NN$.
Si ottiene allora che esiste $x_n$ che soddisfa le due condizioni:
\[
  \abs{x_n-a}<\frac 1 n, \qquad \abs{f(x_n)-f(a)}\ge \eps.
\]
La prima condizione garantisce che si abbia $a_n \to a$.
Ma la seconda condizione impedisce che si abbia $f(x_n)\to a$.
Abbiamo quindi trovato un assurdo.
\end{proof}

\begin{theorem}[limite del valore assoluto]
\mymark{**}
\mymargin{limite!del valore assoluto}
\label{th:limite_abs}
Se $a_n\in \RR$ è una successione che ammette limite $a_n \to a \in \bar \RR$
allora
\[
  \abs{a_n} \to \abs{a}.
\]
Viceversa se $\abs{a_n} \to 0$ allora $a_n \to 0$.

Lo stesso vale per le successioni $a_n\in \CC$ che hanno limite
$a_n\to a\in \bar \CC$ (rimpiazzando il valore assoluto
con il modulo).
\end{theorem}
%
\begin{proof}
Se $a\in \RR$ è sufficiente applicare il teorema~\ref{th:cont_sequenziale}
alla funzione $f(x)=\abs{x}$ che sappiamo essere una funzione continua.
Quando $a$ è infinito la condizione $a_n\to a$ implica chiaramente
$\abs{a_n}\to +\infty$ in quanto sia la condizione $a_n>M$
che la condizione
$a_n < -M$ implicano $\abs{a_n}>M$ quando $M>0$.

Per la seconda parte è sufficiente osservare che essendo
$\big\lvert\abs{x}\big\rvert
 = \abs{x}$ la definizione di limite $a_n\to 0$ è equivalente a
 $\abs{a_n-0}=\abs{a_n}\to 0$.

 La stessa identica dimostrazione è valida anche per le successioni di numeri complessi.
\end{proof}

\begin{theorem}[limite della somma]
\label{th:limite_somma}
\mymark{***}
\mymargin{limite!della somma}
Siano $a_n \to a$ e $b_n \to b$ successioni reali con $a,b\in \bar \RR$.
Se $a$ e $b$ non sono
infiniti di segno opposto
allora
\[
    a_n + b_n \to a+b.
\]
Se $a$ e $b$ non sono infiniti con lo stesso segno allora
\[
   a_n - b_n \to a-b.
\]

Lo stesso risultato vale per successioni complesse con $a,b\in \bar \CC$
supponendo che $a$ e $b$ non siano entrambi $\infty$.
\end{theorem}
%
\begin{proof}
\mymark{**}
Consideriamo inizialmente il caso in cui $a,b$ siano entrambi limiti finiti.
Allora per ogni $\eps>0$ esiste un $N$ (che, al solito, sarà il massimo tra un $N_a$ ed un $N_b$) tale per cui per ogni $n> N$ si ha
$\abs{a_n -a} < \eps/2$ e $\abs{b_n - b} < \eps/2$.

Risulta allora che per ogni $n> N$ si ha
\[
  \abs{(a_n + b_n) - (a+b)} \le \abs{a_n -a} + \abs{b_n -b} < \eps/2 + \eps/2 = \eps.
\]
Cioè $a_n+b_n \to a+b$, come volevamo dimostrare.

Se $a =+\infty$ e $b\neq -\infty$ allora la successione $b_n$ è inferiormente
limitata cioè esiste $K\in \RR$ tale che $b_n \ge K$ per ogni $n$ e quindi
$a_n+b_n \ge a_n + K$.
Visto che $a_n \to +\infty$ sappiamo che per ogni $M\in \RR$ esiste
$N\in \NN$ tal che per $n>N$ si ha $a_n > M - K$.
Ma allora $a_n + b_n >M$ da cui si ottiene la validità della definizione
di limite $a_n + b_n \to +\infty$.

Per completare la dimostrazione osserviamo che se $a_n\to \ell$ con $\ell \in
\bar \RR$, allora $-a_n \to -\ell$. Si tratta semplicemente di cambiare i segni
nella definizione di limite.

Dunque se $a=-\infty$ e $b\neq +\infty$ possiamo cambiare segno a entrambe
le successioni e ricondurci al caso precedente. Questo completa la prima
parte della dimostrazione.

Per dimostrare la seconda parte (il limite della differenza è uguale alla differenza dei limiti) ci si riconduce alla prima parte, ricordando che la
differenza è la somma con l'opposto.

Per quanto riguarda le successioni complesse, se $a$ e $b$ sono finiti ci si
riconduce al caso reale in quanto la convergenza sul piano complesso si riconduce
alla convergenza di parte reale e parte immaginaria.
Se $a=\infty$ e $b\in \CC$ allora significa che $\abs{a_n}\to +\infty$ e
basta
utilizzare la disuguaglianza triangolare inversa per ottenere che:
\[
  \abs{a_n+b_n} \ge \abs{a_n} - \abs{b_n} \to +\infty - b = +\infty
\]
da cui $a_n+b_n\to \infty$. Analogo il caso $a\in \CC$, $b=\infty$.
\end{proof}

\begin{theorem}[prodotto di limitata per infinitesima]
\mymark{**}
\mymargin{prodotto limitata per infinitesima}
Se $a_n$ è una successione limitata e $b_n\to 0$ allora
$a_n\cdot b_n \to 0$.
Il risultato è valido sia per le successioni reali
che per le successioni complesse.
\end{theorem}
%
\begin{proof}
  Se $a_n$ è limitata significa che esiste $M>0$ tale che $\abs{a_n}\le M$ per
  ogni $n\in \NN$.
  Per la definizione di limite applicata a $b_n\to 0$, per ogni $\eps>0$
  esiste $N\in \NN$ tale che per ogni $n>N$ si ha $\abs{b_n -0}=\abs{b_n} < \eps / M$.
  Allora per ogni $n>N$ si ha
  \[
    \abs{a_n\cdot b_n - 0} = \abs{a_n\cdot b_n} \le M \cdot \abs{b_n}
    < M \cdot \frac \eps M = \eps.
  \]
  Dunque è verificata la definizione di limite $a_n \cdot b_n \to 0$.
\end{proof}
\begin{theorem}[limite del prodotto]
\label{th:limite_prodotto}
\mymark{***}
\mymargin{limite!del prodotto}
Siano $a_n\in \RR$ e $b_n\in \RR$ successioni
che ammettono limite $a_n \to a\in \bar \RR$ e $b_n \to b\in \bar \RR$.
Se escludiamo il caso in cui uno dei due limiti
è zero e l'altro è infinito
allora risulta
\[
  a_n \cdot b_n \to a\cdot b.
\]

Lo stesso risultato vale per successioni $a_n,b_n\in \CC$ con limite in $\bar \CC$.
\end{theorem}
\begin{proof}
\mymark{**}
Se $a$ e $b$ sono entrambi finiti si osserva che
\begin{align*}
  \abs{a_n \cdot b_n - a\cdot b}
  &= \abs{a_n \cdot b_n - a_n \cdot b + a_n \cdot b - a\cdot b}\\
  &\le \abs{a_n} \cdot\abs{b_n - b} + \abs{a_n -a} \cdot \abs{b}.
\end{align*}
Sappiamo che $b_n - b \to 0$ e $a_n - a \to 0$ (limite della differenza)
e sappiamo che $a_n$ è limitata e ovviamente la successione costante $b$ è anch'essa limitata.
Dunque (prodotto di limitata per infinitesima) si ha $a_n(b_n-b)\to 0$ e $(a_n-a)b\to 0$. E ancora applicando il limite della somma si ottiene
infine che $a_n b_n - ab\to 0$ il che è equivalente (sommo $ab$) ad $a_n b_n\to ab$, come volevamo dimostrare.

Se $a = +\infty$ e $b>0$ allora per la definizione di limite applicata a $b_n$
esiste $N_b\in \NN$ tale che per ogni $n>N$ si ha $b_n > b/2$. La definizione
di limite applicata ad $a_n$ ci dice invece che per ogni $M\in \RR$
esiste $N_a\in \NN$ tale che per ogni $n>N_a$ si ha $a_n > 2M/b$.
Deduciamo che per ogni $n> N=\max\{N_a, N_b\}$ si ha
\[
  a_n\cdot b_n > \frac{2M}{b}\frac{b}{2} = M.
\]
Si ottiene dunque la validità della definizione di limite $a_n b_n\to +\infty$.
Se $a= +\infty$ e $b<0$ oppure $a=-\infty$ e $b>0$ si può cambiare segno
ad una delle due successioni e ricondursi al caso precedente.
\end{proof}

\begin{theorem}[limite del reciproco]
\label{th:limite_reciproco}
\mymark{**}
\mymargin{limite!del reciproco}
Sia $a_n\in \RR$ una successione che ammette limite $a_n \to a\in \bar \RR$.
Se $a\neq 0$ allora
\[
  \frac{1}{a_n} \to \frac{1}{a}.
\]
Se $a = 0$ ma $a_n>0$ per ogni $n\in \NN$ allora
\[
  \frac{1}{a_n} \to +\infty.
\]
Se $a=0$ ma $a_n<0$ per ogni $n\in \NN$ allora
\[
  \frac{1}{a_n} \to -\infty.
\]

Nel caso complesso non ci sono eccezioni.
Se $a_n\to a \in \bar \CC$ allora $1/a_n \to 1/a\in \bar \CC$.
\end{theorem}
%
\begin{proof}
\mymark{*}
Nel caso $a$ sia finito e $a\neq 0$ abbiamo già dimostrato
che la funzione $f(x)=1/x$ è continua nel punto $a$
(teorema~\ref{th:cont_reciproco}) dunque
il risultato segue dal
teorema~\ref{th:cont_sequenziale}: $f(a_n)\to f(a)$.

Se $a$ è infinito allora
(che sia $a=+\infty$ o $a=-\infty$)
per ogni $\eps > 0$ esiste $N\in \NN$ tale che per ogni $n>N$ si ha
$\abs{a_n} > 1/\eps$. E dunque per ogni $n>N$
\[
\abs{\frac{1}{a_n}} < \eps
\]
che significa che $1/a_n \to 0$, come volevamo dimostrare.

Consideriamo il caso $a=0$ con $a_n>0$ per ogni $n\in \NN$. In tal caso
dalla definizione di limite $a_n \to 0$ sappiamo che
per ogni $M>0$ esiste $N\in \NN$ tale che per $n>N$ si ha
$a_n = \abs{a_n} < 1/M$. Dunque per $n>N$ si ha $1/a_n > M$ ed
abbiamo ottenuto la validità della definizione di limite $1/a_n \to +\infty$.

Il caso $a=0$ con $a_n<0$ si riconduce al precedente cambiando
segno ad $a_n$.
\end{proof}

\begin{theorem}[limite del rapporto]
\label{th:limite_rapporto}
\mymark{**}
\mymargin{limite!del rapporto}
Se $a_n \to a$ e $b_n\to b$ allora
\[
  \frac{a_n}{b_n} \to \frac{a}{b}
\]
escludendo il caso in cui $a$ e $b$ siano entrambi infiniti o entrambi nulli.
\end{theorem}
%
\begin{proof}
\mymark{*}
Possiamo ricondurci ai teoremi precedenti osservando che
\[
   \frac{a_n}{b_n} = a_n \cdot \frac{1}{b_n}.
\]
\end{proof}

Si osservi che
per come abbiamo definito le operazioni
su $\bar \RR$
i teoremi precedenti si possono riassumere dicendo
che ognuna delle uguaglianze
\begin{align*}
\lim (a_n + b_n)
&= \lim a_n + \lim b_n \\
\lim (a_n - b_n)
&= \lim a_n - \lim b_n \\
\lim (a_n \cdot b_n)
&= \lim a_n \cdot \lim b_n \\
\lim \frac {a_n}{b_n}
&= \frac{\lim a_n}{\lim b_n}
\end{align*}
è valida ogni volta che l'espressione sul lato destro è definita.
%
\section{successioni estratte}

Le successioni sono funzioni $\NN \to \RR$ dunque possiamo
utilizzare la definizione~\ref{def:monotonia} per dire
se una successione è (strettamente) crescente/decrescente.
Per le successioni, tuttavia, si possono dare delle definizioni
equivalenti utilizzando il principio di induzione,
come viene esplicitato nel seguente
teorema.

\begin{theorem}[successioni monotòne]
\mymark{***}
Una successione $a_n$ è
\begin{enumerate}
\item \emph{crescente}: se per ogni $n\in \NN$ si ha $a_{n+1} \ge a_n$;
\item \emph{decrescente}: se per ogni $n\in \NN$ si ha $a_{n+1} \le a_n$;
\item \emph{strettamente crescente}: se per ogni $n\in \NN$ si ha $a_{n+1}>a_n$;
\item \emph{strettamente decrescente}: se per ogni $n\in \NN$ si ha
$a_{n+1}<a_n$;
\end{enumerate}
\end{theorem}
%
\begin{proof}
  Consideriamo ad esempio la prima condizione (funzione crescente).
Ovviamente se $a_n$ è crescente si ha $a_{n+1}\ge a_n$ in quanto $n+1 > n$.
Viceversa supponiamo che per ogni $n\in \NN$ si abbia $a_{n+1}\ge a_n$.  Allora
chiaramente $a_1\ge a_0$, $a_2\ge a_1 \ge a_0$, $a_3 \ge a_2 \ge a_1 \ge a_0$...
Per induzione si può quindi dimostrare che  $a_n \ge a_{n-1} \ge \dots \ge a_2
\ge a_1$.

Gli altri casi si svolgono in maniera analoga.
\end{proof}

\begin{theorem}[limite di successioni monotòne]
\mymark{***}
\mymargin{limite!di successioni monotòne}
Ogni successione monotòna ammette limite.
Più precisamente: se $a_n$ è crescente allora $\lim a_n = \sup a_n$,
se $a_n$ è decrescente allora $\lim a_n = \inf a_n$.
\end{theorem}
%
\begin{proof}
\mymark{***}
Supponiamo sia $a_n$ crescente e sia $\ell = \sup a_n$.
Se $\ell$ è finito sappiamo che (caratterizzazione del $\sup$)
per ogni $\eps>0$ esiste $N\in \NN$ tale che $a_N> \ell -\eps$.
Ma siccome $a_n$ è crescente si avrà che per ogni $n>N$ vale
$a_n \ge a_N > \ell-\eps$.
D'altra parte sappiamo anche che $\ell\ge a_n$ per ogni $n\in \NN$
e dunque, mettendo insieme le due cose, si ottiene
\[
  \forall \eps>0 \colon \exists N\in \NN \colon \forall n\in \NN\colon
   n>N \implies \ell-\eps < a_n \le \ell < \ell + \eps.
\]
Abbiamo dunque verificato la definizione di limite $a_n \to \ell$.

Se  $\ell=+\infty$ sappiamo che $a_n$ non è superiormente limitata, cioè per ogni $M\in \RR$ esiste $N\in \NN$ tale che $a_N \ge M+1 > M$.
Essendo però $a_n$ crescente otteniamo anche che per ogni $n>N$ si
 ha $a_n \ge A_N > M$. Dunque si ottiene
 \[
 \forall M\in \RR\colon\forall N\in \NN\colon \forall n \in \NN\colon
  n>N \implies a_n > M
 \]
 che è la definizione di limite $a_n \to +\infty$.

Non può essere $\ell = -\infty$ in quanto il $\sup a_n \ge a_0 > -\infty$.
\end{proof}

\begin{definition}[sottosuccessione]
\mymark{*}
Se $a_n$ è una successione e $n_k$ è una successione strettamente crescente i cui valori sono numeri naturali, allora la successione
$b_k = a_{n_k}$ si dice essere una \myemph{sottosuccessione} di $a_n$
(o anche \emph{successione estratta} da $a_n$).
\end{definition}

Ricordando che una successione $a_n$ non è altro che una funzione
$\vec a\colon \NN \to \RR$, la successione $n_k$ corrisponde ad una funzione
$\vec n\colon \NN \to \NN$ e la sottosuccessione $a_{n_k}$ corrisponde alla
funzione composta $\vec a \circ \vec n$.

Si osservi che nella definizione precedente la variabile $n$ rappresenta
una variabile muta quando scriviamo la successione $a_n$, ma
rappresenta anche il nome della successione fissata $n_k$.
Questo sovraccarico
di significato è voluto e se usato correttamente rende più semplice
le notazioni, in quanto la successione $n_k$ viene sostituita alla
variabile $n$, con lo stesso nome, nella successione $a_n$.
La sottosuccessione $a_{n_k}$ risulta essere una successione nella variabile $k$, non nella variabile $n$.

\begin{example}
Sia $a_n = n^2$ la successione dei quadrati perfetti:
\begin{center}
\begin{tabular}{c|ccccccccc}
$n$   & $0$ & $1$ & $2$ & $3$ & $4$  & $5$  & $6$  & \dots \\ \hline
$a_n$ & $0$ & $1$ & $4$ & $9$ & $16$ & $25$ & $36$ & \dots
\end{tabular}
\end{center}
Consideriamo la successione dei numeri pari $n_k = 2k$.
la corrispondente sottosuccessione dei quadrati perfetti
$b_k = a_{n_k}$
rappresenta la successione dei quadrati dei numeri pari:
\begin{center}
\begin{tabular}{c|ccccccccc}
$k$       & $0$ & $1$ & $2$ & $3$ & $4$  & $5$  & $6$  & \dots \\ \hline
$n_k$ & $0$ & $2$ & $4$ & $6$ & $8$ & $10$ & $12$ & \dots \\
$a_{n_k}$ & $0$ & $4$ & $16$ & $36$ & $64$ & $100$ & $144$ & \dots
\end{tabular}
\end{center}
Si ha in pratica
  $b_k = a_{n_k} = a_{2k} = (2k)^2$.

Abbiamo in effetti \emph{estratto} alcuni dei termini della successione
originaria.
\end{example}

\begin{example}
Se $a_n = (-1)^n$ e $n_k=2k$ allora $a_{n_k} = 1$.
Vediamo quindi che una successione che non ammette limite
può contenere una sottosuccessione che invece ha limite.
\end{example}

Osserviamo che se $\vec n\colon \NN\to \NN$ è una
funzione strettamente crescente
(cioè $n_k=\vec n(k)$ è una successione strettamente crescente di indici)
allora posto $A=\vec n(\NN)=\{n_k\colon k\in \NN\}$ si ha che
$\vec n \colon \NN \to A$ è una bigezione. Quindi $A$ è un insieme infinito.
Viceversa dato un qualunque insieme infinito $A\subset \NN$ esiste una
unica successione $\vec n\colon \NN \to A$ bigettiva e strettamente crescente:
basterà porre, per induzione,
$n_0 = \min A$, $n_1 = \min \{n\in A \colon n > n_0\}$
e, in generale,
 $n_{k+1} = \min\{ n\in A \colon n > n_k\}$.

Dunque possiamo identificare le sottosuccessioni di una successione
$\vec a \colon \NN \to \RR$ con le restrizioni ai sottoinsiemi infiniti di $\NN$.
Nell'esempio precedente, si è considerata la sottosuccessione
di tutti i termini con indice pari $n_k=2k$ per ottenere la sottosuccessione
$a_{n_k} = a_{2k}$. Si può equivalentemente pensare di prendere l'insieme di
tutti i numeri pari $A=2\NN$ e considerare la successione ristretta ai soli
indici pari:
\[
  a_0, a_2, a_4, \dots
\]
Se rinumeriamo gli indici pari usando tutti i numeri naturali otteniamo
la sottosuccessione $b_k=a_{n_k}$:
\[
  b_0 = a_0,\ b_1 = a_2,\ b_2 = a_4,\ \dots,\ b_k = a_{n_k},\ \dots
\]


\begin{theorem}[cambio di variabile nei limiti]
\mymargin{cambio di variabile nei limiti}
Se $n_k$ è una successione di numeri naturali con $n_k\to +\infty$
e se $a_n$ è una qualunque successione che ammette limite
allora
\begin{equation}\label{eq:472454}
  \lim_k a_{n_k} = \lim_n a_n.
\end{equation}

Se $a_{n_k}$ è una sottosuccessione di $a_n$ allora in particolare $n_k\to +\infty$
e quindi anche in questo caso vale~\eqref{eq:472454}.
\end{theorem}
%
\begin{proof}
Le definizioni di $a_n \to \ell$ e $n_k \to +\infty$ sono le seguenti
(usiamo la notazione con gli intorni per non dover distinguere
i casi di $\ell$ finito / infinito ma si potrebbe ugualmente
procedere con le definizioni usuali).
Se indichiamo con $\vec a(n)=a_n$ e con $\vec n(k)=n_k$
le condizioni  $a_n\to \ell$ e $n_k\to +\infty$ si possono
esprimere come segue:
\begin{gather*}
 \forall U \in \B_\ell \colon \exists V \in \B_{+\infty} \colon
  \vec a (V) \subset U,\\
 \forall V \in \B_{+\infty} \colon \exists W \in \B_{+\infty} \colon
  \vec n(W) \subset V.
\end{gather*}
Mettendo insieme le due definizioni si ottiene
\[
  \forall U \in \B_\ell \colon
  \exists V \in \B_{+\infty} \colon
  \exists W \in \B_{+\infty} \colon
  \vec n(W) \subset V \text{ e }
  \vec a(V) \subset U
\]
e dunque
\[
  \forall U \in \B_\ell \colon \exists W \in \B_{+\infty} \colon
  \vec a(\vec n(W)) \subset U
\]
che corrisponde esattamente alla definizione di limite $a_{n_k} = a(n(k)) \to \ell$.

Per la seconda parte è sufficiente verificare che se $n_k$
è strettamente crescente
e a valori in $\NN$, deve necessariamente essere $n_k\to +\infty$.
Notiamo infatti che $\vec n\colon \NN\to \NN$
è iniettiva (per la stretta monotonia) e dunque rappresenta una
corrispondenza biunivoca tra il suo dominio $\NN$ e l'insieme dei
suoi valori $\{n_k \colon k \in \NN\}$. Significa dunque che l'insieme
dei valori è un insieme infinito di numeri naturali che quindi non può che
essere illimitato. Dunque $\sup n_k = +\infty$ e dalla monotonia
otteniamo $\lim n_k = \sup n_k = +\infty$.
\end{proof}



\begin{theorem}[Bolzano-Weierstrass]
\mymark{***}
\mymargin{Bolzano-Weierstrass}
\index{teorema!di Bolzano-Weierstrass}
Se $a_n$ è limitata allora esiste una sottosuccessione
$a_{n_k}$ convergente. Il teorema è valido sia per successioni reali
che complesse.
\end{theorem}
%
\begin{proof}
Facciamo dapprima il caso $a_n\in \RR$.
\mymark{***}
Sia $A_0=\inf a_n$ e $B_0=\sup a_n$. Essendo $a_n$ limitata
sia $A_0$
che $B_0$ sono finiti e ogni termine della successione sta
nell'intervallo $[A_0,B_0]$. Definiamo $n_0$: ovviamente si avrà $a_{n_0} = a_0 \in [A_0, B_0]$.

Consideriamo il punto medio $M_0 = (A_0+B_0)/2$ dell'intervallo $[A_0,B_0]$ e consideriamo i due mezzi intervalli $[A_0,M_0]$ e $[M_0,B_0]$. Tutti i termini della successione stanno in almeno
uno di questi due intervalli.
Se consideriamo gli indici $n\in \NN$ della successione $a_n$, uno dei due sotto-intervalli deve contenere termini
della successione per infiniti indici.
Chiamiamo $[A_1, B_1]$
tale sottointervallo,
chiamiamo $n_1$ il più piccolo naturale maggiore di $n_0=0$
per cui $a_{n_1} \in [A_1,  B_1]$.

Ripetiamo il procedimento.
Consideriamo il punto medio $M_1$ dell'intervallo $[A_1,B_1]$.
Per costruzione l'intervallo contiene termini della successione
per infiniti indici dunque uno dei due sotto-intervalli $[A_1,M_1]$
o $[M_1,B_1]$ deve anche lui
contenere termini della successione per infiniti indici. Chiamiamo
$[A_2, B_2]$ tale intervallo e definiamo $n_2$ come il più piccolo
naturale maggiore di $n_1$ per cui $a_{n_2}\in [A_2, B_2]$.

Si può procedere così all'infinito (formalmente: tramite una definizione per induzione)
e ottenere quindi le successioni $A_k$, $B_k$ e $n_k$ che soddisfano le seguenti proprietà (da verificare con il principio di induzione):
\begin{enumerate}
\item $A_k$ è crescente, $B_k$ è decrescente, $A_k \le B_k$;
\item $B_k - A_k = (B_0-A_0)/2^k$;
\item $a_{n_k} \in [A_k, B_k]$.
\end{enumerate}

Essendo $A_k$ monotona sappiamo che esiste $\ell = \lim A_k$. Essendo poi $A_0 \le A_k \le B_k \le B_0$ sappiamo che $A_k$ è limitata, quindi $\ell$ è finito.
Inoltre
\[
 \lim B_k = \lim A_k + \frac{B_0-A_0}{2^k} = \lim A_k = \ell
 \]
 e dunque
passando al limite nelle disuguaglianze
\[
   A_k \le a_{n_k} \le B_k
\]
si ottiene (teorema dei carabinieri)
\[
  a_{n_k} \to \ell.
\]
Questo conclude la dimostrazione per le successioni reali.

Se $a_n$ è una successione di numeri complessi, si potrà scrivere
$a_n = x_n + i y_n$ con
$x_n$ e $y_n$ successioni reali. Se $a_n$ è limitata significa che $\abs{a_n}$
è superiormente limitata. Ma risulta $\abs{x_n} \le \abs{a_n}$ e
$\abs{y_n}\le \abs{a_n}$ quindi se $a_n$ è limitata anche la parte reale
$x_n$ e la parte immaginaria $y_n$ sono successioni limitate.
Allora $x_n$ ammette una sotto-successione convergente $x_{n_k}$.
Ma $y_{n_k}$ è anch'essa limitata e quindi anch'essa ammette una
sotto-sotto-successione $y_{n_{k_j}}$ convergente.
Dunque la sotto-sotto-successione $a_{n_{k_j}}$ è convergente.
\end{proof}

\begin{corollary}\label{cor:12394}
  Sia $a_n$ una successione qualunque (reale o complessa). Allora esiste
  una sottosuccessione $a_{n_k}$ che ha limite (in $\bar \RR$ o in $\bar \CC$).

  Più precisamente se la successione è limitata allora ammette una estratta convergente,
  se è reale e superiormente illimitata allora ammette
  una estratta con limite $+\infty$,
  se è reale e inferiormente illimitata ammette
  una estratta con limite $-\infty$,
  se è complessa e illimitata ammette
  una estratta con limite $\infty$.
\end{corollary}
%
\begin{proof}
  Se $a_n$ è limitata il risultato segue dal teorema di Bolzano-Weierstrass.

  Se $a_n$ è reale e non è superiormente limitata
  allora per ogni $M\in \RR$ esiste $n$ tale che $a_n \ge M$.
  Allora scegliamo $n_0\in \NN$ in modo che $a_{n_0}\ge 0$ e poi
  definiamo, induttivamente $n_k$ in modo che
  \[
   a_{n_{k+1}}\ge \max\{k+1, a_0, a_1, \dots, a_{n_k}\}.
  \]
  Se $a_n$ è superiormente limitata è sempre possibile soddisfare la precedente
  relazione. Chiaramente $a_{n_k} \ge k \to +\infty$ e quindi abbiamo
  mostrato che esiste una sottosuccessione che tende a $+\infty$.

  In modo analogo si svolge il caso in cui $a_n$ si inferiormente illimitata.

  Se $a_n$ è una successione illimitata complessa allora significa che
  $\abs{a_n}$ è superiormente illimitata e quindi, per quanto detto sopra,
  esiste una sottosuccessione tale che $\abs{a_{n_l}}\to +\infty$.
  Ma questo significa che $a_{n_k} \to \infty$.
\end{proof}

Il seguente teorema è molto importante dal punto di vista
culturale, ma non verrà utilizzato nella teoria seguente.

\begin{theorem}[Cantor: secondo metodo diagonale]
\label{th:cantor_secondo}%
\mymargin{non numerabilità dei reali}%
\index{secondo metodo diagonale di Cantor}%
\index{Cantor}%
\index{teorema!di Cantor}%
L'insieme dei numeri reali non è numerabile: $\#\RR > \#\NN$.
\end{theorem}
%
\begin{proof}
E' sufficiente dimostrare che $\#[0,1] > \#\NN$ in quanto
chiaramente $\#\RR \ge \#[0,1]$.
Supponiamo per assurdo che esista una funzione biettiva $a\colon \NN \to [0,1]$.
Questa funzione corrisponde dunque ad una successione $a_n$.
Consideriamo l'intervallo $[0,1]$ e dividiamolo in tre intervalli  di lunghezza $1/3$: $[0,1/3]$, $[1/3,2/3]$, $[2/3,1]$. Il punto $a_0$ non può stare in tutti e tre questi intervalli. Sia $[A_0,B_0]$ un intervallo (dei tre) che non contiene $a_0$: $a_0 \not \in [A_0,B_0]$.
Dividiamo anche $[A_0,B_0]$ in tre intervalli di lunghezza $1/9$.
Almeno uno di questi tre intervalli, che chiamiamo $[A_1,B_1]$,
non contiene $a_1$: $a_1 \not \in [A_1,B_1]$.
Procediamo così all'infinito in maniera simile al teorema precedente.
Otterremo due successioni $A_n$, $B_n$ che soddisfano queste proprietà:
\begin{enumerate}
\item $A_n$ crescente, $B_n$ decrescente;
\item $0\le A_n \le B_n \le 1$;
\item $a_n \not \in [A_n, B_n]$;
\item $B_n - A_n = 1/3^{n+1}$.
\end{enumerate}

Essendo $A_n$ monotona e limitata, essa ha limite finito $\lim A_n = \ell$.
Fissato $n$ osserviamo che per ogni $k\ge n$ si ha
\[
  A_n \le A_k \le B_n
\]
passando al limite in $k$ (con $n$ fissato) si ottiene
\[
  A_n \le \ell \le B_n
\]
che significa che $\ell \in [A_n, B_n]$ per ogni $n\in \NN$
(in particolare $\ell \in [0,1]$).
Visto che invece $a_n \not \in [A_n, B_n]$ risulta che per
ogni $n\in \NN$ si ha $\ell \neq a_n$.
Dunque il numero $\ell$ non è un termine della successione $a_n$
ovvero la funzione $a\colon \NN \to [0,1]$ non è suriettiva.
\end{proof}

Il seguente lemma
non verrà usato in seguito ma può essere interessante di per sé.
Si osservi che tale lemma fornisce una dimostrazione alternativa
del teorema di Bolzano-Weierstrass in quanto ogni sottosuccessione monotona
ammette limite.

\begin{lemma}[estratte monotone]
Ogni successione $a_n\in \RR$ ammette una estratta monotona.
\end{lemma}
%
\begin{proof}
Consideriamo l'insieme $P$ dei punti di ``picco'', ovvero degli indici
di quei termini della successione che sono maggiori o uguali a tutti i termini
seguenti:
\[
  P = \{n\in \NN\colon m\ge n \implies a_n\ge a_m\}.
\]
Se $P$ è finito significa che esiste un indice $n_1\in \NN$ tale
che non ci sono picchi da $n_1$ in poi. In particolare $n_1$ non è un punto di
picco
quindi deve esistere $n_2>n_1$ tale che $a_{n_1}<a_{n_2}$.
Ma neanche $n_2$ è un punto di picco quindi deve esistere $n_3>n_2$ tale
che $a_{n_2}<a_{n_3}$... procedendo induttivamente si riesce quindi a definire
una successione $n_k$ di indici tali che $a_{n_k}$ risulta essere strettamente
crescente.

Se, viceversa, $P$ è infinito allora elencando in ordine i suoi elementi otterremo
una successione $n_1 < n_2 < n_3, \dots$ di indici ognuno dei quali corrisponde ad un valore
di picco. Se $j>k$ si ha dunque $n_j>n_k$ ed essendo $n_k\in P$ significa che
$a_{n_j} \le a_{n_k}$. Dunque la successione $a_{n_k}$ risulta essere decrescente.
\end{proof}


\section{punti limite}

\begin{proposition}[proprietà caratteristica della convergenza]
  \label{prop:convergenza}
Sia data una successione e un candidato punto limite $\ell$.
Se da ogni sottosuccessione è possibile estrarre una sotto-sottosuccessione
che ha limite $\ell$ allora l'intera successione ha limite $\ell$.
\end{proposition}
%
\begin{proof}
  Sia $a_n$ la successione.

  Se per assurdo non fosse $a_n \to \ell$
  esisterebbe un intorno $B\in \B_\ell$
  per cui frequentemente $a_n \not \in B$.
  Questo significa che esiste una sottosuccessione
  $a_{n_k} \not \in B$. Formalmente
  la successione di indici $n_k$ può essere definita
  ricorsivamente:
  \[
    \begin{cases}
      n_0 = \min \{n\in \NN \colon a_n \not \in B\}, \\
      n_{k+1} = \min \{n\in \NN \colon n>n_k, a_n \not \in B\}.
    \end{cases}
  \]
  Ma allora da $a_{n_k}$
  non è possibile estrarre una sotto-sottosuccessione
  che abbia limite $\ell$, e questo è contrario alle
  ipotesi.
\end{proof}

\begin{definition}[punti limite, limite superiore, limite inferiore]
Sia $a_n$ una successione.
Una quantità $\ell\in \bar\RR$ si dice essere
un \myemph{punto!limite} di $a_n$ se esiste una sottosuccessione
$a_{n_k}$ che ha limite $\ell$. Se chiamiamo $L\subset \bar\RR$ l'insieme
dei punti limite possiamo definire il \emph{limite superiore} e il
\emph{limite inferiore}
\mymargin{limite!superiore/inferiore}
rispettivamente come
\[
\limsup_{n\to +\infty} a_n = \sup L, \qquad
\liminf_{n\to -\infty} a_n = \inf L.
\]
Potremo anche scrivere più brevemente $\limsup_n a_n$ o anche $\limsup a_n$
al posto di $\displaystyle \limsup_{n\to +\infty} a_n$.
Lo stesso vale per il $\liminf$.
\end{definition}

\begin{example}
La successione $a_n=(-1)^n$ ha due punti limite:
\[
  \limsup_{n\to +\infty}(-1)^n = 1, \qquad \liminf_{n\to+\infty} (-1)^n = -1.
\]
Infatti la sottosuccessione dei termini con indice pari è costante $1$ mentre
quella dei termini di indice dispari è costante $-1$.
\end{example}
%
\begin{example}
  Visto che $\#\QQ=\#\NN$ esiste una funzione $\vec a \colon \NN \to \QQ$
  surgettiva. Utilizzando la proprietà di densità dei numeri razionali
  si può dimostrare che l'insieme dei punti limite della corrispondente successione
  $a_n = \vec a(n)$ è tutto $\bar \RR$.
\end{example}
%
\begin{theorem}[proprietà del limite superiore/inferiore]
Sia $a_n$ una successione reale e sia $L$ l'insieme dei suoi punti
limite. Allora
\begin{enumerate}
  \item $L\neq \emptyset$

  \item $\limsup a_n \ge \liminf a_n$;

  \item se $\limsup a_n = \liminf a_n = \ell$ allora $\lim a_n = \ell$;

  \item
  l'insieme dei punti limite è chiuso per passaggio al limite:
  se $x_k\in L$ e $x_k \to \ell$ per qualche $\ell \in \bar \RR$ allora $\ell \in L$;

  \item
  esiste una sottosuccessione $a_{n_k}$ tale che
  $a_{n_k} \to \limsup a_n$ (idem per quanto riguarda
  il $\liminf$);

  \item $a_n$ è superiormente limitata se e solo se $\limsup a_n < +\infty$;
  $a_n$ è inferiormente limitata se e solo se $\liminf a_n > -\infty$;

  \item la condizione
  $\limsup a_n = \ell$ è equivalente a
  \begin{align*}
   \forall \ell' > \ell &\colon a_n < \ell' \text{ definitivamente} \\
   \forall \ell' < \ell &\colon a_n > \ell' \text{ frequentemente}
  \end{align*}
  e la condizione $\liminf a_n = \ell$ è equivalente a
  \begin{align*}
  \forall \ell' > \ell &\colon a_n < \ell' \text{ frequentemente} \\
  \forall \ell' < \ell &\colon a_n > \ell' \text{ definitivamente}
  \end{align*}

  \item $\displaystyle \limsup_{n\to +\infty} a_n = \lim_{n\to +\infty} \enclose{\sup_{k\ge n} a_k}$,
  $\quad\displaystyle \liminf_{n\to +\infty} a_n = \lim_{n\to +\infty} \enclose{\inf_{k\ge n} a_k}$.

 \end{enumerate}
\end{theorem}
%
\begin{proof}
  Per il punto 1.\
  il corollario~\ref{cor:12394} garantisce che sia $L\neq \emptyset$ e
  quindi $\sup L \ge \inf L$ da cui discende il punto 2.

  Per il punto 3.\ se $\limsup = \liminf = \ell$ significa che $L=\{\ell\}$
  ha un solo elemento. Ma per il corollario al teorema di Bolzano Weierstrass
  sappiamo che da ogni sottosuccessione è possibile estrarre una sotto-sottosuccessione
  che ammette limite. Per definizione di $L$ tale limite deve essere in $L$
  e quindi non può che essere $\ell$. Allora per la proposizione~\ref{prop:convergenza}
  deduciamo che l'intera successione ha limite $\ell$.

  Per il punto 4.\ sia $x_k\in L$ una successione di punti limite
  e sia $\ell\in \bar \RR$ tale che $x_k \to \ell$ per $k\to +\infty$.
  Se $\ell = +\infty$ basta dimostrare che $a_n$ non può essere superiormente
  limitata. Ma questo è chiaro, perché se fosse $a_n \le M$ per qualche $M\in \RR$
  allora si avrebbe $x_k\le M$ (per i teoremi di confronto) e quindi non potrebbe
  essere $x_k \to \ell=+\infty$. Discorso analogo vale se $\ell = -\infty$.
  Supponiamo allora che sia $\ell \in \RR$.
  Per dimostrare che
  $\ell \in L$ dobbiamo trovare una sottosuccessione $a_{n_k}$ di $a_n$ tale
  che $a_{n_k}\to \ell$ per $k\to +\infty$.

  Basterà trovare una successione strettamente crescente
  di indici $n_k$ tale
  che si abbia $\abs{a_{n_k} - x_k}<\frac 1 {k}$ per ogni
  $k\ge 1$.
  Ma questo è sempre possibile farlo perché per ogni $k$
  il punto $x_k\in L$ è limite di una sottosuccessione di
  $a_n$ e quindi fissato $k$ si ha $\abs{a_n-x_k}<\frac 1 k$
  frequentemente in $n$.
  Dunque è possibile definire $n_k$ ricorsivamente
  \[
   \begin{cases}
    n_0 = 0,\\
    n_{k+1} = \min\{n\in \NN \colon n>n_k, \abs{a_n-x_{k+1}}<\frac{1}{k+1}\}.
   \end{cases}
  \]
  Ma allora si ha ovviamente
  \[
    \abs{a_{n_k} - \ell}
    \le \abs{a_{n_k} - x_k} + \abs{x_k -\ell}
    \le \frac 1 k + \abs{x_k-\ell} \to 0+0 = 0.
  \]

  Il punto 5.\ si può esprimere dicendo che $\sup L\in L$ e $\inf L \in L$.
  Visto che $L\neq \emptyset$ la caratterizzazione del $\sup$
  garantisce che esista una successione $x_n\in L$ tale che $x_n \to \sup L$.
  Ma allora per il punto precedente si deve avere $\sup L \in L$.
  Lo stesso per l'$\inf$.

  Il punto 6.\ segue dal corollario~\ref{cor:12394}: se una successione
  non è limitata superiormente (o inferiormente)
  esiste una estratta che tende a $+\infty$ (o a $-\infty$) e quindi
  $\limsup  = +\infty$ (o $\liminf = -\infty$).
  Viceversa se $\limsup = +\infty$ (o se $\liminf = -\infty$)
  per il punto 3.\ sappiamo esistere
  una sottosuccessione che tende a $+\infty$ (o a $-\infty$)
  quindi né la sottosuccessione
  né l'intera successione può essere superiormente (o inferiormente) limitata.

  Per il punto 7.\ facciamo la dimostrazione per il $\limsup$ (per il $\liminf$
  sarà analogo). Se $\limsup a_n = \ell$ e se fosse frequentemente
  $a_n \ge \ell' > \ell$ allora esisterebbe una sottosuccessione $a_{n_k}\ge \ell'$.
  Tale sottosuccessione avrebbe una sotto-sottosuccessione che ammette un limite
  $\ge \ell'> \ell$.
  Ma questo è assurdo perché essendo $\ell=\sup L$ nessuna sottosuccessione
  può avere limite maggiore di $\ell$.
  Dunque definitivamente deve essere $a_n < \ell'$ per ogni $\ell'>\ell$.
  Se invece fosse definitivamente $a_n \le \ell' < \ell$ ogni
  sottosuccessione dovrebbe avere limite non superiore ad $\ell'$
  e questo è assurdo in quanto essendo $\sup L = \ell > \ell'$
  devono esistere successioni che hanno limite maggiore di $\ell'$.

  Viceversa se per ogni $\ell'>\ell$ si ha $a_n<\ell'$ definitivamente
  significa che per ogni sottosuccessione $a_{n_k}$ si ha
  $a_{n_k} < \ell'$ definitivamente e quindi se tale sottosuccessione
  ammette limite si deve avere $\lim a_{n_k} \le \ell'$. Dunque
  ogni $\ell'>\ell$ è un maggiorante di $L$ e quindi $\sup L\le \ell$.
  Se poi per ogni $\ell'<\ell$ si ha $a_n>\ell'$ frequentemente
  significa che esiste una sottosuccessione $a_{n_k}$ che
  ha limite $\ge \ell'$. Dunque per ogni $\ell'<\ell$ risulta
  $\sup L \ge \ell'$ da cui $\sup L \ge \ell$. Dunque $\sup L=\ell$
  come volevamo dimostrare.

  Per il punto 8.\ facciamo la dimostrazione per il $\limsup$ (per il $\liminf$
  sarà analogo).
  Posto $A_n = \sup_{k\ge n} a_k$ osserviamo che $A_n$ è decrescente
  in quanto all'aumentare di $n$ l'insieme
  $\{k\colon k \ge n\}$ decresce (nel senso dell'inclusione insiemistica)
  e quindi
  il $\sup$ non aumenta.
  Dunque $\ell=\lim A_n$ esiste certamente in $\bar \RR$.
  Inoltre per definizione di limite per ogni $\ell'>\ell$
  si dovrà avere $A_n < \ell'$ definitivamente e quindi
  (essendo ovviamente $a_n \le A_n$) si avrà
  anche $a_n < \ell'$ definitivamente.
  Viceversa preso $\ell'<\ell$ si avrà definitivamente
  $A_n > \ell'$. Ma questo significa che esiste $k>n$
  per cui si ha $a_k > \ell'$ e quindi
  risulta $a_n > \ell'$ frequentemente.
  Per il punto precedente si può quindi concludere che
  $\ell = \limsup a_n$.
\end{proof}

\begin{theorem}[operazioni con $\limsup$ e $\liminf$]
Siano $a_n$ e $b_n$ due successioni reali. Si ha:
\begin{enumerate}
  \item
  $\limsup (-a_n) = -\liminf a_n$, $\qquad\liminf (-a_n) = -\limsup a_n$;

  \item
    se $\lambda \ge 0$ allora
    \[
    \limsup (\lambda \cdot a_n) = \lambda \cdot \limsup a_n, \qquad
    \liminf (\lambda \cdot a_n) = \lambda \cdot \liminf a_n;
    \]

  \item
  si ha
  \begin{gather*}
  \limsup (a_n + b_n) \le \limsup a_n + \limsup b_n,
  \\
  \liminf (a_n + b_n) \ge \liminf a_n + \liminf b_n;
  \end{gather*}

  \item
  se $a_n \ge b_n$ allora
  \[
   \limsup a_n \ge \limsup b_n, \qquad \liminf a_n \ge \liminf b_n.
  \]

\end{enumerate}
\end{theorem}
%
\begin{proof}
  Per il punto 1.\ basti osservare che se la successione $a_n$ ha $L$ come
  insieme dei punti limite allora la successione $-a_n$ ha $-L$ come punti
  limite. Quindi $\sup (-L) = -\inf L$ e $\inf(-L) = -\sup L$.

  Per il punto 2.\ si osservi che se $L$ è l'insieme dei punti limite
  di $a_n$ allora l'insieme dei punti limite di $\lambda a_n$ è $\lambda L$.
  Se $\lambda \ge 0$ si ha dunque $\sup \lambda L = \lambda \sup L$
  e $\inf \lambda L = \lambda \inf L$ (se $\lambda<0$ invece $\inf$ e $\sup$
  si scambiano, come nel punto 1).

  Per il punto 3.\ consideriamo il caso del $\limsup$ (per il $\liminf$ sarà analogo).
  Se $\ell = \limsup(a_n+b_n)$ significa che esiste una sottosuccessione
  di $a_n+b_n$ che ha limite $\ell$. Ma posso estrarre una sotto-sottosuccessione
  tale che anche il primo addendo $a_n$ abbia limite. E poi posso estrarre
  una sotto-sotto-sottosuccesione in modo che anche il secondo addendo $b_n$
  abbia limite. Dunque, a meno di sottosuccessioni, avremo $a_n \to a$, $b_n \to b$
  con $a+b=\ell$.
  Ma allora per definizione di $\limsup$ si avrà $\limsup a_n \ge a$
  e $\limsup b_n \ge b$ da cui $\limsup a_n + \limsup b_n \ge a+b = \ell$.

  Per il punto 4.\ si osserva che se $\ell = \limsup a_n$ allora
  per ogni $ell'>\ell$ si ha definitivamente
    $a_n<\ell'$ e di conseguenza anche $b_n < \ell'$ definitivamente.
    Dunque $\limsup b_n \le \ell$.
    Se invece poniamo $\ell = \liminf b_n$ allora
    per ogni $\ell'<l$ si ha definitivamente $b_n \ge \ell'$ ma
    allora anche $a_n \ge \ell'$ defitiviamente e quindi
    $\liminf a_n \ge \ell$.
\end{proof}

\section{il teorema degli zeri}

\begin{theorem}[degli zeri]
\mymark{***}%
\mynote{teorema degli zeri}%
\index{teorema!degli zeri}%
\label{th:zeri}%
Sia $I\subset \RR$ un intervallo, $f\colon I \to \RR$ una funzione
continua, $a,b\in I$ tali che $f(a)\le 0$ e $f(b)\ge 0$.
Allora esiste $c\in I$ tale che $f(c)=0$.
\end{theorem}

\begin{proof}
\mymark{***}
La dimostrazione che adottiamo è di particolare rilevanza in quanto
non solo permette di dimostrare l'esistenza del punto $c$ che risolve
$f(x)=0$
ma ci presenta
un algoritmo, il \myemph{metodo!di bisezione},
\index{bisezione!metodo di}
che può essere effettivamente utilizzato per approssimare
tale soluzione.

Possiamo supporre senza perdere di  generalità che sia $a<b$.
Poniamo $A_0 = a$, $B_0= b$ e consideriamo il punto medio $C_0 = (A_0+B_0)/2$.
Scegliamo tra i due intervalli $[A_0, C_0]$ e $[C_0,B_0]$ quello per cui
il segno ai due estremi è discorde (o, caso fortunato, nullo).
Più precisamente se $f(C_0)\ge 0$ poniamo $[A_1,B_1] = [A_0,C_0]$ altrimenti
scegliamo $[A_1,B_1] = [C_0,B_0]$ così si ha, in ogni caso,
$f(A_1)\le 0$, $f(B_1)\ge 0$.

Consideriamo il punto medio $C_1$ del nuovo intervallo $[A_1,B_1]$ e ripetiamo
il procedimento indefinitamente. Quello che otteniamo sono due successioni
$A_n$, $B_n$ con queste proprietà (che potrebbero essere dimostrate per induzione):
\begin{enumerate}
\item $A_n < B_n$, $B_n - A_n = (b-a)/2^n$;
\item $A_n$ è crescente, $B_n$ è decrescente;
\item $f(A_n)\le 0$, $f(B_n)\ge 0$.
\end{enumerate}

Essendo $A_n$ monotòna sappiamo che $A_n$ converge $A_n\to c$.
Inoltre visto che $A_n \in [a,b]$ anche $c\in [a,b]$ (per la permanenza del
segno delle successione $A_n-a$ e $b-A_n$).
Passando al limite nell'uguaglianza $B_n = A_n + (b-a)/2^n$
si ottiene che anche $B_n \to c$. Essendo $f$ continua
avremo
\[
f(A_n) \to f(c), \qquad
f(B_n) \to f(c).
\]
Ma $f(A_n)\le 0$ e quindi per la permanenza del segno anche $f(c)\le 0$.
D'altra parte $f(B_n) \ge 0$ e quindi $f(c)\ge 0$.
Si ottiene dunque $f(c) = 0$, come volevamo dimostrare.
\end{proof}

\begin{example}\label{ex:75445}
Si voglia risolvere l'equazione
\[
  x^5-x-1=0.
\]
Posto $f(x) = x^5-x-1$ è chiaro che la funzione $f\colon \RR\to \RR$
è continua (in quanto composizione di funzioni continue).
Osserviamo che $f(0) = -1$ e $f(2)=29$, dunque la funzione
soddisfa le ipotesi del teorema degli zeri sull'intervallo $[0,2]$.
Sappiamo quindi che l'equazione in questione ha almeno una soluzione
in tale intervallo.

Utilizzando il metodo di bisezione possiamo determinare una soluzione
con precisione arbitraria. Posto $A_0=0$, $B_0=2$ abbiamo verificato che
$f(A_0)<0$ e $f(B_0)>0$.
Prendiamo il punto
medio $C_0=1$ e calcoliamo la funzione: $f(C_0)=-1 < 0$. Sappiamo
allora che una soluzione deve essere compresa nell'intevallo
$[A_1,B_1] = [C_0,B_0] = [1,2]$ perché anche in tale intervallo valgono le ipotesi
del teorema degli zeri.
Il punto medio di tale intervallo è $C_1=3/2 = 1.5$
e risulta $f(3/2) = 163/32>0$ dunque l'intervallo successivo
che andremo a considerare è $[A_2,B_2]=[1,3/2]$.
Per non dover lavorare con troppe cifre decimali invece di suddividere
esattamente a metà quest'ultimo intervallo consideriamo un punto
intermedio $C_2 = 6/5 = 1.2$ dove si ha $f(C_2)=901/3125>0$.
Sappiamo allora che una soluzione è compresa nell'intervallo
$[A_3,B_3] = [1,1.2]$. Prendiamo il punto medio $C_3=11/10=1.1$
e troviamo $f(C_3) = -48949/10^5 <0$. Abbiamo quindi ottenuto
che esiste $x\in (1.1,1.2)$ tale che $f(x)=0$. Sappiamo quindi
che $\abs{x-1.15} < 0.05$ cioè abbiamo trovato $x$ con un errore
inferiore a $0.05$.

Con molta pazienza si può procedere
con il metodo di bisezione fino ad arrivare a verificare
che $f(116/10^2)$ $=$ $-596583424/10^{10}<0$ e $f(117/10^2)=224480357/10^{10}>0$ da cui
si ottiene che una soluzione è compresa tra $1.16$ e $1.17$ con un errore
inferiore a $0.005$.
Con il calcolatore (si veda ad esempio il codice a pagina \pageref{code:bisection})
si possono ottenere più cifre significative: $x=1.1673039782614187\ldots$
\end{example}

\begin{table}
\begin{center}
\begin{tabular}{r}
1.\small
4142135623 7309504880 1688724209 6980785696 7187537694 \\ \small
8073176679 7379907324 7846210703 8850387534 3276415727 \\ \small
3501384623 0912297024 9248360558 5073721264 4121497099 \\ \small
9358314132 2266592750 5592755799 9505011527 8206057147 \\ \small
0109559971 6059702745 3459686201 4728517418 6408891986 \\ \small
0955232923 0484308714 3214508397 6260362799 5251407989 \\ \small
6872533965 4633180882 9640620615 2583523950 5474575028 \\ \small
7759961729 8355752203 3753185701 1354374603 4084988471 \\ \small
6038689997 0699004815 0305440277 9031645424 7823068492 \\ \small
9369186215 8057846311 1596668713 0130156185 6898723723 \\ \small
5288509264 8612494977 1542183342 0428568606 0146824720 \\ \small
7714358548 7415565706 9677653720 2264854470 1585880162 \\ \small
0758474922 6572260020 8558446652 1458398893 9443709265 \\ \small
9180031138 8246468157 0826301005 9485870400 3186480342 \\ \small
1948972782 9064104507 2636881313 7398552561 1732204024 \\ \small
5091227700 2269411275 7362728049 5738108967 5040183698 \\ \small
6836845072 5799364729 0607629969 4138047565 4823728997 \\ \small
1803268024 7442062926 9124859052 1810044598 4215059112 \\ \small
0249441341 7285314781 0580360337 1077309182 8693147101 \\ \small
7111168391 6581726889 4197587165 8215212822 9518488472
  \end{tabular}
\end{center}
\caption{Le prime 1000 cifre decimali del numero $\sqrt 2$
calcolate con il metodo di bisezione usato nella dimostrazione
del teorema~\ref{th:zeri}.
Si veda il codice a pagina~\pageref{code:bisection}.}
\label{fig:cifre_sqrt2}
\index{$\sqrt 2$!cifre decimali}
\index{cifre!$\sqrt 2$}
\end{table}

\begin{corollary}[proprietà dei valori intermedi]
\mymark{**}
\mymargin{proprietà!dei valori intermedi}
Sia $I\subset \RR$ un intervallo e $f\colon I \to \RR$ una
funzione continua.
Allora se $f$ assume due valori $y_1$ e $y_2$ allora $f$
assume anche tutti i valori intermedi tra $y_1$ e $y_2$.
Detto altrimenti: una funzione continua
manda intervalli in intervalli.
\end{corollary}
%
\begin{proof}
Se $y_1$ e $y_2$ sono valori assunti da $f$ significa
che esistono $x_1,x_2 \in I$ tali che $f(x_1)= y_1$ e $f(x_2)=y_2$.
Allora scelto $y$ si consideri la funzione $g(x) = f(x)-y$.
Se $y$ è intermedio tra $y_1$ e $y_2$ la funzione $g$ assumerà
segni opposti in $x_1$ e $x_2$ e dunque, per il teorema degli zeri,
dovrà esserci un punto $x$ in cui $g$ si annulla. In tale punto
si avrà dunque $f(x)=y$, come volevamo dimostrare.
\end{proof}

\begin{lemma}
Se $I$ è un intervallo di $\RR$ ogni funzione $f\colon I \to \RR$
iniettiva e continua è strettamente monotona.
\end{lemma}
%
\begin{proof}
Si può osservare che una funzione è strettamente monotona se mantiene i valori
intermedi cioè se dati tre punti
$x<y<z$ risulta sempre che $f(y)$ è un valore intermedio tra $f(x)$ e $f(z)$:
\[
  f(x)< f(y) <f(z) \qquad\text{oppure} \qquad f(x)> f(y) > f(z).
\]
Se ciò non accadesse, ad esempio se fosse $f(y)>f(z)>f(x)$ con $x<y<z$
allora per la continuità di $f$ dovrebbe esistere un valore intermedio
tra $x$ e $y$ in cui la funzione assume il valore $f(z)$. Ma allora la funzione
non sarebbe iniettiva.
\end{proof}

\begin{lemma}[caratterizzazione delle funzioni monotone continue]
Sia $I\subset\RR$ un intervallo e sia $f\colon I \to \RR$ una
funzione crescente. Allora sono equivalenti
\begin{enumerate}
\item $f$ è continua;
\item per ogni $x\in I$
  \begin{enumerate}
    \item se $x\neq \inf I$ allora $f(x) = \sup f(\{y \in I\colon y<x\})$,
    \item e se $x\neq \sup I$ allora $f(x) = \inf f(\{y \in I \colon y>x\})$.
  \end{enumerate}
\end{enumerate}

Risultato analogo vale per le funzioni decrescenti.
\end{lemma}
%
\begin{proof}
Sia $A_x = \{y \in I\colon y<x \}$ e $B_x = \{y\in I \colon y>x\}$.
Se $f$ è crescente risulta sempre
\[
f(A_x) \le f(x) \le f(B_x)
\]
in quanto se $a\in A_x$ e $b\in B_x$ allora $a<x<b$ e quindi $f(a)\le f(x) \le f(b)$. Dunque
\[
\sup f(A_x) \le f(x) \le \inf f(B_x).
\]

Supponiamo che $f$ sia continua e consideriamo un
qualunque $x\in I$.
Se $x\neq \inf I$ dobbiamo mostrare che $\sup f(A_x)\ge f(x)$.
La successione $x_n = x - 1/n$ sta in $A_x$ per $n$
sufficientemente grande, dunque $\sup f(A_x) \ge f(x_n)$.
Ma visto che $f(x_n)\to f(x)$ per confronto si ottiene $\sup f(A_x) \ge f(x)$.
Analogamente se $x\neq \sup I$ prendendo la successione $x_n = x+1/n$
si trova che $\inf f(B_x) \le f(x)$.

Supponiamo ora di avere $x\in I$ che non è un estremo di $I$ e che valga
$\sup f(A_x) = f(x) = \inf f(B_x)$.
Per ogni $\eps>0$, per la caratterizzazione di $\sup$ e $\inf$
dovranno allora esistere $y\in A_x$ e $z\in B_x$ tali che
$f(x) = \sup f(A_x) < f(y) +\eps$ e $f(x) = \inf f(B_x) > f(z)-\eps$.
Se $x_n\in I$ e $x_n \to x$ per $n$ abbastanza grande si dovrà avere
$y < x_n < z$ e quindi, per la monotonia di $f$: $f(y)\le f(x_n)\le f(z)$
da cui
\[
  f(x)-\eps < f(x_n) < f(x)+\eps.
\]
Significa allora che $f(x_n)\to f(x)$.

Se $x$ fosse un estremo di $I$, ad esempio se $x=\inf I$, si ripete
lo stesso ragionamento ma solo sul lato destro di $x$: per ogni $\eps>0$
esisterà $z\in B_x$ tale che $f(x) = \inf f(B_x) \ge f(z)-\eps$.
Ma se $x_n\in I$ dovrà essere $x_n\ge x$ (in quanto $x$ è l'estremo inferiore
di $I$) e quindi si avrà comunque
\[
  f(x) \le f(x_n) < f(x) + \eps
\]
da cui segue, per l'arbitrarietà di $\eps$, $f(x_n)\to f(x)$.
\end{proof}

\begin{theorem}[continuità della funzione inversa]
  \label{th:inversa_continua}
\mymark{*}
\mynote{continuità della funzione inversa}
\index{continuità!della funzione inversa}
Sia $I\subset \RR$ un intervallo e sia $f\colon I \to \RR$ una
funzione continua strettamente crescente.
Allora posto $J=f(I)$ anche $J$ è un intervallo, $f\colon I\to J$ è
invertibile e $f^{-1}\colon J\to I$ è anch'essa
continua e strettamente crescente.

Risultato analogo vale per le funzioni strettamente decrescenti.
\end{theorem}
%
\begin{proof}
Che $J=f(I)$ sia un intervallo segue direttamente dal teorema dei valori intermedi, essendo $f$ continua. Essendo $f$ strettamente
crescente $f$ risulta essere iniettiva e quindi $f\colon I \to J$ è biettiva.
Esiste dunque la funzione inversa $f^{-1}\colon J \to I$.

Mostriamo ora che la stretta monotonia di $f^{-1}$ segue dalla stretta
monotonia di $f$. Presi $y_1 < y_2$ in $J$ poniamo $x_1=f^{-1}(y_1)$ e $x_2=f^{-1}(y_2)$ cosicché $y_1=f(x_1)$ e $y_2=f(x_2)$.
Se per assurdo fosse $f^{-1}(y_1)\ge f^{-1}(y_2)$ si avrebbe $x_1 \ge x_2$
e quindi $f(x_1)\ge f(x_2)$ cioè $y_1\ge y_2$, contro l'ipotesi $y_1 < y_2$.

Per mostrare che $f^{-1}$ è continua utilizziamo
il lemma precedente.
Dato $y \in J$ con $y\neq \inf J$, consideriamo l'insieme
$A'_y=\{t \in J\colon t<y\}$.
Essendo $f$ monotona e invertibile,
posto $x=f^{-1}(y)$ e $A_x=\{ s\in I\colon s<x\}$
si ha $f^{-1}(A'_y) = A_x$ e chiaramente $\sup A_x = x$.
Dunque $\sup f^{-1}(A'_y) = x = f^{-1}(y)$. In maniera analoga
si dimostra che $\inf f^{-1}(\{t\in J \colon t>y\}) = f^{-1}(y)$
quando $y\neq \sup J$. Dunque $f^{-1}$ è continua,
come volevamo dimostrare.
\end{proof}

\begin{exercise}\label{ex:inversa_monotona}
Sia $f\colon I\to J$ una funzione continua e invertibile definita tra due
intervalli $I\subset \RR$ e $J\subset \RR$. Dimostrare che $f$ è strettamente
monotona (e quindi $f^{-1}$ è continua).
\end{exercise}

%%%%%%%%%%%%%%%%%%%%%%%%
%%%%%%%%%%%%%%%%%%%%%%%%
%%%%%%%%%%%%%%%%%%%%%%%%
%%%%%%%%%%%%%%%%%%%%%%%%
\section{il teorema di Weierstrass}

\begin{definition}[massimi e minimi assoluti]
\mymark{**}
Sia $A$ un insieme e sia $f\colon A \to \RR$.
Il \emph{massimo} $M$ e il \emph{minimo} $m$
\mymargin{massimo/minimo}%
\index{massimo}%
\index{minimo}%
di $f$ su $A$ sono,
se esistono, i valori
\begin{align*}
  M &= \max f(A) = \max\{f(x)\colon x \in A\} \\
  m &= \min f(A) = \min\{f(x)\colon x \in A\}.
\end{align*}
Si indicano anche con
\mymargin{$\max$, $\min$}
\index{$\max$}
\index{$\min$}
\[
  M = \max_{x\in A} f(x), \qquad m = \min_{x\in A} f(x).
\]

Diremo inoltre che $x_0$ è un \emph{punto di massimo assoluto}
per $f$ se $f(x_0) = M$ e diremo che $x_0$ è un \emph{punto di minimo assoluto} per $f$ se $f(x_0) = m$.
\mymargin{massimo/minimo assoluto}
\index{punto!di massimo assoluto}
\index{punto!di minimo assoluto}
\index{massimo!assoluto}
\index{minimo!assoluto}

Equivalentemente $x_0$ è un massimo assoluto se $f(x_0) \ge f(x)$ per ogni $x\in A$ e invece $x_0$ è un minimo assoluto se $f(x_1) \le f(x)$ per ogni $x\in A$.
\end{definition}

Si faccia attenzione a non confondere i \emph{valori} massimo e minimo con i \emph{punti} di massimo e minimo. I valori sono elementi della immagine della funzione, i punti sono elementi del dominio.

\begin{example}
Si consideri la funzione $f\colon \RR \to \RR$ definita da $f(x) = \frac{1}{1+x^2}$.
Per ogni $x$ si ha $x^2\ge 0$ quindi $1+x^2\ge 1$ e dunque $f(x)\le 1$.
Inoltre si ha $f(0)=1$. Dunque $1$ è il massimo della funzione
e il punto $x=0$ è un punto di massimo (in questo caso l'unico).

Ma chiaramente $f(x)>0$ per ogni $x\in \RR$.
Inoltre $f(n) \to 0$ per $n\to +\infty$ e quindi $\inf f(\RR) = 0$.
Se $f$ avesse minimo si dovrebbe avere $\min f(\RR) = \inf f(\RR) = 0$ ma $0$ non può essere minimo in quanto $f(x) >0$ per ogni $x$.
Risulta quindi che la funzione $f$ non ha minimo.
\end{example}

\begin{lemma}[successioni minimizzanti/massimizzanti]
\mynote{successioni mi\-ni\-miz\-zan\-ti/mas\-si\-miz\-zan\-ti}
\index{successioni!minimizzanti}
\index{successioni!massimizzanti}
Sia $A$ un insieme non vuoto e
sia $f\colon A \to \RR$ una funzione. Allora esistono
due successioni $a_n$ e $b_n$ di punti di $A$ tali che
\[
  \lim_{n\to +\infty} f(a_n) = \inf f(A), \qquad
  \lim_{n\to +\infty} f(b_n) = \sup f(A).
\]
\end{lemma}
%
\begin{proof}
Ricordiamo che $f(A) = \{f(x)\colon x \in A\}$ è l'immagine
della funzione $f$. Facciamo la dimostrazione per l'estremo inferiore,
risultato analogo si potrà ottenere per l'estremo superiore.

Sia $m=\inf f(A)$.
Se $m=-\infty$ significa che $f(A)$ non è inferiormente limitato, in particolare per ogni $n\in \RR$ esiste $a_n$ tale che $f(a_n) < - n$.
Dunque (per confronto) $f(a_n) \to -\infty$ come volevamo dimostrare.

Se $m\in \RR$ per le proprietà caratterizzanti l'estremo inferiore sappiamo che per ogni $\eps>0$ esiste $a\in A$ tale che $f(a) < m + \eps$.
Per ogni $n\in\NN$ possiamo scegliere $\eps=1/n$ e ottenere quindi una successione $a_n$ tale che $f(a_n) < m + 1/n$.
D'altra parte essendo $m$ un minorante di $f(A)$ sappiamo che $m \le f(a_n)$. Abbiamo dunque $m \le f(a_n) < m+ 1/n$ e per il teorema dei carabinieri possiamo quindi concludere che $f(a_n) \to m$ per $n\to +\infty$.
\end{proof}

\begin{theorem}[Weierstrass]
\mymark{***}
\mynote{teorema di Weierstrass}
\index{teorema!di Weierstrass}
Siano $a,b\in \RR$ con $a\le b$
e sia $f\colon [a,b]\to \RR$ una funzione continua.
Allora
esistono punti di massimo e di minimo per $f$ su $[a,b]$.
\end{theorem}
%
\begin{proof}
\mymark{***}
Dimostriamo solamente che $f$ ha minimo, per il massimo la dimostrazione procede
infatti in maniera del tutto analoga.

Sia $m= \inf f([a,b])$.
Per il lemma precedente sappiamo che esiste una successione $a_n$ minimizzante ovvero tale che
$a_n \in A$ e $f(a_n)\to m$ per $n\to +\infty$.

Per il teorema di Bolzano-Weierstrass dalla successione $a_n$ possiamo estrarre una sottosuccessione $a_{n_k}$ convergente: $a_{n_k} \to x_0$.
Visto che $a_{n_k} \in [a,b]$ si avrà, per il teorema della permanenza del segno, anche $x_0 \in [a,b]$ (si applichi la permanenza del segno alle successioni $a_{n_k}-a$ e $b-a_{n_k}$).

Dunque abbiamo una successione $a_{n_k}\to x_0$ con $a_{n_k}\in [a,b]$ e
$x_0 \in [a,b]$. Essendo $f$ continua si avrà dunque $f(a_{n_k}) \to f(x_0)$.
Ma noi sapevamo che $f(a_n)\to m$ e dunque anche $f(a_{n_k}) \to m$.
Concludiamo quindi che $f(x_0) = m$ cioè $m$, l'estremo inferiore,
è un valore assunto dalla funzione ed è quindi un minimo.
Dal canto suo $x_0$ è un punto di minimo assoluto.
\end{proof}

\begin{corollary}[limitatezza delle funzioni continue]
Sia $f\colon [a,b]\to \RR$ una funzione continua. Allora $f$ è limitata.
\end{corollary}
\begin{proof}
Visto che $f$ ha massimo $M$ e minimo $m$ si ha $f(x)\in [m,M]$ per ogni $x\in[a,b]$.
Ovviamente $m>-\infty$ e $M<+\infty$ in quanto $m$ e $M$ sono valori della funzione $f$.
\end{proof}

%%%%%%%%%%%%%
%%%%%%%%%%%%%
%%%%%%%%%%%%%
\section{potenze e radici $n$-esime}
%%%%%%%%%%%%%
%%%%%%%%%%%%%
%%%%%%%%%%%%%
\index{potenza}
\index{radice}

\begin{theorem}[invertibilità della funzione potenza]
\mymark{*}
\mymargin{invertibilità della funzione potenza}
Per ogni $n\in \NN\setminus\{0\}$, la funzione
\begin{align*}
  f\colon [0,+\infty) &\to [0,+\infty)\\
   x &\mapsto x^n
\end{align*}
è strettamente crescente e biettiva.
Inoltre, se $n$ è dispari, la funzione
\begin{align*}
  f\colon \RR & \to \RR\\
    x &\mapsto x^n
\end{align*}
è strettamente crescente e biettiva.
\end{theorem}
%
\begin{proof}
La funzione $f$ su $[0,+\infty)$ è strettamente crescente perché
per l'assioma di monotonia del prodotto di numeri reali si
osserva che
se $x>y\ge 0$ allora $x^2 > y^2$, $x^3 > y^3$ e così via
(la dimostrazione andrebbe formalizzata, al solito, utilizzando il principio di
induzione).
Essendo strettamente crescente $f$ è anche iniettiva.
Per mostrare che $f\colon [0,+\infty) \to [0,+\infty)$ è suriettiva, consideriamo
un qualunque $y\in[0,+\infty)$ e prendiamo la funzione $g(x)=f(x)-y$.
Chiaramente $g(0) = 0^n - y = -y \le 0$. Se $y\le 1$ allora $g(1) = 1^n - y \ge 0$
altrimenti, se $y>1$, si ha $y^n>y$ e quindi $g(y) = y^n-y \ge 0$.
In ogni caso abbiamo verificato che esistono $a,b$ tali che $g(a)\le 0$ e $g(b)\ge 0$

Osserviamo che, per il teorema sul limite del prodotto,
se $x_k \to x$ allora $x_k^2 = x_k\cdot x_k \to x\cdot x = x^2$.
Questo dimostra che la funzione $x^2$ è continua.
Per induzione si dimostra che $x^n$ è continua per ogni $n\in \NN$.
Per il teorema sul limite della differenza risulta che anche $g$ è
continua.
Dunque possiamo applicare il teorema degli zeri
per determinare l'esistenza di un $x\in[0,+\infty)$ tale che $g(x)=0$.
Visto che $x$ risolve l'equazione $f(x)=y$ e la surgettività è dimostrata.

Se $n$ è dispari si ha $(-x)^n = -(x^n)$.
In particolare per $x<0$ si ha $x^n<0$. Dunque se $x> y \ge 0$ si
ha $-x < -y \le 0$ e
\[
  (-x)^n = -x^n < -y^n \le 0.
\]
E' quindi facile verificare che la funzione $x^n$
risulta strettamente crescente e bigettiva su tutto $\RR$.
\end{proof}

\begin{definition}[radice $n$-esima]
\mymark{***}
\mymargin{radice $n$-esima}
Se $n\in \NN$ è pari e non nullo
chiamiamo $\sqrt[n]{x}$ la funzione
inversa di $x^n$ come funzione $[0,+\infty)\to [0,+\infty)$.
Se $n\in \NN$ è dispari chiamiamo $\sqrt[n]{x}$
la funzione inversa di $x^n$ come funzione $\RR \to \RR$.
\end{definition}

Se ricordiamo come era stata definita la radice quadrata $\sqrt x$ nel teorema Teorema~\ref{th:radice_quadrata} ci rendiamo conto che $\sqrt {x} = \sqrt[2]{x}$.
La radice seconda viene usualmente chiamata radice quadrata e
analogamente la radice terza viene usualmente chiamata \emph{radice cubica}.

\begin{theorem}[proprietà della radice $n$-esima]
\mymargin{proprietà!della radice}
Per ogni $n,m \in \NN\setminus\{0\}$, $a\ge 0$, $b>0$ si ha:
\begin{enumerate}
\item $\sqrt[n]{a^n} = a$;
\item $\displaystyle \sqrt[nm]{a} = \sqrt[n]{\sqrt[m]{a}}$;
\item $\displaystyle \sqrt[n]{a\cdot b} = \sqrt[n]{a}\sqrt[n]{b}$;
\item $\displaystyle \sqrt[n]{\frac a b} = \frac{\sqrt[n]{a}}{\sqrt[n]{b}}$;
\item la funzione $x\mapsto \sqrt[n]{x}$ è continua e strettamente crescente.
\end{enumerate}
\end{theorem}

\begin{proof}
Le proprietà della radice vengono dedotte dalle corrispondenti proprietà
della potenza intera, sfruttando il fatto che la radice è la funzione inversa.
L'ultima proprietà discende dal teorema~\ref{th:inversa_continua}.
\end{proof}

\begin{theorem}[disuguaglianza di Bernoulli]
\mymark{**}
\mynote{disuguaglianza di Bernoulli}
\index{Bernoulli}
\index{disuguaglianza!di Bernoulli}
Se $x > -1$ e $n\in \NN$ si ha
\[
(1+x)^n \ge 1 + nx.
\]
\end{theorem}
%
\begin{proof}
\mymark{**}
Lo dimostriamo per induzione su $n$. Per $n=0$, sostituendo si ottiene $1\ge 1$.
Supponendo che sia verificata la disuguaglianza per un certo $n$:
\[
(1+x)^n \ge 1 + nx
\]
moltiplicando ambo i membri per $1+x > 0$ si ottiene
\[
(1+x)^{n+1} \ge (1+x) (1+nx) = 1 + (n+1)x + n x^2
\ge 1 + (n+1)x
\]
che è proprio quello che volevamo dimostrare.
\end{proof}

\begin{theorem}[limite della radice $n$-esima]
\label{th:limite_radice}
\mymark{**}
\mymargin{limite!della radice $n$-esima}
Se $a>0$
\[
   \lim_{n\to +\infty} \sqrt[n]{a} = 1
\]
\end{theorem}
%
\begin{proof}
\mymark{*}
Consideriamo innanzitutto il caso $a\ge 1$.
Posto $x=\sqrt[n]{a}-1$ nella disuguaglianza di Bernoulli, si ha
\[
a
= (1+x)^n
\ge 1 + nx
= 1 + n (\sqrt[n]{a}-1)
\]
da cui
\[
 \sqrt[n]{a} \le 1 + \frac{a-1}{n} \to 1 + 0 = 1.
\]
D'altra parte se $a\ge 1$ si ha $\sqrt[n]{a} \ge 1$
e dal confronto tra limiti si ottiene la tesi.

Se $a<1$ basta osservare che
\[
\sqrt[n]{a} = \frac{1}{\sqrt[n]{\frac 1 a}} \to 1
\]
per il caso precedente applicato con $1/a$ al posto di $a$.
\end{proof}

Siamo ora intenzionati a definire le potenze $x^y$ con esponente $y\in \RR$.
La funzione $f(x) = a^x$ si ottiene dal seguente teorema.

\begin{theorem}[funzione esponenziale]
  \label{th:esponenziale}
\mynote{funzione esponenziale}
\index{funzione!esponenziale}
Per ogni $a>0$ esiste una unica funzione $f\colon \RR \to \RR$
con le seguenti proprietà:
\begin{enumerate}
\item $f(1) = a$;
\item per ogni $x,y\in \RR\colon f(x+y) = f(x)\cdot f(y)$;
\item $f$ è monotòna.
\end{enumerate}
Tale funzione risulta inoltre essere positiva, e
continua, soddisfa la relazione $f(-x)= 1/f(x)$ e
se $p\in \ZZ$ e $q\in \NN\setminus\{0\}$ risulta
\[
  f\enclose{\frac p q} = \sqrt[q]{a^p}.
\]
Inoltre se $a>1$ tale funzione è strettamente crescente,
se $a<1$ tale funzione è strettamente decrescente e se $a=1$
tale funzione è costante.
\end{theorem}
%
\begin{proof}
Dimostriamo innanzitutto che $f$ non si può annullare mai.
Infatti per ogni $x\in \RR$
\[
   a = f(1) = f(x+1-x) = f(x)\cdot f(1-x)
\]
e dunque se fosse $f(x)=0$ si avrebbe $a=0$ che abbiamo escluso per ipotesi.
Possiamo anzi dire che $f$ non è mai negativa in quanto
\[
  f(x) = f(x/2 + x/2) = f(x/2)^2 \ge 0.
\]
Osserviamo anche che
\[
  f(0) = f(0+0) = f(0)\cdot f(0)
\]
da cui, dividendo per $f(0)$ si ottiene $f(0)=1$.
E' anche facile verificare, per induzione, che
per ogni $n\in \NN$ si ha
\[
 f(nx) = (f(x))^n
\]
infatti $f((n+1)x) = f(nx+x)=f(nx)\cdot f(x)$.
In particolare se $n\in \NN$ si ha
\[
  f(n) = f(1)^n = a^n.
\]
Ma poi
\[
1=f(0) = f(x-x) = f(x)\cdot f(-x)
\]
e quindi $f(-x)=1/f(x)$ e quindi $f(nx) = (f(x))^n$ per ogni $n\in \ZZ$.
In particolare se $p\in \ZZ$ e $q \in \NN\setminus\{0\}$ si ha
\[
 a^p = f(p) = f(q\cdot p/q) = (f(p/q))^q
\]
da cui
\begin{equation}\label{eq:def_pow_0}
 f(p/q) = \sqrt[q]{a^p}.
\end{equation}

Dunque la funzione $f(x)$ è univocamente determinata per ogni $x\in \QQ$.
Osserviamo che, per le proprietà delle radici, se $p/q = n/m$ allora
\[
 \sqrt[m]{a^n}
 = \sqrt[mp]{a^{np}}
 = \sqrt[nq]{a^{np}}
 = \sqrt[q]{a^p}
\]
e questo significa che $f$ può effettivamente essere definita
coerentemente su tutto $\QQ$
tramite
la \eqref{eq:def_pow_0}.

Supponiamo d'ora in poi che sia $a>1$
e verifichiamo che in tal caso $f$ deve essere strettamente crescente su $\QQ$.
Se $x,y \in \QQ$
con $x<y$ si avra $x=p/q$, $y=n/m$ con $pm < nq$ allora
$a^{pm} < a^{nq}$ e quindi
\[
f(x)
= \sqrt[q]{a^p}
= \sqrt[qm]{a^{pm}}
< \sqrt[qm]{a^{nq}}
= \sqrt[m]{a^n}
= f(y).
\]

Definiamo ora
\begin{align*}
  A_x &= \{t\in \QQ\colon t<x\} = \QQ \cap (-\infty,x),
  \\
  B_x &= \{t\in \QQ\colon t>x\} = \QQ \cap (x,+\infty).
\end{align*}
Ovviamente $A_x < x < B_x$ (nel senso che per ogni $\alpha \in A_x$
e per ogni $\beta \in B_x$ si ha $\alpha < x < \beta$).
Siccome vogliamo che $f$ sia crescente si dovrà avere
$f(A_x) \le f(x) \le f(B_x)$ da cui in particolare
$\sup f(A_x) \le f(x) \le \inf f(B_x)$.

Vogliamo ora mostrare che deve essere
$\sup f(A_x) = \inf f(B_x)$
cosicché $f(x)$ sarà univocamente determinata
per ogni $x\in \RR$ da:
\begin{equation}\label{eq:def_pow_2}
  f(x) = \sup f(A_x) = \inf f(B_x).
\end{equation}
(ricordiamo infatti che su $\QQ$ $f$ è già stata univocamente determinata
e $A_x$ e $B_x$ sono sottoinsiemi di $\QQ$).

Dato qualunque $x\in \RR$ esiste $k\in \NN$ tale che $x < k$.
Scelto comunque $n \in \NN$, per la densità dei razionali
esistono $y,z\in \QQ$ tali che
\[
  y < x < z < k
\]
e tali che $z-y< \frac 1 n$. Allora si avrà
\begin{align*}
  0 \le f(z) - f(y)
  &= f(y)\cdot\enclose{\frac{f(z)}{f(y)}-1}
  \le f(k)\cdot \enclose{f(z-y)-1}\\
  &\le f(k)\cdot \enclose{f(1/n)-1}.
\end{align*}
Chiaramente $y\in A_x$ e $z\in B_x$ dunque
\[
\inf f(B_x) - \sup f(A_x) \le f(z) - f(y) \le f(k) \cdot (f(1/n)-1).
\]
Ricordiamo ora che $f(1/n) = \sqrt[n]{a} \to 1$ (teorema precedente)
dunque il lato destro della precedente disuguaglianza può essere reso
minore di qualunque $\eps>0$
e quindi, come voluto,
dovrà essere $\inf f(B_x) = \sup f(A_x)$.

Mostriamo ora che $f(x)$ deve essere strettamente crescente su tutto $\RR$,
ricordando
che abbiamo già verificato che $f$ è strettamente crescente su $\QQ$.
Ma presi $x,y\in \RR$ con $x<y$ esistono $z,w\in \QQ$ tale che $x<z<w<y$.
Allora $z\in B_x$ e $w\in A_y$, dunque
\[
  f(x) = \inf f(B_x) \le f(z) < f(w) \le \sup A_y = f(y).
\]

Il teorema di caratterizzazione delle funzioni monotone e continue
ci assicura che le condizione $f(x) = \sup A_x$ e $f(x)= \inf B_x$
garantiscono la continuità di $f$.

Infine, la relazione
\[
  f(x+y) = f(x) \cdot f(y)
\]
è verificata se $x,y \in \QQ$ (abbiamo costruito $f$ su $\QQ$ in modo
che questa relazione fosse valida).
Ma se $x,y \in \RR$ per densità esisteranno $x_n, y_n\in \QQ$ tali
che $x_n\to x$ e $y_n\to y$. Allora passando al limite
nella relazione
\[
  f(x_n+y_n) = f(x_n)\cdot f(y_n)
\]
sfruttando la continuità di $f$ si ottiene il risultato voluto.

Nel caso $a<1$ si procede in maniera analoga, la funzione risulterà strettamente
 decrescente su $\QQ$ e si potrà estendere su tutto $\RR$ mantenendo
 la monotonia. Nel caso $a=1$ si vede che la funzione $f$ è costante $f(x)=1$
 su $\QQ$ e quindi
per mantenere la monotonia l'unico modo per estendere $f$ a tutto $\RR$ è quello
di mantenere la stessa costante.
\end{proof}

\begin{definition}[potenze con esponente reale]
\index{potenza!con esponente reale}
Sia $a\in \RR$, $a>0$ e $x\in \RR$.
Definiamo $a^x$ come
l'unica funzione definita dal teorema precedente.
\mymargin{$a^x$}
\end{definition}

\begin{exercise}
Si dimostri che $a^{xy} = (a^x)^y$
e $(a\cdot b)^x = a^x \cdot b^x$.
(sfruttare l'unicità data dal teorema della funzione esponenziale).
\end{exercise}

Osserviamo che abbiamo dato due diverse definizioni di $x^y$.
La prima è quella delle potenze intere, valida se $x\in \RR$ e $y\in \ZZ$
e se $x\neq 0$ quando $y<0$. La seconda, quella delle potenze con esponente
reale è valida per $x>0$ e $y\in \RR$. Il teorema precedente ci garantisce
che le due definizioni coincidono quando $x>0$ e $y\in \ZZ$.

\begin{theorem}[limite dell'esponenziale]
\label{th:limite_esponenziale}
\mymargin{limite!dell'esponenziale}
Sia $a>1$.
%Sia $a>1$ e $b>0$.
\begin{enumerate}
\item se $x_n \to +\infty$ allora $a^{x_n} \to +\infty$;
\item se $x_n \to -\infty$ allora $a^{x_n} \to 0$;
% \item se $a_n \to +\infty$ allora $(a_n)^b \to +\infty$.
\end{enumerate}
\end{theorem}
%
\begin{proof}
La disuguaglianza di Bernoulli garantisce che
\[
  a^n = (1+(a-1))^n \ge 1 + n(a-1).
\]
Ma allora se $x_n \to +\infty$
\[
  a^{x_n} \ge a^{\lfloor x_n \rfloor} \ge 1 + \lfloor x_n \rfloor (a-1)
  \ge 1 + (x_n-1)\cdot(a-1)
  \to +\infty.
\]
Se $x_n \to -\infty$ allora $-x_n \to +\infty$ ed essendo
$a^{-x_n} = 1/ a^{x_n}$ si ottiene il risultato voluto.
\end{proof}

%%%%%%%%%%%%%%%
%%%%%%%%%%%%%%%
%%%%%%%%%%%%%%%
\section{il logaritmo}
%%%%%%%%%%%%%%%
%%%%%%%%%%%%%%%
%%%%%%%%%%%%%%%

Fissato $a>1$ consideriamo la funzione esponenziale
$f\colon \RR \to \RR$, $f(x) = a^x$.
Sappiamo che $f$, essendo strettamente crescente, è iniettiva.
Il teorema sulla funzione esponenziale ci dice che $f(x)>0$ e quindi
$f(\RR) \subset (0,+\infty)$.
Visto però che $a^n\to +\infty$ e $a^{-n} \to 0$ scopriamo che
$\sup f(\RR)=+\infty$ e $\inf f(\RR)=0$.
Per la continuità di $f$ risulta però che $f(\RR)$ sia un intervallo
e dunque necessariamente $f(\RR) = (0,+\infty)$.
Dunque $f\colon \RR \to (0,+\infty)$ è biettiva.
Lo stesso vale nel caso $0<a<1$ (semplicemente per il fatto che $a^x = (1/a)^{-x}$).

\begin{definition}[logaritmo]
Per $a>0$, $a\neq 1$ definiamo $\log_a\colon (0,+\infty)\to \RR$,
chiamato \emph{logaritmo in base $a$}\mymargin{logaritmo},
la funzione inversa di $a^x$.
Si ha dunque
\[
  \log_a x = y \iff a^y = x.
\]
\end{definition}

\begin{theorem}[proprietà del logaritmo]
\mymargin{proprietà!del logaritmo}
Se $a>0$, $a\neq 1$
\begin{enumerate}
  \item $\log_a(a^x) = x$;
  \item $\log_a(x\cdot y)= \log_a x + \log_a y$;
  \item $\log_a(x^y) = y \log_a (x)$;
  \item $\log_a(1) = 0$, $\log_a(a)=1$;
  \item  $\displaystyle  \log_a x = \frac{\log_b x}{\log_b a}$
  (qualunque sia $b>0$, $b\neq 1$);
  \item se $a>1$ la funzione $\log_a(x)$ è strettamente crescente,
  se $0<a<1$ è strettamente decrescente;
  \item $\log_a(x)$ è una funzione continua;
  \item se $a>1$ e $x_n \to +\infty$ allora $\log_a(x_n)\to +\infty$;
  \item se $a>1$ e $x_n \to 0$, $x_n>0$ allora $\log_a(x_n)\to -\infty$.
\end{enumerate}
%
\begin{proof}
Tutte queste proprietà si ricavano direttamente dalle analoghe proprietà
dell'esponenziale.
\end{proof}
\end{theorem}

Per completezza enunciamo il seguente teorema che però
probabilmente non conviene memorizzare.
L'idea veramente rilevante, che si usa nella dimostrazione,
è il fatto che quando in una potenza variano sia la base
che l'esponente conviene riscrivere la potenza fissando
una base $c>1$ qualunque
tramite la seguente identità:
\[
  a^b = c^{b\cdot \log_c a}.
\]
In questo modo la base rimane fissata e quello
che varia è solo l'esponente.

\begin{theorem}[limite della potenza]
\mymargin{limite!della potenza}
Siano $a_n$ e $b_n$ successioni
Se $a_n\to a$ e $b_n \to b$
sono successioni convergenti e $a>0$
allora
\[
  (a_n)^{b_n} \to a^b.
\]
Inoltre se $a_n \to a$, $a_n>0$, $b_n \to b$:
\begin{enumerate}
\item se $a<1$ e $b=+\infty$ allora ${a_n}^{b_n} \to 0$;
\item se $a<1$ e $b=-\infty$ allora ${a_n}^{b_n} \to +\infty$;
\item se $a>1$ e $b=+\infty$ allora ${a_n}^{b_n} \to +\infty$;
\item se $a>1$ e $b=-\infty$ allora ${a_n}^{b_n} \to 0$;
\item se $a=+\infty$ e $b>0$ allora ${a_n}^{b_n} \to +\infty$.
\end{enumerate}

\end{theorem}
%
\begin{proof}
Scelto un qualunque numero $c>1$ (ad esempio $c=2$)
si ha
\[
  a_n^{b_n} = c^{\log_c a_n^{b_n}}
   = c^{b_n \cdot \log_c a_n}.
\]

Per la continuità del logaritmo si ha $\log_c a_n \to \log_c a$
(oppure $\log_c a_n \to +\infty$ se $a=+\infty$),
per il teorema sul limite del prodotto si ha $b_n\cdot \log_c a_n \to b\cdot \log_c a$
e infine, per la continuità dell'esponenziale si ottiene
\[
  c^{b_n \cdot \log_c a_n} \to c^{b\log_c a} = a^b.
\]

Negli altri casi si procede con la stessa dimostrazione
e si osserva che il prodotto $b_n \cdot \log a_n$ non risulta essere
una forma indeterminata.
\end{proof}

Rimangono esclusi i seguenti casi (forme indeterminate):
\begin{enumerate}
\item $a=0$, $b=0$ forma indeterminata ``$0^0$'';
\item $a=1$, $b=+\infty$ forma indeterminata ``$1^{+\infty}$'';
\item $a=1$, $b=-\infty$ forma indeterminata ``$1^{-\infty}$'';
\item $a=+\infty$, $b=0$ forma indeterminata ``$(+\infty)^0$''.
\end{enumerate}

%%%%%%%%%%%%%%%%%
%%%%%%%%%%%%%%%%%
%%%%%%%%%%%%%%%%%
%%%%%%%%%%%%%%%%%
\section{la costante di Nepero}

La funzione esponenziale è legata ad un modello di crescita che si trova spesso
in natura: la \myemph{crescita esponenziale}.
Prendiamo come esempio una popolazione di batteri che cresce senza
limitazioni di spazio e di nutrimento oppure
alla crescita di un capitale dovuto ad una rendita finanziaria.

Supponiamo che una popolazione che al tempo $t_0=0$
ammonta ad un certo numero $c$ di batteri, al tempo
$t>0$ raggiunga una numerosità $q(t,c)$.
Se lascio crescere la popolazione per un ulteriore
tempo $s>0$ troverò al tempo $t+s$ la stessa
popolazione che avrei al tempo $s$ se al tempo
zero fossi partito con la popolazione $q(t)$:
\[
  q(t+s,c) = q(s,q(t,c)).
\]
L'equazione precedente si chiama proprietà
di \emph{semigruppo}
\index{semigruppo}%
continuo.
Ma fissato $t$ la popolazione $q(t,c)$ deve
essere proporzionale a $c$ perché ogni batterio
ha la sua discendenza indipendentemente dalla numerosità
totale della popolazione. In pratica
si deve avere $q(t,c) = k(t) \cdot c$ per una opportuna
funzione $k(t)$ che non dipende da $c$.
Dunque
\[
  q(t+s,c)
  = q(s,q(t,c))
  = k(s) \cdot q(t,c)
  = k(s) \cdot k(t) \cdot c
\]
da cui
\[
  k(t+s) = k(s) \cdot k(t).
\]
In base al teorema~\ref{th:esponenziale}
possiamo affermare
che $k(t)$ è una funzione esponenziale $k(t)=a^t$
per una qualche costante $a$.
La costante $a$ può essere determinata mediante la formula:
\[
  a = k(1) = \frac{q(1,c)}{c}
\]
ma questa espressione non ha un preciso significato fisico in quanto
dipende dall'unità di tempo scelta.

La costante a cui possiamo dare significato è invece l'aumento relativo
istantaneo della popolazione. Possiamo infatti supporre che
se lasciamo la popolazione crescere per un tempo $\Delta t$ molto piccolo,
si otterrà un aumento di popolazione proporzionale al tempo $\Delta t$
(stiamo qui anticipando il concetto di derivata) e alla popolazione:
\begin{equation}\label{eq:488464}
  q(\Delta t,c) = c + r c \Delta t = (1+r \Delta t) c.
\end{equation}
La costante $r$ rappresenta quindi l'aumento
relativo istantaneo della popolazione (nel caso dell'investimento
$r$ sarebbe il tasso di interesse istantaneo).
Questa definizione ha senso
quando $\Delta t$ è piccolo in quanto non tiene conto del fatto che
nell'intervallo di tempo $[t,t+\Delta t]$ la popolazione che si
è aggiunta genera anch'essa nuova popolazione (ovvero l'interesse
accumulato genera anch'esso interesse\footnote{%
In effetti la scoperta della costante $e$
è dovuta a Jacob Bernoulli nel 1683.
Bernoulli si chiedeva qual è l'interesse annuo effettivo
che si ottiene da un capitale che dia una rendita
giornaliera (interesse composto).
Se un capitale è investito ad un tasso di interesse $r$,
si intende che ogni giorno il capitale
viene aumentato di un fattore $1+r/365$
e quindi dopo un anno il fattore moltiplicativo è $(1+r/365)^365$.
}).

Per calcolare l'aumento della popolazione su tempi ``grandi'' possiamo
suddividere gli intervalli temporali in $n$ intervallini di ampiezza
$\Delta t$ e applicare in ognuno di essi la relazione \eqref{eq:488464}.
Si trova:
\begin{align*}
 q(\Delta t,c) &= c(1+r\Delta t) \\
 q(2\Delta t,c) &= q(\Delta t,c) (1+r\Delta t)  = c(1+r\Delta t)^2\\
 &\vdots \\
 q(n\Delta t) &= c(1+r\Delta t)^n.
\end{align*}
Dunque, ponendo $\Delta t=t/n$ si ha
\[
  q(t,c) = c\enclose{1+r\frac{t}{n}}^n
\]
in particolare per $t=1/r$ si ottiene:
\[
  q(1/r,c) = c\enclose{1+\frac{1}{n}}^n
\]
e ricordando che $q(t,c)=c a^t$ otteniamo:
\[
  a^{\frac 1 r} = \enclose{1+\frac{1}{n}}^n.
\]
Se per $n\to +\infty$ (che corrisponde a $\Delta t \to 0$)
la quantità sul lato destro tende ad un numero $e$ (che chiameremo costante
di Nepero) avremo allora
\[
  a = e^r, \qquad q(t,c) = c e^{rt}
\]
che è la relazione che lega le due costanti $a$ e $r$ che definiscono
la crescita esponenziale.

Risulta in effetti valido il seguente.

\begin{theorem}[costante di Nepero]
\mymark{**}
La successione
\[
  a_n = \enclose{1+\frac 1 n}^n
\]
è crescente e limitata, dunque è convergente.
\end{theorem}
%
\begin{proof}
Dimostriamo innanzitutto che $a_n$ è crescente, cioè che
per ogni $n\ge 2$ si ha $a_n \ge a_{n-1}$.
E' chiaro che $a_n>0$ per ogni $n$,
quindi ci riconduciamo a
verificare che $\frac{a_n}{a_{n-1}} \ge 1$.

Si ha
\begin{align*}
\frac{a_n}{a_{n-1}}
&= \frac{\enclose{1+\frac 1 n}^n}{\enclose{1+\frac 1 {n-1}}^{n-1}}
= \frac{\enclose{\frac{n+1}{n}}^n}{\enclose{\frac{n}{n-1}}^{n-1}}\\
&= \enclose{\frac{n+1}{n}\cdot\frac{n-1}{n}}^n \cdot \frac{n}{n-1}
= \enclose{\frac{n^2- 1}{n^2}}^n \cdot \frac{n}{n-1}
\end{align*}
Osserviamo ora che la disuguaglianza di Bernoulli garantisce
\[
  \enclose{\frac{n^2 -1}{n^2}}^n
  = \enclose{1-\frac{1}{n^2}}^n
  \ge 1 - \frac{n}{n^2} = 1 - \frac{1}{n} = \frac{n-1}{n}
\]
da cui si ottiene, come volevamo, $a_n / a_{n-1} \ge 1$ cioè
$a_n$ è crescente.

Se ora consideriamo la successione
\[
  b_n = \enclose{1+\frac 1 n}^{n+1}
\]
osserviamo che si ha
\[
  b_n = \enclose{1+\frac 1 n}^n \cdot \enclose{1+\frac 1 n}
   = a_n\cdot \enclose{1+\frac 1 n} > a_n.
\]
Per dimostrare che $a_n$ è limitata sarà quindi sufficiente dimostrare
che $b_n$ è superiormente limitata. Vedremo ora che $b_n$ è decrescente (e quindi $a_n \le b_n \le b_1$ è superiormente limitata).

Procediamo in maniera analoga a quanto fatto per $a_n$:
\begin{align*}
\frac{b_{n-1}}{b_n}
& = \frac{\enclose{1+\frac{1}{n-1}}^n}{\enclose{1+\frac{1}{n}}^{n+1}}
  = \frac{\enclose{\frac{n}{n-1}}^n}{\enclose{\frac{n+1}{n}}^{n+1}}
  = \enclose{\frac{n}{n-1}\cdot\frac{n}{n+1}}^n\cdot\frac{n}{n+1} \\
& = \enclose{\frac{n^2}{n^2-1}}^n \frac{n}{n+1}
  = \enclose{1 + \frac{1}{n^2-1}}^n \frac{n}{n+1}.
\end{align*}
In base alla disuguaglianza di Bernoulli otteniamo
\[
  \enclose{1 + \frac{1}{n^2-1}}^n
  \ge 1 + n \frac{1}{n^2-1}
  \ge 1 + n\frac{1}{n^2}  = 1+\frac{1}{n} = \frac{n+1}{n}.
\]
Mettendo insieme le due stime si ottiene dunque $b_{n-1}/b_n \ge 1$
che è quanto ci rimaneva da dimostrare.
\end{proof}

E' quindi giustificata la seguente.

\begin{definition}[costante di Nepero]
\mymark{***}
Definiamo la \emph{costante di Nepero}%
\mynote{costante di Nepero}%
\index{costante!di Nepero}%
\index{$e$}%
\footnote{Nepero è l'italianizzazione del nome del
matematico scozzese John Napier (1550-1617).
Il nome $e$ è stato introdotto da Eulero (Leonhard Euler 1707-1783)
%http://eulerarchive.maa.org//docs/originals/E853.pdf
}
\[
  e = \lim_{n\to +\infty} \enclose{1+\frac 1 n}^n.
\]
\end{definition}

Sapendo che
\[
  \enclose{1+\frac 1 n}^n \le e \le \enclose{1+\frac 1 n}^{n+1}
\]
e ponendo $n=1$ otteniamo $2\le e \le 4$.

\begin{exercise}\label{ex:4876765}
Posto $a_n = \frac{n^n}{n!}$ mostrare che $\frac{a_{n+1}}{a_n} \to e$.
\end{exercise}

\begin{theorem}[limiti che si riconducono al numero $e$]
\mymargin{limiti che si riconducono al numero $e$}
Se $a_k \to 0$, $a_k\neq 0$ allora, per $k\to +\infty$,
\[
  \enclose{1+a_k}^{\frac 1 {a_k}} \to e.
\]
\end{theorem}
%
\begin{proof}
Caso 1. Se $a_k > 0$ allora consideriamo la successione di naturali
$n_k = \lfloor 1/a_k \rfloor$ cosicché
\[
  n_k \le \frac{1}{a_k} \le n_k + 1
\qquad
\text{e}
\qquad
 \frac{1}{n_k+1} \le a_k \le \frac 1 {n_k}.
\]
dunque
\[
\enclose{1+\frac 1 {n_k+1} }^{n_k}
  \le \enclose{1+a_k}^{\frac 1 {a_k}}
  \le \enclose{1+\frac 1 {n_k}}^{n_k + 1}.
\]
Osserviamo ora che essendo che per $k\to +\infty$ anche
$n_k \to +\infty$ si deve
avere (in base al teorema di cambio di variabile nei limiti)
\[
 \lim_k \enclose{1 + \frac{1} {n_k}}^{n_k+1}\!\!\!
 = \lim_n \enclose{1+ \frac 1 n}^{n+1}\!\!\!
 = \lim_n \enclose{1+ \frac 1 n}^n\!\cdot\enclose{1+\frac 1 n}
 = e
\]
ed essendo anche $n_k + 1 \to +\infty$
\[
  \lim_k \enclose{1 + \frac 1 {n_k+1}}^{n_k}
  = \lim_n \enclose{1+\frac 1 n}^{n-1}
  = \lim_n \frac{\enclose{1+\frac 1 n}^{n}}{1+\frac 1 n}
  = e.
\]
Dunque, per confronto tra i limiti, si ottiene $(1+a_k)^{\frac 1 {a_k}}\to e$.

Caso 2. Se $a_k<0$ possiamo scrivere $a_k = -\abs{a_k}$ da cui
\[
  \enclose{1+a_k}^{\frac 1 {a_k}}
  = \frac{1}{\enclose{1-a_k}^{\frac 1{a_k}}}
\]
e, procedendo come nel caso precedente, ci si riconduce al limite
\[
  \lim_n \frac{1}{\enclose{1-\frac 1 n}^n}.
\]
Per quest'ultimo osserviamo che si ha
\[
 \frac{1}{\enclose{1-\frac 1 n}^n}
 = \frac{1}{\enclose{\frac{n-1}{n}}^n}
 = \enclose{\frac{n}{n-1}}^n
 = \enclose{1 + \frac{1}{n-1}}^n
\]
che per quanto visto in precedenza (mettendo $n+1$ al posto di $n$)
ha anch'esso limite $e$.

Caso generale. Se $a_k$ ha segno variabile posso considerare
le due sottosuccessione dei termini di segno positivo e dei termini di segno
negativo. Per quanto visto nei casi precedenti entrambe le successioni
convergono ad $e$ e quindi è immediato verificare che l'intera
successione converge ad $e$.
\end{proof}

\begin{corollary}
Per ogni $x\in \RR$ si ha
\[
  \lim_{n\to +\infty} \enclose{1+ \frac x n}^n = e^x.
\]
\end{corollary}
%
\begin{proof}
Infatti, per il teorema precedente, posto $a_n = x/n$ si ha
\[
\lim_{n\to +\infty}\enclose{1+\frac x n}^{\frac n x} = e.
\]
Ma allora
\[
\enclose{1+ \frac x n}^n = \enclose{\enclose{1+\frac x n}^{\frac n x}}^x
\to e^x
\]
\end{proof}

\begin{definition}[logaritmi naturali]
Vedremo che il numero $e$ risulta essere una base naturale per la funzione
esponenziale e di conseguenza per il logaritmo. Il logaritmo in base
$e$ viene chiamato \myemph{logaritmo naturale} e viene indicato con $\ln = \log_e$.
\end{definition}

In alcuni testi si utilizza l'operatore $\log$, indicato senza una base esplicita,
ma la definizione non è completamente condivisa.
In certi testi (per lo più in ambito matematico)
si definisce $\log  = \ln = \log_e$,
in altri testi si considera $\log = \log_{10}$.

\begin{corollary}[limite notevole]
Se $a_n \to 0$, $a_n>0$ allora
\[
 \lim_{n\to +\infty} \frac{\ln \enclose{1+ a_n}}{a_n} = 1.
\]
\end{corollary}
%
\begin{proof}
Per ricondursi al teorema precedente basta osservare che
\[
  \frac{\ln(1+a_n)}{a_n}
  = \ln \enclose{(1+a_n)^{\frac 1 {a_n}}}.
\]
\end{proof}

\begin{exercise}
Mostrare che
\[
  \lim n^n\cdot \enclose{\frac{n+1}{n^2+1}}^n = e.
\]
\end{exercise}

\begin{exercise}
Mostrare che
\[
\lim n\cdot \ln\enclose{1 + \frac 1 n} = 1.
\]
\end{exercise}

%%%%%%%%%%%%%%%%%%%
%%%%%%%%%%%%%%%%%%%
%%%%%%%%%%%%%%%%%%%
%%%%%%%%%%%%%%%%%%%
\section{ordini di infinito}

\begin{theorem}[criterio della radice]
\label{th:criterio_radice}
\mymark{***}
\mynote{criterio della radice}
\index{criterio!della radice per successioni}
Sia $a_n$ una successione a termini positivi tale che
\[
  \sqrt[n]{a_n} \to \ell
\]
con $\ell \in \bar \RR$.
Allora se $\ell<1$ si ha $a_n \to 0$ se invece $\ell > 1$ si ha $a_n \to +\infty$.
\end{theorem}
%
\begin{proof}
\mymark{**}
Consideriamo prima il caso $\ell < 1$.
Se $\lim \sqrt{a_n} = \ell$ significa che per ogni $\eps>0$ la successione
$\sqrt{a_n}$ risulta definitivamente minore di $\ell +\eps$.
Scegliendo opportunamente $\eps$ (ad esempio $\eps = (1-\ell)/2$) si potrà
avere $q = \ell+\eps < 1$. Dunque avremo definitivamente $\sqrt[n]{a_n}< q$
ovvero $a_n < q^n$. Per ipotesi $a_n\ge 0$
e quindi, tolto un numero finito di termini, si ottiene $0 \le a_n < q^n \to 0$
da cui $a_n \to 0$ (in quanto l'aver tolto un numero finito di termini non
cambia né il carattere né il limite della successione).

Se $\ell>1$ si potrà procedere in maniera analoga. Esisterà $q$ con $1 < q < \ell$ tale che definitivamente $\sqrt[n]{a_n} > q$ da cui $a_n > q^n \to +\infty$.
\end{proof}

\begin{proof}[Dimostrazione alternativa]
  Si può osservare che
  \[
    a_n = \enclose{\sqrt[n]{a_n}}^n
     = e^{n \cdot \ln \sqrt[n]{a_n}}.
  \]
  Se $\ell <1$ allora il logaritmo tende ad un numero negativo,
  l'argomento dell'esponenziale tende a $-\infty$ e quindi l'esponenziale tende a zero. Se invece $\ell>1$ il logaritmo tende ad un numero positivo e quindi l'esponenziale tende a $+\infty$.
\end{proof}

\begin{theorem}[criterio del rapporto]
\label{th:criterio_rapporto}
\index{criterio!del rapporto per le successioni}
\index{rapporto!criterio del}
  Sia $a_n$ una successione reale a termini non negativi
  $a_n \ge 0$ tale che esista il limite del rapporto di due termini consecutivi:
  \[
     \frac{a_{n+1}}{a_n} \to \ell \in \bar \RR.
  \]
  Se $\ell < 1$ allora $a_n \to 0$, se $\ell >1$ allora $a_n \to +\infty$.
\end{theorem}
%
\begin{proof}
Supponiamo sia $\ell<1$. Posto $q=(1+\ell)/2$ si ha $q < \ell < 1$ e posto $\eps=q-\ell>0$ per la definizione di limite $\frac{a_{n+1}}{a_n}\to \ell$ dovrà esistere un $N\in \NN$ tale
che per ogni $n\ge N$ si abbia:
\[
  \frac{a_{n+1}}{a_n} < \ell + \eps = q
\]
ovvero $a_{n+1} < q \cdot a_n$. In particolare si avrà:
\begin{align*}
  a_{N+1} &< q \cdot a_N \\
  a_{N+2} &< q \cdot a_{N+1} < q^2\cdot a_N \\
  a_{N+3} &< q \cdot a_{N+2} < q^3\cdot a_N \\
  \vdots
\end{align*}
ed è chiaro che per induzione potremo dimostrare che per
ogni $k\in \NN$ si ha
\[
  a_{N+k} < q^k\cdot a_N.
\]
Osserviamo però che $q^k \cdot a_N \to 0$ per $k\to +\infty$
in quanto $q<1$ e quindi $q^k \to 0$. Dunque, tolti i primi $N$ termini, la successione $a_n$ tende a zero. Ma i primi $N$ termini non influenzano né il carattere né il limite della successione e quindi l'intera successione $a_n$ tende a zero.

Il caso $\ell>1$ si fa in maniera analoga. Si sceglie $q$ tale
che $1<q<\ell$ e si trova, in maniera analoga al caso precedente,
che per un certo $N\in \NN$ e per ogni $k\in \NN$ si ha
\[
  a_{N+k} > q^k \cdot a_N \to +\infty.
\]
\end{proof}

Osserviamo che, nel teorema precedente (ma anche nel criterio della radice),
non si può concludere alcunché nel
caso in cui sia $\ell = 1$.
Ad esempio le due successioni $a_n = 1/n$ e $b_n = n$
hanno limiti diversi ($a_n \to 0$, $b_n\to +\infty$) ma per entrambe
il limite del rapporto di termini consecutivi tende ad $\ell=1$.

\begin{exercise}
Mostrare che
\[
  \lim \frac{n^n}{n!} = +\infty.
\]
Posto $a_n = 2^n / n!$ possiamo provare ad applicare il criterio del rapporto.
Ci si riconduce all'esercizio~\ref{ex:4876765} dove abbiamo già visto
che $a_{n+1} / a_n \to e$.
Visto che $e>1$ per il criterio del rapporto possiamo affermare che
$a_n \to +\infty$.
\end{exercise}

\begin{theorem}[Cesàro, relazione tra rapporto e radice]
\label{th:criterio_cesaro}
\mymark{*}
\mynote{criterio del rapporto alla Cesàro}
\index{Cesàro}
\index{criterio!del rapporto alla Cesàro}
Sia $a_n$ una successione a termini positivi.
Se
\[
  \frac{a_{n+1}}{a_n} \to \ell \in \RR
\]
allora
\[
 \sqrt[n]{a_n}\to \ell.
\]
\end{theorem}
%
\begin{proof}
Chiaramente deve essere $\ell\ge 0$. Supponiamo inizialmente
che sia $\ell>0$ e scegliamo qualunque $\eps>0$, con $\eps < \min\{1,\ell\}$.
Per la definizione di limite $a_{n+1}/a_n \to \ell$ esisterà $N$ tale che per $n>N-1$ si abbia
\[
  \ell -\eps < \frac{a_{n+1}}{a_n} < \ell + \eps.
\]
Osserviamo che in generale, per ogni $n\ge N$ si ha
\[
 a_n  = \frac{a_n}{a_{n-1}}\cdot\frac{a_{n-1}}{a_{n-2}}
 \cdot\cdots\cdot
     \frac{a_{N+2}}{a_{N+1}}\cdot\frac{a_{N+1}}{a_N}
     \cdot a_N.
\]
Dunque potendo stimare,
dall'alto e dal basso,
ognuno degli $n-N$ fattori di questo
prodotto di rapporti, si ottiene
\[
(\ell -\eps)^{n-N} \cdot a_N < a_n < (\ell +\eps)^{n-N} \cdot a_N
\]
da cui
\[
(\ell -\eps)^{\frac{n-N}{n}} \cdot \sqrt[n]{a_N}
< \sqrt[n]{a_n}
< (\ell +\eps)^{\frac{n-N}{n}} \cdot \sqrt[n]{a_N}
\]
ovvero
\begin{equation}\label{eq:39846}
(\ell -\eps) \cdot \sqrt[n]{\frac{a_N}{(\ell -\eps)^N}}
< \sqrt[n]{a_n}
< (\ell +\eps) \cdot \sqrt[n]{\frac{a_N}{(\ell -\eps)^N}}.
\end{equation}
Ricordiamo ora che per $n\to +\infty$ si ha (teorema~\ref{th:limite_radice})
\[
\sqrt[n]{\frac{a_N}{(\ell -\eps)^N}} \to 1
\]
dunque passando al $\liminf$ e al $\limsup$ nelle disuguaglianze \eqref{eq:39846}
si ottiene
\[
 \ell-\eps \le \liminf_n \sqrt[n]{a_n}\le \limsup_n \sqrt[n]{a_n}\le \ell+\eps.
\]
Essendo questo vero per ogni $\eps>0$ si ottiene
\[
\ell \le \liminf_n \sqrt[n]{a_n}\le \limsup_n \sqrt[n]{a_n}\le \ell
\]
dunque le disuguaglianze sono uguaglianze, $\liminf$ e $\limsup$ coincidono
dunque il limite esiste ed è $\ell$, come volevamo dimostrare.
\end{proof}

\begin{exercise}
Si applichi il risultato precedente per
verificare che
\[
   \lim \sqrt[n]{n} = 1
\]
e (più difficile)
\[
  \lim \frac{n}{\sqrt[n]{n!}} = e.
\]
\end{exercise}

Si osservi che il teorema~\ref{th:criterio_cesaro} fornisce una dimostrazione
alternativa del teorema~\ref{th:criterio_rapporto} in quanto lo riconduce
al teorema~\ref{th:criterio_radice}.

\begin{definition}[ordine di infinito/infinitesimo]
\mymark{***}
\mymargin{ordine di infinito/infinitesimo}
Se $a_n$ e $b_n$ sono successioni a termini positivi, diremo che
per $n\to +\infty$ la successione $a_n$ è \emph{molto più piccola}
della successione $b_n$ e scriveremo $a_n \ll b_n$ se vale
\mymargin{$\ll$}
\[
\frac{a_n}{b_n} \to 0.
\]
Se $a_n \ll b_n$ diremo anche che $b_n$ è \emph{molto più grande}
di $a_n$ e scriveremo $b_n \gg a_n$.
\mymargin{$\gg$}
\end{definition}

\begin{theorem}[ordini di infinito]
\mymargin{ordini di infinito}
\mymark{***}
Per ogni $a>1$, $\alpha>0$ si ha, per $n\to +\infty$
\[
\log_a n \ll n^\alpha \ll a^n \ll n! \ll n^n.
\]
Più in generale se $x_n \to +\infty$
è una qualunque successione si ha,
per $n\to+\infty$
\[
\log_a(x_n) \ll (x_n)^\alpha \ll a^{x_n}.
\]
\end{theorem}
%
\begin{proof}
\mymark{**}
Cominciamo col mostrare che $a^n \ll n!$
applicando il criterio del rapporto alla successione $\frac{a^n}{n!}$:
\[
\frac{\displaystyle \frac{a^{n+1}}{(n+1)!}}{\displaystyle \frac{a^n}{n!}}
= \frac{a^{n+1}}{a^n}\cdot \frac{n!}{(n+1)!}
= a \cdot \frac {1}{n + 1} \to 0 < 1.
\]
Dunque si ha, come richiesto $a^n / n! \to 0$.
Si procede in modo analogo per mostrare che $n! \ll n^n$:
\begin{align*}
\frac{(n+1)!}{n!}\cdot \frac{n^n}{(n+1)^{n+1}}
&= (n+1) \cdot \enclose{\frac{n}{n+1}}^n \frac {1}{n+1}\\
&= \frac{1}{\enclose{1+\frac 1 n}^n} \to \frac 1 e < 1.
\end{align*}

Per dimostrare che
$n^\alpha \ll a^n$
si può procedere con il criterio del rapporto, come nei casi precedenti:
\[
\frac{(n+1)^\alpha}{n^\alpha}\cdot \frac{a^n}{a^{n+1}}
= \frac 1 a \cdot \enclose{\frac{n+1}{n}}^\alpha \to \frac 1 a \cdot 1^\alpha = \frac 1 a < 1
\]
da cui $n^\alpha / a^n \to 0$.

Se ora $x_n\to +\infty$ è qualunque
cerchiamo di ricondurci ad una successione a valori interi.
Osserviamo che si ha
\[
\lfloor x_n \rfloor
\le x_n
\le \lfloor x_n \rfloor + 1
\]
da cui, per monotonia,
\[
\lfloor x_n \rfloor^\alpha
\le x_n^\alpha
\le (\lfloor x_n \rfloor + 1)^\alpha
= \lfloor x_n \rfloor^\alpha \enclose{1+ \frac{1}{\lfloor x_n \rfloor}}^\alpha
\]
e
\[
a^{\lfloor x_n \rfloor}
\le a^{x_n}
\le a^{\lfloor x_n \rfloor + 1}
= a \cdot a^{\lfloor x_n \rfloor}.
\]
Dunque
\[
\frac{\lfloor x_n \rfloor^\alpha}{a \cdot a^{\lfloor x_n \rfloor}}
\le \frac{x_n^\alpha}{a^{x_n}}
\le \frac{\lfloor x_n \rfloor^\alpha \enclose{1+ \frac{1}{\lfloor x_n \rfloor}}^\alpha}
    {a^{\lfloor x_n \rfloor}}.
\]
Ma ora, se $n\to +\infty$ sapendo che $\lfloor x_n\rfloor \to +\infty$ si ha
che (per il teorema di sostituzione del limite)
\[
\lim \frac{\lfloor x_n \rfloor^\alpha}{a^{\lfloor x_n \rfloor}} = 0
\qquad
\text{e}
\qquad
\lim \frac{\lfloor x_n \rfloor^\alpha }
    {a^{\lfloor x_n \rfloor}} = 0
\]
da cui segue che $\frac{x_n^\alpha}{a^{x_n}}\to 0$.

Per dimostrare l'ultima relazione, $\log_a(x_n)\ll (x_n)^\alpha$,
consideriamo la successione $y_n = \alpha \cdot \log_a x_n$
cosicché $a^{y_n} = x_n^\alpha$.
Notiamo che se $x_n\to +\infty$
anche $y_n \to +\infty$.
Dunque, per le proprietà precedenti,
sappiamo che $y_n \ll a^{y_n}$ e dunque
\[
\frac{\log_a x_n}{x_n^\alpha}
= \frac{1}{\alpha}\cdot\frac{y_n}{a^{y_n}} \to 0.
\]
\end{proof}

\begin{exercise}
Calcolare i seguenti limiti
\begin{gather*}
  \lim_{n\to +\infty} \frac{\displaystyle \ln\sqrt{n^2+n^n}}
  {\displaystyle e^{1 + \ln n}\cdot \ln(n^2-n\sqrt n)}, \\
  \lim_{n\to +\infty} \frac{\sqrt{n! + 2^n}}{3^n}, \qquad
  \lim_{n\to +\infty} \frac{\sqrt{(2n)!}}{n^n}, \\
  \lim_{n\to +\infty} \sqrt[n]{e^n + \sqrt{10^n}}.
\end{gather*}
\end{exercise}
