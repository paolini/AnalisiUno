\section{teoria degli insiemi}

Nei paragrafi precedenti abbiamo visto che i predicati possono essere
combinati tra loro e quantificati.
Ma quali sono i predicati elementari che possiamo considerare per fondare 
tutta la matematica?
La teoria degli insiemi ci fornisce la formalizzazione del concetto
di \emph{appartenenza}: il predicato di base (l'unico che ci servirà) è
un predicato della forma $x \in A$ che significa ``l'oggetto $x$ è un elemento
dell'insieme $A$''.
La teoria degli insiemi non spiega (né tantomento definisce)
cosa siano gli oggetti e cosa siano gli insiemi perché questi sono concetti
primitivi (come lo sono i \emph{punti} e le \emph{rette} per la geometria euclidea).
La teoria degli insiemi ci fornisce, semplicemente, le regole formali che
possono essere utilizzate per trattare i predicati della forma $x\in A$.
Ad esempio un \emph{assioma} 
della teoria degli insiemi è il seguente.
\begin{axiom}[insieme vuoto]
\[
  \exists A \colon \lnot \exists x\colon x \in A
\]
\end{axiom}
L'assioma significa: ``esiste un insieme $A$ che non contiene alcun elemento $x$''
ovvero stiamo dicendo che esiste l'\emph{insieme vuoto}%
\mymargin{insieme vuoto}%
\index{insieme vuoto} che normalmente viene
denotato con il simbolo $\emptyset$.
Scriveremo quindi $\exists\emptyset\colon \forall x\colon \lnot(x\in \emptyset)$.
Possiamo quindi introdurre un contesto dimostrativo 
in cui $\emptyset$ è una costante e l'assioma diventa più
semplicemente $\forall x\colon \lnot (x\in \emptyset)$.

Altri opportuni assiomi della teoria degli insiemi garantiscono l'esistenza
dell'unione $A\cup B$, intersezione $A\cap B$ e differenza $A\setminus B$
di due insiemi qualunque $A$ e $B$. Tali operazioni
tra insiemi possono essere ricondotte alla relazione di appartenenza
e possono quindi essere definite come segue.

\begin{axiom}[operazioni tra insiemi]
Se $A$ e $B$ sono insiemi allora esistono gli insiemi  
$A\cup B$ (unione), $A\cap B$ (intersezione) e 
$A\setminus B$ (differenza).
\mymargin{unione intersezione differenza}%
\index{unione}%
\index{intersezione}%
\index{differenza}%
Formalmente:
\begin{align*}
    \forall A,B\colon \exists A\cup B \colon \forall x\colon x\in U &\iff (x\in A) \lor (x\in B),\\
    \forall A,B\colon \exists A\cap B \colon \forall x\colon x\in I &\iff (x\in A) \land (x\in B),\\
    \forall A,B\colon \exists A\setminus B \colon \forall x\colon x\in D &\iff (x\in A) \land \lnot (x \in B).
\end{align*}
\end{axiom}

La negazione dell'appartenenza $\lnot (x \in A)$ viene usualmente
abbreviata con $x \not \in A$.

Sempre utilizzando la semplice relazione di appartenenza possiamo definire
le relazioni di inclusione e uguaglianza tra insiemi:
$A \subset B$ si legge ``$A$ è un sottoinsieme di $B$'',
$A \supset B$ si legge ``$A$ è un sovrainsieme di $B$''
e $A=B$ si legge ``$A$ è uguale a $B$''. 
Queste relazioni sono definite dalle seguenti proprietà:
\begin{align*}
  A \subset B &\iff \forall x\colon (x\in A \implies x\in B)\\
  A \supset B &\iff \forall x\colon (x\in A \implied x\in B)\\
  A = B &\iff \forall x\colon (x\in A \iff x \in B).
\end{align*}
A parole diremo che $A$ è un sottoinsieme di $B$, 
$A \subset B$, se ogni elemento di $A$ è anche elemento di $B$.
Diremo che $A$ e $B$ sono uguali, $A=B$, 
se $A$ e $B$ hanno gli stessi elementi.
Ovviamente la relazione $\supset$ non è altro 
che la relazione inversa di $\subset$ cioè risulta 
$A\supset B$ se e solo se $B\subset A$.
Scriveremo anche $A \neq B$ per indicare
la relazione opposta dell'uguaglianza ovvero: $\lnot(A=B)$.
Si noti che $A=B$ è equivalente a $(A\subset B) \land (B\subset A)$.

% In molti sistemi formali l'uguaglianza è un predicato primitivo con la 
% proprietà che oggetti uguali possono essere sostituiti uno all'altro 
% in ogni proposizione. Per recuperare tale proprietà 
% dobbiamo imporre il seguente.
% 
% \begin{axiom}[estensionalità]
% Se $x=y$ allora il predicato $x\in A$ è equivalente 
% al predicato $y\in A$.
% \end{axiom}

La relazione $A \subset B$ significa che ogni elemento di $A$ 
è anche elemento di $B$ e quindi non ci sono elementi di $A$ che non stiano 
in $B$ cioè $A$ non è più grande di $B$. 
Come stratagemma mnemonico si osservi che il simbolo 
$\subset$ è orientato in modo che l'insieme più piccolo 
stia dal lato più stretto della relazione così come 
nella disuguaglianza $3 \le 5$ il numero più piccolo 
sta dal lato più stretto del simbolo $\le$.
Per dimostrare che la relazione $A \subset B$ è vera 
bisognerà prendere un qualunque elemento di $A$ e dimostrare 
che tale elemento è anche elemento di $B$.
Per dimostrare che vale l'uguaglianza $A=B$ tipicamente si procede 
dimostrando separatamente le due inclusioni $A\subset B$ e $B\subset A$.

Risulta molto utile la possibilità di costruire insiemi di insiemi.
Per questo motivo la teoria degli insiemi usualmente non fa distinzione
tra oggetti e insiemi di oggetti. Nella relazione primitiva $x\in A$ anche
$x$ può essere un insieme. Possiamo allora immaginare che ogni oggetto del
nostro universo sia un insieme. In questo modo la relazione di uguaglianza $A=B$
che abbiamo definito sopra risulta ben definita per ogni coppia di oggetti
(o insiemi, che è lo stesso) $A$ e $B$.

Visto che gli elementi di un insieme $\mathcal A$ sono a loro volta insiemi,
è possibile considerare l'unione $\bigcup \mathcal A$
e, se\mynote{%
L'intersezione di una famiglia vuota di insiemi darebbe l'insieme 
universo, che vedremo non può essere definito.
} 
$\mathcal A \neq \emptyset$, l'intersezione $\bigcap \mathcal A$ di tutti gli elementi
di $\mathcal A$.
Questo estende il concetto di unione e intersezione anche a famiglie
eventualmente infinite.

\begin{axiom}[unione e intersezione arbitraria]
Se $\mathcal A$ è un insieme qualunque esiste l'insieme $\bigcup \mathcal A$
e se $\mathcal A \neq \emptyset$ esiste l'insieme $\bigcap \mathcal A$
con le seguenti proprietà:
\begin{align*}
  x \in \bigcup \mathcal A & \iff \exists A \in \mathcal A \colon x\in A, \\
  x \in \bigcap \mathcal A & \iff \forall A \in \mathcal A \colon x\in A.
\end{align*}
\end{axiom}
Una notazione alternativa è la seguente:
\[
 \bigcup \mathcal A = \bigcup_{A\in \mathcal A} A, \qquad 
 \bigcap \mathcal A = \bigcap_{A\in \mathcal A} A.  
\]
Questa notazione mette in evidenza il fatto che l'insieme 
$\bigcup \mathcal A$ è l'unione di tutti gli elementi $A$ dell'insieme 
$\mathcal A$ mentre $\bigcap \mathcal A$ è l'intersezione 
di tutti gli elementi di $\mathcal A$.
L'insieme $\mathcal A$ viene spesso chiamata una \emph{famiglia}
di insiemi perché i suoi elementi vengono trattati come insiemi 
più che come oggetti.

Un altro assioma della teoria degli insiemi garantisce che per ogni
$x$ esiste un insieme il cui unico elemento è $x$. 
\begin{axiom}[singoletto]
  Se $x$ è un insieme esiste l'insieme $\ENCLOSE{x}$, 
  chiamato \emph{singoletto}%
\mymargin{singoletto}%
\index{singoletto}
  tale che:
  \[
    y \in \ENCLOSE{x} \iff y=x.
  \]
\end{axiom}
Facendo l'unione di singoletti possiamo definire (per elencazione) insiemi che contengono
un numero finito di oggetti:
\begin{align*}
  \ENCLOSE{a,b} &= \ENCLOSE{a} \cup \ENCLOSE{b} \\
  \ENCLOSE{a, b, c} &= \ENCLOSE{a,b} \cup \ENCLOSE{c}\\
  \ENCLOSE{a, b, c, d} &= \ENCLOSE{a,b,c} \cup \ENCLOSE{d}\\
  &\quad\vdots
\end{align*}

Si faccia però attenzione: l'insieme $\ENCLOSE{a,b}$ contiene due elementi
solamente se $a\neq b$, infatti se $a=b$ si potrà facilmente verificare
che
\begin{equation}\label{eq:4775523}
\ENCLOSE{a,a} = \ENCLOSE{a}.
\end{equation}

\begin{exercise}
  Verificare \eqref{eq:4775523} utilizzando le definizioni formali date in precedenza.
\end{exercise}
\begin{proof}[Svolgimento.]
Utilizzando le definizioni di unione, di singoletto e di disgiunzione logica
si ha l'equivalenza dei seguenti
predicati:
\begin{gather*}
  x \in \ENCLOSE{a,a}  \\
  x \in \ENCLOSE{a} \cup \ENCLOSE{a}\\
  (x \in \ENCLOSE{a}) \lor (x \in \ENCLOSE{a})\\
  (x = a) \lor (x = a) \\
  x = a \\
  x \in \ENCLOSE{a}
\end{gather*}
e dunque $\ENCLOSE{a,a}=\ENCLOSE{a}$ per la definizione di uguaglianza tra insiemi.
\end{proof}

Se $P(x)$ è un predicato in una sola variabile $x$ vorremmo poter
definire l'insieme di tutti gli oggetti $x$ 
che rendono vero il predicato $P$.
Sorprendentemente se aggiungessimo questo assioma 
(chiamato assioma di specificazione ingenua) nella teoria degli insiemi
avremmo un paradosso%
\index{teoria!ingenua degli insiemi}%
\index{insiemi!teoria ingenua}%
\index{Cantor, Georg}%
\index{Frege}%
\index{Russell}%
\mynote{Georg Cantor (1845--1918), Bertrand Russell (1872--1970) 
vedi note storiche a pag.~\pageref{nota:Cantor}}.

\begin{theorem}[paradosso di Russell]
\label{th:Russell}%
Non esiste un insieme $R$ tale che per ogni $x$ si abbia
\[
   x\in R \iff x\not \in x
\]
(un tale insieme si potrebbe definire così
tramite un assioma di specificazione \emph{ingenua}: $R=\ENCLOSE{x\colon x\not \in x}$).
\end{theorem}
%
\begin{proof}
  Se esistesse avremmo:
  \[
    R \in R 
    \iff R\not \in R
  \]
  e questo è in contrasto
  con il principio di non contraddizione
  (un predicato e la sua negazione non possono essere entrambi veri).
\end{proof}

Per evitare l'incoerenza è necessario limitare l'\emph{assioma di specificazione}
alla costruzione di sottoinsiemi di insiemi già costruiti.

\begin{axiom}[specificazione]
  Se $P$ è un qualunque predicato con una variabile libera $x$
  e se $B$ è un insieme, allora esiste l'insieme 
  $\ENCLOSE {x\in B\colon P(x)}$ formato 
  da tutti gli elementi di $B$ che soddisfano il predicato $P$:
\[
  a \in \ENCLOSE{x\in B\colon P(x)} \iff (a \in B \land P(a)).
\]
\end{axiom}

Se ora proviamo a ripetere il ragionamento di Russell dato un insieme $A$ 
possiamo costruire l'insieme $R=\ENCLOSE{x\in A\colon x\not \in x}$
e scopriremmo che non può essere $R\in R$ 
(altrimenti dovrebbe essere $R\not \in R$ assurdo).
Ma potrebbe benissimo essere $R\not \in R$ se inoltre $R\not \in A$.
Dunque si deduce che dato qualunque insieme $A$ c'è un insieme 
che non è elemento di $A$: dobbiamo quindi rassegnarci al fatto che non
esiste un insieme \emph{universo} contenente qualunque altro insieme,
ma almeno non otteniamo una contraddizione%
\mynote{%
D'altra parte non possiamo comunque escludere che una contraddizione 
esista. 
Infatti Goedel ha dimostrato che non è possibile dimostrare che 
non ci siano contraddizioni nel sistema assiomatico da noi considerato.
}.

Per completare la teoria degli insiemi introduciamo anche il concetto di
\emph{insieme delle parti}%
\mymargin{insieme delle parti}%
\index{insieme!delle parti}
\index{$\mathcal P(\cdot)$}% 
cioè l'insieme di tutti i sottoinsiemi di un insieme dato.
\begin{axiom}[insieme delle parti]
\label{def:insieme_parti}%
Se $A$ è un insieme esiste l'insieme $\P(A)$ delle parti di $A$
che è l'insieme di tutti i sottoinsiemi di $A$:
\begin{equation}\label{eq:insieme_delle_parti}
  x \in \mathcal P(A) \iff x \subset A.
\end{equation}
\end{axiom}

\subsection{gli ordinali finiti di Von Neumann}

Con i soli assiomi che abbiamo introdotto fin'ora possiamo 
costruire un numero arbitrariamente grande di insiemi finiti
tramite una costruzione dovuta a Von Neumann. 
Possiamo definire 
\begin{equation}\label{eq:vonNeumann}
  \begin{aligned}
  0&=\emptyset, &
  1&=0\cup\ENCLOSE{0}, &
  2&=1\cup\ENCLOSE{1}, &
  3&=2\cup\ENCLOSE{2}, &
  4&=3\cup\ENCLOSE{3},\\
  5&=4\cup\ENCLOSE{4}, &
  6&=5\cup\ENCLOSE{5}, &
  7&=6\cup\ENCLOSE{6}, &
  8&=7\cup\ENCLOSE{7}, &
  9&=8\cup\ENCLOSE{8}.
  \end{aligned}
\end{equation}
Esplicitando queste definizioni si osserva che:
$0=\ENCLOSE{}$ è un insieme con zero elementi, 
$1=\ENCLOSE{0}=\ENCLOSE{\ENCLOSE{}}$ è un insieme con un elemento (lo zero),
$2=\ENCLOSE{0,1}=\ENCLOSE{\ENCLOSE{},\ENCLOSE{\ENCLOSE{}}}$ è un insieme con due elementi (i due precedenti), 
$3=\ENCLOSE{0,1,2}=\ENCLOSE{\ENCLOSE{},\ENCLOSE{\ENCLOSE{}},\ENCLOSE{\ENCLOSE{},\ENCLOSE{\ENCLOSE{}}}}$ 
è un insieme con tre elementi (i tre precedenti),
etc.
Procedendo a piacere è possibile definire un numero arbitrariamente 
grande di insiemi finiti, uno diverso dall'altro.

Questi particolari insiemi esistono certamente e potremo quindi utilizzarli 
nel seguito per fare esempi di insiemi finiti.
E' anche possibile (ma non necessario) utilizzare questi insiemi 
per definire i numeri naturali (si veda l'Assioma di infinito~\ref{axiom:infinito}).

\begin{exercise}
Verificare che $4\in 5$, 
che $4\subset 5$ 
e che $4\neq 5$.
\end{exercise}

\subsection{relazioni}

Grazie agli assiomi precedenti è possibile definire il prodotto cartesiano
di due insiemi: sarà questo un concetto molto importante nel seguito.
Innanzitutto dobbiamo definire il concetto di \emph{coppia}%
\mymargin{coppia}%
\index{coppia}: dati
$a,b$ definiamo la coppia $(a,b)$ con primo elemento $a$ e secondo elemento $b$
come un oggetto che ha questa proprietà:%
\mynote{%
Un modo formale per definire la coppia tramite l'utilizzo di insiemi 
è dovuto a Kuratowski:
\[
   (a,b) = \ENCLOSE{\ENCLOSE{a},\ENCLOSE{a,b}}.
\]
Non è difficile verificare che questa definizione soddisfa la 
proprietà~\eqref{eq:coppia}.
Così possiamo definire l'insieme prodotto
\begin{align*}
  A\times B = \big\{&C\in \mathcal P(\mathcal P(A\cup B))\colon \\ 
  & \exists a\colon \exists b\colon a\in A, b\in B, \\
  & C=\ENCLOSE{\ENCLOSE{a},\ENCLOSE{a,b}}\big\}
\end{align*}
}%
\begin{equation}\label{eq:coppia}
  (a, b) = (a', b') \iff (a=a') \land (b=b').
\end{equation}
Stiamo cioè richiedendo che una coppia venga identificata dai due
elementi che la compongono in cui, però, è importante anche l'ordine in
cui vengono elencati (a differenza dell'insieme $\ENCLOSE{a,b}$, in cui
l'ordine degli elementi è irrilevante).
Nel seguito useremo anche la notazione $a \mapsto b$ per indicare 
la coppia $(a,b)$ e la chiameremo \emph{freccia}%
\mymargin{freccia}%
\index{freccia} da $a$ in $b$.

Il \emph{prodotto cartesiano} $A\times B$ di due insiemi $A$ e $B$
è l'insieme di tutte le coppie
il cui primo elemento sta in $A$ e
il secondo elemento sta in $B$:
\[
  (a, b) \in A \times B \iff (a\in A \land b\in B).
\]

Se rappresentiamo gli elementi di $A$ come dei punti su una retta
orizzontale (asse delle $x$) e gli elementi di $B$ come dei punti
su una retta verticale (asse delle $y$) gli elementi di $A\times B$
possono essere rappresentati come i punti del piano che proiettati sull'asse
delle $x$ vanno in $A$ e proiettati sull'asse delle $y$ vanno in $B$
(si veda la figura~\ref{fig:funzione}). 
  
Una \emph{relazione}%
\mymargin{relazione}%
\index{relazione} $R$ tra gli elementi di un insieme $A$ e gli elementi
di un insieme $B$ non è altro che un sottoinsieme di $A\times B$
ovvero $R\in \mathcal P(A\times B)$.
Per una relazione $R\subset A\times B$ si userà la notazione infissa
$aRb$ per indicare $(a,b)\in R$.
Oppure con la notazione delle freccie potremo 
scrivere $a \stackrel R \mapsto b$
per indicare che la freccia $a\mapsto b$ è un elemento di $R$. 
Ad esempio se prendo $A=B=\ENCLOSE{1,2,3}$ e considero la relazione 
$R=\ENCLOSE{(1,2),(1,3),(2,3)}$
il predicato $aRb$ rappresenta l'usuale relazione d'ordine $a<b$ sui
tre numeri considerati.

\subsubsection{relazioni di equivalenza}

\begin{definition}[relazione di equivalenza]
\label{def:equivalenza}%
\label{def:insieme_quoziente}%
Sia $R\subset A\times A$ una relazione sull'insieme $A$. Diremo che 
$R$ è una \emph{relazione di equivalenza}%
\mymargin{relazione di equivalenza}%
\index{relazione!di equivalenza} se valgono le seguenti proprietà
(tipiche dell'uguaglianza)
per ogni $x,y,z\in A$:
\begin{enumerate}
  \item riflessiva: $x R x$;
  \item simmetrica: se $x R y$ allora $y R x$;
  \item transitiva: se $x R y$ e $yRz$ allora $x R z$.
\end{enumerate}
Se $R$ è una relazione di equivalenza diremo che $x$ è equivalente a $y$ 
(tramite $R$) quando $xRy$.
Dato $x \in A$ l'insieme di tutti gli elementi $y\in A$ che sono equivalenti 
a $x$ si chiama \emph{classe di equivalenza}%
\mymargin{classe di equivalenza}%
\index{classe!di equivalenza} di $x$. 
Si definisce in questo modo:
\[
  [x]_R = \ENCLOSE{y\in A \colon xRy}.  
\]
Se $R$ è una relazione di equivalenza su $A$ definiamo 
l'\emph{insieme quoziente}
\mymargin{insieme quoziente}%
\index{insieme!quoziente}%
$A/R$
come l'insieme di tutte le classi di equivalenza:
\[
 A/R 
 = \ENCLOSE{[x]_R\colon x\in A} 
 = \ENCLOSE{B\in \mathcal P(A)\colon \exists x\in A\colon B=[x]_R}.  
\]
\end{definition}

Se $R$ è una relazione di equivalenza su $A$ 
l'insieme quoziente $A/R$ rappresenta l'insieme 
degli oggetti di $A$ in cui vengono identificati tra loro gli oggetti tra loro 
equivalenti.
Spesso diremo che $A/R$ rappresenta $A$ a meno della equivalenza $R$.

Ad esempio se $A=\ENCLOSE{1,2,3,4,5}$ e prendiamo $D=\ENCLOSE{1,3,5}$ (i numeri dispari)
e $P=\ENCLOSE{2,4}=A\setminus D$ 
(i numeri pari) possiamo definire una relazione $R$ su $A$ mediante la proprietà:
\[
 x R  y \iff (x\in P \land y\in P) \lor (x\in D \land y\in D).
\]
La relazione $R$ rappresenta la proprietà degli elementi di $A$ 
di avere la stessa ``parità''. 
Si avrà $[1]_R = [3]_R=[5]_R= D$ e $[2]_R=[4]_R=P$.
Dunque 
\[
   A/R = \ENCLOSE{P,D} = \ENCLOSE{\ENCLOSE{2,4},\ENCLOSE{1,3,5}}
\]
e, in effetti, questo insieme ha due elementi $P$ e $D$ che è quello 
che si ottiene identificando i numeri di $A$ in base alla proprietà 
di essere pari o dispari. 

L'insieme quoziente $A/R$ è una \emph{partizione}%
\mymargin{partizione}%
\index{partizione} di $A$ in quanto 
è formato da insiemi disgiunti la cui unione è tutto $A$. In effetti 
dare una partizione di $A$ è equivalente a dare una relazione 
di equivalenza.

\subsubsection{relazioni d'ordine}

\begin{definition}[relazione d'ordine]
  \label{def:ordine}%
  Una relazione
  $\le$ su un insieme $X$ viene detta
  \emph{relazione d'ordine}%
\mymargin{relazione d'ordine}%
\index{relazione!d'ordine}
  se soddisfa le seguenti proprietà (per ogni $x,y,z\in X$):
  \begin{enumerate}
    \item[1.] riflessiva: $x\le x$;
    \item[2.] antisimmetrica: $x\le y \land y\le x \implies x=y$;
    \item[3.] transitiva: $x\le y \land y\le z \implies x\le z$.
  \end{enumerate}
  Si dice inoltre che la relazione d'ordine $\le$
  è una relazione d'\emph{ordine totale}
\mymargin{ordine totale}%
  (o \emph{lineare}) 
  \index{ordinamento!totale}%
  \index{ordinamento!lineare}%
  \index{totale!ordine}%
  \index{lineare!ordine}%
  e si dice che $X$ è \emph{totalmente ordinato} se vale
  \index{totalmente ordinato}%
  \begin{enumerate}
    \item[4.] dicotomia: $x\le y \lor y\le x$.
  \end{enumerate}
\end{definition}

\begin{definition}
  \label{def:massimo}%
  \label{def:minimo}%
  \label{def:maggiorante}%
  \label{def:minorante}%
  \label{def:sup}%
  \label{def:inf}%
  Sia $\le$ una relazione d'ordine su un insieme $X$
  sia $A\subset X$ e sia $x\in X$.

  Diremo che $x$ è un \emph{maggiorante} di $A$ se per ogni $a\in A$
  si ha $a\le x$.
  \mymargin{maggiorante, minorante}%
  Diremo che $x$ è un \emph{minorante} di $A$ se per ogni $a\in A$
  si ha $x\le a$.
  
  Diremo che $x$ è il \emph{massimo} di $A$ 
  e scriveremo $x=\max A$,
  se $x$ è un maggiorante di $A$ e $x\in A$.
  \mymargin{massimo, minimo}
  Diremo che $x$ è il \emph{minimo} di $A$ 
  e scriveremo $x=\min A$,
  se $x$ è un minorante di $A$ e $x\in A$.
  
  Diremo che $x$ è l'\emph{estremo superiore} di $A$ 
  e scriveremo $x=\sup A$,
  se $x$ è un maggiorante di $A$
  e non esistono maggioranti di $A$ più piccoli di $x$
  ($x$ è il minimo dei maggioranti).
  \mymargin{estremo superiore e inferiore}%
  Diremo che $x$ è l'\emph{estremo inferiore} di $A$ 
  e scriveremo $x=\inf A$,
  se $x$ è un minorante di $A$
  e non esistono minoranti di $A$ più grandi di $x$
  ($x$ è il massimo dei minoranti).

  Se l'insieme $A$ ha almeno un maggiorante $x\in X$, diremo che $A$ 
  \mymargin{limitato}%
  \index{limitato}%
  \index{insieme!limitato}%
  è \emph{superiormente limitato}. 
  Se $A$ ha almeno un minorante diremo che $A$ è \emph{inferiormente limitato}.
  Se $A$ è sia superiormente che inferiormente limitato diremo che $A$ è \emph{limitato}.
\end{definition}

Si osservi che il massimo e il minimo di un insieme $A$, se esistono, 
sono unici. Infatti se $x$ e $y$ sono entrambi massimi di $A$
si deve avere $x\le y$ in quanto $y$ è massimo, ma anche $y\le x$ 
visto che anche $x$ è massimo. 
Dunque $x=y$ per la proprietà antisimmetrica. 
Lo stesso vale per il minimo.
Visto che l'estremo superiore è il minimo dei maggioranti, e l'estremo inferiore 
è il massimo dei minoranti, anche estremo superiore ed estremo inferiore,
se esistono, sono unici.
Se il massimo (o il minimo) esiste, allora coincide con 
l'estremo superiore (o inferiore). 
Viceversa vedremo degli esempi in cui l'estremo superiore esiste ma non 
è elemento dell'insieme (dunque non è un massimo, e il massimo non esiste).

In generale, se $\le$ è una relazione d'ordine su $X$ si definisce la 
relazione inversa $\ge$ ponendo $x\ge y$ quando $y\le x$.
La proprietà riflessiva identifica gli ordinamenti \emph{larghi}.
Si definisce il corrispondente ordinamento stretto $<$
ponendo $x < y$ quando $x\le y \land x\neq y$
e la relazione inversa $>$ ponendo $x>y$ quando $y<x$.

\begin{example} 
  Sia $X=\ENCLOSE{1,2,3,4}$. 
  Si può verificare che c'è una unica relazione d'ordine totale
  su $X$ per cui si ha $1 \le 2$, $2 \le 3$ e $3 \le 4$
  (scriveremo più brevemente $1 \le 2 \le 3 \le 4$).
  Questa relazione è definita dall'insieme di coppie:
  \[
    R_\le = \ENCLOSE{
          (1,1),(1,2),(1,3),(1,4),
          (2,2),(2,3),(2,4),
          (3,3),(3,4),
          (4,4)}
  \]
  nel senso che $x\le y$ se $(x,y)\in R_\le$.
  La relazione inversa $\ge$ corrisponde a:
  \[
    R_\ge = \ENCLOSE{
          (1,1),(2,1),(3,1),(4,1),
          (2,2),(3,2),(4,2),
          (3,3),(4,3),
          (4,4)}
  \]
  e le relazioni strette $<$ e $>$ corrispondono a:
  \begin{align*}
    R_< &= \ENCLOSE{
          (1,2),(1,3),(1,4),
          (2,3),(2,4),
          (3,4)}, \\
    R_> &= \ENCLOSE{
          (2,1),(3,1),(4,1),
          (3,2),(4,2),
          (4,3)}.
  \end{align*}

  Preso $A=\ENCLOSE{1,3}\subset X$ si ha $\max A = \sup A = 3$ e 
  $\min A = \inf A = 1$.
  I maggioranti di $A$ sono $3$ e $4$. 
  C'è un unico minorante che è $1$.
\end{example}

\begin{example}
Se prendiamo l'insieme $A=\ENCLOSE{1,2,3}$ e consideriamo la relazione
su $X=\mathcal P(A)$ definita da $x\le y$ se $x\subset y$ otteniamo una relazione 
d'ordine che non è totale. 
Infatti se $x=\ENCLOSE{1,2}$ e $y=\ENCLOSE{2,3}$ non è vero né che $x\subset y$ 
né che $y\subset x$.
Si ha $\max \P(A) = \sup \P(A) = A$ e $\min \P(A) = \inf \P(A) = \emptyset$.
\end{example}

\begin{example}
L'insieme $\NN$ dei numeri naturali (verrà introdotto nel capitolo~\ref{sec:naturali}) non ha maggioranti 
quindi non ha né massimo né estremo superiore. 
Infatti scelto 
qualunque numero naturale ce n'è sempre un'altro più grande.

Se all'insieme dei numeri naturali aggiungiamo un elemento $\infty$,
che definiamo essere più grande di ogni numero naturale, 
otteniamo un insieme totalmente ordinato $X=\NN\cup\ENCLOSE{\infty}\supset \NN$
tale che $\max X = \infty$ e $\sup \NN = \infty$ in $X$.
Ma $\infty$ non è un elemento di $\NN$ quindi non è un massimo di $\NN$ in $X$.
\end{example}

\begin{definition}[ordinamento denso]
  \label{def:ordinamento_denso}%
  Un ordinamento si dice essere
  \emph{denso}%
\mymargin{ordinamento denso}%
\index{denso} o \emph{divisibile} 
  se tra due punti distinti esiste sempre un punto intermedio:
  \index{denso!sottoinsieme}%
  \index{denso!ordinamento}%
  \index{sottoinsieme!denso}%
  \index{ordinamento!denso}%
  \mynote{Più in generale se $X$ è un insieme totalmente 
  ordinato e $A\subset X$ si dice che $A$ è denso in $X$ 
  se dati comunque $x,y\in X$ con $x<y$ esiste $c\in A$ tale che 
  $x<c<y$.}
  \[
   x < y \implies \exists c \colon x < c < y.
  \]
\end{definition}

\begin{example}
Un esempio di ordinamento denso è quello dato sull'insieme dei punti 
di una retta geometrica orientata.
Dati due punti distinti $P$ e $Q$ sulla retta, 
esiste sempre un punto intermedio tra $P$ e $Q$ (ad esempio il punto medio).

Intuitivamente osserviamo che se ho un ordinamento denso su un insieme 
e l'insieme contiene almeno due punti, allora l'insieme 
contiene infiniti punti visto che c'è un punto intermedio, 
e poi un altro punto intermedio 
tra il punto intermedio e uno dei due punti, e così via...
\end{example}

\begin{definition}[ordinamento continuo]
  \label{def:ordinamento_continuo}%
  Un ordinamento denso $\le$ su un insieme $X$ si dice essere
  \emph{continuo}%
\mymargin{ordinamento continuo}%
\index{continuo}
  \index{continuità!ordinamento}%
  \index{ordine!continuo}%
  (o \emph{Dedekind-completo})
  \index{Dedekind-completo}%
  \index{completo!Dedekind}%
  se dati comunque $A,B$ sottoinsiemi non vuoti di $X$
    tali che per ogni $a\in A$ e per ogni $b\in B$ risulta $a\le b$
    (concisamente scriveremo $A\le B$ e diremo che $A$ e $B$ sono 
    \emph{separati})
    allora esiste $c\in X$ tale che per ogni $a\in A$ e ogni $b\in B$ 
    si ha $a\le c \le b$ (concisamente scriveremo $A\le c \le B$
    e diremo che $c$ è un elemento di separazione tra $A$ e $B$).
\end{definition}

A parole la precedente definizione si esprime dicendo che un insieme ordinato
denso è \emph{continuo} se ogni coppia di sottoinsiemi separati 
ammette un elemento di separazione.
In molti testi questa condizione viene chiamata \emph{completezza}
o, più precisamente, \emph{completezza di Dedekind}.
\index{completezza!di Dedekind}%
\index{Dedekind!completo}%

\begin{example}
Un insieme può essere denso ma non continuo. 
Se ad esempio prendiamo due punti distinti $P$, $Q$ su una retta orientata, 
in modo che sia $P<Q$, poi prendiamo il punto 
medio $M$ tra $P$ e $Q$, poi il punto medio tra tutti i punti precedenti 
e così via,
intuitivamente otteniamo un insieme di punti $X$ che risulta essere denso.
Ma se prendiamo l'insieme $A$ dei punti di $X$ che distano da $P$ 
meno del doppio di quanto distano da $Q$, 
e l'insieme $B=X\setminus A$ è chiaro che i due insiemi sono non vuoti e 
separati, ma non esiste un punto di separazione tra $A$ e $B$ in quanto 
tale punto dovrebbe avere distanza $\frac 1 3 PQ$ da $P$ mentre tutti i punti 
di $X$ hanno una distanza da $P$ che si esprime con una frazione di $PQ$ 
che ha come denominatore una potenza di $2$.
\end{example}

\mynote{
Eviteremo di usare la definizione di continuità sugli ordinamenti che non sono 
densi, altrimenti tutti gli insiemi finiti avrebbero un ordinamento continuo 
e anche insiemi come $\NN$ e $\ZZ$ risulterebbero continui, cosa che 
va contro l'intuizione di continuità che abbiamo in mente.
}

\begin{theorem}[esistenza del $\sup$]%
  \label{th:sup}%
  \mymark{**}%
  Sia $X$ un insieme con un ordinamento continuo.
  Se $A\subset X$ è un insieme non vuoto
  e superiormente limitato, allora esiste $\sup A$ (l'estremo superiore).
  Se $A\subset X$ è un insieme non vuoto e inferiormente limitato 
  allora esiste $\inf A$ (l'estremo inferiore).

  Se $X$ ha minimo allora $\sup\emptyset = \min X$
  e se $X$ ha massimo allora $\inf\emptyset = \max X$.
  \end{theorem}
  %
  \begin{proof}
  \mymark{*}
  Consideriamo l'insieme dei maggioranti di $A$:
  \[
  B = \ENCLOSE{ b\in \RR \colon b \ge A}.
  \]
  Preso $a\in A$ e $b\in B$ essendo che $b$ è un maggiorante di $A$,
  si deve avere $a\le b$. Dunque $A$ e $B$ sono separati.
  Inoltre $A$ è non vuoto per ipotesi e $B$ è non vuoto perché 
  $A$ è superiormente limitato e dunque esiste almeno un maggiorante di $A$.
  Dunque, essendo per ipotesi l'ordinamento continuo,
  deve esistere $c\in X$ che è un maggiorante di $A$ e un minorante di $B$.
  Visto che $c$ è un maggiorante di $A$ si ha $c\in B$.
  Ma $c$ è anche un minorante di $B$ dunque $c=\min B$.
  Essendo $c$ il minimo dei maggioranti di $A$ si ha, per definizione,
  $c= \sup A$.
ì
  Dimostrazione analoga vale per l'estremo inferiore.

  Per quanto riguarda l'insieme vuoto notiamo che ogni $x\in X$ 
  è sia un maggiorante che un minorante dell'insieme vuoto.
  Dunque l'insieme dei maggioranti e l'insieme dei minoranti coincidono con $X$.
  Il minimo dei maggioranti è dunque $\min X$ e il massimo dei minoranti è $\max X$.
\end{proof}

Osserviamo che l'esistenza del $\sup$
degli insiemi non vuoti e superiormente limitati
è in effetti una condizione equivalente all'assioma di continuità.
Infatti se $A\le B$ certamente $\sup A$, se esiste, è elemento 
di separazione tra $A$ e $B$.

La retta geometrica è un esempio di insieme che 
vorremmo avesse la proprietà di continuità.
Questo perché quando si traccia una retta che attraversa una figura 
geometrica vorremmo avere la certezza che, 
se la retta ha un punto interno alla figura ed un punto esterno, 
allora abbia anche un punto di intersezione (per noi questo sarà il teorema 
degli zeri~\ref{th:zeri}).
Questi punti di intersezione possono essere  
visti come punti di separazione tra i punti
interni e i punti esterni alla figura:
per questo motivo siamo interessati alle proprietà di continuità.

La questione è delicata e non banale. 
Invece che puntare a formalizzare gli assiomi della geometria di Euclide
noi avremo un approccio più algebrico nel senso che andremo a definire 
la retta geometrica utilizzando l'insieme $\RR$ dei numeri reali.
Gran parte di questo capitolo sarà sostanzialmente dedicato a questo scopo.
Ma si veda la discussione su Euclide e Hilbert nelle note storiche 
a fine capitolo.

