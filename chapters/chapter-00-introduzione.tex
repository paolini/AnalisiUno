\chapter*{Introduzione}

%% questa parte di testo viene estratta dallo script make-docs.sh e
%% inserita nei files README.md e index.html
%% Le righe prese in considerazione sono quelle che terminano
%% con il commento README
%% lo script fa del suo meglio per convertire il LaTeX in mardown e HTML
%% i {comment} sono utilizzati per testo destinato solamente al README

Queste note sono nate come appunti per il corso di Analisi Matematica %% README
del corso di studi in Fisica dell'Università %% README
di Pisa negli anni accademici 2017/18, 2018/19, 2019/20, 2020/21 e 2021/22. %% README
 %% README

Le note (come il corso a cui fanno riferimento) %% README
riguardano l'analisi delle funzioni di una variabile %% README
reale. %% README
Gli argomenti trattati sono serie e successioni numeriche, %% README
il calcolo differenziale e il calcolo integrale. %% README
Viene fatta un minimo di analisi funzionale allo scopo di considerare, %% README
come ultimo argomento, lo studio delle equazioni differenziali ordinarie. %% README
Da subito vengono introdotti i numeri complessi che vengono utilizzati %% README
laddove possono aiutare a dare una visione più unitaria e concettualmente %% README
più semplice degli argomenti trattati (in particolare nello studio delle serie %% README
di potenze, nella definizione delle funzioni trigonometriche, nella risoluzione delle equazioni differenziali lineari). %% README
%% README

Le note sono estensive, non c'è alcun tentativo di concisione. %% README
L'obiettivo è quello di raccogliere tutti quei risultati che non sempre è %% README
possibile esporre in maniera dettagliata e rigorosa a lezione. %% README
Troveremo, ad esempio, %% README
la costruzione dei numeri reali,
definizioni equivalenti della funzione esponenziale e una definizione %% README
analitica (tramite serie di potenze) %% README
delle funzioni trigonometriche (e di $\pi$). %% README
Proponiamo la dimostrazione del teorema fondamentale dell'algebra, %% README
della formula di Stirling e di Wallis, %% README
e dell'irrazionalità di $e$ e di $\pi$. %% README
Viene proposta una definizione formale dei simboli di Landau %% README
$o$-piccolo e $O$-grande con i relativi teoremi per trattare queste espressioni. %% README
Lo stesso viene fatto per il simbolo di integrale indefinito. %% README
Vengono trattati quei risultati algebrici che permettono di %% README
giustificare gli algoritmi per il calcolo delle primitive %% README
delle funzioni razionali e per risolvere le equazioni differenziali %% README
lineari con il metodo di similarità. %% README
 %% README

Le figure non sono frequenti ma a margine di molte di esse  %% README
è presente un \emph{QR-code} (un quadrato formato da una nuvola di pixel) %% README
che permette di accedere alla figura  %% README
\emph{online} e modicarne i parametri.  %% README
Alla pagina \url{https://paolini.github.io/AnalisiUno/} 
sono inseriti tutti i collegamenti alle figure interattive.
\begin{comment}
Di seguito in questa pagina trovate l'elenco %% README
con i collegamenti alle figure interattive. %% README
\end{comment}
 %% README
\myqrcode{https://paolini.github.io/AnalisiUno/}{}{questa pagina web}
Queste note sono rese disponibili liberamente sia in formato PDF che %% README
in forma di sorgente %% README
\begin{comment}
LaTeX. %% README
\end{comment}
\LaTeX{}.
Un modo per sostenere questo progetto e mantenerlo disponibile liberamente 
per tutti, è quello di acquistarne una copia cartacea.

Il materiale è costantemente in evoluzione %% README
e certamente contiene errori e incoerenze. Ogni suggerimento o commento è %% README
benvenuto! %% README


\section*{contributi}

Ringrazio:
%
Valerio Amico,
Rico Bellani,
Fabio Bensch,
Jacopo Bernardini,
Elia Bonistalli,
Nicolò Bottiglioni,
Antonino Calderone,
Davide Campanella Galanti,
Alessandro Canzonieri,
Giulio Carlo,
Luca Casagrande,
Alessandro Casini,
Marco Catapano,
Giulio Cianti,
Luca Ciceri,
Tommaso Ceccotti,
Giorgio Ciocca,
Luca Ciucci,
Francesco Debortoli,
Martino Dimartino,
Igor Di Tota,
Edoardo Fedi,
Diego Fortini,
Giorgio Gioffré,
Michele Pio Iallorenzi,
Irene Iorio,
Marco Labella,
Davide Labrosciano,
Andrea La Mendola,
Viviana Lippolis,
Jacopo Lombardi,
Marco Malucchi,
Francesco Maria Manuello,
Luigina Mazzone,
Roberto Menta,
Michele Monti,
Elena Morano,
Aurora Mugnai,
Francesco Nagni,
Marco Nuti,
Ruben Pariente,
Daniele Pavarini,
Guglielmo Pellitteri,
Paolo Pennoni,
Davide Perrone,
Lorenzo Pierfederici,
Alessandro Rayan Ahmed,
Riccardo Maria Rini,
Mattia Ripepe,
Francesco Rodriguez,
Elisa Sabadini,
Gabriele Scarci,
Gabriele Sclafani,
Maria Antonella Secondo,
Gabriele Siffredi,
Irene Silvestro,
Alessio Squilloni,
Antonio Tagliente,
Francesco Enrico Teofilo,
Federico Tettamanti,
Laura Toni,
Giacomo Trupiano,
Bianca Turini,
Francesco Vaselli,
Antoine Venturini,
Matteo Vilucchio,
Piero Viscone,
Dariel Vllamasi
%
che hanno segnalato errori e correzioni.

Ringrazio i colleghi Vincenzo Tortorelli e Pietro Majer
che mi hanno dato molti suggerimenti preziosi.

\section*{pronuncia delle lettere straniere}

\emph{inglesi:}
\begin{center}
\begin{minipage}{3cm}
$J$ $j$ gei (\emph{i} lunga) \index{j}\\
$K$ $k$ cappa (key) \index{k}\\
$W$ $w$ vu doppia \index{w}\\
$X$ $x$ ics \index{x}\\
$Y$ $y$ ipsilon (\emph{i} greca) \index{y}
\end{minipage}
\end{center}
%
\emph{greche:}
\begin{center}
\begin{minipage}{3cm}
$A$ $\alpha$ alfa \index{$\alpha$} \\
$B$ $\beta$ beta \index{$\beta$}\\
$\Gamma$ $\gamma$ gamma \index{$\gamma$} \\
$\Delta$ $\delta$ delta \index{$\delta$} \\
$E$ $\eps$ epsilon \index{$\eps$} \\
$Z$ $\zeta$ zeta\footnotemark[1] \\
$H$ $\eta$ eta  \\
$\Theta$ $\theta$ teta \index{$\theta$}
\end{minipage}%
\begin{minipage}{3cm}
$I$ $\iota$ iota  \\
$K$ $\kappa$ kappa\footnotemark[2]  \\
$\Lambda$ $\lambda$ lambda \index{$\lambda$} \\
$M$ $\mu$ mi (mu) \index{$\mu$} \\
$N$ $\nu$ ni (nu) \index{$\nu$} \\
$\Xi$ $\xi$ csi  \\
$O$ $o$ omicron  \\
$\Pi$ $\pi$ pi\footnotemark[3] \index{$\pi$}
\end{minipage}%
%
\begin{minipage}{3cm}
$P$ $\rho$ ro \index{$\rho$} \\
$\Sigma$ $\sigma$ sigma \index{$\sigma$} \index{$\Sigma$} \\
$T$ $\tau$ tau  \index{$\tau$}\\
$Y$ $\upsilon$ ipsilon\footnotemark[4]\\
$\Phi$ $\phi$ fi \index{$\phi$} \\
$X$ $\chi$ chi \index{$\chi$} \\
$\Psi$ $\psi$ psi \index{$\psi$} \\
$\Omega$ $\omega$ omega \index{$\omega$}
\end{minipage}
\end{center}
%
\emph{ebraiche:}
\begin{center}
\begin{minipage}{3cm}
$\aleph$ alef
\end{minipage}
\end{center}

\footnotetext[1]{pronunciato usualmente con la $z$ dolce per distinguerlo dalla $z$.}
\footnotetext[2]{in inglese si distingue la pronuncia tra $k$ \emph{key} e $\kappa$ \emph{kappa}.}
\footnotetext[3]{in italiano \emph{pi greco} in inglese \emph{pai}, per distinguere da $p$.}
\footnotetext[4]{si pronuncia \emph{upsilon} per distinguerla da $y$.}
