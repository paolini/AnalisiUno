\section{somma per parti}
\begin{theorem}[somma per parti]
\label{th:somma_per_parti}%
Siano $a_k$ e $B_k$ successioni (reali o complesse).
Posto
\[
  A_n = \sum_{k=0}^{n-1} a_k, \qquad
  b_n = B_{n+1} - B_n
\]
si ha
\mymargin{somma per parti}%
\index{somma!per parti}%
\begin{equation}\label{eq:somma_per_parti}
 \sum_{k=m}^{n-1} a_k B_k = A_n B_n - A_m B_m - \sum_{k=m}^{n-1} A_{k+1}b_k.
\end{equation}
\end{theorem}
%
\begin{proof}
Osserviamo che $a_n = A_{n+1}-A_n$ dunque
\begin{align*}
  A_n B_n - A_m B_m
  &= \sum_{k=m}^{n-1} (A_{k+1}B_{k+1}-A_k B_k)\\
  &= \sum_{k=m}^{n-1} (A_{k+1}B_{k+1}-A_{k+1}B_k + A_{k+1}B_k - A_k B_k)\\
  &= \sum_{k=m}^{n-1} A_{k+1}b_k + \sum_{k=m}^{n-1} a_k B_k.
\end{align*}
\end{proof}

Se prendiamo $a_k=(-1)^k$ si può osservare che $A_n = \sum_{k=0}^{n-1} (-1)^k$
è una successione limitata in quanto $A_n = 1$ se $n$ è dispari mentre
$A_n=0$ se $n$ è pari.
Se invece scegliamo una successione $B_n$ è positiva,
decrescente e infinitesima
e poniamo $b_n = B_{n+1}-B_n$
si ha
\[
  \sum_{k=0}^{+\infty} \abs{b_k}
  = \lim_{n\to+\infty}\sum_{k=0}^{n-1} (B_k-B_{k+1})
  = \lim_{n\to +\infty} (B_0 - B_n) = B_0 < +\infty.
\]
Dunque il seguente teorema è una estensione del criterio
di Leibniz per le serie a segni alterni.

\begin{theorem}[criterio di Dirichlet]%
\label{th:dirichlet}%
Siano $a_n$ e $B_n$ successioni (reali o complesse)
e poniamo $\displaystyle A_n = \sum_{k=0}^{n-1} a_k$,
$b_n = B_{n+1} - B_n$.
Se $A_n$ è limitata,
$B_n\to 0$
e $\sum \abs{b_n} < +\infty$
allora la serie $\sum a_k B_k$ è convergente.
\end{theorem}
%
\begin{proof}
Per la formula di somma per parti~\eqref{eq:somma_per_parti}
 si ha
\begin{equation}\label{eq:3498954}
 \sum_{k=0}^{n-1} a_k B_k
 = A_n B_n - A_0 B_0 - \sum_{k=0}^{n-1} A_{k+1}b_k.
\end{equation}
Il primo addendo $A_n B_n$ tende a zero per $n\to +\infty$
in quanto prodotto di una successione limitata per una infinitesima.
Il secondo addendo è costante.
La serie $\sum A_{k+1} b_k$ è assolutamente convergente
in quanto essendo $\abs{A_n}\le L$ limitata
e $\sum b_n$ assolutamente convergente si ha
\[
  \sum \abs{A_{k+1}} \cdot \abs{b_k}
  \le L \cdot \sum \abs{b_k} < +\infty.
\]

Dunque il limite della somma sul lato destro converge e quindi
la somma sul lato sinistro
dell'equazione~\eqref{eq:3498954}
è convergente per $n\to +\infty$.
\end{proof}

\begin{example}\label{ex:serie_log}
Fissato $z\in \CC$, $\abs{z} \le 1$, $z\neq 1$ la serie
\[
  \sum_{k=1}^{+\infty} \frac{z^k}{k}
\]
è convergente.
\end{example}
\begin{proof}
Si noti che per $\abs{z}<1$ si può facilmente applicare
il criterio del rapporto o della radice alla serie dei moduli.
Ma per $\abs{z}=1$ quei criteri non si applicano e
bisogna invece utilizzare il teorema~\ref{th:dirichlet}.

Posto $a_k=z^k$ si osserva che $\sum a_k$ è una
serie geometrica e (essendo $z\neq 1$) si ha
\[
  A_n = \sum_{k=0}^{n-1} z^k = \frac{1-z^n}{1-z}
\]
che è limitata, infatti:
\[
  \abs{A_n} = \frac{\abs{1-z^n}}{\abs{1-z}} \le \frac{1+\abs{z^n}}{\abs{1-z}}
  = \frac{2}{\abs{1-z}}.
\]
Mentre posto $B_n = 1/n$ è chiaro che, essendo $B_n$ reale,
decrescente si ha
\begin{align*}
\sum_{k=1}^{+\infty} \abs{B_{k+1}-B_{k}}
&= \lim_{n\to +\infty}\sum_{k=1}^{n-1} (B_k-B_{k+1})
= \lim_{n\to +\infty}(B_1 - B_n)\\
&= \lim_{n\to +\infty}\enclose{1 - \frac{1}{n}} = 1
< +\infty.
\end{align*}
Si applica quindi il teorema~\ref{th:dirichlet}
per ottenere la convergenza
della serie data.

Si osservi che se $\abs{z}>1$ la serie in questione non
converge perché il termine generico $z^k/k$ non è infinitesimo.
Per $z=1$ si ottiene la serie armonica, che pure non converge.
\end{proof}

\begin{exercise}
  La serie
  \[
    \sum_{k=1}^{+\infty} \frac{\sin k}{k}
  \]
  è convergente.
  \end{exercise}
  \begin{proof}
  Applichiamo il teorema~\ref{th:dirichlet}.
  Posto $a_k = \sin k$ e $B_k=1/k$
  si ha
  \[
    A_n = \sum_{k=0}^{n-1} \sin k = \Im \sum_{k=0}^{n-1} e^{ik}.
  \]
  Osserviamo allora che $e^{ik}=(e^i)^k$ e dunque $A_n$ è la parte immaginaria
  della somma di una serie geometrica. 
  Si può quindi calcolare esplicitamente
  \[
    A_n = \Im \frac{1-(e^i)^n}{1-e^i}
  \]
  da cui
  \[
    \abs{A_n} \le \abs{\frac{1-e^{in}}{1-e^i}} \le \frac{1+\abs{e^{in}}}{\abs{1-e^i}}
    = \frac{2}{\abs{1-e^i}}
  \]
  e dunque $A_n$ è limitata.
  
  D'altro canto posto $B_k = 1/k$ è chiaro che $B_k$ è
  decrescente e infinitesima.
  \end{proof}
  
%% \begin{comment}
%% section{somme su insiemi qualunque}
%% 
%% La teoria che abbiamo visto finora ci permette di definire
%% la somma di una sequenza di numeri $\vec a \colon \NN\to \RR$
%% (o $\vec a \colon \NN \to \CC$).
%% In questo capitolo daremo una definizione di somma che può
%% essere applicata a una funzione definita su un insieme di indici qualunque, non necessariamente l'insieme $\NN$ dei
%% numeri naturali.
%% 
%% Scopriremo che la definizione che stiamo introducendo se
%% applicata al caso $A=\NN$ corrisponde esattamente alla convergenza assoluta.
%% 
%% \begin{definition}[funzioni sommabili]
%% Sia $A$ un insieme qualunque.
%% Sia $A_n$ una successione di sottoinsiemi finiti di $A$.
%% Diremo che $A_n$ tende ad $A$ e scriveremo
%% \[
%%   A_n \to A
%% \]
%% se per ogni insieme finito $B\subset A$ esiste $N\in \NN$
%% tale che per ogni $n\ge N$ si ha $A_n\supset B$.
%% 
%% Sia ora $f$ una funzione reale (o complessa) definita
%% su $A$: $f\colon A \to \RR$ (oppure $f\colon A \to \CC$).
%% 
%% Se $A = \ENCLOSE{a_1, a_2, \dots, a_m}$ è finito si può definire
%% \[
%%   \sum_{a \in A} f(a) = f(a_1) + f(a_2) + \dots + f(a_m).
%% \]
%% (la definizione può essere formalizzata mediante una
%% induzione sulla cardinalità dell'insieme finito $A$).
%% 
%% Se invece $A$ è infinito scriveremo
%% \[
%%   \sum_{a\in A} f(a) = S
%% \]
%% con $S\in \bar \RR$ (oppure $S\in \bar \CC$)
%% se per ogni successione di sottoinsiemi finiti $A_n \to A$
%% si ha
%% \begin{equation}\label{eq:589421}
%%   \lim_{n\to +\infty}\sum_{a\in A_n}f(a) = S.
%% \end{equation}
%% 
%% In tal caso diremo che esiste la somma di $f$ su $A$.
%% Se tale somma $S$ è finita diremo che $f$ è sommabile su $A$.
%% \end{definition}
%% 
%% \begin{theorem}[invarianza per permutazioni]
%%   Sia $f$ una funzione definita su un insieme $A$
%%   a valori reali o complessi. Sia $\sigma\colon A \to A$
%%   una qualunque funzione bigettiva.
%%   Allora la funzione $f$ ha somma su $A$ se e solo se
%%   la funzione $f\circ \sigma$ ha somma su $A$ e in tal
%%   caso le somme coincidono:
%%   \[
%%     \sum_{x\in A} f(x) = \sum_{x\in A} f(\sigma(x)).
%%   \]
%% \end{theorem}
%% %
%% \begin{proof}
%%   Basta osservare che se $A_n$ è una qualunque successione
%%   di sottoinsiemi finiti di $A$ tale che $A_n\to A$
%%   allora anche $B_n=\sigma(A_n)$ è una successione
%%   di sottoinsiemi finiti di $A$ tale che $B_n\to A$.
%%   E viceversa.
%%   Dunque le definizione di somma per $f$ e per $\sigma\circ f$ sono equivalenti.
%% \end{proof}
%% 
%% \begin{theorem}[collegamento tra serie e somme arbitrarie]
%% Sia $a_k$ con $k\in \NN$, una successione numerica (reale o complessa).
%% 
%% Se $S$ è un numero finito (reale o complesso)
%% sono equivalenti:
%% \begin{enumerate}
%%   \item la funzione $f(k)=a_k$ è sommabile su $\NN$ e
%% \begin{equation}\label{eq:488484}
%%   \sum_{k\in \NN} a_k = S;
%% \end{equation}
%% \item la serie $\sum_k a_k$ è assolutamente
%% convergente e
%% \begin{equation}\label{eq:497494}
%%   \sum_{k=0}^{+\infty} a_k = S.
%% \end{equation}
%% \end{enumerate}
%% 
%% Se $a_k\ge 0$ (dunque reale) allora anche
%% \[
%%   \sum_{k\in \NN} a_k = +\infty
%% \qquad
%% \text{e}
%% \qquad
%% \sum_{k=0}^{+\infty} a_k = +\infty
%% \]
%% sono equivalenti.
%% \end{theorem}
%% %
%% \begin{proof}
%%   Se vale \eqref{eq:488484} allora per ogni successione
%%   di insiemi finiti $A_n \to \NN$ si deve avere
%%   \[
%%     S_n = \sum_{k\in A_n} a_k  \to S.
%%   \]
%%   In particolare se scegliamo $A_k=\ENCLOSE{0,1,2, \dots, n}$
%%   risulta che $S_n$ non è altro che la successione delle somme parziali della serie $\sum a_k$ 
%%   e dunque deve valere \eqref{eq:497494}.
%%   Ma più in generale se prendiamo qualunque permutazione $\sigma$
%%   si dovrà avere $\sum_k a_{\sigma(k)} = S$
%%   e quindi la serie deve convergere assolutamente altrimenti 
%%   si violerebbe il teorema~\ref{th:convergenza_condizionata} 
%%   di convergenza condizionata.
%% 
%%   Viceversa supponiamo che $\sum a_n$ sia una serie assolutamente convergente e poniamo
%%   \[
%%     S = \sum_{k=0}^{\infty} a_k.
%%   \]
%%   Dovremo dimostrare che per ogni successione
%%   $A_n\to A$ di sottoinsiemi finiti vale~\eqref{eq:589421}.
%%   Cioè dovremo dimostrare che per ogni $\eps>0$
%%   esiste $N\in \NN$ tale che per ogni $n>N$ si ha
%%   \begin{equation}\label{eq:4389567}
%%     \abs{\sum_{k\in A_n} a_k - S} < \eps.
%%   \end{equation}
%%   Dato $\eps>0$ per il teorema~\ref{th:coda} esiste $M\in \NN$
%%   tale che
%%   \[
%%     \sum_{k=M+1}^{+\infty} \abs{a_k} < \eps.
%%   \]
%%   Ma visto che $A_n \to A$ deve esistere $N\in \NN$
%%   tale che per ogni $n\ge N$ si ha $A_n \supset \ENCLOSE{0,1,2,\dots ,M}$
%%   e allora per ogni $n\ge N$
%%   \[
%%     \sum_{k \in A_n} a_k - S = \sum_{k=0}^M a_k + \sum_{k\in A_n, k>M} a_k - \sum_{k=0}^{+\infty} a_k
%%     = \sum_{k\in A_n, k>M} a_k - \sum_{k=M+1}^{+\infty} a_k
%%   \]
%%   ma è chiaro che
%%   \[
%%     \abs{\sum_{k\in A_n, k>M} a_k - \sum_{k=M+1}^{+\infty} a_k}
%%     \le \sum_{k=M+1}^{+\infty} \abs{a_k} < \eps
%%   \]
%%   da cui si ottiene, come voluto, \eqref{eq:4389567}.
%% \end{proof}
%% 
%% \end{comment}
%% \begin{comment}
%% \begin{theorem}[somme più che numerabili]
%%   Sia $f$ una funzione definita su un insieme $A$
%%   a valori reali o complessi. Se $f$ è sommabile
%%   su $A$ allora l'insieme
%%   \[
%%     \ENCLOSE{x\in A\colon f(x)\neq 0}
%%   \]
%%   è al più numerabile.
%% \end{theorem}
%% %
%% \begin{proof}
%% Supponiamo dapprima che $f$ abbia valori reali.
%% Se l'insieme $\ENCLOSE{f\neq 0}$ fosse più che numerabile almeno uno dei due insiemi $\ENCLOSE{f>0}$ o $\ENCLOSE{f<0}$
%% sarebbe più che numerabile.
%% Senza perdita di generalità possiamo supporre che sia
%% il primo cioè che $A_+ = \ENCLOSE{x \in A\colon f(x)>0}$
%% sia più che numerabile. Considero allora per ogni $n\in \NN$
%% gli insiemi $A_n = \ENCLOSE{x\in A\colon \frac 1 {f(x)} \in (n,n+1]}$. Almeno uno di questi insiemi
%% deve essere infinito perché se gli $A_n$ fossero
%% tutti finiti allora l'insieme $A=\bigcup_n A_n$ sarebbe
%% numerabile. Dunque esiste $N\in \NN$ tale che $A_N$ è infinito.
%% Posto $\eps = \frac{1}{N+1}>0$
%% esistono dunque $x_0,x_1, \dots, x_k, \dots$ infiniti punti di $A$ tali che $f(x_k) \ge \eps$.
%% 
%% ***DUBBIO*** Se $A$ è più che numerabile
%% esiste $A_n$ finito tale che $A_n\to A$?
%% \end{proof}
%% \end{comment}
%% \begin{comment}
%% 
%% \begin{theorem}[somme iterate]
%% Sia $f$ una funzione definita su un insieme $A$ a
%% valori reali o complessi. Supponiamo che sia
%% \[
%%   A = \bigcup_{k\in K} A_k
%% \]
%% dove $K$ è un arbitrario insieme di indici e
%% $A_k$ sono sottoinsiemi di $A$ a due a due disgiunti.
%% Allora
%% \[
%%   \sum_{x\in A} f(x) = \sum_{k\in K} \sum_{x\in A_k} f(x)
%% \]
%% se almeno uno dei due lati dell'uguaglianza esiste.
%% \end{theorem}
%% \end{comment}

%%%%%%%%%%
