\section{la serie esponenziale}

Consideriamo la funzione $f \colon \CC \to \CC$
definita da 
\begin{equation}\label{eq:def_exp}
f(z) = \sum_{k=0}^{+\infty} \frac{z^k}{k!}.
\end{equation}
La funzione è definita su tutto $\CC$ in quanto (grazie al criterio del rapporto)
è facile verificare che il raggio di convergenza di questa serie è $R=+\infty$.

Il nostro obiettivo sarà quello di dimostrare che $f(z)=e^z$ per ogni $z\in \CC$.
In effetti si potrebbe prendere~\ref{eq:def_exp} come definizione 
dell'esponenziale complesso e definire la funzione esponenziale 
reale e le funzioni trigonometriche di conseguenza.

\begin{theorem}[proprietà della serie esponenziale]
\label{th:exp_complesso}%
Sia $f$ la funzione definita da ~\eqref{eq:def_exp}.
Si ha:
\begin{enumerate}
\item
$\displaystyle f(0) = 1$;

\item
$\displaystyle f(\bar z) = \overline{f(z)}$;

\item
per ogni $z,w \in \CC$
\[
  f(z+w) = f(z) \cdot f(w);
\]

\item
per ogni $z\in \CC$ si ha $f(z) \neq 0$ e
\[
 f(-z) = \frac{1}{f(z)};
\]

\item la funzione $f\colon \CC \to \CC$ è continua;

\item si ha
\begin{equation}\label{eq:limite_exp_complesso}
   \lim_{z\to 0}\frac{f(z)-1}{z} = 1.
\end{equation}
\end{enumerate}
\end{theorem}
%
\begin{proof}
\mymark{*}
\begin{enumerate}
\item
La proprietà $f(0)=1$ si ottiene per verifica diretta (ricordiamo che $0^0=1$
e $0!=1$).

\item La proprietà $f(\bar z) = \overline{f(z)}$ si ottiene
passando al limite la seguente uguaglianza tra le somme parziali
che sfrutta le proprietà del coniugio di somma e prodotto:
\[
\sum_{k=0}^{n} \frac{\bar z^k}{k!} = \overline{\sum_{k=0}^n \frac{z^k}{k!}}.
\]


\item
Consideriamo la matrice infinita
\[
m_{k,j}  = \frac{z^k}{k!} \cdot \frac{w^j}{j!}.
\]
Allora da un lato
\begin{align*}
 f(z+w)
 &= \sum_{n=0}^{+\infty} \frac{(z+w)^n}{n!}\\
 &= \sum_{n=0}^{+\infty} \frac{1}{n!}\sum_{k=0}^n \frac{n!}{k!(n-k)!} z^k\cdot w^{n-k}\\
 &= \sum_{n=0}^{+\infty} \sum_{k=0}^n m_{k, n-k}
\end{align*}
e dall'altro
\begin{align*}
 f(z) \cdot f(w)
 &= \sum_{k=0}^{+\infty}\frac{z^k}{k!} \sum_{j=0}^{+\infty}\frac{w^j}{j!}
 = \sum_{k=0}^{+\infty}\sum_{j=0}^{+\infty}\frac{z^k}{k!} \frac{w^j}{j!} \\
 &= \sum_{k=0}^{+\infty}\sum_{j=0}^{+\infty} m_{k,j}.
\end{align*}

Il risultato segue dunque dal teorema~\ref{th:somma_Cauchy}.

\item
Visto che
\[
  1 = f(0) = f(z-z) = f(z) \cdot f(-z)
\]
ricaviamo che $f(z)\neq 0$ e $f(-z) =  1 / f(z)$.

\item
La continuità discende dal risultato generale sulla continuità della
somma di una serie di potenze: teorema~\ref{th:continuita_somma_serie}.

\item
Direttamente dalla definizione~\eqref{eq:def_exp}
si ottiene
\[
 f(z) - 1 = \sum_{k=1}^{+\infty} \frac{z^k}{k!}
 = z \cdot \sum_{k=0}^{+\infty} \frac{z^k}{(k+1)!}.
\]
Osserviamo ora che la serie $\sum \frac{z^k}{(k+1)!}$ ha raggio di
convergenza infinito e quindi è assolutamente convergente per ogni $z\in \CC$.
In particolare la somma di tale serie è continua e
quindi si ha
\[
\lim_{z\to 0}\frac{\exp(z) - 1}{z}
   =  \lim_{z\to 0} \sum_{k=0}^{+\infty} \frac{z^k}{(k+1)!}
   = \sum_{k=0}^{+\infty}\frac{0^k}{(k+1)!} = 1.
\]
\end{enumerate}
\end{proof}

\begin{theorem}[coincidenza tra funzione esponenziale e serie esponenziale]%
\label{th:exp=ex}%
\label{th:serie_esponenziale}%
\index{funzione!esponenziale}%
\mymark{***}%
Per ogni $x\in \RR$ si ha
\[
  \sum_{k=0}^{+\infty} \frac{x^k}{k!} = e^x.
\]
\end{theorem}
%
\begin{proof}\mymark{**}
Vogliamo applicare il teorema~\ref{th:isomorfismo}.
Posto $f(x)=\sum_{k=0}^{+\infty} \frac{x^k}{k!}$, 
$f\colon \RR\to\RR$ sappiamo,
per il teorema precedente, che $f(x+y)=f(x)\cdot f(y)$.
Dobbiamo verificare che $f$ è crescente.
Innanzitutto notiamo che se $x> 0$ la serie che definisce
$f(x)$ è a termini positivi e il primo termine è $1$.
Dunque per ogni $x> 0$ si ha $f(x)>1$.
In particolare posto $a=f(1)$ si ha $a>1$.
Inoltre se $y>x$ si ha $f(y-x)>1$ e dunque
\[
 f(y) = f(x+y-x) = f(x)\cdot f(y-x) > f(x).
\]
La funzione è quindi strettamente crescente. 
Dunque è un omomorfismo monotono dal gruppo additivo $\RR$ al 
gruppo moltiplicativo $(0,+\infty)$.
Fissato $a=f(1)$ sappiamo,
per quanto visto nella sezione~\ref{sec:esponenziale},
che c'è una unica funzione con queste proprietà e dunque 
deve essere: 
\[
  f(x)= a^x.
\]
Non ci rimane che dimostrare che $a=e$.
Ma grazie al teorema~\ref{th:exp_complesso}
appena dimostrato sappiamo che per $x\to 0$ si ha
\[
  \frac{f(x)-1}{x}\to 1.
\]
D'altra parte sappiamo che $f(x)=a^x$ e quindi
per il limite notevole (corollario~\ref{cor:limite_notevole_e})
\[
\frac{f(x)-1}{x} = \frac{a^{x}-1}{x}
= \frac{e^{x \ln a}-1}{x \ln a} \cdot \ln a
\to \ln a.
\]
Deduciamo quindi che $\ln a= 1$ ovvero che $a=e$.
\end{proof}

Con una certa fatica sarebbe anche possibile dimostrare che 
effettivamente $f(z)=e^z$ (nel teorema precedente l'abbiamo dimostrato 
solamente per $x\in \RR$). 
Si tratterebbe di verificare che la funzione 
$\phi(x) = f\enclose{i\frac{x}{2\pi}}$
ha le proprietà enunciate nel teorema~\ref{th:omomorfismo_U}.
Sarà però più semplice utilizzare le derivate, quindi rimandiamo questa 
verifica al prossimo capitolo.

% Aver distinto le due definizioni di $e^x$ (tramite funzione potenza) e di $\exp(x)$ (tramite somma della serie esponenziale) è puramente strumentale.
% C'è una unica funzione esponenziale che può essere definita in
% un modo o nell'altro. Non ci si fissi quindi con l'identificare
% le due diverse notazioni $e^x$ ed $\exp(x)$ con le due diverse definizioni.
% Ogni testo avrà una sua definizione di funzione esponenziale che può
% essere per certi versi arbitraria salvo poi ritrovare le proprietà
% caratterizzanti di tale funzione.
% 
% D'ora in poi scriveremo, per ogni $z\in \CC$
% \[
%   e^z = \exp z
% \]
% considerando quindi $\exp z$ l'estensione a tutto il piano complesso
% della funzione esponenziale già definita
% sulla retta reale.

Nel corollario~\ref{cor:limite_notevole_ex}
abbiamo dimostrato che 
per ogni $x\in \RR$ si ha
\[
  e^x = \lim_{n\to +\infty} \enclose{1+\frac x n}^n.  
\]
Il seguente teorema ci dice che la serie esponenziale $f(z)$ 
soddisfa la stessa proprietà per ogni $z \in \CC$.
Dunque in effetti ci fornisce una dimostrazione 
alternativa dell'identità $f(x) = e^x$ e un indizio ulteriore 
della coincidenza tra $f(z)$ ed $e^z$.

\begin{theorem}[limite notevole esponenziale complesso]%
\label{th:limite_notevole_esponenziale_complesso}%
\mymark{*}%
Per ogni $z\in \CC$ risulta
\[
  \sum_{k=0}^{+\infty} \frac{z^k}{k!} = \lim_{n \to +\infty} \enclose{1+\frac z n}^n  
\]
\end{theorem}
%
\begin{proof}
\mymark{*}
Utilizzando lo sviluppo del binomio osserviamo che si ha
\[
 \enclose{1+\frac z n}^n
 = \sum_{k=0}^n \binom{n}{k} \frac{z^k}{n^k}
 = \sum_{k=0}^n \frac{z^k}{k!} \cdot \frac{n!}{n^k\cdot (n-k)!}.
\]
Posto per ogni $k\le n$
\begin{align*}
 c(n,k)
  &= \frac{n!}{n^k\cdot (n-k)!}
  = \frac{n \cdot (n-1) \cdot \ldots \cdot(n-k+1)}{n^k} \\
  &= \frac{n}{n}\cdot {\frac {n-1} n} \cdot \frac {n-2} {n} \cdot \ldots \cdot \frac{n-k+1}{n}
\end{align*}
osserviamo che $0\le c(n,k)\le 1$ in quanto
prodotto di numeri non negativi minori o uguali ad $1$.
Inoltre, fissato $k$, si ha $c(n,k)  \to 1$ per $n\to +\infty$
in quanto ogni fattore $\frac{n-j}{n}$ tende
a $1$ per $n\to +\infty$ (si noti che a $k$ fissato il numero di fattori $k$ è
fissato).


Sia $z\in \CC$ fissato e sia
\[
  S = \sum_{k=0}^{+\infty} \frac{\abs{z}^k}{k!}.
\]
Sappiamo che la serie esponenziale è assolutamente convergente
per ogni $z\in \CC$ (in quanto il raggio di convergenza è $+\infty$)
quindi $S$ è un numero reale (finito).
Dunque per il teorema \ref{th:coda} (della coda) sappiamo che per ogni $\eps>0$
esiste $M$ tale che
\[
   \sum_{k=M+1}^{+\infty} \frac{\abs{z}^k}{k!} < \eps.
\]

Fissato $k\le M$ visto che $c(n,k)\to 1$
esiste $N_k > M$ tale che per ogni $n>N_k$
si abbia $1-c(n,k) < \eps$
(ricordiamo che $c(n,k)\le 1$).
Prendiamo allora
\[
  N=\max\ENCLOSE{N_k\colon k\le M}
\]
cosicchè per ogni $n>N$ e per ogni $k\le M$ si avrà $0 \le 1-c(n,k) < \eps$.
Allora, per ogni $n>N$, possiamo spezzare la somma da $0$ a $n$ nelle due
somme da $0$ a $M$ e da $M+1$ a $n$:
\begin{align*}
\abs{\sum_{k=0}^n \frac{z^k}{k!} - \enclose{1+\frac z n}^n}
&= \abs{\sum_{k=0}^n \enclose{\frac{z^k}{k!} - c(n,k)\frac{z^k}{k!}}}
= \abs{\sum_{k=0}^n  (1-c(n,k))\frac{z^k}{k!}} \\
&\le \sum_{k=0}^n  (1-c(n,k))\frac{\abs{z}^k}{k!} \\
  &= \sum_{k=0}^{M} (1-c(n,k)) \frac{\abs{z}^k}{k!}
   + \sum_{k=M+1}^n (1-c(n,k)) \frac{\abs{z}^k}{k!} \\
&\le \sum_{k=0}^{M} \eps \frac{\abs{z}^k}{k!}
   + \sum_{k=M+1}^n \frac{\abs{z}^k}{k!} \\
&\le  \eps \sum_{k=0}^{+\infty} \frac{\abs{z}^k}{k!}
    + \eps \\
&\le \eps S + \eps
= \eps (S+1).
\end{align*}

Visto che $\eps>0$ era arbitrario abbiamo verificato
tramite la definizione che
\[
\sum_{k=0}^n \frac{z^k}{k!} - \enclose{1+\frac z n}^n \to 0
\]
cioè
\[
\lim_{n\to +\infty} \enclose{1+\frac z n}^n = \sum_{k=0}^{+\infty} \frac{z^k}{k!}.
\]
\end{proof}

\begin{table}
\begin{center}
\begin{tabular}{r}
\ttfamily\footnotesize 2.7182818284 5904523536 0287471352 6624977572 4709369995 \\
\ttfamily\footnotesize   9574966967 6277240766 3035354759 4571382178 5251664274 \\
\ttfamily\footnotesize   2746639193 2003059921 8174135966 2904357290 0334295260 \\
\ttfamily\footnotesize   5956307381 3232862794 3490763233 8298807531 9525101901 \\
\ttfamily\footnotesize   1573834187 9307021540 8914993488 4167509244 7614606680 \\
\ttfamily\footnotesize   8226480016 8477411853 7423454424 3710753907 7744992069 \\
\ttfamily\footnotesize   5517027618 3860626133 1384583000 7520449338 2656029760 \\
\ttfamily\footnotesize   6737113200 7093287091 2744374704 7230696977 2093101416 \\
\ttfamily\footnotesize   9283681902 5515108657 4637721112 5238978442 5056953696 \\
\ttfamily\footnotesize   7707854499 6996794686 4454905987 9316368892 3009879312 \\
\ttfamily\footnotesize   7736178215 4249992295 7635148220 8269895193 6680331825 \\
\ttfamily\footnotesize   2886939849 6465105820 9392398294 8879332036 2509443117 \\
\ttfamily\footnotesize   3012381970 6841614039 7019837679 3206832823 7646480429 \\
\ttfamily\footnotesize   5311802328 7825098194 5581530175 6717361332 0698112509 \\
\ttfamily\footnotesize   9618188159 3041690351 5988885193 4580727386 6738589422 \\
\ttfamily\footnotesize   8792284998 9208680582 5749279610 4841984443 6346324496 \\
\ttfamily\footnotesize   8487560233 6248270419 7862320900 2160990235 3043699418 \\
\ttfamily\footnotesize   4914631409 3431738143 6405462531 5209618369 0888707016 \\
\ttfamily\footnotesize   7683964243 7814059271 4563549061 3031072085 1038375051 \\
\ttfamily\footnotesize   0115747704 1718986106 8739696552 1267154688 9570350354
\end{tabular}
\end{center}
\caption{Le prime 1000 cifre decimali del numero $e$
calcolate con il metodo utilizzato nella dimostrazione
del teorema~\ref{th:approx_e}.
Si veda il codice a pagina~\pageref{code:compute_e}.}
\label{fig:cifre_e}
\index{$e$!cifre decimali}
\index{cifre!$e$}
\end{table}

\begin{theorem}[approssimazione dell'esponenziale]
  \label{th:approx_exp}%
  \label{th:approx_e}%
  \index{$\exp$!approssimazione}%
  \index{approssimazione!di $\exp$}%
  Se $\abs{x}\le 1$ si ha 
  \begin{equation}\label{eq:stima_exp}
    \abs{e^x - \sum_{k=0}^n \frac{x^k}{k!}}
    \le \frac{\abs{x}^{n+1}}{n\cdot n!}.
  \end{equation}
  In particolare ponendo $x=1$ si trova
  \begin{equation}\label{eq:stima_e}
     0 < e - \sum_{k=0}^n \frac{1}{k!} \le \frac{1}{n \cdot n!}
  \end{equation}
  e per $n=5$ si ottiene
  \begin{equation}
    2.716 < e < 2.719
  \end{equation}
  \end{theorem}
  %
  \begin{proof}
  La coda della serie esponenziale può essere stimata 
  tramite la somma di una serie geometrica:
  \begin{align*}
    \abs{f(x) - \sum_{k=0}^n \frac {x^k} {k!}}
    & \le \sum_{k=n+1}^{+\infty} \frac{\abs{x}^k}{k!} 
     = \sum_{k=0}^{+\infty}\frac{\abs{x}^{n+k+1}}{(n+k+1)!} \\
    & \le \sum_{k=0}^{+\infty}\frac{\abs{x}^{n+1+k}}{(n+1)^{k+1}\cdot n!}
     = \frac{\abs{x}^{n+1}}{(n+1)\cdot n!}\sum_{k=0}^{+\infty} \enclose{\frac{\abs{x}}{(n+1)}}^k \\
    & = \frac{\abs{x}^{n+1}}{(n+1)\cdot n!} \cdot \frac{1}{1-\frac{\abs{x}}{(n+1)}} 
     = \frac{\abs{x}^{n+1}}{(n+1-\abs{x})\cdot n!}.
  \end{align*}
  Se $\abs{x}\le 1$ si ottiene quindi la stima~\eqref{eq:stima_exp}.
  Se $x=1$ la serie esponenziale è a termini positivi e dunque 
  approssima $e$ per difetto: si ottiene dunque~\eqref{eq:stima_e}.
  Per $n=5$ si ha
  \[
   \sum_{k=0}^5 \frac{1}{k!} = 1 + 1 + \frac 1 2 + \frac{1}{6} + \frac {1}{24} + \frac{1}{120}
   = \frac{326}{120}
  \]
  Dunque da un lato
  \[
    e \ge \frac{326}{120} \ge 2.716
  \]
  e dall'altro
  \[
   e \le \frac{326}{120} + \frac{1}{5\cdot 5!}
     \le 2.717 + 0.002 = 2.719
  \]
\end{proof}
  
\begin{theorem}[irrazionalità di $e$]
\mymargin{$e\not\in \QQ$}%
\index{$e\not\in \QQ$}%
\mymark{**}%
\index{irrazionalità!di $e$}%
\index{$e$!è irrazionale}%
Il numero $e$ è irrazionale.
\end{theorem}
%
\begin{proof}
Supponiamo per assurdo che sia $e=p/q$ con $p\in \ZZ$ e $q \in \NN$.
Possiamo supporre $q>1$
(non importa che la frazione sia ridotta ai minimi termini).

Per ogni $n\in \NN$ si ha 
\[
  n! \cdot e = \sum_{k=0}^{+\infty} \frac{n!}{k!}
   = \sum_{k=0}^n \frac{n!}{k!} + n!\sum_{k=n+1}^{+\infty} \frac{1}{k!}.
\]
Il primo addendo nella somma precedente è intero
in quanto se $k\le n$ il rapporto $n!/k!$ è intero.
Grazie al teorema~\ref{th:approx_e}
per il secondo addendo se $n>1$ si ha:
\[
0 < n! \cdot \sum_{k=n+1}^{+\infty} \frac{1}{k!}
\le \frac{1}{n} < 1.
\]
Risulta dunque che per ogni $n\ge 2$ il numero $n! e$
è strettamente compreso tra due interi consecutivi e quindi non 
è mai intero. Dunque $e$ non può essere razionale perché 
se fosse $e=\frac p q$ si avrebbe $n! e \in \ZZ$ per ogni 
$n>q$.
\end{proof}

\begin{comment}
\subsection{definizione alternativa delle funzioni trigonometriche}

In questa sezione proponiamo una definizione alternativa 
delle funzioni trigonometriche a partire dall'esponenziale complesso.

\begin{definition}[funzioni trigonometriche]%
\label{def:sincos}%
\index{$\cos$}%
\index{$\sin$}%
\index{funzioni!trigonometriche}%
\index{funzioni!circolari}%
\index{trigonometria!funzioni elementari}%
Per ogni $x\in \RR$ si potrà definire
\[
  \cos x = \Re \enclose{e^{ix}}, \qquad
  \sin x = \Im \enclose{e^{ix}}
\]
cosicché valga per definizione la
\emph{formula di Eulero}%
\mymargin{formula di Eulero}%
\index{formula!di Eulero}%
\index{Eulero!formula di}%
\mymark{***}%
\[
  e^{ix} = \cos x + i \sin x.
\]
\end{definition}

\begin{theorem}[proprietà delle funzioni seno e coseno]
\index{proprietà!delle funzioni seno e coseno}%
\mymark{***}%
Se definiamo 
\[
\cos x = \Re e^{ix},
\]
Le funzioni $\sin$ e $\cos$ appena definite
soddisfano le stesse proprietà 
delle corrispondenti funzioni definite 
nel capitolo~\ref{sec:funzioni_trigonometriche}:
\mynote{Queste proprietà caratterizzano univocamente le funzioni 
$\sin$ e $\cos$ e quindi effettivamente coincidono con quelle 
che avevamo già definito. 
La dimostrazione più semplice richiede però l'utilizzo delle 
derivate.
}
\begin{enumerate}
\item
$\displaystyle
\cos(x) = \frac{e^{ix}+e^{-ix}}{2}$,
$\displaystyle
\sin(x) = \frac{e^{ix}-e^{-ix}}{2i}$;
\item
$\sin(-x) = -\sin x$ (la funzione $\sin$ è dispari),
$\cos(-x) = \cos x$ (la funzione $\cos$ è pari);
\item identità fondamentale della trigonometria:
\index{trigonometria!identità fondamentale}%
\index{identità!fondamentale della trigonometria}%
\index{formula!fondamentale della trigonometria}%
\[
\cos^2 x + \sin^2 x = 1;
\]
\item
formule di addizione:
\index{formula!di addizione}%
\index{trigonometria!formule di addizione}%
\index{addizione!formule della trigonometria}%
\begin{gather*}
\cos(\alpha+\beta) = \cos \alpha \cos \beta - \sin \alpha \sin \beta,\\
\sin(\alpha+\beta) = \sin \alpha \cos \beta + \cos \alpha \sin \beta;
\end{gather*}
\item
le funzioni $\cos\colon \RR\to\RR$ e $\sin \colon \RR \to \RR$
sono continue;
\item si ha
\index{sviluppo in serie!coseno}%
\index{sviluppo in serie!seno}%
\index{$\cos$!sviluppo in serie}%
\index{$\sin$!sviluppo in serie}%
\begin{align}
\label{eq:serie_cos}
\cos x &= \sum_{k=0}^{+\infty} (-1)^k\frac{x^{2k}}{(2k)!}
  = 1 - \frac{x^2}{2} + \frac{x^4}{4!} - \frac{x^6}{6!} + \dots
\\
\label{eq:serie_sin}
\sin x &= \sum_{k=0}^{+\infty} (-1)^k\frac{x^{2k+1}}{(2k+1)!}
  = x - \frac{x^3}{6} + \frac{x^5}{5!} - \frac{x^7}{7!} + \dots
\end{align}
\item
vale il seguente limite notevole
\[
 \lim_{x\to 0} \frac{\sin x}{x} = 1
 \qquad\text{e}\qquad
 \lim_{x\to 0} \frac{1-\cos x}{x^2} = \frac 1 2.
\]
\end{enumerate}
\end{theorem}
%
%
\begin{proof}
\mymark{*}
\begin{enumerate}
\item
Essendo $\overline{e^{ix}} = e^{\overline{ix}} = e^{-ix}$
discende dalla formula~\eqref{eq:re_im} per il calcolo
di parte reale ed immaginaria.

\item
Si verifica direttamente con le formule precedenti.

\item
Per $x\in \RR$ si ha da un lato
\[
  \abs{\exp (ix)}^2 = \exp(ix)\cdot \overline{\exp(ix)}
  = \exp(ix)\cdot \exp(-ix)
  = \exp(0) = 1
\]
e dall'altro
\[
  \abs{\exp(ix)}^2 = \abs{\cos x + i \sin x}^2
    = \cos^2 x + \sin^2 x.
\]

\item
Grazie alla formula che esprime l'esponenziale
della somma:
\begin{align*}
\cos(\alpha+\beta) + i \sin(\alpha + \beta)
&= \exp(i(\alpha + \beta))
= \exp(i\alpha) \cdot \exp(i\beta) \\
&= (\cos \alpha + i \sin \alpha) \cdot (\cos \beta + i \sin \beta)\\
&= \cos \alpha \cos \beta - \sin \alpha \sin \beta \\
&\quad + i (\sin \alpha \cos \beta + \cos \alpha \sin \beta)
\end{align*}
e uguagliando parte reale e parte immaginaria si ottengono le formule
di addizione.

\item
Visto che la funzione $\exp\colon \CC \to \CC$ è continua
anche la sua restrizione all'asse immaginario lo è.
E dunque anche parte reale (coseno) e parte immaginaria (seno) lo sono.

\item
Sia $x\in \RR$.
Osservando che $i^{2k} = (i^2)^k = (-1)^k$ e $i^{2k+1}= i \cdot i^{2k}
= i\cdot(-1)^k$ suddividendo i termini pari e dispari
della serie che definisce l'esponenziale si ha:
\begin{align*}
  \exp(ix)
  &= \sum_{k=0}^{+\infty} \frac{i^k x^k}{k!}
  = \sum_{k=0}^{+\infty}\frac{i^{2k} x^{2k}}{(2k)!}
    + \sum_{k=0}^{+\infty}\frac{i^{2k+1} x^{2k+1}}{(2k+1)!}\\
  &= \sum_{k=0}^{+\infty}\frac{(-1)^k x^{2k}}{(2k)!}
    +  i \sum_{k=0}^{+\infty}\frac{(-1)^k x^{2k+1}}{(2k+1)!}.
\end{align*}
Osserviamo ora che le due serie che compaiono a destra dell'uguaglianza
sono a termini reali e quindi la loro somma è reale.
Dunque queste due serie coincidono con la parte reale e la parte immaginaria di $\exp(ix) = \cos x + i \sin x$.

\item
Per la corrispondente proprietà dell'esponenziale
sappiamo che per $x \to 0$ si ha
\[
  \frac{e^{i x}-1}{i x} \to 1.
\]
Ma
\[
  \frac{e^{ix}-1}{i x}
  = \frac{\cos x - 1 + i \sin x}{i x}
  = \frac{\sin x}{x} + i\frac{1- \cos x}{x}.
\]
Scopriamo dunque che
\[
  \frac{1-\cos x}{x} \to 0
  \qquad\text{e}\qquad
  \frac{\sin x}{x} \to 1.
\]
D'altra parte si ha
\begin{align*}
  \frac{1-\cos x}{x^2}
  &=\frac{(1-\cos x)\cdot(1+\cos x)}{x^2(1+\cos x)}
  = \frac{1-\cos^2 x}{x^2} \cdot \frac{1}{1+\cos x}\\
  &= \enclose{\frac{\sin x}{x}}^2\cdot \frac 1{1+\cos x}
  \to 1 \cdot \frac 1 {1+\cos 0} = \frac 1 2.
\end{align*}

\end{enumerate}
\end{proof}

Definiamo $\pi$ come la misura in radianti dell'angolo piatto, 
cioè dell'angolo identificato dal numero $e^{i\pi} = -1$ 
sul piano complesso.

\begin{theorem}[definizione di $\pi$]
\index{$\pi$!definizione}%
\index{$\tau$!definizione}%
\label{th:pi}%
\mymargin{$\pi$}%
Definiamo $\pi$ come il più piccolo numero reale
positivo in cui si annulla la funzione $\sin x$.
Si ha $\pi \in [2.8,4]$.
Le funzioni $e^{ix}$, $\sin x$ e $\cos x$ sono
periodiche di periodo $\tau = 2\pi$:
\mymargin{$\tau$}%
\[
  e^{i(x+2\pi)} = e^{ix}, \qquad
  \cos(x + 2\pi) = \cos x, \qquad
  \sin(x + 2\pi) = \sin x.
\]
La funzione $\sin x$ è strettamente crescente nell'intervallo
$\closeinterval{-\frac \pi 2}{\frac \pi 2}$
mentre la funzione $\cos x$ è strettamente decrescente 
nell'intervallo $\closeinterval{0}{\pi}$.
Si ha infine:
\begin{align*}
\sin(\pi + x) = -\sin x, \qquad 
\cos(\pi + x) = -\cos x.
\end{align*}

Vale inoltre la celeberrima formula di Eulero:
\index{Eulero!formula di}%
\index{formula!di Eulero}%
\[
  e^{i\pi} + 1 = 0.
\]
\end{theorem}
%%
%%
\begin{proof}
Il teorema~\ref{th:approx_exp} ci permette di approssimare la 
funzione esponenziale $e^{ix}$ per $x\in [0,1]$. 
Ci proponiamo quindi di determinare l'andamento delle funzioni 
$\sin x$ e $\cos x$ in tale intervallo e definire in tale intervallo 
il valore di $\alpha=\frac \pi 4$.
Tramite tale valore $\alpha$ e tramite le formule di addizione riusciremo 
a determinare l'andamento delle funzioni $\sin$ e $\cos$ su tutta la 
retta reale.

Se $x\in[0,1]$ grazie al teorema~\ref{th:approx_exp} applicato 
con $n=3$ sappiamo che 
\begin{align*}
  \abs{e^{ix} - \enclose{1 + ix  + \frac{(ix)^2}{2} + \frac{(ix)^3}{6}}}
  \le \frac{\abs{ix}^4}{18}
\end{align*}
sviluppando
\begin{align*}
  \abs{\cos x + i \sin x - 1 - ix + \frac{x^2}{2} + i\frac{x^3}{6}}
  \le \frac{x^4}{18}.
\end{align*}
Il valore assoluto delle parti reale e immaginaria di un numero 
complesso sono minori del modulo del numero complesso.
Dunque si ottiene:
\begin{align}\label{eq:86035}
    \abs{\cos x - 1 + \frac{x^2}{2}} \le \frac{x^4}{18}\smallskip
    \qquad\text{e}\qquad
    \abs{\sin x - x + \frac{x^3}{6}} \le \frac{x^4}{18}.
\end{align}
Per la funzione $\sin x$ si ha dunque
\begin{align*}
    x - \frac{x^3}{6} - \frac{x^4}{18}
    \le \sin x 
    \le x - \frac{x^3}{6} + \frac{x^4}{18}
\end{align*}
e sapendo che $0<x<1$ possiamo affermare che 
\[
  x - \frac{x^3}{6} - \frac{x^4}{18}
  \ge x - \frac{x}{6} - \frac{x}{18}
  = \frac{7}{9} x
\]
e 
\[
  x - \frac{x^3}{6} + \frac{x^4}{18}
  \le x - \frac{x^3}{6} + \frac{x^3}{18}
  \le x.
\]
Dunque se $x\in[0,1]$ si ha 
\begin{equation}\label{eq:4757614}
  \frac{7}{9} x \le \sin x \le x.
\end{equation}
Per il coseno da un lato osserviamo che da~\eqref{eq:86035}
si ottiene, se $x\in [0,1]$:
\[
\cos x 
  \ge 1 - \frac{x^2}{2} -\frac{x^4}{18}   
  \ge 1 - \frac{1}{2} - \frac{1}{18}
  = \frac 4 9 > 0.
\]
Dall'altro lato si ha
\[
\cos x 
\le 1 - \frac{x^2}{2} + \frac{x^4}{18}
= 1 - \frac{x^2}{2} + \frac{x^2}{18}
\le 1 - \frac{4}{9} x^2.
\]

Vogliamo ora dimostrare che la funzione $\cos x$ 
è strettamente decrescente sull'intervallo $[0,1]$.
Infatti se $x\in[0,1]$ e $t\in(0,1]$ si ha 
\begin{align*}
  \cos(x+t) 
  = \cos x \cos t - \sin x \sin t
  \le \cos x \cos t < \cos x
\end{align*}
in quanto $\sin x\ge 0$, $\sin t \ge 0$ 
e $\cos t < 1 - \frac{5}{9}t^2 < 1$ se $t>0$.
Dunque la funzione $\cos x$ è strettamente 
decrescente su $[0,1]$. 
Visto che $\cos^2 x + \sin^2 x = 1$ 
deduciamo che la funzione $\sin^2 x$ 
è strettamente crescente in $[0,1]$.
Inoltre $\sin x$ è positiva su tale 
intervallo e dunque anche $\sin x$
è strettamente crescente in $[0,1]$.

Vogliamo ora dimostrare che esiste 
$\alpha \in [0,1]$ tale per cui 
$\sin \alpha = \frac{\sqrt 2}{2}$.
Visto che $\sin x$ è una funzione continua, 
e $\sin 0 = 0$, per utilizzare 
il teorema~\ref{th:zeri} dei valori intermedi 
basterà verificare che $\sin 1 > \frac{\sqrt 2}{2}$.
E infatti si ha:
\[
\sin 1 \ge \frac 7 9 > \frac{\sqrt 2}{2}.
\]
Dunque esiste un unico $\alpha\in [0,1]$ 
tale che $\sin \alpha = \frac{\sqrt 2}{2}$.
Allora anche $\cos \alpha = \frac{\sqrt 2}{2}$ 
(in quanto $\sin^2 + \cos^2 = 1$)
e quindi
\[
  e^{2i\alpha} 
  = \enclose{\frac{\sqrt 2}{2} \enclose{1+i}}^2
  = \frac 1 2 \enclose{1+2i-1} = i.
\]
Dunque $\alpha$ è la misura in radianti di metà 
angolo retto.

Se poniamo $\pi = 4\alpha$ scopriamo dunque che 
\[
  e^{i\pi} 
  = \enclose{e^{2i\alpha}}^2
  = i^2 = -1
\]
da cui $\sin \pi = 0$ e $\cos \pi = -1$.
Analogamente da $e^{2i\alpha}=i$ 
troviamo che $\sin \frac \pi 2 = 1$
$\cos \frac \pi 2 = 0$.

Dalla stima~\eqref{eq:4757614} si ottiene 
\[
  \frac 7 9 \alpha 
  \le \frac{\sqrt 2}{2}
  \le \alpha
\]
da cui $\pi=4\alpha \in [2.8, 4]$.

Abbiamo verificato che sull'intervallo 
$\closeinterval{0}{\frac \pi 4}$ la funzione $\sin x$ è strettamente 
crescente mentre $\cos x$ è strettamente decrescente.
Dalle formule di addizione possiamo quindi dedurre che 
$\sin\enclose{\frac \pi 2 - x} = \cos x$ 
e $\cos\enclose{\frac \pi 2 -x} = \sin x$.
Dunque anche sull'intervallo $\closeinterval{\frac \pi 4}{\frac \pi 2}$ 
la funzione $\sin x$ è crescente e $\cos x$ è decrescente.
Ancora, tramite le formula di addizione troviamo che 
$\sin(\pi - x)=\sin x$ e $\cos(\pi - x) = -\cos x$ 
da cui possiamo dedurre che sia la funzione $\sin x$ 
che la funzione $\cos x$ sono strettamente decrescenti 
sull'intervallo $\closeopeninterval{\frac \pi 2}{\pi}$.
Possiamo dunque affermare che $\sin x>0$ se $0<x<\pi$ 
e dunque $\pi$ è il più piccolo reale positivo in cui 
la funzione $\sin$ si annulla.

Visto che $\sin(\pi+x) = -\sin x$ e $\cos(\pi+x)=-\cos x$
possiamo stabilire l'andamento delle funzioni $\sin$ e $\cos$
anche sull'intervallo $\closeinterval{\pi}{2\pi}$.
Infine essendo $e^{2i\pi} = 1$ si osserva che la funzione 
$e^{2ix}$ e di conseguenza le funzioni $\sin x$ e $\cos x$
sono periodiche di periodo $2\pi$.
\end{proof}
\end{comment}

%%%%
%%%%
%%%%
