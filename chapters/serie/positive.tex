\section{serie a termini positivi}

\index{serie!a termini positivi}
Nel seguito considereremo serie i cui termini sono numeri reali
positivi (o almeno non negativi).
Quando scriveremo $a_n >0$ (o $a_n \ge 0$) sarà sempre
sottointeso che $a_n\in \RR$ visto che per i numeri complessi non
reali non abbiamo definito la relazione d'ordine.

\begin{theorem}[carattere delle serie a termini positivi]\label{th:serie_positiva}
\mymark{***}%
\mymargin{carattere delle serie a termini positivi}%
\index{carattere!di una serie a termini positivi}%
Se $a_n\ge 0$
la serie $\sum a_n$ è regolare:
o converge oppure diverge a $+\infty$.
\end{theorem}
%
\begin{proof}
\mymark{***}
Se $a_n \ge 0$ essendo $S_{n+1} = S_n + a_n$ significa che
la successione $S_n$ delle somme parziali è crescente.
Dunque il limite delle $S_n$ esiste e non può essere negativo.
\end{proof}

\begin{theorem}[criterio del confronto]
\mymark{**}
\mymargin{criterio del confronto}
\index{criterio!del confronto per serie}
Siano $\sum a_n$ e $\sum b_n$ serie a
termini positivi che si confrontano: $0\le a_n\le b_n$.
Allora
\[
  \sum_{k=0}^{+\infty} a_k \le \sum_{k=0}^{+\infty} b_k.
\]
In particolare se $\sum b_n$ converge anche $\sum a_n$ converge
e se $\sum a_n$ diverge anche $\sum b_n$ diverge.

Quest'ultimo risultato vale anche se $0 \le a_n \ll b_n$.
\end{theorem}
%
\begin{proof}
\mymark{*}
Se $S_n$ sono le somme parziali di $\sum a_n$ e $R_n$ sono le somme
parziali di $\sum b_n$ si ha $S_n \le R_n$ e il risultato
si riconduce al confronto tra successioni.

Nel caso in cui $a_n \ll b_n$ per definizione sappiamo che $\frac{a_n}{b_n}\to 0$
e quindi dalla definizione di limite sappiamo che
esiste $N$ tale che per ogni $n>N$ si ha (avendo scelto $\eps=1$)
\[
  \frac{a_n}{b_n} < 1.
\]
Dunque si ottiene $a_n \le b_n$ per tutti gli $n$ tranne al più un numero
finito. Sapendo che il carattere della serie non cambia se si modifica
la serie su un numero finito di termini ci si riconduce al caso precedente.
\end{proof}

\begin{example}\label{ex:52573}
\mymark{***}
La serie
\begin{equation}\label{eq:296453}
 \sum_{k=1}^{+\infty} \frac{1}{k^2}
\end{equation}
è convergente.
Infatti osservando che si ha per ogni $n>0$
\[
  \frac{1}{(n+1)^2} \le \frac{1}{n(n+1)}
\]
possiamo affermare che
\[
  \sum_{k=1}^{+\infty} \frac{1}{k^2}
  = 1 + \sum_{k=1}^{+\infty} \frac{1}{(k+1)^2}
  \le 1+ \sum_{k=1}^{+\infty} \frac{1}{k(k+1)}
  = 2
\]
in quanto ci siamo ricondotti alla
serie telescopica di Mengoli che ha somma pari a $1$.

Sappiamo quindi che la serie~\eqref{eq:296453} è convergente
senza sapere esattamente quale sia la sua somma.
Possiamo però trovare numericamente delle approssimazioni
della somma, facendo la somma dei primi termini
e stimando l'errore tramite la serie di Mengoli,
di cui sappiamo calcolare la somma.
Infatti se $S_N$ è la somma parziale dei primi
$N$ termini e $S = \lim S_N$ è la somma della serie,
essendo $1/(k+1)^2 \le 1/(k^2+k)$ si ha
\[
S_N
\le S
\le S_N + \sum_{k=N}^{+\infty} \frac{1}{k(k+1)}
\le S_N + \frac{1}{N}.
\]
Per calcolare le prime 6 cifre decimali esatte basterà
quindi sommare il primo milione di termini della serie.
Lo si può fare, ad esempio, con il codice riportato
a pagina~\pageref{code:series}, ottenendo $S\approx 1.644934$

Utilizzando strumenti molto più avanzati
Eulero \index{Eulero}
(Leonard Euler, 1707--1783) è
riuscito ad esprimere la somma di questa serie mediante costanti matematiche fondamentali
(noi lo faremo con un metodo diverso nell'esercizio~\ref{ex:Basilea}).
\end{example}

\begin{corollary}[criterio del confronto asintotico]
\mymark{*}
\mymargin{criterio del confronto asintotico}
\index{criterio!del confronto asintotico}
Se $a_n$ e $b_n$ sono successioni a termini positivi,
asintoticamente equivalenti (definizione~\ref{def:ordine_infinito}),
allora le serie corrispondenti $\sum a_n$ e $\sum b_n$
hanno lo stesso carattere.
\end{corollary}
%
\begin{proof}
\mymark{*}
Le serie a termini positivi non possono essere indeterminate
quindi è sufficiente verificare che se una serie converge, converge anche l'altra.
Essendo $a_n / b_n$ convergente tale rapporto deve anche essere
limitato (teorema~\ref{th:limitatezza_successione_convergente}), 
quindi esiste $C\in \RR$ tale che
\[
   a_n \le C \cdot b_n.
\]
Se la serie $\sum b_n$ converge anche $\sum C \cdot b_n$ converge e, per confronto,
converge anche $\sum a_n$.

Viceversa, scambiando il ruolo di $a_n$ e $b_n$ si verifica che se $a_n$
converge, converge anche $b_n$.
\end{proof}

\begin{example}
La serie
\[
\sum_n \frac{n^2+2n+3}{2n^4-n^3+n+1}
\]
è convergente. Infatti si può facilmente verificare che
\[
   \frac{n^2+2n+3}{2n^4-n^3+n+1} \sim \frac{1}{2n^2}.
\]
Ma sappiamo che la serie $\sum 1/n^2$ è convergente, di conseguenza
anche la serie $\sum 1/(2n^2)$ lo è (per linearità della somma, 
teorema~\ref{th:linearita_somma_serie})
e quindi, per confronto
asintotico, anche la serie data è convergente.
\end{example}

Nel capitolo \ref{sec:serie_segno_variabile} vedremo che il criterio di convergenza 
asintotica può essere applicato solamente alle serie a termini positivi.

\begin{theorem}[criterio della radice]
\mymargin{criterio della radice}
\index{criterio!della radice}
Sia $\sum a_n$ una serie a termini non negativi
(cioè $a_n\ge 0$) tale che
\mymark{***}
$\sqrt[n]{a_n} \to \ell \in [0,+\infty]$.
Se $\ell<1$ allora la serie converge.
Se $\ell>1$ allora la serie diverge.

Più in generale il risultato è valido con
\[
  \ell = \limsup \sqrt[n]{a_n}
\]
anche nel caso in cui il limite di $\sqrt[n]{a_n}$ non dovesse esistere.
\end{theorem}
%
\begin{proof}
\mymark{***}
Nel caso $\ell < 1$
prendiamo $q$ con $\ell < q < 1$ e poniamo $\eps = q-\ell$.
Per la definizione di limite $\sqrt[n]{a_n}\to \ell$
(ma basta che sia $\limsup \sqrt[n]{a_n}=\ell$)
sappiamo
esistere $N$ tale che per ogni $n > N$ si abbia
\[
  \sqrt[n]{a_n} < \ell + \eps = q
\]
cioè
\[
   a_n < q^n.
\]
Sapendo che $\sum q^n$ converge, sapendo anche che il carattere
della serie non cambia modificando un numero finito di termini,
per confronto possiamo concludere che anche la serie $\sum a_n$ converge.

Se $\ell>1$ si ha che $\sqrt[n]{a_n}>1$ e quindi $a_n>1$ per infiniti valori di $n$. La successione $a_n$ non è infinitesima e quindi la serie non può convergere.
\end{proof}

\begin{example}
La serie
\[
  \sum_k 2^{(\ln k) - k}
\]
è convergente. Infatti si ha
\[
 \sqrt[k]{2^{\ln k - k}}
 = 2^{\frac{\ln k - k}{k}}
 = 2^{\frac{\ln k }k - 1}
 \to 2^{-1}
 = \frac{1}{2}
 < 1.
\]
\end{example}

\begin{theorem}[criterio del rapporto]
\mymark{***}
\mymargin{criterio del rapporto}
\index{criterio!del rapporto per serie}
Sia $\sum a_n$ una serie a termini non negativi
($a_n\ge 0$)
tale che $a_{n+1} / a_n \to \ell \in [0,+\infty]$.
Se $\ell <1$ allora la serie converge.
Se $\ell > 1$ la serie diverge.
\end{theorem}
%
\begin{proof}
\mymark{*}
Non sarebbe difficile fare una dimostrazione diretta, simile alla dimostrazione fatta per il criterio della radice.
Possiamo però osservare che
per il criterio di convergenza alla Cesàro (teorema~\ref{th:criterio_cesaro}) si ha $\sqrt[n]{a_n} \to \ell$
quindi ci riconduciamo al criterio della radice senza dover fare ulteriori dimostrazioni.
\end{proof}

\begin{example}
\mymark{***}
Per ogni $x\ge 0$ la serie
\[
  \sum \frac{x^n}{n!}
\]
converge.
\end{example}
%
\begin{proof}
Applichiamo il criterio del rapporto.
Posto $a_n = x^n / n!$ si ha
\[
\frac{a_{n+1}}{a_n}
= \frac{x^{n+1}}{(n+1)!}\cdot \frac{n!}{x^n}
= \frac{x}{n+1} \to 0 < 1.
\]
Dunque la serie converge.
\end{proof}

\subsection{associatività della somma di una serie}

\begin{example}
Posto $a_k = (-1)^k$ sappiamo che la serie corrispondente:
\[
  1 - 1 + 1 - 1 + 1 \dots
\]
è indeterminata perché le somme parziali sono
\[
  S_n = \sum_{k=0}^n (-1)^k
  = \begin{cases}
   1 & \text{se $n$ pari}\\
   0 & \text{se $n$ è dispari.}
  \end{cases}
\]
Se però associamo i termini a due a due otteniamo la serie
\[
  (1-1) + (1-1) + (1-1) \dots = 0 + 0 + 0 + \dots = 0
\]
che è convergente.
Il motivo è che la successione delle somme
parziali di questa nuova serie è una particolare estratta (quella
con gli indici pari) della successione delle somme parziali della
serie originale. Quindi non ci dovrebbe sorprendere il fatto che
nonostante la serie originale fosse indeterminata è possibile
associare i termini della serie in modo da ottenere una serie
regolare (in questo caso convergente).
In effetti per il teorema~\ref{th:bolzano_weierstrass} 
di Bolzano-Weierstrass sappiamo che
è sempre possibile estrarre una sottosuccessione regolare (convergente o divergente)
da qualunque successione.
Anche quando da una serie indeterminata si estrae una serie
convergente la somma della serie può dipendere da come i termini
vengono associati.
Se ad esempio nella serie precedente prendessimo
solamente le somme di indice dispari
otterremmo:
\[
  1 + (-1+1) + (-1+1) + \dots = 1.
\]
\end{example}

L'esempio precedente ci mostra che associando i termini di una
serie indeterminata è possibile ottenere serie con carattere
e somma diversi.
Il seguente teorema ci dice che questo fenomeno \emph{cattivo}
può solo avvenire quando si parte da una serie indeterminata.
Se la serie è regolare allora possiamo associarne i termini
senza modificarne né il carattere né la somma.
In particolare questo è vero per le serie a termini positivi.

\begin{theorem}[associatività delle serie regolari]%
\label{th:serie_associativa}%
Se $\sum a_k$ è una serie, scelta comunque
una successione crescente $k_n$ con $k_0=0$
possiamo considerare la serie $\sum b_n$
i cui termini
\[
  b_n = \sum_{j=k_n}^{k_{n+1}-1} a_j
\]
si ottengono associando i termini di $a_k$ a gruppi
consecutivi delimitati dalla successione di indici
$k_n$.

Se la serie $\sum a_k$ è regolare (convergente o divergente)
allora anche la serie $\sum b_n$ è regolare e si ha
\[
\sum_{n=0}^{+\infty} b_n
= \sum_{k=0}^{+\infty} a_k.
\]

In particolare questo vale se la serie $\sum a_k$ è a termini
positivi.
\end{theorem}
%
\begin{proof}
Siano $S_k = \sum_{j=0}^k a_j$ le somme parziali della
serie $\sum a_j$. Allora le somme parziali della serie $\sum b_n$
non sono altro che la sottosuccessione $S_{k_n}$.
Dunque se $S_k$ converge anche ogni sua sottosuccessione
converge allo stesso limite.

Il teorema~\ref{th:serie_positiva} ci dice
che le serie a termini positivi sono regolari e quindi
soddisfano le ipotesi del teorema.
\end{proof}

\subsection{la serie armonica}

Osserviamo che il criterio del rapporto non si applica alla
\emph{serie armonica}%
\mymargin{serie armonica}%
\index{serie!armonica}%
\[
  \sum_k \frac{1}{k}
\]
in quanto
\[
 \frac{\frac{1}{k+1}}{\frac{1}{k}}
 = \frac{k}{k+1} \to 1.
\]

Per capire se la serie armonica converge o diverge presentiamo il metodo
di \emph{condensazione} che verrà enunciato in generale nel prossimo teorema
ma che può essere meglio compreso se applicato al caso particolare
della serie armonica.

Mostreremo che la serie armonica diverge.
L'idea è semplicemente quella di associare gli addendi della serie armonica
in gruppi di lunghezza potenze di due e stimare la somma di ogni gruppo dal basso
con il termine più piccolo (cioè l'ultimo) di ogni gruppo:
\begin{align*}
 \sum_{k=1}^{+\infty} \frac{1}{k}
 & = 1 + \frac 1 2
     + \enclose{\frac 1 3 + \frac 1 4}
     + \enclose{\frac 1 5 + \frac 1 6 + \frac 1 7 + \frac 1 8}
     + \dots\\
 & > 1 + \frac 1 2 + 2 \cdot \frac 1 4 + 4 \cdot \frac 1 8 + \dots \\
   & = 1 + \frac 1 2 + \frac 1 2 + \frac 1 2 + \dots
    = +\infty.
\end{align*}

\begin{theorem}[criterio di condensazione di Cauchy]%
\label{th:condensazione}%
\mymark{**}%
\mymargin{criterio di condensazione di Cauchy}%
\index{criterio!di condensazione di Cauchy}%
\index{condensazione!criterio di Cauchy}%
\index{Cauchy!criterio di condensazione}%
Sia $a_n$ una successione decrescente di numeri reali non negativi:
$a_n \ge 0$.
Allora la serie $\sum a_k$ converge se e solo se converge
la serie
\[
  \sum 2^k a_{2^k}.
\]
\end{theorem}
%
\begin{proof}
\mymark{**}
Supponiamo per comodità che le somme partano da $k=1$.
Si tratta di raggruppare i termini $a_k$ in gruppi di potenze di due:
\begin{align*}
  b_0 & = a_1, \\
  b_1 &= a_2 +  a_3, \\
  b_2 &= a_4 +  a_5 +  a_6 +  a_7, \\
  b_3 &= a_8 +  a_9 +  a_{10} + \dots + a_{15}, \\
  &\vdots\\
  b_n &= a_{2^n} +  a_{2^{n}+1} + \dots + a_{2^{n+1}-1},\\
  &\vdots
\end{align*}
Grazie al teorema~\ref{th:serie_associativa} sulla associatività
delle serie a termini positivi sappiamo che
  \[
  \sum_{k=1}^{+\infty} a_k = \sum_{n=0}^{+\infty} b_n.
  \]
Ogni termine $b_n$ è la somma di $2^n$ termini della
successione $a_k$ che, essendo la successione decrescente,
possono essere stimati dall'alto e dal basso con il primo
e l'ultimo termine di ogni somma, dunque:
\[
  2^n a_{2^{n+1}-1} \le b_n \le 2^n a_{2^{n}}.
\]
Per ipotesi la serie $\sum 2^n a_{2^n}$ è convergente
dunque per confronto anche la serie $\sum b_n$ è convergente.

Viceversa se la serie $\sum 2^n a_{2^n}$ è divergente
anche la serie $\sum 2^{n+1} a_{2^{n+1}}$ è divergente
e sapendo che $b_n\ge 2^n a_{2^{n+1}-1}\ge \frac 1 2 2^{n+1} a_{2^{n+1}}$
otteniamo, per confronto, che anche la serie $\sum b_n$ è divergente.
\end{proof}

\begin{corollary}[serie armonica generalizzata]
\mymark{***}
\mymargin{serie armonica generalizzata}%
\index{serie!armonica!generalizzata}
La serie
\[
 \sum_n \frac{1}{n^\alpha}
\]
converge se $\alpha>1$,
diverge se $0\le \alpha\le 1$.
\end{corollary}
%
\begin{proof}
\mymark{***}
Applichiamo il criterio di condensazione. Posto $a_n = \frac 1{n^\alpha}$ Si ha
\[
  \sum_n 2^n a_{2^n} = \sum_n 2^n \frac{1}{(2^n)^\alpha}
  = \sum_n 2^{n(1-\alpha)}
  = \sum_n \enclose{2^{(1-\alpha)}}^n
\]
che è una serie geometrica di ragione $q=2^{1-\alpha}$.
Se $\alpha>1$ allora $q<1$ e la serie armonica è convergente
se invece $\alpha \le 1$ allora $q\ge 1$ e la serie
armonica è divergente.
\end{proof}

\begin{exercise}
Utilizzare il criterio di condensazione per dimostrare che la serie
\[
  \sum \frac{1}{n \cdot \ln n}
\]
diverge.
\end{exercise}

\begin{exercise}
  Determinare i valori dei parametri $\alpha\in \RR$ e $\beta\in \RR$
  per i quali la seguente serie risulta essere convergente:
  \[
    \sum \frac{1}{n^\alpha (\ln n)^\beta}.
  \]
\end{exercise}

%%%
%%%
