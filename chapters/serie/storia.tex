\section{successione armonica}

Ci si può chiedere come mai la sottosuccessione
\[
 1, \frac 1 2, \frac 1 3, \frac 1 4, \frac 1 5, \dots
\]
viene chiamata \emph{armonica}.
La storia racconta che Pitagora si affacciò un giorno all'officina di un fabbro 
da cui provenivano i suoni dei martelli che battevano sulle incudini.
Si accorse che alcuni martelli producevano suoni tra loro armonici, mentre 
altri no. 
Pitagora studiò il fenomeno cercando di produrre suoni di diverse frequenze 
utilizzando delle corde con tensioni diverse. 
Sostanzialmente scoprì che suoni armonici corrispondevano a frequenze 
commensurabili, ovvero che potevano essere scritte come rapporti di numeri interi.

Ovviamente l'armonicità è una sensazione soggettiva prodotta dal nostro cervello 
ma possiamo ipotizzare come mai una frequenza che è multiplo di un'altra 
possa essere percepita come armonica.
Prendiamo come esempio uno strumento a corda. 
Se pizzichiamo la corda induciamo una vibrazione che 
può essere descritta come una sovrapposizione di onde sinusoidali
(si veda la teoria delle serie di Fourier, nel capitolo~\ref{sec:convergenza_integrale}).
Essendo la corda fissata agli estremi una possibile oscillazione 
è quella in cui la corda assume la forma del grafico di una funzione sinusoidale 
con semiperiodo lungo quanto la corda stessa.
Ma ci sono altre possibili oscillazioni che si possono sovrapporre a questa. 
Ad esempio una oscillazione con periodo uguale alla lunghezza della corda, 
dove si crea un nodo (un punto in cui la corda non oscilla) al centro della corda.
Se tocchiamo la corda ad un terzo della sua lunghezza forzeremo la creazione di un 
nodo in quel punto. Questo impedirà la formazione di queste due prime onde fondamentali 
e ci permetterà di sentire solo le \emph{armoniche} corrispondenti ad oscillazioni 
con almeno un nodo in quel punto.
In buona sostanza ci sarà una sovrapposizioni di onde sinusoidali (pure)
con frequenze che sono multiplo della frequenza fondamentale.
Ovviamente man mano che le frequenze aumentano la loro intensità diminuisce tendendo 
a zero, altrimenti la sovrapposizione (somma della serie) 
delle sinusoidi sarebbe divergente.
La diversa intensità di queste frequenze armoniche determinerà il timbro dello strumento 
musicale. In un flauto, ad esempio, c'è un \emph{ventre} invece che un nodo 
nell'estremità aperta del tubo e si otterranno di conseguenza solo i multipli dispari della frequenza fondamentale.
Visto che è molto difficile ottenere un suono puro, la presenza di armoniche 
è onnipresente nei suoni prodotti in natura. 
Dunque può essere considerato naturale che due suoni con frequenze 
una multiplo dell'altra, siano percepiti come armonici, 
mentre suoni con frequenze in rapporti lontani da un numero intero "piccolo"
saranno percepiti come dissonanti.

Possiamo basarci sulle considerazioni precedenti per capire come vengono definite 
le note musicali nel sistema temperato della musica occidentale.
Scegliamo un suono di riferimento prodotto ad esempio da una corda di una 
certa lunghezza e chiamiamolo \emph{tonica}.
Se riduciamo a metà la lunghezza di questa corda otteniamo un suono 
con frequenza fondamentale doppi, diremo una \emph{ottava} (cosa c'entra il numero 8 lo scopriremo 
più avanti) sopra la tonica.
La nostra percezione ci dice chiaramente che questo suono è più acuto 
del precedente, ma è talmente simile che siamo portati a dargli lo stesso nome. 
Lo stesso accade se raddoppiamo la corda, ovvero raddoppiamo la frequenza. 
Otterremo la \emph{stessa} nota ma più grave, diremo una \emph{ottava} sotto la tonica. 
La nostra percezione delle frequenze è dunque \emph{logaritmica} e non lineare.
Il passaggio da una frequenza a quella doppia è percepito come lo stesso intervallo 
indipendentemente dalla frequenza di partenza. 
Se triplichiamo la frequenza otteniamo una nota \emph{diversa} ma molto assonante 
con la tonica. La chiamiamo \emph{dominante}.
L'intervallo tra tonica e dominante viene chiamato \emph{quinta} (vedremo più avanti cosa 
c'entra il numero 5).
Visto che raddoppiare e dimezzare la frequenza non cambia la percezione della nota,
la dominante può essere rappresentata dal rapporto $\frac 3 2$, intermedio tra 1 e 2.
D'altra parte la relazione armonica che c'è tra la tonica e la dominante 
può essere invertita, dividendo la frequenza per tre. 
Possiamo riportare la frequenza $\frac 1 3$ nell'intervallo tra $1$ e $2$ 
moltiplicando per quattro (ovvero salendo di due ottave) si ottiene 
il rapporto $\frac 4 3$ che viene chiamato \emph{sottodominante}
in quanto è inferiore a quello di $\frac 3 2$.

Se la nota con la frequenza fondamentale da cui siamo partiti la chiamiamo Do
(tonica), 
le note con frequenza doppia e metà la chiameremo sempre Do, e saranno le ottave 
superiori ed inferiori rispetto al Do di partenza. 
Le frequenze triple e un terzo saranno chiamate Sol e Fa rispettivamente.
Nell'intervallo tra due Do consecutivi con frequenze $f$ e $2f$ ci saranno un Fa
di frequenza $\frac 4 3 f$ e un Sol di frequenza $\frac 3 2 f$. 
Se passiamo alla scala logaritmica potremo capire meglio cosa sta accadendo in quanto 
i rapporti tra frequenze diventano differenze tra i loro logaritmi.
Facendo i logaritmi in base $2$ dei rapporti con la frequenza della nota Do 
di riferimento, i Do corrisponderanno a numeri interi. 
Il Do di riferimento
corrisponderà a $0$, l'ottava sopra (frequenza doppia) all'$1$ la seconda ottava sopra al $2$,
l'ottava inferiore al $-1$ e così via.
La frequenza tripla dei Sol, riportata nell'intervallo $[0,1]$ corrisponderà a 
$\log_2\frac 3 2 \approx 0.585$
e quella un terzo dei Fa a $\log_2 \frac 4 3 \approx 0.415$.

\begin{figure}
    \begin{center}
        \begin{tabular}{l|ccccccccccccc}
            note   &   Do   &   Reb           &   Re     &    Mib        &   Mi          &    Fa    &  Solb/Fa\#                       &   Sol    &  Lab           &   La          &   Sib         &   Si            \\
            quinte &   $0$  &  $-5$           &  $+2$    &    $-3$       &  $+4$         &   $-1$   &  $-6/+6$                         &  $+1$    &  $-4$          &  $+3$         &  $-2$         &  $+5$           \\
            freq   &   $1$  &$\frac{128}{243}$&$\frac 98$&$\frac{32}{27}$&$\frac{81}{64}$&$\frac 43$&$\frac{1024}{729}/\frac{729}{512}$&$\frac 32$&$\frac{128}{81}$&$\frac{27}{16}$&$\frac{16}9$   &$\frac{243}{128}$\\
        $\log_2$   &\small 0&\small .075 &\small  .170& \small  .245& \small  .340 & \small  .415& \small  .490 / .510         &\small  .585&\small  .660&\small  .755  &\small  .830  &\small  .925    \\
              temp &\small 0&\small .083 &\small  .167& \small  .250& \small  .333 & \small  .417& \small  .500                &\small  .583&\small  .667&\small  .750  &\small  .833  &\small  .917    \\     
        \end{tabular}
    \end{center}
    \end{figure}
    
Possiamo ora cercare di completare la scala delle note musicali aggiungendo le frequenze triple 
e un terzo delle note Fa e Sol che abbiamo aggiunto. 
Visto che l'intervallo tra Do e Sol e tra Fa e Do si chiama intervallo di quinta (per un motivo che vedremo)
quello che stiamo cercando di costruire si chiama circolo delle quinte.
La frequenza tripla di una nota nella scala logaritmica a meno di ottave corrisponde a sommare $0.585$
ovvero sottrarre $0.415$. Salendo (di una quinta) da un Sol si ottiene una nota 
di frequenza $\frac 9 8$ del Do di riferimento il cui logaritmo ha parte frazionaria $\approx 0.170$.
Chiamiamo $Re$ questa nota. 
Proseguendo in questo modo, moltiplicando e dividendo le frequenze per $3$, 
si ottengono delle note che si spargono sull'intervallo delle frequenze 
tra $f$ e $2f$. 
Le frequenze multiple di $3$ partendo dal Do ci danno nell'ordine le note: 
Sol, Re, La, Mi, Si e Fa\#. Dividendo le frequenze per 3 si 
ottengono nell'ordine le note: Fa, Sib, Mib, Lab, Reb e Solb.
Le frequenze delle note che abbiamo chiamato Fa\# e Solb sono molto vicine 
tra loro: $\frac{3^6}{2^9} = \frac{729}{512} \approx 1.42$ 
e $\frac{2^10}{3^6} = \frac{1024}{729} \approx 1.40$.
I logaritmi in base due di queste frazioni sono $\log_2 \frac{729}{512} \approx 0.510$ 
e $\log_2 \frac{1024}{729} \approx 0.490$. 
Le due frequenze risultano praticamente indistinguibili
e possiamo quindi identificare le due note come la stessa nota.
La differenza tra le frequenze di queste due note si chiama \emph{comma musicale}.
Si ottengono quindi 12 note diverse distribuite in modo quasi uniforme nell'intervallo 
tra la frequenza di base $f$ e la sua ottava $2f$.
La correzione di questo errore di costruzione può essere distribuita sulle diverse 
note in modo diverso, ottenendo diversi sistemi di temperamento.
La scala \emph{ben temperata} (che si è imposta dopo che Bach ne ha mostrato
la versatilità nella composizione) distribuisce il comma in modo uniforme
tra le note, ottenendo una scala in cui la frequenza tra due note consecutive 
(semitono) ha come rapporto esattamente la dodicesima radice di due: 
$\sqrt[12]{2}\approx 1.06$ in questo modo i logaritmi delle frequenze suddividono 
l'intervallo $[0,1]$ in dodici parti uguali di ampiezza $\frac 1 {12} \approx 0.83$.

Quello che abbiamo determinato fin'ora è il motivo per cui sono state scelte 
12 diverse note musicali, 
che si riconduce in ultima analisi al fatto che 12 è una potenza di 3 che si approssima 
molto bene tramite potenze di 2 in quanto $3^12$ è molto vicino a $2^19$.
La sequenza delle più piccole potenze di $3$ che meglio approssimano le potenze 
di $2$ (in errore relativo) è: 1, 2, 5, 12, 41, 53, 306, \dots
dunque la scelta di 12 note è una scelta molto naturale (anche la scelta di una scala 
pentatonica è comune in molte culture musicali).

Abbiamo quindi visto come il rapporto $\frac 1 2$ determina l'univocità delle note 
(note con frequenza doppia sono la stessa nota) e come il rapporto $\frac 1 3$ 
determina la scelta delle note musicali (e il loro numero).

Le armoniche successive possono essere legate alla \emph{armonia musicale} ovvero 
alla composizione degli accordi e delle scale. 
Il rapporto $\frac 1 4$ non genera nuove note, in quanto come $\frac 1 2$ rappresenta 
la stessa nota musicale. Il rapporto successivo è dunque $\frac 1 5$ che approssimativamente 
rappresenta un Mi (se, come sopra, abbiamo scelto il Do come nota di riferimento).
Dunque le frequenze $1,3,5$ generano l'accordo di maggiore, formato dalle 
note Do, Sol, Mi. 
Componendo gli accordi maggiori sulla tonica (Do, Mi, Sol) 
sulla dominante (Sol, Si, Re) e sulla sottodominante (Fa, La, Do) si ottengono
le sette note della scala maggiore: Do, Re, Mi, Fa, Sol, La, Si.
Per questo motivo l'intervallo tra Do e Sol si chiama \emph{quinta}, in quanto comprende 
cinque delle sette note e l'intervallo tra Do e Fa si chiama \emph{quarta}.
Inoltre l'intervallo che comprende l'intera scala si chiama \emph{ottava} in quanto 
comprende come ottava nota il Do superiore.

% ranking scale 
% sorted([(abs(1-2**round(log(3**n)/log(2))/3**n),n) for n in range(100)])[:10]
% [(0.0, 0), (0.0020859537426888286, 53), (0.009418849414340569, 94), (0.01152885180860852, 41), (0.013459631454855736, 12), (0.015517509028936116, 65), (0.023190618041241784, 82), (0.025329407756841116, 29), (0.026738101230810996, 24), (0.02876828053116498, 77)]
%
% 53 0.0020859537426888286
% 41 0.01152885180860852
% 12 0.013459631454855736
% 5 0.053497942386831365
% 2 0.11111111111111116
% 1 0.33333333333333326






