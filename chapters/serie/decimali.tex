\section{rappresentazione posizionale dei numeri reali}
%%%
%%%
Quando scriviamo $\frac{3}{8} = 0.375$ intendiamo che vale 
\[
\frac 3 8 = \frac{3}{10} + \frac{7}{10^2} + \frac{5}{10^3}.  
\]
Più in generale data una base $d\in \NN$, $d\ge 2$, ($d=10$ nel 
caso della rappresentazione decimale)
consideriamo l'insieme $\Enclose{d} = \ENCLOSE{0,1,2,\dots, d-1}$ 
delle cifre in base $d$.
Una sequenza infinita di cifre sarà quindi un elemento 
$\vec a \in \Enclose{d}^\NN$, $\vec a = (a_0,a_1,\dots, a_n, \dots)$
con $a_k\in \Enclose{d}$.
Potremo quindi considerare il numero ``$0.a_0 a_1 a_2 \ldots$'' rappresentato 
dalla sequenza di cifre $\vec a$:
\[
  r(\vec a) = \sum_{k=0}^{+\infty} \frac{a_k}{d^{k+1}}.
\]
Chiaramente $r(\vec a)\in [0,1]$ in quanto essendo $0\le a_k\le d-1$ 
risulta
\begin{equation}\label{eq:10445934}
 0 \le r(\vec a) \le \sum_{k=0}^{+\infty} \frac{d-1}{d^{k+1}}
  = \frac{d-1}{d}\sum_{k=0}^{+\infty}\frac 1 {d^k}
  = \frac{d-1}{d}\cdot \frac{1}{1-\frac 1 d} = 1.
\end{equation}

Ogni numero $x\in [0,1)$ ammette una rappresentazione in cifre $x=r(\vec a)$ 
con $\vec a \in \Enclose{d}^\NN$. 
Infatti per ogni $N\in \NN$ possiamo scrivere 
\[
  \lfloor x\cdot 10^N \rfloor= \sum_{k=0}^{N-1} a_k 10^{N-1-k}
\]
e al crescere di $N$ otteniamo una sequenza di cifre $a_k\in \Enclose{d}$ 
tali che 
\[
  \abs{x \cdot d^N - \sum_{k=0}^{N-1} a_k \cdot d^{N-1-k}} \le 1
\]
da cui 
\[
  \abs{x - \sum_{k=0}^{N-1} \frac{a_k}{d^{k+1}}} \le \frac{1}{d^{N}}
\]
che, facendo tendere $N\to +\infty$, significa $r(\vec a) = x$.
Il numero $x=1$ può essere anch'esso rappresentato, basta prendere 
$a_k=d-1$ per ogni $k\in \NN$ cosicché si ottiene l'uguaglianza 
nel lato destro di \eqref{eq:10445934}. 
In base $d=10$ questo si esprime dicendo che 
\[
 0.999\ldots = 1.  
\]

Ci possiamo chiedere se è possibile che lo stesso numero 
abbia due rappresentazioni in cifre distinte.
Supponiamo quindi che esistano $\vec a,\vec b\in \Enclose{d}^\NN$
con $\vec a \neq \vec b$ 
tali che $r(\vec a) = r(\vec b)$. 
Sia $m = \min\ENCLOSE{n\in\NN\colon a_n\neq b_n}$ la posizione 
della prima cifra diversa tra $\vec a$ e $\vec b$.
Si ha allora 
\[
 r(b) - r(a) = \sum_{k=0}^{+\infty}\frac{b_k - a_k}{d^{k+1}}
 = \frac{b_m - a_m}{d^{m+1}} + \sum_{k=m+1}^{+\infty} \frac{b_k - a_k}{d^{k+1}}
 = A+B
\]
con 
\[
\abs{A} = \abs{\frac{b_m - a_m}{d^{m+1}}} \ge \frac{1}{d^{m+1}}  
\]
e 
\begin{align*}
\abs{B} &= \abs{\sum_{k=m+1}^{+\infty} \frac{b_k - a_k}{d^{k+1}}}
        \le \sum_{k=m+1}^{+\infty} \frac{d-1}{d^{k+1}}\\
        &= \frac{d-1}{d^{m+2}}\sum_{k=0}^{+\infty}\frac{1}{d^k}
        = \frac{d-1}{d^{m+2}}\cdot \frac{1}{1-\frac 1 d} = \frac{1}{d^{m+1}}.
\end{align*}
Dunque, per disuguaglianza triangolare inversa,
\[
0 = \abs{r(b)-r(a)}\ge \abs{A} - \abs{B} \ge 0.  
\]
Significa che tutte le disuguaglianze sono in realtà uguaglianze
e quindi deve essere $\abs{b_m-a_m}=1$ 
e per ogni $k>m$ deve essere $\abs{b_k-a_k}=d-1$. 
Supponendo che sia $b_m=a_m+1$ (l'altro caso è analogo)
per $k>m$ dovrà necessariamente essere $b_k=0$ e $a_k=d-1$.

Ad esempio se $d=10$, $\vec a = (1,2,3,9,9,9,9,\dots )$ 
e $\vec b = (1,2,4,0,0,0,0,\dots)$ si avrà 
$r(\vec a) = r(\vec b) = 0.124$.

\begin{theorem}[Cantor]
  \label{th:cantor_secondo}
L'insieme di Cantor
\[
  C = \{r(\vec a)\colon \vec a \in \ENCLOSE{0,2}^\NN\}
\]
ha cardinalità
\[
\# C = \# \mathcal P(\NN).
\]
In particolare $\#\RR > \#\NN$.
\end{theorem}
%
\begin{proof}
Vogliamo verificare che $r\colon \ENCLOSE{0,2}^\NN \to \closeinterval{0}{1}$
è iniettiva.
Abbiamo già visto in generale che $r(\vec a) = r(\vec b)$ solamente 
se la prima cifra diversa in $\vec a$ e $\vec b$ differisce di una unità.
Ma siccome tutte le cifre di $\vec a$ e $\vec b$ per ipotesi sono $0$ oppure 
$2$, non differiscono di una unità e quindi $r$ è iniettiva.
Dunque $\# C = \#\ENCLOSE{0,2}^\NN$.
D'altra parte $\#\ENCLOSE{0,2}^\NN = \#\mathcal P(\NN)$ in quanto 
ogni $\vec a \in \ENCLOSE{0,2}^\NN$ può essere messo in corrispondenza 
biunivoca con l'insieme $\vec a^{-1}(\ENCLOSE{0})$ degli indici $k\in \NN$
per cui $a_k=0$.

Dal teorema~\ref{th:Cantor} deduciamo che $\# C = \#\mathcal P(\NN) > \#\NN$ 
e visto che $C\subset \RR$ a maggior ragione $\# \RR \ge \# C >\#\NN$.

D'altra parte possiamo mostrare che $\#\RR \le \# \mathcal P(\NN)$
in quanto possiamo costruire una funzione $f\colon \RR \to \mathcal P(\QQ)$
in questo modo:
\[
f(x) = \ENCLOSE{q\in \QQ\colon q<x}.
\]
Per la densità di $\QQ$ in $\RR$ sappiamo che se $x<y$ 
esiste $q\in \QQ$ con $x<q<y$ e dunque $q\in f(y)\setminus f(x)$ 
da cui $f(x)\neq f(y)$. 
Dunque $f$ è iniettiva, che significa $\#\RR\le \# \mathcal P(\QQ)$
Ma $\#\NN = \#\QQ$ dunque $\#\mathcal P(\QQ) = \#\mathcal P(\NN)$ 
e la dimostrazione è conclusa.
\end{proof}

L'insieme $C$ definito nella precedente dimostrazione con $d=3$ 
è lo stesso \emph{insieme di Cantor}%
\mymargin{insieme di Cantor}%
\index{insieme!di Cantor}
\index{Cantor!insieme di}% 
considerato nell'esempio~\ref{ex:insieme_Cantor}.
Infatti si potrebbe mostrare facilmente che 
$C=\frac 1 3 C \cup (\frac 2 3 + \frac 1 3 C)$.

%%%%%%%%%%%%%
%%%%%%%%%%%%%
