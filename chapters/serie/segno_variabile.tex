\section{serie a segno variabile}
\label{sec:serie_segno_variabile}
\index{serie!a segni alterni}

La serie $\sum \frac{(-1)^k}{k+1}$ la cui somma si può scrivere come
\mymargin{serie armonica a segni alterni}%
\index{serie!armonica!a segni alterni}%
\[
1 - \frac{1}{2} + \frac{1}{3} - \frac{1}{4} +  \frac{1}{5} \dots
\]
non è assolutamente convergente
(in quanto la serie $\sum \frac 1 {k+1}$ è divergente) ma ha il termine generico
infinitesimo. 
Non abbiamo quindi nessun criterio che ci permetta di
determinarne il carattere.
Possiamo però sfruttare il fatto che i segni sono \emph{alterni} cioè
che i termini di indice pari hanno segno opposto ai termini di indice dispari. 
Si nota infatti che posto
\[
  S_n = \sum_{k=0}^n \frac{(-1)^k}{k+1}
\]
si ha
\begin{align*}
S_{2n+2}
  &= S_{2n+1} + \frac{1}{2n+3}
  = S_{2n} - \frac{1}{2n+2} + \frac{1}{2n+3}
  < S_{2n}\\
S_{2n+3}
  &= S_{2n+2} - \frac{1}{2n+4}
  = S_{2n+1} + \frac{1}{2n+3} - \frac{1}{2n+4}
  > S_{2n+1} \\
\end{align*}
Dunque la successione delle somme parziali di indice pari è decrescente mentre
sui termini di indice dispari è crescente. Avremo quindi che entrambe
le sottosuccessioni hanno limite: $S_{2n} \to S$, $S_{2n+1} \to R$.

Ma
\[
  S - R = \lim_{n\to +\infty} (S_{2n} - S_{2n+1}) = \lim_{n\to+\infty}\frac{1}{2n+2} = 0.
\]
Dunque $S=R$ e l'intera successione ha limite $S$. 
D'altra parte $S \le S_0$ in quanto $S_{2n}$ è decrescente e $S\ge S_1$ in quanto $S_{2n+1}$ è crescente. 
Concludiamo che $S$ è finito e dunque la serie è convergente.%
\mynote{%
Per determinare il valore della somma di questa serie ci serviranno degli strumenti più avanzati.
Si veda l'equazione~\ref{eq:serie_ln2}.
}

Questa dimostrazione può essere resa più in generale nel seguente.

\begin{theorem}[criterio di Leibniz per le serie a segno alterno]
\label{th:Leibniz}%
\mymark{***}%
\mymargin{criterio di Leibniz}%
\index{criterio!di Leibniz}%
\index{teorema!di Leibniz}%
\index{Leibniz!criterio di}%
Sia $b_n$ una successione monotòna e infinitesima. Allora
la serie
\[
  \sum (-1)^{n} b_n
\]
è convergente.

Più precisamente se $\displaystyle S_n = \sum_{k=0}^n (-1)^k b_k$
sono le somme parziali
si osserva che la somma della serie $S= \lim S_n$ è sempre compresa
tra due termini consecutivi della successione $S_n$. 
Cioè, se $b_k\ge 0$, si ha
\[
  S_{2n+1} \le S \le S_{2n}.
\]
\end{theorem}
%
\begin{proof}
\mymark{***}
Senza perdere di generalità possiamo supporre che la successione $b_n$ sia decrescente e quindi $b_n \ge 0$ (visto che il limite è zero).
Posto
\[
 S_n = \sum_{k=0}^n (-1)^k b_k
\]
si ha
\begin{align*}
  S_{2n+2} &= S_{2n} - b_{2n+1} + b_{2n+2} \\
  S_{2n+3} &= S_{2n+1} + b_{2n+2} - b_{2n+3}.
\end{align*}
Essendo $b_n$ decrescente si ha $b_{2n+2} < b_{2n+1}$ e $b_{2n+3} < b_{2n+2}$ da cui
\[
  S_{2n+2} < S_{2n}, \qquad S_{2n+3} > S_{2n+1}.
\]
Dunque le successioni $S_{2n}$ e $S_{2n+1}$ sono monotone e di conseguenza
hanno limite:
\[
  S_{2n} \to S, \qquad S_{2n+1} \to R
\]
con $S, R  \in [-\infty, +\infty]$.
D'altronde, essendo $b_n$ infinitesima
\[
  S - R
  = \lim_{n\to +\infty} S_{2n} - S_{2n+1}
  = \lim_{n\to +\infty} b_{2n+1} = 0.
\]
Dunque $S=R$. Inoltre essendo $S_{2n}$ decrescente si ha
$S \le S_0$ ed essendo $S_{2n+1}$ crescente si ha $S\ge S_1$.
Dunque $S$ è finito e la serie converge.

Abbiamo anche ottenuto che
$S_{2n-1} \le S \le S_{2n}$ e $S_{2n+1} \le S \le S_{2n}$
dunque è verificata anche la seconda parte dell'enunciato.
\end{proof}

Il teorema precedente si applica, ad esempio, alle seguenti serie,
che risultano essere convergenti anche se non assolutamente 
convergenti:
\[
   \sum \frac{(-1)^n}{\sqrt n}, 
   \sum \frac{(-1)^n}{n\ln n}.
\]

Si faccia attenzione alla verifica delle ipotesi nel teorema precedente.
Visto che si tratta di una serie alternata, non si può applicare 
il criterio di confronto asintotico per ricondurre i termini $b_n$ 
ai termini di una serie più semplice da studiare. 
In particolare la monotonia dei termini $b_n$ non è garantita 
dall'esistenza di una successione asintotica e monotona. 
Si veda il seguente.

\begin{example}
Sia $a_k = \frac{(-1)^k}{\sqrt k} - \frac{1}{k}$.
Si noti che $a_k \sim c_k = \frac{(-1)^k}{\sqrt k}$
ma le due serie $\sum a_k$ e $\sum c_k$ hanno caratteri diversi:
la prima è divergente in quanto si spezza nella differenza 
di una serie convergente e di una divergente,
mentre la seconda è convergente per il criterio di 
Leibniz (teorema~\ref{th:Leibniz}).
\end{example}

Grazie al criterio di Leibniz è facile 
produrre esempi di serie convergenti
ma non assolutamente convergenti.
Il seguente teorema ci dice che per le serie di questo tipo 
la somma della serie dipende dall'ordine in cui sono stati presi i termini:
anzi, si può ottenere come somma qualunque valore si voglia.

\begin{theorem}[convergenza condizionata]%
\label{th:convergenza_condizionata}%
\index{convergenza!condizionata di una serie}%
\mymargin{convergenza condizionata}%
Sia $\sum a_k$ una serie convergente ma non assolutamente convergente a termini reali.
Allora fissato qualunque $x \in [-\infty , +\infty]$ esiste un riordinamento
$\sigma \colon \NN \to \NN$ biettivo tale che
\[
  \sum_{k=0}^{+\infty}  a_{\sigma(k)} = x.
\]
\end{theorem}
%
\begin{proof}
Dividiamo i termini della successione $a_k$ in termini maggiori o uguali a zero
e in termini negativi. Sia $a^+_k$ la sottosuccessione dei termini non negativi
e $-a^-_k$ la sottosuccessione
dei termini negativi (quindi $a^+_k\ge 0$ e $a^-_k > 0$). Si avrà
\begin{align*}
  \sum_{k=0}^n a_k &= \sum_{k=0}^{n^+} a^+_k - \sum_{k=0}^{n^-} a^-_k \\
  \sum_{k=0}^n \abs{a_k} &= \sum_{k=0}^{n^+} a^+_k + \sum_{k=0}^{n^-} a^-_k
\end{align*}
dove $n^+ +1$ e $n^-+1$ sono rispettivamente
il numero di termini non-negativi e negativi
tra i primi $n+1$ termini della successione $a_k$.

Osserviamo ora che dovrà essere
\[
\sum_{k=0}^{+\infty} a_k^+ = +\infty \qquad \text{e} \qquad
\sum_{k=0}^{+\infty} a_k^- = +\infty.
\]
Innanzitutto le somme esistono perché le serie sono a termini non negativi.
Se entrambe queste somme fossero finite allora la serie $\sum\abs{a_k}$ sarebbe convergente, ma per ipotesi abbiamo assunto che $\sum a_k$ non fosse
assolutamente convergente.
Quindi almeno una delle due somme è infinita. Se la somma dei termini positivi
fosse infinita e quella dei termini negativi fosse finita potremmo però
concludere che anche la somma della serie $\sum a_k$ sarebbe infinita.
Viceversa se la somma dei termini positivi fosse finita e quella dei termini
negativi fosse infinita la somma $\sum a_k$ sarebbe $-\infty$. Ma per ipotesi
abbiamo richiesto che la serie $\sum a_k$ fosse convergente.

Fissato $x\in \RR$ possiamo quindi cominciare a sommare i termini positivi
$a^+_k$ finché non si raggiunge o si supera il valore $x$. A quel punto cominciamo a sommare i termini negativi finché non torniamo sotto al valore $x$.
Poi continuiamo a sommare i termini positivi finché non si torna a superare $x$
e di nuovo poi continuiamo con i termini negativi finché non si torna a scendere
sotto $x$. Intermezzando opportunamente termini positivi e termini negativi
riusciamo quindi ad ottenere delle somme parziali che oscillano intorno al valore 
di $x$ e si avvicinano sempre di più a $x$ in quanto ad ogni cambio di ``rotta'' 
la distanza da $x$ è inferiore al valore assoluto dell'ultimo termine sommato 
e la successione dei termini $a_k$ è infinitesima in quanto la serie $\sum a_k$ è convergente.

Stessa cosa si può fare per ottenere una somma $x=+\infty$. Fissata una qualunque
successione $x_n \to +\infty$ comincio a sommare i termini positivi finché non supero il valore $x_1+a_1^-$. 
Poi sommo un solo termine negativo, $-a_1^-$ e ottengo una somma maggiore di $x_1$. 
Poi sommo tanti positivi finché non supero $x_2+a_2^-$. 
Poi sommo un altro unico termine negativo e così via. 
Chiaramente le somme tenderanno a $+\infty$.

Il caso $x=-\infty$ si tratta in maniera analoga.
\end{proof}

