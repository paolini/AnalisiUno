\section{serie telescopiche}

Una serie scritta nella forma
\[
  \sum (a_{k} - a_{k+1})
\]
viene detta \emph{telescopica}
\mymargin{serie telescopica}%
\index{serie!telescopica}
in quanto i singoli termini della somma (come i tubi di un cannocchiale),
si semplificano uno con l'altro (permettendo al cannocchiale di chiudersi):
\[
  S_n 
  = \sum_{k=0}^{n-1} (a_{k} - a_{k+1})
  = \sum_{k=0}^{n-1} a_k - \sum_{k=1}^{n} a_k
  = a_0 - a_n.
\]

In linea teorica ogni serie può essere scritta in forma telescopica,
data $S_n$ basta infatti scegliere $a_n=-S_n$, 
affinché valga la relazione precedente. 
Scrivere una serie in forma telescopica è quindi equivalente a 
determinare la successione delle somme parziali.

\begin{example}[serie di Mengoli]
\mymark{**}
Si ha
\[
  \sum_{n=1}^{+\infty} \frac{1}{n(n+1)} = 1.
\]
\end{example}
%
\begin{proof}
\mymark{**}
Infatti
\[
  \sum_{k=1}^n \frac{1}{k(k+1)}
  = \sum_{k=1}^n \enclose{\frac{1}{k} - \frac{1}{k+1}}
  = \sum_{k=1}^n \frac{1}{k} - \sum_{k=2}^{n+1} \frac{1}{k}
  = 1 - \frac{1}{n+1} \to 1.
\]
\end{proof}

