\chapter{spazi di funzioni}

%%%%%%%%%%%%%%%%%%%%%%%%
%%%%%%%%%%%%%%%%%%%%%%%%

\begin{figure}
  \centering
  \begin{tikzpicture}[node distance=2cm]
    \draw node at (0,0)[name=insieme]{insieme};
    \draw node at (-1,-1)[name=sv]{sp. vettoriale};
    \draw node at (-1,-3)[name=sn]{sp. normato};
    \draw node at (-1,-5)[name=euclideo]{sp. euclideo};
    \draw node at (2,-1)[name=st]{sp. topologico};
    \draw node at (2,-2)[name=sm]{sp. metrico};
    \draw node at (2,-3.5)[name=completo]{sp. completo};
    \draw node at (1,-4.5)[name=banach]{Banach};
    \draw node at (1,-6)[name=hilbert]{Hilbert};
    \draw[->](sv) -- (insieme);
    \draw[->](st) -- (insieme);
    \draw[->](sn) -- (sv);
    \draw[->](sm) -- (st);
    \draw[->](euclideo) -- (sn);
    \draw[->](hilbert) -- (euclideo);
    \draw[->](hilbert) -- (banach);
    \draw[->](banach) -- (completo);
    \draw[->](completo) -- (sm);
    \draw[->](sn) -- (sm);
    \draw[->](banach) -- (sn);

    \begin{scope}[green]
    \draw node at (-2,0)[name=insiemeT]{$\in,\subset,\cup$};
    \draw (insiemeT) -- (insieme);
    \draw node at (-3,-1)[name=svT]{$+,\cdot$};
    \draw (svT) -- (sv);
    \draw node at (-3,-3)[name=snT]{$\abs{v}$};
    \draw (snT) -- (sn);
    \draw node at (-3,-4.5)[name=euclideoT]{$\langle v,w\rangle$};
    \draw (euclideoT) -- (euclideo);
    \draw node at (5,0)[name=stT]{$\lim, \stackrel\circ A, \bar A$};
    \draw (stT) -- (st);
    \draw node at (5,-1.75)[name=smT]{$d(x,y)$, Lip};
    \draw (smT) -- (sm);
    \end{scope}

    \begin{scope}[blue]
    \draw node at (5,-1)[name=Rbar,left]{$\bar \RR$};
    \draw[->](Rbar) -- (st);
      \draw node at (5,-2.5)[name=Q,left]{$\QQ$};
      \draw[->](Q) -- (sm);
      \draw node at (5,-4)[name=Sn,left]{$\mathbb S^n$};
      \draw[->](Sn) -- (completo);
      \draw node at (4.5,-5)[name=Cn,left]{$\mathbb C^n$, $L^p$, $\ell^p$};
      \draw[->](Cn) -- (banach);
      \draw node at (4,-6.5)[name=L2,left]{$L^2$, $\ell^2$};
      \draw[->](L2) -- (hilbert);
      \draw node at (1,-7)[name=Rn]{$\RR^n$};
      \draw[->](Rn) -- (hilbert);
    \end{scope}
  \end{tikzpicture}
\caption{Strutture matematiche che astraggono lo spazio $\RR^n$. La freccia nera significa: ``è un'', la freccia blu: ``è un esempio di'', la linea verde: ``è una operazione tipica di''.}
\end{figure}


\begin{definition}[spazio metrico]
\mymark{**}
\label{def:distanza}
Diremo
che $d\colon X\times X\to\RR$ è una \myemph{distanza} su $X$
se per ogni $x,y,z\in X$ valgono le seguenti proprietà
\begin{enumerate}
\item
  $d(x,y)\ge 0$ (positività);
\item
  $d(x,y)\le d(x,z) + d(z,y)$ (disuguaglianza triangolare);
\item
  $d(x,y)=0$ se e solo se $x=y$ (separazione);
\item
  $d(x,y) = d(y,x)$ (simmetria).
\end{enumerate}

Se $d$ è una distanza
diremo che $X$ è uno
\emph{spazio metrico}
\mynote{spazio metrico}%
\index{spazio!metrico}%
(metrizzato da $d$).
\end{definition}

Osserviamo che dalle disuguaglianze triangolari:
\[
  d(x,z) \le d(x,y) + d(y,z), \qquad
  d(y,z) \le d(x,y) + d(x,z)
\]
si ottiene la \emph{disuguaglianza triangolare inversa}:
\mynote{disuguaglianza triangolare inversa}
\index{disuguaglianza!triangolare!inversa}
\[
  d(x,y) \ge \abs{d(x,z) - d(y,z)}.
\]


\begin{definition}[spazio normato]
\mymark{*}
\label{def:norma}
Sia $V$ uno spazio vettoriale sul campo $\RR$.
Una funzione $\phi \colon V\to \RR$ si
dice essere una \myemph{norma} su $V$ se
per ogni $v,w\in V$ e per ogni $\lambda \in \RR$
valgono le seguenti proprietà:
\begin{enumerate}
\item
  $\phi(v) \ge 0$ (positività);
\item
  $\phi(\lambda v) = \abs{\lambda} \cdot \phi(v)$ (omogeneità e simmetria);
\item
  $\phi(v+w) \le \phi(v) + \phi(w)$
  (disuguaglianza triangolare);
\item
  $\phi(v)=0$ se e solo se $v=0$ (separazione).
\end{enumerate}

Se $\phi$ è una norma su $V$ diremo che
$V$
è uno spazio vettoriale \myemph{normato} da $\phi$.

Se $\phi$ è una norma la funzione $d(v,w) = \phi(v-w)$
è chiaramente una distanza che si chiama
\emph{distanza indotta}
\mynote{distanza indotta}
\index{distanza!indotta}
da $\phi$.
In particolare ogni spazio normato è anche uno spazio metrico rispetto alla
distanza indotta dalla norma. Inoltre la disuguaglianza triangolare
assicura che la norma $\phi\colon V\to \RR$ risulta essere una funzione continua.
\end{definition}

Usualmente si utilizzano le notazioni $\abs{v}$ o $\Abs{v}$ per indicare
una norma $\phi(v)$.

\begin{definition}[spazio euclideo]
\label{def:prodotto_scalare}
Sia $V$ uno spazio vettoriale sul campo $\RR$.
Una funzione $b\colon V\times V \to \RR$ si dice
essere un \myemph{prodotto scalare} su $V$ se
$b$ è una forma bilineare, simmetrica e definita
positiva, ovvero se
per ogni $u,v,w\in V$ e per ogni $\lambda, \mu \in \RR$
valgono le seguenti proprietà:
\begin{enumerate}
\item $b(v,v) \ge 0$ (positività);
\item $b(\lambda u + \mu v, w) = \lambda b(u,w) + \mu b(v,w)$
e $b(u,\lambda v+ \mu w) = \lambda b(u,v) + \mu b(u,w)$
(bilinearità);
\item $b(v,w) = b(w,v)$ (simmetria);
\item $b(v,v) = 0$ se e solo se $v=0$ (non degenerazione).
\end{enumerate}

Se $b$ è un prodotto scalare su $V$ diremo che $V$ è uno
spazio \emph{euclideo} (con prodotto scalare $b$).

Usualmente si utilizzano le notazioni $v\cdot w$, $(v,w)$, $\langle v,w\rangle$
o $\langle v \vert w\rangle$ per denotare un prodotto scalare $b(v,w)$.
\end{definition}

Se $b$ è un prodotto scalare su $V$ il teorema seguente ci garantisce che
che la funzione $\phi(v) = \sqrt{\langle v,v\rangle}$ è una norma su $V$.

Dunque uno spazio euclideo ha, in modo naturale, una struttura di spazio
normato e di spazio metrico.

\begin{theorem}[proprietà del prodotto scalare]
Sia $V$ uno spazio euclideo con prodotto scalare $\langle v, w\rangle$.
Si definisca $\abs{v} = \sqrt{\langle v,v\rangle}$.
Allora $\abs{v}$ è una norma su $V$
e per ogni $v,w\in V$ valgono le seguenti proprietà:
\begin{enumerate}
\item \myemph{sviluppo del binomio}
\[
  \abs{v+w}^2 = \abs{v}^2 + 2 \langle v,w\rangle + \abs{w}^2 ;
\]
\item \myemph{teorema di Pitagora}
\index{teorema!di Pitagora}%
\index{Pitagora!teorema}%
\begin{equation}\label{eq:Pitagora}
  \langle v,w\rangle=0 \implies \abs{v+w}^2 = \abs{v}^2 + \abs{w}^2:
\end{equation}
\item \myemph{disuguaglianza di Young}
\index{Young!disuguaglianza di}%
\begin{equation}\label{eq:Young}
  \langle v,w\rangle \le \frac{\abs{v}^2 + \abs{w}^2}{2};
\end{equation}
\item \myemph{disuguaglianza!di Cauchy-Schwarz}
\index{Cauchy-Schwarz!disuguaglianza di}%
\begin{equation}\label{eq:Cauchy_Schwarz}
   \langle v,w \rangle \le \abs{v} \cdot \abs{w};
\end{equation}
\item \myemph{proprietà del parallelogramma}
\index{parallelogramma!proprietà del}
\begin{equation}\label{eq:parallelogramma}
  \abs{v+w}^2 + \abs{v-w}^2 = 2 \abs{u}^2 + 2\abs{v}^2.
\end{equation}
\item \emph{continuità}
\[
  \text{se $v_k\to v$ e $w_k\to w$ allora
  $\langle v_k,w_k\rangle \to \langle v,w \rangle$}.
\]
\end{enumerate}
\end{theorem}
%
\begin{proof}
Osserviamo innanzitutto che $\abs{v}$ è ben definita per ogni $v\in V$
in quanto il prodotto scalare $\langle v,v\rangle$ per definizione non è mai negativo.
Allora in generale si ha, sfruttando la bilinearità,
l'usuale sviluppo del quadrato del binomio:
\begin{align*}
  \abs{v+w}^2
  &= \langle v+w, v+w\rangle
  = \langle v+w, v\rangle + \langle v+w, w\rangle \\
  &= \langle v,v\rangle + 2\langle v,w \rangle + \langle w,w\rangle
  = \abs{v}^2 + 2\langle v,w \rangle + \abs{w}^2.
\end{align*}
Il teorema di Pitagora segue immediatamente.
Ma allora si ottiene facilmente la disuguaglianza di Young:
\[
  0 \le \abs{v-w}^2 = \abs{v}^2 + \abs{w}^2 - 2 \langle v,w\rangle
\]
e la proprietà del parallelogramma:
\[
  \abs{v+w}^2 + \abs{v-w}^2
  = \abs{v} + 2\langle v,w \rangle + \abs {w}
    + \abs{v} - 2\langle v,w\rangle + \abs{w}.
\]
Ora se $\abs{v}=\abs{w}=1$ la disuguaglianza di Young ci dice che
\[
  \langle v,w \rangle \le 1.
\]
Ma allora per ogni $v\neq 0$ e $w \neq 0$ si ottiene
la disuguaglianza di Cauchy-Schwarz:
\[
\frac{\langle v,w\rangle}{\abs{v}\cdot \abs{w}}
 = \langle \frac{v}{\abs{v}} , \frac{w}{\abs{v}}\rangle
 \le 1.
\]
Se invece $v=0$ o $w=0$ la disuguaglianza di Cauchy-Schwarz è ovvia in quanto
per ogni $u\in V$ si ha
$\langle 0, u\rangle = \langle u,0\rangle = 0$ per bilinearità.

Per quanto riguarda la continuità
ricordiamoci che $v_k \to v$ significa $\Abs{v_k-v}\to 0$.
Possiamo allora scrivere
\begin{align*}
  \langle v_k,w_k\rangle - \langle v,w\rangle
&= \langle v_k,w_k\rangle - \langle v, w_k\rangle +  \langle v,w_k\rangle - \langle v,w\rangle \\
&= \langle v_k - v, w_k\rangle + \langle v, w_k -w\rangle
\end{align*}
e, utilizzando Cauchy-Schwarz se $v_k\to v$ e $w_k\to w$ otteniamo
\begin{align*}
\abs{\langle v_k,w_k\rangle - \langle v,w\rangle}
&\le \abs{v_k-v}\cdot \abs{w_k} + \abs{v}\cdot \abs{w_k-w}\\
&\to 0\cdot \abs{w} + \abs{v}\cdot 0 = 0.
\end{align*}
\end{proof}

Lo spazio vettoriale $\RR^n$ ha una struttura euclidea canonica,
come mostrato nel seguente.



\begin{definition}[struttura euclidea di $\RR^n$]
\label{def:124124}
\mymark{*}
Un vettore $\vec x\in \RR^n$ è definito come una $n$-upla di numeri reali:
\[
  \vec x = (x_1, \dots, x_n).
\]
Su $\RR^n$ possiamo allora definire il prodotto scalare:
\index{prodotto scalare!di $\RR^n$}
\[
  \vec x\cdot \vec y = \sum_{k=1}^n x_k y_k.
\]
Questo prodotto scalare induce la norma euclidea:
\index{norma!euclidea di $\RR^n$}%
\[
  \abs{\vec x} = \sqrt{\vec x \cdot \vec x}
  = \sqrt{x_1^2 + x_2^2+ \dots + x_n^2}.
\]
La distanza indotta da tale norma si chiama \myemph{distanza euclidea}:
\index{distanza!euclidea di $\RR^n$}%
\[
  d(\vec x, \vec y)
  = \abs{\vec x-\vec y}
  = \sqrt{(x_1-y_2)^2 + \dots + (x_n-y_n)^2}.
\]

Nel caso $n=1$ la norma coincide con il valore assoluto e questo
giustifica l'aver utilizzato la stessa notazione.

Se identifichiamo $\CC$ con $\RR^2$ associando al numero complesso $z=x+iy$
il punto $(x,y)\in \RR^2$ possiamo osservare che la norma euclidea
coincide con il modulo del numero complesso:
\[
  \abs{z} = \sqrt{x^2 + y^2} = \abs{(x,y)}.
\]

Dunque $\RR$, $\CC$, $\RR^n$,
sono spazi euclidei, spazi normati e spazi metrici rispetto alla struttura
euclidea canonica.
\end{definition}

\begin{example}[distanza Manhattan]
Su $\RR^2$ possiamo definire una norma, chiamata \myemph{norma!Manhattan},
\index{Manhattan!norma}
come segue:
\[
  \phi(\vec x) = \abs{x_1} + \abs{x_2}.
\]
La distanza indotta $d(p,q)$
\index{distanza!Manhattan}%
\index{Manhattan!distanza}%
rappresenta la lunghezza del percorso più breve per
andare da $p$ a $q$ muovendosi solamente in orizzontale o verticale
(come se fossimo sulle strade di Manhattan).

La norma Manhattan non è euclidea nel senso che non è possibile definire
un prodotto scalare che induca tale norma. Infatti se esistesse un
tale prodotto scalare dovrebbe essere valida la proprietà del parallelogramma
\eqref{eq:parallelogramma} e invece osserviamo che scelto $v=(1,0)$ e $w=(0,1)$
si ha
\[
  \phi(v+w)^2 + \phi(v-w)^2
  = 8
  \neq 4
  = 2\phi(v)^2 + 2\phi(w)^2.
\]

L'insieme $B_R(\vec x_0) = \{\vec x \in \RR^2\colon \phi(\vec x-\vec x_0) < R\}$
dei punti di $\RR^2$ che distano meno di $R$ dal  punto $\vec x_0$
(si veda la definizione~\ref{def:palla})
è un quadrato
con le diagonali, di lunghezza $2R$, parallele agli assi coordinati.
Se come norma $\phi$ avessimo scelto la norma euclidea canonica di $\RR^2$
l'insieme $B_R(\vec x_0)$ sarebbe risultato essere un cerchio di raggio $R$
centrato in $\vec x_0$.
In generale le norme indotte da un prodotto scalare si riconoscono
dalla forma di questi insiemi: solo quando si ottengono ellissi
(o ellissoidi se siamo in dimensione più alta) la norma è euclidea.
Per le altre norme si potranno ottenere dei generici insiemi convessi simmetrici
rispetto al centro $\vec x_0$.
\end{example}

\begin{example}[norma $p$]
\label{ex:norma_p}
Per ogni $p\ge 1$ si può definire su $\RR^n$ la norma
\index{norma!$p$ in $\RR^n$}%
\[
  \abs{\vec x}_p = \sqrt[p]{\abs{x_1}^p + \abs{x_2}^p + \dots + \abs{x_n}^p}.
\]

Si può inoltre definire
\[
  \abs{\vec x}_\infty = \lim_{p\to +\infty} \abs{\vec x}_p
   = \max\{\abs{x_1}, \abs{x_2}, \dots, \abs{x_n}\}.
\]

Per $p=2$ si ottiene la norma euclidea della definizione~\ref{def:124124}
$\abs{v}_2=\abs{v}$.
Per $p=1$
su $\RR^2$ si ottiene la norma Manhattan.
Per $p=+\infty$ si ottiene di nuovo
la norma Manhattan a meno di una rotazione di 45 gradi e di un riscalamento
di fattore $\sqrt 2$:
\[
  \abs{(x_1,x_2)}_\infty
  = \abs{\enclose{\frac{x_1-x_2} 2, \frac {x_1+x_2}2}}_1
\]

Si potrebbe dimostrare che solo per $p=2$ la norma $\abs{\vec x}_p$ è indotta
da un prodotto scalare in quanto solo per $p=2$ è valida
la proprietà del parallelogramma~\eqref{eq:parallelogramma}.
\end{example}


\section{spazi metrici}
\index{spazio!metrico}

\begin{definition}[distanza indotta]
Se $d\colon X\times X \to \RR$ è una distanza e $A \subset X$
allora restringendo $d$ a $A \times A$ si ottiene ancora (ovviamente) una distanza.
Tale restrizione si chiama \emph{distanza indotta}
\mynote{distanza indotta}
\index{distanza!indotta}
da $X$ su $A$.
Dunque se $A$ è un sottoinsieme di uno spazio metrico $(X,d)$ anche $A$ ha una struttura di spazio metrico.
\end{definition}

\begin{example}[sfera]
Se $X\subset \RR^n$ la distanza euclidea di $\RR^n$ induce
su $X$ una struttura di spazio metrico. Se $X$ non è un sottospazio vettoriale di $\RR^n$ abbiamo quindi esempi di spazi metrici che non sono spazi normati. Ad esempio
la \myemph{sfera $n$-dimensionale}
\[
  \mathbb S^n = \{\vec x \in \RR^{n+1}\colon \abs{\vec x} = 1\}
\]
è uno spazio metrico con la distanza indotta da $\RR^n$.

Per $n=1$ si osserva che $\mathbb S^1$ è la circonferenza unitaria nel piano, per $n=2$ si ottiene l'usuale sfera unitaria nello immersa nello spazio tridimensionale.
\end{example}

\begin{definition}[palla]
\label{def:palla}%
\mymark{*}
Sia $(X,d)$ uno spazio metrico.
Per ogni $r>0$ e per ogni $x_0\in X$
definiamo la \myemph{palla} di raggio $r$ centrata in
$x_0$ come l'insieme
\[
  B_r(x_0) = \{x\in X \colon d(x,x_0) < r\}.
\]
\end{definition}

\begin{definition}[relazioni e proprietà topologiche]
\mymark{*}
\label{def:466342}
Sia $(X,d)$ uno spazio metrico.
Un insieme $A\subset X$ si dirà essere un insieme
\myemph{aperto} in $X$ se per ogni $x\in A$ esiste $r>0$
tale che $B_r(x) \subset A$.
Un insieme $A\subset X$ si dirà essere un insieme \myemph{chiuso} in $X$ se il suo complementare $X\setminus A$ è aperto.

La famiglia di tutti gli insiemi aperti si chiama \myemph{topologia} dello spazio metrico $X$. Tutte le definizioni che seguono non dipendono dalla distanza $d$ ma solamente dalla topologia: basterà usare aperti qualunque al posto delle palle $B_r(x)$.

Se $A\subset X$ è un insieme qualunque
$x\in X$ è un punto qualunque diremo che:
\begin{enumerate}
\item
$x$ è \myemph{punto!interno} ad $A$ se esiste $r>0$ tale che $B_r(x) \subset A$;
chiameremo
\myemph{parte!interna}
di $A$ l'insieme dei punti interni di $A$
e la denoteremo con $\stackrel\circ A$;
\item
$A$ è un \myemph{intorno} di $x$ se $x$ è punto interno ad $A$;
\item
$x$ è \myemph{punto!esterno} ad $A$ se è interno al complementare di $A$ ovvero esiste $r>0$ tale che $B_r(x) \cap A = \emptyset$;
chiameremo \myemph{parte!esterna} di $A$ l'insieme dei punti esterni ad $A$;
\item
$x$ è \myemph{punto!di frontiera} per $A$ se non è né interno né esterno ad $A$ ovvero per ogni $r>0$ l'insieme $B_r(x)$ contiene punti di $A$ e di $X\setminus A$;
chiameremo \myemph{frontiera} (o bordo) di $A$ l'insieme dei punti di frontiera che denoteremo con $\partial A$.
\item
$x$ è \myemph{punto!di aderenza} di $A$ se è interno o di frontiera ovvero se per ogni $r>0$ si ha $B_r(x) \cap A \neq \emptyset$;
chiameremo \emph{chiusura} di $A$ l'insieme dei punti di aderenza,
che denoteremo con $\bar A$;
\mymargin{chiusura}
\index{chiusura}
\item
$x$ è \myemph{punto!di accumulazione} di $A$ se
è punto di aderenza per $A \setminus \{x\}$ ovvero se
per ogni $r>0$ l'insieme $A \cap B_r(x)$ contiene punti diversi da $x$, chiameremo \myemph{derivato} di $A$ l'insieme dei punti di accumulazione (che si potrebbe denotare con $A'$);
\item
$x$ è \myemph{punto!isolato} di $A$ se è un punto di $A$ ma non di accumulazione per $A$ cioè se esiste $r>0$ per cui
$B_r(x) \cap A = \{x\}$.
\end{enumerate}
\end{definition}

\begin{theorem}[le palle sono aperte]
Sia $(X,d)$ uno spazio metrico, sia $x\in X$ e $r>0$. Allora la palla $B_r(x)$ è un insieme aperto in $X$.
\end{theorem}
%
\begin{proof}
Sia $y\in B_r(x)$: è sufficiente trovare $\rho>0$ tale che $B_\rho(y) \subset B_r(x)$. Prendendo $\rho = r-d(y,x)$ si osserva che $\rho >0 $ e, per la disuguaglianza triangolare,
dato $z \in B_\rho(y)$ si ha
\[
  d(z,x) \le d(z,y) + d(y,x) < \rho + d(y,x) = r
\]
da cui $B_\rho(y)\subset B_r(x)$ come volevamo dimostrare.
\end{proof}

\begin{comment}
\section{spazi topologici}

\begin{definition}[spazio topologico]
Sia $X$ un insieme e sia $\tau\subset \P(X)$ una famiglia di sottoinsiemi di $X$.
Diremo che $X$ è uno \myemph{spazio topologico}, che $\tau$ è una
\myemph{topologia} e che gli elementi di $\tau$ sono \myemph{aperti}
se:
\begin{enumerate}
\item $\emptyset \in \tau$, $X\in \tau$ (il vuoto e l'intero spazio sono aperti);
\item $A \subset \tau \implies \bigcup A \in \tau$ (unione qualunque di aperti è aperta);
\item $A,B\in \tau \implies A \cap B \in \tau$ (intersezione finita di aperti è aperta).
\end{enumerate}
Diremo inoltre che lo spazio topologico è \myemph{separato} (o $T_2$ o di Hausdorff)
\index{spazio topologico!separato}
\index{spazio topologico!$T_2$}
\index{spazio topologico!di Hausdorff}
\index{$T_2$}
\index{separato}
se inoltre vale
\begin{enumerate}
\item[4.] dati $x,y\in X$ con $x\neq y$ esistono $U,V\in \tau$ con $x\in U$,
$y\in V$ e $U\cap V=\emptyset$.
\end{enumerate}

Diremo che un sottoinsieme $A\subset X$ è \myemph{chiuso} se il complementare
è aperto ($X\setminus A\in \tau$).
Diremo che $A\subset X$ è un \myemph{intorno} del punto $a\in X$ se esiste $U\in \tau$
tale che $x\in U \subset A$.

Tutte le proprietà date nella definizione~\ref{def:466342} possono quindi essere
date in uno spazio topologico utilizzando gli intorni (o gli aperti) al posto
delle palle.
\end{definition}

\begin{theorem}[topologia indotta dalla metrica]
Sia $X$ uno spazio metrico e sia $\tau$ l'insieme di tutti gli aperti
di $X$ ovvero:
\[
 \tau = \{A \subset X\colon \forall a \in A \exists \rho>0 \colon B_\rho(a)\subset A\}.
\]
Allora $\tau$ è una topologia e $X$ è uno spazio topologico separato rispetto alla topologia
$\tau$.
\end{theorem}

\begin{definition}[base di intorni]
Sia $X$ un insieme e per ogni $x\in X$ sia data $\B_x\subset \P(X)$
una famiglia di sottoinsiemi di $X$. Diremo che $\B_x$ è una
\myemph{base di intorni} di $x$ se valgono le seguenti proprietà:
\begin{enumerate}
\item $\B_x$ è non vuoto;
\item se $B\in \B_x$ allora $x \in B$;
\item se $A,B\in \B_x$ allora esiste $C\in \B_x$ tale che $C\subset A\cap B$;

\item Se un insieme contiene un intorno di un punto anch'esso è un intorno di quel punto.
\item L'intersezione di due intorni di un punto è un intorno del punto.
\end{enumerate}

Se per ogni $x\in X$ l'insieme $\B_x$ è una base di intorni di $x$ potremo
definire
\[
  \tau = \{A \subset X\colon \forall x\in A \exists B \in \B_x \colon B\subset A\}
\]
e risulta che $\tau$ è una topologia e quindi $X$ è uno spazio topologico
rispetto a tale topologia (indotta dalla base di intorni).

Se inoltre vale anche la seguente proprietà:
\begin{enumerate}
\item dati $x,y\in X$, $x\neq y$, esistono $B\in \B_x$ e $C\in \B_y$ tali che
$B\cap C=\emptyset$;
\end{enumerate}
allora $X$ sarà uno spazio topologico separato.
\end{definition}
%


Le proprietà seguenti sono conseguenza degli assiomi precedenti e sono quindi valide in ogni spazio topologico.

\begin{enumerate}
\item
La parte interna di un qualunque $A\subset X$ è il più grande (ovvero l'unione di ogni) aperto contenuto in $A$. La chiusura di $A$ è il più piccolo (ovvero l'intersezione di ogni) chiuso contenente $A$.
In particolare la parte interna è sempre aperta e la parte interna di un aperto è tutto l'insieme. La chiusura è un insieme chiuso e la chiusura di un chiuso è l'insieme stesso.
\item
La frontiera di un insieme è chiusa. Parte interna, frontiera e parte esterna sono tre insiemi disgiunti (rispettivamente aperto, chiuso e aperto) la cui unione è tutto lo spazio.
\end{enumerate}

\begin{theorem}
Se $X$ è uno spazio metrico la famiglia di insiemi
\[
  \tau = \{A \subset X \colon \forall a \in A \exists \rho>0 \colon B_\rho(x) \subset A\}
\]
risulta essere una topologia su $X$ che si chiama \emph{topologia indotta}
dalla metrica di $X$.
\end{theorem}
%
\begin{proof}
L'insieme vuoto è aperto in quanto non ci sono punti su cui è necessario verificare la proprietà che definisce gli aperti. Anche l'intero spazio è aperto in quanto ogni palla è contenuta in esso. Dunque $\emptyset, X$ sono aperti e di conseguenzi i rispettivi complementari: $X, \emptyset$ sono chiusi.

Che l'unione qualunque di aperti sia aperta è ovvio: preso un punto dell'unione tale punto è contenuto in uno degli aperti. Dunque c'è una palletta centrata nel punto e contenuta nell'aperto. Ma allora essa è anche contenuta nell'unione.
Che l'intersezione di chiusi sia un chiuso si ottiene passando ai complementari: il complementare di un chiuso è un aperto e il complementare dell'intersezione è l'unione dei complementari.

L'intersezione qualunque di chiusi è uguale al complementare (rispetto ad $X$) dell'unione dei complementari. Per definizione il complementare di un chiuso è aperta e dunque l'unione dei complementari è aperta. Dunque il suo complementare è un chiuso.

Consideriamo ora l'intersezione $A\cap B$ di due aperti $A$ e $B$. Se $x$ è un punto dell'intersezione sappiamo che esistono due palle $B_r(x)$ e $B_s(x)$ tali che $B_r(x)\subset A$ e $B_s(x) \subset B$. La più piccola delle due è contenuta in $A\cap B$ e questo dimostra che $A\cap B$ è aperto. Passando ai complementari si ottiene che l'unione di due chiusi è un chiuso.

Dati $x,y \in X$ se $x\neq y$ allora $r=d(x,y)/2>0$. Osserviamo allora che $B_r(x)$ e $B_r(y)$ sono disgiunte in quanto se se esistesse $z\in B_r(x) \cap B_r(y)$ si avrebbe $d(x,y)\le d(x,z) + d(z,y) < r + r = d(x,y)$ che è assurdo.

Abbiamo quindi dimostrato le quattro proprietà degli aperti. Passiamo alle proprietà degli intorni. Per definizione un insieme $U$ è un intorno di $x$ se esiste un aperto $A$ tale che $x\in A \subset U$. Necessariamente quindi $x\in U$. E se $V\supset U$ allora $x\in A \subset V$ e dunque anche $V$ è intorno di $x$. Se $U$ e $V$ sono intorni dovranno esistere due aperti $A$ e $B$ tali che $x\in A \subset U$ e $x \in B \subset V$. Ma allora $A\cap B$ è aperto e $x \in A \cap B \subset U \cap V$ dunque $U\cap V$ è anch'esso un intorno di $x$. Per il quarto punto basta osservare che ogni intorno contiene, per definizione, un aperto che contiene il punto e un aperto è, sempre per definizione, intorno di ogni suo punto. Per il punto 5 già sappiamo che due punti distinti sono contenuti in aperti disgiunti, e gli aperti sono sempre intorni di ogni loro punto.

Dimostriamo ora che la parte interna di un insieme $A$ è l'unione di tutti gli aperti contenuti in $A$. Da un lato se $x$ è un punto interno ad $A$ allora esiste $r>0$ tale che $B_r(x)\subset A$. Essendo $B_r(x)$ aperto risulta quindi che $x$ sta nell'unione degli aperti contenuti in $A$. Viceversa se $x$ sta nell'unione di tutti gli aperti contenuti in $A$ deve esistere un aperto $U$ tale che $x\in U \subset A$. Ma allora esiste $r>0$ tale che $B_r(x) \subset U \subset A$ e dunque $x$ è punto interno ad $A$. La proprietà analoga per i chiusi si ottiene passando ai complementari.

La frontiera di un insieme è, per definizione, l'insieme dei punti che non sono né interni né esterni. D'altra parte è chiaro che un punto non può essere contemporaneamente interno ed esterno. Dunque la frontiera è il complementare dell'unione della parte interna e della parte esterna, ed è quindi un insieme chiuso.
\end{proof}
\end{comment}

\section{continuità}

\begin{definition}[convergenza in uno spazio metrico]
\mymark{**}
Sia $(X,d)$ uno spazio metrico.
Diremo che una successione di punti $x_n \in X$
\mynote{convergenza}
\index{convergenza!in uno spazio metrico}
\index{spazio!metrico!convergenza}
converge ad un punto $x\in X$
e scriveremo $x_n\to x$ per $n\to +\infty$ se
\[
  d(x_n, x) \to 0 \qquad \text{per $n \to +\infty$}.
\]

Un insieme $A\subset X$ si dice essere
\myemph{sequenzialmente chiuso}
\index{spazio!metrico!sequenzialmente chiuso}
se presa una successione $x_k\in A$ se $x_k\to x$ converge ad un punto $x\in X$ allora $x\in A$.
\end{definition}

Si noti che l'usuale convergenza in $\RR$ non è altro
che la convergenza di $\RR$
visto come spazio metrico con la distanza euclidea $d(x,y) = \abs{x-y}$.

\begin{theorem}[chiusura e chiusura sequenziale]
Sia $(X,d)$ uno spazio metrico.
Un insieme $A\subset X$ è chiuso se e solo se è sequenzialmente chiuso.
\end{theorem}
%
\begin{proof}
Per dimostrare che un insieme sequenzialmente chiuso è chiuso dimostriamo l'equivalente contropositiva: se $A$ non è chiuso allora $A$ non è sequenzialmente chiuso. Sia dunque $A\subset X$ un insieme non chiuso. Significa che c'è un punto $y \in X\setminus A$ che non è esterno ad $A$.
Ciò vuol dire che per ogni $r>0$ l'insieme $B_r(y)\cap A$ è non vuoto. Per ogni $k\in \NN$ posso allora
scegliere $r=1/k$ e quindi so che esiste un punto $x_k\in B_r(y) \cap A$ ovvero $x_k \in A$ e $d(x_k,y) < 1/k$. Dunque $x_k \to y$ con $x_k\in A$ ma $y\not \in A$: significa che $A$ non è sequenzialmente chiuso.

Viceversa supponiamo che $A$ sia chiuso e dimostriamo che
allora è anche sequenzialmente chiuso.
Sia allora $x_k \in A$ una successione convergente ad un punto di $X$: $x_k \to x$.
Dobbiamo mostrare che $x\in A$.
Per definizione di convergenza
sappiamo che per ogni $\eps>0$ esiste $K\in \NN$ tale per ogni $k> K$ si ha $x_k \in B_\eps(x)$.
In particolare $A \cap B_\eps(x) \neq \emptyset$.
Risulta quindi che $x$ non è esterno ad $A$ e quindi, essendo $A$ chiuso, $x\in A$.
\end{proof}

\begin{definition}[continuità]
\mymark{**}
Una funzione $f\colon X \to Y$ definita tra due spazi metrici
si dice essere \myemph{sequenzialmente continua} nel punto $x\in X$ se
per ogni successione convergente $x_n \to x$ in risulta che $f(x_n)\to f(x)$.
Diremo che $f$ è \emph{sequenzialmente continua} se è sequenzialmente continua
in ogni punto $x$ del suo dominio $X$.

Una funzione $f\colon X \to Y$ definita tra due spazi metrici
si dice essere \myemph{continua} in un punto $x\in X$ se
\begin{equation}\label{eq:continuita_metrico}
 \forall \eps>0\colon \exists \delta>0\colon \forall y \in X \colon
 d(x,y)< \delta \implies d(f(x), f(y)) < \eps.
\end{equation}
Diremo che $f$ è \emph{continua} se
è continua in ogni punto $x$ del suo dominio.
\end{definition}

Anche in questo caso abbiamo considerato due diverse nozioni di continuità che
in generale (in spazi topologici) potrebbero non coincidere ma nel caso degli
spazi metrici sono equivalenti, come dimostriamo nel seguente teorema.

\begin{theorem}[definizioni equivalenti di continuità]
Sia $f\colon X \to Y$ una funzione definita tra due spazi metrici $X$ e $Y$.
Allora $f$ è sequenzialmente continua in un punto $x\in X$ se e solo se è
continua nel punto $x$.

Inoltre $f$ è continua se e solo se
per ogni $A$ aperto in $Y$ risulta che $f^{-1}(A)$ è aperto in $X$
(la controimmagine di un aperto è aperta).
\end{theorem}
%
\begin{proof}
Supponiamo che $f$ sia sequenzialmente continua in $x$
e, per assurdo, supponiamo che $f$ non sia continua nello stesso punto $x$.
Allora negando~\eqref{eq:continuita_metrico} otteniamo che esiste $\eps>0$
tale che per ogni $\delta>0$ in particolare per ogni $\delta = \frac{1}{k}$
con $k\in \NN$ esiste un punto $y_k$ tale che $d(y_k,x)<\frac 1 k$ ma
$d(f(y_k),f(x))\ge \eps$. Significa che $y_k \to x$ e quindi,
se $f$ è sequenzialmente continua, dovrebbe essere $f(y_k)\to f(x)$. Ma
questo è in contraddizione con la condizione $d(f(y_k),f(x))\ge \eps$.

Viceversa supponiamo che $f$ sia continua in $x$ e consideriamo una qualunque
successione $x_k \to x$.
Per dimostrare che $f(x_k)\to f(x)$ dobbiamo verificare che per ogni
$\eps>0$ esiste $K\in \NN$ tale che per ogni $k>K$ si ha $d(f(x_k),f(x)) < \eps$.
Ma dalla continuità di $f$, dato $\eps>0$ sappiamo esistere $\delta>0$ per cui
se $d(y,x)<\delta$ allora $d(f(y),f(x))<\eps$. Visto che $x_k\to x$
certamente esiste $K\in \NN$ tale che per ogni $k>K$ si ha $\abs{x_k-k}<\delta$
e quindi $\abs{f(x_k)-f(x)}< \eps$ come volevamo dimostrare.

Mostriamo ora che se $f$ è continua e $A$ è aperto in $Y$ allora $f^{-1}(A)$
è aperto in $X$.
Dato un punto qualunque $x_0\in f^{-1}(A)$ sappiamo che $f(x_0)\in A$ dunque, essendo
$A$ aperto, esiste $\eps>0$ tale che $B_\eps(f(x_0))\subset A$.
Per la continuità di $f$ esiste allora $\delta>0$ tale che se $\abs{x-x_0} < \delta$
allora $\abs{f(x)-f(x_0)}<\eps$ e quindi $f(x)\in A$. Significa che $B_\delta(x_0)\subset f^{-1}(A)$.
Questo è vero per ogni $x_0\in f^{-1}(A)$ quindi tale insieme è aperto.

Viceversa supponiamo che la controimmagine di ogni aperto sia un aperto e
dimostriamo che la funzione è continua in ogni punto.
Preso un punto $x_0\in X$ e un $\eps>0$ consideriamo l'aperto $B_\eps(f(x_0))$.
La sua controimmagine è l'insieme $\{x\colon \abs{f(x)-f(x_0)} < \eps\}$ e
per ipotesi sappiamo che è aperto. Significa allora che esiste $\delta>0$
tale $B_\delta(x_0)$ è contenuto in tale insieme, ovvero per ogni $x\in B_\delta(x_0)$
cioè $\abs{x-x_0}<\delta$
risulta $f(x) \in B_\eps(f(x_0))$
cioè $\abs{f(x)-f(x_0)}<\eps$.
Abbiamo quindi verificato la definizione di
continuità nel punto $x_0$.
\end{proof}



\begin{definition}[spazi limitati]
\mymark{*}
Sia $X$ uno spazio metrico o un sottoinsieme di uno sottospazio metrico. Si dirà che $X$ è
\myemph{limitato} se è contenuto in una palla ovvero se
esiste $x_0\in X$ e $R>0$ tale che $X\subset B_R(x_0)$.
\end{definition}

\begin{definition}[compattezza sequenziale]
\mymark{**}
Sia $X$ uno spazio metrico o un sottoinsieme di uno
spazio metrico. Si dirà che $X$ è
\myemph{sequenzialmente compatto} se da ogni
successione $x_k \in X$ è possibile estrarre una sottosuccessione $x_{k_j}\to x$
convergente ad un punto $x\in X$.
\end{definition}

La compattezza viene definita in forma più generale negli spazi topologici
(argomento che non vogliamo trattare). Sugli spazi metrici si osserva poi che
la compattezza è equivalente alla compattezza sequenziale. Dunque ci capiterà
di scrivere più brevemente \myemph{compatto}
al posto di \emph{sequenzialmente compatto} e anche se formalmente la
definizione di compatto è diversa (e non l'abbiamo introdotta) gli enunciati
rimangono comunque validi.

Il teorema di Bolzano-Weierstrass afferma che gli intervalli $[a,b]$ con
$a,b\in \RR$ sono compatti.
Più in generale si può dimostrare che tutti gli insiemi chiusi e limitati
di $\RR^n$ sono compatti. In generale questo risultato non è vero in qualunque
spazio metrico (un esempio negativo è dato dalla convergenza uniforme, come
vedremo più avanti) ma l'implicazione inversa è sempre vera, come enunciato nel
seguente teorema.

\begin{theorem}
\mymark{**}
Se $A$ è un sottoinsieme sequenzialmente
compatto di uno spazio metrico $X$
allora $A$ è chiuso e limitato.
\end{theorem}
%
\begin{proof}
Chiaramente $A$ è chiuso in quanto presa una successione $x_k\in A$ convergente a punto $x\in X$
sappiamo che esiste una sottosuccessione convergente ad un punto di $A$. Ma necessariamente ogni sottosuccessione converge ad $x$ quindi $x\in A$. Se $A$ non fosse limitato
fissato $a\in A$ per ogni $k\in \NN$ dovrebbe esistere un punto $x_k\in A$ tale che $x_k \not\in B_k(a)$
cioè $d(x_k,a) > k$. Supponiamo allora che esista una sottosuccessione convergente $x_{k_j}\to x \in A$. Allora per la disuguaglianza triangolare inversa si avrebbe
\[
  d(x, a) \ge d(x_{k_j}, a) - d(x_{k_j},x)
   \ge k_j - d(x_{k_j},x) \to +\infty - 0 = +\infty.
\]
Ma questo è assurdo in quanto $d(x,a)\in \RR$.
\end{proof}

\begin{theorem}[Weierstrass: le funzioni continue mandano compatti in compatti]
Sia $f\colon X \to Y$ una funzione continua tra
due spazi metrici $X$ e $Y$.
Se $K\subset X$ è sequenzialmente compatto allora
anche $f(K)$ è sequenzialmente compatto.
\end{theorem}
%
\begin{proof}
Sia $y_k \in f(K)$ una qualunque successione. Allora
esiste $x_k \in K$ tale che $f(x_k) = y_k$.
Essendo $K$ compatto possiamo estrarre una sottosuccessione convergente: $x_{k_j}\to x$. Essendo $f$ continua si ha
\[
  y_{k_j} = f(x_{k_j}) \to f(x) \in f(K).
\]
\end{proof}

Nel caso $X=Y=\RR$ recuperiamo l'usuale teorema di Weierstrass, in quanto se $f\colon [a,b]\to \RR$ è continua essendo $[a,b]$ compatto risulta che $f([a,b])$ è compatto. Ma i compatti di $\RR$ sono chiusi e limitati quindi hanno massimo e minimo in quanto l'estremo superiore e l'estremo inferiore sono finiti e sono punti di aderenza dell'insieme.

\section{completezza}

\begin{definition}[successioni di Cauchy]
\mymark{***}
\index{successione!di Cauchy}
\index{Cauchy!successione di}
Sia $(X,d)$ uno spazio metrico e $x_k$ una successione di punti di $X$.
Diremo che $x_k$ è una
\emph{successione di Cauchy}
\mynote{successione di Cauchy}
se
\[
 \forall \eps>0\colon \exists n\in \NN\colon \forall j>n \colon \forall k > n \colon d(x_j,x_k) < \eps.
\]
\end{definition}

La proprietà che definisce le successioni di Cauchy
potrebbe essere scritta con la notazione
\[
  \lim_{j,k \to +\infty} d(x_j, x_k) = 0.
\]

\begin{theorem}[le successioni convergenti sono di Cauchy]
\mymark{**}
Sia $x_k\to x$ una successione convergente in uno spazio metrico $(X,d)$. Allora $x_k$ è di Cauchy.
\end{theorem}
%
\begin{proof}
\mymark{**}
Per definizione se $x_k \to x$ si ha
\[
  \forall \eps>0\colon \exists n\in \NN \colon
  \forall k>n \colon d(x_k,x)< \eps.
\]
Applicando la disuguaglianza triangolare, per ogni $j,k>n$
si ottiene il risultato desiderato:
\[
  d(x_j, x_k) \le d(x_k,x) + d(x,x_j) \le 2\eps.
\]
\end{proof}

\begin{definition}[completezza]
\mymark{***}
\mynote{completezza}
\index{completezza}
Uno spazio metrico $(X,d)$ si dice essere \myemph{completo}
se ogni successione di Cauchy è convergente.
\end{definition}

\begin{theorem}[completezza dei compatti]
Ogni spazio metrico compatto è completo.
\end{theorem}
%
\begin{proof}
Visto che lo spazio è compatto ogni successione di Cauchy
ammette una sottosuccessione convergente. Ma se una sottosuccessione di una successione di Cauchy è convergente ad un punto, è facile osservare che l'intera successione di Cauchy converge a quel punto.
\end{proof}

\begin{theorem}[chiusi in spazi compatti e in spazi completi]
Sia $(X,d)$ uno spazio metrico e sia $A\subset X$ un sottoinsieme chiuso in $X$. Se $X$ è compatto allora anche $A$ è compatto, se $X$ è completo allora anche $A$ è completo.
\end{theorem}
\begin{proof}
Se $X$ è compatto da ogni successione in $A$ si può estrarre una sottosuccessione convergente ad un punto di $X$. Ma siccome $A$ è chiuso il punto sta in $A$ e dunque la sottosuccessione è convergente in $A$.

Una successione di Cauchy in $A$ è di Cauchy anche in $X$. Se $X$ è completo tale successione converge ad un punto di $x$. Se $A$ è chiuso tale punto è in $A$ e dunque la successione converge in $A$.-
\end{proof}

\begin{definition}[spazio di Banach]
Uno spazio vettoriale normato si dice essere uno
\emph{spazio di Banach}%
\mynote{spazio di Banach}%
\index{spazio!di Banach}%
\index{Banach!spazio di}
se, come spazio metrico, risulta essere completo.
Se la norma è euclidea, cioè deriva da un prodotto scalare, lo spazio si dirà
\emph{spazio di Hilbert}.
\mynote{spazio di Hilbert}%
\index{spazio!di Hilbert}
\end{definition}

\begin{lemma}
\label{lm:cauchy_limitata}
Ogni successione di Cauchy è limitata
(più precisamente: se $x_n$ è una successione di Cauchy in uno spazio metrico $X$ allora l'insieme $\{x_n\colon n\in \NN\}$
è un insieme limitato).
\end{lemma}
%
\begin{proof}
Sia $x_k \in \RR$ una successione di Cauchy.
Fissato $\eps =1$ sappiamo che esiste $N\in \NN$
per cui per ogni $k,j>N$ si ha
$d(x_k,x_j) < 1$. In particolare per ogni $k>N$ si ha
\[
  d(x_k, x_{N+1}) < 1.
\]
Dunque posto
\[
  R = \max\{d(x_0,x_1), d(x_0,x_2), \dots, d(x_0,x_N), d(x_0, x_{N+1}) + 1\}
\]
si osserva che per ogni $k\in \NN$ si ha $d(x_0, x_k)\le R < R+1$ in quanto se $k \le N$ abbiamo scelto appositamente $R$ in modo che sia più grande di $d(x_0,x_k)$ e se $k > N$ allora
\[
  d(x_0,x_k) \le d(x_0,x_{N+1}) + d(x_{N+1},x_k)
    \le d(x_0,x_{N+1}) + 1 \le R.
\]

Significa quindi che per ogni $x\in \NN$ si ha $x_k in B_{R+1}(x_0)$ che è la definizione di limitatezza in uno
spazio metrico.
\end{proof}

\begin{lemma}
\label{lm:cauchy_estratta_convergente}
Se una successione di Cauchy ha una sottosuccessione convergente, allora l'intera successione è convergente.
\end{lemma}
%
\begin{proof}
Sia $x_k$ la successione di Cauchy e sia $x_{k_j}\to x$ una  sottosuccessione convergente.
Allora per ogni $\eps>0$
esiste $m$ tale che se $k,j>m$ allora $d(x_k ,x_j) < \eps$.
Visto che $x_{k_j} \to x$ possiamo trovare $j$ tale che $k_j > m$ e tale che $d(x_{k_j},x) < \eps$. Ma allora
\[
  d(x_k,x) \le d(x_k, x_{k_j}) + d(x_{k_j},x)
   \le 2 \eps.
\]
E questo è vero per ogni $k > m$ da cui risulta verificata la definizione di limite $x_k \to x$.
\end{proof}

\begin{theorem}[completezza di $\RR$]
\mymark{***}
\mynote{$\RR$ è completo}
\index{completezza!di $\RR$}
$\RR$ è completo.
\end{theorem}
%
\begin{proof}
\mymark{***}
Dobbiamo dimostrare che se $x_k$ è una successione di Cauchy in $\RR$ allora $x_k$ converge.
Per il lemma~\ref{lm:cauchy_limitata} sappiamo che $x_k$ è limitata.
Ma allora per il teorema di Bolzano-Weierstrass sappiamo che $x_k$ ha
una estratta convergente.
Grazie al lemma~\ref{lm:cauchy_estratta_convergente} possiamo quindi concludere
che la successione $x_k$ è essa stessa convergente.
\end{proof}

\begin{corollary}[completezza di $\RR^n$ e $\CC$]
Gli spazi $\RR^n$ e $\CC$ (con la usuale distanza euclidea)
sono completi.
\end{corollary}
%
\begin{proof}
Basta osservare che la convergenza (o la condizione di Cauchy) di una successione in $\RR^n$ si ha se e solo se ogni componente è convergente (o di Cauchy) in $\RR$. Dunque essendo $\RR$ completo anche $\RR^n$ lo è. Come spazio metrico $\CC$ è isomorfo ad $\RR^2$ dunque anch'esso è completo.
\end{proof}


\begin{definition}[lipschitz]
\mymark{***}
Sia $f\colon X \to Y$ una funzione definita tra due spazi metrici e sia $L\ge 0$.
Diremo che $f$ è $L$-lipschitziana se
\index{funzione!lipschitziana}
per ogni $x,y \in X$ si ha
\[
  d(f(x),f(y)) \le L d(x,y).
\]
Diremo che $f$ è lipschitziana se esiste $L\ge 0$ tale che $f$ sia $L$-lipschitziana.
\end{definition}

\begin{theorem}
\mymark{*}
Se $f\colon X \to Y$ è lipschitziana allora
$f$ è sequenzialmente continua, cioè
\[
  x_k \to x \implies f(x_k)\to f(x).
\]
\end{theorem}
%
\begin{proof}
Se $x_k\to x$ significa che $d(x_k,x) \to 0$, quindi
\[
  d(f(x_k), f(x)) \le L \cdot d(x_k,x) \to 0.
\]
\end{proof}

Osserviamo che la distanza $d(x,y)$ di uno spazio metrico $X$ risulta sempre essere una funzione $1$-lip\-schit\-zia\-na rispetto ad ognuna delle due variabili $x$ e $y$. Infatti per la disuguaglianza triangolare inversa si ha
\[
  \abs{d(x_1,y) - d(x_2,y)} \le d(x_1, x_2).
\]
Di conseguenza la norma di uno spazio normato è anch'essa $1$-lip\-schit\-zia\-na. In particolare la distanza e la norma risultano essere funzioni continue.

\begin{theorem}[delle contrazioni o punto fisso di Banach-Caccioppoli]
\mymark{***}
\mynote{teor. contrazioni}
\index{teorema!di Banach-Caccioppoli}
\index{teorema!delle contrazioni}
\index{punto!fisso}
\index{contrazione}
Sia $X$ uno spazio metrico completo non vuoto e sia $f\colon X \to X$ una funzione $L$-lipschitziana con $L<1$ (diremo che $f$ è una \emph{contrazione}).
Allora esiste ed è unico un punto
$x\in X$ tale che $f(x) = x$.
\end{theorem}
%
\begin{proof}
\mymark{***}
Si consideri un qualunque punto $p \in X$ e si definisca
la successione $x_k\in X$ tramite la definizione ricorsiva
\[
\begin{cases}
  x_0 = p \\
  x_{k+1} = f(x_k).
\end{cases}
\]
Visto che $f$ è $L$-lipschitziana si avrà
\begin{align*}
  d(x_2, x_1) &= d(f(x_1),f(x_0)) \le L \cdot d(x_1,x_0) \\
  d(x_3, x_2) &= d(f(x_2),f(x_1)) \le L \cdot d(x_2,x_1)
  \le L^2 \cdot d(x_1, x_0) \\
  d(x_4, x_3) &= d(f(x_3),f(x_2)) \le L \cdot d(x_3,x_2)
  \le L^3 \cdot d(x_1, x_0) \\
  &\vdots
\end{align*}
possiamo quindi dimostrare induttivamente che
per ogni $m\in \NN$ si ha
\[
  d(x_{m+1}, x_m) \le L^m \cdot d(x_1, x_0).
\]
Ma allora per ogni $k\in \NN$ e per ogni $j>k$
utilizzando la disuguaglianza triangolare e facendo la somma della progressione geometrica
si ha
\[
  d(x_k,x_j) \le \sum_{m=k}^{j-1} d(x_m, x_{m+1})
   \le \sum_{m=k}^{j-1} L^m \cdot d(x_1, x_0)
   = \frac{L^k-L^j}{1-L} d(x_1,x_0).
\]
Visto che $L<1$ se $k\to +\infty$ e $j>k$ questa quantità tende a zero e quindi
risulta che $x_k$ è una successione di Cauchy. Essendo per ipotesi $X$ completo sappiamo che la successione converge $x_k \to x$ ad un punto $x\in X$.
Per la continuità di $f$, passando al limite nell'equazione
$x_{k+1} = f(x_k)$ si ottiene $x = f(x)$.
Abbiamo quindi trovato un punto fisso.
Se $y\in X$ fosse un altro punto fisso si avrebbe:
\[
  d(x,y) = d(f(x),f(y)) \le L \cdot d(x,y)
\]
che è assurdo se $L<1$ e $x\neq y$.
\end{proof}
