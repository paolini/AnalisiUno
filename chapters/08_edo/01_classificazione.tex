\section{classificazione}

Le equazioni differenziali sono una classe di \emph{equazioni funzionali}
\mymargin{equazioni funzionali}
\index{equazione!funzionale}
ovvero
equazioni in cui l'incognita non è un numero (come accade nelle equazioni algebriche) ma è una funzione. La funzione incognita $u$, sarà quindi funzione di una variabile indipendente $u=u(x)$.
Ad esempio $u$ potrebbe essere la traiettoria di un proiettile che descrive la posizione nello spazio in funzione del tempo $x$. Se nelle equazioni algebriche il nome di gran lunga più utilizzato per l'incognita è $x$, nelle equazioni funzionali a seconda dei contesti le convenzioni possono cambiare in maniera drastica. Si dovrà utilizzare un nome per la funzione e un nome per la sua variabile indipendente: $x=x(t)$, $y=y(x)$, $y=y(t)$, $u=u(x)$ sono alcune delle scelte più utilizzate.
Se l'incognita è una funzione $u=u(x)$ le operazioni che possono comparire nell'equazioni sono,
oltre le usuali operazioni algebriche che agiscono sui singoli valori $u(x)$ della funzione,
anche operatori che agiscono sulla funzione $u$ in sé.
Se l'equazione funzionale oltre alle operazioni algebriche comprende anche
l'operatore derivata, si dirà che è una \mymargin{equazione differenziale}%
\index{equazione!differenziale}%
\emph{equazione differenziale}.
Di contro ci potranno ad esempio essere equazioni che coinvolgono l'operatore integrale e si chiameranno
\emph{equazioni integrali}.

Ci sono quindi diversi concetti che possono aiutare a classificare le equazioni differenziali.
Innanzitutto supporremo sempre che le nostre equazioni siano \emph{senza ritardo}%
\mymargin{senza ritardo}\index{senza ritardo}
(o \emph{senza memoria})
cioè che la funzione $u$ e le sue derivate $u'$, $u''$\dots,
vengano tutte calcolate nello stesso punto $x$.
In caso contrario si parla di
equazioni \emph{differenziali funzionali} (in particolare \emph{equazioni con ritardo}
perché le derivate dipendono tipicamente dai valori assunti nel passato),
argomento che non toccheremo.

Se la funzione incognita è funzione di una singola variabile si dirà che l'equazione è una
\emph{equazione differenziale ordinaria}
\mymargin{ODE}%
\index{equazione!differenziale!ordinaria}%
(abbreviato
\emph{EDO} \index{EDO}%
in italiano,
\emph{ODE} \index{ODE}%
per gli anglosassoni).
Di contro se la funzione incognita è funzione di più variabili l'equazione si chiamerà
\emph{equazione differenziale alle derivate parziali}
\mymargin{PDE}%
\index{equazione!differenziale!alle derivate parziali}%
 (abbreviato \emph{EDP} \index{EDP} in italiano e \emph{PDE}
 \index{PDE} in lingua inglese).
In questo corso, centrato sulle funzioni di una variabile, tratteremo quindi solamente le equazioni differenziali ordinarie.
L'\emph{ordine}
\mymargin{ordine}%
\index{ordine!equazione differenziale}%
\index{equazione!differenziale!ordine}%
dell'equazione differenziale è il numero massimo di derivate successive che vengono applicate alla funzione incognita.
La funzione $u$ sarà definita su un intervallo $I$ della retta reale: $u\colon I \to \RR$
e la funzione e le derivate vengono tutte valutate nello stesso punto. In tal caso
la forma più generale di equazione differenziale ordinaria di ordine $n$ si potrà dunque scrivere come:
\begin{equation}\label{eq:3784643}
  F(x,u(x),u'(x),u''(x), \dots, u^{(n)}(x)) = 0
\end{equation}
con $F\colon \Omega \to \RR$ una funzione data, definita su un insieme $\Omega \subset I\times \RR^{n+1}$.
Una funzione $u\colon I \to \RR$ si dice essere una \emph{soluzione
dell'equazione differenziale} \eqref{eq:3784643} se $u$ è derivabile almeno
$n$ volte in ogni punto $x\in I$ e se \eqref{eq:3784643} è soddisfatta
per ogni $x\in I$.

Usualmente si tende a semplificare la notazione evitando di scrivere sempre
esplicitamente il punto $x$ in cui viene calcolata la funzione.
Sarà quindi usuale
scrivere l'equazione \eqref{eq:3784643}
nella forma abbreviata:
\[
  F(x,u,u',u'', \dots, u^{(n)}) = 0
\]
rendendo anche più evidente il fatto che l'incognita è $u$,
l'intera funzione, e non un singolo valore $u(x)$.


Se ad esempio scegliamo $n=2$ e $F(x,u,v,z)= z+\sin u + v$ otteniamo l'equazione differenziale:
\begin{equation}\label{eq:39872}
 u''(x) + \sin u(x) + u'(x) = 0
\end{equation}
che, in opportune unità di misura, è l'equazione del moto di un pendolo smorzato,
dove $x$ rappresenta il tempo e $u$ la misura dell'angolo di inclinazione del
pendolo rispetto alla verticale.
Osserviamo che nell'esempio precedente la funzione $F$ non dipende direttamente
dalla variabile $x$. Equazioni con questa proprietà si dicono
\emph{equazioni autonome}
\mymargin{equazioni autonome}
\index{equazione!differenziale!autonoma}
ed è immediato osservare che se $u(x)$ è soluzione anche una sua traslazione
temporale $v(x) = u(x-x_0)$ è soluzione dell'equazione
(il moto del pendolo non dipende dall'ora in cui si svolge).

Una equazione scritta nella forma \eqref{eq:3784643} si dice \emph{equazione in forma implicita}
\mymargin{equazione in forma implicita}%
\index{equazione!differenziale!in forma implicita}%
e per analogia con le equazioni algebriche (si pensi all'equazione
$u^2(x) + x^2 = 1$) ci si aspetta che le soluzioni di tale equazioni siano
meglio rappresentate da curve piuttosto che da grafici di funzione.
Risulta in effetti che la teoria delle equazioni differenziali si applica con
molta maggiore efficacia alle \emph{equazioni in forma normale}
\mymargin{equazioni in forma normale}
\index{equazione!differenziale!in forma normale}
che sono le equazioni differenziali di ordine $n$ che possono essere scritte
esplicitando la dipendenza dalla derivata di ordine massimo:
\begin{equation}\label{eq:366793}
 u^{(n)}(x) = f(x, u(x), u'(x), \dots, u^{n-1}(x))
\end{equation}
dove $f\colon \Omega \to \RR$ è una funzione definita su $\Omega\subset I\times \RR^n$.
L'equazione del pendolo si può scrivere in questa forma,
scegliendo $f(x,u,v) = -\sin u - v$.

Più in generale potremmo considerare
\emph{sistemi di equazioni differenziali}
\mymargin{sistemi}%
\index{sistemi di equazioni differenziali}%
\index{equazione!differenziale!sistema}%
in più incognite.
Possiamo rappresentare un sistema di $k$ equazioni ordinarie in $m$ incognite nella forma:
\begin{equation}\label{eq:375456}
  \vec F(x, \vec u(x), \vec u'(x), \dots, \vec u^{n}(x)) = \vec 0
\end{equation}
dove $\vec u$ è una funzione vettoriale $\vec u \colon I \to \RR^m$ 
le cui componenti sono le $m$ funzioni incognite:
\[
  \vec u(x) = (u_1(x), \dots, u_m(x))
\]
mentre la funzione $\vec F \colon I \times (\RR^m)^{n+1} \to \RR^k$ è stavolta
una funzione a valori vettoriali $\vec F = (F_1, \dots, F_k)$ cosicché l'equazione vettoriale
\eqref{eq:375456} è effettivamente equivalente ad un sistema di $k$ equazioni:
\[
\begin{cases}
F_1(x,\vec u(x), \vec u'(x)\dots, \vec u^{(n)}(x)) = 0\\
F_2(x,\vec u(x), \vec u'(x)\dots, \vec u^{(n)}(x)) = 0\\
\quad\vdots \\
F_k(x,\vec u(x), \vec u'(x)\dots, \vec u^{(n)}(x)) = 0\\
\end{cases}
\]
Vedremo che per le equazioni ordinarie del primo ordine è naturale,
come accade per le equazioni algebriche,
avere lo stesso numero di equazioni e di incognite dunque è tipico avere singole
equazioni scalari del primo ordine
(cioè in cui l'incognita è una funzione a valori nel campo
degli scalari $\RR$)
o sistemi di $k=m$ equazioni del primo ordine con incognita una funzione
vettoriale $\vec u$ (cioè una funzione a valori nello spazio vettoriale $\RR^m$)
ovvero con $m$ incognite $u_1,\dots, u_m$ funzioni scalari.

Una importante osservazione è il fatto generale che una equazione differenziale
di ordine $n$ può essere ricondotta ad un sistema di $n$ equazioni
differenziali del primo ordine.
Basta infatti considerare come incognita il vettore (chiamato \emph{jet})
\[
  \vec u = (u, u', u'', \dots, u^{(n-1)})
\]
comprendente tutte le derivate della funzione scalare $u$ fino all'ordine $n-1$.
L'equazione \eqref{eq:3784643}, di ordine $n$, risulta infatti equivalente
al sistema di $n$ equazioni del primo ordine nella variabile
$\vec u = (u_1, \dots, u_n)$
\[
  \begin{cases}
    u_2(x) = u_1'(x)\\
    u_3(x) = u_2'(x)\\
    \quad \vdots \\
    u_{n}(x) = u_{n-1}'(x)\\
    F(x, u_1(x), u_2(x), \dots, u_n(x), u_n'(x)) = 0
  \end{cases}
\]
Nel caso, più interessante,
delle equazioni in forma normale
l'equazione~\eqref{eq:366793} di ordine $n$
diventa un sistema di $n$ equazioni normali
del primo ordine
in $n$ incognite
\[
  \begin{cases}
  u_1'(x) = u_2(x)\\
  u_2'(x) = u_3(x)\\
  \quad \vdots\\
  u_{n-1}'(x) = u_n(x)\\
  u_n'(x) = f(x,u_1(x), u_2(x), \dots, u_n(x))
  \end{cases}
\]
ovvero una equazione differenziale vettoriale
del primo ordine
\[
  \vec u'(x) = \vec f(x, \vec u(x)).
\]
\begin{comment}
avendo definito $\vec f = (f_1, \dots, f_n)$
come
\begin{gather*}
  f_1(x, y_1,\dots, y_n) = y_2 \\
  f_1(x, y_1,\dots, y_n) = y_3 \\
  \quad \vdots \\
  f_{n-1}(x, y_1,\dots, y_n) = y_n \\
  f_n(x, y_1,\dots, y_n) = f(x,y_1, \dots, y_n)
\end{gather*}
\end{comment}

Nell'esempio del pendolo \eqref{eq:39872} si avrà come incognita una funzione vettoriale
$\vec u(x) = (u(x),u'(x))$ le cui componenti sono posizione e velocità angolare.
Il codominio di tale funzione si chiama \emph{spazio delle fasi}.
L'equazione (essendo autonoma tralasciamo la dipendenza da $x$)
si scriverà nella forma $\vec u' = \vec f(\vec u)$ con
$\vec f(y_1,y_2) = (y_2, -\sin(y_1) - y_2)$.

\subsection{equazioni lineari}

Un caso molto particolare ma decisamente importante è quello in cui la funzione
$F$ (per le equazioni in forma implicita) o la funzione $f$
(per le equazioni in forma normale) sono funzioni lineari
per ogni $t$
rispetto alla variabile $u$.
In tal caso diremo che l'equazione è
\mymargin{equazioni lineari omogenee}%
\index{equazione!differenziale!lineare omogenea}%
\emph{lineare omogenea}.
Più precisamente si avrà\mynote{%
Per evidenziare il fatto che una funzione $L$ è lineare tenderemo ad utilizzare 
le parentesi quadre (invece che le tonde) per racchiudere l'argomento della 
funzione: $L[v] = L(v)$.
}
\[
F(x,u(x),u'(x), \dots, u^{(n)}(x)) = A_x[u(x), u'(x), \dots, u^{(n)}(x)]
\]
con $A_x\colon \RR^{n+1}\to \RR$
operatore lineare per ogni $x$ ovvero $A_x$ si rappresenta tramite un vettore
i cui coefficienti sono funzioni della variabile $x$:
\[
  A_x[\vec y] = \sum_{k=0}^n a_k(x) y_k
\]
e l'equazione differenziale si scrive nella forma
\[
  a_0(x) u(x) + a_1(x) u'(x) + \dots + a_n(x) u^{(n)}(x) = 0.
\]
Nel caso in cui i coefficienti $a_k(t)$ non dipendano da $x$
(cioè siano funzioni costanti) diremo che l'equazione è
lineare
\mymargin{coefficienti costanti}%
\index{equazione!differenziale!lineare a coefficienti costanti}%
\emph{a coefficienti costanti}.

E' facile osservare che l'insieme delle soluzioni di una equazione lineare 
omogenea è uno spazio vettoriale: $u=0$ è sempre soluzione, 
se $u$ è una soluzione e $\lambda \in \RR$ anche $\lambda u$ è soluzione 
e se $u$ e $v$ sono due soluzioni anche $u+v$ è soluzione
\mymargin{principio di sovrapposizione}%
\index{principio!di sovrapposizione}%
\index{principio!di sovrapposizione}%
(principio di sovrapposizione).

In effetti
se le funzioni $u$ sono definite su un intervallo $I$ possiamo
identificare $F$ con un funzionale $L\colon \RR^I \to \RR^I$ definito da
\[
  L[u](x) = F(x,u(x), u'(x), \dots, u^{(n)}(x))
\]
Se l'equazione differenziale è lineare allora $L$ è un operatore lineare
sullo spazio vettoriale $\RR^I$ e lo spazio delle soluzioni dell'equazione
differenziale non è altro che $\ker L$, il nucleo dell'operatore,
ed è noto che $\ker L$ è un sottospazio vettoriale.
Nonostante $\RR^I$ sia uno spazio vettoriale di dimensione infinita scopriremo
che (sotto opportune ipotesi) lo spazio vettoriale delle soluzioni
ha dimensione finita $n$, uguale al grado dell'equazione.

Nel caso in cui la funzione $F$ (o la corrispondente $f$)
sia affine\mynote{%
Una funzione $f\colon V\to W$ definita tra due spazi vettoriali si dice 
essere \emph{affine} se può essere scritta nella forma $f(v)=L[v] + q$
con $L\colon V\to W$ operatore lineare e $q\in W$ fissato.
Una equazione lineare $L[v]=w$ può quindi sempre essere scritta nella forma 
$f(v)=0$ con $f$ affine.
} 
si dirà che l'equazione differenziale
è una equazione
\mymargin{equazione lineari}%
\index{equazione!differenziale!lineare}%
\emph{lineare (non omogenea)}.
L'equazione non omogenea avrà la forma:
\[
  a_0(x) u(x) + a_1(x) u'(x) + \dots + a_n(x) u^{(n)}(x) = g(x).
\]
Se $v_0$ e $v_1$ sono due soluzioni di questa equazione è chiaro che la
differenza $u=v_1 - v_0$ è soluzione dell'equazione omogenea
\[
  a_0(x) u(x) + a_1(x) u'(x) + \dots + a_n(x) u^{(n)}(x) = 0.
\]
Dunque quest'ultima si chiama equazione omogenea associata alla non omogenea
e se $v_0$ è una soluzione particolare (qualunque) dell'equazione non omogenea
ogni soluzione $v$ dell'equazione non omogenea si scrive nella forma
\[
  v = v_0 + u
\]
con $u$ soluzione dell'omogenea associata.
Per trovare tutte le soluzioni di una equazione non omogenea è dunque
sufficiente trovare una soluzione particolare della non omogenea 
e tutte le soluzioni della equazione omogenea associata.

L'equazione del pendolo \eqref{eq:39872} non è lineare
ma quando l'angolo $u$ è piccolo (cioè per \emph{piccole oscillazioni}) si ha $\sin u \sim u$.
Facendo questa \emph{linearizzazione} si ottiene l'equazione
\[
  u''(x)  + u(x) + u'(x) = 0.
\]
Questa è una equazione lineare omogenea.
Se sul pendolo agisce una forza esterna (pendolo forzato) l'equazione diventa
\[
  u''(x) + u(x) + u'(x) = g(x)
\]
dove $g(x)$ rappresenta l'entità di una forza esterna variabile nel tempo.
Questa equazione è lineare non omogenea.

Come nel caso generale le equazioni lineari di ordine $n$ si riconducono
a sistemi lineari di $n$ equazioni del primo ordine.

\subsection{problema di Cauchy}

Andremo a dimostrare non solo che le equazioni lineari di ordine $n$ (omogenee o non omogenee) 
hanno come insieme 
delle soluzioni uno spazio lineare (vettoriale o affine) di dimensione $n$, 
ma, più in generale, che l'insieme dell1e soluzioni di una equazione differenziale 
anche non lineare, può essere descritto tramite $n$ parametri
(dunque tale insieme di soluzioni può essere pensato come una ``superficie'' 
di dimension $n$ nello spazio vettoriale di dimensione infinita 
di tutte le funzioni derivabili).

L'idea è che le equazioni ordinarie di ordine $n$ 
hanno una forma di \emph{determinismo}:
se conosco la soluzione $u(x_0)$ in un istante $x_0$ 
e conosco anche tutte le sue derivate $u'(x_0), u''(x_0), \dots, u^{(n-1)}(x_0)$ 
in quell'istante, allora la soluzione esiste ed è univocamente determinata
(teorema~\ref{th:cauchy_lipschitz_ordine_n}).

I valori $u(x_0), u'(x_0), \dots, u^{(n-1)}(x_0)$ si chiamano \emph{condizioni iniziali}.
e il problema di determinare una soluzione di una equazione differenziale
\mymargin{condizione iniziale}%
\index{condizione iniziale}%
con le condizioni iniziali fissate si chiama \emph{problema di Cauchy}:
\mymargin{problema di Cauchy}%
\index{problema!di Cauchy}%
\[
  \begin{cases}
  u^{(n)}(x) = f(x, u(x), \dots, u^{n-1}(x))\\
  u(x_0) = y_0, \\
  u'(x_0) = y_1, \\
  \vdots\\
  u^{(n-1)}(x_0) = y_{n-1}.
  \end{cases}
\]
Vedremo appunto che, sotto opportune ipotesi, il problema di Cauchy
ammette una unica soluzione e questo significa, sostanzialmente,
che l'insieme di tutte le soluzioni dell'equazione differenziale
può essere descritto dal variare
degli $n$ parametri $y_0, y_1, \dots, y_{n-1}$.

Una delle ipotesi fondamentali per garantire l'unicità delle
soluzioni di un problema di Cauchy è che le soluzioni siano
definite su un intervallo. E' chiaro infatti che se ho due
soluzioni definite su due intervalli separati, potrei unire
le due soluzioni e ottenere una soluzione definita sull'unione dei
due intervalli pur potendo scegliere condizioni iniziali 
indipendenti in ognuno dei due intervalli. 
Viceversa una soluzione definita
su un intervallo $I$ è soluzione anche se ristretta ad ogni sottoinsieme
di $I$ (che sia o no un intervallo).
Se restringo una soluzione o unisco due soluzioni ottengo formalmente
soluzioni diverse in quanto i domini di definizione sono diversi, ma
queste soluzioni assumono, dove sono definite, gli stessi valori.
Per evitare questa ambiguità considereremo sempre solamente
le soluzioni definite su un \emph{intervallo massimale}%
\mymargin{intervallo massimale}\index{intervallo massimale} cioè
soluzioni definite su un intervallo che non possono essere
estese su intervalli più grandi.

