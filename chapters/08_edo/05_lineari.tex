\section{equazioni lineari di ordine $n$}
\label{sec:edo_lineari}

\index{equazione!differenziale!lineari di ordine $n$}
\mymargin{equazioni lineari di ordine $n$}
Le equazioni differenziali ordinarie lineari di ordine $n$ in forma normale possono essere scritte nella forma:
\begin{equation}\label{eq:edo_lineare_ordine_n}
  u^{(n)}(x) + a_{n-1}(x) \cdot u^{(n-1)}(x) + \dots + a_1(x) \cdot u'(x) + a_0(x) \cdot u(x) = b(x)
\end{equation}
con $a_k\colon A \to \RR$, $b\colon A\to \RR$ funzioni continue definite su uno stesso dominio $A\subset \RR$.
Nel caso $b(x) = 0$ l'equazione si dice essere \emph{omogenea}%
\mymargin{omogenea}\index{omogenea}
e si può scrivere come:
\begin{equation}\label{eq:edo_lineare_omogenea_ordine_n}
  u^{(n)}(x) + a_{n-1}(x) \cdot u^{(n-1)}(x) + \dots + a_1(x)\cdot u'(x) + a_0(x)\cdot u(x) = 0.
\end{equation}
In generale l'equazione \eqref{eq:edo_lineare_ordine_n}
viene chiamata \emph{equazione non omogenea}
\mymargin{equazione non omogenea}
\index{equazione!differenziale!non omogenea}
e la corrispondente equazione~\eqref{eq:edo_lineare_omogenea_ordine_n}
viene chiamata \emph{equazione omogenea associata}.
\mymargin{equazione omogenea associata}
\index{equazione!differenziale!omogenea associata}

\begin{theorem}[struttura delle soluzioni di una equazione lineare]%
\label{th:edo_lineare_ordine_n}%
\mymark{***}%
Siano $a_k\in C^0(I)$ con $I\subset \RR$
un intervallo aperto.

\begin{enumerate}
\item
L'insieme $V$ delle soluzioni dell'equazione lineare omogenea~\eqref{eq:edo_lineare_omogenea_ordine_n}
è un sottospazio vettoriale di $C^n(I)$ di dimensione $n$.
Inoltre, fissato un punto qualunque $x_0\in I$ l'operatore $J\colon V \to \RR^n$
(chiamato \emph{jet}%
\mymargin{jet}\index{jet}) definito da
\begin{equation}\label{eq:jet}
  J[u] = (u(x_0), u'(x_0), u''(x_0), \dots, u^{(n-1)}(x_0))
\end{equation}
è un operatore lineare bigettivo (cioè un isomorfismo di spazi vettoriali).

\item
L'insieme delle soluzioni dell'equazione non omogenea~\eqref{eq:edo_lineare_ordine_n}
è un sottospazio affine di $C^n(A)$ di dimensione $n$,
parallelo al sottospazio delle soluzioni dell'equazione omogenea
associata~\eqref{eq:edo_lineare_omogenea_ordine_n}.
In particolare se $u_*$ è una soluzione particolare dell'equazione non
omogenea~\eqref{eq:edo_lineare_ordine_n} ogni altra soluzione $u$ di
\eqref{eq:edo_lineare_ordine_n} si scrive nella forma
\[
  u = u_* + v
\]
con $v$ soluzione dell'equazione omogenea associata.
\end{enumerate}
\end{theorem}
%
\begin{proof}
\mymark{***}
Innanzitutto il teorema~\ref{th:edo_esistenza_globale} di esistenza globale garantisce che
le soluzioni delle equazioni~\eqref{eq:edo_lineare_ordine_n}
e~\eqref{eq:edo_lineare_omogenea_ordine_n} esistono e sono funzioni in $C^n(I)$.

Possiamo riscrivere l'equazione~\eqref{eq:edo_lineare_ordine_n} nella forma
\[
  L[u] = b
\]
(usiamo le parentesi quadre per delimitare l'argomento quando si tratta di una 
applicazione lineare)
con
\[
  L[u] = u^{(n)} + \sum_{k=0}^{n-1} a_k u^{(k)}
\]

L'equazione omogenea~\eqref{eq:edo_lineare_omogenea_ordine_n}
risulta quindi essere
\[
  L[u] = 0.
\]
Si osservi che $L\colon C^n(I) \to C^0(I)$ è un operatore lineare in quanto la
somma, la derivata e la moltiplicazione per una funzione sono operatori lineari
sullo spazio vettoriale delle funzioni.
Dunque l'insieme $V$ delle soluzioni dell'equazione omogenea non è altro che
$\ker L$ che notoriamente è uno spazio vettoriale visto che se $L[u]=0$ e $L[v]=0$
allora anche $L[\lambda u + \mu v] = \lambda L[u] + \mu L[v] = 0$.

Possiamo ora determinare la dimensione di tale spazio, mettendo in
corrispondenza le soluzioni dell'equazione con un loro dato iniziale,
tramite l'applicazione $J[u]$ definita da~\eqref{eq:jet}.
Chiaramente $J$ è lineare perché l'operatore derivata e la valutazione in un
punto sono operatori lineari. Osserviamo che $J$ è suriettivo perché dato un
qualunque $\vec y\in \RR^n$ per il teorema \ref{th:cauchy_lipschitz_ordine_n}
di esistenza (globale) di soluzioni per il problema di Cauchy di ordine $n$
sappiamo esistere una soluzione $u\in V$ tale che $J[u]=\vec y$. Ma $J$ è anche
iniettivo perché se $u,v\in V$ sono due soluzioni con $J[u]=J[v]$ significa
che $u$ e $v$ verificano lo stesso problema di Cauchy.
Per l'unicità della soluzione risulta quindi $u=v$.
Abbiamo quindi mostrato che $J\colon V \to \RR^n$ è un isomorfismo di spazi
 vettoriali, quindi $\dim V=n$.

Per quanto riguarda l'equazione non omogenea
sia
\[
W = \ENCLOSE{v\in C^n(A)\colon L[v] = b}
\]
l'insieme di tutte le soluzioni.
Se consideriamo una soluzione particolare $v_0\in W$ e se $v\in W$ è una
qualunque altra soluzione, si osserva che
\[
  L[v-v_0] = L[v] - L[v_0] = b-b = 0.
\]
Significa che $u=v-v_0$ è soluzione dell'equazione omogenea associata:
$u\in V=\ker L$.
Dunque ogni soluzione $v$ dell'equazione non omogenea si può scrivere nella
forma $v = v_0 + u$ con $v_0$ soluzione particolare della non omogenea e
$u$ soluzione generale dell'equazione omogenea associata ovvero
\[
  W = v_0 + V.
\]
\end{proof}

\begin{theorem}[maggiore regolarità delle soluzioni]%
  \label{th:maggiore_regolarita}%
Se $u(x)$ è una soluzione dell'equazione differenziale lineare~\eqref{eq:edo_lineare_ordine_n} e se i coefficienti $a_1, \dots, a_{n-1}, b$ sono funzioni di classe $C^m$ per un certo $m\in \NN$ allora la soluzione è di classe $C^{m+n}$. In particolare se i coefficienti sono di classe $C^\infty$ le soluzioni sono anch'esse di classe $C^\infty$.
\end{theorem}
%
\begin{proof}
Se $u$ è soluzione di~\eqref{eq:edo_lineare_ordine_n}
per definizione sappiamo che $u$ è derivabile $n$ volte nell'intervallo $I$
su cui è definita.
Se scriviamo l'equazione in forma normale:
\[
  u^{(n)}(x) = b(x) - \sum_{k=0}^{n-1}a_k(x) u^{(k)}(x)
\]
essendo $a_k \in C^0$ sappiamo che $u^{(n)}$ è continua dunque $u$ è di classe $C^n$.
Supponiamo ora che sia $u\in C^j$ per qualche $j\ge n$.
I coefficienti dell'equazione sono di classe $C^m$ e vengono moltiplicati per le derivate di $u$ che sono almeno di classe $C^{j-n+1}$.
Se $j-n+1 \le m$ allora il lato destro della precedente equazione è di classe $j-n+1$.
Dunque $u^{(n)}\in C^{j-n+1}$ da cui $u\in C^{j+1}$.
Un passo alla volta è quindi possibile incrementare la regolarità di $u\in C^j$ finché $j-n+1\le m$ cioè finché $j\le m+n-1$. A quel punto otteniamo $u\in C^{j+1} = C^{m+n}$.

Se i coefficienti sono di classe $C^\infty$ il procedimento non termina mai
e si ottiene dunque che anche $u$ è di classe $C^\infty$.
\end{proof}

