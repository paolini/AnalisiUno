\section{il problema di Cauchy}
%%%%%
%%%%%
%%%%%
%%%%%

Il problema di Cauchy per i sistemi del primo ordine
consiste nel
trovare una soluzione di un sistema di 
$n$ equazioni differenziali ordinarie del primo ordine in $n$ incognite
$u_1, \dots, u_n$
con un dato iniziale fissato. 
Cioè dato $x_0\in \RR$,
e $\vec y_0 \in \RR^n$ si cerca un intervallo $I\subset \RR$ con $x_0\in I$ 
e una funzione $\vec u \colon I\to \RR^n$,
$\vec u = (u_1, \dots, u_n)$, tale
che $\vec u$ sia derivabile e soddisfi:
\begin{equation}\label{eq:problema_cauchy}
\begin{cases}
 \vec u'(x) = \vec f(x, \vec u(x)), \qquad \forall x\in I\\
 \vec u(x_0) = \vec y_0.
\end{cases}
\end{equation}

\subsection{esistenza e unicità locale}

\begin{theorem}[Cauchy-Lipschitz: esistenza e unicità locale]%
  \label{th:cauchy_lipschitz}%
  \index{Cauchy!teorema di Cauchy-Lipschitz}%
  \index{teorema!di Cauchy-Lipschitz}%
  \index{teorema!di esistenza e unicità locale}%  
Siano $a\le x_0 \le b$, $R>0$, $\vec y_0 \in \RR^n$. 
Posto $K=\ENCLOSE{\vec y\in \RR^n \colon \abs{\vec y - \vec y_0}\le R}\subset \RR^n$ 
sia $\vec f\colon [a,b]\times K\to \RR^n$ una funzione con le proprietà:
\begin{enumerate}
  \item $\vec f$ è continua,
  \item esiste $L>0$ tale che per ogni $x\in [a,b]$, ed ogni $\vec y_1,\vec y_2 \in K$ 
  si ha 
  \begin{equation}\label{eq:hp_lipschitz}
  \abs{\vec f(x, \vec y_2) - \vec f(x,\vec y_1)} \le L \abs{\vec y_2- \vec y_1}.  
  \end{equation}
\end{enumerate}

Allora esiste $\delta>0$ tale che posto 
scelto comunque un intervallo $I$ con 
$x_0 \in I \subset [a,b]\cap [x_0-\delta, x_0+\delta]$ 
esiste una unica funzione derivabile $\vec u\colon I \to K$ 
che risolve il problema di Cauchy~\eqref{eq:problema_cauchy}.
\end{theorem}
%
\begin{proof}
  \emph{Passo 1.}
  Per prima cosa vogliamo dimostrare che una funzione $\vec u\colon I\to K\subset \RR^n$
  è derivabile e risolve il problema di Cauchy~\eqref{eq:problema_cauchy} 
  se e solo se 
  $\vec u$ è continua e risolve l'equazione integrale
  \begin{equation}\label{eq:problema_cauchy_integrale}
  \vec u(x) = \vec y_0 + \int_{x_0}^x \vec f(t,\vec u(t))\, dt.
  \end{equation}

  Se $\vec u$ è derivabile e 
  $\vec u'(x) = \vec f(x, \vec u(x))$ per ogni $x\in[a,b]$ allora 
  siccome $\vec f$ è continua la funzione $\vec f(t,\vec u(t))$ è continua 
  (teorema~\ref{th:continuita_funzione_composta_metrico})
  e quindi $\vec u$ è di classe $C^1$. 
  Posso allora integrare ambo i membri 
  e ottenere:
  \[
   \int_{x_0}^x \vec u'(t)\, dt = \int_{x_0}^x \vec f(t,\vec u(t))\, dt.  
  \]
  Per il teorema fondamentale del calcolo il lato sinistro è uguale 
  a $\vec u(x) - \vec u(x_0) = \vec u(x) - \vec y_0$ e quindi si ottiene 
  la validità di~\eqref{eq:problema_cauchy_integrale}.

  Viceversa se $\vec u$ è continua e soddisfa~\eqref{eq:problema_cauchy_integrale}
  allora, sempre per il teorema fondamentale del calcolo possiamo affermare
  che l'integrale nel lato destro nell'uguaglianza~\eqref{eq:problema_cauchy_integrale}
  è una funzione derivabile con derivata $\vec f(x,\vec u(x))$ e dunque anche 
  il lato sinistro è derivabile e vale $\vec u'(x) = \vec f(x,\vec u(x))$ per 
  ogni $x\in I$. 
  Abbiamo già osservato che se $\vec u$ è continua questa funzione (che è la derivata 
  di $\vec u$) è anch'essa continua e quindi $\vec u$ è di classe $C^1$.
  Per $x=x_0$ l'integrale si annulla e quindi si ottiene 
  anche $\vec u(x_0) = \vec y_0$ concludendo che in effetti $\vec u$ risolve 
  il problema di Cauchy~\eqref{eq:problema_cauchy}.

  \emph{Passo 2.}
  Visto che $[a,b]\times K$ è un insieme chiuso e limitato in $\RR \times \RR^n$ 
  il teorema di Weierstrass garantisce che esiste $M>0$ tale che 
  $\abs{\vec f(x,y)}\le M$ per ogni $x\in [a,b]$ e ogni $y\in K$.
  Scegliamo $\delta >0$ tale che $\delta \le \frac{R}{M}$ e $\delta < \frac {1}{L}$
  e prendiamo l'intervallo $I = [a,b]\cap [x_0-\delta,x_0+\delta]$.
  Consideriamo ora l'insieme $X=C(I,K)\subset C(I,\RR^n)$ di tutte le funzioni continue $\vec u\colon I\to K$
  e mettiamo su $X$ la distanza uniforme $d_\infty$.
  E' facile osservare che $X$ è un sottospazio chiuso di $\B(I,\RR^n)$ (le funzioni limitate definite su $I$)
  e visto che quest'ultimo è uno spazio completo (teorema~\ref{th:limitate_completo}) anche $X$ 
  è completo (teorema~\ref{th:chiuso_completo}).

  Definiamo su $X$ l'operatore $T\colon X \to C(I,\RR^n)$ come segue:
  \[
    T(\vec u)(x) = \vec y_0 + \int_{x_0}^x \vec f(t,\vec u(t))\, dt.
  \]
  Osserviamo che l'equazione~\eqref{eq:problema_cauchy_integrale} può essere riscritta
  come equazione di punto fisso: $T(\vec u) = \vec u$. 
  Il nostro intento è quindi quello di utilizzare il teorema~\ref{th:banach-caccioppoli} delle contrazioni.

  \emph{Passo 3.}
  Dimostriamo innanzitutto che $T(X)\subset X$ cosicché possiamo pensare a $T$ come ad un operatore 
  $T\colon X\to X$ che manda $X$ in sé.
  Posto $\vec v = T(\vec u)$ si ha
  (teorema~\ref{th:stima_modulo_integrale}):
  \begin{align*}
    \abs{\vec v(x)-\vec y_0} 
    &= \abs{\int_{x_0}^x \vec f(t, \vec u(t))\, dt} 
    \le \abs{\int_{x_0}^x \abs{\vec f(t,\vec u(t))}\, dt}\\
    &\le \abs{\int_{x_0}^x M\, dt } 
    = \abs{x-x_0}\cdot M \le \delta \cdot M.
  \end{align*}
Visto che abbiamo scelto $\delta \le \frac{R}{M}$ si ottiene che $v(x) \in K$ per ogni $x\in [a,b]$ 
e quindi effettivamente $T(u)\in K$.

\emph{Passo 4.}
Dimostriamo infine che $T$ è una contrazione. 
Se $\vec u,\vec v\in X$ si ha, usando l'ipotesi~\eqref{eq:hp_lipschitz} con $x=t$, $\vec y_1=\vec u(x)$
$\vec y_2 = \vec v(x)$,
\begin{align*}
\abs{T(\vec u)(x)-T(\vec v)(x)} 
  &= \abs{\int_{x_0}^x \vec f(t,\vec u(t))\, dt - \int_{x_0}^x \vec f(t,\vec v(t))\, dt} \\
  &\le \abs{\int_{x_0}^x \abs{\vec f(t,\vec u(t))- \vec f(t,\vec v(t))}\, dt }\\
  &\le \abs{\int_{x_0}^x L\cdot  \abs{\vec u(t), \vec v(t)}\, dt} 
  \le \abs{\int_{x_0}^x L \cdot d_\infty(\vec u, \vec v)\, dt}  \\
  &=\abs{x-x_0}\cdot L \cdot d_\infty(\vec u,\vec v)
  \le \delta \cdot L \cdot d_\infty(\vec u,\vec v).
\end{align*}
Avendo scelto $\delta < \frac{1}{L}$ si ha $\delta L<1$ e quindi effettivamente l'operatore $T$ 
è una contrazione.
Sono dunque verificare le ipotesi del teorema~\ref{th:banach-caccioppoli} delle contrazioni
che garantiscono l'esistenza e unicità del punto fisso $\vec u$.

Ma abbiamo già osservato che $T(\vec u) = \vec u$ è equivalente al problema di Cauchy~\eqref{eq:problema_cauchy}
e quindi la dimostrazione è terminata.
\end{proof}

\begin{definition}[Cauchy-Lipschitz]%
\label{def:cauchy_lipschitz}%
Diremo che una funzione $\vec f\colon \Omega\subset \RR\times \RR^n \to \RR^m$
soddisfa la \emph{proprietà di Cauchy-Lipschitz}%
\mymargin{proprietà di Cauchy-Lipschitz}\index{proprietà di Cauchy-Lipschitz} se $f$ è continua e se
per ogni $(x_0,\vec y_0)\in \Omega$ esistono $a<x_0$, $b>x_0$, $R>0$ ed $L>0$
tali che le ipotesi del teorema~\ref{th:cauchy_lipschitz} sono verificate.
\end{definition}
 
La condizione di Cauchy-Lipschitz può sembrare piuttosto complicata
da verificare. Una condizione molto più semplice  è che la funzione $\vec f$
sia di classe $C^1$. La seguente proposizione ci dice che tale
condizione è sufficiente per applicare il teorema di Cauchy-Lipschitz.

\begin{proposition}%
\mymark{***}%
Sia $\Omega \subset \RR\times \RR^n$ un insieme aperto.
Se $\vec f\in C^1(\Omega,\RR^m)$ 
allora $f$ soddisfa la condizione
di Cauchy-Lipschitz (definizione~\ref{def:cauchy_lipschitz}).

In particolare, grazie al teorema~\ref{th:cauchy_lipschitz} 
per ogni $(x_0,\vec y_0)\in \Omega$ esiste un intervallo aperto 
$I$ contenente il punto $x_0$
ed esiste una unica soluzione $u$ 
del problema di Cauchy~\eqref{eq:problema_cauchy} definita 
su tale intervallo $I$.
\end{proposition}
%
\begin{proof}
\mymark{***}%
Se prendiamo un qualunque cilindro $[a,b]\times K$ con 
\[
  K=\ENCLOSE{\vec y\in \RR^n\colon \abs{\vec y-\vec y_0}\le R}, 
\]
$\vec y_0\in \Omega$, $K\subset \Omega$
sappiamo che ogni derivata parziale
$\partial f_k / \partial y_j$ è continua ed è quindi limitata
(per il teorema di Weierstrass) su $[a,b]\times K$. 
Sia $L_{k,j}$ il massimo di $\abs{\partial f_k / \partial y_j}$ su $K$ e sia
$L$ il massimo degli $L_{k,j}$ per $k=1,\dots m$ e $j=1\dots,n$.
Allora presi $\vec y, \vec z\in K$ si può scomporre l'incremento vettoriale 
$\vec z - \vec y$ lungo le $n$ direzioni coordinate e applicare il teorema di Lagrange lungo ogni direzione.
Per ogni $k = 1,\dots, m$ si ha quindi:
\begin{align*}
\MoveEqLeft
\abs{f_k(x,\vec z)-f_k(x,\vec y)}\\
&\le \sum_{j=1}^n \abs{f_k(x,z_1, \dots, z_j, y_{j+1}, \dots, y_n) - f_k(x,z_1, \dots, z_{j-1}, y_j, \dots, y_n)}\\
&\le \sum_{j=1}^n L_{k,j} \abs{z_j-y_j}
\le \sum_{j=1}^n L_{k,j} \abs{\vec z-\vec y}
\le n L \abs{\vec z-\vec y}
\end{align*}
da cui
\begin{align*}
  \abs{\vec f(x,\vec z) - \vec f(x,\vec y)}
  &= \sqrt{\sum_{k=1}^m \abs{f_k(x,\vec z) - f_k(x,\vec y)}^2} 
  \le \sqrt{\sum_{k=1}^m n^2 L^2 \abs{\vec z-\vec y}^2} \\
  &\le n\sqrt{m} L \abs{\vec z-\vec y} = L' \abs{\vec z-\vec y}.
\end{align*}
\end{proof}

Il teorema di Cauchy-Lipschitz garantisce l'esistenza di una soluzione
definita in un
piccolo intervallo $[x_0-\delta_0, x_0+\delta_0]$ intorno al punto iniziale $x_0$.
Se scegliamo $x_1 = x_0+\delta_0$ possiamo applicare nuovamente lo stesso
teorema per trovare una soluzione definita nell'intervallo
$[x_1, x_1+\delta_1]$
che coincida con la precedente soluzione nel punto $x_1$.
Le due soluzioni possono
essere incollate per ottenere una soluzione definita sull'unione dei due intervalli.
Questo procedimento può essere ripetuto infinite volte ma gli intervalli
potrebbero diventare sempre più piccoli, quindi non c'è garanzia di ottenere una
soluzione definita su intervalli sempre più grandi.
Nel seguito cercheremo di capire quanto grande può essere l'intervallo
\emph{massimale} su cui si riesce ad estendere la soluzione.

\begin{definition}[soluzione/intervallo massimale]
Sia $I\subset \RR$ un intervallo non vuoto e sia $\vec u \colon I \to \RR^n$
una soluzione di una equazione differenziale.

Se $J\supset I$, $J \neq I$ è un intervallo, se $\vec v\colon J \to \RR^n$
risolve la stessa equazione differenziale
e se $\vec v(x)=\vec u(x)$ per ogni $x\in I$,
diremo che $\vec v$ è una \emph{estensione della soluzione}%
\mymargin{estensione della soluzione}\index{estensione della soluzione} $u(x)$.

Se la soluzione $\vec u$ non ammette estensioni, diremo che $\vec u$ è una
soluzione definita su un intervallo massimale o, più semplicemente, diremo
che $\vec u$ è una \emph{soluzione massimale}%
\mymargin{soluzione massimale}\index{soluzione massimale}.
\end{definition}

\begin{proposition}[caratterizzazione delle soluzioni massimali]
\label{prop:edo_massimale}
Sia $\Omega \subset \RR\times \RR^n$ un insieme aperto
e supponiamo che $\vec f\colon \Omega\subset \RR\times\RR^n\to\RR^n$
sia una funzione continua che soddisfa la condizione
di Cauchy-Lipschitz (definizione~\ref{def:cauchy_lipschitz}).
Allora una soluzione $\vec u$ dell'equazione
\begin{equation}\label{eq:edo_normale_ordine_uno}
  \vec u'(x) = \vec f(x, \vec u(x))
\end{equation}
definita su un intervallo $I$ è massimale se e solo se
$I$ è aperto e per ogni $K$ compatto $K\subset \Omega$
e per ogni $x_0 \in I$ esistono $x_1,x_2\in I$, $x_1 < x_0 < x_2$ tali che 
$(x_1, \vec u(x_1))\not \in K$ e $(x_2,\vec u(x_2)) \not \in K$.
Significa cioè che il grafico di $\vec u$ cioè la curva $(x,\vec u(x))$ esce da qualunque compatto  
$K\subset \Omega$ sia facendo crescere $x$ verso destra che facendo calare $x$ verso sinistra.
Intuitivamente il grafico della soluzione massimale tende ad arrivare ai limiti di $\Omega$.
\end{proposition}
%
\begin{proof}
\emph{Prima implicazione}.
Supponiamo che $\vec u\colon I \to \RR^n$ sia una soluzione di~\eqref{eq:edo_normale_ordine_uno} 
definita su un intervallo massimale $I$. Dovrà essere $(x,\vec u(x))\in \Omega$ per ogni $x\in I$.

Mostriamo innanzitutto
che $I$ è un intervallo aperto a destra: se non lo fosse si avrebbe che $x_0=\sup I \in I$. 
Allora $(x_0,\vec u(x_0))\in \Omega$ e per il teorema di esistenza e unicità locale
potremmo trovare un piccolo intervallino $J=[x_0,x_0+\delta]$ una funzione
$\vec v\colon J\to \RR^n$ che risolve~\eqref{eq:edo_normale_ordine_uno}
con $\vec v(x_0)=\vec u(x_0)$.
Facendo l'unione dei due grafici ottengo una estensione di $\vec u$ a tutto l'intervallo
$I\cup J$ che è strettamente più grande di $I$. 
Dunque $\vec u$ non poteva essere una soluzione massimale.

Mostriamo ora che la curva $(x,\vec u(x))$ esce da ogni compatto $K\subset \Omega$ sia da destra che da sinistra. 
Supponiamo per assurdo che esista un compatto $K$ tale che $(x,\vec u(x))\in K$ per ogni $x\in I$.
Il grafico $(x,\vec u(x))$ è limitato per $x\in I$ dunque certamente $I$ deve essere limitato. 
Inoltre abbiamo visto che $I$ è aperto e quindi $I=(a,b)$ con $a,b\in \RR$, $a<b$.

Allora per ogni $x\in I=(a,b)$ si ha
\[
  \abs{\vec u'(x)} = \abs{\vec f(x, \vec u(x))}
   \le \sup_K \abs{\vec f}.
\]
Visto che $\vec f$ è continua 
la funzione $\abs{\vec f}$ ha massimo su $K$ per il teorema di Weierstrass e quindi esiste 
$M\ge 0$ tale che $\abs{\vec u'(x)} \le M$ per ogni $x\in (a,b)$. 
Significa che ogni componente di $\vec u$ è $M$-lipschitziana, in particolare
ogni componente è uniformemente continua (teorema~\ref{th:lipschitz_uniformemente_continua}).
Dunque la funzione $\vec u$ può essere estesa con continuità
(teorema~\ref{th:estensione_uniformemente_continua})
ad una funzione $\vec v \colon [a,b] \to \RR^n$ definita anche agli estremi dell'intervallo.
Visto che $(x,\vec u(x)) \in K$ per ogni $x\in(a,b)$ e visto che
$\vec v(b) = \lim_{x\to b^-}\vec u(x)$ essendo $K$ chiuso possiamo affermare che
$(b,\vec v(b)) \in K$.
Lo stesso vale per $x\to a^+$ dunque $(a,\vec v(a))\in K$
e dunque $(x,\vec v(x))\in K \subset \Omega$ per ogni $x\in [a,b]$.
Vogliamo ora verificare che $\vec v$ è soluzione dell'equazione differenziale~\eqref{eq:edo_normale_ordine_uno}.
Chiaramente $\vec v$ soddisfa l'equazione per ogni $x\in (a,b)$
in quanto su $(a,b)$ coincide con $\vec u$ che è soluzione.
Consideriamo l'estremo $b$.
Visto che $\vec v$ è continua in $b$ e $\vec f$ è continua in $(b,\vec v(b))$ si ha
\[
  \lim_{x\to b^-} \vec v'(x)
  = \lim_{x\to b^-} \vec f(x,\vec  v(x))
  = \vec f(b,\vec v(b)) \in \RR^n
\]
Sappiamo però che se il limite della derivata di una funzione continua esiste ed è finito, allora la funzione è derivabile nel punto limite e la derivata è continua in quel punto
(proposizione~\ref{prop:4384774}).
Dunque $\vec v$ è derivabile in $b$ e anche in quel punto soddisfa l'equazione differenziale. Lo stesso vale nel punto $a$. Dunque siamo riusciti a trovare una estensione $\vec v$ di $\vec u$ contraddicendo l'ipotesi che $\vec u$ fosse una soluzione massimale.

\emph{Seconda implicazione}. Supponiamo ora $\vec u\colon I\to \RR^n$ sia una soluzione dell'equazione 
differenziale~\eqref{eq:edo_normale_ordine_uno} definita su un intervallo $I$ e tale che la curva 
$(x,\vec u(x))$ esca da ogni compatto $K$ al variare di $x\in I$. 
Vogliamo dimostrare che $\vec u$ è soluzione massimale. 
Innanzitutto $I$ non può essere compatto, altrimenti anche $K=\ENCLOSE{(x,\vec u(x))\colon x \in I}$ 
sarebbe un compatto di $\Omega$ e ovviamente la curva non esce da $K$.
Sia $a=\inf I$ e $b=\sup I$. 
Siccome $I$ non è chiuso esso non contiene uno dei due estremi: supponiamo sia $b$. 
Allora se $\vec u$ fosse estendibile a destra esisterebbe una estensione continua 
$\vec v\colon (a,b]\to \RR^n$ che coincide con $\vec u$ su $(a,b)$ e che è continua 
nel punto $x=b$ e che soddisfa l'equazione differenziale quindi, in particolare, $(b,\vec v(b))\in \Omega$.
Ma allora scelto $x_0 \in (a,b)$
prendiamo $K=\ENCLOSE{(x,\vec v(x))\colon x\in [x_0,b]}$. 
Visto che $\vec v\colon[x_0,b]\to \RR^n$ è continua è facile verificare che $K$ è chiuso è limitato, 
$K\subset \Omega$. Ma ovviamente $(x,\vec v(x))\in K$ per ogni $x\ge x_0$, contro le ipotesi.
\end{proof}

\begin{proposition}[separazione delle soluzioni]
\label{prop:separazione_soluzioni}%
Se $\vec f\colon \Omega\subset \RR \times \RR^n\to \RR^n$
verifica la condizione di Cauchy-Lipschitz (definizione~\ref{def:cauchy_lipschitz}),
se $\vec u\colon I \to \RR^n$ e $\vec v\colon J\to\RR^n$ sono due soluzioni
dell'equazione differenziale
$\vec u'(x) = \vec f(x,\vec u(x))$
definite su due intervalli $I,J \subset \RR$  
e se esiste $x_0\in I\cap J$ tale che $\vec u(x_0) = \vec v(x_0)$ allora $\vec u(x) = \vec v(x)$ per ogni $x\in I\cap J$.
Detto in altri termini: nelle ipotesi del teorema di esistenza e unicità i grafici di due soluzioni diverse definite in un intervallo non possono toccarsi.
\end{proposition}
%
\begin{proof}
Sia $x_0 \in I\cap J$ un punto in cui $\vec u(x_0) = \vec v(x_0)$
e supponiamo per assurdo che esista $x_2 \in I\cap J$ tale che $\vec u(x_2)\neq \vec v(x_2)$. In tal caso possiamo considerare il punto
\[
   x_1 = \sup \ENCLOSE{x \in [x_0,x_2] \colon \vec u(x) = \vec v(x)}.
\]
Su tutto l'intervallo $[x_0,x_1)$ si ha quindi $\vec u(x) = \vec v(x)$ e per continuità dovrà dunque anche essere $\vec u(x_1) =  \vec v(x_1)$.
In pratica $x_1$ è l'ultimo punto di contatto tra le due soluzioni.
Ponendo allora il punto $(x_1,\vec u(x_1))$ come dato iniziale del problema di Cauchy scopriamo che $\vec u$ e $\vec v$ sono localmente due soluzioni di tale problema. Per l'unicità locale le due soluzioni devono coincidere in un piccolo intorno del punto $x_1$, diciamo in particolare che devono coincidere su $[x_1,x_1+\eps]$ ma questo è in contraddizione con la definizione di $x_1$.
\end{proof}

\begin{theorem}[esistenza di soluzioni massimali]
\label{th:edo_esistenza_massimali}
Sia $\vec u\colon I\to \RR^n$ una soluzione dell'equazione differenziale~\eqref{eq:edo_normale_ordine_uno}
definita su un intervallo non vuoto $I\subset \RR$
e supponiamo che tale equazioni soddisfi il teorema di esistenza e unicità locale 
(questo succede ad esempio se $f\in C^1$).
Se $\vec u$ non è essa stessa una soluzione massimale, esiste
sempre una estensione massimale $\vec v\colon J\to \RR^n$, $J\supset I$.
\end{theorem}
%
\begin{proof}
Supponiamo che $\vec u$ non sia massimale, dunque $\vec u$ ammette estensioni. 
%In particolare ammette estensioni definite su intervalli aperti, in quanto se una soluzione è definita su un intervallo che non è aperto posso sempre estenderla in un intorno aperto dei punti di frontiera dell'intervallo mediante il teorema di esistenza locale ottenendo quindi una estensione definita su un intervallo aperto.

Sia $\mathcal F$ l'insieme di tutte le estensioni di $\vec u$.
%definite su intervalli aperti. 
Più precisamente ogni $\vec w\in \mathcal F$ è una funzione 
$\vec w\colon J_{\vec w}\to \RR^n$ 
definita su un intervallo aperto $J_{\vec w}\supset I$ 
che soddisfa l'equazione differenziale~\eqref{eq:edo_normale_ordine_uno} 
e che coincide con $\vec u$ su $I$. 
Definiamo $\vec v\colon J\to\RR^n$ come segue:
\begin{align*}
J &= \bigcup_{\vec w \in \mathcal F} J_{\vec w}\\
\vec v(x) &= \vec w(x) \qquad\text{se $x\in J_{\vec w}$}.
\end{align*}
Si osservi che dato $x\in I$ se $\vec w_1, \vec w_2\in\mathcal F$ 
sono due estensioni entrambe definite su un punto $x$, 
allora esse coincidono in $x$ per la 
Proposizione~\ref{prop:separazione_soluzioni}, 
dunque $\vec v(x)$ è univocamente definita.

%Essendo ogni $J_{\vec w}$ aperto è chiaro che $J$ è aperto. 
Chiaramente $J$ deve essere un intervallo, 
in quanto tutti i $J_{\vec w}$ hanno in comune i punti di $I$.
Verifichiamo ora che $\vec v$ soddisfa l'equazione differenziale. 
Preso un punto $x\in J$ deve esistere $\vec w$ tale che $x\in J_{\vec w}$. 
Inoltre se $x$ non è un estremo di $J$ posso scegliere $\vec w$ in modo 
che $x$ non sia un estremo di $J_{\vec w}$.
Sappiamo che $\vec w'(x) = f(x,\vec w(x))$ e visto che $\vec u$ coincide 
con $\vec w$ su $J_{\vec w}$ possiamo dedurre che 
$\vec u$ è derivabile e che le derivate coincidono:
\[
\vec u'(x) = \vec w'(x) = f(x,\vec w(x)) = f(x,\vec u(x)).
\]
Dunque anche $\vec u$ è soluzione di~\eqref{eq:edo_normale_ordine_uno}.
In pratica abbiamo verificato che $\vec v\in \mathcal F$ ed è quindi una estensione di $\vec u$. 
Questa è una estensione massimale perché se ci fosse una estensione definita su un intervallo 
più grande 
%ce ne sarebbe anche una definita su un intervallo più grande aperto, 
%ma questo è impossibile perché
sarebbe anch'essa inclusa in $\mathcal F$.
\end{proof}

\subsection{esistenza globale}

\begin{theorem}[esistenza globale]
\label{th:edo_esistenza_globale}
Sia $I = (a,b) \subset \RR$ un intervallo aperto, $\Omega=I\times \RR^n$ e sia
$\vec f\colon \Omega \to \RR^n$, $\vec f = \vec f(x,\vec y)$ una funzione che
soddisfa la proprietà di Cauchy-Lipschitz
e che sia sublineare in $\vec y$ uniformemente rispetto a $x$, cioè
esistono $m,q\in \RR$ tali che:
\[
  \abs{\vec f(x,\vec y)} \le m\abs{\vec y} + q,
  \qquad\forall (x,\vec y)\in \Omega.
\]
Allora per ogni $x_0\in I$ e per ogni $\vec y_0\in \RR^n$ esiste una funzione
$\vec u\colon I \to \RR^n$ (definita su tutto $I$) soluzione del problema di Cauchy~\eqref{eq:problema_cauchy}.
\end{theorem}
%
La dimostrazione richiede un lemma preliminare.
\begin{lemma}[Gronwall]
Siano $a,b\in \RR$ con $a<b$ e siano $m,q\ge 0$.
Sia $\vec u \colon [a,b) \to \RR^n$ una funzione di classe $C^1$
tale che
\[
  \abs{\vec u'(t)} \le m \abs{\vec u(t)} + q
  \qquad \forall t\in [a,b).
\]
Allora $\vec u$ è limitata.
\end{lemma}
%
\begin{proof}
Ricordando che
\[
  (\abs{\vec u}^2)'
  = (\vec u\cdot \vec u)'
  = 2 \vec u \cdot \vec u'
\]
per ogni $x\in [a,b)$ possiamo fare la seguente stima:
\begin{align*}
\Enclose{\ln(1+\abs{\vec u(t)}^2)}_{a}^{x}
&= \int_{a}^{x}
  \enclose{\ln(1+\abs{\vec u(t)}^2)}' \, dt
= \int_{a}^{x}
\frac{2\vec u(t)\cdot \vec u'(t)}{1+\abs{\vec u(t)}^2}\, dt \\
&\le 2 \int_{a}^{x} \frac{\abs{\vec u(t)} \abs{\vec u'(t)}}{1+\abs{\vec u(t)}^2}\, dt
\le 2\int_{a}^{x} \frac{\abs{\vec u(t)}\enclose{m\abs{\vec u(t)}+q}}{1+\abs{\vec u(t)}^2}\, dt\\
&\le 2\int_{a}^{x} \enclose{m + \frac{q\abs{\vec u(t)}}{1+\abs{\vec u(t)}^2}}\, dt\\
&\le (x-a)(2m + q) \le (b-a)(2m+q)
\end{align*}
avendo anche sfruttato la disuguaglianza $s/(1+s^2) \le 1/2$ con $s=\abs{\vec u(t)}$.
Dunque
\[
  \ln(1+\abs{\vec u(x)}^2) \le \ln(1+\abs{\vec u(a)}^2) + (b-a)(2m+q)
\]
è una funzione limitata e di conseguenza anche $\abs{\vec u(x)}$ è limitata.
\end{proof}

\begin{proof}[Dimostrazione teorema~\ref{th:edo_esistenza_globale}]
Sia $\vec u$ una soluzione massimale del problema di Cauchy e sia
$J\subset I$ l'intervallo massimale su cui $\vec u$ è definita.
Sia $b=\sup J$. Sappiamo quindi che $\vec u$ è definita su $[x_0,b)$
e quindi per il lemma precedente deve essere limitata: esiste cioè
$M>0$ per cui $\abs{\vec u(x)}\le M$ per ogni $x\in[x_0,b)$.
Se consideriamo il compatto
\[
K=[x_0,b] \times \overline{B_M(\vec 0)}
 = \ENCLOSE{(x,\vec y)\in \RR \times \RR^n\colon x\in [x_0,b], \abs{\vec y}\le M}
\]
abbiamo quindi verificato che $(x,\vec u(x))\in K$ per ogni $x\in [a_0,b)$.
Se fosse $b\in I$ si avrebbe $K\subset \Omega$ e quindi questo sarebbe
in contraddizione con la proposizione~\ref{prop:edo_massimale} che
ci dice che ogni soluzione massimale esce da qualunque compatto contenuto
in $\Omega$. Significa allora che $b\not \in I$ e cioè che $b = \sup J = \sup I$
(non può essere $\sup J>\sup I$ in quanto $J\subset I$).

In maniera analoga (o per simmetria) si può dimostrare che $\inf J=\inf I$
e dunque $J=I$ come volevamo dimostrare
\end{proof}

