\section{teorema fondamentale del calcolo}

\begin{theorem}[del valor medio]%
  \label{th:media_integrale}%
  \mymark{***}%
  Siano $a,b\in \RR$, $a<b$ e sia
  $f\colon \closeinterval{a}{b} \to \RR$ una funzione Riemann-integrabile.
  Allora
  \begin{equation}\label{eq:media_inf_sup}
    \inf_{\closeinterval{a}{b}} f 
    \le \frac{\int_a^b f(x)\, dx}{b-a} 
    \le \sup_{\closeinterval{a}{b}} f.
  \end{equation}
  Inoltre se $f$ è continua 
  esiste un punto $c \in \closeinterval{a}{b}$
  tale che
  \[
  \frac{\int_a^b f}{b-a} = f(c).
  \]
  \end{theorem}
  %
  La quantità
  \[
    \frac{\int_a^b f}{b-a}
  \]
  si chiama \emph{valor medio integrale} di $f$ su $[a,b]$ e spesso
  si indica con il simbolo
  \[
    \dashint_a^b f.
  \]
  %
  \begin{proof}
  \mymark{***}
  Per ogni $x\in [a,b]$ si ha 
  \[
    \inf_{\closeinterval{a}{b}} f \le f(x) \le \sup_{\closeinterval{a}{b}}f.
  \]
  Integrando si ottiene
  \[
    \inf_{\closeinterval{a}{b}} f \cdot (b-a)
    \le \int_a^b f(x)\, dx 
    \le \sup_{\closeinterval{a}{b}}f\cdot (b-a)
  \]
  e dividendo per $b-a$ arriviamo a~\eqref{eq:media_inf_sup}.

  Se la funzione $f$ è continua su $[a,b]$ per il teorema di Weierstrass 
  $\inf$ e $\sup$ sono valori assunti dalla funzione 
  e la media integrale $\dashint_a^b f$ è un valore intermedio tra il 
  minimo e il massimo.
  Ma per il teorema~\ref{th:valori_intermedi} dei valori intermedi 
  possiamo infine affermare che esiste un punto $c\in [a,b]$
  dove la funzione assume tale valore.
\end{proof}
  
\mymargin{teor. fondamentale}
\begin{theorem}[Torricelli-Barrow: teorema fondamentale del calcolo]
\mymark{***}%
\index{teorema!fondamentale del calcolo}%
\index{teorema!di Torricelli-Barrow}%
\index{Torricelli-Barrow!teorema di}%
\label{th:torricelli-barrow}%
Sia $I\subset \RR$ un intervallo, sia $x_0 \in I$
 e sia $f\colon I\to \RR$ una funzione continua.
Allora la \emph{funzione integrale}
\index{funzione!integrale}
$F\colon I \to \RR$
\mymargin{funzione integrale}
\[
  F(x) = \int_{x_0}^x f
\]
è ben definita, è derivabile e si ha per ogni $x\in I$
\[
  F'(x) = f(x).
\]
In particolare essendo $f\in C^0(I)$ si ha $F\in C^1(I)$.

Inoltre se $G\colon I \to \RR$ è una qualunque funzione tale che
$G'(x) = f(x)$ per ogni $x\in I$, allora per ogni $a,b \in I$ si ha
\mymargin{formula fondamentale del calcolo}
\index{formula!fondamentale del calcolo}%
\[
  \int_a^b f = G(b) - G(a).
\]
\end{theorem}
%
\begin{proof}
\mymark{***}
Osserviamo innanzitutto che la funzione $f$, essendo continua, è integrabile
su ogni intervallo chiuso e limitato contenuto in $I$. Dunque l'integrale
$\int_{x_0}^x f$ è ben definito.

Per ogni $h\neq 0$, se $x+h \in I$ per l'additività dell'integrale
si ha
\[
\frac{F(x+h) - F(x)}{h} = \frac{\int_{x_0}^{x+h} f - \int_{x_0}^x f}{h}
 = \frac{\int_x^{x+h} f}{h}.
\]
Applicando ora il teorema del valor medio possiamo
affermare che esiste un punto $\xi(h)$ nell'intervallo di estremi $x$ e $x+h$
tale che
\[
  \frac{\int_x^{x+h} f}{h} = f(\xi(h)).
\]
Per $h\to 0$, si ha $\xi(h) \to x$ e, per continuità di $f$,
$f(\xi(h)) \to f(x)$.
Dunque abbiamo mostrato che $F$ è derivabile in $x$:
\[
 \lim_{h\to 0}\frac{F(x+h)-F(x)}{h} = f(x)
\]
e $F'(x) = f(x)$.

Dunque se $a,b\in I$ sono punti qualunque si ha:
\[
\int_a^b f = \int_{x_0}^b f - \int_{x_0}^a f = F(b) - F(a).
\]
E se $G\colon I \to \RR$ è una qualunque  funzione tale che $G'(x)=f(x)$ si
avrà $G'(x) = F'(x)$ per ogni $x\in I$ e dunque $(G-F)' = 0$ su $I$.
Per i criteri di monotonia possiamo concludere che $G-F$ è costante su $I$:
$G-F = c$. Dunque si ha
\[
 \int_a^b f = F(b) - F(a) = (G(b) - c) - (G(a) - c) = G(b) - G(a).
\]
\end{proof}

\begin{example}
Si voglia calcolare
\[
  \int_0^b x^2\, dx.
\]
Basterà osservare che posto $G(x)=\frac{x^3}{3}$ si ha $G'(x)=x^2$
e quindi, grazie alla formula fondamentale del calcolo integrale si
ha
\[
  \int_0^b x^2\, dx = G(b) - G(0) = \frac{b^3}{3}.
\]
E' evidente quanto questo metodo risolutivo sia molto più semplice
e potente di quello utilizzato nell'esempio~\ref{ex:integrale_quadrato}.
\end{example}

\begin{definition}[primitiva]
\label{def:primitiva}%
\mymark{***}%
Sia $A \subset \RR$ e sia $f\colon A \to \RR$ una funzione qualunque.
Una funzione $F\colon A \to \RR$ si dice essere una \emph{primitiva}%
\mymargin{primitiva}%
\index{primitiva}
(o \emph{antiderivata})
\index{antiderivata}
di $f$ se $F$ è derivabile e $F'(x)=f(x)$ per ogni $x\in A$.
\end{definition}

Il teorema fondamentale del calcolo integrale può dunque essere espresso nel
modo seguente: ogni funzione $f$ continua, definita su un intervallo,
ammette almeno una primitiva e se $F$ è una qualunque primitiva di $f$ si ha
\[
  \int_a^b f = F(b) - F(a).
\]
Per indicare la differenza $F(b)-F(a)$ si usano
talvolta le seguenti notazioni:
\[
  \Enclose{F(x)}_{x=a}^b = \Enclose F_a^b
  = F(x) \vert_{x=a}^b
  = F \vert_a^b = F(b) - F(a).
\]

Il calcolo degli integrali si riduce quindi alla determinazione delle primitive
ovvero ad invertire l'operatore di derivata.
Risulterà quindi importante avere degli strumenti per determinare le primitive
di una funzione.

\begin{definition}[integrale indefinito]
\mymark{***}
L'insieme di tutte le primitive di una funzione $f\colon A \to \RR$
si indica con il simbolo
\[
  \int f
  \qquad\text{oppure}\qquad
  \int f(x) \, dx
\]
e si chiama \emph{integrale indefinito}.
Il motivo di questa notazione (e del nome) deriva dal teorema fondamentale del
calcolo integrale, in
base al quale se $f\colon[a,b]\to \RR$
è continua si ha
\[
  \int_a^b f = \Enclose{\int f}_a^b.
\]
\end{definition}

\begin{remark}
Si faccia attenzione però che nel caso di funzioni non continue è possibile
che le funzioni integrali non siano primitive. Ad esempio
se si prende la funzione di Heaviside $H(x)$ definita nell'esempio~\ref{ex:heaviside}
si può osservare che $F(x)=\int_0^x H(t)\, dt = \abs{x}$
vale $0$ se $x<\le 0$ e vale $x$ se $x\ge 0$.
Dunque $F$ è una funzione integrale ma non è una primitiva di $H$
(e l'insieme delle primitive, in questo caso, è vuoto).
\end{remark}

Se pensiamo all'operatore lineare $D$ definito sull'insieme delle funzioni
derivabili $Df = f'$
si può pensare a $\int f$ come all'insieme delle controimmagini di $f$
tramite $D$ ovvero:
\[
  \int f = D^{-1}(\ENCLOSE{f}) = \ENCLOSE{F \colon DF =f }.
\]
Dunque $\int f$ è l'insieme delle soluzioni $F$ dell'equazione lineare
non omogenea $DF =f$. L'equazione omogenea associata è $DF = 0$.
Se le funzioni sono definite su un intervallo allora le soluzioni dell'equazione omogenea sono le costanti (come enunciato nel seguente teorema).
Dunque in tal caso lo spazio $\int f$ è uno spazio affine di dimensione $1$
parallelo allo spazio vettoriale delle funzioni costanti. 
Ma se scegliamo un dominio che non è un intervallo 
(come nell'esempio~\ref{ex:primitive_non_connesso}) 
si avrà uno spazio la cui dimensione è pari al numero di intervalli che compongono il dominio.

\begin{theorem}[proprietà delle primitive]
\label{th:primitive}
\mymark{***}
Sia $f\colon I \to \RR$
una funzione continua definita su un intervallo non vuoto $I\subset \RR$. Allora
\begin{enumerate}
\item esiste almeno una primitiva $F$ di $f$;
\item data una primitiva $F$ di $f$ ogni altra
primitiva $G$ differisce da $F$ per una costante: $\exists c\in \RR\colon G= F+c$.
\end{enumerate}

Detto in altri termini se $f$ è continua $\int f$ non è vuoto e se inoltre 
il dominio di $f$ è un intervallo e $F\in \int f$ è una qualunque primitiva, allora
\[
  \int f = \ENCLOSE{F+c \colon c \in \RR}.
\]
\end{theorem}

Osserviamo che l'insieme delle funzioni costanti
su un intervallo
non è altro che $\ker D$ ovvero lo spazio di annullamento dell'operatore derivata.
Stiamo dunque semplicemente osservando che le controimmagini di un operatore
lineare sono spazi affini paralleli al nucleo dell'operatore.

\begin{proof}
\mymark{***}
Scelto un punto $x_0\in I$ possiamo
considerare la funzione integrale
\[
  F(x) = \int_{x_0}^x f(t)\, dt.
\]
Il teorema fondamentale del calcolo integrale
ci assicura che $F$ è una primitiva di $f$.

Viceversa se $F$ e $G$ sono due primitive di $f$ allora si ha:
\[
  F' = G' = f.
\]
Posto $H=G-F$ avremo quindi $H'=0$ sull'intervallo $I$. Per i criteri di
monotonia sappiamo quindi che $H$ è costante, ovvero esiste $c\in \RR$ tale
che $H(x)=c$ per ogni $x\in I$. Dunque si ottiene, come voluto: $G=F+H=F+c$.
\end{proof}

Se la funzione $f$ non è definita su un intervallo l'insieme delle primitive
avrà dimensione maggiore di $1$, come si vede nel seguente.

\begin{example}[primitive sugli insiemi non connessi]
\label{ex:primitive_non_connesso}%
\mymark{*}%
Consideriamo la funzione $f(x) = 1/x$. Osserviamo che
$f\colon (-\infty, 0) \cup (0,+\infty)\to \RR$ è definita sull'unione di due
intervalli. Per verifica diretta possiamo osservare che la funzione
$F(x) = \ln \abs{x}$ è una primitiva di $f$. Per ottenere l'insieme di tutte
le primitive possiamo aggiungere ad $F$ una qualunque funzione con derivata
nulla sul dominio di $f$. Le funzioni con derivata nulla sono costanti su ogni
intervallo e quindi troviamo che per ogni $c_1, c_2\in \RR$ la funzione
\[
G(x) =
\begin{cases}
  \ln (x) + c_1 &\text{se $x>0$},\\
  \ln (-x) + c_2 & \text{se $x<0$}
\end{cases}
\]
è una primitiva di $f$ e non ci sono altre primitive.

In questo caso lo spazio delle primitive ha dimensione $2$ in quanto il nucleo
dell'operatore derivata sullo spazio delle funzioni definite sull'unione di
due intervalli ha dimensione $2$.
Questo è l'esempio più semplice di un fenomeno piuttosto generale per cui gli
operatori differenziali su un certo spazio risultano strettamente legati alla
topologia dello spazio stesso. In questo caso la dimensione del nucleo
dell'operatore derivata $D$ è uguale al numero di componenti connesse del
dominio delle funzioni nel dominio di $D$.
\end{example}

