\section{integrali impropri}

La definizione di integrale di Riemann è stata data solamente per funzioni
limitate definite su un intervallo
limitato.
Vogliamo ora estendere la definizione di integrale alle funzioni illimitate e
agli intervalli illimitati.
Lo faremo riconducendoci, tramite un limite, al caso già studiato.

\begin{definition}[integrabilità locale]
\label{def:localmente_riemann}
Sia $f\colon A \to \RR$ una funzione definita su un insieme $A\subset \RR$.
Diremo che $f$ è \emph{localmente Riemann-integrabile} se
per ogni intervallo chiuso e limitato $[\alpha,\beta]\subset A$ risulta
che $f$ sia limitata e Riemann-integrabile su $[\alpha,\beta]$.
\end{definition}

\begin{remark}
Se $f\colon A \to \RR$ è una funzione continua, allora su ogni
intervallo $[\alpha,\beta]\subset A$ la funzione $f$ risulta essere continua
e quindi limitata e Riemann-integrabile per il teorema~\ref{th:integrabilita_continue}.
Lo stesso vale per le funzioni monotone:
grazie al teorema~\ref{th:integrabilita_monotone}
sappiamo che una funzione monotona è limitata e
Riemann-integrabile su ogni intervallo $[\alpha,\beta]$ dunque anch'essa
è localmente Riemann-integrabile. Sarà molto raro avere a che fare con
funzioni localmente Riemann-integrabili che non si riconducono
a questi due casi.
\end{remark}

\begin{definition}[integrale improprio]
\label{def:integrale_improprio}
\index{integrale!improprio}
Se $f$ è una funzione localmente Riemann-integrabile sull'intervallo $[a,b)$
con $a\in \RR$, $b\in (a,+\infty]$ definiamo l'integrale improprio
di $f$ su $[a,b)$ come:
\[
  \int_a^b f(x)\, dx = \lim_{\beta \to b^-} \int_a^\beta f(x)\, dx
\]
se il limite a lato destro esiste (finito o infinito).

Se $f$ è una funzione localmente Riemann-integrabile sull'intervallo $(a,b]$
con $b\in \RR$, $a\in [-\infty,b)$ definiamo l'integrale improprio
di $f$ su $(a,b]$ come:
\[
  \int_a^b f(x)\, dx = \lim_{\alpha \to a^+} \int_\alpha^b f(x)\, dx
\]
se il limite esiste (finito o infinito).

Se $f$ è una funzione localmente Riemann-integrabile sull'intervallo $(a,b)$
con $a\in [-\infty,+\infty)$, $b\in(a,+\infty]$ preso un qualunque
punto $c\in (a,b)$
definiamo l'integrale improprio di $f$ su $(a,b)$ come:
\begin{align*}
  \int_a^b f(x)\, dx &=
  \int_a^c f(x)\, dx + \int_c^b f(x)\, dx \\
  &= \lim_{\alpha \to a^+}\int_\alpha^c f(x)\,dx
    + \lim_{\beta\to b^-}\int_c^\beta f(x)\, dx
\end{align*}
sempre che entrambi i limiti esistano (finiti o infiniti) e non siano infiniti
di segno opposto (cosicché la loro somma è ben definita). In base
all'osservazione~\ref{rem:4821341} l'integrale non dipende
dal punto $c$ scelto.

Se $I$ è un intervallo di estremi $a,b\in [-\infty, +\infty]$, $a<b$
ed esistono un numero finito di punti $x_0,\dots,x_n \in I$ tali che
$a = x_0 < x_1 < \dots < x_n = b$ e se
$f$ risulta essere localmente Riemann-integrabile sull'insieme
$A=I\setminus\ENCLOSE{x_0, \dots, x_n}$ allora definiamo
l'integrale improprio di $f$ su $A$ come:
\[
  \int_a^b f(x)\, dx = \sum_{k=1}^n \int_{x_{k-1}}^{x_k} f(x)\, dx
\]
se ogni integrale improprio $\int_{x_{k-1}}^{x_k} f(x)\, dx$ esiste (finito o infinito)
e se la somma è ben definita in quanto tutti gli integrali infiniti
hanno lo stesso segno.

Se l'integrale improprio $\int_a^b f(x)\, dx$ esiste ed è finito
(in base ad una delle definizioni precedenti), diremo che
la funzione $f$ è
\mymargin{integrabile}
\emph{integrabile in senso improprio}
(o \emph{integrabile in senso generalizzato})
e diremo che l'integrale \emph{converge}.
\mymargin{convergente}%
\index{convergenza!integrale}%
\index{integrale!convergente}%
\index{integrale!improprio!convergente}%
Se invece l'integrale esiste ma non è finito, diremo che l'integrale
\emph{diverge}.
\mymargin{divergente}%
\index{divergenza!integrale}%
\index{integrale!divergente}%
\index{integrale!improprio!divergenza}%
Negli altri casi diremo che l'integrale è
\emph{indeterminato}.
\mymargin{indeterminato}%
\index{integrale!improprio!indeterminato}%
\index{integrale!indeterminato}%

Determinare il \emph{carattere}
\mymargin{carattere}%
\index{carattere!dell'integrale}%
\index{integrale!carattere}%
\index{integrale!improprio!carattere}%
dell'integrale $\int_a^b f(x)\, dx$
significa dire se tale integrale è convergente, divergente o indeterminato.

Se gli estremi di integrazione sono scambiati, $a>b$,
risulta utile utilizzare anche per gli integrali impropri
la convenzione
già introdotta per gli integrali di Riemann:
\[
  \int_a^b f(x) \, dx = - \int_b^a f(x)\, dx.
\]
\end{definition}

\begin{remark}\label{rem:4821341}
Nella definizione di integrale improprio sull'intervallo
aperto $(a,b)$ la scelta del punto $c\in(a,b)$ non influenza la definizione.
Se infatti scegliamo due diversi punti $c,c'\in(a,b)$ per ogni $\alpha,\beta \in (a,b)$
si ha, grazie alla additività dell'integrale di Riemann:
\[
  \int_\alpha^c f(x)\,dx + \int_c^\beta f(x)\,dx =
  \int_\alpha^{c'} f(x)\, dx + \int_{c'}^\beta f(x)\, dx.
\]
E quindi i limiti, se esistono, sono uguali.
\end{remark}

\begin{example}
La funzione $f(x)= \ln x$ può essere integrata in senso improprio sull'intervallo
$[1,+\infty)$ e si ha:
\begin{align*}
 \int_1^{+\infty} f(x)\,dx
 &= \lim_{\beta\to +\infty} \int_1^\beta \ln(x)\, dx
 = \lim_{\beta\to +\infty}\Enclose{x\ln x - x}_1^\beta \\
 &= \lim_{\beta\to +\infty} \enclose{\beta \ln \beta - \beta + 1}
 = +\infty.
\end{align*}
Anche sull'intervallo $(0,1]$ la funzione $\ln x$ ha integrale improprio
\[
  \int_0^1 \ln x\, dx = \lim_{\alpha \to 0^+}\Enclose{x \ln x -x}_\alpha^1
  = \lim_{\alpha\to 0^+} (-1-\alpha \ln \alpha + \alpha)  = -1.
\]
Dunque sull'intervallo $(0,+\infty)$ la funzione $\ln x$ ha integrale improprio
\[
  \int_0^{+\infty}\ln x\, dx = \int_0^1 \ln x\, dx + \int_1^{+\infty}\ln x\, dx
   = -1 + (+\infty) = +\infty.
\]
Diremo quindi che la funzione $f(x)=\ln x$ non è integrabile in senso improprio
sull'intervallo $(0,+\infty)$ ma l'integrale esiste ed è divergente.
\end{example}

Per abbreviare le notazioni intenderemo:
\[
  \Enclose {F(x)}_a^b
  = \lim_{x\to b^-}F(x) - \lim_{x\to a^+}F(x)
\]
osservando che se $F$ è definita e continua agli estremi $a$ e $b$ questa notazione coincide con l'usuale
\[
\Enclose{F(x)}_a^b = F(b) - F(a).
\]

Potremo quindi scrivere:
\begin{align*}
  \int_0^1 \ln x\, dx
  &= \Enclose{x\ln x- x}_0^1
  = 1 \cdot\ln 1 -1 - \lim_{x\to 0^+} (x\ln x - x) \\
  &= -1  - 0 = -1.
\end{align*}

Osserviamo che se $f\colon(a,b) \to \RR$ è continua e $F\colon(a,b)\to \RR$
è una sua primitiva
allora si avrà
\begin{align*}
  \int_a^b f(x)\, dx
   &= \lim_{\alpha\to a^+}\int_{\alpha}^c f(x)\, dx
   + \lim_{\beta\to b^-}\int_c^{\beta} f(x)\, dx \\
   & = \lim_{x\to a} (F(c)-F(x)) + \lim_{x\to b} (F(x)-F(c)) \\
   & = \lim_{x\to b} F(x) - \lim_{x\to a} F(x)
   = \Enclose{F(x)}_a^b.
\end{align*}
Dunque la formula fondamentale del calcolo integrale rimane formalmente identica
per gli integrali impropri.

\begin{example}
Per calcolare l'integrale
\[
  \int_{-\infty}^{+\infty} \frac{1}{1+x^2}\, dx
\]
possiamo scegliere un qualunque punto $c\in \RR$ e calcolare
\begin{align*}
\int_{-\infty}^{+\infty}\frac{1}{1+x^2}\, dx
&=
\lim_{\alpha\to-\infty}\int_\alpha^c \frac{1}{1+x^2}\, dx
  + \lim_{\beta\to+\infty}\int_c^\beta \frac{1}{1+x^2}\, dx\\
  &= [\arctg x]_{-\infty}^c
  + [\arctg x]_c^{+\infty}  \\
  &= \arctg c - (-\frac\pi 2) + \frac \pi 2 - \arctg c
  = \pi.
\end{align*}

Ma più semplicemente possiamo scrivere:
\[
 \int_{-\infty}^{+\infty}\frac{1}{1+x^2}\, dx
 = \Enclose{\arctg x}_{-\infty}^{+\infty}
 = \frac \pi 2 - \enclose{-\frac \pi 2} = \pi.
 \]

\end{example}

\begin{remark}
Se $f\colon[a,b]\to\RR$ è limitata e Riemann-integrabile su
$[a,b]$ allora,
in base al teorema~\ref{th:integrale_continuo}
si ha:
\[
 \int_a^b f(x)\, dx
  = \lim_{\alpha \to a^+} \int_\alpha^b f(x)\, dx
  = \lim_{\beta\to b^-}\int_a^\beta f(x)\,dx.
\]
Dunque in questo caso gli integrali impropri su $[a,b)$ e su $(a,b]$ coincidono
con l'usuale integrale di Riemann (e sono quindi convergenti).
Allo stesso modo se $f$ è localmente integrabile su $[a,b)$
l'integrale improprio su $[a,b)$ e l'integrale improprio su
$(a,b)$ coincidono.
Lo stesso succede se la funzione è localmente integrabile
su $(a,b)$ e consideriamo l'insieme $A=(a,b)\setminus\ENCLOSE{x_0,\dots,x_n}$.
L'integrale su $(a,b)$ coincide con l'integrale su $A$ grazie all'additività
dell'integrale rispetto al dominio.

Questo giustifica l'aver utilizzato la stessa notazione $\int_a^b$ sia per l'integrale
di Riemann su $[a,b]$ sia per i diversi integrali impropri su $[a,b)$, $(a,b]$,
$(a,b)$ o $(a,b)\setminus\ENCLOSE{x_0, \dots, x_n}$.
\end{remark}

\begin{example}
La funzione $\sin x$ pur essendo integrabile su ogni intervallo chiuso
e limitato (in quanto funzione continua) non ammette integrale
improprio sull'intervallo $[0,+\infty)$ in quanto l'integrale:
\[
  \int_0^\beta \sin(x)\, dx
  = \Enclose{-\cos(x)}_0^\beta
  = -\cos(\beta)+\cos(0) = 1-\cos(\beta)
\]
non ammette limite per $\beta\to +\infty$.
\end{example}

\begin{example}
La funzione $2x/(x^2+1)$ non è integrabile in senso improprio su $\RR$ in quanto si ha
\begin{align*}
\int_{0}^{+\infty} \frac {2x} {1+x^2}\, dx
 &= \Enclose{\ln (1+x^2)}_0^{+\infty} = +\infty \\
\int_{-\infty}^0 \frac{2x}{1+x^2}\, dx
 &= \Enclose{\ln(1+x^2)}_{-\infty}^0 = -\infty
\end{align*}
e $(+\infty)+ (-\infty)$ è indefinito.
\end{example}

\begin{example}
Si ha
\begin{align*}
  \int_{-1}^2 \frac{1}{\sqrt[3]{x}}\, dx
  &= \int_{-1}^0 \frac{1}{\sqrt[3] {x}}\, dx
   + \int_{0}^2 \frac{1}{\sqrt[3] x}\, dx \\
  &= \Enclose{\frac 3 2 \sqrt[3] {x^2}}_{-1}^0  + \Enclose{\frac 3 2 \sqrt[3] {x^2}}_0^2
  = 0 - \frac 3 2 + \frac 3 2 \sqrt[3]{4} - 0 \\
  &= \frac 3 2 (\sqrt[3] 4-1).
\end{align*}
\end{example}

\begin{example}
Si ha
\[
  \int_0^{+\infty} \frac{1}{x}\, dx
  = \Enclose{\ln x}_0^{+\infty}
  = +\infty - (-\infty) = +\infty
\]
e
\[
  \int_{-\infty}^{0} \frac{1}{x}\, dx
  = \Enclose{\ln(-x)}_{-\infty}^{0}
  = -\infty - (+\infty) = -\infty.
\]
Allora l'integrale
\[
  \int_{-\infty}^{+\infty} \frac{1}{x}\, dx
\]
non è definito in quanto somma di infiniti di segno opposto.
\end{example}

\begin{remark}
Attenzione a non dimenticare i punti ``cattivi'' all'interno dell'intervallo
di integrazione. Se vogliamo valutare:
\[
  \int_{-1}^1 \frac{1}{x}\, dx
\]
Potremmo essere portati a pensare che questo integrale sia uguale a:
\[
 \Enclose{\ln \abs{x}}_{-1}^1 = \ln \abs{1} - \ln \abs{-1} = 0.
\]
Invece questo integrale non è definito in quanto bisogna considerare
che la funzione integranda non è definita e comunque non è limitata in un
intorno di $x=0$ e quindi l'intervallo $[-1,1]$ va spezzato nell'unione
dei due intervalli $[-1,0)$ e $(0,1]$ da cui:
\[
  \int_{-1}^1 \frac{1}{x}\, dx = \Enclose{\ln \abs{x}}_{-1}^0 + \Enclose{\ln\abs{x}}_0^{1}
   = -\infty + (+\infty)
\]
ed essendoci una somma di infiniti con segno opposto la somma non ha senso
e l'integrale improprio non è definito.
\end{remark}

\begin{theorem}[proprietà dell'integrale improprio]
\label{th:proprieta_integrale_improprio}
Se $f\le g$ e se entrambi gli integrali sono definiti, risulta:
\index{integrale!improprio!monotonia}%
\mymargin{monotonia}%
\index{monotonia}%
\[
  \int_a^b f(x)\, dx \le \int_a^b g(x)\, dx.
\]

Siano $a,b,c \in \bar \RR$ con $a\le c \le b$.
Allora nella seguente uguaglianza
\index{integrale!improprio!additività}%
\mymargin{additività}%
\index{additività}%
\[
  \int_a^b f(x)\, dx = \int_a^c f(x)\, dx + \int_c^b f(x)\, dx
\]
se almeno uno dei due lati è definito allora anche l'altro lato lo è e
l'uguaglianza è valida.

Siano $\lambda, \mu \in \RR$.
\index{integrale!improprio!linearità}%
\mymargin{linearità}%
\index{linearità}%
\[
  \int_a^b \enclose{\lambda f(x) + \mu g(x)}\, dx
  = \lambda \int_a^b f(x) + \mu \int_a^b g(x)
\]
se il lato destro è ben definito (esistono gli integrali impropi di $f$ e $g$
e le operazioni hanno senso).

Sia $g\colon (a,b)\to (c,d)$ una funzione $C^1$
e $f\colon (c,d) \to \RR$ una funzione continua
e integrabile in senso improprio su $(c,d)$.
Siano $-\infty \le a < b \le +\infty$ e
$-\infty \le c < d \le +\infty$ tali che
\[
c = \lim_{x\to a^+} g(x),
\qquad
d = \lim_{x\to b^-} g(x).
\]
Se $f$ è integrabile in senso improprio su $(c,d)$
allora $f \circ g$ è integrabile su $(a,b)$ e
si ha
\index{integrale!improprio!cambio di variabile}%
\mymargin{cambio di variabile}%
\index{cambio di variabile}%
\[
\int_a^b f(g(x)) g'(x)\, dx = \int_c^d f(y)\, dy.
\]

Se invece gli estremi sono scambiati:
\[
d = \lim_{x\to a^+} g(x),
\qquad
c = \lim_{x\to b^-} g(x)
\]
si avrà, nelle stesse ipotesi:
\[
  \int_a^b f(g(x)) g'(x) \, dx = \int_d^c f(y)\, dy.
\]

Se $F$ è derivabile su $(a,b)$, $g$ è continua 
e $G$ è una primitiva di $g$ allora vale la formula:
\index{integrale!improprio!per parti}%
\mymargin{integrazione per parti}%
\index{integrazione per parti}%
\[
 \int_a^b F(x)g(x)\, dx = \Enclose {F(x) G(x)}_a^b - \int_a^b F'(x)G(x)\, dx  
\]
se le quantità sul lato destro esistono (finite o infinite) e la loro 
differenza non è una forma indeterminata.
\end{theorem}
%
\begin{proof}
Tutte queste proprietà sono già state dimostrate per l'integrale delle funzioni
limitate su intervalli limitati. Si possono facilmente estendere all'integrale
improprio laterale osservando che il limite mantiene le proprietà richieste.
Di conseguenza le proprietà valgono per additività anche per l'integrale
improprio bilaterale, facendo attenzione che non si produca una somma
indeterminata $+\infty + (-\infty)$.
Infine, sempre per additività, le formule sono valide per gli integrali
impropri multilaterali.
\end{proof}

\begin{exercise}\label{ex:821685}
  Si calcoli il seguente integrale doppio:
  \[
    \int_0^{+\infty} \Enclose{\int_0^{+\infty} \sin t \cdot e^{-tx}\, dt}\, dx.
  \]
\end{exercise}
%
\begin{proof}
L'integrale più interno può essere calcolato per parti:
\begin{align*}
  I(x) &= \int_0^{+\infty} \sin t\cdot e^{-tx}\, dt \\
  &=  \Enclose{-\cos t \cdot e^{-tx}}_0^{+\infty}
    - x \int_0^{+\infty} \cos t \cdot e^{-tx}\, dt \\
  &= 0 + 1 - x\enclose{\Enclose{\sin t \cdot e^{-tx}}_0^{+\infty} 
  +x \int_0^{+\infty}\sin t \cdot e^{-tx}\, dt}
\end{align*}
da cui 
\[
  I(x) = 1 - x^2 I(x), \qquad I(x) = \frac{1}{1+x^2}   
\]
e 
\[
 \int_0^{+\infty} \frac{1}{1+x^2}\, dx = \Enclose{\arctg x}_0^{+\infty} = \frac \pi 2.
\]
\end{proof}

\begin{theorem}[integrabilità delle funzioni positive]
\label{th:integrabilita_positive}
\mymark{**}
Sia $f\colon (a,b)\to \RR$ una funzione localmente Riemann-integrabile
non negativa: $f\ge 0$. Allora l'integrale di $f$ esiste (finito o infinito)
ed è non negativo:
\[
  \int_a^b f(x)\, dx \ge 0.
\]
\end{theorem}
%
\begin{proof}
\mymark{**}
Sia $c\in (a,b)$ un punto fissato e sia
\[
  F(x) = \int_c^x f(t)\, dt.
\]
Essendo $f$ localmente Riemann-integrabile la funzione $F$ è ben definita
ed essendo $f\ge 0$ risulta che $F$ è crescente in quanto se $x_2>x_1$
essendo $f\ge 0$ si ha
\[
  F(x_2)-F(x_1) = \int_{x_1}^{x_2} f(t)\, dt \ge \int_{x_1}^{x_2} 0\, dt = 0.
\]
Dunque esistono i limiti:
\begin{align*}
  \int_c^b f(x)\, dx  &= \lim_{\beta\to b^-} F(\beta)\\
  \int_a^c f(x)\, dx  &= \lim_{\alpha\to a^+} -F(\alpha)
\end{align*}
ed essendo $F$ crescente entrambi i limiti sono non negativi e quindi
la somma dei due limiti è ben definita.
\end{proof}

Anche se non sappiamo calcolare esplicitamente un integrale,
è spesso possibile determinarne la convergenza confrontandolo,
tramite il teorema seguente, con un integrale noto.

\begin{theorem}[criterio di confronto e confronto asintotico]
\mymark{**}
Siano $f, g \colon$ $ [a,b)\to \RR$ (con $a \le b \le +\infty$)
funzioni localmente Riemann-integrabili. Supponiamo inoltre che
per ogni $x\in [a,b)$ si abbia $f(x)\ge 0$ e $g(x)\ge 0$.
In queste ipotesi sappiamo che i due integrali:
\begin{equation}\label{eq:921754}
  \int_a^b f(x)\, dx, \qquad \int_a^b g(x)\, dx
\end{equation}
esistono entrambi (finiti o infiniti) e sono non negativi.
Inoltre abbiamo i seguenti criteri di confronto.

\begin{enumerate}
\item \emph{Confronto puntuale ``$\le$''.}
Supponiamo che per ogni $x\in [a,b)$ si abbia $f(x) \le g(x)$. Allora
si ha:
\begin{equation}\label{eq:467143}
  \int_a^b f(x)\, dx \le \int_a^b g(x)\, dx.
\end{equation}
In particolare se l'integrale di $g$ è convergente anche l'integrale di
$f$ è convergente mentre se l'integrale di $f$ è divergente anche l'integrale
di $g$ è divergente.

\item \emph{Confronto asintotico ``$\ll$''.}
Supponiamo che per $x\to b^-$ si abbia $f \ll g$ (cioè $f/g\to 0$).
Allora risulta che
se l'integrale di $g$ è convergente allora anche l'integrale di $f$ è convergente
mentre se l'integrale di $f$ è divergente anche l'integrale di $g$ è divergente.

\item \emph{Confronto asintotico ``$\sim$''.}
Supponiamo che per $x\to b^-$ si abbia $f\sim g$ (cioè $f/g\to 1$).
Allora i due integrali~\eqref{eq:921754}
hanno lo stesso carattere (entrambi convergenti oppure entrambi divergenti).
\end{enumerate}

Risultati analoghi valgono per funzioni definite su un intervallo aperto a
sinistra: $(a, b]$ con $-\infty \le a \le b$,
in tal caso nei criteri asintotici si faranno i limiti per $x\to a^+$.
\end{theorem}
%
\begin{proof}
\mymark{**}
Gli integrali di $f$ e $g$ esistono grazie al teorema~\ref{th:integrabilita_positive}.

Se $f\le g$ la disuguaglianza~\eqref{eq:467143} è garantità dalla proprietà di monotonia
(teorema~\ref{th:proprieta_integrale_improprio}).

Se $f(x)/g(x) \to 0$ per $x\to b^-$ significa che esiste un punto $c\in[a,b)$
tale che per ogni $x\in [c,b)$ si ha $f(x)/g(x)\le 1$ cioè $f(x)\le g(x)$.
Dunque, per l'additività e la monotonia dell'integrale, possiamo affermare che
\[
  \int_a^b f
  = \int_a^c f + \int_c^b f
  \le \int_a^c f + \int_c^b g
  = \int_a^c f + \int_a^b g - \int_a^c g.
\]
Sappiamo che gli integrali di $f$ e $g$ sull'intervallo
$[a,c]$ sono convergenti in quanto $f$ e $g$ sono
limitate e Riemann-integrabili su $[a,c]$.
Dunque se l'integrale $\int_a^b g$ è convergente anche l'integrale
$\int_a^b f$ è convergente.
Viceversa se $\int_a^b f$ è divergente allora anche $\int_a^b g$ è divergente.

Nel caso in cui $f(x)/g(x)\to 1$ per $x\to b^-$ possiamo trovare
un punto $c\in[a,b)$ tale per cui $\frac 1 2 \le f(x)/g(x) \le 2$
per ogni $x\in [c,b)$. Si avrà allora per ogni $x\in [c,b)$
\[
f(x) \le 2g(x), \qquad g(x) \le 2 f(x)
\]
e si potrà quindi procedere come nel caso precedente.
\end{proof}

Nei casi più frequenti il teorema precedente si applica confrontando la funzione con una potenza:
\[
  (x-x_0)^\alpha, \qquad \alpha \in \RR.
\]

Sarà quindi utile sapere, come termine di paragone,
per quali $\alpha$ queste funzioni hanno integrale convergente
come nel seguente esempio.

\begin{example}
\label{ex:416145}%
\mymark{***}%
Sia $a>0$ e $p\in \RR$. Gli integrali
\[
  \int_{a}^{+\infty}\frac{1}{x^p}\, dx,
  \qquad
  \int_{-\infty}^{-a}\frac{1}{(-x)^p}\, dx
\]
sono convergenti se e solo se $p>1$.

Sia $x_0\in [a,b]$ e $p\in\RR$.
Gli integrali
\[
  \int_a^{x_0} \frac{1}{(x-x_0)^p}\, dx,
  \qquad
  \int_{x_0}^b \frac{1}{(x-x_0)^p}\, dx
\]
sono convergenti se e solo se $p<1$.

La verifica si fa facilmente valutando il limite della primitiva
$F(x) = x^{1-p}/(1-p)$ negli estremi dell'intervallo.
\end{example}

\begin{example}
\label{ex:integrale_gaussiana_finito}%
\index{integrale!gaussiana}%
\index{funzione!gaussiana}%
\index{Gauss!funzione di}%
L'integrale
\[
 \int_{-\infty}^{+\infty}e^{-x^2}\, dx
\]
è convergente in quanto per $x\to +\infty$ risulta $e^{-x^2} \ll 1/x^2$
e l'integrale di $1/x^2$ è convergente (esempio~\ref{ex:416145})
in un intorno di $+\infty$ e di $-\infty$.
Nell'esercizio~\ref{ex:integrale_gaussiana} dimostreremo 
che questo integrale è pari a $\sqrt \pi$.
\end{example}

\begin{example}[la funzione $\Gamma$ Eulero]%
  \label{ex:gamma_eulero}%
  \index{$\Gamma$ funzione di Eulero}%
  \index{funzione!$\Gamma$ di Eulero}%
  Si definisce $\Gamma\colon (0,+\infty) \to \RR$ come
  \[
    \Gamma(x) = \int_0^{+\infty} e^{-t} t^{x-1}\, dt.
  \]
  L'integrale converge per ogni $x>0$ in quanto posto
  $f(t) = e^{-t} t^{x-1}$ per $t\to 0^+$ si ha $f(t) \sim t^{x-1}$ che ha
  integrale convergente in un intorno di $0$ mentre per $t\to +\infty$ si ha
  $f(t) \ll e^{-t/2}$ che ha integrale convergente in un intorno di $+\infty$.
  
  Integrando per parti si ottiene una interessante proprietà della funzione $\Gamma$:
  \begin{align*}
  \int_0^{+\infty} e^{-t}t^{x}\, dt
  &= \Enclose{-e^{-t}t^{x}}_0^{+\infty}
  + \int_0^{+\infty} e^{-t}xt^{x-1}\, dt\\
  &= x\int_0^{+\infty} e^{-t}t^{x-1}\, dt
  \end{align*}
  cioè $\Gamma(x+1) = x\Gamma(x)$.
  Osservando che $\Gamma(1)=1=0!$ si può quindi dimostrare, per induzione,
  che $\Gamma(n+1) = n!$ per ogni $n\in \NN$.
  Abbiamo quindi trovato una funzione che estende il fattoriale
  da $\NN$ a tutto l'intervallo di numeri reali $(-1,+\infty)$.
  
  Nel capitolo~\ref{sec:scambio_integrale_limite} dimostreremo 
  che la funzione $\Gamma$ è derivabile.
  
  Non siamo ora in grado di calcolare il valore di $\Gamma(x)$ se $x$ 
  non è intero, ma possiamo ricondurre il valore di $\Gamma(1/2)$ 
  all'integrale della gaussiana, facendo il cambio 
  di variabile $t=s^2$, $dt=2s ds$:
  \begin{align*}
  \Gamma(1/2) 
  &= \int_0^{+\infty} \frac{e^{-t}}{\sqrt t}\, dt  
  = \int_0^{+\infty} \frac{e^{-s^2}}{s} \cdot 2s\, ds
  = \int_{-\infty}^{+\infty} e^{-s^2}\, ds.
  \end{align*}
  Nell'esercizio~\ref{ex:integrale_gaussiana} troveremo il valore 
  dell'integrale della gaussiana e potremo 
  quindi concludere che $\Gamma(1/2) = \sqrt \pi$.
  Questo ci permette di calcolare il valore di $\Gamma$ su 
  tutti i semi-interi:
  \[
  \Gamma(3/2) = \frac 1 2 \Gamma(1/2) = \frac{\sqrt \pi}{2}, \qquad
  \Gamma(5/2) = \frac 3 2 \Gamma(3/2) = \frac{3\sqrt \pi}{4}, \dots 
  \]
\end{example}
    
\begin{example}
L'integrale
\[
  \int_{-\infty}^{+\infty}\frac{1}{1+x^2}\, dx
\]
è convergente in quanto per $x\to +\infty$ risulta $1/(1+x^2) \sim 1/x^2$.

L'integrale
\[
  \int_1^{+\infty} \frac{1}{\ln x}\, dx
\]
è divergente perché per $x\to +\infty$
si ha $1/\ln x \gg 1/x$ e per $x\to 1^+$ si ha
$1/\ln x \sim 1/(x-1)$. Per $p=1$ gli integrali
di $1/x$ a $+\infty$ e di $1/(x-1)$ in $1$
sono entrambi divergenti.

Se $f\colon [a,+\infty)$ è una funzione localmente Riemann-integrabile
e se $f(x)\to \ell \neq 0$ per $x\to +\infty$ allora l'integrale
di $f$ è divergente. Se $\ell>0$ esisterà $c>a$ tale che per $x>c$
si abbia $f(x)\ge 0$ (permanenza del segno). Visto che
$f(x)\sim \ell$ per $x\to +\infty$ possiamo quindi concludere
che $\int_c^{+\infty} f(x) = +\infty$ e
quindi anche
$\int_a^{+\infty} f(x) = +\infty$.
Se $\ell<0$ si può cambiare segno ad $f$ e ripetere l'argomento precedente
si scopre quindi che in questo caso l'integrale di $f$ è $-\infty$.
\end{example}

\begin{exercise}
Si mostri con un esempio che se una funzione $f$ ha integrale
convergente sull'intervallo $[0,+\infty)$ non
è detto che $f(x)\to 0$ per $x\to +\infty$.
\end{exercise}

% \begin{exercise}[difficile]
% Mostrare che l'integrale
% \[
%   \int_0^{+\infty} \exp\enclose{x^2 \ln \frac{2+\cos(2\pi x)}{3}} \, dx
% \]
% è finito. Ma la funzione integranda non tende a zero per $x\to +\infty$.
% \end{exercise}

\begin{exercise}
Dire se i seguenti integrali sono convergenti:
\[
  \int_0^{+\infty} \frac{1}{(x^2-1)\cdot \ln^2 x}\, dx, \qquad
  \int_0^{+\infty} \frac{\ln^2 x}{x^2-1}\, dx.  
\]
\end{exercise}

\begin{exercise}
Determinare i valori di $p\in \RR$ per i quali 
il seguente integrale è convergente:
  \[
  \int_0^{+\infty} \frac{\sqrt x-\ln x}{(x-\sin x)^p}\, dx.  
  \]
\end{exercise}


\begin{theorem}[criterio di convergenza assoluta]
\label{th:convergenza_assoluta_integrale}
\mymark{**}%
Sia $f\colon (a,b)\to \RR$ una funzione localmente Riemann-integrabile.
Allora anche $\abs{f}$ è localmente Riemann-integrabile e se
\[
\int_a^b \abs{f(x)}\, dx < +\infty
\]
(cioè $\abs{f}$ è integrabile in senso improprio)
allora anche $f$ è integrabile in senso improprio e
\[
  \abs{\int_a^b f(x)\,dx}
  \le \int_a^b \abs{f(x)}\, dx < +\infty.
\]
\end{theorem}
%
%
\begin{proof}
Poniamo $f = f^+ - f^-$ con $f^+\ge 0$ e $f^-\ge 0$.
Se $f$ è localmente Riemann-integrabile
$f^+$ e $f^-$ sono localmente Riemann-integrabili
(grazie al teorema~\ref{th:reticolo}).
Osserviamo che si ha $f = f^+ - f^-$
e $\abs{f} = f^+ + f^-$.
Visto che $0\le f^+ \le \abs{f}$ possiamo
applicare i criteri di confronto e
affermare che $f^+$ ha integrale convergente.
Lo stesso vale per $f^-$ e dunque per $f=f^+ - f^-$.
Inoltre
\[
  \abs{\int_a^b f }
  = \abs{\int_a^b f^+ - \int_a^b f^-}
  \le \int_a^b f^+ + \int_a^b f^-
  = \int_a^b \abs{f}.
\]
\end{proof}

\begin{definition}[convergenza assoluta]
\index{integrale!improprio!convergenza assoluta}%
\index{integrale!convergenza assoluta}%
\index{convergenza!assoluta!integrale}%
\index{integrabile!assolutamente}%
\index{assolutamente!integrabile}%
Quando
\[
  \int_a^b \abs{f(x)}\, dx <+\infty
\]
diremo che l'integrale di $f$ è \emph{assolutamente convergente}
o che $f$ è \emph{assolutamente integrabile} (in senso improprio).
Il teorema
precedente ci garantisce che una funzione assolutamente integrabile
(se localmente Riemann-integrabile) è integrabile.
\end{definition}

\begin{example}
L'integrale
\[
  \int_0^{+\infty} \sin(x^2)\cdot e^{-x}\, dx
\]
è assolutamente convergente (e quindi convergente)
in quanto la funzione integranda è continua,
e si ha
\[
 \abs{\sin(x^2)\cdot e^{-x}} \le e^{-x}
\]
da cui, per confronto,
\[
 \int \abs{\sin(x^2)\cdot e^{-x}}\, dx  \le
  \int_0^{+\infty} e^{-x}\, dx < +\infty.
\]
\end{example}

\begin{example}[funzione integrabile ma non assolutamente]
\label{ex:48864}%
\mymark{*}%
Vogliamo mostrare che l'integrale
\[
  \int_\pi^{+\infty} \frac{\sin x}{x}\, dx  
\]
è convergente ma non assolutamente convergente.

Un modo per farlo è quello di ricondursi alle serie.
Posto 
\[
  a_k = \int_{k\pi}^{(k+1)\pi} \frac{\sin x}{x}\, dx
\]
si ha
\[
  \abs{a_k} = \int_{k\pi}^{(k+1)\pi} \frac{\abs{\sin x}}{x}\, dx
\]
e dunque 
\[
\int_\pi^{+\infty} \frac{\sin x}{x}\, dx
= \sum_{k=1}^{+\infty} a_k, \qquad
\int_\pi^{n\pi} \frac{\abs{\sin x}}{x}\, dx
= \sum_{k=1}^{n-1} \abs{a_k}.
\]



Si osservi che 
\[
   \int_\pi^{n\pi} \frac{\sin x}{x}\, dx
   = \sum_{k=1}^{n-1} \int_{k\pi}^{(k+1)\pi} \frac{\sin x}{x}\, dx
\]
e che 



Integrando per parti,
\begin{align*}
  \int_1^{+\infty} \frac{\sin x}{x}\, dx
  &= \Enclose{\frac{-\cos x}{x}}_1^{+\infty} -
  \int_1^{+\infty} \frac{\cos x}{x^2} \, dx \\
  &= \cos 1 - \int_1^{+\infty} \frac{\cos x}{x^2}\, dx.
\end{align*}
Osserviamo ora che la funzione $\frac{\cos x}{x^2}$ è integrabile
per il criterio della convergenza assoluta, in quanto:
\[
  \abs{\frac{\cos x}{x^2}} \le \frac{1}{x^2}
\]
che è integrabile su $[1,+\infty)$.
Dunque la nostra funzione ha integrale convergente anche in $[1,+\infty)$
e dunque ha integrale convergente su tutto $(0,+\infty)$.
Nell'esercizio~\ref{ex:integrale_sin_integrale} vedremo che
$\int_0^{+\infty} \frac{\sin x}{x} = \frac \pi 2$.

Tale funzione non è però integrabile assolutamente.
Infatti si ha,
per ogni $k\in \NN$
\begin{align*}
  \int_{k\pi}^{(k+1)\pi} \abs{\frac{\sin x} x}\, dx
  &\ge \int_{k\pi}^{(k+1)\pi} \frac{\abs{\sin x}}{(k+1)\pi}\, dx\\
  &= \frac{\int_0^\pi \sin x \, dx}{(k+1)\pi}
  = \frac{2}{(k+1)\pi}
\end{align*}
da cui
\begin{align*}
  \int_0^{+\infty} \abs{\frac{\sin x}{x}}\, dx
  & = \lim_{b\to +\infty} \int_0^b \abs{\frac{\sin x}{x}}\, dx 
  \ge \lim_{b\to+\infty} \sum_{k=0}^{\lfloor \frac{b}{\pi}\rfloor-1} \int_{k\pi}^{(k+1)\pi}\abs{\frac{\sin x} x}\, dx\\
  &= \sum_{k=0}^{+\infty} \frac{2}{(k+1)\pi} = +\infty.
\end{align*}
\end{example}

\begin{exercise}
  Si provi ad applicare il metodo utilizzato nell'esempio \ref{ex:48864} ad 
  un generico integrale della forma $\int_1^{+\infty} f\cdot g$ identificando le ipotesi 
  su $f$ e su $g$ che garantiscono la convergenza dell'integrale.
  Si dovrebbe ottenere un criterio analogo al teorema~\ref{th:dirichlet} di Dirichlet
  per la convergenza delle serie.
\end{exercise}


I criteri che abbiamo visto fin'ora mettono in evidenza una notevole
analogia tra serie e integrali. Determinare la convergenza
di un integrale è però, in generale, più semplice che determinare
la convergenza della corrispondente serie.
Nell'esercizio seguente, ad esempio, se $\alpha=1$ è sufficiente trovare una primitiva
delle funzioni integrande, per determinare il carattere dell'integrale.

\begin{exercise}
Si determini per quali $\alpha,\beta\in \RR$ risultano convergenti
gli integrali:
\[
  \int_0^{\frac 1 2} \frac{1}{x^\alpha \abs{\ln x}^\beta}\, dx,
  \qquad
  \int_2^{+\infty} \frac{1}{x^\alpha \ln^\beta x} \, dx.
\]
\end{exercise}

\begin{theorem}[collegamento tra serie ed integrali impropri]
\mymark{**}
Sia $f\colon$ $[1,+\infty)\to \RR$ una funzione decrescente non negativa
e sia $a_k=f(k)$.
Allora la serie
\[
   \sum_{k=1}^{+\infty} a_k
\]
è convergente se e solo se è convergente l'integrale
\[
  \int_1^{+\infty} f(x)\, dx.
\]

E, più precisamente, per ogni $n\in \NN\cup\ENCLOSE{+\infty}$ si ha
\begin{equation}\label{eq:39185}
  0
  \le \sum_{k=1}^{n} a_k - \int_1^{n}f(x)\, dx
  \le a_1.
\end{equation}

\end{theorem}
%
\begin{proof}
\mymark{**}
Visto che la funzione $f$, è non negativa e monotona,
l'integrale improprio di $f$ su $[1,+\infty)$ esiste (finito o infinito)
per il teorema~\ref{th:integrabilita_positive}.
Anche la serie $\sum f(k)$ essendo a termini non negativi risulta essere determinata.

Essendo $f$ decrescente, per ogni $x \in [k,k+1]$ si ha
\[
  f(k+1) \le f(x) \le f(k)
\]
integrando su $[k,k+1]$ si ottiene
\[
  f(k+1) \le \int_{k}^{k+1} f(x)\, dx \le f(k)
\]
e sommando per $k=1,\dots, n-1$ si ottiene:
\[
  \sum_{k=2}^{n} f(k) \le \int_{1}^{n} f(x)\, dx \le \sum_{k=1}^{n-1} f(k)
\]
che implica la~\eqref{eq:39185} se $n\in \NN$. Facendo
tendere $n\to+\infty$ la ~\eqref{eq:39185} rimane comunque verificata.
La prima disequazione ci permette di stimare la serie con l'integrale
e la seconda ci permette viceversa di stimare l'integrale con la serie.
Dunque se uno dei due converge anche l'altro converge, come volevamo dimostrare.
\end{proof}

\begin{example}[stima asintotica di $\ln n!$]
\label{ex:498124}%
\index{fattoriale!stima asintotica}%
\mymargin{stima asintotica $\ln n" $}%
Osserviamo che
\[
  \ln (n!) = \ln \prod_{k=1}^n k = \sum_{k=1}^n \ln k.
\]
Applichiamo lo stesso procedimento utilizzato nel teorema precedente alla
funzione $f(x) = \ln x$. Visto che il logaritmo è crescente
(invece che decrescente) otterremo delle stime rovesciate, ma analoghe.
 Ripetendo con attenzione i conti, si ottiene:
\[
    \int_1^n \ln(x)\, dx \le  \sum_{k=2}^n \ln k \le \int_2^{n+1} \ln(x) \, dx
\]
e ricordando che $\int \ln x = x \ln x - x$ si ottiene
\[
  n \ln n - n + 1 \le \ln(n!) \le (n+1) \ln (n+1) - n - 2\ln 2 + 1
\]
da cui, dividendo ambo i membri per $n \ln n$ e facendo tendere $n\to +\infty$ si trova
\[
 \frac{\ln n!}{n \ln n}\to 1
\]
ovvero
\[
  \ln (n!) \sim n \ln n \qquad \text{per $n\to +\infty$.}
\]

Si faccia attenzione che in generale non possiamo fare l'esponenziale di ambo i membri di una equivalenza asintotica
e sperare che l'equivalenza asintotica rimanga valida.
In effetti trovare una stima asintotica per $n!$ è molto più complicato (rispetto alla stima che abbiamo 
trovato per il suo logaritmo).
Lo faremo nel teorema~\ref{th:stirling} trovando la formula di Stirling.
Si veda anche l'esercizio~\ref{ex:7340098}.

\end{example}

