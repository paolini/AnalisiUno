\section{integrale di Riemann}

\begin{definition}[integrale di Riemann]%
\label{def:integrale}%
\mymark{***}%
\index{integrale!definizione}%
\index{definizione!di integrale}%
\index{Riemann!integrale di}%
Siano $a,b\in \RR$, $a \le b$.
Un insieme $P\subset [a,b]$ si dice essere una \emph{suddivisione di Riemann}%
\mymargin{suddivisione di Riemann}%
\index{suddivisione!di Riemann}
\index{Riemann!suddivisione di}
\index{partizione di Riemann}
\index{Riemann!partizione di}
dell'intervallo $[a,b]$ se $P$ è un insieme finito tale che $a,b\in P$.
In particolare $P$ si
potrà scrivere come
\[
 P = \ENCLOSE{ x_0, x_1, \dots, x_N}
\]
con
\[
  a = x_0 < x_1 < \dots < x_{N-1} < x_N = b.
\]

Sia $f\colon [a,b] \to \RR$ una funzione limitata.
Data una qualunque suddivisione $P$ di $[a,b]$ definiamo
rispettivamente le \emph{somme superiori} e le \emph{somme inferiori}
\mymargin{somme superiori/inferiori}%
\index{somme superiori/inferiori}
\index{Riemann!somme superiori}
\index{Riemann!somme inferiori}
come
\begin{align*}
S^*(f,P)
&= \sum_{k=1}^N (x_k - x_{k-1}) \cdot \sup f([x_{k-1},x_k]) \\
S_*(f,P)
&= \sum_{k=1}^N (x_k - x_{k-1}) \cdot \inf f([x_{k-1},x_k]).
\end{align*}
Definiamo infine
\begin{align*}
  I^*(f) &= \inf \ENCLOSE{S^*(f,P) \colon \text{$P$ suddivisione di $[a,b]$}}
  \\
  I_*(f) &= \sup \ENCLOSE{S_*(f,P) \colon \text{$P$ suddivisione di $[a,b]$}}.
\end{align*}

Se $I^*(f) = I_*(f)$ diremo che $f$ è
\emph{Riemann-integrabile}
\mymargin{integrale di Riemann}%
\index{integrale!di Riemann}%
\index{Riemann!integrale di}%
\index{integrabilità}%
\index{integrale}%
\index{integrale!definizione}%
e diremo che l'\emph{integrale} di $f$ su $[a,b]$ è
il valore comune $I^*(f)=I_*(f)$ che verrà denotato con
\[
  \int_a^b f
  \qquad{\text{oppure con}} \qquad
  \int_a^b f(x)\, dx.
\]

Se $b<a$ e se $f$ è Riemann-integrabile su $[b,a]$
definiamo per convenzione:
\[
  \int_a^b f = -\int_b^a f.
\]
\end{definition}

\begin{lemma}
Sia $f\colon[a,b]\to \RR$ una funzione limitata.
Se $P$ e $Q$ sono due suddivisioni qualunque dell'intervallo
$[a,b]$ si ha
\begin{equation}\label{eq:5112309}
  S_*(f,P) \le S_*(f,P\cup Q)\le S^*(f,P\cup Q) \le S^*(f,Q).
\end{equation}
Di conseguenza $I_*(f) \le I^*(f)$.
\end{lemma}
\begin{proof}
  Sia $P$ una qualunque suddivisione di $[a,b]$ e sia $y\in [a,b]$ un punto qualunque. Posto $P' = P \cup \ENCLOSE{y}$ vogliamo mostrare
  che si ha
  \begin{equation}\label{eq:39543}
    S_*(f,P) \le S_*(f,P') \le S^*(f,P') \le S^*(f,P).
  \end{equation}
  Se $y\in P$ non c'è niente da dimostrare in quanto
  risulterebbe $P'=P$ e la disuguaglianza $S_*(f,P') \le S^*(f,P')$ è sempre verificata in quanto ogni estremo superiore che compare nella definizione di $S^*$ è maggiore o uguale al corrispondente
  estremo inferiore che compare nella definizione di $S_*$.
  Supponiamo allora che $y \not \in P$ e dunque che $y$ sia compreso tra due punti consecutivi $x_{k-1}, x_k$ della suddivisione $P$:
  \[
    a= x_0 < x_1 < \dots < x_{k-1} < y < x_k < \dots < x_N=b.
  \]
  Allora le somme che definiscono $S_*(f,P)$ e $S_*(f,P')$ 
  differiscono solo sull'intervallo $[x_{k-1},x_k]$. 
  % e si ha
  % \begin{align*}
  %   S_*(f,P') - S_*(f,P)
  %  &= (y-x_{k-1})\cdot \!\!\inf_{[x_{k-1},y]}\!\!\! f
  %  + (x_k - y)\cdot\! \inf_{[y,x_k]}\! f\\
  %  &\quad - (x_k - x_{k-1})\cdot \!\!\inf_{[x_{k-1}, x_k]}\!\!\!f
  % \end{align*}
  Ma osservando che
  \[
  \inf f([x_{k-1}, x_k])
  \le \inf f([x_{k-1},y])
  \qquad \text{e} \qquad
  \inf f([x_{k-1}, x_k])
  \le \inf f([y,x_k])
  \]
  si ottiene 
  \begin{align*}
    (x_k - x_{k-1})\cdot \inf f([x_{k-1}, x_k])
    & = (y - x_{k-1} + x_k - y)\cdot \inf f([x_{k-1}, x_k]) \\
    & \le (y-x_{k-1})\cdot \inf f([x_{k-1},y]) + (x_k - y)\cdot \inf f([y,x_k])
  \end{align*}
  e dunque $S_*(f,P) \le S_*(f,P')$.
  In maniera analoga si ottiene $S^*(f,P) \ge S^*(f,P')$.
  Dunque \eqref{eq:39543} è dimostrata.
  Ma allora se $P$ e $Q$ sono suddivisioni qualunque osserviamo che $P\cup Q$ 
  si può ottenere da $P$ aggiungendo uno alla volta i punti di $Q$. 
  Iterando la \eqref{eq:39543} si ottiene \eqref{eq:5112309}.
  Facendo l'estremo inferiore al variare di $Q$
  si ottiene $S_*(f,P) \le I^*(f)$ e facendo l'estremo superiore al variare 
  di $P$ si ottiene $I_*(f) \le I^*(f)$.
  \end{proof}


\begin{theorem}[criteri di integrabilità]
\label{th:criteri_integrabilita}%
\mymark{*}%
\mymargin{criteri di integrabilità}%
\index{criterio!di integrabilità}%
Sia $f\colon[a,b]\to \RR$ una funzione limitata.
\begin{enumerate}
\item
La funzione $f$ è Riemann-integrabile se e solo se
per ogni $\eps>0$ esiste una suddivisione $P$
tale che
\[
  S^*(f,P) - S_*(f,P) < \eps.
\]

\item
Se $f$ è Riemann-integrabile su $[a,b]$ allora
esiste una successione $P_n$ di suddivisioni tali che
\begin{equation}\label{eq:93765}
  \lim_{n\to +\infty} S^*(f,P_n)
  = \lim_{n\to+\infty} S_*(f,P_n)
  = \int_a^b f.
\end{equation}
Viceversa se $f$ è limitata ed esiste una successione $P_n$ 
di suddivisioni di $[a,b]$ per cui si ha
\begin{equation*}
  \lim_{n\to +\infty} \enclose{S^*(f,P_n) - S_*(f,P_n)} = 0
\end{equation*}
allora la funzione $f$ è Riemann-integrabile e vale~\eqref{eq:93765}.
\end{enumerate}
\end{theorem}
%
\begin{proof}
Se esiste una suddivisione $P$ tale che $S^*(f,P)-S_*(f,P) < \eps$ possiamo immediatamente concludere che
\[
I^*(f) - I_*(f) \le S^*(f,P) - S_*(f,P) < \eps.
\]
Se questo è vero per ogni $\eps >0$ deduciamo che $I^*(f) - I_*(f) = 0$ e dunque che $f$ è Riemann-integrabile.

Viceversa qualunque sia $f$, per le proprietà
di $\sup$ e $\inf$
esistono $Q$ e $R$ suddivisioni tali che
\[
  I^*(f) \ge S^*(f,Q) - \frac\eps 2
  \qquad\text{e}\qquad
  I_*(f) \le S_*(f,R) + \frac\eps 2
\]
da cui, per il lemma precedente, ponendo $P=Q\cup R$
se $f$ è Riemann-integrabile
si ottiene
\begin{align*}
S^*(f,P)-S_*(f,P) 
&\le S^*(f,Q) - S_*(f,R) \\
&\le I^*(f) + \frac\eps 2 - \enclose{I_*(f) - \frac \eps 2} = \eps.
\end{align*}

Dimostriamo ora il secondo punto.
Supponiamo dapprima che $f$ sia Riemann-integrabile su $[a,b]$.
Allora per il punto precedente per ogni $n\in \NN$ ponendo $\eps=1/n$ possiamo trovare una suddivisione $P_n$ tale che
\[
  S^*(f,P_n) - S_*(f,P_n) < \frac 1 n
\]
da cui
\[
  I^*(f) \le S^*(f,P_n) \le S_*(f,P_n) + \frac 1 n
   \le I_*(f) + \frac 1 n
\]
perciò passando al limite per $n\to +\infty$,
essendo $I^*(f) = I_*(f) = \int_a^b f$ deve valere
\[
  \lim S^*(f,P_n) = \lim S_*(f,P_n) = \int_a^b f.
\]

Viceversa se
\[
 \lim_{n\to +\infty} S^*(f,P_n) - S_*(f,P_n) = 0
\]
per ogni $\eps>0$ esiste $n$ tale che
\[
  S^*(f,P_n) - S_*(f,P_n) < \eps.
\]
Per il punto precedente concludiamo che $f$ è Riemann-integrabile.
D'altra parte sappiamo che
\[
  S_*(f,P_n) \le I_*(f) = \int_a^b f = I^*(f) \le S^*(f,P_n)
\]
dunque se $S^*(f,P_n) - S_*(f,P_n) \to 0$ necessariamente
l'integrale coincide con i limiti di $S^*(f,P_n)$ e di
$S_*(f,P_n)$.
\end{proof}

Il seguente esercizio è stato svolto da Archimede (287 a.C. - 212 a.C.) 
per calcolare l'area di un settore di parabola (metodo di esaustione).

\begin{example}[calcolo dell'integrale mediante suddivisioni]%
\label{ex:integrale_quadrato}%
Mostriamo che per ogni $b>0$ la funzione $f(x)=x^2$ è Riemann-integrabile sull'intervallo $[0,b]$ e si ha
\[
 \int_0^b x^2\, dx = \frac{b^3}{3}.
\]
\end{example}
\begin{proof}
Consideriamo le suddivisioni \emph{equispaziate} dell'intervallo $[0,b]$, cioè dividiamo $[0,b]$ in $N$ intervalli ognuno di ampiezza $b/N$:
\[
P_N = \ENCLOSE{\frac{kb}{N}\colon k \in 0, 1, \dots, N}.
\]
Si ha
\[
  S^*(f,P_N) 
   = \sum_{k=1}^N \frac b N \sup f\enclose{\Enclose{\frac{k-1}{N} b,\frac k N b}}
   = \frac{b}{N} \sum_{k=1}^N \frac{k^2b^2}{N^2}
   = \frac{b^3}{N^3} \sum_{k=1}^N k^2.
\]
Ricordiamo ora che vale (si veda esercizio~\ref{ex:somma_quadrati}):
\[
  \sum_{k=1}^N k^2 = \frac{N(N+1)(2N+1)}{6} = \frac{2N^3+3N^2+N}{6}
  \sim \frac 1 3 N^3,\qquad \text{per $N\to +\infty$}.
\]
Dunque si ha
\[
  S^*(f,P_N) \to \frac {b^3} 3, \qquad \text{per $N\to +\infty$}.
\]
Analogamente si trova
\[
  S_*(f,P_N) 
  = \sum_{k=1}^N \frac{b}{N}\inf f\enclose{\Enclose{\frac{k-1} N b,\frac k N b}} 
  = \frac{b}{N}\sum_{k=1}^N \frac{(k-1)^2b^2}{N^2}
  = \frac{b^3}{N^3} \sum_{k=0}^{N-1} k^2
\]
e dunque 
\[
  S_*(f,P_N) \sim \frac {b^3} {N^3} \frac {(N-1)^3} 3 \to \frac {b^3} 3, \qquad \text{per $N\to +\infty$}.
\]
La dimostrazione si conclude quindi applicando
il criterio \eqref{eq:93765} del teorema~\ref{th:criteri_integrabilita} precedente.
\end{proof}

\begin{theorem}[integrale della costante]
\label{th:integrale_costante}
Se $f\colon[a,b]\to \RR$ è costante: $f(x) = c$ allora
$f$ è Riemann-integrabile e si ha
\[
  \int_a^b f = c\cdot (b-a).
\]
\end{theorem}
%
\begin{proof}
Visto che su ogni $A\subset [a,b]$ si ha
\[
  \sup_A f = \inf_A f = c
\]
è facile verificare che si ha
\[
  S^*(f,P) = S_*(f,P) = c\cdot (b-a)
\]
qualunque sia la suddivisione $P$ di $[a,b]$. Il risultato segue immediatamente.
\end{proof}

Non tutte le funzioni sono Riemann-integrabili come ci mostra il seguente esempio.
\begin{example}[funzione di Dirichlet]
\mymark{**}
\mymargin{funzione di Dirichlet}
\index{funzione!di Dirichlet}
\index{integrabilità!controesempio}
Sia $a<b$ e sia $f\colon[a,b]\to \RR$ la funzione definita da
\[
 f(x) =
 \begin{cases}
   1 & \text{se $x\in \QQ$}\\
   0 & \text{se $x\not \in \QQ$}.
 \end{cases}
\]
Allora $f$ non è Riemann-integrabile.
\end{example}
%
\begin{proof}
\mymark{*}
Sia $P=\ENCLOSE{x_0, x_1, \dots, x_N}$ con $a=x_0 < x_1 < \dots < x_N = b$
una qualunque suddivisione di $[a,b]$.
Allora basta osservare che, per la densità dei razionali (teorema~\ref{th:densita_frazioni}),
e degli irrazionali (esercizio~\ref{ex:densita_irrazionali}),
in qualunque intervallino $I=[x_{k-1}, x_k]$ sono presenti infiniti punti
razionali e infiniti punti irrazionali. Dunque $\sup f(I)=1$ e $\inf f(I)=0$ e di conseguenza
\begin{align*}
  S^*(f,P) &= \sum_{k=1}^N (x_k - x_{k-1})\cdot 1 = b-a \\
  S_*(f,P) &= \sum_{k=1}^N (x_k - x_{k-1})\cdot 0 = 0
\end{align*}
da cui $I^*(f) = b-a \neq 0 = I_*(f)$.
\end{proof}

%%%%%
