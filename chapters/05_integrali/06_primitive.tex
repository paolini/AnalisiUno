\section{calcolo delle primitive}

Come già detto utilizzeremo la notazione $\int f$ per indicare le primitive
della funzione $f$.
Ma invece di scrivere $F\in \int f$
per indicare che $F$ è una primitiva di $f$
potremo scrivere
più semplicemente ma con abuso di notazione $\int f = F$
ricordando (come facevamo con la notazione degli $o$-piccolo) che tale relazione
non è affatto simmetrica.
Tradizionalmente si scrive
\begin{equation}\label{eq:4137878}
  \int \cos x\, dx = \sin x + c 
\end{equation}
ma potremmo anche scrivere più semplicemente:
\mynote{%
Il modo di scrivere \eqref{eq:4137878} è solo apparentemente 
migliore di~\eqref{eq:489417} in quanto in alcuni casi non 
tutte le primitive posso essere determinate a meno di una costante
(si veda l'esempio~\ref{ex:primitive_non_connesso}).
Vale quindi la regola (valida in generale per tutte le notazioni che richiedono 
una interpretazione) di usare con cautela e cognizione di causa la notazione 
degli integrali indefiniti: bisogna essere consapevoli del significato di quello 
che scriviamo e dobbiamo utilizzare una notazione che sia comprensibile a chi ci ascolta.
}%
\begin{equation}\label{eq:489417}
  \int \cos x\, dx = \sin x.
\end{equation}

\begin{theorem}[primitive di alcune funzioni elementari]
\index{integrale!funzioni elementari}
Si ha
per ogni $\alpha \in \RR$, $\alpha\neq -1$
\begin{gather*}
\int x^\alpha\, dx \ni \frac{x^{\alpha+1}}{\alpha+1},
\qquad
\int \frac{1}{x}\, dx \ni \ln\abs{x}
\\
\int e^x \, dx \ni e^x,
\qquad
\int \cos x\, dx \ni \sin x,
\qquad
\int \sin x\, dx \ni -\cos x
\\
\int \cosh x \, dx \ni \sinh x,\qquad
\int \sinh x \, dx \ni \cosh x, \\
\int \frac{1}{1+x^2}\, dx \ni \arctg x, \qquad
\int \frac{1}{\sqrt{1-x^2}}\, dx \ni \arcsin x, \\
\int \frac{1}{\sqrt{x^2+1}}\, dx \ni \settsinh x = \ln\enclose{x+\sqrt{x^2+1}}\\
\int \frac{1}{\sqrt{x^2-1}}\, dx \ni \settcosh x = \ln\enclose{x+\sqrt{x^2-1}}.
\end{gather*}
\end{theorem}
%
\begin{proof}
E' sufficiente fare riferimento alla corrispondente tabella
delle derivate delle funzioni elementari (teorema~\ref{th:derivate_elementari}).
\end{proof}

\begin{theorem}[linearità dell'integrale indefinito]
Per ogni $\lambda,\mu \in \RR$ e se $f$, $g$ sono funzioni qualunque si ha:
\[
  \int \enclose{\lambda f + \mu g} \supset \lambda \int f + \mu \int g
\]
\end{theorem}
%
\begin{proof}
Ogni elemento dell'insieme che si trova sul lato destro
si scrive nella forma $\lambda F + \mu G$ con $F\in \int f$ e $G\in \int g$.
Dunque si ha $F'=f$ e $G'=g$ da cui
\[
  (\lambda F + \mu G)' = \lambda f + \mu g
\]
e quindi
\[
  \lambda F + \mu G \in \int (\lambda f + \mu g)
\]
come dovevamo dimostrare.
\end{proof}

\begin{example}\label{ex:234045}
  Calcolare $\int (2x-1)^2\, dx$.
\end{example}
\begin{proof}[Svolgimento.]
  Si ha 
  \[
  \int (x-1)^2\, dx = \int \enclose{4x^2 - 4x +1}\, dx 
  \ni 4\frac{x^3}{3} - 4 \frac {x^2}{2} + x
  = \frac 4 3 x^3 - 2 x^2 + x.
  \]
\end{proof}

\begin{theorem}[cambio di variabile negli integrali]
Valgono le seguenti proprietà:
\begin{enumerate}
\item
se $g\colon A \to \RR$ è derivabile e $f\colon g(A) \to \RR$
allora
\mymargin{sostituzione diretta}%
\index{sostituzione!diretta}%
\index{integrale!sostituzione diretta}%
\begin{equation}\label{eq:sostituzione_1}
  \int f(g(x)) g'(x)\, dx \supset
  \Enclose{\int f(y) \, dy}_{y=g(x)}
\end{equation}
dove si intende
\[
 \Enclose{F(y)}_{y=g(x)} = F(g(x));
\]

\item
se $g\in C^1([a,b])$ e $f\in C^0(g([a,b]))$ allora
\mymargin{cambio di variabile}%
\index{cambio!di variabile}%
\index{integrale!cambio di variabile}%
\begin{equation}\label{eq:sostituzione_2}
 \int_{g(a)}^{g(b)} f(x)\, dx = \int_a^b f(g(t))\, g'(t)\, dt;
\end{equation}

\item
se $I$ è un intervallo, $g\in C^1(I)$ è iniettiva e $f\in C^0(g(I))$
allora $g^{-1}$ è definita sull'intervallo $g(I)$
e si ha
\mymargin{sostituzione inversa}%
\index{sostituzione!inversa}%
\index{integrale!sostituzione inversa}%
\begin{equation}\label{eq:239455}
  \int f(x)\, dx \supset \Enclose{\int f(g(t)) g'(t)\, dt}_{t=g^{-1}(x)}
\end{equation}
\end{enumerate}
\end{theorem}
%
\begin{proof}
Per~\eqref{eq:sostituzione_1}
basta osservare che se $F(y)$ è una qualunque primitiva di $f(y)$ 
allora $F(g(x))$ è una primitiva di $f(g(x))\cdot g'(x)$
(basta applicare la formula di derivazione della funzione composta).
Dunque ogni elemento dell'insieme di destra in~\eqref{eq:sostituzione_1}
è elemento dell'insieme di sinistra.

Per dimostrare~\eqref{eq:sostituzione_2} sappiamo che $f$, essendo continua, ammette almeno una
primitiva $F(x)$. Per il punto precedente sappiamo che $F(g(t))$ è una
primitiva di $f(g(t))g'(t)$ (basta farne la derivata per verificarlo).
Dunque, utilizzando la formula fondamentale del calcolo, si ottiene:
\[
\int_{g(a)}^{g(b)}\!\! f(x) \, dx
= \Enclose{F(x)}_{g(a)}^{g(b)}
= F(g(b)) - F(g(a))
= \Enclose{F(g(t))}_a^b
=\int_a^b\! f(g(t))g'(t)\, dt.
\]

Per dimostrare~\eqref{eq:239455}
sia $F$ una qualunque funzione elemento dell'insieme sul lato destro della
relazione.
Si avrà $F(x) = H(g^{-1}(x))$ con $H(t)$ primitiva
di $f(g(t))g'(t)$. 
Ma allora, per la formula fondamentale del calcolo integrale,
fissato $t_0 \in I$ si ha, per~\eqref{eq:sostituzione_2},
\mynote{Se $g^{-1}$ fosse derivabile 
la dimostrazione sarebbe immediata, ma in generale 
può capitare che la derivata di $g$ si annulli in 
qualche punto e quindi $g^{-1}$ potrebbe non 
essere derivabile.}%
\begin{align*}
  F(x) - F(g(t_0))
  &= H(g^{-1}(x)) - H(t_0)
  = \int_{t_0}^{g^{-1}(x)} f(g(t)) g'(t)\, dt
  = \int_{g(t_0)}^x f(s)\, ds
\end{align*}
da cui, per il teorema~\ref{th:torricelli-barrow}, 
possiamo fare la derivata di 
ambo i membri per ottenere
$F'(x) = f(x)$. 
Dunque la funzione $F$ è elemento dell'insieme a 
sinistra della relazione~\eqref{eq:239455},
come volevamo dimostrare.
\end{proof}

Le formule del teorema precedente si scrivono usualmente nella forma
\[
  \int f(g(x)) g'(x)\, dx = \int f(y) \, dy
\]
dove si intende che le variabili $x$ e $y$ devono soddisfare la relazione
$y=g(x)$ (o, viceversa, $x=g^{-1}(y)$).
Per memorizzare tale formula si usa normalmente definire il
\emph{differenziale} di una funzione $g$ come $dg(x) = g'(x)\, dx$
(coerentemente con la notazione $g' = dg / dx$)
cosicché se $y=g(x)$ si ha $dy = g'(x)\, dx$.
Non daremo qui una definizione formale di cosa sia un differenziale
ma senz'altro utilizzeremo questa comoda notazione, pensandola
semplicemente come una facilitazione tipografica.

\begin{example}
Calcolare $\int (2x-1)^2\, dx$.
\end{example}
\begin{proof}[Svolgimento.]
Possiamo fare il cambio di variabile $y=2x-1$, $dy = 2\, dx$.
Dunque 
\begin{align*}
\int (2x-1)^2\, dx 
&= \frac 1 2 \int (2x-1)^2\, 2dx
\supset \frac 1 2 \Enclose{\int y^2\, dy}_{y=2x-1}
\ni \frac 1 2 \Enclose{\frac{y^3}{3}}_{y=2x-1}\\
&= \frac 1 6 \enclose{2x-1}^3 = \frac{8x^3 - 12x^3 + 6 x - 1}{6}
= \frac 4 3 x^3 - \frac 2 x^2 + 1 - \frac 1 6.
\end{align*}
Osserviamo che questa primitiva è diversa da quella trovata 
Nell'esempio~\ref{ex:234045} ma differisce da questa per una costante.
Dunque entrambi i risultati sono corretti.
\end{proof}
\begin{example}
Calcolare
$\int \cos^2 x \, dx$.
\end{example}
\begin{proof}[Svolgimento.]
Ricordando che $\cos(2t) = \cos^2 t - \sin^2 t = 2\cos^2 t - 1$ si
ha $\cos^2 t = \frac{1+\cos(2t)}2$ (formula di bisezione).
Dunque
\[
\int \cos^2 t\, dt
= \int\frac{1+\cos(2t)}{2}\, dt
= \int \frac 1 2 \, dt + \int \frac{\cos(2t)}{2}\, dt.
\]
Chiaramente $\int \frac 1 2 \, dt  = \frac t 2$.
Nel secondo integrale
possiamo fare un cambio di variabile, ponendo
$2t=s$ da cui $2dt = ds$:
\begin{align*}
\int \frac{\cos(2t)}{2}\, dt
&= \frac 1 4 \int \cos(2t)\, 2dt
= \frac 1 4 \Enclose{\int \cos s \, ds}_{s=2t}
= \frac 1 4 \Enclose{\sin s}_{s=2t}\\
&= \frac 1 4 \sin(2t)
= \frac{1}{2}\sin t \cdot \cos t.
\end{align*}
In definitiva otteniamo
\[
  \int \cos^2 x \, dx = \frac{t+\sin t \cdot \cos t}{2}.
\]
\end{proof}


\begin{example}
Calcolare
$\int \sqrt{1-x^2}\, dx$.
\end{example}
\begin{proof}
La funzione integranda è definita per $x\in [-1,1]$.
Ci viene in mente di operare la sostituzione $x=\sin t$
con $t\in [-\pi/2, \pi/2]$.
Osserviamo che su $[-\pi/2,\pi/2]$ la funzione $\sin t$ è derivabile,
invertibile e la sua inversa è $t = \arcsin x$.
Informalmente si ha
\[
 x= \sin t, \qquad dx = \cos t \, dt
\]
da cui si ottiene la formula
\[
 \int \sqrt{1-x^2}\, dx = \Enclose{\int \sqrt{1-\sin^2(t)} \cos t\, dt}_{t=\arcsin x}.
\]
Osserviamo ora che per $t\in [-\pi/2, \pi/2]$ risulta $\sqrt{1-\sin^2(t)}=\cos t$
e dunque l'integrale
diventa
\[
 \int \sqrt{1-\sin^2(t)}\cos t\, dt = \int \cos^2(t)\, dt.
\]
Quest'ultimo integrale lo abbiamo calcolato nell'esercizio precedente.
Dunque otteniamo:
\begin{align*}
 \int \sqrt{1-x^2}\, dx
 &\stackrel{(x=\sin t)}= \int \cos^2(t)\, dt
 = \frac{t + \sin t \cos t}{2} \\
 &= \frac{t + (\sin t) \sqrt{1-\sin^2 t}}{2} \\
 &\stackrel{(t=\arcsin x)}= \frac{\arcsin x + x \sqrt{1-x^2}}{2}.
\end{align*}
\end{proof}

Si osservi che il grafico $y=\sqrt{1-x^2}$ dell'esempio precedente
è una semicirconferenza e dunque l'integrale 
\[
 \int_{-1}^1 \sqrt{1-x^2}\, dx = \Enclose{\frac{\arcsin x +x\sqrt{1-x^2}}{2}}_{-1}^{1}
 = \frac{\pi}{2}.
\]
rappresenta l'area di un semicerchio unitario.

\begin{exercise}
Calcolare
\[
 \int \sqrt{1+x^2}\, dx, \qquad \int \sqrt{x^2-1}\, dx.  
\]

Suggerimento: in analogia con l'integrale di $\sqrt{1-x^2}$
di può effettuare la sostituzione $t=\sinh x$ nel primo 
integrale e $t=\cos h x$ nel secondo.
\end{exercise}

\begin{theorem}[integrazione per parti]
\mymark{*}%
\mymargin{integrale per parti}%
\index{integrale!per parti}%
Sia $f\colon A\subset \RR \to \RR$ una funzione qualunque, sia $g\colon A \to\RR$
una funzione derivabile
e sia $F \in \int f$.
Allora
\[
  \int f\cdot g \supset F \cdot g - \int F \cdot g'.
\]

In particolare se $f\in C^0([a,b])$ e $g\in C^1([a,b])$
e $F \in \int f$, si ha
\[
  \int_a^b f\cdot g = \Enclose{F\cdot g}_a^b - \int_a^b F \cdot g'.
\]
\end{theorem}
%
\begin{proof}
\mymark{*}
Ogni funzione dell'insieme di destra si scrive nella forma
$F\cdot g - H$ con $H \in \int F \cdot g'$.
Dunque $H' = F \cdot g'$ e, per ipotesi, $F'=f$ da cui
\[
(F\cdot g - H)' = F' \cdot g + F \cdot g' - H' = F' \cdot g
= f\cdot g
\]
che è quanto dovevamo dimostrare.

La seconda parte del teorema
deriva direttamente dalla formula fondamentale del calcolo integrale
(valida in quanto sia $f\cdot g$ che $F \cdot g'$ sono funzioni continue),
osservando che
\[
\Enclose{F\cdot g - \int F \cdot g'}_a^b
= \Enclose{F\cdot g}_a^b - \int_a^b F \cdot g'.
\]
\end{proof}

\begin{example}
Si voglia calcolare
\[
  \int x \cos x\, dx.
\]
Il metodo di integrazione per parti ci permette
di ricondurre l'integrale di un prodotto ad un integrale
di un prodotto in cui uno dei fattori viene integrato e l'altro derivato.
In questo caso sarà conveniente derivare il fattore $x$
e integrare il fattore $\cos x$ in modo da ricondursi all'integrale di
$1\cdot \sin x$, che sappiamo svolgere.
Precisamente si ha
\[
 \int x \cos x\, dx = x \sin x - \int 1 \cdot \sin x \, dx
  = x \sin x + \cos x.
\]
\end{example}

\begin{example}
  \label{ex:822309}
Si voglia calcolare
\[
 \int e^x \cos x\, dx.
\]
In questo caso se utilizziamo l'integrazione per parti possiamo ricondurre
questo integrale a $\int e^x \sin x$. Integrando ancora per parti ci si
ricondurrà nuovamente ad $\int e^x \cos x$. Se però in questi passaggi si
riottiene la quantità originale con un segno cambiato, si potrà risolvere
l'equazione ottenuta per trovare il risultato cercato.

Precisamente:
\begin{align*}
\int e^x \cos x\, dx
&= e^x \sin x - \int e^x \sin x\, dx\\
 &= e^x \sin x - \Enclose{e^x(-\cos x) - \int e^x(-\cos x)\, dx} \\
 &= e^x \sin x + e^x \cos x - \int e^x \cos x \, dx
\end{align*}
da cui:
\[
 2 \cdot \int e^x \cos x\, dx  = e^x \sin x + e^x \cos x
\]
ovvero
\[
  \int e^x \cos x\, dx = \frac{e^x(\sin x + \cos x)}{2}.
\]
\end{example}

\begin{theorem}[ancora primitive di funzioni elementari]
\mymark{***}
Si ha
\begin{align*}
  \int \ln x\, dx  &= x \ln x - x, \\
  \int \arctg x\, dx &= x \arctg x - \ln \sqrt{1+x^2},\\
  \int \arcsin x\, dx &= x \arcsin x + \sqrt{1-x^2},\\
  \int \arccos x\, dx &= x \arccos x - \sqrt{1-x^2}.
\end{align*}
\end{theorem}
%
\begin{proof}
\mymark{***}
L'idea, in tutti i casi, è che la derivata della funzione integranda
è molto più semplice della funzione stessa (nei primi due casi è una funzione 
razionale negli altri due una funzione irrazionale ma non trascendente).
Può quindi risultare utile applicare l'integrazione
per parti nella forma:
\[
  \int f(x)\, dx = \int 1\cdot f(x)\, dx = x f(x) - \int x f'(x)\, dx.
\]
Nel primo caso si ha:
\[
\int \ln x\, dx = x \ln x - \int x \frac{1}{x}\, dx
 = x \ln x - \int 1 dx = x \ln x - x.
\]
Nel secondo caso:
\[
\int \arctg x\, dx = x \arctg x - \int \frac{x}{1+x^2}\, dx.
\]
Operiamo quindi un cambio di variabile $y=1+x^2$:
\begin{align*}
\int \frac{x}{1+x^2}\, dx
&= \frac{1}{2}\int \frac{1}{1+x^2} 2x \, dx
\stackrel{y=1+x^2} = \frac{1}{2}\int \frac 1 y\, dy \\
&= \frac{\ln y}{2} = \frac{1}{2}\ln\enclose{1+x^2}
\end{align*}
da cui, in conclusione:
\[
 \int \arctg x\, dx = x \arctg x - \frac 1 2 \ln\enclose{1+x^2}.
\]
Nel terzo caso:
\[
 \int 1\cdot \arcsin x\, dx = x\cdot \arcsin x - \int\frac{x}{\sqrt{1-x^2}}\, dx  
\]
e operando la sostituzione $y=1-x^2$ si arriva a 
\[
  \int \arcsin x\, dx = x \cdot \arcsin x + \sqrt{1-x^2}.  
\]
L'ultimo integrale si svolge in maniera analoga.
\end{proof}

\begin{remark}
Se si applica il metodo di integrazione per parti nella ricerca della primitiva
della funzione $\frac{1}{x\ln x}$, integrando il fattore $\frac 1 x$ e
derivando il fattore $\frac 1 {\ln x}$ si ottiene un fenomeno a prima vista
sconcertante:
\begin{align*}
  \int \frac{1}{x\ln x}\,dx
  &= \ln x \cdot \frac{1}{\ln x} - \int \ln x \cdot \frac{\frac 1 x}{-\ln^2 x}\, dx\\
  &= 1 + \int \frac{1}{x\ln x}\, dx.
\end{align*}
Si potrebbe infatti pensare di poter semplificare gli integrali ai due lati
dell'uguaglianza per ottenere $0=1$.
Questo non si può fare perché, ricordiamolo, l'integrale indefinito è un insieme
di funzioni (le primitive della funzione $\frac{1}{x\ln x}$ in questo caso)
e sappiamo che sommando una costante ad una primitiva si ottiene un'altra primitiva.
Quindi non ci deve stupire il fatto che sommando $1$ all'insieme delle primitive
questo rimanga invariato.

Per la cronaca: l'integrale in questione può essere calcolato per sostituzione,
ponendo $y=\ln x$ trovando $F(x) = \ln\abs{\ln x}$ come una delle primitive.
\end{remark}
%
%
%
