\section{proprietà degli integrali}
%%%%%
\index{integrali!proprietà}%
\index{proprietà!degli integrali}%

\begin{theorem}[additività dell'integrale]
\mymark{*}%
\mymargin{additività dell'integrale}%
\index{additività dell'integrale}%
\index{integrale!additività}%
\label{th:additivita_integrale}%
Sia $f\colon [a,b]\to \RR$ una funzione limitata e sia $c\in [a,b]$.
Allora $f$ è Riemann-integrabile su $[a,b]$ se e solo se
$f$ è Riemann-integrabile su $[a,c]$ e su $[c,b]$.
E in tal caso risulta
\begin{equation}\label{eq:36645}
  \int_a^b f = \int_a^c f + \int_c^b f.
\end{equation}

In base alla convenzione
\[
    \int_b^a f = -\int_a^b f
\]
la formula \eqref{eq:36645} è valida non solo se $a\le c\le b$ ma anche
se $a,b,c$ sono in qualunque ordine, purché la funzione $f$ sia integrabile
sull'intervallo che contiene tutti e tre i punti $a,b,c$.
\end{theorem}
%
\begin{proof}
\mymark{*}
Supponiamo che $f$ sia integrabile su $[a,c]$ e su $[c,b]$.
Allora, in base ai criteri di integrabilità, per ogni $\eps>0$ esisteranno una
suddivisione $P$ di $[a,c]$ e una suddivisione $Q$ di $[c,b]$ tali che
\[
  S^*(f,P) - S_*(f,P) < \frac \eps 2,
  \qquad
  S^*(f,Q) - S_*(f,Q) < \frac \eps 2.
\]
L'insieme $R=P\cup Q$ risulta essere una suddivisione di $[a,b]$ su cui si avrà
\begin{equation}\label{eq:56632}
S^*(f,R) = S^*(f,P) + S^*(f,Q), \qquad
S_*(f,R) = S_*(f,P) + S_*(f,Q)
\end{equation}
e dunque
\[
S^*(f,R) - S_*(f,R) \le \frac \eps 2 + \frac \eps 2 = \eps.
\]
Applicando nuovamente il criterio di integrabilità in senso invertito otteniamo
dunque l'integrabilità di $f$ su $[a,b]$ e le equazioni
\eqref{eq:56632} garantiscono l'additività dell'integrale rispetto al dominio.

Viceversa se $f$ è integrabile su $[a,b]$ il criterio di integrabilità
ci garantisce che per ogni $\eps>0$ esiste una suddivisione $R$ di $[a,b]$ tale che
\[
S^*(f,R) - S_*(f,R) < \eps.
\]
Se ora consideriamo $R' = R \cup \ENCLOSE{c}$ sappiamo che $S^*(f,R') \le S^*(f,R)$
e $S_*(f,R') \ge S_*(f,R)$ dunque anche $R'$ soddisfa la proprietà
\[
S^*(f,R') - S_*(f,R') < \eps.
\]
Ma ora è chiaro che posto $P=R'\cap[a,c]$ e $Q=R'\cap[c,b]$ risulta che $P$
e $Q$ siano suddivisioni di $[a,c]$ e $[c,b]$ rispettivamente e che
\begin{align*}
  S^*(f,R') &= S^*(f,P) + S^*(f,Q), \\
  S_*(f,R') &= S_*(f,P) + S_*(f,Q).
\end{align*}
Dunque si ha
\begin{align*}
(S^*(f,P) - S_*(f,P)) + (S^*(f,Q) - S_*(f,Q))
&= S^*(f,R') - S_*(f,R')
< \eps.
\end{align*}
Visto che entrambi gli addendi $S^*-S_*$ sono non negativi
risulta che valgono separatamente le disuguaglianze
\[
S^*(f,P) - S_*(f,P) < \eps, \qquad
S^*(f,Q) - S_*(f,Q) < \eps.
\]
Dunque $f$ è integrabile sia su $[a,c]$ che su $[c,b]$.
E nuovamente possiamo osservare che l'integrale è additivo sul dominio.
\end{proof}  

\begin{theorem}[monotonia dell'integrale]%
\label{th:monotonia_integrale}%
\mymark{*}%
Sia $a\le b$ e siano
$f,g\colon [a,b]\to \RR$ due funzioni Riemann-integrabili.
Se per ogni $x\in [a,b]$ si ha $f(x) \le g(x)$ allora
\[
  \int_a^b f(x) \le \int_a^b g(x).
\]

In particolare se $f\ge 0$ allora $\int_a^b f \ge 0$.
\end{theorem}
%
\begin{proof}
Chiaramente se $f \le g$ si avrà che il $\sup$ di $f$ su qualunque intervallo sarà minore o uguale al $\sup$ di $g$ sullo stesso intervallo. Dunque su ogni suddivisione $P$ di $[a,b]$ si avrà:
\[
  S^*(f,P) \le S^*(g,P)
\]
da cui si ottiene immediatamente $I^*(f) \le I^*(g)$ e il risultato segue.
\end{proof}

\begin{theorem}[linearità dell'integrale]
\mymark{*}
\label{th:integrale_lineare}
Siano $f,g\colon [a,b]\to \RR$ due funzioni Riemann-integrabili
e siano $\lambda, \mu \in \RR$. Allora $\lambda f + \mu g$
è Riemann-integrabile e si ha
\[
  \int_a^b (\lambda f + \mu g) = \lambda \int_a^b f + \mu \int_a^b g.
\]

In particolare l'insieme delle funzioni Riemann-integrabili su $[a,b]$ risulta
essere uno spazio vettoriale reale e l'integrale è una
applicazione lineare su tale spazio, a valori in $\RR$.
\end{theorem}
%
\begin{proof}
\mymark{*}
Mostriamo innanzitutto che
\begin{equation}\label{eq:20043}
  \int_a^b (-f) = -\int_a^b f.
\end{equation}
Questo deriva dal fatto che su qualunque insieme $A$ si ha
$\sup_A (-f) = -\inf_A f$ e dunque per una qualunque suddivisione $P$
si ha
\[
  S^*(-f,P) = -S_*(f,P).
\]
Se ne deduce che $I^*(-f) = -I_*(f)$ e, analogamente, $I_*(-f) = -I^*(f)$.
Dunque se $f$ è Riemann-integrabile anche $-f$ lo è e vale la proprietà \eqref{eq:20043}.

Ora se $\lambda \ge 0$ vogliamo mostrare che vale
\begin{equation}\label{eq:10032}
  \int_a^b \lambda f = \lambda \int_a^b f.
\end{equation}
Semplicemente si osserva che $\sup_I \lambda f = \lambda \sup_I f$ e dunque
$S^*(\lambda f,P) = \lambda S^*(f,P)$ per ogni suddivisione $P$.
Ne consegue che $I^*(\lambda f) = \lambda I^*(f)$. In maniera analoga si può
mostrare che $I_*(\lambda f) = \lambda I_*(f)$. Dunque se $f$ è
Riemann-integrabile anche $\lambda f$ (con $\lambda \ge 0$) lo è e vale \eqref{eq:10032}.

Mettendo assieme \eqref{eq:20043} e $\eqref{eq:10032}$ si ottiene
che $\eqref{eq:10032}$ vale per ogni $\lambda \in \RR$.
Lo stesso sarà vero se mettiamo $g$ al posto di $f$ e $\mu$ al posto di $\lambda$.
Per concludere la dimostrazione sarà dunque sufficiente
mostrare che vale anche
\begin{equation}\label{eq:80003}
\int_a^b (f+g) = \int_a^b f + \int_a^b g.
\end{equation}
Osserviamo che su qualunque insieme $A$ si ha
\[
  \sup_A (f+g) \le \sup_A f + \sup_A g.
\]
Infatti per ogni $x\in A$ si ha $\sup_A f\ge f(x)$ 
e $\sup_A g \ge g(x)$ dunque $\sup_A f + \sup_A g$
è maggiore di $f(x)+g(x)$ per ogni $x\in A$ 
ed è quindi non inferiore a $\sup_A (f+g)$ che 
è il minimo dei maggioranti.
Questo significa che
\[
  S^*(f+g) \le S^*(f) + S^*(g).
\]
analogamente si potrà dimostrare che
\[
  S_*(f+g) \ge S_*(f) + S_*(g).
\]
Si ottiene dunque
\[
  I^*(f+g) \le I^*(f) + I^*(g)
  \qquad\text{e}\qquad
  I_*(f+g) \ge I_*(f) + I_*(g)
\]
e dunque se $f$ e $g$ sono integrabili anche $f+g$ risulta integrabile
e vale la \eqref{eq:80003}.

Per concludere che l'insieme delle funzioni integrali sia uno spazio vettoriale
è sufficiente osservare che, grazie al teorema~\ref{th:integrale_costante},
la funzione $0$ risulta integrabile.
\end{proof}

\begin{definition}\label{def:osc}
Sia $A\subset \RR$. Definiamo 
l'\emph{oscillazione} di $A$ come:
\[
  \osc A \defeq \sup A - \inf A.
\]
Nel caso $A=f(D)$ sia l'immagine di una funzione 
definiamo l'\emph{oscillazione} di $f$ sull'insieme 
$D$ come l'oscillazione della sua immagine, ovvero:
\[
   \osc_{x\in D} f(x) 
   \defeq \osc f(D) 
   = \sup_{x\in D} f(x) - \inf_{x\in D} f(x).  
\]
\end{definition}

\begin{theorem}[integrabilità delle funzioni continue]
  \mymark{***}%
  \label{th:integrabilita_continue}%
  \mymargin{integrabilità delle funzioni continue}%
  \index{integrabilità!funzioni continue}%
  Sia $f\colon [a,b]\to \RR$ una funzione continua.
  Allora $f$ è limitata e Riemann-integrabile.
  \end{theorem}
  %
  \begin{proof}
  \mymark{***}
  Per il teorema di Weierstrass sappiamo che $f$ è limitata.
  Per il teorema di Heine-Cantor sappiamo che $f$ è uniformemente continua,
  dunque per ogni $\eps>0$ esiste un $\delta>0$ tale che
  \[
   \abs{x-y} < \delta \implies \abs{f(x)-f(y)} < \eps.
  \]
  Possiamo allora considerare una suddivisione $P_\delta$ con la proprietà che
  gli intervalli individuati dalla suddivisione abbiano tutti ampiezza minore di
  $\delta$ (ad esempio potremmo prendere la suddivisione formata da
  $(b-a)/\delta+2$ punti equispaziati in $[a,b]$). Su ogni intervallo $I$ di tale
  suddivisione si avrà che se $x,y\in I$ allora $\abs{f(x)-f(y)} < \eps$ da cui
  si deduce $\sup_I f - \inf_I f \le \eps$.
  In particolare, sommando su tutti gli intervalli, si avrà
  \begin{align*}
    S^*(f,P_\delta) - S_*(f,P_\delta)
    &= \sum_{k=1}^N (x_k-x_{k-1})\enclose{\sup_{[x_{k-1},x_k]} f - \inf_{[x_{k-1},x_k]} f} \\
    &\le \eps \sum_{k=1}^N (x_k - x_{k-1})
     = \eps (b-a).
  \end{align*}
  Visto che questa quantità può essere resa arbitrariamente piccola per
  $\eps \to 0$, in base ai criteri di integrabilità possiamo concludere che la
  funzione $f$ è integrabile.
  \end{proof}
  
  \begin{theorem}[integrabilità delle funzioni monotone]
  \label{th:integrabilita_monotone}%
  \mymargin{integrabilità delle funzioni monotone}%
  \index{integrabilità!funzioni monotone}%
  Sia $f\colon [a,b]\to \RR$ una funzione monotona. Allora $f$ è limitata e
  Riemann-integrabile.
  \end{theorem}
  %
  \begin{proof}
  Supponiamo, per fissare le idee, che $f$ sia crescente.
  
  Chiaramente $f$ è limitata in quanto $f(a) \le f(x) \le f(b)$ per ogni
  $x\in [a,b]$.
  
  Per avere l'integrabilità e sufficiente mostrare
  che esiste una successione di suddivisioni $P_n$
  tale che $S^*(f,P_n) - S_*(f,P_n) \to 0$.
  Consideriamo la suddivisione equispaziata
  $P_n=\ENCLOSE{x_k \colon k=0,1, \dots, n}$ con $x_k=a+k(b-a)/N$.
  In tal caso su ogni intervallino $[x_{k-1},x_k]$ si ha
  \[
    \sup f([x_{k-1}, x_k]) = f(x_k),
    \qquad
    \inf f([x_{k-1}, x_k]) = f(x_{k-1}).
  \]
  Dunque la differenza tra le somme superiori
  e le somme inferiori è telescopica
  e si ha, per $n\to +\infty$
  \begin{align*}
  S^*(f,P) - S_*(f,P)
  &= \sum_{k=1}^n \frac{b-a}{n} f(x_k)
    - \sum_{k=1}^n \frac{b-a}{n} f(x_{k-1}) \\
  &= \frac{b-a}{n}(f(b)-f(a)) \to 0.
  \end{align*}
  E' quanto volevamo dimostrare.
  \end{proof}
  
  \begin{example}[funzione di Heaviside]
  \label{ex:heaviside}
  \mymargin{funzione di Heaviside}
  \index{funzione!di Heaviside}
  Sia $a<0<b$.
  La funzione $H\colon [a,b] \to \RR$ definita da
  \[
  H(x) =
  \begin{cases}
  1 & \text{se $x\ge 0$}\\
  0 & \text{se $x< 0$}
  \end{cases}
  \]
  è crescente quindi integrabile.
  \end{example}
  
\begin{theorem}[integrabilità della funzione composta]
  \label{th:composta_integrabile}%
  \index{integrabilità!della funzione composta}%
  \index{composizione!integrabilità}%
  Sia $\phi\colon [\alpha,\beta]\to \RR$ una funzione 
  continua e sia $f\colon [a,b]\to [\alpha,\beta]$ una 
  funzione Riemann-integrabile.
  Allora la funzione composta 
  $\phi\circ f\colon[a,b]\to \RR$ è 
  limitata e Riemann-integrabile.
\end{theorem}
%
\begin{proof}
Per i criteri di integrabilità, teorema~\ref{th:criteri_integrabilita},
fissato $\eps>0$ basta trovare una suddivisione $P$
per cui si abbia $S^*(\phi\circ f,P) - S_*(\phi\circ f,P)\le \eps(b-a)$.
Per il teorema~\ref{th:heine_cantor} di Heine-Cantor 
la funzione $\phi$ è uniformemente continua su $[\alpha,\beta]$ 
e quindi esiste $\delta>0$ tale che se $\abs{y_1-y_2}< \delta$
allora $\abs{\phi(y_1)-\phi(y_2)}< \eps$. 
Per il teorema~\ref{th:weierstrass} di Weierstrass, invece, 
esiste $M$ tale che $\abs{\phi(y)}\le M$ per ogni $y\in [\alpha,\beta]$.
In particolare questo garantisce che la funzione composta $\phi \circ f$ 
sia limitata: $\phi(f(x))\le M$ per ogni $x\in [a,b]$.
Scelto $\sigma = \frac{\eps \delta}{4M} (b-a)$ 
(vedremo alla fine perché si sceglie così)
per l'integrabilità di $f$ sappiamo che 
esiste una suddivisione $P$ fatta 
dai punti $x_0=a < x_1 < \dots < x_n=b$ tale che 
\mynote{L'operatore $\osc$ è stato introdotto 
nella definizione~\ref{def:osc}}%
\[
  S^*(f,P) - S_*(f,P) 
  = \sum_{k=1}^n (x_k - x_{k-1})
  \cdot \osc f([x_{k-1},x_k])
  \le \sigma.
\]

Vogliamo dimostrare che la seguente quantità è piccola:
\[
 S^*(\phi\circ f,P) - S_*(\phi\circ f,P)
 = \sum_{k=1}^n (x_k - x_{k-1}) \cdot \osc \phi(f([x_{k-1},x_k])).
\]
Se $\osc f([x_{k-1},x_k]) \le \frac{\delta }2$, 
allora per ogni $t,s\in[x_{k-1},x_k]$ si ha 
$\abs{f(t)-f(s)}\le \frac{\delta}{2}$
e quindi, per l'uniforme continuità di $\phi$,
$\abs{\phi(f(t))-\phi(f(s))}<\eps$.
Dunque 
\[
  \osc f([x_{k-1},x_k]) \le \frac{\delta}{2} 
  \implies  
  \osc \phi(f([x_{k-1},x_k]))\le \eps.
\] 
D'altra parte gli intervalli in cui 
$\osc f([x_{k-1},x_k]) > \frac{\delta}{2}$ 
dovranno essere pochi perché altrimenti la funzione non sarebbe 
integrabile.
Se chiamiamo $K$ l'insieme degli indici $k$ per cui si 
ha $M_k-m_k\ge \frac \delta 2$, dovrà infatti essere 
\[
  \frac{\delta}{2}\sum_{k\in K} (x_k-x_{k-1})
  \le \sum_{k\in K} (x_k-x_{k-1})\cdot \osc f([x_{k-1},x_k])
  \le S^*(f,P) - S_*(f,P) \le \sigma.
\]
Visto che la funzione $\phi$ assume valori tra $-M$ e $M$, 
si ha  
\[
   \sum_{k\in K} (x_k-x_{k-1})
   \osc \phi(f([x_{k-1},x_k]))
   \le 2M \sum_{k\in K} (x_k-x_{k-1})
   \le \frac{4M\sigma}{\delta} = \eps (b-a).
\]

In conclusione: 
\begin{align*}
  S^*(\phi\circ f,P)-S_*(\phi\circ f,P)
  \le 
  \eps (b-a) + 
  \sum_{k\not \in K} (x_k-x_{k-1}) \eps
  = 2\eps (b-a)
\end{align*}
che per i criteri di integrabilità ci dà la tesi.
\end{proof}

\begin{remark}
Se nel teorema precedente si prende $f(x)=x$ si ottiene 
il teorema~\ref{th:integrabilita_continue} di integrabilità delle funzioni continue.
\end{remark}

Dati $a,b\in \RR$ definiamo:
\begin{align*}
  a^+ &= \begin{cases}
    a & \text{se $a>0$,}\\
    0 & \text{altrimenti,}
  \end{cases} &
  a^- &= \begin{cases}
    -a & \text{se $a<0$,}\\
    0 & \text{altrimenti,}
  \end{cases} \\
  a \vee b &= \begin{cases}
    a & \text{se $a\ge b$,}\\
    b & \text{altrimenti,}
  \end{cases} &
  a \wedge b &= \begin{cases}
    a & \text{se $a\le b$,}\\
    b & \text{altrimenti.}
  \end{cases}
\end{align*}
I valori $a^+$ e $a^-$ si chiamano rispettivamente 
\emph{parte positiva} e \emph{parte negativa} di $a$.
L'operazione $a\vee b = \max\ENCLOSE{a,b}$ restituisce il massimo tra $a$ e $b$,
mentre $a\wedge b = \min\ENCLOSE{a,b}$ restituisce il minimo tra $a$ e $b$.
Si noti che valgono le seguenti proprietà:
\[
   a^+ = a \vee 0, \qquad
   a^- = -(a \wedge 0), \qquad
   a = a^+ - a^-, \qquad
   \abs{a} = a^+ + a^-.
\]
Inoltre ci si può facilmente convincere che risulta:
\[
  a + b = (a \vee b) + (a \wedge b), \qquad
  \abs{a-b} = (a \vee b) - (a \wedge b).
\]

Se $f$ e $g$ sono funzioni a valori reali definiamo 
di conseguenza le funzioni $f^+$, $f^-$, $f\vee g$ e $f\wedge g$:
\begin{gather*}
  f^+(x) = (f(x))^+, \qquad
  f^-(x) = (f(x))^-, \\
  (f\vee g)(x) = f(x) \vee g(x), \qquad
  (f\wedge g)(x) = f(x) \wedge g(x).
\end{gather*}

\begin{theorem}[proprietà di reticolo]%
\label{th:reticolo}%
Se $f$ è una funzione Riemann-integrabile sull'intervallo $[a,b]$ allora
anche $\abs{f}$, $f^+$ e $f^-$ sono integrabili e, se
anche $g$ è Riemann-integrabile su $[a,b]$, allora anche $f\vee g$ e $f\wedge g$
sono integrabili sullo stesso intervallo.
\end{theorem}
%
\begin{proof}
  Basta osservare che le funzioni 
  $y\mapsto \abs{y}$, $y\mapsto y^+$ e $y\mapsto y^-$ 
  sono funzioni continue, e applicare il teorema~\ref{th:composta_integrabile}
  di integrabilità della funzione composta.
  \end{proof}
  %
  \begin{proof}[Dimostrazione alternativa.]
  Possiamo fare una dimostrazione che non richiede 
  il teorema~\ref{th:composta_integrabile}
  ma, sostanzialmente, ne ricalca la dimostrazione nel 
  caso particolare, molto più semplice, che ci interessa.

  Dimostriamo innanzitutto che se $f$ è integrabile anche $f^+$ lo è.
  Visto che $\abs{f^+(x)-f^+(y)}\le \abs{f(x)-f(y)}$ 
  è facile convincersi che su ogni insieme $A$ risulta: 
  \mynote{L'operatore $\osc$ è stato introdotto 
  nella definizone~\ref{def:osc}}%
  $\osc_A f^+ \le \osc_A f$.
  Dunque su qualunque suddivisione $P$
  si ottiene:
  \[
    0 \le S^*(f^+,P) - S_*(f^+,P) \le S^*(f,P) - S_*(f,P).
  \]
  Per il primo criterio di integrabilità del 
  teorema~\ref{th:criteri_integrabilita} essendo $f$ integrabile 
  sappiamo che per ogni $\eps>0$ esiste $P$ tale che il lato 
  destro dell'equazione precedente è inferiore ad $\eps$.
  Ma allora anche il lato sinistro lo è e dunque, 
  per lo stesso criterio, anche $f^+$ è integrabile.    
  In maniera analoga si dimostra che se $f$ è integrabile 
  anche $f^-$ è integrabile.
  \mynote{
    Oppure si può osservare che $f^- = -(-f)^+$ e sfruttare il 
    fatto che se $f$ è integrabile già sappiamo che $-f$ 
    è integrabile.
  }
  
  Per quanto riguarda $\abs f$, $f\vee g$ e $f\wedge g$  
  basta osservare che%
  \mynote{Per ogni $a,b\in \RR$ 
  è facile convincersi che valgono 
  le seguenti identità
  \[
    \begin{cases}
    a = a^+ - a^-\\
    \abs{a} = a^+ + a^-\\
    \end{cases}
  \]\[
    \begin{cases}
      a + b = (a\vee b) + (a\wedge b)\\
      \abs{a - b} = (a\vee b) - (a\wedge b) 
    \end{cases}
  \]}
  \begin{gather*}
    \abs{f} = f^+ + f^-, \qquad
  %  f^+ = \frac{\abs{f} + f}{2}, \qquad
  %  f^- = \frac{\abs{f} - f}{2}, \\
    f\wedge g = \frac{f + g - \abs{f-g}}{2}, \qquad
    f\vee g = \frac{f+g + \abs{f-g}}{2}
  \end{gather*}
  e visto che combinazioni lineari di funzioni integrabili 
  sono funzioni integrabili (teorema~\ref{th:integrale_lineare})
  si ottiene l'integrabilità delle funzioni a lato sinistro
  delle uguaglianze.
\end{proof}

\begin{remark}
Osserviamo che se $f$ e $g$ sono funzioni continue, 
allora anche $f^+$, $f^-$, $\abs f$, $f\vee g$ e $f\wedge g$ sono funzioni continue
e dunque Riemann integrabili grazie al teorema~\ref{th:integrabilita_continue}.
Dunque il teorema precedente è utile solamente quando si ha a che fare con funzioni 
Riemann integrabili che non sappiamo essere continue. 
\end{remark}

\begin{theorem}
\label{th:integrale_modulo}%
Se $a\le b$ ed $f$ è una funzione limitata e Riemann-integrabile su $[a,b]$
si ha 
\[
  \abs{\int_a^b f(x)\, dx} \le \int_a^b \abs{f(x)}\, dx.
\]
In generale (tolta l'ipotesi $a \le b$) si ha
\begin{equation}\label{eq:489732}
  \abs{\int_a^b f(x)\, dx} 
  \le 
  \abs{\int_a^b \abs{f(x)}}. 
\end{equation}
\end{theorem}
\begin{proof}
Osserviamo che le funzione $\abs{f}$, $f^+$, $f^-$ sono Riemann-integrabili
per il teorema~\ref{th:reticolo} precedente.

Se $a\le b$, essendo $f^+\ge 0$ e $f^-a\ge 0$,
per monotonia (teorema~\ref{th:monotonia_integrale})
sappiamo che $\int_a^b f^+\ge 0$ 
e $\int_a^b f^-\ge 0$, cosicché, 
per linearità (teorema~\ref{th:integrale_lineare}):
\begin{align*}
\abs{\int_a^b f} 
&= \abs{\int_a^b f^+ - \int_a^b f^-}\\
&\le \abs{\int_a^b f^+} + \abs{\int_a^b f^-}
= \int_a^b f^+ + \int_a^b f^-
= \int_a^b \abs{f}.   
\end{align*}
Se $b\le a$ ci si riconduce al caso precedente scambiando 
gli estremi di integrazione.
\end{proof}

Il seguente risultato è una versione continua della disuguaglianza di convessità 
per le combinazioni baricentriche.
Si può notare che il teorema~\ref{th:integrale_modulo} precedente è un caso particolare 
del seguente con $\phi(x)=\abs{x}$.
%
\begin{theorem}[disuguaglianza di Jensen]
  \label{th:jensen}%
  \index{disuguaglianza!di Jensen}%
  \index{Jensen!disuguaglianza di}%
  \index{teorema!Jensen}%
Sia $f\colon[a,b]\to I\subset \RR$ una funzione continua a valori 
in un intervallo $I$ e sia $\phi\colon I \to \RR$ una funzione convessa e continua.
Sia inoltre $g\colon [a,b]\to \RR$ una funzione continua,
non negativa e tale che 
\[
    \int_a^b g(x)\, dx = 1
\]
Allora si ha 
  \[
  \phi\enclose{\int_a^b f(x) g(x) \, dx} \le \int_a^b \phi(f(x)) g(x)\, dx.
  \]
Nel caso particolare in cui si scelga $g$ la funzione costante $g(x)=1/(b-a)$ si ottiene 
la formulazione più espressiva:
\[
  \phi\enclose{\dashint_a^b f(x)\, dx} \le \dashint_a^b \phi(f(x))\, dx
\]
dove con $\dashint$ si è indicata la media integrale:
\[
  \dashint_a^b f(x)\, dx = \frac{1}{b-a} \int_a^b f(x)\, dx.
\]
\end{theorem}
%
\begin{proof}
Le ipotesi di continuità di $f$, $g$ e $\phi$ sono sovrabbondanti, ma ci servono 
per avere una garanzia immediata che entrambe le funzioni integrande
siano effettivamente Riemann-integrabili.
Posto 
\[
  y_0 = \int_a^b f(x) g(x)\, dx 
\]
scegliamo una funzione affine $L(y) = my + q$ tale che $L(y_0)=\phi(y_0)$ 
e $L(y)\le \phi(y)$ per ogni $y\in I$ (grazie al teorema~\ref{th:supporto_convessa}).
Per la linearità dell'integrale si ha: 
\[
  L(y_0)
  = \int_a^b mf(x)g(x)\, dx + q \int_a^b g(x)\, dx
  = \int_a^b L(f(x)) g(x)\, dx.
\]
Allora per la monotonia dell'integrale risulta
\[
  \phi(y_0) = L(y_0) = \int_a^b L(f(x)) g(x)\, dx \le \int_a^b \phi(f(x)) g(x)\, dx.
\]
\end{proof}
    

\begin{theorem}[continuità dell'integrale]
\label{th:integrale_continuo}
Sia $f\colon [a,b]\to \RR$ una funzione limitata e tale
che per ogni $c\in (a,b]$ risulta che $f$ sia Riemann-integrabile
su $[c,b]$. Allora $f$ è Riemann-integrabile su $[a,b]$ e risulta
\[
  \int_a^b f = \lim_{c\to a^+} \int_c^b f.
\]

Analogamente se $f$ è integrabile su ogni intervallo $[a,c]$
con $c\in [a,b)$ allora $f$ è integrabile su $[a,b]$ e vale
\[
  \int_a^b f = \lim_{c\to b^-} \int_a^c f.
\]
\end{theorem}
%
\begin{proof}
Dimostriamo solamente la prima parte, visto che la seconda si tratta in
maniera del tutto analoga. Supponiamo quindi che $f$ sia integrabile su ogni
intervallo $[c,b]$ con $c\in (a,b]$. Sappiamo inoltre che $f$ è limitata su
tutto $[a,b]$ e quindi esiste $M>0$ tale che $\abs{f(x)}\le M$ per ogni
$x\in [a,b]$. Fissato $\eps>0$ qualunque,
scegliamo $c = a + \eps/(4M)$ e, sapendo che $f$ è integrabile su $[c,b]$,
consideriamo una suddivisione $Q_\eps$ di $[c,b]$ tale che
\[
  S^*(f, Q_\eps) - S_*(f, Q_\eps) < \frac \eps 2.
\]
Ma allora posto $P_\eps = \ENCLOSE{a} \cup Q_\eps$ otteniamo:
\begin{align*}
  S^*(f,P_\eps) &= S^*(f,Q_\eps) + (c-a)\sup_{[a,c]} f
    \le S^*(f,Q_\eps) + M \cdot (c-a) \le S^*(f,Q_\eps) + \frac \epsilon 2\\
  S_*(f,P_\eps) &= S_*(f,Q_\eps) + (c-a)\inf_{[a,c]} f
    \ge S_*(f,Q_\eps) - M \cdot (c-a) \ge S_*(f,Q_\eps) - \frac \epsilon 2
\end{align*}
da cui
\begin{align*}
  S^*(f,P_\eps)-S_*(f,P_\eps)
  &\le S^*(f,Q_\eps) - S_*(f,Q\eps) + 2 M \cdot (c-a)\\
  &< \frac{\eps}{2} + 2 M \frac{\eps}{4M}
  = \eps.
\end{align*}
Dunque, la funzione $f$ è integrabile. Ma si ha
\begin{align*}
\int_a^b f \le S^*(f,P_\eps) \le S^*(f,Q_\eps) + \frac{\eps}{4}
\le S_*(f,Q_\eps) + \frac 3 4 \eps
\le \int_{a+\frac {\eps}{4M}}^b f + \frac {3}{2} \eps \\
\int_a^b f \ge S_*(f,P_\eps) \ge S_*(f,Q_\eps) - \frac{\eps}{4}
\ge S^*(f,Q_\eps) - \frac 3 4 \eps
\ge \int_{a+\frac {\eps}{4M}}^b f - \frac {3}{2} \eps
\end{align*}
da cui, passando al limite per $\eps \to 0^+$, si ottiene
\[
  \lim_{c\to a^+} \int_c^b f \le \int_a^b f \le \lim_{c\to a^+} \int_c^b f
\]
e quindi l'uguaglianza.
\end{proof}

\begin{example}
La funzione
\[
  f(x) = \begin{cases}
  \sin\frac 1 x & \text{se $x\neq 0$,}\\
  0 & \text{se $x=0$}
  \end{cases}
\]
è integrabile su ogni intervallo chiuso e limitato $[a,b]$.
\end{example}
\begin{proof}
Su ogni intervallo $[\eps,b]$ con $\eps>0$ e $b>\eps$ la funzione
è integrabile in quanto su tali intervalli è continua (la funzione è continua su
tutto $\RR\setminus\ENCLOSE{0}$). Inoltre la funzione è limitata, quindi per il teorema
precedente possiamo concludere che è integrabile su $[0,b]$ per ogni $b>0$.

In maniera analoga (per simmetria) la funzione è integrabile su $[a,0]$
per ogni $a<0$.
Dunque, per additività, la funzione è integrabile su ogni $[a,b]$ con $a<0$ e $b>0$
e di conseguenza (sempre grazie al teorema~\ref{th:additivita_integrale})
è integrabile su ogni intervallo $[a,b]$.
\end{proof}

