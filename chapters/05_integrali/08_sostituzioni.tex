\section{integrali che si riconducono a funzioni razionali}

E' importante sapere che di qualunque funzione razionale è possibile
scriverne la primitiva utilizzando i metodi della sezione precedente.
Allo stesso modo è utile sapere che ci sono altre casistiche che si
riconducono all'integrazione di una funzione razionale.

\emph{Funzioni razionali in $e^x$.}
Se la funzione integranda $f(x)$ si scrive nella forma
\[
  f(x) = R(e^{\lambda x})
\]
con $R$ funzione razionale e $\lambda\neq 0$,
allora si può risolvere l'integrale tramite la
sostituzione $t = e^{\lambda x}$. Infatti si ha
\[
\begin{cases}
 e^{\lambda x} = t\\
 x = \frac{\ln t}{\lambda}\\
 dx = \frac{1}{\lambda t}\, dt
 \end{cases}
\]
e la funzione integranda diventa una funzione razionale:
\[
 \int R(e^{\lambda x})\, dx = \Enclose{\int \frac{R(t)}{\lambda t}\, dt}_{t=e^{\lambda x}}
\]

\begin{example}
Si voglia risolvere l'integrale
\[
  \int \frac{2\sqrt{e^x} + e^{2x}}{e^x-4}\, dx.
\]
\end{example}
\begin{proof}[Soluzione.]
Scriviamo la funzione integranda in funzione di $e^{\frac x 2}$:
\[
  \frac{2\sqrt{e^x} + e^{2x}}{e^x-4}
  =\frac{2e^{\frac x 2}+e^{4\frac x 2}}{e^{2\frac x 2}-4}.
\]
Facendo il cambio di variabile $t=e^{\frac x 2}$, $x=2\ln t$, $dx=\frac 2 t\, dt$
si ottiene una funzione razionale in $t$:
\[
  \int \frac{2\sqrt{e^x} + e^{2x}}{e^x-4}\, dx
  = \int \frac{2t + t^4}{t^2-4}\cdot \frac 2 t dt
  = 2\int \frac{2+t^3}{t^2-4}\, dt.
\]
Facendo la divisione tra i polinomi e la riduzione ai fratti semplici si ottiene
\[
 \int 2t\, dt + \int \frac{5}{t-2}\, dt + \int \frac{3}{t+2}\, dt
 = t^2 + 5\ln\abs{t-2} + 3\ln\abs{t+2}
\]
e quindi sostituendo $t=e^{\frac x 2}$ si ottiene il risultato
\[
 e^x + 5 \ln \abs{\sqrt{e^x}-2} + 3 \ln\enclose{\sqrt{e^x}+2}.
\]
\end{proof}

\emph{Funzioni razionali in $\sin^2 x$, $\cos^2 x$, $\sin x \cdot \cos x$.}
Se la funzione integranda $f(x)$ si scrive nella forma
\[
  f(x) = R(\sin^2 x, \cos^2 x, \sin x \cdot \cos x)
\]
con $R$ funzione razionale (cioè rapporto di polinomi nelle tre variabili indicate)
 allora si può risolvere l'integrale
tramite la sostituzione $t=\tg x$. Infatti
osservando che risulta
\[
  1 + \tg^2 x = 1 + \frac{\sin^2 x}{\cos^2 x} = \frac{1}{\cos^2 x}
\]
da cui
\begin{align*}
\cos^2 x &= \frac{1}{1+\tg^2 x}\\
\sin^2 x &= \tg^2 x \cdot\cos^2 x\\
\sin x \cdot \cos x &= \tg x \cdot \cos^2 x
\end{align*}
ponendo $t = \tg x$ si ha:
\begin{equation}\label{eq:466324}
\begin{cases}
  \cos^2 x = \frac{1}{1+t^2}\\
  \sin^2 x = \frac{t^2}{1+t^2}\\
  \sin x \cdot \cos x = \frac{t}{1+t^2}\\
  dx = \frac{1}{1+t^2}\, dt
\end{cases}
\end{equation}
e la funzione integranda diventa una funzione razionale.

\begin{example}
\label{ex:35663}
Si voglia calcolare
\[
  \int \frac{1}{\cos x \cdot ( \sin x + \cos x)}\, dx.
\]
\end{example}
\begin{proof}[Soluzione.]
La funzione integranda si può scrivere nella forma
\[
  \frac{1}{\sin x \cdot \cos x + \cos^2x}.
\]
Effettuando la sostituzione~\eqref{eq:466324}
si ottiene
\[
  \int \frac{1}{\frac t {1+t^2} + \frac{1}{1+t^2}}\cdot \frac{1}{1+t^2}\, dt
  = \int \frac{1}{t+1}\, dt = \ln \abs{t+1} = \ln \abs{\tg x + 1}.
\]
\end{proof}

\emph{Più in generale funzioni razionali di $\sin x$ e $\cos x$.}
Se la funzione integranda $f(x)$ si scrive nella forma
\[
  f(x) = R(\sin x, \cos x)
\]
con $R$ funzione razionale, allora si può risolvere l'integrale
tramite la sostituzione $t=\tg \frac{x}{2}$. Infatti con tale sostituzione si ha
(usando le formule di bisezione e riconducendosi al caso precedente)
\begin{equation}\label{eq:3675323}
  \begin{cases}
    \cos x = \cos^2 \frac x 2 - \sin^2 \frac x 2 = \frac{1-t^2}{1+t^2} \\
    \sin x = 2 \sin \frac x 2 \cos \frac x 2 = \frac{2t}{1+t^2}\\
    dx = \frac{2}{1+t^2}\, dt.
  \end{cases}
\end{equation}
Di nuovo con questa sostituzione la funzione integranda diventa razionale.

\begin{remark}
Si osservi che la sostituzione~\eqref{eq:3675323} potrebbe essere
sempre utilizzata al posto della~\eqref{eq:466324} in quanto più generale.
Ma, usualmente, se è possibile usare la sostituzione~\eqref{eq:466324}
l'integrale risulta
poi più semplice da calcolare. Si faccia la prova con l'integrale
dell'esempio~\ref{ex:35663}!
\end{remark}

\begin{example}
Si voglia calcolare
\[
 \int \frac{1}{\sin x}\, dx.
\]
\end{example}
\begin{proof}[Soluzione.]
Utilizzando la sostituzione~\eqref{eq:3675323}
si ottiene
\begin{align*}
  \int \frac{1}{\sin x}\, dx
  &= \int \frac{1}{\frac{2t}{1+t^2}}\frac{2}{1+t^2}\, dt \\
  &= \int \frac{1}{t}\, dt = \ln\abs t = \ln \abs{\tg \frac x 2}.
\end{align*}
\end{proof}

\emph{Funzioni razionali con radicali.}
Se la funzione $f(x)$ si scrive nella forma:
\[
  f(x) = R(\sqrt[n] x)
\]
con $R$ funzione razionale, allora si può risolvere l'integrale tramite
la sostituzione $x = t^n$. Infatti con tale sostituzione si ha
\begin{equation}\label{eq:4675821}
\begin{cases}
  \sqrt[n] x = t\\
  dx = n t^{n-1}\, dt.
\end{cases}
\end{equation}

\begin{example}
Si voglia calcolare
\[
  \int \frac{\sqrt[4]{x}}{\sqrt[2]{x} + \sqrt[3]{x}}\, dx.
\]
\end{example}
\begin{proof}[Soluzione.]
Si osservi che la funzione integranda può essere scritta
come funzione razionale di $t=\sqrt[12]{x}$ (abbiamo scelto
il minimo comune multiplo tra i radicandi in gioco: $12 = \mathit{mcm}\ENCLOSE{4,2,3}$)
\[
  \frac{\sqrt[4]{x}}{\sqrt{x} + \sqrt[3]{x}}
  = \frac{\sqrt[12]{x^3}}{\sqrt[12]{x^6} + \sqrt[12]{x^4}}.
\]
Dunque utilizzando la sostituzione \eqref{eq:4675821} con $n=12$ si ha
\begin{align*}
\int \frac{t^3}{t^6 + t^4}\cdot 12 t^{11}\, dt
&= 12 \int \frac{t^{14}}{t^6+t^4}\, dt
 = 12 \int \frac{t^{10}}{t^2+1}\, dt.
\end{align*}
Procedendo con la divisione tra polinomi si ottiene
\begin{align*}
\MoveEqLeft{12 \int \Enclose{t^8 - t^6 + t^4 - t^2 + 1 - \frac{1}{t^2+1}}\, dt} \\
&= 12 \Enclose{\frac{t^9}{9} - \frac{t^7}{7} + \frac{t^5}{t} - \frac{t^3}{3} + t - \arctg t} \\
&= \frac 4 3 \sqrt[4]{x^3} - \frac{12}{7}\sqrt[12]{x^7}
+ \frac{12}{5}\sqrt[12]{x^5} - 4 \sqrt[4]{x} + 12 \sqrt[12]{x} - 12 \arctg \sqrt[12]{x}.
\end{align*}
\end{proof}

