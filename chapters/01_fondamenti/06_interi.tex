\section{i numeri interi}

L'insieme $\NN$ dei numeri naturali con l'operazione di addizione 
non è un gruppo in quanto gli elementi non hanno opposto in $\NN$.
Per ovviare a questo vogliamo definire un insieme più grande 
che contenga anche gli opposti dei numeri naturali e su cui si possa 
quindi definire l'operazione di sottrazione. 

In questo caso ha senso rappresentare la coppia $(m,n)$ 
tramite la freccia $m\mapsto n$ perché la differenza $n-m$ 
rappresenta la quantità che aggiunta ad $m$ dà $n$
e lo si può pensare come ad una freccia (o vettore) che parte da $m$ 
ed arriva ad $n$ sull'insieme dei numeri naturali.
Dobbiamo però identificare le coppie che rappresentano la stessa differenza,
dunque poniamo (si veda la definizione~\ref{def:insieme_quoziente}  
di insieme quoziente)
\[
  \ZZ \defeq (\NN \times \NN)/\sim
  \qquad\text{dove}\qquad
  (m\mapsto n)\sim (m' \mapsto n')\iff n+m' = n'+m.
\]
Un numero intero $k\in \ZZ$ sarà quindi una classe di equivalenza:
$k = \Enclose{m \mapsto n}_\sim$. 
\mynote{Se $n\ge m$ la classe di equivalenza $[m\mapsto n]_\sim$
coincide con il grafico della funzione $\NN\colon \NN$ che trasla 
i numeri naturali di $n-m$ unità verso destra. 
Dunque coincide con la funzione che aggiunge $n-m$ ad ogni numero
naturale. 
Per come abbiamo definito l'operazione di addizione risulta quindi, 
$[m\mapsto n]_\sim = +(n-m)$.
Se invece $n<m$ la classe di equivalenza $[m\mapsto n]_\sim$
coincide con il grafico della funzione che trasla i numeri naturali
di $m-n$ unità verso sinistra ed è definita solo sui naturali 
maggiori o uguali ad $m-n$. 
Questa funzione è proprio l'operazione di sottrazione, dunque formalmente 
$[m\mapsto n]_\sim = -(m-n)$. Formalmente possiamo quindi scrivere 
\[
 \ZZ = \ENCLOSE{+n\colon n\in \NN} \cup \ENCLOSE{-n\colon n\in \NN}
\]
che risulta essere perfettamente aderente alla nostra intuizione.
}
D'ora in poi scriviamo semplicemente $k=\Enclose{m\mapsto n}$
tralasciando il simbolo $\sim$ che indica l'equivalenza.
Intuitivamente il numero intero $\Enclose{m\mapsto n}$ con $n,m\in \NN$ 
rappresenta il numero $n-m\in \ZZ$.
L'addizione su $\ZZ$ si definisce facilmente:
\[
 \Enclose{m \mapsto n}+\Enclose{M\mapsto N} = \Enclose{m+M \mapsto n+N}.
\]
\mynote{Se i numeri interi vengono interpretati come \emph{traslazioni}
sui numeri naturali, la somma di due numeri interi coincide, dove è definita,
con la composizione delle due traslazioni.}
%
Si verifica facilmente che la definizione precedente è \emph{ben posta}
ovvero non dipende dalla scelta di $(n,m)$ e $(N,M)$ all'interno 
della loro classe di equivalenza: se $n+m' = n'+m$ e $N+M'=N'+M$ 
allora $(n+N)+(m'+M')=(n'+N')+(m+M)$.

L'addizione che abbiamo definito su $\ZZ$ è commutativa e associativa, 
il numero $\Enclose{0 \mapsto 0}$
è elemento neutro e ogni elemento ha opposto, infatti 
$\Enclose{m \mapsto n} + \Enclose{n \mapsto m} 
= \Enclose{m+n \mapsto n+m} = \Enclose{0\mapsto 0}$
cioè $-\Enclose{m \mapsto n}=\Enclose{n\mapsto m}$.
Dunque $\ZZ$ è un gruppo additivo.

Possiamo definire facilmente un ordinamento su $\ZZ$:
$\Enclose{m\mapsto n} \ge \Enclose{M\mapsto N}$ se 
$n+M\ge N+m$.
E' molto facile verificare che l'ordinamento è compatibile 
con l'addizione e quindi $\ZZ$ risulta essere un 
gruppo abeliano totalmente ordinato (definizione~\ref{def:gruppo_ordinato}).

Osserviamo che possiamo identificare un numero naturale $n\in \NN$ 
con la classe di equivalenza $\phi(n) = \Enclose{0\mapsto n}\in \ZZ$.
La corrispondenza $\phi\colon \NN\to\ZZ$ è iniettiva e rispetta sia l'addizione:
$\phi(n+m) = \phi(n)+\phi(m)$ che l'ordinamento: $n\ge m \iff \phi(n)\ge \phi(m)$.
Dunque l'insieme $\phi(\NN)\subset \ZZ$ è una copia isomorfa dei 
numeri naturali e soddisfa quindi gli assiomi di Peano, come $\NN$.
\mynote{
Se $n,m\in \NN$, $n\ge m$ risulta 
$\Enclose{m \mapsto n} = \Enclose{n\mapsto 0}-\Enclose{m\mapsto 0} 
= \phi(n)-\phi(m)$.
}%
Sarà quindi comodo rinominare $\NN$ identificandolo con $\phi(\NN)$ così 
ottenendo che $\NN\subset \ZZ$.
Preso qualunque $n\in \ZZ$ si avrà che $n\in \NN$ se $n\ge 0\in \ZZ$ 
altrimenti  si avrà $-n\in \NN$. 
Dunque $\ZZ = \NN \cup (-\NN)$ e $0\in \NN\subset \ZZ$ 
è l'unico numero intero che è l'opposto di se stesso.

L'operazione di moltiplicazione che abbiamo definito su $\NN$ può essere 
estesa in modo naturale a tutto $\ZZ$. 
Se vogliamo mantenere la proprietà distributiva dovremo avere, 
per $n,m\in\NN$
\[
  0 = 0 \cdot n = (m+(-m))\cdot n = m\cdot n + (-m)\cdot n
\]
per cui poniamo, per definizione, per ogni $n,m\in \NN$:
\[
  (-m) \cdot n = -(m\cdot n), \qquad  m \cdot (-n) = -(m\cdot n).
\]
In questo modo la moltiplicazione $x\cdot y$ 
è ben definita per ogni $x,y\in \ZZ$ e soddisfa la proprietà distributiva.

Con le definizioni date è noioso ma facile dimostrare 
che valgono le usuali proprietà della
moltiplicazione su $\ZZ$.

\begin{theorem}[proprietà moltiplicazione]
  La moltiplicazione su $\ZZ$ soddisfa le seguenti proprietà.
  \begin{enumerate}
    \item[1.] elemento neutro e assorbente: $n\cdot 1 = n$, $n\cdot 0 = 0$,
    \item[2.] proprietà associativa: $(n\cdot m)\cdot k = n \cdot (m\cdot k)$,
    \item[3.] proprietà commutativa: $n\cdot m = m\cdot n$,
    \item[4.] proprietà distributiva: $k\cdot(m+n) = k\cdot m + k\cdot n$. 
  \end{enumerate}
\end{theorem}

In effetti scopriamo che $\ZZ$ è un anello abeliano con unità
(definizione~\ref{def:anello}).

Se $n,m\in\ZZ$ con $m\neq 0$ e 
\mymargin{multiplo}%
se esiste $k\in\ZZ$ tale 
che $n=km$ diremo che $n$ è un \emph{multiplo}
di $m$ oppure che $m$ è un \emph{divisore}%
\mymargin{divisore}%
\index{divisore} di $n$
e scriveremo:
\[
  \frac{n}{m} = k.  
\]

I multipli di $2$ si chiamano numeri \emph{pari}%
\mymargin{pari}%
\index{pari},
gli interi non pari si dicono \emph{dispari}%
\mymargin{dispari}%
\index{dispari}. 


\subsection{frazioni e numeri razionali}

Così come abbiamo esteso i numeri naturali affinché l'operazione di addizione abbia 
una operazione inversa (la sottrazione) in modo simile possiamo estendere l'insieme 
dei numeri interi in modo che anche la moltiplicazione abbia una operazione inversa 
(la divisione). 
La frazione $\frac p q$ può essere definita come coppia 
di numeri interi $p\in \ZZ$ e $q\in \NN\setminus \ENCLOSE 0$. 
E l'insieme dei numeri razionali si ottiene identificando 
frazioni equivalenti, ovvero: 
\[
  \QQ = ((\NN \setminus\ENCLOSE 0)\times \ZZ)/\sim  
\]
dove la relazione di equivalenza $\sim$ è definita ponendo 
$(q,p) \sim (q',p')$ quando $p\cdot q' = p'\cdot q$.
La coppia $(q,p)$ rappresenta la frazione $\frac p q$ e, 
per maggiore chiarezza, nel seguito scriveremo $\frac p q$ 
al posto di $(q,p)$.

\mynote{La coppia $(q,p)$ può essere anche denotata tramite 
la freccia $q\mapsto p$ 
e rappresenta la dilatazione di $\ZZ$ che manda il numero $q$ nel numero $p$.
Frazioni equivalenti rappresentano la stessa dilatazione:
ad esempio se dilato $\ZZ$ di un fattore $\frac 3 2$, 
il numero $2$ andrà nel 
numero $3$ e il numero $8$ andrà nel numero $12$.
In questo senso la frazione $\frac 3 2$ viene posta equivalente
alla frazione $\frac{12} 8$.
La moltiplicazione tra frazioni corrisponde alla composizione 
dei riscalamenti: se compongo la dilatazione $2\mapsto 3$ 
con la dilatazione $3\mapsto 5$ ottengo la dilatazione 
$2\mapsto 5$. Infatti $\frac{3}{2}\cdot \frac{5}{3} = \frac{5}{2}$.
}%
La moltiplicazione e la addizione tra numeri razionali viene definita
sulle classi di equivalenza:
\[
 \Enclose{\frac p q} \cdot \Enclose{\frac{p'}{q'}} 
 = \Enclose{\frac{p\cdot p'}{q\cdot q'}} 
 \qquad 
 \Enclose{\frac{p}{q}} + \Enclose{\frac{p'}{q'}} 
 = \Enclose{\frac{p\cdot q' + p'\cdot q}{q\cdot q'}}.
\]
Possiamo inoltre definire un ordinamento ponendo
\[
 \Enclose{\frac p q} \le \Enclose{\frac{p'}{q'}}
 \quad\iff  \quad p \cdot q' \le p' \cdot q.
\]
Si verifica facilmente che queste definizioni \emph{passano al quoziente} 
nel senso che il prodotto di frazioni equivalenti è equivalente al prodotto 
delle frazioni. 
Dunque la moltiplicazione e l'addizioni 
e la relazione d'ordine sono ben definite su $\QQ$.

Le frazioni del tipo $\frac{q}{q}$ sono tutte equivalenti tra loro e rappresentano 
l'elemento neutro della moltiplicazione in $\QQ$.
Le frazioni del tipo $\frac{p}{1}$ rappresentano 
i numeri interi $\ZZ$ all'interno di $\QQ$.

Con le operazioni appena definite si può verificare che $\QQ$ 
risulta essere un campo ordinato e denso.
Inoltre identificando il numero razionale $\Enclose{\frac p 1}_\sim$
con il numero intero $p\in \ZZ$ si può pensare, ridefinendo opportunamente 
$\ZZ$, che sia $\ZZ \subset \QQ$.

Una operazione che non può essere definita su $\QQ$ è, ad esempio, 
l'estrazione della radice quadrata. 
Il seguente teorema ci dice, ad esempio, che non possiamo definire in 
$\QQ$ la radice quadrata di $2$.

\begin{theorem}[Pitagora, irrazionalità di $\sqrt 2$]
  \mymark{**}%
  \label{th:pitagora}%
  L'equazione $x^2=2$ non ha soluzioni in $\QQ$.
  \end{theorem}
  %
  \begin{proof}
  \mymark{*}%
  Supponiamo $x\in \QQ$ sia una soluzione di $x^2=2$.
  Allora si potrà scrivere $x=p/q$ con $p\in \ZZ$ e $q\in \NN$, $q\neq 0$.
  Possiamo anche supporre che la frazione $p/q$ sia ridotta ai minimi
  termini cioè che $p$ e $q$ non abbiano fattori in comune.
  Moltiplicando l'equazione
  $(p/q)^2=2$ per $q^2$ si ottiene $p^2 = 2 q^2$.
  Risulta quindi che $p^2$ è pari.
  Ma allora anche $p$ è pari (perché il quadrato di un dispari è dispari).
  Ma se $p$ è pari allora $p^2$ è multiplo di quattro.
  Ma allora anche $2q^2$ è multiplo di quattro e quindi $q^2$ è pari.
  Dunque anche $q$ è pari. Ma avevamo supposto che $p$ e $q$ non avessero
  fattori in comune quindi questo non può accadere.
\end{proof}

Il fatto che $x^2=2$ non ha soluzione in $\QQ$ ci permette di 
asserire che l'ordinamento di $\QQ$ non è continuo.
\mymargin{$\QQ$ non è continuo}%
Infatti possiamo considerare i seguenti sottoinsiemi di $\QQ$:
\[
 A = \ENCLOSE{x\in \QQ\colon x^2\le 2, x\ge 0}, \qquad
 B = \ENCLOSE{x\in \QQ\colon x^2\ge 2, x\ge 0}. 
\]
Chiaramente se $a\in A$ e $b\in B$ si ha $a\le b$ perché se 
invece fosse $a>b$ avremmo, per monotonia, $a^2>b^2$ 
e quindi non potrebbe essere $a^2 \le 2$ e $b^2\ge 2$.
Ovviamente $0^2 \le 2$ mentre $2^2\ge 2$ dunque né $A$ né $B$ 
è vuoto.
Se $\QQ$ fosse continuo ci dovrebbe allora essere un elemento di 
separazione $c$ cioè $c\in \QQ$ tale che $a\le c\le b$ per ogni 
$a\in A$ e $b\in B$.
Ci chiediamo allora se $c^2$ è minore o maggiore di $2$.
Se $c^2<2$ possiamo porre $\eps = 2-c^2 > 0$. 
Esiste allora 
un numero razionale $\delta>0$ tale che $\delta<1$ 
e $\delta < \frac{\eps}{2c+1}$.
Allora si avrebbe:
\[
 (c+\delta)^2 
  = c^2+2c\delta + \delta^2 
  < c^2 + 2c\delta + \delta 
  = c^2 + \delta\cdot(2c+1) 
  < c^2 + \eps = 2.
\]
Ma questo significa che $c+\delta \in A$ e quindi $c$ non può essere 
un maggiorante di $A$.
Se invece fosse $c^2>2$ 
essendo $1\in A$ dovrà anche essere $c\ge 1>0$.
Dunque si può prendere $\delta = \frac{c^2-2}{2c}>0$
e si avrebbe 
\[
 (c-\delta)^2 
 = c^2 - 2c\delta + \delta^2
 > c^2 - 2c\delta = 2
\]
da cui risulta che $c-\delta \in B$ e quindi $c$ non può essere
un minorante di $B$.
Dobbiamo quindi concludere che deve essere $c^2=2$.
Ma, per il teorema~\ref{th:pitagora}, questo non è possibile.
Significa che $A$ e $B$ non hanno elemento di separazione in $\QQ$.

