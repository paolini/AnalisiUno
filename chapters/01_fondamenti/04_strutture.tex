\section{strutture algebriche}

Andremo presto a definire strutture numeriche su cui sono definite le operazioni 
di addizione e moltiplicazione. 
In generale vogliamo definire cosa si intende con una \emph{operazione}.
\mymargin{operazione}%
\index{operazione}%

Una \emph{operazione} sull'insieme $A$ è una funzione 
$f\colon A\to(A\to A)$.
Dati $a,b\in A$ avremo dunque che $f(a)$ è una funzione $A\to A$
e quindi $f(a)(b)$ sarà un elemento di $A$. 
Tipicamente le operazioni si denotano con simboli come $+$, $\cdot$, $/$, $\times$ etc...
Se $*$ è una operazione, invece di scrivere $*(a)(b)$ scriveremo 
$a*b$.

Si noti che una funzione $f\colon A\to (A\to A)$ è equivalente ad una funzione
$g\colon A\times A \to A$ ponendo $g(a,b) = f(a)(b)$. 
Si può dunque pensare equivalentemente ad una operazione 
come ad una funzione di due variabili.

\begin{example}
La usuale operazione di addizione $+$ definita sugli insiemi numerici è una operazione.
Se $\NN$ è l'insieme dei numeri naturali (lo definiremo tra poco)
possiamo pensare a $+\colon \NN\to (\NN\to \NN)$ come ad una funzione che ad un 
numero $n\in \NN$ associa la funzione $+n$ che incrementa di $n$ il suo argomento:
$(+3)(5)=3+5=8$.
\end{example}

Quando i simboli delle operazioni vengono utilizzati in una espressione si pone 
il problema di stabilire la precedenza delle operazioni.
Se ad esempio scriviamo $a*b*c$ dobbiamo decidere se intendiamo
$(a*b)*c$ (associatività a sinistra) oppure $a*(b*c)$ (associatività a destra).
Se $(a*b)*c=a*(b*c)$ diremo che l'operazione è associativa, in tal 
caso ovviamente non importa inserire le parentesi.
Se non si inseriscono le parentesi ususalmente si intende che l'associazione è a sinistra.
Ad esempio $5-4-3 = (5-4)-3$ e non $5-(4-3)$. 
In alcuni rari casi l'associatività è a destra, un esempio tipico è l'elevamento 
a potenza: $2^{3^2} = 2^{(3^2)} \neq (2^3)^2$.
Il motivo è che $(a^b)^c$ si scrive più facilmente $a^{b\cdot c}$.
Nei linguaggi di programmazione che implementano l'operazione di elevamento 
a potenza bisogna fare molta attenzione: in alcuni linguaggi (ad esempio Fortran) 
l'elevamento a potenza 
associa a sinistra come le altre operazioni, in altri linguaggi (ad esempio Python)
associa a destra.

Quando compaiono operazioni diverse nella stessa espressione, alcune operazioni 
hanno la precedenza su altre.
Ad esempio in $a+b\cdot c$ la moltiplicazione ha la precedenza sull'addizione
e quindi l'espressione va interpretata come $a+(b\cdot c)$ non come $(a+b)\cdot c$.
Addizione e sottrazione hanno la stessa precedenza e vanno quindi interpretate da sinistra a destra:
$a-b+c = (a-b)+c$. Per la divisione noi useremo, raramente, il simbolo $/$, 
in tal caso la divisione associa a sinistra ed ha la stessa precedenza della 
moltiplicazione: $a/b/c=(a/b)/c$, $a/b\cdot c=(a/b)\cdot c$.
Questa regola è a volte controintuitiva, per cui noi eviteremo di scrivere 
espressioni di questo tipo, soprattutto perché il puntino della moltiplicazione 
può essere omesso, come in $a/bc$ rendendo l'espressione ancora più ambigua.
Nel dubbio: usare le parentesi!
Per denotare la divisione utilizzeremo più spesso il simbolo di frazione $\frac a b$ che non richiede 
di specificare l'associatività e la precedenza in quanto la lunghezza della 
linea di frazione determina come vengono associati gli operandi: 
\[
  \frac{\,\frac a b\,}{c}=(a/b)/c, \qquad 
  \frac{a}{\,\frac bc\,}=a/(b/c).
\]  

Gli insiemi numerici che ci proponiamo di definire (naturali, interi, razionali, reali, complessi)
sono dotati tutti delle due operazioni di addizione e moltiplicazione
oltre che (salvo i numeri complessi) di una relazione di ordine.
In base alle proprietà che queste operazioni soddisfano potranno rientrare nelle 
seguenti strutture algebriche astratte. 

Per adesso facciamo un elenco delle strutture che incontreremo più avanti. 
Man mano che individueremo gli esempi di insiemi che rientrano in queste strutture
sarà più chiaro il significato di queste definizioni.

\begin{definition}[monoide]%
  \label{def:monoide}%
  Sia $A$ un insieme su cui è definita una operazione $*$.
  Se l'operazione è associativa: $x*(y*z) = (x*y)*z$ 
  ed ha elemento neutro $e\in A$ tale che $e*x = x*e = x$
  diremo che $A$ è un \emph{monoide}.
  Se inoltre l'operazione è commutativa, $x*y=y*x$ 
  diremo che $A$ è un \emph{monoide commutativo} (o abeliano).
  
  Quando per l'operazione $*$ viene utilizzato il simbolo di addizione $+$
  diremo che il monoide è additivo. In tal caso l'elemento neutro 
  si chiama \emph{zero} e si indica con $0$.
  Se utilizziamo invece il simbolo di moltiplicazione $x\cdot y$
  diremo che il monoide è moltiplicativo. In tal caso l'elemento 
  neutro si chiama \emph{unità} e si indica usualmente con $1$.
  \end{definition}
  
  \begin{example}
  L'insieme dei numeri naturali è un monoide commutativo sia con l'operazione di addizione $+$ 
  (monoide additivo) che con l'operazione di moltiplicazione 
  $\cdot$ (monoide moltiplicativo).
  \end{example}
  
  \begin{definition}[gruppo]
  \label{def:gruppo}%
  Un insieme $G$ su cui è definita una \emph{operazione} $*$ 
  si dice essere un \emph{gruppo}%
\mymargin{gruppo}%
\index{gruppo} se l'operazione
  ha le seguenti proprietà:
  \begin{enumerate}
    \item associativa: $\forall x,y,z\in G\colon (x*y)*z = x*(y*z)$;
    \index{proprietà!associativa}%
    \index{associatività}%
    \item esistenza elemento neutro: 
    \index{elemento!neutro}%
    \index{neutro}%
    $\exists e\in G\colon \forall x\in G \colon e*x=x*e = x$;
    \item esistenza inverso: 
    \index{elemento!inverso}%
    \index{inverso}%
    $\forall x\in G\colon \exists y\in G\colon x*y=y*x=e$.
  \end{enumerate}
  Inoltre il gruppo si dice essere \emph{abeliano}%
\mymargin{gruppo abeliano}%
\index{abeliano} o \emph{commutativo}
  se vale la proprietà:
  \begin{enumerate}
    \item[4.] commutativa: $\forall x,y\in G\colon x*y = y*x$.
    \index{proprietà!commutativa}%
    \index{commutatività}% 
  \end{enumerate}
  
  Quando l'operazione viene denotata con il simbolo $+$ (addizione)
  diremo che il gruppo è additivo, denoteremo con $0$ 
  \index{zero}%
  (zero) l'elemento neutro e l'inverso di $x$ verrà chiamato \emph{opposto}
  e si denota con $-x$.
  \index{opposto}%
  Se invece si usa il simbolo $\cdot$ (moltiplicazione)
  diremo che il gruppo è moltiplicativo, l'elemento neutro potrà 
  essere denotato con il simbolo $1$ (uno o unità) e 
  \index{uno}\index{unità}%
  l'inverso di $x$ potrà essere chiamato \emph{reciproco}
  e si denota usualmente con $x^{-1}$ o $\frac 1 x$.
  \index{reciproco}%
\end{definition}
  
L'elemento neutro di un gruppo è unico. 
Se infatti $x$ e $y$ fossero due elementi neutri 
si avrebbe $x = x*y = y$. 
Anche l'inverso è unico: infatti se $y$ e $z$ fossero 
due inversi di $x$ si avrebbe $y = y * x * z = z$.

\begin{example}
  L'insieme dei numeri interi $\ZZ$ (lo definiremo più avanti) 
  con l'operazione di addizione 
  è un gruppo abeliano. 
  L'elemento neutro è lo zero, l'opposto di $x$ è $-x$.
  Su $\ZZ$ definiremo anche l'operazione di moltiplicazione,
  ma $\ZZ$ con l'operazione di moltiplicazione non è un gruppo 
  in quanto non tutti gli elementi hanno inverso moltiplicativo in $\ZZ$.
\end{example}

\begin{example}[gruppo simmetrico]
  Se $X$ è un insieme qualunque si può considerare l'insieme
  $X!$ delle funzioni bigettive su $X$:
  \[
    X! = \ENCLOSE{f\colon X\to X\colon f \text{ bigettiva}}.
  \]
  Quando $X$ è un insieme finito le funzioni bigettive su $X$ si chiamano 
  anche \emph{permutazioni} in quanto, appunto, permutano gli elementi di $X$.

  Per esercizio si può verificare che l'operazione $\circ$ di composizione 
  rende $X!$ un gruppo.
  Tale gruppo si chiama \emph{gruppo simmetrico} di $X$ 
  (viene anche usualmente denotato con $S_X$).
  Il gruppo simmetrico non è abeliano se $X$ ha più di due elementi.
\end{example}

\begin{definition}[anello e campo]
  \label{def:anello}%
  \label{def:campo}%
  Sia $A$ un insieme su cui sono definite due operazioni: 
  addizione e moltiplicazione.  
  Diremo che $A$ è un \emph{anello} se $A$ è un gruppo abeliano rispetto alla 
  addizione, 
  se la moltiplicazione è associativa $(x\cdot y)\cdot z = x\cdot (y\cdot z)$ 
  e se vale la proprietà distributiva $x\cdot(y+z) = x\cdot y + x\cdot z$,
  $(x+y)\cdot z = x\cdot z + y\cdot z$.

  Se inoltre esiste un elemento $1\in A$ neutro per la moltiplicazione 
  diremo che $A$ è un anello \emph{con unità}.

  Se la moltiplicazione è commutativa diremo che $A$ è un \emph{anello abeliano}.

  Se $A$ è un anello ed $A\setminus \ENCLOSE{0}$ risulta essere un gruppo abeliano
  per l'operazione di moltiplicazione 
  (dunque c'è una unità $1\neq 0$ e ogni elemento non nullo ha inverso moltiplicativo)
  diremo che $A$ è un \emph{campo}.
\end{definition}

\begin{example}
  L'insieme dei numeri interi $\ZZ$ (lo definiremo tra poco) è un anello abeliano con unità
  ma non è un campo.
  L'insieme dei numeri razionali $\QQ$ (lo definiremo tra poco) è un campo.
\end{example}

Un anello $A$ è un gruppo additivo, per cui negli anelli l'elemento neutro 
dell'addizione si indica con $0$. 
L'inverso additivo di $a\in A$ si chiama \emph{opposto} e si indica con $-a$.
E' quindi definita anche l'operazione di sottrazione: $a-b = a+(-b)$.

Lo stesso vale in un campo $R$ dove, inoltre, ogni elemento non nullo ha 
anche un inverso moltiplicativo. Se $x\in R\setminus\ENCLOSE{0}$ il suo
inverso moltiplicativo si chiama \emph{reciproco} e si indica con $1/x$ oppure $x^{-1}$.
Dunque è definita anche l'operazione di divisione: $a/b = a\cdot (1/b)$
se $b\neq 0$.

In generale se $A$ è un anello allora valgono
le familiari proprietà:
\[
  0\cdot x = x\cdot 0 = 0, \qquad
  (-1)\cdot x = x \cdot (-1) = -x.
\]
Per la prima abbiamo: 
\[
  0\cdot x = 0\cdot x + x + (-x) = (0+1)\cdot x + (-x) = x + (-x) = 0
\]
e scrivendo gli addendi in ordine opposto si ottiene anche $x\cdot 0 = 0$.
Diremo che $0$ è \emph{elemento assorbente}
\mymargin{elemento assorbente}%
\index{elemento!assorbente}%  
\index{assorbente}%
per la moltiplicazione.
Allora possiamo dimostrare anche la seconda proprietà:
\[
   (-1)\cdot x = (-1)\cdot x + x + (-x) = (-1 + 1)\cdot x + (-x) = 0 + (-x) = -x
\]
e anche in questo caso scrivendo gli addendi in ordine opposto si ottiene $x\cdot(-1)=-x$.

Se $A$ è un campo, vale la legge di \emph{annullamento del prodotto}:
\index{annullamento!prodotto}%
\mymargin{annullamento prodotto}%
\[
  x\cdot y = 0 \iff x=0 \lor y=0.
\]
L'implicazione verso sinistra è la proprietà assorbente, che abbiamo già 
verificato. 
Viceversa se $x\cdot y = 0$ e $x\neq 0$ allora possiamo moltiplicare
ambo i membri per $x^{-1}$ e ottenere $y=x^{-1}\cdot 0 = 0$.

\begin{definition}[gruppo totalmente ordinato]
  \label{def:gruppo_ordinato}%
  \label{def:campo_ordinato}%
  Diremo che $G$ è un \emph{gruppo totalmente ordinato}
\mymargin{gruppo totalmente ordinato}%
\index{gruppo!ordinato}% 
  se $G$ è un gruppo
  (con operazione $*$),
  se è anche un insieme totalmente ordinato (con relazione $\le$)
  e se l'operazione del gruppo mantiene 
  l'ordinamento ovvero vale la proprietà
  \begin{enumerate}
    \item[1.] monotonia:
    se $x\le y$ allora per ogni $z$ si ha:
      \[
      x*z \le y*z \qquad\text{e}\qquad z*x \le z*y.
      \] 
  \end{enumerate}
  Diremo che il gruppo totalmente ordinato, 
  è \emph{denso} o \emph{continuo} se la relazione 
  d'ordine ha anche tali proprietà.

  Diremo che $R$ è un \emph{campo ordinato}
\mymargin{campo ordinato}%
\index{campo!ordinato}%
  se è un campo, 
  se rispetto alla addizione 
  è un gruppo abeliano totalmente ordinato
  (dunque l'addizione preserva l'ordinamento 
  $x\le y \implies x+z\le y+z$)
  e se inoltre anche la moltiplicazione per un numero positivo 
  mantiene l'ordinamento:
  \begin{enumerate}
    \item[2.] monotonia: 
    se $x\le y$ e $0\le z$ allora $x\cdot z \le y\cdot z$. 
  \end{enumerate} 

  Diremo che il campo ordinato è \emph{continuo} 
  se l'ordinamento è continuo.
\end{definition}

\begin{example}
Vedremo che $\ZZ$ (l'insieme dei numeri interi) è un esempio di gruppo additivo totalmente ordinato 
ma non denso.
\end{example}

\begin{example}
Vedremo che $\QQ$ (i numeri razionali) è un esempio di campo ordinato denso ma non continuo 
mentre $\RR$ (i numeri reali) è un (in un certo senso l'unico) esempio di campo ordinato continuo.
\end{example}

\begin{example}
Un esempio banale di gruppo totalmente ordinato denso e continuo è $X=\ENCLOSE{0}$.
\end{example}

Se $G$ è un gruppo additivo totalmente ordinato con elemento neutro $0$ si ha:
\[
  x\le y  \iff 0 \le y-x.
\]
Significa che l'ordinamento è univocamente determinato 
dal confronto con l'elemento neutro, ovvero dal sapere 
quali sono gli elementi positivi.

Se $G$ è un gruppo additivo totalmente ordinato, possiamo esprimere una proprietà di monotonia 
per le disuguaglianze strette: se $x<y$ allora $x+z<y+z$ e $z+x<z+y$.
Si tratta di osservare che se vale l'uguaglianza $x+z=y+z$,
sommando $-z$ ad ambo i membri si ottiene $x=y$.

Se $R$ è un campo ordinato e prendiamo $x\in R$, 
Possiamo osservare che $x\ge 0$ se e solo se $-x\le 0$. 
Infatti basta sommare alla prima disequazione $-x$
per ottenere la seconda e viceversa sommare $x$ alla seconda per 
ottenere la prima (monotonia dell'addizione).

Verifichiamo ora la validità della regola dei segni per il prodotto:
se $x\ge 0$ e $y\ge 0$ allora $x\cdot y \ge 0$,
$(-x)\cdot (-y) \ge 0$, $(-x)\cdot y = x \cdot (-y) \le 0$.
La prima proprietà, $x\cdot y\ge 0$ è data per ipotesi (proprietà di monotonia 
del prodotto applicata a $y\ge 0$ moltiplicando ambo i membri per $x\ge 0$).
Di conseguenza $-(x\cdot y)\le 0$, ovvero $(-x)\cdot y \le 0$ 
e $x\cdot (-y) \le 0$. 
Cambiando ancora segno a quest'ultima disuguaglianza si ottiene $(-x)\cdot(-y)\ge 0$.

Dunque possiamo dedurre che per ogni $x\in R$ si ha $x^2=x\cdot x\ge 0$
perché sia che $x\ge 0$ sia che $x\le 0$ il prodotto sarà sempre $\ge 0$.

In particolare possiamo dedurre che $1=1\cdot 1\ge 0$. 
E visto che per ipotesi di campo, $1\neq 0$ si deve avere $1>0$.

\begin{exercise}
Sia $R$ un campo ordinato, $m\in R$, $m>0$.
Allora $m^{-1}>0$.
Inoltre (se $m>0$) per ogni $x,y\in R$ si ha:
\[
  x<y \iff m\cdot x < m\cdot y.
\]
\end{exercise}

\begin{exercise}
Dimostrare che ogni campo ordinato è denso.
\end{exercise}
\begin{proof}[Svolgimento.]
Dati $x,y\in R$ con $x<y$ dobbiamo trovare $z\in R$ tale che $x<z<y$.
Se $x<y$ si ha, $x+x<y+x<y+y$, 
\end{proof}


