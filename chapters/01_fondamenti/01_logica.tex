\section{logica}

Se, come diceva Galileo, la matematica è il linguaggio della natura,
è importante che la matematica sia essa stessa espressa in un linguaggio
che risulti essere il più possibile oggettivo e non ambiguo.
La \emph{logica} è la disciplina matematica che si occupa
dello studio e della formalizzazione del linguaggio matematico.
In questo capitolo riassumiamo in maniera sintetica ed intuitiva
alcuni concetti e contemporaneamente
fissiamo le notazioni che verranno utilizzate nel seguito.

La logica studia i \emph{sistemi formali}%
\mymargin{sistemi formali}%
\index{sistemi formali} (anche detti \emph{sistemi logico-deduttivi}
o \emph{sistemi assiomatici)} che sono delle descrizioni meccaniche, non ambigue, 
di un linguaggio formale. 
Parliamo al plurale di \emph{sistemi formali} in quanto ogni ambito della matematica 
(o di altre scienze) potrebbe sviluppare un proprio sistema formale specializzato per quell'ambito. 
Il primo sistema formale è stato sviluppato da Euclide per descrivere le proprietà 
di punti, rette e circonferenze del piano (la \emph{geometria euclidea}).
Al tempo di Euclide la formalizzazione era ancora intuitiva e incompleta, 
la formalizzazione moderna 
di tale sistema è stata completata da Hilbert, dopo duemila anni.
Peano ha introdotto un sistema assiomatico per descrivere le proprietà dei numeri naturali.
Dedekind ha individuato gli assiomi per descrivere i numeri reali.
Cantor ha introdotto la \emph{teoria degli insiemi} all'interno della quale è stato 
possibile includere tutte le altre teorie matematiche. 
Tale teoria è stata formalizzata da Zermelo e Fraenkel
ed è questa la teoria che useremo nel nostro corso.

Tutti i sistemi formali descrivono un linguaggio.
Per descrivere un linguaggio dobbiamo dire innanzitutto quali sono i \emph{simboli}%
\mymargin{simboli}%
\index{simboli}
di quel linguaggio. In generale possiamo pensare ai simboli come alle lettere 
(o caratteri, nel linguaggio dell'informatica) che possono essere utilizzati per comporre le frasi.
Simboli tipici delle teorie matematiche sono ad esempio: 
\texttt{x}, \texttt{y}, \texttt{5}, \texttt{7}, 
\texttt{+}, \texttt{=}, \texttt{)}, \texttt{:} etc.
Nei linguaggi informatici i simboli corrispondono ai caratteri presenti sulla tastiera 
di un computer. 
Nelle lingue naturali si prenderebbero come simboli le singole lettere dell'alfabeto
a cui aggiungere eventualmente i caratteri per la punteggiatura.
Una sequenza finita di simboli si chiama \emph{formula}%
\mymargin{formula}%
\index{formula} (potremmo anche chiamarle \emph{frasi}).
Ad esempio: \texttt{x7)+} potrebbe essere una formula formata da quattro simboli.
Tra tutte le formule un sistema formale deve individuare quelle che si potrebbero chiamare 
\emph{formule ben formate} ovvero le formule a cui effettivamente vogliamo dare significato.
Ad esempio la formula $x+5=7$ potrebbe essere una formula ben formata perché siamo in grado 
di dargli un significato.
Il sistema formale non dà un significato alle formule ben formate 
(il significato è una estrapolazione della nostra mente) ma semplicemente deve dare delle regole 
per determinare quali siano le formule ben formate e quali no.
Visto che ogni simbolo che utilizziamo può essere rappresentato al computer, possiamo pensare 
alle formule come alle stringhe dei linguaggi di programmazione e possiamo pensare che il sistema formale 
deve descrivere un algoritmo (cioè un procedimento meccanico) in grado di determinare 
se una formula è ben formata oppure no.
Tra le \emph{formule ben formate} il sistema formale deve infine specificare quali 
siano i \emph{teoremi}. 
Di nuovo questo deve essere fatto mediante un algoritmo puramente meccanico 
in modo da garantire che i teoremi risultino oggettivi e universali: 
non ci può essere disaccordo sulla validità di un teorema, eventualmente 
ci può essere disaccordo 
sulla interpretazione di tale teorema.
Tipicamente i sistemi formali sono \emph{deduttivi}.
Nei sistemi \emph{deduttivi} si identificano alcune formule che vengono 
chiamate
 \emph{assiomi} e che vengono immediatamente riconosciuti come teoremi.
Ad esempio vedremo che il primo assioma della teoria degli insiemi è 
\[
  \exists X\colon \not \exists y\colon y\in X
\]
che si potrebbe leggere 
\begin{displayquote}
esiste un insieme $X$ per il quale nessun $y$ è elemento di $X$
\end{displayquote}
ed esprime l'esistenza dell'insieme vuoto.
Oltre agli assiomi in un sistema deduttivo vengono specificate delle 
\emph{regole di inferenza} cioè dei modi in cui 
le formule possono essere modificate o composte in modo tale che se le formule 
di partenza sono teoremi anche la formula ottenuta lo è.
I \emph{sillogismi} di Aristotele possono essere utilizzati come esempi di regole 
di inferenza. 
Supponiamo che le formule 
\texttt{"Socrate è un uomo"} 
e \texttt{"l'uomo è un animale"}
siano entrambe teoremi. Allora possiamo pensare di definire una regola di inferenza 
che mi dice che se \texttt{"X è un Y"} è un teorema 
e \texttt{"Y è uno Z"}
è un teorema allora anche \texttt{"X è uno Z"} è un teorema. 
Con questa regola di inferenza è quindi possibile dedurre che 
\texttt{"Socrate è un animale"}
è un teorema.
E' chiaro che se le regole formali possono essere definite meccanicamente 
tramite un algoritmo allora anche i teoremi possono essere determinati meccanicamente 
mediante un algoritmo.
La ricerca in matematica consiste nell'esplorare lo spazio delle formule ben formate 
per determinare quali siano effettivamente teoremi. 
Dare la \emph{dimostrazione}%
\mymargin{dimostrazione}%
\index{dimostrazione} di un teorema significa esibire tutta la catena delle formule 
e delle regole di inferenza che permettono di ottenere il teorema a partire dagli 
assiomi.

\subsection{proposizioni, operatori logici}

Tutto questo è un processo meccanico, ma in realtà è ovvio che i sistemi formali 
vengono definiti in modo tale da aver per noi un qualche tipo di significato intuitivo.
In tal modo la ricerca delle dimostrazioni non è un processo puramente meccanico 
ma segue delle linee di pensiero che possono richiedere intuizione, inventiva e anche 
senso estetico. 
Tipicamente a livello intuitivo vogliamo assegnare un valore di verità alle formule 
ben formate: vorremmo cioè dire che alcune formule sono 
\emph{vere}%
\mymargin{vero}%
\index{vere} 
ed altre sono 
\emph{false}%
\mymargin{falso}%
\index{false}. 
In tal caso le formule ben formate vengono usualmente chiamate \emph{proposizioni}%
\mymargin{proposizioni}%
\index{proposizione}
(nel linguaggio naturale diremmo: \emph{affermazioni}).
Un esempio di proposizione (falsa) potrebbe essere: 
\texttt{2+2=5}.

E' possibile combinare più proposizioni mediante
gli operatori logici. Se $P$ e $Q$ sono proposizioni
si può costruire la proposizione $P \land Q$
chiamata \emph{congiunzione logica}%
\mymargin{congiunzione}%
\index{congiunzione logica}.
Tale proposizione
si può leggere ``$P$ e $Q$'' ed è una proposizione
che risulta essere vera solamente nel caso in cui sia
$P$ che $Q$ siano vere
(si veda la tabella~\ref{tab:verita_operatori_logici}
per un riassunto schematico).
Spesso la congiunzione logica è sottointesa:
se si fa un elenco di proposizioni $P,Q,R$ 
si intende usualmente la loro congiunzione $P \land Q \land R$
cioè si intende che devono essere tutte vere.
La \emph{disgiunzione logica}%
\mymargin{disgiunzione}%
\index{disgiunzione logica} denotata
con $P \lor Q$
si può leggere ``$P$ o $Q$'' ed è una proposizione che
è vera se almeno una tra $P$ e $Q$ è vera.
La \emph{negazione logica}%
\mymargin{negazione}%
\index{negazione logica} denotata con $\lnot P$ è una
proposizione che si può leggere ``non $P$'' che
è vera quando $P$ è falsa ed è falsa quando $P$ è vera.

Operatori logici molto utilizzati sono le \emph{implicazioni}%
\mymargin{implicazioni}%
\index{implicazione}.
La proposizione $P\Rightarrow Q$ si può leggere ``$P$ implica $Q$''
e significa che $Q$ è vera se $P$ è vera. Non si confonda
il valore di verità di $P\Rightarrow Q$ con il valore di verità
di $Q$. Se $P$ è vera allora $P\Rightarrow Q$ è vera o falsa
a seconda che $Q$ sia vera o falsa. Ma se $P$ è falsa allora
l'implicazione $P\Rightarrow Q$ è vera indipendentemente dal
valore di $Q$. In effetti $P\Rightarrow Q$ è equivalente a
$Q \lor \lnot P$ perché per la verità di $P\Rightarrow Q$
basta che $Q$ sia vera (quando $P$ è vera) oppure che $P$ sia falsa.

La freccia inversa $P\Leftarrow Q$ si può utilizzare per
invertire l'implicazione: è equivalente a $Q \Rightarrow P$.
Se valgono entrambe le implicazioni
$(P \Leftarrow Q) \land (P\Rightarrow Q)$
è facile convincersi che $P$ e $Q$ devono avere lo stesso
valore di verità: diremo quindi che sono equivalenti e
scriveremo $P \Leftrightarrow Q$.

Nella tabella~\ref{tab:verita_operatori_logici} sono riportati
tutti i valori di verità che si possono ottenere combinando
tra loro due proposizioni. Nella tabella~\ref{tab:operatori_logici}
sono riportate alcune proprietà di tali operatori: queste
proprietà possono essere comprese interpretando il loro significato
ma possono anche essere dedotte meccanicamente. 
Per verificare meccanicamente la validità di una di queste 
espressioni logiche è sufficiente fare una tabella in cui si 
inseriscono tutti i possibili valori di verità di $P$, $Q$ ed $R$
(in totale saranno $8$ casi) e per ognuno di questi si dovrà
calcolare il valore di verità di ogni operazione svolta e verificare 
che ogni espressione completa assume il valore $V$ (vero).
Diremo che queste sono \emph{tautologie}
\index{tautologia}%
\mymargin{tautologia}%
ovvero espressioni logiche la cui validità non dipende
dal valore di verità dei suoi termini.

\begin{table}
\begin{center}
  \begin{tabular}{cc|cccccc}
    $P$ & $Q$ & $\neg P$ & $P\land Q$ & $P\lor Q$ & $P\Rightarrow Q$ &
    $P\Leftarrow Q$ & $P\Leftrightarrow Q$ \\\hline
    \texttt{F} & \texttt{F} & \texttt{V} & \texttt{F} & \texttt{F} & \texttt{V} & \texttt{V} & \texttt{V} \\
    \texttt{F} & \texttt{V} & \texttt{V} & \texttt{F} & \texttt{V} & \texttt{V} & \texttt{F} & \texttt{F} \\
    \texttt{V} & \texttt{F} & \texttt{F} & \texttt{F} & \texttt{V} & \texttt{F} & \texttt{V} & \texttt{F} \\
    \texttt{V} & \texttt{V} & \texttt{F} & \texttt{V} & \texttt{V} & \texttt{V} & \texttt{V} & \texttt{V} \\
    \end{tabular}
\end{center}
\caption{La tabella di verità degli operatori logici. 
$\texttt{F}$ significa \emph{falso}, $\texttt{V}$ significa \emph{vero}.}
\label{tab:verita_operatori_logici}
\end{table}

\begin{table}
\begin{tabular}{rcll}
                          &$\neg (P \land \neg P)$&                              & non contraddizione \\
                         &$P \lor \neg P$&                                       & terzo escluso \\
                         $\neg \neg P$ & $\iff$ & $ P$                           & doppia negazione\\
                                    $P \land Q$ & $\iff$ & $ Q \land P$                   & simmetria\\
                                     $P \lor Q$ & $\iff$ & $ Q \lor P$                    & \\
                              $\neg (P\land Q)$ & $\iff$ & $ (\neg P) \lor (\neg Q)$      & formule di De Morgan\\
                               $\neg (P\lor Q)$ & $\iff$ & $ (\neg P) \land (\neg Q)$     & \\
                            $(P\land Q) \lor R$ & $\iff$ & $ (P\lor R) \land (Q \lor R)$  & proprietà distributiva\\
                            $(P\lor Q) \land R$ & $\iff$ & $ (P\land R) \lor (Q \land R)$ & \\
                            $(P \Rightarrow Q)$ & $\iff$ & $ (Q \Leftarrow P)$            & antisimmetria\\
                            $(P\Rightarrow Q)$ & $\iff$ & $ (\neg Q\Rightarrow\neg P)$   & implicazione contropositiva\\
                        $\neg (P\Rightarrow Q)$ & $\iff$ & $ P \land (\neg Q)$            & controesempio\\
                             $P\Rightarrow Q$ & $\iff$ & $ \lnot(P \land (\neg Q))$     & dimostrazione per assurdo
\end{tabular}
\caption{Alcune proprietà degli operatori logici. Queste proposizioni sono tutte vere qualunque siano i valori di verità delle proposizioni $P$ e $Q$. In particolare le equivalenze logiche ci dicono che le proposizioni ai due lati dell'equivalenza assumono sempre lo stesso valore di verità e quindi sono interscambiabili.}
\label{tab:operatori_logici}
\end{table}

Per definire formalmente la logica del calcolo proposizionale
dobbiamo avere un sistema sottostante che ci dice chi sono 
le proposizioni di base, ovvero i termini che possiamo sostituire al posto 
di $P$, $Q$, $R$. 
Queste saranno proposizioni (ovvero formule ben formate).
Una volta definite le proposizioni di base,
ogni formula della forma
$(P)\land(Q)$, $(P)\lor(Q)$, $(P)\implies(Q)$, 
$(P)\impliedby (Q)$, $(P)\Leftrightarrow(Q)$ e
$\lnot(P)$
è anch'essa una proposizione (ovvero è ben formata) 
se $P$ e $Q$ lo sono. 
Il sistema formale utilizza i simboli: 
$\land$ $\lor$ $\implies$ $\impliedby$ $\Leftrightarrow$
$\lnot$ $($ $)$ oltre a tutti i simboli che servono per 
comporre le proposizioni di base.

Il caso più semplice possibile è quello in cui abbiamo 
solamente due proposizioni di base $F$ e $V$ e due assiomi: 
$V$ e $\lnot (F)$. 
L'interpretazione del sistema è che $V$ è vero mentre $F$ è falso.

Le regole di inferenza del calcolo delle proposizioni
possono essere date in modi diversi. 
\index{regole di inferenza}%
\mymargin{regole di inferenza}%
Proviamo a spiegare (senza troppi formalismi) il sistema 
che viene chiamato \emph{deduzione naturale}. 
Questo sistema presenta per ognuno degli operatori logici
(negazione, congiunzione, disgiunzione, implicazione) 
due regole formali: una che permette l'introduzione dell'operatore e una 
che permette l'eliminazione l'operatore.

Le regole più semplici sono quelle che riguardano la congiunzione 
logica. Se $P$ e $Q$ sono teoremi possiamo dedurre che anche 
$(P)\land (Q)$ è un teorema (introduzione della congiunzione).
\index{congiunzione logica}%
\index{introduzione!congiunzione}%
\mymargin{introduzione congiunzione}%
Viceversa se $(P)\land (Q)$ è un teorema possiamo dedurre 
che sia $P$ che $Q$ sono teoremi (eliminazione della congiunzione).
\index{eliminazione!congiunzione}%
\mymargin{eliminazione congiunzione}%
Ad esempio se abbiamo il teorema $1+1=2$ e il teorema $2+1=3$ 
allora possiamo dedurre che $(1+1=2) \land (2+1=3)$ è un teorema.
Viceversa se $(1+1=2)\land (2+1=3)$ è un teorema allora possiamo 
dedurre che $2+1=3$ è un teorema.

Anche l'introduzione della disgiunzione logica è una regola 
\index{disgiunzione logica}%
molto semplice: se $P$ è un teorema allora anche $(P)\lor (Q)$ 
e $(Q)\lor (P)$ sono teoremi, qualunque sia la proposizione $Q$.
\index{introduzione disgiunzione}%
\mymargin{introduzione disgiunzione}%
Ad esempio se $1+1=2$ è un teorema possiamo dedurre 
che anche $(1+1=2) \lor (1+1\neq 2)$ è un teorema.
La regola di eliminazione della disgiunzione è leggermente 
\mymargin{eliminazione disgiunzione}%
più complicata. Se  $(P)\lor (Q)$ è un teorema e se entrambi 
$(P)\implies (R)$ e $(Q)\implies (R)$ sono teoremi, possiamo 
dedurre che $R$ è un teorema.
Ad esempio se abbiamo i teoremi: 
\[
\begin{gathered}
  (x>1) \implies (x^2>1)\\ 
  (x<-1) \implies (x^2>1)\\  
  (x>1) \lor (x<-1)
\end{gathered}
\]
allora possiamo dedurre che $x^2>1$ 
è un teorema.

Per quanto riguarda l'implicazione logica, la regola di eliminazione 
\index{implicazione logica}%
si chiama \emph{modus ponens} ed è forse la più importante 
\index{modus ponens}%
\mymargin{modus ponens}%
delle regole di inferenza.
La regola ci dice che se abbiamo il teorema 
$P$ e il teorema $(P)\implies (Q)$ possiamo affernare che anche 
$Q$ è un teorema.
Ad esempio se abbiamo dimostrato $(1+1=2) \implies (2+1=3)$ 
e abbiamo dimostrato $1+1=2$ allora possiamo dedurre $2+1=3$.
Questa regola formale dà significato all'implicazione logica.

La regola di \emph{introduzione della implicazione} è più complicata 
\mymargin{introduzione implicazione}%
da formalizzare e richiede l'introduzione di un nuovo concetto:
il \emph{contesto} dimostrativo.
\index{contesto dimostrativo}%
\mymargin{contesto dimostrativo}%
Un \emph{contesto} è in pratica un nuovo sistema formale che 
eredita tutti gli assiomi e le regole formali del sistema formale 
originario ma con delle proprietà in più che permettono tipicamente di 
ottenere teoremi che non era possibile ottenere nel sistema formale 
originale.
Per l'introduzione dell'implicazione logica la regola 
è la seguente. Per poter affermare che $(P)\implies (Q)$ 
è un teorema introduciamo un nuovo contesto in cui 
tutti i teoremi già dimostrati sono assiomi e inoltre 
anche $P$ è un assioma aggiuntivo. 
Se in questo contesto riusciamo ad ottenere $Q$ come
teorema allora possiamo dire che $(P)\implies (Q)$ è un teorema 
nel sistema formale originale.
Si noti che $Q$ potrebbe non essere un teorema nel contesto 
orginale, lo è solo nel contesto ipotetico in cui abbiamo 
supposto (per ipotesi, appunto) che $P$ fosse un teorema.

Proviamo per esempio a dimostrare la regola più controversa
delle tavole di verità ovvero
che $F\implies V$ (dove $F$ sta per falso e 
$V$ sta per vero). 
Quello che dobbiamo fare è verificare che se $\lnot P$ 
è un teorema (cioè $P$ è falso) e se $Q$ è un teorema 
(cioè $Q$ è vero) allora $P\implies Q$ è un teorema.
Dovremmo usare la regola di introduzione dell'implicazione logica.
Dunque supponiamo per ipotesi che $P$ sia un teorema. 
Visto che $Q$ era un teorema nel contesto originale 
$Q$ rimane un teorema nel contesto ipotetico in cui 
abbiamo supposto che $P$ sia un assioma.
Possiamo quindi, fuori dal contesto, dedurre $P\implies Q$ 
come volevasi dimostrare.

Ad esempio al posto di $P$ possiamo mettere $2\neq 2$ 
e al $Q$ possiamo mettere $1=1$. 
Vorremo allora dimostrare che $2\neq 2 \implies 1=1$ 
è un teorema.
Proviamo a farlo formalmente con tutti i dettagli.

Supponiamo di avere un sistema formale 
contenente i simboli $1$, $2$, $=$, $\neq$ oltre che
tutti i simboli della logica proposizionale 
(ovvero $\land$, $\lor$, $\lnot$, $\implies$, $($ e $)$).
Chiamiamo \emph{costanti} i simboli $1$ e $2$.
Supponiamo che tutte le formule della forma 
$x\neq y$ e $x=y$ siano proposizioni 
(formule ben formate) se al posto di $x$ e $y$ 
mettiamo qualunque costante. 
Supponiamo che $x\neq y$ sia un sinonimo di $\lnot (x=y)$,
Supponiamo infine che $x=x$ sia un assioma se al posto 
di $x$ mettiamo qualunque costante.

Fuori dal sistema formale, nell'esibire una dimostrazione,
faremo un elenco dei nostri teoremi, numerando ogni riga 
per poterci fare riferimento. 
Usiamo le parentesi graffe $\{$ e $\}$ per racchiudere 
i contesti dimostrativi. 
Tra parentesi quadre indichiamo le regole formali che stiamo 
utilizzando. 
\begin{theorem}[esempio di dimostrazione formale]
  $2 \neq 2 \implies 1=1$.
\end{theorem}
%  
\begin{proof}
  \begin{align}
    &\{ & &\text{[inizio contesto ipotetico]} \\ 
    &\quad 2\neq 2 & & \text{[ipotesi]}  \\
    &\quad 1=1 & & \text{[assioma]} \\
    &\} & & \text{[fine del contesto ipotetico]} \\
    &2\neq 2 \implies 1=1 & & \text{[introduzione implicazione]}
  \end{align}
\end{proof}

% Ad esempio possiamo dimostrare che 
% $1=2 \implies (2=3 \implies 1=3)$.
% Si inizia con l'introdurre un contesto ipotetico in cui $1=2$.
% In questo contesto introduciamo un nuovo \emph{sotto}-contesto 
% in cui anche $2=3$. Nel sotto-contesto abbiamo come teoremi
% (ipotetici) sia $1=2$ che $2=3$. 
% Un assioma dell'uguaglianza (che non abbiamo ancora menzionato)
% permetterà di dire che se $x=y$ allora in un qualunque teorema 
% si possono rimpiazzare le occorrenze di $x$ con $y$ e viceversa.
% Dunque se (ipoteticamente) $1=2$ e $2=3$ possiamo rimpiazzare 
% $2$ con $1$ nell'uguaglianza $2=3$ per ottenere $1=3$.
% Si noti che $1=3$ è falso. 
% Ma sarebbe effettivamente un teorema nell'ipotesi (assurda)
% che $1=2$ e $2=3$.
% Dunque possiamo dedurre $2=3 \implies 1=3$ nel contesto 
% in cui abbiamo ipotizzato $1=2$. 
% E possiamo dedurre $1=2 \implies (2=3 \implies 1=3)$ nel
% contesto generale. Quest'ultimo è un vero e proprio teorema,
% valido senza ipotesi aggiuntive.

Ritorniamo alle regole di inferenza del calcolo proposizionale,
in particolare alla negazione logica. 
\index{negazione logica}%
\mymargin{rimozione negazione}%
Per la rimozione la regola è quella della doppia negazione: 
se $\lnot (\lnot (P))$ è un teorema allora anche $P$ è un teorema.
Ad esempio se $\lnot (\lnot (1=1))$ è un teorema allora anche $1=1$ 
è un teorema.
\mymargin{introduzione negazione}%
La regola di introduzione della negazione logica è la 
dimostrazione per assurdo.
\index{dimostrazione!per assurdo}%
\index{assurdo!dimostrazione per}%
Se introduciamo un contesto formale in cui assumiamo (per assurdo)
che $P$ sia un teorema e, nel contesto, riusciamo ad ottenere 
una contraddizione (ad esempio $\lnot (P)$) allora possiamo 
affermare che (fuori dal contesto) $\lnot (P)$ è un teorema.

Ad esempio se $1=1$ è un teorema possiamo dimostrare 
che anche $\lnot \lnot (1=1)$ è un teorema.
Infatti supponiamo per assurdo $\lnot \lnot \lnot (1=1)$.
Allora, tramite eliminazione della doppia negazione possiamo 
dedurre $\lnot (1=1)$. Ma questo è in contraddizione 
con $\lnot \lnot (1=1)$. Dunque abbiamo dimostrato che 
$\lnot \lnot (1=1)$.

Le regole formali che abbiamo introdotto sono sufficienti 
per dimostrare la validità delle tabelle di verità degli 
operatori logici.

\begin{exercise}
Si provino ad utilizzare le regole formali di inferenza del 
calcolo proposizionale per dimostrare la validità delle 
tavole di verità.
\end{exercise}

\subsection{predicati, quantificatori}

Un \emph{predicato}%
\mymargin{predicato}%
\index{predicato} è una proposizione che contiene
una o più variabili il cui valore di verità, quindi,
può dipendere dal valore assegnato alle variabili.
Un esempio di predicato è $x+2=5$ che risulta essere vero se $x=3$
e falso altrimenti.

Le variabili $x,y,\dots$ non sono altro che simboli 
del sistema formale. 
Usualmente vengono utilizzate le ultime lettere dell'alfabeto.
Siccome le lettere dell'alfabeto sono in numero finito
e visto che vogliamo invece avere la possibilità di introdurre 
sempre nuove variabili, decidiamo che 
una variabile può essere composta da una sequenza di più simboli. 
Si può ad esempio usare l'apice: $x$, $x'$, $x''$ oppure 
usare una sequenza di lettere: $\textrm{area}$, $\sin$
per avere una quantità illimitata di nomi diversi a disposizione.

Se un predicato dipende da una o più variabili queste
si chiamano \emph{variabili libere}%
\mymargin{variabili libere}%
\index{variabili libere}. E' possibile
\emph{chiudere} una variabile libera mediante un quantificatore.
Il \emph{quantificatore universale}%
\mymargin{quantificatore universale}%
\index{quantificatore universale} denotato col simbolo
$\forall$ serve ad affermare che il predicato è vero
per ogni possibile valore della variabile quantificata.
Ad esempio la proposizione $\forall x\colon x+2=5$ significa:
``per ogni $x$ si ha $x+2=5$'' ed è una proposizione falsa.
Il quantificatore $\forall x$ rende \emph{muta} la variabile
$x$ del predicato $x+2=5$ nel senso che il valore di verità 
della proposizione
non dipende più dal valore di quella variabile, che non è
più una variabile libera ma funge solo da segnaposto.
In effetti se cambio nome ad una variabile muta il valore 
di verità non cambia. Ad esempio scrivere $\forall x\colon x+1=1+x$
è equivalente a $\forall z\colon z+1=1+z$.

Il \emph{quantificatore esistenziale}%
\mymargin{quantificatore esistenziale}%
\index{quantificatore esistenziale} denotato col simbolo
$\exists$ serve ad affermare che il predicato è vero per
almeno un valore della variabile quantificata.
Ad esempio la proposizione $\exists x\colon x+2=5$ significa:
``esiste almeno un $x$ per cui risulta $x+2=5$''.
Quello che si ottiene è una proposizione in cui la variabile
$x$ è muta. In questo caso la proposizione è vera in quanto
per $x=3$ il predicato è vero.

Un predicato può dipendere da più variabili ed è quindi
possibile inserire più quantificatori. In tal caso l'ordine
dei quantificatori può essere rilevante.
Delle seguenti proposizioni la prima è vera, la seconda
invece è falsa:
\begin{gather*}
\forall x\colon \exists y\colon x+2=y \\
\exists y\colon \forall x\colon x+2=y.
\end{gather*}
Intuitivamente la prima affermazione ci dice che per ogni numero $x$ 
esiste un numero $y$ che differisce di $2$ da $x$.
La seconda ci dice che c'è un numero $y$ che differisce di $2$ 
da qualunque numero $x$.

\begin{table}
  \begin{center}
    \begin{tabular}{lcl}
    $\lnot \forall x \colon P(x)$ & $\iff$ & $\exists x \colon \lnot P(x)$\\
    $\lnot \exists x \colon P(x)$ & $\iff$ & $\forall x \colon \lnot P(x)$\\
    \end{tabular}
  \end{center}
  \caption{Le formule di De Morgan per i quantificatori logici.}
  \label{tab:proprieta_quantificatori}
\end{table}

Risulta utile saper trasformare un quantificatore logico
nell'altro per mezzo delle negazioni con le formule 
di De Morgan riportate
nella tabella~\ref{tab:proprieta_quantificatori}.

Spesso il quantificatore universale viene sottointeso.
Ad esempio se si scrive $x+y=y+x$
si intende che tutte le variabili libere sono quantificate 
tramite quantificatore universale: 
$\forall x\colon \forall y\colon x+y=y+x$.
Il segno di interpunzione $\colon$ può essere omesso
e una quantificazione può essere fatta contemporaneamente 
su più variabili. 
Potremo quindi scrivere: $\forall x\forall y\colon x+y=y+x$ 
o anche $\forall x,y\colon x+y = y+x$.
A volte, se non crea ambiguità, si potranno scrivere 
i quantificatori anche a fine frase 
\[
  x+y=y+x \quad \forall x\forall y.
\]

Una proposizione non può contenere variabili libere, 
altrimenti il suo valore di verità potrebbe dipendere da quelle 
variabili e non sarebbe quindi ben definito.
\mymargin{costanti}%
\index{costanti}%
Si può però intendere che alcune variabili sono \emph{costanti},
cioè rappresentano un oggetto ben preciso.
Ad esempio nella proposizione $\forall x: \pi > -x^2$ si intende 
che $\pi$ è una costante, ovvero un ben preciso numero.
Anche le cifre $0$, $1$, $2$ sono usualmente sempre intese come
costanti.

Insieme al calcolo dei predicati si introduce usualmente anche il simbolo di uguaglianza \texttt{=}.
Come viene definita l'uguaglianza dipende da come sono definiti gli oggetti trattati 
dal nostro sistema formale, cosa che noi faremo per la teoria degli insiemi nel prossimo paragrafo.
Ma comunque venga definita l'uguaglianza si richiede che valga il seguente.
\begin{axiom}[uguaglianza]
\[
  \forall x\colon x=x.  
\]
\end{axiom}
Inoltre si richiede che oggetti uguali possano essere 
sostituiti uno all'altro in un teorema:
se $x=y$ e $P(x)$ sono teoremi allora anche $P(y)$ lo è.

Anche per il calcolo dei predicati dobbiamo specificare le regole 
di inferenza per l'introduzione e l'eliminazione dei quantificatori.

\mymargin{eliminazione $\forall$}%
\index{quantificatore!universale}%
L'eliminazione del quantificatore universale è molto semplice. 
Se abbiamo un teorema della forma $\forall x\colon P(x)$
e se nel contesto dimostrativo è stata introdotta una costante 
$c$, allora possiamo dedurre $P(c)$.
\mymargin{introduzione $\forall$}%
Viceversa, per introdurre il quantificatore universale 
dobbiamo aprire un contesto dimostrativo in cui supponiamo 
che $x$ sia una costante di cui non abbiamo alcuna informazione 
(cioè $x$ non è usata come costante al di fuori del contesto).
Se nel contesto riusciamo a dimostrarare una proposizione $P(x)$ possiamo 
allora dedurre, fuori contesto, il teorema $\forall x\colon P(x)$.

In pratica si dirà: fissato $x$ qualunque, abbiamo dimostrato $P(x)$
quindi possiamo affermare che $\forall x\colon P(x)$.

\index{quantificatore!esistenziale}%
\mymargin{eliminazione $\exists$}%
Per quanto riguarda il quantificatore esistenziale la regola 
di eliminazione ci dice che se abbiamo un teorema della forma 
$\exists x\colon P(x)$ possiamo introdurre un contesto dimostrativo 
in cui è definita una costante $c$ che ha la proprietà $P(c)$.
Se in tale contesto riesco ad ottenere un teorema che non dipende 
da $c$ allora quel teorema è valido anche fuori dal contesto.
\mymargin{introduzione $\exists$}%
La regola di introduzione ci dice invece che
se abbiamo un teorema della forma $P(c)$,
con $c$ una qualunque costante,
possiamo allora dedurre $\exists x\colon P(x)$.

Come esempio enunciamo e dimostriamo formalmente un semplice 
teorema.

\begin{theorem}[esempio di dimostrazione formale]
  \[ \forall x \exists y\colon x=y \]
\end{theorem}
\begin{proof}
  \begin{align}
    &\forall x\colon x = x & & \text{[assioma]} \label{ff1}\\
    &\{ & &\text{[fissiamo $x$ qualunque]}\\
    &\quad x=x  & & \text{[eliminazione $\forall$ da \eqref{ff1}]}\\ 
    &\quad \exists y\colon x=y & &\text{[introduzione $\exists$]}\\
    &\} & &\\
    &\forall x\exists y\colon x=y & &\text{[introduzione $\forall$]} 
  \end{align}
\end{proof}

