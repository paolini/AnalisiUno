\section{note storiche}

\label{nota:Peano}%
\index{Peano!Giuseppe}%
\emph{Giuseppe Peano} (1858--1932), matematico torinese, contribuì a porre 
i fondamenti della logica matematica. 
La notazione $\exists$ per il quantificatore universale si deve a lui.
La definizione originale di Peano prendeva $1$ come primo numero
naturale ma nella matematica moderna risulta più comodo includere anche $0$ 
tra i numeri naturali, così come si considera il vuoto tra gli insiemi.

\label{nota:Galileo}%
\index{Galileo!Galilei}%
\emph{Galileo Galilei} (1564--1642) osservò che i quadrati 
perfetti: $1,4,9,16,\dots$ sono da un lato una piccola parte 
di tutti i numeri naturali (questi numeri si distanziano 
sempre di più tra loro) ma d'altro canto sono tanti quanti i numeri naturali 
perché la corrispondenza $n\mapsto n^2$ è biunivoca.

\label{nota:Cantor}%
\index{Cantor!Georg}%
\label{nota:Russell}%
\index{Russell!Bertrand}%
\label{nota:Frege}%
\index{Frege!Gottlob}%
La teoria degli insiemi
è stata introdotta da \emph{Georg Cantor} (1845--1918) senza una vera formalizzazione logica
(oggi la chiameremmo \emph{teoria ingenua degli insiemi}).
\emph{Gottlob Frege} (1848--1925) fu il primo matematico che tentò di formalizzare 
la teoria degli insiemi di Cantor. 
Nel 1902 \emph{Bertrand Russell}, avendo letto il lavoro di Frege, 
gli invio una lettera che enunciava il paradosso da lui scovato:
``Mi trovo in completo accordo con lei in tutte le parti essenziali, in particolare
quando lei rifiuta ogni elemento psicologico dando un grande valore
all'ideografia %[Begriffsschrift]
per il fondamento della matematica e della logica formale [\dots] c'è solo
un punto dove ho incontrato una difficoltà [...]''.
La risposta di Frege (22 giugno 1902) è deprimente:
``La sua scoperta della contraddizione mi ha causato una grandissima sorpresa e,
direi, costernazione, perché ha scosso le basi su cui intendevo costruire l'aritmetica.''

\label{nota:Euclide}%
\label{nota:Hilbert}%
\index{Euclide}%
\index{assiomi!di Euclide}%
\index{Hilbert!David}%
\index{geometria euclidea}%
\emph{Euclide} (circa 300 a.C.) ha introdotto il metodo assiomatico in geometria. 
Il suo trattato ``gli Elementi'', è considerato il libro
che in assoluto ha avuto maggiore impatto nella storia della matematica.
Gli assiomi introdotti da Euclide corrispondono alle costruzioni geometriche fatte 
con riga e compasso.
Se consideriamo l'insieme dei punti che possono essere costruiti a partire da 
un insieme finito di punti dati, otteniamo un insieme denso che però 
non è completo. 
E' ben noto, infatti, che tramite riga e compasso non è possibile 
fare la \emph{trisezione dell'angolo} (cioè dividere 
un angolo generico in tre parti uguali)
né la \emph{duplicazione del cubo}
(cioè costruire il lato di un cubo il cui volume sia il doppio di un cubo di dato lato)
né la \emph{quadratura del cerchio} (cioè costruire il lato di un quadrato
con la stessa area di un cerchio di dato raggio).
I primi due problemi richiedono la costruzione delle radici cubiche, mentre 
il terzo problema richiede la costruzione di $\sqrt\pi$.
Tramite la teoria di Galois si è dimostrato che tramite costruzioni con riga e 
compasso si possono costruire solo quei numeri che si possono esprimere 
a partire dai numeri interi utilizzando, oltre alle quattro operazioni, 
l'estrazione di radice quadrata. 
E' facile costruire, ad esempio, la diagonale di un quadrato di dato lato: 
questo corrisponde alla costruzione di $\sqrt 2$.
Ovvio invece che i numeri trascendenti (come $\sqrt\pi$) non possono essere costruiti, visto 
che le operazioni ammissibili non ci fanno uscire dall'insieme dei numeri algebrici.
La costruzione di un poligono regolare inscritto in un cerchio di dato raggio 
è equivalente alla costruzione delle radici complesse $n$-esime dell'unità, ed 
è quindi legato alla costruibilità di particolari numeri algebrici. 
Solo grazie all'apporto di Gauss (1777-1855) c'è stato un avanzamento rispetto alle conoscenze di Euclide 
su quali fossero i poligoni regolari costruibili.
Sorprendentemente 
mentre i poligoni regolari con $7$, $9$, $11$ e $13$ lati non sono costruibili 
con riga e compasso, il poligono regolare con $17$ lati è costruibile
(Gauss-Wentzel). 

Tutta questa discussione dovrebbe rendere chiaro che non è ovvio come si possa 
definire lo spazio geometrico in modo rigoroso 
in modo che risulti essere completo. 
In effetti solo nel 1900 \emph{David Hilbert} (1862--1943) ha dato una definizione assiomatica rigorosa
della geometria euclidea. 
La proprietà di completezza viene catturata da Hilbert mediante un assioma di massimalità:
lo spazio euclideo è un insieme di punti che soddisfano certi assiomi 
(gli usuali postulati di Euclide risistemati da Hilbert) e inoltre 
è il più grande insieme di punti con tali proprietà. 
Dunque deve contenere i limiti delle successioni di Cauchy 
(oppure i punti di separazione degli insiemi lineari separati, 
se pensiamo alla continuità dell'ordinamento invece che alla completezza) 
perché altrimenti tali punti potrebbero essere aggiunti allo spazio senza 
violare gli altri assiomi ma violando quindi la massimalità.


\label{nota:Dedekind}
\index{Dedekind!Richard}%
Richard Dedekind (1831--1916) per spiegare
le motivazioni che lo hanno portato a trovare una 
definizione rigorosa dei numeri reali,
scrive:
``Viene spesso sostenuto che il calcolo differenziale 
tratta le grandezze continue e, 
tuttavia, non viene mai data una spiegazione 
di questa continuità; perfino le esposizioni 
più rigorose del calcolo differenziale non basano
le loro dimostrazioni sulla continuità ma, 
in maniera più o meno consapevole, si appellano o 
a nozioni geometriche o suggerite dalla geometria,
oppure, dipendono da teoremi che non 
non sono stati dimostrati in modo 
puramente aritmetico.''