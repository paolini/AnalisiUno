\section{i numeri naturali}
\label{sec:naturali}

Sebbene le proprietà dei numeri naturali dovrebbero essere intuitivamente già ben note,
la loro formalizzazione non è banale.
In particolare alcune dimostrazioni risultano complicate perché quando dobbiamo dimostrare 
molti fatti ovvi, bisogna fare attenzione a non utilizzare proprietà che ci risultano ovvie 
ma che ancora non sono state dimostrate.
Per questo motivo è perfettamente legittimo se in prima lettura saltiamo le dimostrazioni 
per capire il senso generale del discorso e poi, se necessario, torniamo indietro per
approfondire i dettagli.

I numeri naturali $0,1,2,\dots$ sono i numeri che utilizziamo per contare o per 
fare le iterazioni. C'è un primo numero naturale, che per noi sarà $0$, e poi
per ogni numero naturale $n$ ce n'è uno successivo che chiameremo $\sigma(n)$. 
Partendo da $0$ e passando al successivo si raggiungono tutti i numeri naturali.

\begin{definition}[assiomi di Peano]
  \label{def:assiomi_peano}%
  \label{def:naturali}%
  \mynote{Giuseppe Peano (1858--1932) vedi note storiche a pag.~\pageref{nota:Peano}} 
  \index{numeri!naturali}%
  \index{$\NN$}%
  \index{Peano}%
  \index{assiomi!di Peano}%
  \index{insieme!induttivo}%
  \index{induttivo}%
Dato un insieme $\NN$, 
un elemento $0\in \NN$ 
e una funzione $\sigma\colon \NN\to\NN$ 
diremo che $(\NN,0,\sigma)$
soddisfa gli assiomi di Peano,
o più semplicemente che $\NN$ è un insieme di numeri naturali,
se valgono le seguenti proprietà: 
\mynote{La funzione $\sigma$ ci dà il successore di ogni numero. 
Definiremo $1=\sigma(0)$, $2=\sigma(2)$, $3=\sigma(2)$, \dots
Il primo assioma di Peano dice che numeri diversi hanno 
numeri successori diversi. 
Il secondo assioma ci dice che il primo 
numero naturale, per noi lo $0$, 
non è il successore di nessun'altro mentre ogni altro numero 
ne ha uno precedente. Il terzo assioma, il principio di induzione, 
ci dice che ogni numero naturale può essere raggiunto dallo $0$ 
facendo un numero finito di passaggi al successore.}
\begin{enumerate}
  \item $\sigma$ è iniettiva;
  \item $\sigma(\NN) = \NN \setminus\ENCLOSE{0}$;
  \item se $A\subset \NN$ e 
  \begin{enumerate} 
    \item[(i)] $0\in A$ 
    \item[(ii)] $n\in A \implies \sigma(n)\in A$
  \end{enumerate}
  allora $A=\NN$.
\end{enumerate}
\end{definition}

D'ora in avanti, quando utilizziamo il simbolo $\NN$, supporremo sempre che $\NN$ sia un insieme che soddisfa la precedente definizione.
Vedremo nel Teorema~\ref{th:unicitaN} che due diversi insiemi che soddisfano gli assiomi di Peano 
sono isomorfi, cioè possono essere messi in corrispondenza biunivoca mantenendo la stessa struttura.
In questo senso possiamo dire che gli assiomi di Peano caratterizzano univocamente 
l'insieme dei numeri naturali. 
Diremo quindi che $\NN$ è \emph{l}'insieme dei numeri naturali piuttosto 
che \emph{un} insieme di numeri naturali.

Dato $n\in \NN$ il numero $\sigma(n)$ si chiama il \emph{successore}%
\mymargin{successore}%
\index{successore} di $n$. 
Il numero $0\in \NN$, che per assioma non è il successore di nessun'altro 
numero naturale, si chiama \emph{zero}%
\mymargin{zero}%
\index{zero}. 
Per comodità diamo un nome anche alle altre \emph{cifre decimali}
\mymargin{cifre decimali}%
\index{cifre!decimali}%
ovvero ai primi numeri naturali
(più avanti introdurremo la notazione posizionale per poter rappresentare ogni numero naturale 
con una sequenza di cifre decimali):
\begin{equation}\label{eq:cifre}
\begin{gathered}
 1 \defeq \sigma(0),\quad  
 2 \defeq \sigma(1),\quad
 3 \defeq \sigma(2),\quad 
 4 \defeq \sigma(3),\quad
 5 \defeq \sigma(4),\\ 
 6 \defeq \sigma(5),\quad 
 7 \defeq \sigma(6),\quad 
 8 \defeq \sigma(7),\quad 
 9 \defeq \sigma(8)
\end{gathered}
\end{equation}
e osserviamo che $\sigma$ è l'usuale operazione del contare:
 \[
 0 \stackrel\sigma\mapsto 1 \stackrel\sigma\mapsto 2 \stackrel\sigma\mapsto 
 3 \stackrel\sigma\mapsto 4 \stackrel\sigma\mapsto 5 \stackrel\sigma\mapsto 
 6 \stackrel\sigma\mapsto \dots  n \stackrel\sigma\mapsto \sigma(n) \dots
 \]

I primi due assiomi di Peano (definizione~\ref{def:assiomi_peano}) 
servono a garantire che in questo processo del \emph{contare}%
\mymargin{contare}%
\index{contare}
troviamo sempre numeri diversi (non si torna mai indietro) in quanto nessun numero 
può avere come successore $0$ o un numero già incontrato in precedenza (che 
se non è zero è il successore di un altro numero).
Questa proprietà è per certi versi paradossale
(paradosso di Galileo, o paradosso dell'hotel Hilbert)
\index{paradosso!di Galileo}%
\index{paradosso!degli insiemi infiniti}%
\index{hotel Hilbert}%
\index{Hilbert!hotel}%
e per Dedekind è la definizione di insieme infinito (tra poco la andremo a introdurre).%
\mynote{Galileo Galilei (1564--1642) vedi note storiche a pag.~\pageref{nota:Galileo}}%

Gli assiomi della teoria degli insiemi che abbiamo visto nei capitoli precedenti
non garantiscono l'esistenza di insiemi infiniti, e dunque se 
vogliamo garantire che l'insieme dei numeri naturali effettivamente esiste, 
dobbiamo aggiungere un ulteriore assioma. 

Potremmo semplicemente richiedere per assioma che esiste un insieme $\NN$ che soddisfa gli assiomi
di Peano~\ref{def:assiomi_peano}.
Ad esempio potremmo richiedere 
che esista l'insieme degli ordinali di Von Neumann \eqref{eq:vonNeumann}. 
\mynote{
Se denotiamo con $P(\alpha)$ il predicato 
$(\emptyset\in \alpha \land \forall n\colon n\in \alpha) \implies n\cup\ENCLOSE{n}\in \alpha$
allora l'assioma che garantisce l'esistenza dell'insieme $\omega$ degli ordinali finiti 
può essere scritto nella forma:
  $\exists \omega\colon (P(\omega) \land \forall \alpha\colon P(\alpha)\implies \omega\subset \alpha)$.
  La funzione $\sigma\colon\omega\to\omega$ sarebbe quindi $\sigma(n) = n\cup\ENCLOSE n$.
}

Ma sembra più naturale richiedere più semplicemente che esista almeno un insieme infinito,
vedremo infatti che questo è sufficiente a garantire l'esistenza dell'insieme dei numeri naturali
(teorema~\ref{th:esistenza_naturali}, dove si dimostra che ogni insieme infinito contiene un insieme 
che soddisfa gli assiomi di Peano).

Investighiamo quindi per un attimo cosa significa che un insieme sia infinito.
Per dare una definizione indipendente dall'esistenza dei numeri naturali possiamo usare 
una idea dovuta a Dedekind:
un insieme $X$ è infinito se può essere messo in corrispondenza 
biunivoca con un suo sottoinsieme proprio: 
esiste $f\colon X\to X$ tale che $f$ è iniettiva ma non suriettiva. 
Intuitivamente questo non può succedere su un insieme finito: 
se $X$ è finito ci aspettiamo che ogni suo sottoinsieme sia strettamente più piccolo di $X$
e quindi non possa essere messo in corrispondenza con tutto $X$.

\begin{definition}[insieme infinito]
  \label{def:infinito}%
  Diremo che un insieme $X$ è \emph{finito}
  \index{insieme!finito}%
  \index{finito!insieme}%
  \index{Dedekind!finito}%
  (più precisamente: Dedekind-finito)
  se ogni funzione iniettiva $f\colon X\to X$ è anche suriettiva.
  \mymargin{insieme finito}%
\index{insieme finito}%

  Diremo che un insieme $X$ è \emph{infinito} 
  (più precisamente Dedekind-infinito)
  \index{insieme!infinito}%
  \index{infinito!insieme}%
  \index{Dedekind!infinito}%
  se non è finito ovvero
  se esiste $f\colon X\to X$ iniettiva ma non suriettiva.
  \mymargin{insieme infinito}%
\index{insieme infinito}%
\end{definition}

Come abbiamo già osservato gli assiomi di Peano (definizione~\ref{def:assiomi_peano}) 
richiedono che la funzione $\sigma\colon \NN\to\NN$ sia iniettiva ma non suriettiva.
Risulta quindi che $\NN$ deve essere un insieme infinito.

Ma non si può escludere che tutti gli insiemi siano finiti (e che quindi $\NN$ non esista) 
se non introduciamo il seguente assioma.

\begin{axiom}[infinito]
  \label{axiom:infinito}%
  Esiste un insieme infinito. 
\end{axiom}

Grazie al precedente assioma il teorema seguente garantisce
l'esistenza di un insieme $\NN$ che soddisfa gli assiomi di Peano.

\begin{theorem}[esistenza dei numeri naturali]
  \label{th:esistenza_naturali}%
Se $X$ è un qualunque insieme infinito (definizione~\ref{def:infinito})
esistono $\NN\subset X$, $0\in \NN$ e $\sigma\colon \NN\to\NN$ 
che soddisfano gli assiomi di Peano (definizione~\ref{def:assiomi_peano}).
\end{theorem}
%
\begin{proof}
Se $X$ è infinito esiste $f\colon X\to X$ iniettiva ma non suriettiva. 
Scegliamo arbitrariamente $0\in X\setminus f(X)$. 
Preso un sottoinsieme $I\subset X$ diremo che $I$ è \emph{induttivo}
\mymargin{induttivo}%
\index{induttivo}%
se $(0\in I \land n\in I)\implies f(n)\in I$. 
Possiamo quindi definire:
\[
  \NN = \bigcap \ENCLOSE{I\subset X\colon \text{$I$ induttivo}}.
\]
Si verifica facilmente che $\NN$ è anch'esso un sottoinsieme induttivo di $X$.
\mynote{Nella definizione~\ref{def:restrizione} viene introdotto il simbolo
$\llcorner$ per la restrizione di funzione.}
Dunque per ogni $n\in \NN$ si ha $f(n)\in \NN$ e quindi possiamo 
definire $\sigma\colon \NN\to \NN$ come la restrizione di $f$ 
ad $\NN$: $\sigma=f\llcorner \NN$.

Osserviamo quindi che $\sigma$ soddisfa gli assiomi di Peano.
Il primo assioma è conseguenza dell'iniettività di $f$.
Il secondo è verificato per come abbiamo scelto $0$.
Per verificare il terzo assioma consideriamo un qualunque insieme 
$A\subset \NN$ tale che $0\in A$ e tale che se $n\in A$ anche $n+1\in A$.
Per definizione $A$ è induttivo e quindi certamente 
$\NN\subset A$ visto che $\NN$, per come è definito,
è sottoinsieme di ogni insieme induttivo.
\end{proof}

La costruzione precedente ci dice che l'insieme $\NN$ è il più piccolo insieme infinito
nel senso che: se $X$ è infinito allora $X$ contiene una copia isomorfa di $\NN$. 
\mynote{
Grazie al teorema~\ref{th:cantor_bernstein}, potremmo dire che se abbiamo due insiemi 
che soddisfano gli assiomi di Peano, essendoci una immersione iniettiva di ognuno
nell'altro (teorema~\ref{th:esistenza_naturali}), 
allora esiste una corrispondenza biunivoca tra i due insiemi.
Nel teorema~\ref{th:unicitaN} vedremo che non solo esiste una corrispondenza biunivoca 
ma anche che esiste una tale corrispondenza che preserva la struttura 
(cioè l'operazione $\sigma$ e l'elemento $0$). 
}

\subsection{principio di induzione}

L'ultimo degli assiomi di Peano 
ci dice che se un sottoinsieme dei numeri naturali contiene lo 
zero e contiene il successore di ogni suo elemento, allora contiene tutti i 
numeri naturali.
Serve a garantire che il processo del contare esaurisca tutti 
i numeri naturali, e che quindi non ci siano dei naturali \emph{irraggiungibili}
partendo da zero.
Questa proprietà viene usualmente utilizzata mediante il seguente.

\mynote{%
Formalmente dovremmo dire che $P$ è una funzione da $\NN$ in $\{V,F\}$
dove $V$,$F$ sono due valori distinti che interpretiamo come vero e falso.
Infatti il concetto di \emph{predicato} non è definito all'interno del 
sistema formale, ma è un concetto esterno.
Ma è ovvio che ad ogni predicato $P$ corrisponde 
una funzione $P\colon \NN\to \{V,F\}$ e viceversa.
}

\index{principio!di induzione}%
\index{induzione matematica}%
\begin{theorem}[principio di induzione]
  Sia $P(n)$ un predicato.
  Se 
  \begin{enumerate}
    \item vale $P(0)$
    \item $\forall n\in \NN\colon P(n)\implies P(n+1)$
  \end{enumerate} 
  allora $\forall n\in \NN\colon P(n)$.
\end{theorem}
%
\begin{proof}
  Consideriamo l'insieme $A=\{n\in \NN\colon P(n)\}$.
  Grazie alle ipotesi del teorema possiamo applicare la terza
  proprietà dei numeri naturali per dedurre che $A=\NN$.
  Dunque $P(n)$ è soddisfatta per ogni $n\in\NN$.
\end{proof}

Tramite il principio di induzione è anche possibile 
definire funzioni (o operazioni) per induzione.
Grazie al seguente teorema~\ref{th:induzione}, 
se vogliamo definire una funzione $f\colon\NN\to X$
basterà dichiarare il valore di $f(0)$ e definire $f(n+1)$ 
(cioè $f(\sigma(n)))$ in funzione di $f(n)$.
Questo metodo per definire una funzione si chiama 
\emph{definizione ricorsiva} (o per ricorrenza) in quanto definisce 
il valore della funzione sul termine $n$-esimo ricorrendo
al valore assegnato sui termini precedenti.

\begin{theorem}[definizione per induzione]
  \label{th:induzione}%
  Sia $X$ un insieme, sia $\alpha\in X$ e sia $g\colon X\to X$ una funzione.
  Allora esiste una unica funzione $f\colon \NN \to X$ tale che
  \begin{equation}\label{eq:4835628}
    \begin{cases}
      f(0) = \alpha, \\
      f(\sigma(n)) = g(f(n)).
    \end{cases}
  \end{equation}
  Si avrà dunque
  \[
    f(0) = \alpha,\quad
    f(1) = g(\alpha),\quad
    f(2) = g(g(\alpha)),\quad
    f(3) = g(g(g(\alpha)))\dots
  \]
  Più in generale se abbiamo $\alpha\in X$ e una funzione $g\colon \NN \times X \to X$
  esisterà una unica funzione $f\colon \NN \to X$ tale che
  %
  \begin{equation}
    \begin{cases}
      f(0) = \alpha, \\
      f(\sigma(n)) = g(n, f(n)).
    \end{cases}
  \end{equation}
\end{theorem}
%
\begin{proof}
Dobbiamo ricordarci che le funzioni $f\colon \NN \to X$ non sono altro che relazioni 
e cioè sottoinsiemi del prodotto $\NN\times X$.
Le proprietà~\eqref{eq:4835628} si scrivono dunque nella forma 
$(0,\alpha)\in f$ e $(n,x) \in f \implies (\sigma(n),g(x))\in f$
(se $f(n)=x$ allora $f(\sigma(n))=g(x)$).
L'idea è quindi di prendere il più piccolo sottoinsieme di $\NN\times X$ 
che possa rappresentare una funzione con le proprietà richieste.
Consideriamo dunque la famiglia di insiemi:
\[
\mathcal F = \ENCLOSE{F\in \mathcal P(\NN\times X)\colon 
  (0,\alpha)\in F,\quad (n,x)\in F \Rightarrow (\sigma(n),g(x))\in F}.
\]
Chiaramente $\mathcal F$ non è vuota in quanto $\NN\times X \in \mathcal F$.
Possiamo dunque farne l'intersezione e definire un insieme $f$:
\[
  f = \bigcap_{F\in \mathcal F} F.
\]
L'insieme $f$ che abbiamo definito rappresenta una relazione tra $\NN$ e $X$.
Visto che $(0,\alpha)\in F$ per ogni $F\in \mathcal F$ dovrà essere 
$(0,\alpha)\in f$.
Inoltre se $(n,x)\in f$ allora $(n,x)\in F$ per ogni $F\in \mathcal F$ 
e quindi $(\sigma(n),g(x))\in F$ per ogni $F\in \mathcal F$
da cui $(\sigma(n),g(x))\in f$. Significa che $f\in \mathcal F$.

Vogliamo ora dimostrare che $f$ è una funzione, cioè che è univocamente definita 
su tutto $\NN$.
Per prima cosa consideriamo l'insieme su cui $f$ è definita 
e cioè $A=\ENCLOSE{n\in \NN\colon \exists x\in X\colon (n,x)\in f}$
e dimostriamo, per induzione, che $A=\NN$.
In effetti $(0,\alpha)\in f$ quindi $0\in A$. 
E se $n\in A$ sappiamo che esiste $x\in X$ tale che $(n,x)\in f$ 
e dunque, essendo $f\in \mathcal F$, anche $(\sigma(n),g(x))\in f$
da cui $\sigma(n)\in A$. 
Abbiamo dimostrato che $f$ è definita su tutto $\NN$.

Dimostriamo ora che $f$ è univoca. Consideriamo 
l'insieme su cui $f$ è univocamente definita: 
$B=\ENCLOSE{n\in \NN\colon \exists! x\in X\colon (n,x)\in f}$.
Di nuovo vogliamo dimostrare per induzione che $B=\NN$. 
Per dimostrare che $0\in B$, visto che già sappiamo che $(0,\alpha)\in f$, 
dobbiamo dimostrare che se $x\neq \alpha$ si ha $(0,x)\not\in f$.
Sia dunque $x\neq \alpha$ e consideriamo 
l'insieme $F=f\setminus\ENCLOSE{(0,x)}$.
Chiaramente $F\in \mathcal F$ in quanto $(0,\alpha)\in F$
visto che $(0,\alpha)\in f$ e $(0,\alpha)\neq (0,x)$
inoltre se $(n,y)\in F$ allora $(n,y)\in f$ 
e quindi $(\sigma(n),g(y)) \in f$.
Ma certamente $(\sigma(n),g(y))\neq (0,x)$ in quanto $\sigma(n)\neq 0$ 
dunque $(\sigma(n),g(y))\in F$.
Visto che $F\in \mathcal F$ si deve avere $f\subset F$ e dunque 
$(0,x)\not \in f$. Dunque $f$ è univocamente definita in $0$.

Dobbiamo ora mostrare che se $n\in B$ anche $\sigma(n)\in B$.
Se $n\in B$ significa che c'è un unico $x\in X$ tale che $(n,x)\in f$
e certamente anche $(\sigma(n),g(x))\in f$.
Prendiamo allora $y\neq g(x)$, vorremo dimostrare che $(\sigma(n),y)\not \in f$.
Consideriamo, similmente a prima, l'insieme $F=f\setminus\ENCLOSE{(\sigma(n),y)}$
e cerchiamo di dimostrare che $F\in \mathcal F$.
Chiaramente $(0,\alpha)\in F$ perché $(0,\alpha)\in f$ 
e non può essere $(0,\alpha)=(\sigma(n),y)$ in quanto $\sigma(n)\neq 0$.
Se ora supponiamo che sia $(m,z)\in F$ certamente sarà $(m,z)\in f$ 
e dunque $(\sigma(m),g(z))\in f$: 
dobbiamo mostrare che $(\sigma(m),g(z))\in F$. 
D'altra parte se fosse $(\sigma(m), g(z))=(\sigma(n),y)$ 
dovrebbe essere $m=n$ in quanto $\sigma$ è iniettiva. 
Ma visto che $f$ è univocamente definita su $n$ dovrà allora essere 
anche $(n,z) = (n,x)$ e quindi $g(z)=g(x) \neq y$. 
Dunque $(\sigma(m),g(z))\in F$ e $F\in \mathcal F$.
Ma allora $f\subset F$ e quindi $(\sigma(n),y)\not \in f$.
Significa che $f$ è univocamente definita anche in $\sigma(n)$.
Per induzione $B=\NN$ ed $f$ è una funzione $f\colon \NN\to X$.

Ovviamente visto che $f\in \mathcal F$ sappiamo che $f$ 
soddisfa le proprietà richieste dal teorema.

Nella seconda parte del teorema, dove $g\colon \NN\times X \to X$,
possiamo considerare l'insieme $Y=\NN\times X$ e la funzione 
$G\colon Y\to Y$ definita da $G(n,x) = (\sigma(n), g(n,x))$.
Allora applicando la prima parte possiamo trovare $F\colon \NN\to Y$
tale che $F(0) = (0,\alpha)$ e $F(\sigma(n)) = G(F(n))$.
Basterà prendere come $f(n)$ la seconda componente di $G(n)$.
\end{proof}

\begin{theorem}[addizione su $\NN$]
  \label{th:addizione_naturali}%
  \index{addizione!su $\NN$}%
  \mymargin{addizione}%
Su $\NN$ è definita in modo unico una operazione di addizione 
che soddisfa le seguenti proprietà:
\begin{equation}\label{eq:3408923}
  \begin{cases}
    n + 0 = n,\\
    n + \sigma(m) = \sigma(n+m).
  \end{cases}
\end{equation}

Questa operazione rende $\NN$ un monoide additivo commutativo 
con elemento neutro $0$.
\end{theorem}
\begin{proof}
L'operazione di addizione è univocamente definita 
grazie al teorema~\ref{th:induzione}.

Per dimostrare che $\NN$ è un monoide commutativo dobbiamo verificare
che $0$ è elemento neutro e che valgono la proprietà associativa 
e commutativa.

Chiaramento $n+0=n$ per la definizione~\eqref{eq:3408923}. 
Verifichiamo ora che anche $0+m=m$ e lo facciamo per induzione su $m$.
Se $m=0$ risulta vero per la definizione.
Per quanto riguarda $0+\sigma(m)$ abbiamo per definizione:
$0+\sigma(m)=\sigma(0+m)$ e per ipotesi induttiva $0+m=m$.
Dunque $0+\sigma(m)=\sigma(m)$, che è quanto volevamo dimostrare.

Verifichiamo la proprietà associativa: $n+(m+k) = (n+m)+k$, lo 
facciamo per induzione su $k$. 
Se $k=0$ allora $n+(m+0)=n+m=(n+m)+0$, per le proprietà precedenti.
Per il passo induttivo valutiamo (usando sempre la definizione~\eqref{eq:3408923}):
$n+(m+\sigma(k)) = n+\sigma(m+k) =\sigma(n+(m+k))$.
L'ipotesi induttiva garantisce che $n+(m+k)=(n+m)+k$ e quindi:
$n+(m+\sigma(k)) = \sigma((n+m)+k) = (n+m)+\sigma(k)$
che è quanto volevamo dimostrare.

Verifichiamo infine la proprietà commutativa: $n+m=m+n$.
Lo facciamo per induzione su $m$. 
Se $m=0$ è banale per le proprietà dell'elemento neutro.
Il passo induttivo risulta immediatamente 
grazie alla definizione~\ref{eq:3408923} 
e all'ipotesi induttiva: 
$n+\sigma(m) = \sigma(n+m) = \sigma(m+n) = m+\sigma(n)$.
\end{proof}

\begin{example}\label{ex235}
  Dimostrare che $2+3=5$.
\end{example}  
%
\begin{proof}[Svolgimento]
Ricordiamo che, per definizione \eqref{eq:cifre}, 
$1=\sigma(0)$, $2=\sigma(1)$, $3=\sigma(2)$, $4=\sigma(3)$ e $5=\sigma(4)$.
Dobbiamo usare le proprietà $n+\sigma(m) = \sigma(n+m)$ 
e $n+0=n$:
\begin{align*}
2+3 &= 2 + \sigma(2) = \sigma(2+2) = \sigma(2+\sigma(1))
=\sigma(\sigma(2+1)) = \sigma(\sigma(2+\sigma(0)))\\
&=\sigma(\sigma(\sigma(2+0)))
=\sigma(\sigma(\sigma(2)))
=\sigma(\sigma(3))
=\sigma(4)=5.
\end{align*}
\end{proof}

\mymargin{$\sigma(n)=n+1$}%
Visto che $\sigma(n) = \sigma(n+0) = n+\sigma(0) = n+1$, 
potremo in futuro evitare di utilizzare la funzione \emph{successore} 
e scrivere direttamente $n+1$ per indicare il successore di $n$.

Abbiamo definito l'addizione $n+m$ 
come l'iterazione della funzione successore.
In generale è possibile definire la moltiplicazione per un numero naturale 
come una somma ripetuta, e l'elevamento a potenza (con esponente naturale)
come una moltiplicazione ripetuta.
Questo può essere fatto su $\NN$ come su ogni monoide.

\begin{theorem}[operazioni iterate]
\label{th:operazioni_iterate}%
\label{th:operazione_ripetuta}%
Se $M$ è un monoide additivo (definizione~\ref{def:monoide})
possiamo definire per induzione la moltiplicazione $n\cdot x$ tra un numero 
naturale $n\in \NN$ e un elemento $x$ del monoide:
\[
\begin{cases}
  0\cdot x = 0 \in M \\
  (n+1)\cdot x = n\cdot x + x.
\end{cases}
\]

Valgono inoltre le seguenti proprietà
(valide per ogni $n,m\in \NN$ e ogni $x,y\in M$):
\begin{enumerate}
  \item elemento assorbente: $0\cdot x = 0$ e $n\cdot 0 = 0$;
  \item proprietà distributive: 
    $n\cdot (x+y)=n\cdot x + n\cdot y$
    e $(n+m)\cdot x = n\cdot x + m\cdot x$;
  \item proprietà associativa: $n\cdot (m\cdot x)= (n\cdot m)\cdot x$.
\end{enumerate}

Lo stesso ovviamente si può fare nei monoidi moltiplicativi
se definiamo la potenza $x^n$ come prodotto ripetuto:
\[
\begin{cases}
  x^0 &= 1 \in M \\
  x^{n+1} &= x^n \cdot x,
\end{cases}
\]
le proprietà sono analoghe:
\begin{enumerate}
  \item potenze banali: $x^0=1$, $1^n=1$;
  \item potenza del prodotto: $(x\cdot y)^n = x^n\cdot y^n$;
  \item prodotto di potenze: $x^{n+m}=x^n\cdot x^m$;
  \item potenza di potenza: $(x^m)^n = x^{m\cdot n}$.
\end{enumerate}
\end{theorem}
\begin{proof}
Le proprietà andranno dimostrate per induzione.
Facciamo il caso che $M$ sia un monoide additivo.
La proprietà assorbente $0\cdot x = 0$ è data per definizione,
mentre $n\cdot 0 = 0$ si dimostra immediatamente 
per induzione, infatti per $n=0$ si ha $0\cdot 0 = 0$ 
e il passo induttivo è $(n+1)\cdot 0 = n\cdot 0 + 0 = 0 + 0 = 0$.

Per la prima proprietà distributiva: $n\cdot (x+y)=n\cdot x + n\cdot y$
è verificato il caso base: $0\cdot(x+y) = 0 = 0 + 0 = 0\cdot x + 0\cdot y$.
Il passo induttivo diventa: 
$(n+1)\cdot(x+y)
=n\cdot(x+y)+x+y
=n\cdot x + x + n\cdot y + y 
=(n+1)\cdot x + (n+1)\cdot y$.
La seconda $(n+m)\cdot x = n\cdot x + m\cdot x$
per $n=0$ è banale: $(0+m)\cdot x = m\cdot x = 0\cdot x + m\cdot x$.
Il passo induttivo:
$(n+1+m)\cdot x = (n+m)\cdot x + x 
= n\cdot x + x + m\cdot x
= (n+1)\cdot x + m\cdot x$.

La proprietà associativa: $(n\cdot m)\cdot x = n\cdot (m\cdot x)$
è ovvia per $n=0$: $(0\cdot m)\cdot x = 0 = 0\cdot x = 0\cdot (m\cdot x)$.
Per il passo induttivo usiamo anche la proprietà distributiva:
$((n+1)\cdot m)\cdot x 
= (n\cdot m+m)\cdot x 
= (n\cdot m)\cdot x + m\cdot x
= n\cdot (m\cdot x)+m\cdot x
= (n+1)\cdot (m\cdot x)$.

Per quanto riguarda il monoide moltiplicativo le proprietà sono esattamente 
le stesse, cambia solamente la notazione con cui le scriviamo:
al posto della somma su $M$ scriviamo un prodotto, e al posto del prodotto 
scriviamo una potenza. 
\end{proof}

\begin{exercise}
  Dimostrare che $2\cdot 3 = 6$ e che $2^3 = 8$.
\end{exercise}
  
Definiamo l'ordinamento di $\NN$ 
con l'idea che un numero $m$ è \emph{più grande} di $n$ 
se si ottiene aggiungendo qualcosa ad $n$.
\index{relazione!d'ordine su $\NN$}%
\mymargin{ordinamento di $\NN$}%
\[
    n\le m \iff \exists k\in \NN \colon n+k = m.
\]

\begin{theorem}[proprietà delle operazioni su $\NN$]
\label{th:operazioni_naturali}%
\index{moltiplicazione!su $\NN$}%
\index{elevamento a potenza!su $\NN$}%
Le operazioni di addizione, moltiplicazione, elevamento a potenza 
e l'ordinamento definiti su $\NN$
soddisfano le seguenti proprietà.
Proprietà di addizione e moltiplicazione:
\begin{enumerate}
    \item elemento neutro:
      $n + 0 = 0 + n = n$,
      $n\cdot 1 = 1\cdot n = n$;
    \item elemento assorbente:
      $n\cdot 0 = 0$;
    \item proprietà associativa: 
      $(n+m)+k = n+(m+k)$, $(n\cdot m)\cdot k = n \cdot (m\cdot k)$
    \item proprietà commutativa: 
      $n+m = m+n$, $n\cdot m = m\cdot n$;
    \item proprietà distributiva:
     $k\cdot (n+m) = k\cdot n + k\cdot m$;
    \item proprietà invariantiva:
     se $m+k = n+k$ allora $m=n$;
    \item annullamento del prodotto: se $m\cdot n=0$ allora $m=0$ o $n=0$;
\end{enumerate}
proprietà dell'elevamento a potenza:
\begin{enumerate}
  \item $n^0 = 1$;
  \item $n^1 = n$;
  \item $n^{m+k} = n^m \cdot n^k$;
  \item $(n^m)^k = n^{m\cdot k}$;
  \item $(n\cdot m)^k = n^k \cdot m^k$;
\end{enumerate}
proprietà di ordinamento:
\begin{enumerate}
  \item $n\le n$ (proprietà riflessiva);
  \item se $n\le m$ e $m\le k$ allora $n\le k$ (proprietà transitiva);
  \item se $n\le m$ e $m\le n$ allora $n=m$ (proprietà antisimmetrica);
  \item o $n\le m$ oppure $m\le n$ (dicotomia);
  \item monotonia:
   se $m\le n$ allora 
   $m+k\le n+k$, $m\cdot k \le n\cdot k$,
   $k^m \le k^n$ e $m^k \le n^k$. 
\end{enumerate}
\end{theorem}
%
\begin{proof}
Molte di queste proprietà sono valide in qualunque monoide, e sono già 
state dimostrate nel teorema~\ref{th:operazioni_iterate}:
elemento neutro, elemento assorbente, proprietà associativa, 
distributiva e tutte le proprietà delle potenze.
Che l'addizione sia commutativa e associativa è già stato dimostrato 
nel teorema~\ref{th:addizione_naturali}.
Dimostriamo le proprietà rimanenti.

Dimostrare la proprietà invariantiva $n+k=m+k \implies n=m$ 
per induzione su $k$.
Per $k=0$ la proprietà è banale.  
Per il passo induttivo osserviamo che se vale $m+k+1=n+k+1$
allora, grazie all'iniettività di $\sigma$, 
dovrà essere $m+k = n+k$
e per ipotesi induttiva concludiamo $m=n$ come dovevamo dimostrare.
  
La proprietà $1\cdot n=n$ è conseguenza 
della definizione: $(0+1)\cdot n = 0\cdot n + n = 0 + n = n$. 
La proprietà $n\cdot 1=n$ si può dimostrare per induzione.
Se $n=0$ è data per definizione. 
Per il passo induttivo abbiamo $(n+1)\cdot 1 = n\cdot 1 + 1$
ed essendo $n\cdot 1=n$ per ipotesi induttiva 
si conclude $(n+1)\cdot 1 = n+1$, che è quanto dovevamo dimostrare.

La proprietà commutativa $n\cdot m=m\cdot n$ si può dimostrare 
facilmente per induzione su $n$. 
Per $n=0$ è banale, grazie alla proprietà assorbente.
Per il passo induttivo osserviamo che $(n+1)\cdot m = n\cdot m + m$
e suppendo per ipotesi induttiva che $n\cdot m=m\cdot n$ 
e usando la proprietà distributiva otteniamo:
si ottiene $(n+1)\cdot m = m\cdot n + m\cdot 1 = m\cdot (n+1)$
che è quanto dovevamo dimostrare.

Per dimostrare l'annullamento del prodotto supponiamo per assurdo che 
il prodotto di due numeri diversi da $0$ possa essere nullo.
Un numero naturale diverso da zero è il successore di un altro numero naturale, 
quindi potremmo scrivere $(n+1)\cdot(m+1)=0$. 
Ma $(n+1)\cdot(m+1) = n\cdot m + n + m + 1$ è il successore di un numero naturale  
quindi non può essere zero.

vediamo infinite le proprietà dell'ordinamento.
Il fatto che $n+0=n$ dimostra la proprietà riflessiva: $n\ge n$.
  
Per la proprietà transitiva è sufficiente osservare
che se $n=m+j$ e $m=k+l$ allora $n=k+l+j$.

Per la proprietà antisimmetrica supponiamo di avere $n=m+k$ 
e $m=n+j$. 
Deduciamo che $ m = m+k+j$ da cui $k+j=0$.
Se fosse $k\neq 0$ o $j\neq 0$ il lato sinistro sarebbe il successore 
di un numero naturale, ma $0$ non è il successore di nessun numero naturale.
Questo significa che $k=j=0$ e quindi $n=m$, come volevamo dimostrare.

Per mostrare che l'ordinamento è totale dobbiamo invece 
procedere con una dimostrazione per induzione. 
Per induzione su $n\in \NN$ vogliamo mostrare 
che per ogni $m\in \NN$ si ha $n\le m$ oppure $m\le n$. 
Se $n=0$ il fatto è ovvio in quanto $m=m+0$ e quindi $m\ge 0$
per ogni $m\in \NN$.
Fissati $m$ ed $n$, supponiamo ora di sapere che $n\le m$ oppure $m\le n$
e dimostriamo che allora $n+1\le m$ oppure $m\le n+1$.
Se $m\le n$ significa che $n = m + k$ per un qualche 
$k\in \NN$. Ma allora $n+1 = m + k + 1$ e quindi 
vale anche $m\le n+1$. 
Se invece $n\le m$ significa che esiste $k\in \NN$ 
per cui $m=n+k$. Se $k\neq 0$ allora $k=1+j$ con $j\in \NN$
da cui $m=n+1+j$ e dunque anche $n+1\le m$.
Se $k=0$ allora $n=m$ e dunque $n+1=m+1$ che ci porta 
alla disuguaglianza inversa $m \le n+1$.
In ogni caso il passo induttivo è dimostrato.

Dimostriamo ora la monotonia della somma e del prodotto.
Sia $m\le n$ e sia $k$ qualunque.
Visto che $m\le n$ si ha $n=m+j$. 
Dunque
\[
n+k = m + j + k \ge m+k   
\] 
e
\[
m\cdot k = (n+j)\cdot k = n\cdot k + j\cdot k \ge n\cdot k.  
\]

Per verificare la monotonia delle potenze procediamo per induzione.
Basterà dimostrare che $m^k\le(m+1)^k$ e $k^m\le k^{m+1}$.
Lo facciamo, a loro volta, per induzione:
\[
(m+1)^{k+1} = (m+1)\cdot (m+1)^k \ge m\cdot m^k
\]
e (supponendo $k\ge 1$)
\[
k^{m+1} = k\cdot k^m \ge 1\cdot k^m = k^m. 
\]
\end{proof}

\begin{exercise}[prodotti notevoli]
Usando la proprietà distributiva dimostrare che, per $a,b\in \NN$ si ha:
\[
(a+b)^2 = a^2+2ab+b^2,
\qquad
(a+b)^3 = a^3 + 3a^2b + 3ab^2 + b^3.
\]
\end{exercise}

La proprietà invariantiva ci dice 
che se esiste $k\in \NN$ tale che $n+k=m$
allora tale $k$ è unico. 
\index{differenza}%
\index{sottrazione}%
\mymargin{differenza}
Dunque se $m\ge n$ si può definire la differenza $k=m-n$
come quell'unico $k$ tale che $n+k=m$.

In maniera analoga se esiste $k\in \NN$ tale che $m=k\cdot n$
diremo che $m$ è un \emph{multiplo} di $n$ 
\index{multiplo}%
\mymargin{multipli e divisori}%
ovvero che $n$ \emph{divide} $m$ (si può scrivere $n\vert m$).
Diremo che $n$ è \emph{pari} se $n$ è divisibile per due,
altrimenti diremo che $n$ è \emph{dispari}.
Se $n\neq 0$ e $m=k\cdot n$ allora $k$ è unico 
(lo si verifichi utilizzando la proprietà di annullamento del prodotto)
e si chiama \emph{quoziente} di $m$ diviso $n$, 
\index{quoziente}%
scriveremo: $k=\frac m n$.

Si osservi che abbiamo definito $n^0=1$ per ogni $n\in \NN$,
compreso $n=0$. 
Dunque abbiamo consapevolmente definito $0^0=1$:
questa definizione (controversa) risulterà essere utile.
Si noti invece che $0^n=0$ solo se $n\neq 0$.
Infatti qualunque numero moltiplicato per $0$ dà zero, 
quindi una moltiplicazione ripetuta dà zero 
se c'è almeno un fattore nullo. 
Ma nel prodotto $0^0$ ci sono $0$ fattori $0$ quindi non c'è in effetti nessuna 
moltiplicazione per $0$. 
E' dunque naturale che il risultato sia $1$, 
l'elemento neutro della moltiplicazione.

\begin{exercise}
  Dimostrare che per ogni $n\in \NN$ si ha:
  \[
    n \text{ dispari} \implies \exists k\in \NN\colon n=2k+1.
  \]
\end{exercise}
  
\begin{exercise}
  Dimostrare per induzione che per ogni $n\in \NN$, $n\ge 4$ si ha 
  \[  
    2^n \ge n^2.
  \]
  Ovvero, dimostrare che per ogni $n\in \NN$ si ha 
  \[
    2^{n+4} \ge (n+4)^2.  
  \]
  \end{exercise}
  
\subsection{sequenze finite o ennuple}
\index{ennuple}%
\index{sequenza!finita}%

Abbiamo già visto che se $A$ è un insieme possiamo definire l'insieme 
delle coppie di elementi di $A$ con il prodotto cartesiano tra insiemi: 
$A\times A$. 
Cosa rappresenta un prodotto ripetuto?
Gli insiemi $(A\times A)\times A$ e $A\times(A\times A)$ non sono 
uguali, in quanto il primo è un insieme di coppie il cui primo elemento 
è a sua volta una coppia, il secondo insieme invece è un insieme di coppie 
il cui secondo elemento è una coppia:
\begin{gather*}
  (A\times A)\times A = \ENCLOSE{((a_0,a_1),a_2)\colon a_0,a_1,a_2\in A},
  \\
  A \times (A\times A) = \ENCLOSE{(a_0,(a_1,a_2))\colon a_0,a_1,a_2\in A}.
\end{gather*}
Questi insiemi sono molto simili tra loro e potrebbero essere identificati.
Un altro insieme molto simile è l'insieme $A^{3}$.  
Usando la definizione di Von Neumann, $3 = \ENCLOSE{0,1,2}$, l'insieme 
$A^{3}$ rappresenta l'insieme di tutte le funzioni 
$\vec a\colon \ENCLOSE{0,1,2}\to A$.
La funzione $\vec a$ è univocamente determinata dal suo valore nei 
tre punti del dominio: $a_0 = \vec a(0)$, $a_1=\vec a(1)$, $a_2=\vec a(2)$
in quanto $\vec a = \ENCLOSE{0\mapsto a_0, 1\mapsto a_1, 2\mapsto a_2}$.
Possiamo quindi identificare $\vec a$ con la tripla (o vettore) di valori:
\[
  \vec a = (a_0, a_1, a_2), \qquad a_0,a_1,a_2 \in A.  
\]
I valori $a_0$, $a_1$ e $a_2$ vengono anche chiamate \emph{coordinate}
o \emph{componenti} del \emph{vettore} $\vec a$.
\mymargin{coordinate, componenti, vettore}%
\index{coordinate, componenti, vettore}%
\index{coordinate!vettore}%
\index{componenti!vettore}%
\index{vettore}%
In effetti l'insieme $A^{2}$ può essere identificato con $A\times A$, l'insieme 
$A^{3}$ sarà l'insieme delle triple di elementi di $A$ e in generale se $n\in\NN$ 
identifichiamo con $A^{n}$ l'insieme delle $n$-uple (leggi: ennuple) di elementi 
\index{ennuple}%
di $A$:%
\mynote{
  Stiamo qui identificando $n\in \NN$ con l'insieme $\ENCLOSE{0,1,\dots, n-1}$.}%
\[
   \vec a \in A^{n} \iff 
   \vec a = (a_0, a_1, \dots, a_{n-1}).  
\]

Storicamente ci siamo abituati a contare partendo da $1$ invece che da $0$.
Per questo motivo è usuale numerare gli elementi di una $n$-upla con gli indici 
che vanno da $1$ a $n$ invece che da $0$ a $n-1$.
Dunque se $\vec a \in A^{3}$ sarà usuale scrivere 
$\vec a = (a_1, a_2, a_3)$ invece che $\vec a = (a_0, a_1, a_2)$.

\subsection{somme e prodotti con un numero variabile di addendi}
\index{sommatoria}%
\index{produttoria}%

Introduciamo ora su $\NN$ il concetto di somma (e prodotto) iterato.
Quello che ora facciamo su $\NN$ vale allo stesso modo sugli insiemi 
numerici che andremo a definire più avanti, in particolare lo useremo 
molto sull'insieme $\RR$ dei numeri reali. 

In generale se $A$ è un monoide (definizione~\ref{def:monoide}) additivo 
(o moltiplicativo) possiamo fare somme (o prodotti) di un numero 
arbitrario (ma finito) di addendi (o fattori) di $A$.
Più precisamente se $\vec a \in A^n$ 
è una $n$-upla di elementi di $A$, 
vogliamo definirne la somma e il prodotto:
\mynote{In questo ambito è usuale numerare gli elementi della 
$n$-upla a partire dall'indice $1$ invece che dall'indice $0$.}
\begin{equation}\label{eq:6122112}
\sum_{k=1}^n a_k = a_1 + a_2 + \dots + a_n,
\qquad 
\prod_{k=1}^n a_k = a_1 \cdot a_2 \dots a_n.
\end{equation}
Formalmente bisogna dare una definizione per induzione. 
Basterà dire che la somma (o il prodotto)
di zero addendi (o fattori) vale l'elemento neutro 
dell'operazione cioè $0$ per la somma (e $1$ per il prodotto).
La somma di $n+1$ addendi (o $n+1$ fattori) è la somma dei primi $n$ 
a cui aggiungiamo l'ultimo addendo (o moltiplichiamo per l'ultimo fattore).
Formalmente:
\[
  \begin{cases}
    \displaystyle\sum_{k=1}^{0} a_k = 0, \\
    \displaystyle\sum_{k=1}^{n+1} a_k = \enclose{\sum_{k=1}^{n} a_k} + a_{n+1};
  \end{cases}  \qquad
  \begin{cases}
    \displaystyle\prod_{k=1}^{0} a_k = 1, \\
    \displaystyle\prod_{k=1}^{n+1} a_k = \enclose{\prod_{k=1}^{n} a_k} \cdot a_{n+1}.
  \end{cases}
\]
La variabile $k$ che si trova nelle formule~\eqref{eq:6122112} è \emph{muta}: 
al suo posto si può utilizzare qualunque altra variabile che non
compaia altrove.

E' naturalmente possibile anche fare una somma a partire da un indice 
diverso da $0$:
\[
  \sum_{k=m+1}^{m+n} a_k = \sum_{j=1}^{n} a_{m+j}.
\]
Dal punto di vista mnemonico abbiamo fatto 
un \emph{cambio di variabile} $k=m+j$:
per $j=1$ si trova $k=m+1$ e per $j=n$ si trova $k=m+n$.

Si osservi che risulta:
\[
  x^n = \prod_{k=1}^n x
\]
in quanto la definizione ricorsiva di potenza coincide 
con la definizione ricorsiva del prodotto sul lato destro.

\begin{theorem}
In un monoide commutativo, si ha:
  \[
  \sum_{k=1}^n  \enclose{a_k + b_k} 
  = \sum_{k=1}^n a_k + \sum_{k=1}^n b_k.
  \]

Sia $f\colon A\to A$ una funzione \emph{additiva}
cioè $f(0) = 0$ e $f(x+y)=f(x)+f(y)$.
Allora 
  \[
    \sum_{k=1}^n  f(a_k) = f\enclose{\sum_{k=1}^n a_k}.
  \]

Infine risulta 
  \[
  \sum_{k=1}^{m+n} a_k = \sum_{k=1}^m a_k + \sum_{k=1}^n a_{k+m}
  \]
ovvero 
  \[
  \sum_{k=1}^{m+n} a_k = \sum_{k=1}^m a_k + \sum_{k=m+1}^{m+n} a_k.  
  \]
\end{theorem}
\begin{proof}
  Le dimostrazioni possono essere svolte 
  per induzione.
\end{proof}

\begin{exercise}
  \label{ex:somma_lineare}%
  Dimostrare che 
  \[
    \sum_{k=1}^n k = \frac{n\cdot (n+1)}{2}
  \]
\end{exercise}
Il risultato della somma può essere ricordato nel modo seguente:
la media di una progressione aritmetica $\frac{1+2+ \dots + n}{n}$ 
è uguale alla media tra il primo 
e l'ultimo termine della progressione: $\frac{1+n}{2}$.
\begin{proof}[Svolgimento]
Lo dimostriamo per induzione.
Per $n=0$ la somma è pari a $0$ per definizione.
Se la formula è vera per un certo $n$, si ha 
\[
  \sum_{k=1}^{n+1} k = \enclose{\sum_{k=1}^n k} + (n+1)
   = \frac{n\cdot(n+1)}{2} + (n+1) 
   = \frac{(n+2)(n+1)}{2}
\]
che è quanto volevamo dimostrare.
\end{proof}

\begin{exercise}
\label{ex:somma_quadrati}
  Trovare una formula per calcolare esplicitamente:
  \[
    \sum_{k=1}^n k^2.
  \]
\end{exercise}
\begin{proof}[Svolgimento.]
Vogliamo una formula simile a quelle dimostrate negli esercizi 
precedenti. 
Questa volta, però, la formula la dobbiamo trovare noi.
Presentiamo un metodo che permette, in generale, 
di trovare la formula per calcolare $\sum_{k=1}^n k^{m+1}$
se conosciamo la formula per $\sum_{k=1}^n k^m$.

Osserviamo che la differenza 
di due termini cubici consecutivi risulta essere quadratico:
\[
(k+1)^3 - k^3 = 3 k^2 + 3k + 1.  
\]
Sommando ambo i lati della precedente equazione si ottiene:
\begin{equation}\label{eq:309838}
\sum_{k=1}^n (k+1)^3 - \sum_{k=1}^n k^3 
= 3\sum_{k=1}^n k^2+3\sum_{k=1}^n k+\sum_{k=1}^n 1.
\end{equation}
Al lato destro compare la somma di cui vogliamo calcolare il valore 
(moltiplicata per $3$)
insieme ad altre due somme di cui sappiamo già il valore. 
Basterà allora determinare il valore del lato sinistro dove 
osserviamo che i termini delle due somme si cancellano 
a vicenda,
\mynote{%
si chiama \emph{somma telescopica}
\index{somma!telescopica}%
\index{telescopico}%
in quanto i termini delle due sommatorie si chiudono uno nell'altro 
come i tubi di un cannocchiale.} %
tranne l'ultimo della prima somma 
e il primo della seconda: 
\[
  \sum_{k=1}^n(k+1)^3 - \sum_{k=1}^n k^3 = (n+1)^3 - 1^3.
\]
Moltiplichiamo per $2$ l'equazione~\eqref{eq:309838},
mettiamo in evidenza la somma dei quadrati e 
utilizzando la formula $2\sum_{k=1}^n k = n(n+1)$
per ottenere:
\begin{align*}
  6 \sum_{k=1}^n k^2 
  &=  2(n+1)^3 - 3 n(n+1) - 2n - 2
  = (n+1) \cdot \Enclose{2(n+1)^2 - 3n} - 2(n+1)\\
  &= (n+1)\cdot \Enclose{2n^2+4n+2 -3n - 2}
   = (n+1)\cdot \Enclose{2n^2+n} 
   = n \cdot (n+1)(2n+1)
\end{align*}
da cui 
\begin{equation}\label{eq:somma_quadrati}
  \sum_{k=1}^n k^2 = \frac{n\cdot (n+1)(2n+1)}{6}.
\end{equation}
\end{proof}

\begin{exercise}
Dimostrare che 
\[
  \sum_{k=1}^n k^3 = \frac{n^2\cdot (n+1)^2}{4}.
\]
Si può utilizzare il metodo dell'esercizio precedente per trovare una formula esplicita 
oppure, banalmente, basta verificare per induzione che la formula è valida.
\end{exercise}

\begin{exercise}
  \label{ex:somma_geometrica}%
Si noti che se prendiamo una progressione geometrica 
(potenze successive con la stessa base) 
ad esempio:
\[
  x = 3^3 + 3^4 + 3^5 + 3^6
\]
ogni addendo è uguale al precedente moltiplicato per $3$.
Dunque se moltiplichiamo per la base l'intera somma
\[
  3x = 3^4 + 3^5 + 3^6 + 3^7
\]
si ottiene la somma iniziale con un termine in più alla fine 
e un termine in meno all'inizio. 
\mynote{Anche questa è una somma \emph{telescopica}}
Facendo la differenza tutti i termini si cancellano tranne 
il primo e l'ultimo
\[
 3x - x = 3^7 - 3^3  
\]
e possiamo ricavare $x$.

Si formalizzi il ragionamento precedente per dimostrare che 
fissato $a\neq 1$ si ha per ogni naturale $n$:
  \[
    \sum_{k=0}^n a^k = \frac{a^{n+1}-1}{a-1}.
  \] 
\end{exercise}

Denotiamo con $\Enclose{n} = \ENCLOSE{0,1,2, \dots, n-1}$.
Se $\sigma\colon \Enclose{n} \to \Enclose{n}$
è bigettiva (una tale funzione si chiama \emph{permutazione}%
\mymargin{permutazione}%
\index{permutazione})
allora si può fare il cambio di variabile $k=g(j)$:
\[
    \sum_{k=0}^{n-1} a_k = \sum_{j=0}^{n-1} a_{\sigma(j)}.
\]
Questa uguaglianza può essere dimostrata facilmente nel caso 
in cui $\sigma$ scambi due soli indici lasciando fissi tutti gli altri 
(trasposizione) e poi può essere estesa a tutte le permutazioni
osservando che ogni permutazione si può scrivere come composizione 
di trasposizioni.

Questa proprietà è sostanzialmente la proprietà commutativa della somma 
estesa ad un numero qualunque di addendi.
Grazie a questa proprietà possiamo definire la somma di una funzione 
definita su qualunque insieme finito. 
Se $f\colon X \to M$
dove $M$ è un monoide, 
e $\sigma\colon \Enclose{n} \to X$ è una bigezione
allora si può definire 
\[
  \sum_{x\in X} f(x) = \sum_{j=0}^{n-1} f(\sigma(j))  
\]
in quanto la somma sul lato destro non dipende dalla bigezione $\sigma$ che 
abbiamo scelto.

%%%%%%%%%%%%%%%%%%%%%%%%%%%%%%%%%

\subsection{buon ordinamento e unicità dei numeri naturali}

Si faccia riferimento alle definizioni~\ref{def:minimo}
per il concetto di minimo, maggiorante e minorante.
%
\begin{theorem}[principio del buon ordinamento]
  \label{th:buon_ordinamento}
  Sia $A\subset \NN$, $A\neq \emptyset$. 
  Allora $A$ ha minimo.
\end{theorem}
%
\begin{proof}
  Osserviamo che se $n\in \NN$ è un minorante di $A\subset \NN$ 
  allora o $n\in A$ e quindi $n$ è il minimo di $A$
  oppure anche $n+1$ è un minorante di $A$ in quanto 
  non ci sono numeri naturali strettamente compresi tra $n$ e $n+1$.
  \mynote{Se ci fosse un numero naturale tra $n$ e $n+1$ 
  sottraendo $n$ avremmo un numero naturale tra $0$ e $1$. 
  Ma non esiste $x\in \NN$ tale che $0<x<1$.
  Infatti se $x\in \NN$, $x>0$ allora $x\neq 0$ e quindi 
  esiste $m\in \NN$ tale che $x=m+1$. 
  Ma allora $x\ge 1$.}
  Dunque se $A$ non avesse minimo, per il principio di induzione 
  ogni $n\in\NN$ sarebbe un minorante di $A$.
  In tal caso $A$ dovrebbe essere vuoto perché se esistesse $a\in A$ 
  certamente $a+1$ non sarebbe un minorante di $A$.
\end{proof}

Vogliamo ora dimostrare che l'insieme dei numeri naturali è sostanzialmente 
unico nel senso che se ci sono due insiemi che soddisfano gli assiomi di 
Peano allora è possibile mettere in corrispondenza gli elementi dei due insiemi 
in modo che lo zero vada in zero e numeri corrispondenti abbiano successori 
corrispondenti.

\begin{theorem}[unicità dei numeri naturali]
  \label{th:unicitaN}%
  Se $\NN$ e $\NN'$ sono due insiemi che soddisfano gli assiomi di Peano 
  con zero $0\in \NN$ e $0'\in \NN'$ e funzioni 
  successore $\sigma$ su $\NN$ e $\sigma'$ su $\NN'$ allora
  esiste una funzione bigettiva $f\colon \NN\to \NN'$ tale che 
  $f(0) = 0'$ e $f(\sigma(n)) = \sigma'(f(n))$.
\end{theorem}
%
\begin{proof}
Possiamo definire $f$ per induzione:
\[
\begin{cases}
  f(0) = 0' \\ 
  f(\sigma(n)) = \sigma'(f(n))
\end{cases}  
\]
così rimane solo da dimostrare che $f$ è una bigezione.

Per dimostrare che $f$ è iniettiva consideriamo l'insieme 
\[
  A=\ENCLOSE{a\in \NN\colon \exists b\in \NN\colon b\neq a, f(a)=f(b)}.
\]
Se tale insieme è vuoto allora $f$ è effettivamente iniettiva.
Supponiamo allora per assurdo che $A$ non sia vuoto.
In tal caso possiamo considerare il minimo $a=\min A$ 
(grazie al teorema~\ref{th:buon_ordinamento}).
Dovrà quindi esistere $b\in \NN$ tale che $b\neq a$ e $f(b)=f(a)$.
Ovviamente anche $b\in A$ e quindi dovrà essere $b>a$ in quanto
$a$ è il minimo. Dunque $b>0$ e $f(b) = f(\sigma(b-1))
=\sigma'(f(b-1))$. Se $a=0$ abbiamo $f(a)=0'$ e quindi da $f(a)=f(b)$ 
otteniamo che $0'$ è nell'immagine di $\sigma'$ che è contrario 
agli assiomi di Peano. 
Se invece $a>0$ si avrà, come per $b$,
$f(a)=\sigma'(f(a-1))$ e dunque $\sigma'(f(a-1)) = \sigma'(f(b-1))$.
Per l'iniettività di $\sigma'$ si deduce $f(a-1)=f(b-1)$ da cui 
$a-1 \in A$. Ma questo è assurdo in quanto $a$ era il minimo di $A$.

Per dimostrare che $f$ è surgettiva consideriamo l'immagine 
$B'=f(\NN)$ e usiamo il principio di induzione su $\NN'$ 
per dimostrare che $B'=\NN'$.
Per prima cosa $0'\in B'$ in quanto $0'=f(0)$.
Se poi $n'\in B'$ allora esiste $n\in \NN$ tale che $f(n)=n'$.
Ma allora $f(\sigma(n))=\sigma'(f(n))=\sigma'(n')$ 
e dunque anche $\sigma'(n')\in B$. 
\end{proof}

\subsection{rappresentazione decimale dei numeri naturali}

Per rappresentare in modo efficiente qualunque numero naturale utilizziamo la 
\emph{notazione posizionale decimale}%
\mymargin{notazione posizionale decimale}%
\index{notazione!posizionale decimale}.
Ricordiamo che le cifre decimali $0,1,2,\dots, 9$ sono state definite in~\eqref{eq:cifre}
a pag~\pageref{eq:cifre}.
Il numero rappresentato da una sequenza finita di cifre decimali può essere 
definito ricorsivamente: una sequenza di una sola cifra rappresenta il numero 
corrispondente alla cifra stessa, una sequenza di $n+1$ cifre rappresenta il 
numero rappresentato dalle prime $n$ cifre, moltiplicato per $d=9+1$ (cioè dieci),
e sommato alla ultima cifra. 
Ad esempio la sequenza di cifre $4701$ (leggi: quattromilasettecentouno)
è definita così:
\[ 
  4701 = ((4\cdot d+7)\cdot d+0)\cdot d+1, \qquad d=9+1.
\]
Osservando che $10 = 1\cdot d + 0 = d$ si scriverà:
\begin{align*}
  4701 
  & = ((4\cdot 10 + 7)\cdot 10 +0)\cdot 10 + 1 \\
  & = \mathbf 4\cdot 10^3 + \mathbf 7\cdot 10^2 + \mathbf 0\cdot 10^1 + \mathbf 1 \cdot 10^0. 
\end{align*}

\begin{table}
  \begin{center}
    \def\tabcolsep{3pt}
    \begin{tabular}{>{\small}r|>{\small}r>{\small}r>{\small}r>{\small}r>{\small}r>{\small}r>{\small}r>{\small}r>{\small}r}
      $+$       & 1 & 2 & 3 & 4 & 5 & 6 & 7 & 8 & 9 \\ \hline
      1         & 2 & 3 & 4 & 5 & 6 & 7 & 8 & 9 & 10 \\
      2         & 3 & 4 & 5 & 6 & 7 & 8 & 9 & 10 & 11 \\
      3         & 4 & 5 & 6 & 7 & 8 & 9 & 10 & 11 & 12 \\
      4         & 5 & 6 & 7 & 8 & 9 & 10 & 11 & 12 & 13 \\
      5         & 6 & 7 & 8 & 9 & 10 & 11 & 12 & 13 & 14 \\
      6         & 7 & 8 & 9 & 10 & 11 & 12 & 13 & 14 & 15 \\
      7         & 8 & 9 & 10 & 11 & 12 & 13 & 14 & 15 & 16 \\
      8         & 9 & 10 & 11 & 12 & 13 & 14 & 15 & 16 & 17 \\
      9         & 10 & 11 & 12 & 13 & 14 & 15 & 16 & 17 & 18
      \end{tabular}
      \qquad
      \begin{tabular}{>{\small}r|r>{\small}r>{\small}r>{\small}r>{\small}r>{\small}r>{\small}r>{\small}r>{\small}r}
        $\cdot$       & 2 & 3 & 4 & 5 & 6 & 7 & 8 & 9 \\ \hline
        2         & 4 & 6 & 8 & 10 & 12 & 14 & 16 & 18 \\
        3         & 6 & 9 & 12 & 15 & 18 & 21 & 24 & 27 \\
        4         & 8 & 12 & 16 & 20 & 24 & 28 & 32 & 36 \\
        5         & 10 & 15 & 20 & 25 & 30 & 35 & 40 & 45 \\
        6         & 12 & 18 & 24 & 30 & 36 & 42 & 48 & 54 \\
        7         & 14 & 21 & 28 & 35 & 42 & 49 & 56 & 63 \\
        8         & 16 & 24 & 32 & 40 & 48 & 56 & 64 & 72 \\
        9         & 18 & 27 & 36 & 45 & 54 & 63 & 72 & 81
      \end{tabular}
    \end{center}
    \caption{Le \emph{tabelline} dell'addizione e della moltiplicazione.}
    \label{tab:tabelline}
\end{table}

Procedendo come nell'esempio~\ref{ex235} si può costruire la \emph{tabellina}
della somma, dove sono riportate le somme di tutte le coppie di cifre decimali
(tabella~\ref{tab:tabelline}).
Sfruttando la definizione e la proprietà distributiva del prodotto 
(teorema~\ref{th:operazioni_naturali})
non sarà difficile 
costruire anche la \emph{tabellina} della moltiplicazione, dove sono riportati i prodotti delle 
coppie di cifre decimali. 
Ad esempio una volta dimostrato che $7\cdot 7 = 49$ si può procedere 
a dimostrare che $7\cdot 8 = 56$ come segue:
\begin{align*}
  7\cdot 8 &= 7\cdot (7+1) = 7\cdot 7 + 7 = 49+7 
  = (4\cdot 10 + 9) + 7 = 4\cdot 10 + 16 \\
  &= 4\cdot 10 + 10\cdot 1 + 6 = 5\cdot 10 + 6 = 56.
\end{align*}

Una volta imparate a memoria le \emph{tabelline} è possibile svolgere le operazioni di addizione 
e moltiplicazione su qualunque coppia di numeri naturali espressi in forma decimale. 
L'algoritmo è quello della somma e della moltiplicazione in colonna che abbiamo imparato alla 
scuola elementare. Si può capire la correttezza di questi algoritmi con un esempio.
Proviamo a dimostrare che $34\cdot 56 = 1904$:
\begin{align*}
\mathbf{34}\cdot \mathbf{56} 
  &= \mathbf{34}\cdot (\mathbf 5\cdot 10+ \mathbf 6) 
   = \mathbf{34}\cdot \mathbf 6 + (\mathbf{34}\cdot \mathbf 5) \cdot 10 \\
  &= (\mathbf 3\cdot 10 + \mathbf 4)\cdot \mathbf 6 + (\mathbf 3\cdot 10 + \mathbf 4)\cdot \mathbf 5 \cdot 10 \\
  &= \mathbf{18}\cdot 10 + \mathbf{24} + (\mathbf{15}\cdot 10 + \mathbf{20})\cdot 10 \\
  &= \mathbf{1}\cdot 10^2+(\mathbf 8+\mathbf 2)\cdot 10 + \mathbf 4 + (\mathbf 1\cdot 10^2 + (\mathbf 5+\mathbf 2)\cdot 10 + \mathbf 0)\cdot 10 \\
  &= (\mathbf 1+\mathbf 1)\cdot 10^2 +\mathbf 0\cdot 10 + \mathbf 4 + (\mathbf 1\cdot 10^2 + \mathbf 7\cdot 10 + \mathbf 0)\cdot 10 \\
  &= \mathbf 2\cdot 10^2 + \mathbf 0\cdot 10 + \mathbf 4 + \mathbf 1\cdot 10^3+\mathbf 7\cdot 10^2 + \mathbf 0 \cdot 10\\
  &= \mathbf 1\cdot 10^3 + \mathbf 9\cdot 10^2 + \mathbf 0\cdot 10 + \mathbf 4
  = \mathbf{1904}.
\end{align*}

\subsection{fattoriale e semi-fattoriale}

Definiamo il 
\emph{fattoriale}%
\mymargin{fattoriale}%
\index{fattoriale} 
\index{"!}%
\index{$n$"!}%
di un numero naturale $n$
denotato con $n!$ (leggi: $n$ fattoriale) 
come il prodotto di tutti i numeri naturali da $1$ a $n$:
\[
  n! = \prod_{k=1}^n k.
\]
La definizione ricorsiva del prodotto permette di definire il fattoriale
come segue:
\[
  \begin{cases}
    0! = 1 \\
    (n+1)! = (n+1) \cdot n!
  \end{cases}
\]

\begin{exercise}
  \label{ex:6734098}%
  Utilizzando il principio di induzione
  si dimostri che per ogni $n\in \NN$:
  \[
    2^{n+1} \ge n+1, \qquad
    (n+1)! \ge 2^n, \qquad
    n^n \ge n!
  \]
\end{exercise}

\begin{table}
  \begin{center}
  \begin{tabular}{r|>{\small}r>{\small}r>{\small}r>{\small}r>{\small}r}
  $n$       & 0 & 1 & 2 & 3 & 4 \\
  \footnotesize $n+5$     & 5 & 6 & 7 & 8 & 9 \\ \hline
  $2^n$     & 1 & 2 & 4 & 8 & 16 \\
  \footnotesize $2^{5+n}$ & 32 & 64 & 128 & 256 & 512 \\
  \footnotesize $2^{10+n}$ & 1024 & 2048 & 4096 & 8192 & 16384 \\
  \footnotesize $2^{15+n}$ & 32768 & 65536 & 131072 & 262144 & 524288 \\
  \footnotesize $2^{20+n}$ & 1048576 & 2097152 & 4194304 & 8388608 & 16777216 \\  \hline
  $3^n$                    & 1 & 3 & 9 & 27 & 81 \\
  \footnotesize $3^{5+n}$  & 243 & 729 & 2187 & 6561 & 19683 \\  \hline
  $n!$      & 1 & 1 & 2 & 6 & 24 \\
  \footnotesize $(5+n)!$  & 120 & 720 & 5040 & 40320 & 362880 \\
  \footnotesize $(10+n)!$  & 3628800 & 39916800 & 479001600 & 6227020800 & 87178291200 \\ \hline
  \footnotesize $10^{3n}$  &  & K (chilo) & M (mega) & G (giga) & T (tera) \\ 
  \footnotesize $10^{3(n+5)}$  & P (peta) & E (exa) & Z (zetta) & Y (yotta) \\ \hline
  \footnotesize $2^{10n}$  &  & Ki (chibi) & Mi (mebi) & Gi (gibi) & Ti (tebi) \\
  \footnotesize $2^{10(n+5)}$ & Pi (pebi)& Ei (exbi) & Zi (zebi) & Yi (yobi)
  \end{tabular}
  \end{center}
  \caption{I primi valori (e nomi) di alcune delle sequenze che abbiamo definito.
  Sarà utile in particolare ricordare che  $2^{10}=1024$ 
  è molto vicino a $10^3=1000$: questo giustifica i nomi simili utilizzati
  per le potenze di $2^{10}$ e per le potenze di $10^3$.
  }
  \end{table}
  
  A volte sarà utile considerare anche i prodotti di solamente i numeri
  pari o i numeri dispari fino ad un certo numero $n$. Questo
  si chiama \emph{semi-fattoriale}%
\mymargin{semi-fattoriale}%
\index{semi-fattoriale} e si indica con $n!!$
  Lo possiamo definire separatamente sui numeri pari (cioè 
  i numeri che si possono scrivere nella forma $2n$ con $n\in \NN$)
  e i numeri dispari (che scriviamo nella forma $2n+1$):
  \index{"!"!}
  \index{$n$"!"!}
  \index{doppio fattoriale}%
  \index{fattoriale!doppio}%
  \index{fattoriale!semi-fattoriale}%
  \index{semi-fattoriale}%
  \begin{align*}
    (2n)!! &= \prod_{k=1}^n (2k) = 2 \cdot 4 \cdot 6 \cdots (2n) \\
    (2n+1)!! &= \prod_{k=1}^n (2k+1)1 \cdot 3 \cdot 5 \cdots (2n+1).
  \end{align*}
  
  \begin{remark}
  \label{rem:doppio_fattoriale}%
  Si osservi che risulta
  \[
    (2n)!! = (2\cdot 1) \cdot (2\cdot 2) \cdot (2\cdot 3) \cdots (2\cdot n)
          = 2^n \cdot n!
  \]
  mentre
  \[
    (2n+1)!! = \frac{(2n+1)!}{2n!!} = \frac{(2n+1)!}{2^n\cdot n!}.
  \]
  Queste formule permettono di esprimere il semi-fattoriale utilizzando
  il fattoriale intero e le potenze.
  \end{remark}
  
\begin{exercise}
  Si dia una definizione per induzione del semi-fattoriale
  (separatamente per i pari e per i dispari)
  e si dimostrino, per induzione, le formule nell'osservazione precedente.
\end{exercise}

\subsection{cardinalità finite}

Denotiamo con $[n] = \ENCLOSE{k\in \NN\colon k<n}$
l'insieme dei primi $n$ numeri naturali.
Si avrà $[n]=\ENCLOSE{0,1,\dots, n-1}$.
\mynote{
  Si noti che se $\NN$ è l'insieme 
  degli ordinali di Von Neumann \eqref{eq:vonNeumann} allora $[n]=n$.
}

\begin{definition}[cardinalità finite]
Se $A$ è un insieme e $n\in \NN$ diremo che 
$A$ ha $n$ elementi (o ha cardinalità $n$) e scriveremo 
\mymargin{numero di elementi}%
\[
  \#A = n
\]
se esiste una funzione bigettiva $f\colon [n]\to A$.
\end{definition}

Intuitivamente se $f\colon [n]\to A$ è una qualunque funzione, 
posto $a_1=f(0)$, $a_2=f(1)$, \dots, $a_{n}=f(n-1)$ 
avremo che l'immagine di $f$ è $f([n]) = \ENCLOSE{a_1,a_2,\dots,a_n}$.
Se $f$ è surgettiva $A=f([n])$ e se $f$ è iniettiva 
tutti gli elementi $a_1$, \dots, $a_n$ saranno distinti
e quindi effettivamente l'insieme $A=\ENCLOSE{a_1,a_2,\dots,a_n}$
ha $n$ elementi.

\begin{theorem}[cardinalità finite]
  \label{th:cardinali_finiti}%
Un insieme $A$ è finito
(nel senso di Dedekind, definizione~\ref{def:infinito})
se e solo se esiste $n\in \NN$ tale che $\#A = n$.
Inoltre tale $n$, se esiste, è unico.
\end{theorem}
\begin{proof}
\emph{Passo 1.} Dimostriamo innanzitutto che $[n]$ è finito.
Si tratta di dimostrare che se $f\colon [n]\to [n]$ è iniettiva allora è anche surgettiva.
Lo facciamo per induzione. 
Per $n=0$ notiamo che $[0]=\emptyset$ e c'è una unica funzione $f\colon \emptyset \to \emptyset$
(la funzione vuota: $\emptyset^\emptyset = \ENCLOSE{\emptyset}$)
che in effetti è bigettiva.
Supponiamo ora che $[n]$ sia finito e che $f\colon [n+1]\to [n+1]$ sia iniettiva.
Possiamo allora definire $g\colon[n]\to[n]$ ponendo:
\[
  g(k) = \begin{cases}
    f(k) & \text{se $f(k)\neq n$,}\\
    f(n) & \text{se $f(k)=n$.}
  \end{cases}
\]
Con un poco di attenzione si verifica facilmente che se $f$ è iniettiva allora anche $g$ è iniettiva
(l'idea è che $g$ utilizza gli stessi valori di $f$ scartando il valore $n$, visto che $n\not \in [n]$, 
e utilizzando al suo posto il valore $f(n)$).
Ma allora, per ipotesi induttiva, $g$ è bigettiva.
Dunque anche $f$ è bigettiva perché assume tutti i valori assunti da $g$  
e, avendo un punto in più ed essendo iniettiva deve assumere un ulteriore valore che non può 
che essere $n$.

\emph{Passo 2.}
Dimostriamo che se $\#A=n$ allora $A$ è finito.
Data $f\colon A\to A$ iniettiva dobbiamo dimostrare che $f$ è surgettiva.
Per ipotesi $\#A=n$ dunque esiste una funzione bigettiva $g\colon A\to [n]$.
La funzione $h=g\circ f\circ g^{-1}\colon [n]\to [n]$ è iniettiva, in quanto composizione di funzioni iniettiva.
Dunque, essendo $[n]$ finito per il passo 1, la funzione $h$ è anche surgettiva.
Ma allora anche $f$ è surgettiva in quanto $f=g^{-1}\circ h\circ g$.

\emph{Passo 3.} Dimostriamo che se $f\colon[n]\to A$ è iniettiva
ed esiste $g\colon[n+1]\to A$ iniettiva allora $f$ non può 
essere surgettiva.
Infatti se $f$ fosse bigettiva allora la funzione
$h\colon [n+1]\to [n]$ definita da $h=f^{-1}\circ g$ 
sarebbe iniettiva.
Ma la sua restrizione $h'=h \llcorner [n]$ sarebbe una funzione
iniettiva $h'\colon[n]\to[n]$ ma non assumerebbe mai il valore 
$h(n)$ dunque non sarebbe suriettiva.
Per definizione questo significa che $[n]$ è infinito, 
cosa che abbiamo già escluso al passo 1.

\emph{Passo 4.}
Dimostriamo che se $A$ è finito allora esiste $n\in \NN$ tale che $\#A=n$.
Passiamo alla contronominale e supponiamo che non esista $n\in \NN$ tale che $\#A=n$:
dobbiamo dimostrare che in tal caso $A$ è infinito.
Per fare questo consideriamo le famiglie di funzioni:
\[
  J_n(A) = \ENCLOSE{f\colon [n]\to A\colon f \text{ iniettiva}}.
\]

Ovviamente se $f\in J_{n+1}(A)$ allora $f$ ristretta a $[n]$ è elemento di $J_n(A)$.
Dunque se $J_n(A)=\emptyset$ allora anche $J_{n+1}(A)=\emptyset$.
 
Se $J_n(A)\neq \emptyset$ e $J_{n+1}(A)=\emptyset$ allora possiamo affermare che $\#A=n$ 
perché presa una funzione iniettiva $f\in J_n(A)$ se questa non fosse surgettiva potremmo 
estenderla ad una funzione iniettiva $f'\in J_{n+1}(A)$.

Dunque dovrà essere $J_n(A)\neq \emptyset$ per ogni $n\in \NN$.

Per il passo 3 sappiamo che ogni 
$f\in J_n(A)$ non è suriettiva dunque può essere estesa 
ad una funzione $g\in J_{n+1}(A)$
tale che $g$ coincide con $f$ su $[n]$. 
\mynote{qui si usa l'assioma della scelta}
Deve dunque esistere una successione di funzioni
$f_n\colon [n]\to A$ tali che $f_{n+1}$ coincide con $f_n$ su $[n]$.
I grafici di queste funzioni sono dunque una successione crescente di insiemi e la loro 
unione è il grafico di una funzione $f\colon \NN \to A$ che anch'essa 
coincide con $f_n$ su $[n]$. 
Dunque $f$ è iniettiva perché se fosse $f(n)=f(m)$
con $m<n$ si avrebbe anche $f_n(n)=f_n(m)$ che non può essere visto che ogni $f_n$ è iniettiva.
Sull'immagine di $f$ ho quindi una struttura isomorfa a $\NN$ che posso sfruttare per 
costruire una funzione iniettiva ma non suriettiva su $A$.
Definisco $h\colon A\to A$ ponendo:
\[
  h(x) = \begin{cases}
    f(f^{-1}(x)+1) &\text{se $x$ è nell'immagine di $f$},\\
    x &\text{altrimenti}
  \end{cases}
\]
ovviamente $h$ è iniettiva (in quanto $f$ lo è) 
ma non è suriettiva (in quanto $f(0)$ non è nell'immagine di $h$).

Abbiamo quindi mostrato che $A$ è infinito, come volevamo.

\emph{Passo 5.}
Dimostriamo che se $A\subset B$, $\#A=m$, $\#B=n$ allora $m\le n$.
Se per assurdo fosse $\#A=n$ e $\#A=m$ con $m>n$ allora 
avremmo una contraddizione con quanto affermato al passo 3
in quanto dovrebbe esistere una funzione bigettiva $f\colon[n]\to A$ 
(in quanto $\#A=n$) e dovrebbe esistere una funzione iniettiva 
$g\colon[n+1]\to A$ in quanto essendo $\#A=m$ c'è una funzione iniettiva 
$[m]\to A$ che posso restringere a $[n+1]$.
\end{proof}

\begin{theorem}[operazioni con le cardinalità finite]
  Se $A$ e $B$ sono insiemi finiti risulta:
  \begin{align}
    \#(A\cup B) &= \#A + \#B \quad \text{se $A\cap B=\emptyset$},\\
    \#(A\setminus B) &= \#A - \#B \quad \text{se $B\subset A$},\\
    \#(A\times B) &= \#A \cdot \#B, \\
    \#(B^A) &= \#B^{\#A}, \\
    \#(A!) &= (\#A)!, \\
    \#(\P(A)) &= 2^{\#A}.
  \end{align}
\end{theorem}
\begin{proof}
Per quanto riguarda l'unione disgiunta, se $f\colon [n]\to A$ e $g\colon [m]\to B$
sono bigettive, una funzione bigettiva $h\colon [n+m] \to A\cup B$ si può
definire così:
\[
  h(k) = \begin{cases}
    f(k) & \text{se } k<n,\\
    g(k-n) & \text{se } k\ge n.
  \end{cases}
\] 
Si verifica facilmente che $h$ è bigettiva.
Per quanto riguarda la differenza $A\setminus B$ basta osservare che se $B\subset A$
allora $A = (A\setminus B) \cup B$ e dunque $\#A = \#(A\setminus B) + \#B$ per il punto precedente.

Nei casi rimanenti, senza perdità di generalità, 
possiamo supporre che sia $A=[n]$. E procediamo per induzione su $n$.

Per il prodotto cartesiano il caso $n=0$ corrisponde a $\emptyset\times B = \emptyset$.
Il passo induttivo si dimostra osservando che $[n+1]\times B = [n]\times B \cup \ENCLOSE{n} \times B$
da cui, passando alle cardinalità diventa $(n+1)\cdot m = n\cdot m + m$.

Gli elementi di $B^{[n+1]}$ sono le funzioni $[n+1]\to B$ 
e si possono identificare
univocamente prendendo una funzione $f\colon [n]\to B$ e un valore $b\in B$ da assegnare 
al punto $n$. 
Dunque $B^{[n+1]}$ è in corrispondenza biunivoca con $B^{[n]}\times B$
e passando alle cardinalità si ottiene appunto $m^{n+1} = m^n\cdot m$.

L'insieme $A!$ è l'insieme delle funzioni bigettive $A\to A$. 
Visto che $A$ è finito l'insieme delle funzioni bigettive coincide 
con l'insieme delle funzioni iniettive. 
Supponiamo $A=[n]$ e procediamo per induzione su $n$.
Se $n=0$ si ha $\emptyset!=\ENCLOSE{\emptyset}$ che, passando alle cardinalità, diventa $0!=1$.
Per il passo induttivo osserviamo che una funzione iniettiva $[n+1]\to B$ 
si identifica univocamente scegliendo il valore $b\in B$ da assegnare al punto $n$ 
e poi prendendo una funzione $f\colon [n]\to B\setminus\ENCLOSE{b}$ iniettiva. 
Dunque $[n+1]!$ si identifica con $B\times [n]!$ e, passando alle cardinalità, questo diventa 
la definizione ricorsiva del fattoriale: $(n+1)! = (n+1)\cdot n!$.

Per l'insieme delle parti osserviamo che $\P(A)$ è in corrispondenza biunivoca con
l'insieme delle funzioni $A\to \ENCLOSE{0,1}$: basta associare ad ogni sottoinsieme $X$ di $A$ 
la funzione che vale $1$ su $X$ e $0$ altrimenti. 
Dunque $\#(\P(A)) = \#([2]^A) = 2^{\#A}$.
\end{proof}

\begin{theorem}[principio dei cassetti (o dei piccioni)]
  \index{principio!dei cassetti}%
  \index{cassetti, principio dei}%
  \index{piccioni, principio dei}%
Se $A$ e $B$ sono insiemi finiti e $\#A > n\cdot \#B$ allora 
per ogni $f\colon A \to B$ esiste $b\in B$ tale che 
\[
 \# f^{-1}(\ENCLOSE{b}) \ge n+1.
\]
\end{theorem}
\begin{proof}
Dimostriamo l'implicazione contronominale: se per ogni $b\in B$ si ha $\# f^{-1}(b) < n+1$
allora $\#A \le n\cdot \#B$.
Senza perdere di generalità supponiamo inoltre $A=[m]$.
Data la funzione $f\colon A\to B$ possiamo definire una nuova funzione $g\colon A\to B\times[n]$ 
ponendo:
\[
  g(a) = \left(f(a),\# \ENCLOSE{x\in [a]\colon f(x)=f(a)}\right).
\]
Abbiamo in pratica appiccicato una diversa etichetta ad ogni valore di $f$. 
In questo modo è facile convincersi che $g$ risulta essere iniettiva.
Dunque $\#A \le \#(B\times [n]) = n\cdot \# B$ che è la tesi contronominale.
\end{proof}

\begin{example}
In un gruppo di 1000 persone ci sono almeno tre persone che compiono gli anni lo stesso giorno.
Si applica il principio dei cassetti con $A$ insieme delle persone, 
$B$ insieme dei giorni dell'anno $f\colon A\to B$ che associa ad ogni persona il giorno
del suo compleanno. Visto che $1000 > 366 \cdot 2$ il teorema si applica con $n=2$.
\end{example}

