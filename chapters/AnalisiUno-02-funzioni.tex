\chapter{funzioni continue}

\begin{comment}
Lo scopo principale di questo corso è quello di studiare le
funzioni con dominio e codominio nei numeri reali:
\[
  f \colon A \subset \RR \to \RR.
\]
Ad esempio per ogni $n\in \ZZ$ abbiamo già definito la funzione
potenza:
\[
  f(x) = x^n.
\]
Se $n\ge 0$ si ha $f\colon \RR \to \RR$ (la funzione è definita
per ogni $x\in \RR$), se invece $n<0$ si ha
$f\colon \RR\setminus\ENCLOSE{0}\to \RR$ in quanto le potenze
con esponente negativo non sono definite se la base è $0$.
\end{comment}

\section{limiti}

Se una funzione non è definita in un punto potremmo comunque 
essere interessati a capire se, come per le funzioni continue,
intorno a quel punto i valori assunti dalla funzione si avvicinano 
ad un determinato valore.

\begin{definition}[limite finito]
  \label{def:limite_finito}%
  \index{limite!definizione}%
  Sia $f\colon A\subset \RR \to \RR$ una funzione, 
  $x_0\in \RR$ e $\ell \in \RR$.
  Diremo che $f(x)$ tende a $\ell$ (oppure: ha limite $\ell$) 
  quando $x$ tende a $x_0$ 
  e scriveremo
  \[
    f(x) \to \ell \qquad \text{per $x\to x_0$}
  \]
  se 
  \begin{equation}\label{eq:limite_finito}
    \forall \eps>0\colon \exists \delta>0 \colon 
    \forall x \in A\setminus\ENCLOSE{x_0}\colon
    \abs{x-x_0}<\delta \implies \abs{f(x)-\ell}< \eps.
  \end{equation}

  (Stessa identica definizione può essere data per funzioni 
  di variabile complessa e/o valori complessi.)
\end{definition}

Possiamo osservare che la definizione è molto simile alla 
definizione~\ref{def:continua}. 
In effetti confrontando le due definizioni si deduce che 
la condizione $f(x)\to \ell$ per $x\to x_0$ 
è equivalente a richiedere che l'estensione $\tilde f \colon A \cup \ENCLOSE{x_0}\to \RR$ 
definita da 
\[
 \tilde f(x) = \begin{cases}
    f(x) & \text{se $x\neq x_0$}\\ 
    \ell & \text{se $x=x_0$}
 \end{cases}
\]
sia continua nel punto $x_0$. 
Si osservi che non importa che la funzione $f$ sia definita nel punto 
$x_0$ e se è definita il valore $f(x_0)$ è irrilevante nella 
definizione di limite.

\begin{example}
  Si ha 
  \[
  \frac{\sqrt{x}-1}{x-1} \to \frac 1 2 \qquad \text{per $x\to 1$}. 
  \]
  Infatti, moltiplicando numeratore e denominatore 
  per $\sqrt x + 1$ si nota che se $x\ge 0$, $x\neq 1$ si ha:
  \[
      \frac{\sqrt x-1}{x-1}
      = \frac{x-1}{(x-1)(\sqrt x +1)}
      = \frac{1}{\sqrt x + 1}.
  \]
  Osserviamo che la funzione a lato sinistro è definita sull'insieme 
  $A = [0,1)\cup(1,+\infty)$ mentre la funzione a lato
  destro è definita su tutto l'intervallo $[0,+\infty)$.
  Inoltre quest'ultima funzione è continua 
  (in quanto la radice è una funzione continua e somma e rapporto di funzioni 
  continue è una funzione continua) che assume il valore $\frac 1 2$
  quando $x=1$. 
  Dunque la funzione sinistra pur non essendo definita per $x=1$ 
  ha limite $\frac 1 2$ per $x\to 1$.
\end{example}

Sarà molto utile estendere il concetto di limite 
ai punti all'infinito della retta reale estesa (o del piano complesso).
Per fare ciò la definizione di limite che abbiamo dato deve 
essere opportunamente modificata a seconda che la variabile o il valore 
del limite o entrambi siano finiti o infiniti (distinguendo poi $+\infty$ 
da $-\infty$). 
Molti altri casi devono essere poi presi in considerazione 
per introdurre i limiti destro e sinistro.
Per rendere la trattazione più compatta e omogenea useremo 
quindi l'approccio utilizzato nella \myemph{topologia} che è 
il campo della matematica che si occupa di limiti e continuità nella 
massima generalità possibile. 
Faremo questo senza definire cos'è la \emph{topologia} 
(cosa che ci porterebbe troppo lontano) ma semplicemente 
utilizzando la stessa terminologia in maniera trasparente.

\begin{definition}[intorno]
Sia $U\subset\bar \RR$. Andremo a definire cosa 
significa che $U$ è un \myemph{intorno} di $x_0$
per ogni $x_0\in \bar \RR$.

Se $x_0\in \RR$ diremo che $U$ è un intorno di $x_0$ 
quando vale questa condizione:
\[
\exists \eps>0 \colon \forall x\in \RR\colon 
\abs{x-x_0} < \eps \implies x\in U.
\]
Diremo invece che $U$ è un intorno di $+\infty$ 
se 
\[
 \exists M>0 \colon \forall x\in \bar \RR \colon 
 x > M \implies x\in U;
\]
e analogamente diremo che $U$ è un intorno di $-\infty$
se 
\[
 \exists M>0 \colon \forall x\in \bar \RR\colon 
 x < -M \implies x \in U.  
\]

Dato $x_0\in \bar \RR$ denotiamo con
\[
  \mathcal U_{x_0} = \ENCLOSE{U\in \mathcal P(\bar \RR)
  \colon \text{$U$ è un intorno di $x_0$}}
\]
la famiglia di tutti i suoi intorni.

(Definizioni analoghe possono essere date sul piano complesso: al finito 
la definizione è formalmente la stessa; gli intorni di $\infty\in \bar \CC$ 
sono gli insiemi $U\subset \bar \CC$ per cui esiste $M>0$ tale che per ogni $z\in \bar \CC$ 
se $\abs{z}>M$ si ha $x\in U$.)
\end{definition}

\begin{definition}[limite generale]
  Sia $f\colon A\subset \bar \RR \to \bar \RR$ una funzione, 
  sia $x_0\in \bar \RR$ e $\ell\in \bar \RR$. 
  Scriveremo 
  \[
    f(x)\to \ell\qquad\text{per $x\to x_0$}
  \]
  se vale la seguente condizione:
  \begin{equation}\label{def:limite}
  \forall U\in \mathcal U_\ell\colon 
  \exists V\in \mathcal U_{x_0}\colon 
  \forall x\in A\setminus\ENCLOSE{x_0}\colon 
  x\in V \implies f(x) \in U.
  \end{equation}
\end{definition}

Osserviamo innanzitutto che questa definizione è equivalente 
alla definizione~\ref{def:limite_finito}.
Infatti la proprietà~\eqref{eq:limite_finito} può essere riscritta 
così:
\[
\forall \eps>0\colon \exists \delta>0\colon 
\forall x\in A\setminus\ENCLOSE{x_0}\colon 
 x \in (x_0-\delta,x_0+\delta) \implies f(x) \in (\ell-\eps,\ell+\eps).
\]
Supponiamo che quest'ultima proprietà sia valida e dimostriamo che allora 
anche 
