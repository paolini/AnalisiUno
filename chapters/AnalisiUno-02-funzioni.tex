\chapter{funzioni continue, limiti e successioni}
%\label{ch:successioni}

\section{funzioni continue}
\label{sec:continuita}

Intuitivamente una funzione continua ha la proprietà
che se in un punto $x_0$ assume un valore $y_0=f(x_0)$ allora
in punti abbastanza vicini ad $x_0$ i valori assunti
non saranno molto diversi dal valore $y_0$.
Nella definizione seguente questo viene formalizzato:
fissato il punto $x_0$ e scelto un errore $\eps>0$
che siamo disposti a commettere sui valori della funzione
possiamo trovare un errore $\delta>0$ per cui nei punti
che differiscono da $x_0$ per meno di $\delta$ il valore
della funzione differisce da $y_0$ per meno di $\eps$.

\begin{definition}[continuità su $\RR$]
\label{def:continua}%
\index{funzione!continua}%
\index{continuità}%
Sia $f\colon A \subset \RR \to \RR$ una funzione. Diremo che
$f$ è \myemph{continua nel punto} $x_0\in A$ se
\begin{equation}\label{eq:continuita}
\forall \eps>0 \colon \exists \delta>0 \colon
\forall x\in A\colon
\abs{x-x_0} < \delta \implies \abs{f(x)-f(x_0)} < \eps.
\end{equation}

Diremo che $f\colon A \to \RR$ 
è \myemph{continua} se è continua in ogni punto $x_0 \in A$.
\end{definition}

%%%%%%%%%%%%%%%%%%%
%%%%%%%%%%%%%%%%%%%
%%%%%%%%%%%%%%%%%%%

Attenzione: la funzione $f(x)=\frac{1}{x}$
assume vicino a $x=0$ valori molto diversi tra loro (ad esempio $f(0.01)-f(-0.01)=200$)
ma ciò non toglie che la funzione possa essere continua in quanto il punto
$x=0$ non appartiene al dominio e quindi,
in base alla definizione precedente, non ha senso e non ha importanza
verificare se la funzione è continua in tale punto.

\begin{theorem}[continuità del reciproco]
\label{th:cont_reciproco}%
La funzione $f\colon \RR\setminus\ENCLOSE{0}\to \RR$ definita
da $f(x)=\frac 1 x$ è una funzione continua.
\end{theorem}
%
\begin{proof}
Siano $x,x_0\neq 0$.
Se prendiamo $\delta < \frac{\abs{x_0}}2$
e se $\abs{x-x_0} < \delta$ si avrà,
per disuguaglianza triangolare inversa,
$\abs{x} > \frac{\abs{x_0}}2$. Dunque
\[
\abs{\frac 1 x - \frac 1 {x_0}}
= \frac{\abs{x-x_0}}{\abs{x\cdot x_0}}
\le \frac{2\delta}{\abs{x_0}^2}.
\]
Si ottiene quindi la condizione $\abs{f(x)-f(x_0)}<\eps$
se si sceglie $\delta$ in modo che
risulti anche $\delta < \frac{\abs{x_0}^2\cdot \eps}{2}$.
\end{proof}

\begin{example}
La \myemph{funzione segno} $\sgn\colon \RR \to \RR$ definita da
\[
  \sgn(x) = \begin{cases}
  1 & \text{se $x> 0$}\\
  0 & \text{se $x=0$}\\
  -1 & \text{se $x<0$}
  \end{cases}
\]
è un esempio di funzione non continua.
\end{example}
%
\begin{proof}
Verifichiamo che la funzione non è continua
nel punto $x_0=0$.
Infatti se $x\neq 0$ si ha
\[
\abs{\sgn(x)-\sgn(x_0)} = \abs{\sgn(x)} = 1
\]
e quindi se scegliamo $\eps<1$ non è possibile
trovare $\delta>0$ per cui valga la condizione
di continuità~\eqref{eq:continuita}
nel punto $x_0=0$.
\end{proof}

\begin{exercise}
Dimostrare che le funzioni $f(x) = x$ e $g(x)=\abs{x}$ sono continue.
Dimostrare che le funzioni $h(x) = \lfloor x\rfloor$ e $k(x)=\lceil x \rceil$
non sono continue (ma sono continue in ogni punto
$x_0\in \RR \setminus \ZZ$).
\end{exercise}

\begin{definition}[operazioni sulle funzioni]
Sia $A \subset \RR$ e siano $f,g$ funzioni $A \to \RR$.
Possiamo allora definire
$f+g$, $-f$, $f-g$, $f\cdot g$, $\abs f$, $f^n$ (con $n\in \NN$)
e (se $g(x)\neq 0$ per ogni $x\in A$) anche $f/g$
come funzioni $A \to \RR$ mediante le seguenti ovvie
definizioni
\begin{gather*}
(f+g)(x) = f(x) + g(x), \qquad
(f-g)(x) = f(x) - g(x), \\
(f\cdot g)(x) = f(x) \cdot g(x), \qquad
(f/g)(x) = f(x) / g(x), \\
(-f)(x) = -(f(x)), \qquad
\abs f(x) = \abs{f(x)}, \qquad
f^n(x) = (f(x))^n.
\end{gather*}

Se $f\colon A \to B$ e $g\colon B\to C$ ricordiamo
che abbiamo definito la funzione composta
$g\circ f\colon A \to C$:
\[
  (g\circ f)(x) = g(f(x)).
\]

Se $c\in \RR$ è un numero considereremo a volte $c\colon A \to \RR$
come una funzione \emph{costante}.
Risulta quindi inteso che se $c\in \RR$ e $f\colon A \to \RR$
allora $c\cdot f$ è la funzione definita da
$(c\cdot f)(x) = c\cdot (f(x))$.
La somma e il prodotto per costante rendono l'insieme $\RR^A$
delle funzioni $A \to \RR$ uno spazio vettoriale sul campo $\RR$.
\end{definition}

\begin{theorem}[composizione di funzioni continue]
\label{th:continuita_composizione}%
Se $f$ e $g$ sono funzioni definite e continue
in uno stesso punto $x_0$
allora anche
\[
  f+g, \qquad
  f\cdot g, \qquad
  f-g, \qquad
  \abs{f}, \qquad
  f^n\ \text{(con $n\in \NN$)}
\]
sono funzioni definite e continue nel punto $x_0$.
Se inoltre $g(x_0)\neq 0$ anche la funzione
$f/g$
è definita e continua nel punto $x_0$.

Se $f\colon A\subset \RR \to \RR$ è una funzione continua
nel punto $x_0\in A$ e
$g\colon B\subset \RR \to \RR$ è una funzione
continua nel punto $y_0=f(x_0)\in B$ allora la funzione $g\circ f$
definita in $f^{-1}(B)$ è continua nel punto $x_0$.
\end{theorem}
%
\begin{proof}
Mostriamo prima di tutto la continuità
della composizione $g\circ f$.
Per la continuità di $f$ in $x_0$ e di $g$ in $y_0=f(x_0)$
si ha che per ogni $\eps>0$ esiste un $\gamma>0$
e per ogni $\gamma>0$ esiste un $\delta>0$ per cui
\begin{align*}
 \abs{x-x_0}< \delta &\implies \abs{f(x)-f(x_0)}<\gamma,\\
 \abs{y-y_0}< \gamma & \implies \abs{g(y)-g(y_0)}<\eps
\end{align*}
da cui
\[
\abs{x-x_0}< \delta
\implies \abs{f(x)-f(x_0)}< \gamma
\implies \abs{g(f(x))-g(f(x_0))} < \eps
\]
che non è altro che la condizione di
continuità~\eqref{eq:continuita} per $g\circ f$.

Se $f$ e $g$ sono continue nel punto $x_0$
allora per ogni $\eps'>0$ esistono $\delta_1$
e $\delta_2$ tali che
\begin{align*}
 \abs{x-x_0} < \delta_1 &\implies \abs{f(x)-f(x_0)} < \eps',
 \\
 \abs{x-x_0} < \delta_2 &\implies \abs{g(x)-g(x_0)} < \eps'.
\end{align*}
In particolare scegliendo $\delta = \min\ENCLOSE{\delta_1,\delta_2}$
se $\abs{x-x_0} < \delta$ valgono contemporaneamente
entrambe le stime:
\[
 \abs{f(x)-f(x_0)}< \eps', \qquad
 \abs{g(x)-g(x_0)}< \eps'.
\]

Dunque per la somma osserviamo che se $\abs{x-x_0}<\delta$
si ha
\begin{align*}
 \abs{(f+g)(x) - (f+g)(x_0)}
  &= \abs{f(x)+g(x)-f(x_0)-g(x_0)}\\
  &\le \abs{f(x)-f(x_0)} + \abs{g(x)-g(x_0)}
  \le 2 \eps'.
\end{align*}
Dato $\eps>0$ basterà quindi scegliere $\eps' = \eps / 2$
e $\delta$ come sopra per ottenere la condizione di continuità.

Per il prodotto si ha
\begin{align*}
  \MoveEqLeft \abs{f(x)g(x)-f(x_0)g(x_0)}\\
  &= \abs{f(x)g(x) - f(x_0)g(x) + f(x_0)g(x) - f(x_0)g(x_0)}\\
  &\le \abs{f(x)-f(x_0)}\cdot\abs{g(x)} + \abs{f(x_0)}\cdot \abs{g(x)-g(x_0)} \\
  &\le \eps' \abs{g(x)} + \abs{f(x_0)} \eps'.
\end{align*}
Osserviamo ora che $\abs{g(x)}\le \abs{g(x)-g(x_0)} + \abs{g(x_0)}$
e quindi $\abs{g(x)}\le \abs{g(x_0)}+\eps'$ da cui:
\begin{align*}
  \abs{f(x)g(x)-f(x_0)g(x_0)}
  &\le \eps' \enclose{\abs{g(x_0)}+\eps'} + \abs{f(x_0)} \eps' \\
  &= \eps' \cdot(\abs{f(x_0)} + \abs{g(x_0)} + \eps').
\end{align*}
Possiamo facilmente rendere questa quantità inferiore a
qualunque $\eps>0$: basterà prendere $\eps'<1$ e
\[
  \eps' < \frac{\eps}{\abs{f(x_0)} + \abs{g(x_0)} + 1}.
\]

La funzione $f^n$ è continua per induzione su $n$
in quanto prodotto di funzioni
continue: $f^{n+1} = f^{n} \cdot f$.

Abbiamo già visto che la funzione $h(x) = \frac{1}{x}$ è continua,
dunque se $g$ è continua anche la funzione $\frac{1}{g(x)} = h\circ g$
è continua essendo composizione di funzioni continue. 
Di conseguenza anche la funzione $\frac{f}{g} = f \cdot \frac{1}{g}$
è continua, essendo il prodotto di funzioni continue.

Analogamente la funzione $f-g$ è la somma di $f$ con $-g$ e
la funzione $-g$ è la composizione di $g$ con $h(x)=-x$.
E' immediato verificare che la funzione $h(x)=-x$ è continua
e dunque anche la differenza $f-g$ è continua.
Lo stesso vale per la funzione $\abs f$ che è la composizione
di $f$ con la funzione $h(x) = \abs{x}$.
\end{proof}

Il precedente teorema è molto importante ed utile in quanto
garantisce che ogni funzione definita tramite una espressione
che coinvolge solamente le funzioni e le operazioni
elencate nell'enunciato del teorema, risulta certamente
essere una funzione continua. Come nel seguente.

\begin{example}
La funzione
\[
f(x) = \frac{(x-3)\cdot x -\frac{1}{x+x^2}}{\abs{x-\frac{1-x^3}{x}}}
\]
è continua.

Si intende che tale funzione è definita sull'insieme degli $x\in \RR$
per cui tutte le operazioni coinvolte sono definite ovvero
$f\colon D \subset \RR \to \RR$
con
\[
  D = \ENCLOSE{x\in \RR \colon x+x^2 \neq 0,\ x\neq 0,\ \abs{x-\frac{1-x^3}{x}}\neq 0}.
\]

Per convincersi che questa funzione $f$ è continua
si nota che le funzioni $x$ e le costanti $3$ e $1$ sono
funzioni continue.
Ma allora, per il teorema~\ref{th:continuita_composizione},
anche le funzioni $x-3$ e $x^2=x\cdot x$ sono continue.
Dunque anche $(x-3)\cdot x$, $x+x^2$ e $x^3$ e $1-x^3$ sono continue.
Di conseguenza sono continue pure $\frac 1{1+x^2}$ e $\frac{1-x^3}{x}$.
E poi saranno continue anche $(x-3)\cdot x - \frac 1{1+x^2}$ e $x-\frac{1-x^3}{x}$
e quindi $x-\frac{1-x^3}{x}$ e pure $\abs{x-\frac{1-x^3}{x}}$. Infine sarà
dunque continua $f(x)$.
\end{example}

In particolare è chiaro che le funzioni lineare e quadratiche 
che abbiamo introdotto nelle sezioni precedenti sono funzioni continue
in quanto sono ottenute sommando e moltiplicando tra loro funzioni continue.

%%%%%%%%%%%%%%%%%%%

\begin{theorem}[continuità delle funzioni monotone]
  \label{th:monotona_continua}%
  \mymark{*}%
  \index{funzione!continua}%
  \index{funzione!monotona}%
  \index{funzione!suriettiva}%
  Sia $I\subset \RR$ un intervallo e sia
  $f\colon I \to \RR$ una funzione monotòna.
  Allora $f$ è continua se e solo se $f(I)$ è 
  un intervallo.
\end{theorem}
%
\begin{proof}
Senza perdita di generalità possiamo supporre che $f$ 
sia crescente. 
Dimostriamo innanzitutto che se $f(I)$ è intervallo allora $f$ è continua.

Prendiamo un punto $x_0\in I$ e sia $\eps>0$.
Vogliamo trovare $x_1<x_0$ tale che  
per ogni $x\in [x_1,x_0]\cap I$ si abbia $f(x)>f(x_0)-\eps$.
Siccome $f(I)$ è un intervallo che contiene il punto $f(x_0)$ 
ci sono due possibilità: o $f(x)>f(x_0)-\eps$ per ogni $x\in I$
e quindi possiamo scegliere $x_1<x_0$ a piacere
oppure esiste $x_1\in I$ tale che $f(x_1)=f(x_0) - \frac \eps 2$.
In questo secondo caso dovrà essere $x_1<x_0$ (in quanto $f$ è crescente)
e per monotonia si avrà, come voluto $f(x)\ge f(x_1) > f(x_0)-\eps$ 
per ogni $x\in [x_1,x_0]$.

In modo analogo possiamo trovare $x_2>x_0$ tale 
che per ogni $x\in [x_0,x_2]\cap I$ si abbia $f(x) < f(x_0)+\eps$.
Essendo $f$ crescente se $x\ge x_0$ si avrà anche $f(x)\ge f(x_0)$ 
e se $x\le x_0$ si avrà $f(x) \le f(x_0)$. 
Dunque per ogni $x\in [x_1,x_2]$ si avrà $\abs{f(x)-f(x_0)}<\eps$.
Basterà scegliere $\delta = \min\ENCLOSE{x_0-x_1,x_2-x_0}$ 
per ottenere la continuità di $f$ in $x_0$.

Supponiamo ora che $f$ sia una funzione continua e crescente.
Vogliamo dimostrare che $f(I)$ è un intervallo.

Siano $y_1, y_2 \in f(I)$ e sia $y_0\in \RR$
con $y_1 < y_0 < y_2$.
Vogliamo dimostrare che $y_0\in f(I)$ cioè che esiste $x_0\in I$
tale che $f(x_0)=y_0$.
Sappiamo che esistono $x_1,x_2$ in $I$ tali che
$f(x_1) = y_1$ e $f(x_2) = y_2$.
Dovrà essere $x_1<x_2$ perché $f(x_1)<f(x_2)$ e
$f$ è strettamente crescente.
Definiamo:
\[
 x_0 = \sup A
\qquad
\text{con } A=\ENCLOSE{x\in I\colon f(x)<y_0},
\]
e dimostriamo che $f(x_0)=y_0$. 
Chiaramente $x_0\in [x_1,x_2]$ in quanto 
$x_1\in A$ e $x_2$ è un maggiorante di $A$
e dunque $f$ è definita e continua in $x_0$.

Se fosse $f(x_0)<y_0$ scelto $\eps = y_0-f(x_0)$
per la definizione di continuità dovrebbe esistere
$\delta>0$ tale che per ogni $x\in I$ con $\abs{x-x_0}<\delta$
si abbia $\abs{f(x)-f(x_0)}<\eps$.
In particolare scelto $x=x_0+\frac \delta 2$
si avrebbe $f(x) < f(x_0) + \eps = y_0$.
Ma allora avremmo $x\in A$ che è assurdo in quanto
$\sup A = x_0 < x$.

Se fosse invece $f(x_0)>y_0$ posto $\eps = f(x_0)-y_0$,
grazie alla continuità di $f$ in $x_0$,
possiamo trovare $\delta>0$ tale che per ogni $x\in I$
con $\abs{x-x_0}<\delta$ si abbia $\abs{f(x)-f(x_0)}<\eps$.
In particolare scelto $t = x_0 - \frac \delta 2$
si ha $f(t) > f(x_0)-\eps = y_0$. 
Ma essendo $t<x_0 = \sup A$ sappiamo che $t$ non può essere 
un maggiorante di $A$ dunque deve esistere $x\in A$
tale che $t<x$. 
Ma se $x\in A$ allora $f(x) < y_0 < f(t)$
che è assurdo in quanto $f$ è crescente.

Abbiamo quindi mostrato che $f(x_0)=y_0$ e dunque
che $f(I)$ è un intervallo.
\end{proof}
%
\begin{exercise}
Si dimostri che se $I$ è un intervallo e $f\colon I\to \RR$
è iniettiva allora $f$ è monotona. 
\end{exercise}

Il teorema~\ref{th:monotona_continua} precedente ci permette di affermare che per $a>0$, $a\neq 1$ 
le funzioni 
$\exp_a$ e $\log_a$ sono continue. 
Infatti tali funzioni sono bigezioni monotone tra 
gli intervalli $\RR$ e $\RR_+$.

Grazie al teorema~\ref{th:continuita_composizione} sappiamo 
che se $n\in \NN$ la funzione $x^n$ è continua su tutto il suo 
dominio $\RR$. 
La funzione inversa $\sqrt[n]{x}$ è anch'essa crescente e 
bigettiva come funzione $[0,+\infty)\to[0,+\infty)$ e dunque 
anch'essa risulta essere continua.
Per simmetria (si veda l'esercizio~\ref{ex:simmetrica_continua}) 
possiamo concludere 
%che sia la potenza $x^n$ 
che le radici $\sqrt[n]{x}$ sono 
funzioni continue su tutto il loro dominio.
Anche la funzione $f(x) = x^\alpha$, $f\colon [0,+\infty)\to [0,+\infty]$ 
con $\alpha>0$ è crescente ed è invertibile (l'inversa è $x^{\frac 1 \alpha}$)
e dunque è surgettiva e continua.
Se $\alpha<0$ la funzione $f(x)=x^\alpha$ è definita sull'intervallo 
$(0,+\infty)$ ed è anch'essa continua in quanto 
composizione di funzioni continue: $x^{\alpha} = \frac{1}{x^{-\alpha}}$.

%
\begin{exercise}\label{ex:simmetrica_continua}
  Sia $f\colon \RR\to \RR$ una funzione pari oppure dispari.
  Se la restrizione di $f$ all'intervallo $[0,+\infty)$ 
  è continua allora $f$ è continua su tutto $\RR$.
\end{exercise}


\section{limite di funzione}

Se una funzione $f$ non è definita in un punto $x_0$ non ha senso chiedersi
se in tale punto è continua. 
Possiamo però chiederci se è possibile definire la funzione anche nel punto 
$x_0$ dando un valore opportuno $\ell$ in modo da rendere $f$ continua 
in quel punto. 
Se ciò è possibile diremo che la funzione $f(x)$ ha limite $\ell$ 
per $x$ che tende a $x_0$ e scriveremo:
\[
  f(x) \to \ell \qquad \text{per $x\to x_0$}
\]
(daremo tra poco una definizione con maggiore precisione e generalità).

Ad esempio la funzione $f(x) = \frac{x^2-1}{x-1}$ è una funzione definita per $x\neq 1$
e coincide, se $x\neq 1$ con la funzione lineare $\tilde f(x) = x+1$ definita 
su tutto $\RR$. 
Visto che $\tilde f$ è continua e $\tilde f(1)=2$
scriveremo:
\[
  \frac{x^2-1}{x-1} \to 2
  \qquad \text{per $x\to 1$.}
\]

Se estendiamo $f$ con un valore $\ell$ nel punto $x_0$ otteniamo 
in generale la funzione 
\[
  \tilde f(x) = \begin{cases}
    f(x) & \text{se $x\neq x_0$}\\
    \ell & \text{se $x=x_0$}
  \end{cases}  
\]
e la continuità di $\tilde f$ nel punto $x_0$ si scrive così:
\[
\forall\eps>0\colon \exists \delta>0\colon
\abs{x-x_0}<\delta \implies \abs{\tilde f(x)-\tilde f(x_0)}<\eps.  
\]
Osserviamo che se $x=x_0$ allora ovviamente $\abs{\tilde f(x)-\tilde f(x_0)}
= 0 < \eps$ dunque possiamo supporre, nella condizione precedente, 
che sia $x\neq x_0$ e dunque $\tilde f(x)=f(x)$. 
Inoltre visto che $\tilde f(x_0)=\ell$ si ottiene 
la seguente condizione:
\begin{equation}\label{eq:55338}
\forall \eps>0\colon \exists \delta >0 \colon 
  \forall x\neq x_0\colon
  \abs{x-x_0}<\delta \implies \abs{\tilde f(x) - \ell} < \eps.
\end{equation}
La~\eqref{eq:55338} potrebbe dunque essere utilizzata per definire 
il concetto di limite $f(x)\to \ell$ per $x\to x_0$.
Sarà però molto utile estendere il concetto di limite $f(x)\to \ell$ 
per $x\to x_0$ anche nei casi in cui $\ell$ e/o $x_0$ possano 
essere infiniti (cioè elementi dei reali estesi $\bar \RR$).

Per fare ciò osserviamo che
se definiamo $B_\rho(x_0) = \ENCLOSE{x\in \RR \colon \abs{x-x_0}<\rho}$
la condizione \eqref{eq:55338}
può essere scritta anche nella forma 
\[
  \forall \eps>0\colon \exists \delta >0 \colon 
  x \in B_\delta(x_0)\setminus\ENCLOSE{x_0} \implies \tilde f(x) \in B_\eps(\ell)  
\]
che a sua volta si può scrivere nella forma:
\[
  \forall \eps>0\colon \exists \delta >0 \colon 
  f(B_\delta(x_0)\setminus\ENCLOSE{x_0}) \subset  B_\eps(\ell).    
\]

L'idea è che gli insiemi del tipo\footnote{%
la lettera $B$ sta per \emph{ball} in quanto 
più in generale (se fossimo in $\RR^3$ invece che in $\RR$)
l'insieme dei punti che distano meno di $\rho$ da un punto fissato 
è l'interno di una sfera piena. 
In geometria una sfera piena si chiama \emph{palla} 
se contiene solo i punti interni (e non la superficie sferica)
e si chiama \emph{disco} o \emph{palla chiusa} se contiene 
anche i punti della superficie.
}%
$B_\rho(x_0)$ 
rappresentano i punti \emph{vicini} al punto $x_0$. 
In analogia potremmo pensare che i punti \emph{vicini} 
a $+\infty$ siano i punti di una qualunque semiretta 
del tipo $(M,+\infty]$.
Si dà quindi la seguente.


\begin{definition}[intorno]
Per $x\in \RR$ definiamo la famiglia degli \myemph{intorni} (basilari) di $x$
come l'insieme di tutti gli intervallini aperti, simmetrici, centrati in $x$:
\[
  \B_x = \ENCLOSE{ (x-\eps, x+\eps) \colon \eps>0}.
\]
Definiamo poi le famiglie 
degli \emph{intorni destri} e \emph{intorni sinistri}
\mymargin{intorni!destri/sinistri}
\index{intorni}
di $x$ come
\[
  \B_{x^+} = \ENCLOSE{ [x, x+\eps) \colon \eps>0},
  \qquad
  \B_{x^-} = \ENCLOSE{ (x-\eps , x] \colon \eps>0}
\]
e le famiglie degli intorni di $+\infty$ e $-\infty$ come segue
\[
  \B_{+\infty} = \ENCLOSE{ (\alpha,+\infty], \colon \alpha \in \RR },\qquad
  \B_{-\infty} = \ENCLOSE{ [-\infty, -\beta), \colon \beta\in \RR}.
\]

Per ogni $x\in \bar \RR = [-\infty, +\infty]$
risultano quindi definiti gli intorni $\B_x$ e per
ogni $x\in \RR$ sono definiti gli intorni $\B_{x^+}$ e $\B_{x^-}$.
\end{definition}

\begin{remark}
Sarebbe possibile definire in maniera analoga gli intorni dei punti in $\RR^n$
(per l'analisi di funzioni di più variabili)
o in $\CC$ (per l'analisi complessa).
Ma in questo corso e in questo capitolo in particolare siamo interessati 
allo studio delle funzioni di una singola variabile.

Su $\RR$ c'è una struttura di ordine totale che è utile preservare aggiungendo
due punti all'infinito: $+\infty$ e $-\infty$.
Su $\RR^n$ (e su $\CC$, che in questo contesto possiamo identificare con $\RR^2$)
non c'è una struttura d'ordine naturale e quindi
usualmente si considera un unico punto all'infinito $\infty$ i cui intorni
saranno
\[
  \B_\infty = \ENCLOSE{\ENCLOSE{x \in \RR^n\colon \abs{x}>R}\colon R>0}.
\]

Su $\RR^n$ (e su $\CC$) non esiste il concetto di intorno \emph{destro}
e \emph{sinistro} proprio perché questi concetti presuppongono un ordinamento.

In certi casi può tornare utile considerare un unico punto all'infinito,
denotato con $\infty$, anche in $\RR$ (in molti testi tale punto verrebbe
denotato con il simbolo $\pm\infty$)
e si potrebbero usare le notazioni $+\infty = \infty^-$ e $-\infty = \infty^
+$ visto che gli intorni di $+\infty$ e $-\infty$ sono in effetti intorni
unilaterali del punto all'infinito.
\end{remark}

\begin{definition}[limite di funzione]
\mymark{***}
Sia $A\subset \RR$ e $f\colon A \to \RR$. 
Sia $x_0\in [-\infty,+\infty]$
% un punto di accumulazione di $A$ 
e sia $\ell \in [-\infty,+\infty]$.
Allora diremo che la funzione $f$ ha limite $\ell$ per $x$ che tende a $x_0$ 
e scriveremo\mymargin{limite!di funzione}
\[
  f(x) \to \ell \qquad \text{per $x\to x_0$}
\]
se per ogni intorno di $\ell$ esiste un intorno di $x_0$ tale che
la funzione valutata nell'intorno di $x_0$, tolto eventualmente $x_0$,
assume valori
nell'intorno di $\ell$:
\begin{equation}\label{eq:def_limite}
  \forall V \in \B_\ell \colon \exists U \in \B_{x_0} \colon f(U\setminus\ENCLOSE{x_0}) \subset V.
\end{equation}

La stessa definizione può essere data restringendosi agli intorni destri/sinistri del punto $x_0$ (nel caso $x_0 \in \RR$). Si otterranno quindi le definizioni
di \emph{limite destro} e \emph{limite sinistro}
\mymargin{limite!destro/sinistro}
semplicemente sostituendo $\B_{x_0^+}$ o $\B_{x_0^-}$ al posto di 
$\B_{x_0}$ nella definizione
precedente. 
In tal caso scriveremo $f(x)\to \ell$ per $x\to x_0^+$ per il limite 
destro e $f(x)\to \ell$ per $x\to x_0^-$ per il limite sinistro.

Infine anche il risultato del limite può essere $\ell^+$ o $\ell^-$, 
in tal caso useremo $\B_{\ell^+}$ o $\B_{\ell^-}$ 
al posto di $\B_{\ell}$.
\end{definition}
  
\begin{example}
Si consideri la funzione segno:
\[
\sgn(x) =
\begin{cases}
  1 & \text{se $x>0$},\\
  0 & \text{se $x=0$},\\
  -1 & \text{se $x<0$}.
\end{cases}
\]
Si può verificare che
\[
\sgn(x) \to 1 \qquad \text{per $x\to 0^+$}
\]
e
\[
\sgn(x) \to -1 \qquad \text{per $x\to 0^-$}
\]
\end{example}

Abbiamo già visto che nel caso in cui $x_0\in \RR$ e $\ell\in \RR$ 
siano entrambi finiti 
la definizione di limite $f(x)\to \ell$ per $x\to x_0$
si traduce nella condizione~\eqref{eq:55338}.

Anche negli altri casi esplicitando le definizioni di intorno
si possono ottenere delle condizioni più esplicite.
Ad esempio la condizione $f(x)\to -\infty$ per $x\to x_0^+$
si traduce nel modo seguente:
\[
\forall \beta\in \RR\exists \delta>0 \colon x_0 < x < x_0+\delta 
\implies f(x) < -\beta.  
\]
Visto che ci sono cinque diversi casi per il punto $x_0$:
$x_0\in \RR$, $x_0^+$, $x_0^-$, $+\infty$, $-\infty$ e altrettanti 
casi per $\ell$ (in effetti anche il risultato del limite 
può essere \emph{destro} o \emph{sinistro}), si ottengono 
in tutto 25 diverse definizioni di limite.

In tutte queste definizioni di limite è sottointeso che i punti 
$x$ che vengono presi in considerazione sono punti 
del dominio di $f\colon A \subset \RR \to \RR$.
Accade allora che se esiste un intorno $V\in \B_{x_0}$
per cui $A \cap V \setminus \ENCLOSE{x_0}$ è vuoto allora 
la condizione di limite è vuota ed è quindi sempre verificata 
qualunque sia il valore di $\ell$. 
E' quindi inutile fare il limite per $x\to x_0$ 
se $x_0$ non soddisfa la seguente.

\begin{definition}[punto di accumulazione]
  \mymark{*}
  Siano $A\subset  \RR$ un insieme e $x_0\in [-\infty, +\infty]$.
  Diremo che $x_0$ è un \myemph{punto!di accumulazione} di $A$
  se ogni intorno di $x_0$ contiene punti di $A$ diversi da $x_0$, ovvero:
  \[
   \forall U \in \B_{x_0}\colon (A\setminus \ENCLOSE{x}) \cap U \neq \emptyset.
  \]

  Stessa definizione si può dare per $x_0^+$ e $x_0^-$ 
  utilizzando gli intorni destri/sinistri di $x_0$. 
\end{definition}

\begin{theorem}[unicità del limite]
\mymark{*}
Sia $A\subset \RR$, $f\colon A \to \RR$, $x_0$
punto di accumulazione per $A$ e $\ell_1, \ell_2\in [-\infty,+\infty]$.
Se per $x\to x_0$ si ha
\[
  f(x) \to \ell_1 \qquad\text{e}\qquad f(x) \to \ell_2
\]
allora $\ell_1=\ell_2$.

Risultato analogo si ha per i limiti destro e sinistro: $x\to x_0^+$, 
$x\to x_0^-$.
\end{theorem}
%
\begin{proof}
\mymark{*}
Supponiamo per assurdo che $\ell_1\neq \ell_2$.
Allora esiste un intorno $V_1$ di $\ell_1$ ed un intorno $V_2$ di $\ell_2$
tali che $V_1\cap V_2 = \emptyset$ (basta prendere degli intorni abbastanza piccoli). 
Ma per le definizioni di limite $f(x)\to \ell_1$ e $f(x)\to \ell_2$ 
dovranno esistere $U_1$ e $U_2$ intorni di $x_0$ su cui si ha 
$f(U_1)\subset V_1$ e $f(U_2)\subset V_2$. 
Ma allora $f((A\setminus\ENCLOSE{x_0})\cap U_1)\cap f((A\setminus\ENCLOSE{x_0})\cap U_2)\subset V_1\cap V_2 = \emptyset$... 
e questo è assurdo perché certamente $U_1\cap U_2$ 
contiene punti di $A$ diversi da $x_0$ in quanto 
$U_1$ e $U_2$ sono uno contenuto nell'altro e $x_0$ 
è un punto di accumulazione per $A$.
\end{proof}

Il teorema precedente garantisce che se $x_0$ è un punto di accumulazione 
del dominio di $f$ allora il limite per $x\to x_0$ se esiste è unico. 
In tal caso possiamo dunque dare la seguente.
%
\begin{definition}[operatore di limite]
Sia $f\colon A\subset \RR \to \RR$ una funzione e $x_0$ 
un punto di accumulazione per $A$. 
Se esiste $\ell\in\closeinterval{-\infty}{+\infty}$ tale che
$f(x)\to \ell$ per $x\to x_0$ allora poniamo
\[
  \lim_{x\to x_0} f(x) = \ell.
\]
Lo stesso si può fare per il limite destro $x\to x_0^+$ 
e sinistro $x\to x_0^-$.
\end{definition}

\begin{theorem}[collegamento tra limiti e continuità]%
\mymark{***}%
  Sia $A\subset \RR$, $f\colon A \to \RR$. 
  Se $x_0\in A$ è un punto di accumulazione di $A$
  allora $f$ è continua in $x_0$ se e solo se
  \[
    \lim_{x\to x_0}f(x) = f(x_0).
  \]
  Se $x_0\in A$ non è punto di accumulazione diremo 
  che $x_0$ è un \myemph{punto isolato} di $A$.
  In tal caso la funzione $f$ è sempre continua nel punto $x_0$.
\end{theorem}
  
  \begin{proof}
  In base alla definizione~\ref{def:continua} la funzione $f$ è continua nel
  punto $x_0$ se
  \[
   \forall \eps>0 \colon \exists \delta >0 \colon
   \forall x \in A\colon
   \abs{x-x_0}<\delta \implies \abs{f(x)-f(x_0)} < \eps
  \]
  mentre la definizione di limite $f(x)\to f(x_0)$ per $x\to x_0$
  si espande in
  \[
  \forall \eps>0 \colon \exists \delta>0\colon
  \forall x \in A, x\neq x_0\colon
  \abs{x-x_0}<\delta \implies \abs{f(x)-f(x_0)} < \eps.
  \]
  L'unica differenza è che nella definizione di limite
  c'è la condizione $x\neq x_0$. Ma visto che per $x=x_0$
  si ha $\abs{f(x)-f(x_0)}=0$ tale condizione è in questo caso 
  inutile e quindi le due definizioni sono equivalenti.

  Se il punto $x_0$ è isolato la definizione di continuità
  è sempre verificata in quanto esiste un $\delta>0$ 
  tale che il punto $x=x_0$ è l'unico punto di $A$ 
  nell'intorno $(x_0-\delta,x_0+\delta)$.
  \end{proof}

\begin{example}
  Non è difficile convincersi che gli unici punti di accumulazione 
  per l'insieme $\ZZ$ sono $+\infty$ e $-\infty$.
  Dunque qualunque funzione $f\colon \ZZ \to \RR$ è continua in quanto 
  tutti i punti del suo dominio sono punti isolati.
\end{example}
  
\begin{theorem}[località del limite]
Il limite di una funzione per $x\to x_0$ dipende solamente dai valori di $f$
in un intorno di $x_0$ e non dipende dal valore di $f$ in $x_0$.

Più precisamente: se $A,B\subset \RR$, $x_0\in \RR$ sono tali che 
esiste un intorno $V$ di $x_0$ per cui 
$(A\setminus\ENCLOSE{x_0}) \cap  V = (B\setminus \ENCLOSE{x_0}) \cap V$ 
e $f(x)=g(x)$ per ogni $x\in(A\setminus\ENCLOSE{x_0}) \cap  V$ 
allora se $f(x)\to \ell$ per $x\to x_0$ anche $g(x)\to \ell$ 
per $x\to x_0$.
\end{theorem}
%
\begin{proof}
  Basta osservare che nella definizione di limite 
  non è restrittivo supporre che l'intorno del punto $x_0$ 
  sia sempre preso all'interno dell'intorno $V$ su cui 
  le due funzioni coincidono.
\end{proof}

\begin{theorem}[restrizione del limite]
Se una funzione ha limite $\ell$ per $x\to x_0$ 
e se restringiamo l'insieme di definizione della funzione 
allora il limite della funzione non cambia. 
Più precisamente
se $f\colon A \to \RR$ è una funzione tale che $f(x)\to \ell$ per $x\to x_0$
e se $B \subset A$ e $g\colon B\to \RR$ è la restrizione di $f$ 
a $B$ allora anche $g(x)\to \ell$ per $x\to x_0$. 
\end{theorem}
%
\begin{proof}
Il teorema segue immediatamente dalla definizione di limite se si osserva
che restringendo il dominio la condizione di validità del limite si indebolisce
in quanto gli intorni di $x_0$ vengono intersecati con il dominio della funzione.
\end{proof}

Si osservi che a differenza del teorema sulla località del limite è
possibile che la funzione ristretta $g$ abbia limite quando la funzione
$f$ non aveva limite.
Si osservi anche che in entrambi questi teoremi sarà opportuno 
che $x_0$ sia un punto di accumulazione, altrimenti la condizione $f(x)\to \ell$ 
risulta essere vuota.

\begin{theorem}[legame tra limite, limite destro e limite sinistro]%
\mymark{*}
Sia $A\subset \RR$, $f\colon A \to \RR$ una funzione e $x_0$ un punto di accumulazione
di $A$. Sia $A^+ = A \cap [x_0,+\infty)$ e $A^- = A \cap (-\infty, x_0]$.

Se $x_0$ è punto di accumulazione sia di $A^+$ che di $A^-$
allora si ha
\[
  \lim_{x\to x_0} f(x) = \ell
\]
se e solo se
\[
  \lim_{x\to x_0^+} f(x) = \lim_{x\to x_0^-} f(x) = \ell.
\]

Se $x_0$ è punto di accumulazione di $A^+$ ma non di $A^-$ allora
i limiti
\[
  \lim_{x\to x_0} f(x) \qquad \text{e}\qquad \lim_{x\to x_0^+} f(x)
\]
sono equivalenti. Analogamente se $x_0$ è punto di accumulazione
di $A^-$ ma non di $A^+$ risultano equivalenti
\[
  \lim_{x\to x_0} f(x) \qquad \text{e}\qquad \lim_{x\to x_0^-} f(x).
\]
\end{theorem}
%
\begin{proof}
Si tratta semplicemente di verificare le definizioni di limite sfruttando il fatto che intorni di un punto $x_0$ sono formati dall'unione di intorno destro e intorno sinistro.
\end{proof}

\begin{theorem}[limite della funzione composta/cambio di variabile]
\label{th:limite_composta}
Siano $A\subset \RR$, $B\subset \RR$,
$x_0$ un punto di accumulazione di $A$,
$y_0$ un punto di accumulazione di $B$,
$\ell\in [-\infty,+\infty]$.
Siano $f\colon A \to B$, $g\colon B\to \RR$
funzioni tali che $f(x)\neq y_0$ se $x\neq x_0$ e 
\[
  \lim_{x\to x_0} f(x) = y_0,
\qquad
  \lim_{y\to y_0} g(y) = \ell.
\]
Allora nel secondo limite si può porre $y=f(x)$ e al posto di $y\to y_0$ 
si può mettere $x\to x_0$ (in quanto il primo limite ci dice che se $x\to x_0$ 
allora $y\to y_0$) e dunque vale:
\[
 \lim_{x\to x_0} g(f(x)) = \ell.
\]
\end{theorem}
%
\begin{proof}
Visto che $g(y)\to \ell$
per ogni $U$ intorno di $\ell$ deve esistere un $V$ intorno di $y_0$
tale che $g((B\setminus\ENCLOSE{y_0})\cap V) \subset U$
e visto  che $f(x)\to y_0$ deve esistere un intorno $W$ di $x_0$
tale che $f((A\setminus\ENCLOSE{x_0})\cap W) \subset V$.
Ma visto che per ipotesi $f$ assume valori in $B\setminus\ENCLOSE{y_0}$
si ha anche $f((A\setminus\ENCLOSE{x_0})\cap W)\subset (B\setminus\ENCLOSE{y_0}) \cap V$
e quindi
\[
  g(f((A\setminus\ENCLOSE{x_0})\cap W)) \subset g((B\setminus \ENCLOSE{y_0}) \cap V)
  \subset U
\]
che significa che $g(f(x)) \to \ell$.
\end{proof}

\begin{exercise}
  Si faccia un esempio di una funzione $f\colon\RR\to \RR$ 
  e una funzione $g\colon\RR\to \RR$ tali che 
  \[
  \lim_{x\to 0} f(x) = 0, \qquad 
  \lim_{x\to 0} g(x) = 0
  \]
  ma
  \[
  \lim_{x\to 0} g(f(x)) = 1.
  \]
  Quale ipotesi nel teorema precedente viene a mancare?
\end{exercise}

\begin{theorem}[limite di funzioni monotòne]
  \mark{**}%
  \label{th:limite_monotona}%
Sia $f\colon A \subset \RR \to \RR$ una funzione crescente. 
Se $x_0$ è un punto di accumulazione sinistro per $A$ 
il seguente limite esiste e vale
\[
   \lim_{x\to x_0^-}f(x) = \sup f(\ENCLOSE{x\in A\colon x<x_0})
\]
e se $x_0$ è un punto di accumulazione destro per $A$ 
il seguente limite esiste e vale
\[
   \lim_{x\to x_0^+} f(x) = \inf f(\ENCLOSE{x\in A\colon x>x_0}).
\]
In particolare se $+\infty$ è punto di accumulazione per $A$ 
si ha 
\[
  \lim_{x\to +\infty} f(x) = \sup f(A)
\]
e se $-\infty$ è punto di accumulazione per $A$ si ha 
\[
  \lim_{x\to -\infty} f(x) = \inf f(A).
\]

Gli stessi risultati valgono per le funzioni decrescenti, 
scambiando $\sup$ e $\inf$.
\end{theorem}
%
\begin{proof}
Supponiamo che $f$ sia crescente e consideriamo il limite 
sinistro $x\to x_0^-$. Può anche essere $x_0=+\infty$, in tal 
caso il limite sinistro $x\to x_0^-$ equivale a $x\to +\infty$.
Poniamo $B=\ENCLOSE{x\in A\colon x<x_0}$
$\ell=\sup B$ e ricordiamo le proprietà che caratterizzano 
l'estremo superiore
(sappiamo che $B$ non è vuoto in quanto $x_0^-$ per ipotesi 
è un punto di accumulazione per $A$):
\begin{gather*}
  \forall x \in B \colon f(x) \le \ell \\
  \forall y < \ell \colon \exists \alpha \in B \colon f(\alpha) > y.
\end{gather*}
Siccome $f$ è crescente, dalla seconda condizione 
si ottiene che per ogni $x>\alpha$ si ha $f(x)\ge f(\alpha)> y$
e mettendo insieme le due condizioni si ottiene che per ogni $y<\ell$
esiste $\alpha\in\RR$ tale che per ogni $x>\alpha$ si ha $y<f(x)\le \ell$.
Certamente si ha $\alpha < x_0$ perché $\alpha\in B$.   
Dunque la condizione $x>\alpha$ identifica un intorno sinistro 
del punto $x_0$ e si ha dunque $f(x)\to \ell$ per $x\to x_0^-$.

Ragionamento analogo si può fare per il limite destro  
e per le funzioni decrescenti.
\end{proof}

\begin{exercise}
  Utilizzando il teorema precedente si studino le \emph{discontinuità}
  delle funzioni monotone. 
  Si dimostri quindi che l'insieme dei punti in cui una funzione monotona 
  \emph{non} è continua può essere messo in corrispondenza biunivoca 
  con un sottoinsieme di $\QQ$ e dunque tale insieme è numerabile.
\end{exercise}

Il teorema precedente si applica in particolare alle funzioni 
esponenziali, potenze, radici e logaritmi negli estremi dei loro 
domini. Ad esempio è chiaro che se $x_0\in (0,+\infty)$ 
si ha 
\[
  \lim_{x\to x_0} \log_a x = \log_a x_0
\]
in quanto il logaritmo è una funzione continua. 
Se $a>1$ il logaritmo è crescente e la sua immagine è tutto $\RR$ 
dunque si ha 
\[
  \lim_{x\to 0^+} \log_a x = \inf \RR = -\infty
  \qquad
  \lim_{x\to +\infty} \log_a x = \sup \RR = +\infty.
\]
Per l'esponenziale avremo 
\[
 \lim_{x\to x_0} a^x = a^{x_0}
\]
se $x_0\in \RR$ per continuità. 
Se $a>1$ l'esponenziale è crescente 
ed ha immagine $(0,+\infty)$ dunque 
\[
  \lim_{x\to -\infty} a^x = 0, \qquad 
  \lim_{x\to +\infty} a^x = +\infty.
\]
Se $0<a<1$ i limiti all'infinito 
si scambiano in quanto $a^{x}= \frac{1}{a^{-x}}$.
Per le potenze avremo, se $x_0\in[0,+\infty)$ e $\alpha>0$
\[
  \lim_{x\to x_0} x^\alpha = x_0^\alpha
\]
per continuità. Invece essendo $x^\alpha$ crescente e bigettiva 
su $[0,+\infty)$ si avrà 
\[
  \lim_{x\to +\infty} x^\alpha = +\infty.
\]
Se $\alpha<0$ la funzione $x^\alpha$ è continua, decrescente e bigettiva 
su $(0,+\infty)$ e dunque in tal caso:
\[
  \lim_{x\to +\infty} x^\alpha = 0.
\]
Tutti questi limiti \emph{notevoli} possono essere facilmente 
ricordati se teniamo in mente i grafici delle funzioni elementari
Figura~\ref{fig:esponenziale_logaritmo} e~\ref{fig:potenza_radice}.

Anche la funzione valore assoluto $f(x)\abs{x}$ è separatamente 
monotona sugli intervalli $[0,+\infty)$ e $(-\infty,0]$. 
Dunque sappiamo che 
\[
  \lim_{x\to +\infty} \abs{x} = \abs{+\infty} = +\infty, 
  \lim_{x\to -\infty} \abs{x} = \abs{-\infty} = +\infty.
\]
Se $x_0\in \RR$ ovviamente si ha pure
\[
  \lim_{x\to x_0} \abs{x_0} = \abs{x_0}
\]
in quanto è banale verificare che la funzione $\abs{x}$ è continua.

Ci sarà anche utile osservare che vale anche questa proprietà 
\[
\lim_{x\to x_0} \abs{f(x)} = 0 \iff 
\lim_{x\to x_0} f(x) = 0
\]
in quanto le definizioni di limite $f(x)\to 0$ e $\abs{f(x)}\to 0$ 
coincidono in quanto $\abs{f(x)-0} = \big\lvert{\abs{f(x)} - 0}\big\rvert$.

\begin{theorem}[permanenza del segno]%
\mymark{***}%
\index{permanenza del segno}%
\index{teorema!della permanenza del segno}%
\mynote{permanenza del segno}%
Se
\[
  \lim_{x\to x_0} f(x) > 0
\]
allora esiste un intorno $U$ di $x_0$ tale che 
per ogni $x\in U$ si ha $f(x) > 0$.
\end{theorem}
%
\begin{proof}
Sia $\ell\in [-\infty,+\infty]$ il valore del limite.
Se $\ell>0$ esiste certamente un intorno $U$ di $\ell$ 
tale che $U\subset (0,+\infty]$ (se $\ell\in \RR$ basta prendere 
$(\ell/2,3 \ell/2)$, se $\ell=+\infty$ basta prendere $(1,+\infty]$).
Per la definizione di limite esiste $U$ intorno di $x_0$ 
tale che $f(U)\subset V$ e il risultato segue.
\end{proof}

\begin{theorem}[operazioni con i limiti di funzione]
Se
\[
  \lim_{x\to x_0}f(x) = \ell_1,\qquad
  \lim_{x\to x_0}g(x) = \ell_2
\]
allora si ha
\begin{gather*}
  \lim_{x\to x_0} \enclose{f(x) + g(x)} = \ell_1 + \ell_2, \qquad
  \lim_{x\to x_0} \enclose{f(x) - g(x)} = \ell_1 - \ell_2, \\
  \lim_{x\to x_0} f(x)\cdot g(x) = \ell_1 \cdot \ell_2, \qquad
  \lim_{x\to x_0} \frac{f(x)}{g(x)} = \frac{\ell_1}{\ell_2}
\end{gather*}
sempre che le operazioni utilizzate sul lato destro delle uguaglianze
siano state definite\footnote{%
Si veda la sezione~\ref{sec:reali_estesi}.
I casi in cui le operazioni non sono definite si chiamano 
usualmente \emph{forme indeterminate}
\index{forme indeterminate}
e sono: $+\infty - (+\infty)$, 
$-\infty - (-\infty)$, $+\infty + (-\infty)$, 
$-\infty + (+\infty)$, $0\cdot (+\infty)$, $+\infty \cdot 0$
$0\cdot (-\infty)$, $-\infty \cdot 0$, $\frac 0 0$,
$\frac{+\infty}{+\infty}$, $\frac{-\infty}{-\infty}$,
$\frac{+\infty}{-\infty}$, $\frac{-\infty}{+\infty}$. 
}.
Inoltre se $g(x)\to 0^+$ 
si ha
\[
    \lim_{x\to x_0} \frac{1}{g(x)} = +\infty
\]
(moralmente $\frac{1}{0^+} = +\infty$).
\end{theorem}
%
\begin{proof}
Consideriamo la somma dei limiti
e innanzitutto il caso in cui $\ell_1$ ed $\ell_2$ 
siano entrambi finiti. 
In tal caso dato un qualunque intorno $U=(\ell-\eps,\ell+\eps)$ 
se prendiamo gli intorni $U_1=(\ell_1-\eps/2,\ell_1+\eps/2)$ 
e $U_2 = (\ell_2-\eps/2,\ell_2+\eps/2)$ si possono trovare 
degli intorni $V_1$ e $V_2$ di $x_0$ per cui si ha 
$f(V_1\setminus\ENCLOSE{x_0}) \subset U_1$ e 
$f(V_2\setminus \ENCLOSE{x_0}) \subset U_2$
e a maggior ragione questo risulta se prendiamo $V=V_1\cap V_2$ 
al posto di $V_1$ e $V_2$. 
Visto che $U_1+U_2 = U$ si avrà allora 
$(f+g)(V\setminus\ENCLOSE{x_0})\subset U$.

Sempre nel caso della somma 
se $\ell_1=+\infty$ e $\ell_2\neq -\infty$
allora $\ell_1+\ell_2=+\infty$ e 
dato un intorno di $+\infty$ 
della forma $U=(\alpha,+\infty]$ con $\alpha>0$
possiamo prendere 
$U_1 = (2\alpha,+\infty)$ e $U_2 = (\ell_2-\alpha,\ell_2+\alpha)$ 
se $\ell_2\in \RR$ oppure $U_2 = (\alpha,+\infty)$ se $\ell_2=+\infty$.
In ogni caso si ha $U_1+U_2 = U$ e si procede quindi come nel caso precedente.
Il caso $\ell_1=-\infty$ si dimostra in maniera del tutto analoga.

Per la differenza basta osservare che $f(x)-g(x) = f(x) + (-g(x))$.
Se $\ell_2$ è finito la continuità della funzione $h(y)=-y$
ci garantisce che $-g(x)\to -\ell_2$.
Lo stesso si verifica facilmente quando $\ell_2$ è infinito 
(basta osservare che se $U=(\alpha,+\infty+]$ è un intorno di $+\infty$
allora $-U = [-\infty,-\alpha)$ è un intorno di $-\infty$). 
Dunque il limite della differenza si riconduce al limite della somma.

Per quanto riguarda il prodotto ci ricordiamo che grazie 
al teorema~\ref{th:estensione_additiva} il gruppo additivo totalmente 
ordinato $\RR = \openinterval{-\infty}{+\infty}$ 
è isomorfo al gruppo moltiplicativo $\RR_+ = \openinterval 0 {+\infty}$
e di conseguenza $\bar \RR = \closeinterval{-\infty}{+\infty}$
corrisponde a $\closeinterval {0^+}{+\infty}$
(l'isomorfismo $\RR_+\to \RR$ è dato dalla funzione $\log_a x$ con $a>1$ 
fissato,
tale funzione preserva gli intorni dei punti corrispondenti salvo il fatto 
che gli intorni di $-\infty$ si trasformano in intorni destri di $0$).

Dunque se $f > 0$ e $g > 0$ i risultati per il limite del prodotto 
sono conseguenza dei risultati analoghi per il limite della somma.
Cambiando segno ad una delle due funzioni o ad entrambe ci si può 
ricondurre al caso $f > 0$ e $g > 0$ quando le due funzioni assumono 
un segno costante in almeno un intorno del punto $x_0$.
Per il teorema della permanenza del segno questo è vero se 
$\ell_1\neq 0$ e $\ell_2\neq 0$: il limite del prodotto in questo caso 
è dunque $\ell_1\cdot \ell_2$.
Se invece $\ell_1=0$ e $\ell_2\in\RR$ sappiamo che $f(x)\to 0$ 
è equivalente a $\abs{f(x)}\to 0$ dunque in tal caso si può rimpiazzare $f(x)$ con $\abs{f(x)}\ge 0$
(gli eventuali punti in cui $f(x)=0$ sono irrilevanti e possono essere tolti 
dal dominio di $f$ per garantire $\abs{f}>0$) e ottenere che il limite 
del prodotto è $0$.
Lo stesso accade se $\ell_2=0$ con la funzione $g(x)$.

Per quanto riguarda il rapporto il teorema~\ref{th:estensione_additiva}
ci dice che le proprietà additive dell'opposto diventano (tramite logaritmo)
proprietà moltiplicative del reciproco, almeno se siamo nell'ambito di numeri positivi.
Dunque se $f(x)\to \ell$ con $\ell>0$ 
allora $\frac{1}{f(x)} \to \frac{1}{\ell}$.
Se $\ell=0$ non possiamo concludere niente in quanto le proprietà 
additive a $-\infty$ si rispecchiano nelle proprietà moltiplicative 
in $0^+$, non in $0$. 
Dunque i risultati sul limite del rapporto discendono dai risultati 
sul limite del prodotto con il reciproco.
\end{proof}


Se dobbiamo calcolare un limite in cui compare un elevamento 
a potenza $f(x)^{g(x)}$ con base ed esponente variabile converrà 
scrivere
\[
 f(x)^{g(x)} = a^{g(x)\cdot \log_a f(x)}
\]
per poter applicare il teorema precedente al prodotto $g(x)\cdot \log_a f(x)$.

\begin{example}
  Si calcoli
  \[
  \lim_{x\to +\infty} \frac{2+x^2-x^3}{(x^2-x+2)^2}.
  \]
  \end{example}
  \begin{proof}[Svolgimento.]
  Per $x\to +\infty$ per il teorema sul limite del
  prodotto sappiamo che $x^2\to +\infty$ e
  $x^3\to +\infty$.
  Per il teorema sulla somma dei limiti
  sappiamo che $2+x^2\to +\infty$ ma
  a priori non possiamo applicare tale teorema
  al limite di $(2+x^2)-x^3$ in quanto
  $(+\infty)-(+\infty)$ è una forma indeterminata
  (non rientra nelle ipotesi di quel teorema).
  Bisogna allora intuire che le potenze di $x$
  con esponente maggiore sono preponderanti e vanno
  quindi messe in evidenza tramite
  manipolazioni algebriche. L'espressione
  di cui vogliamo calcolare il limite
  si può quindi riscrivere in questo modo:
  \begin{align*}
  \frac{2+x^2-x^3}{(x^2-x+2)^2}
  &= \frac{x^3 \enclose{\frac 2 {x^3}+\frac{1}{x}-1}}{x^4\enclose{1-\frac 1 x + \frac{2}{x^2}}^2}
  = \frac{1}{x}\cdot \frac{\frac 2 {x^3}+\frac{1}{x}-1}{\enclose{1-\frac 1 x + \frac{2}{x^2}}^2}.
  \end{align*}
  A questo punto l'espressione non presenta più forme indeterminate.
  Applicando i teoremi precedenti possiamo allora affermare che
  si ha
  \begin{align*}
  \lim_{x\to+\infty}\frac{1}{x}\cdot \frac{\frac 2 {x^3}+\frac{1}{x}-1}{\enclose{1-\frac 1 x + \frac{2}{x^2}}^2}
  &= \frac{1}{+\infty} \cdot \frac{\frac 2 {(+\infty)^3}+\frac{1}{+\infty}-1}{\enclose{1-\frac 1 {+\infty} + \frac{2}{(+\infty)^2}}^2}\\
  &= 0 \cdot \frac{0+0-1}{(1-0+0)^2} = 0\cdot (-1) = 0.
  \end{align*}
\end{proof}

L'aver definito le operazioni sulle quantità infinite
risulta in effetti comodo nello svolgimento dei limiti.
Bisogna però essere consapevoli che $\bar \RR$ non è un campo
e quindi le operazioni con i simboli $+\infty$ e $-\infty$
non rispettano molte delle regole che siamo abituati
ad avere sui numeri finiti.
Sarà quindi opportuno ricondursi immediatamente ad una espressione
che abbia senso in $\RR$, sulla quale potremo
applicare le manipolazioni algebriche con più tranquillità.
Per questo motivo in molti testi l'espressione intermedia in cui compaiono 
le operazioni eseguite sulle quantità $+\infty$ e $-\infty$ non è ritenuta valida
e si preferisce scrivere direttamente il risultato.
  
\begin{example}
Si calcoli il seguente limite nel campo complesso:
\[
\lim_{z\to 0} \enclose{\frac 1 {z} - \frac 1 {z^2-z}}. 
\]
\end{example}
\begin{proof}
  Anche in questo caso non possiamo applicare direttamente le regole 
  di calcolo del limite in quanto otterremmo la forma indeterminata 
  $\infty - \infty$.
  Possiamo però operare una semplice manipolazione algebrica
  per eliminare l'indeterminazione:
  \[
    \frac 1 {z} - \frac 1 {z^2-z}
    = \frac{z- 1 - 1}{z^2-z}
    = \frac{z-2}{z^2-z}
    \to \frac{0-2}{0^2-0} = \frac{-2}{0} = \infty 
    \qquad \text{per $z\to 0.$}
  \]
\end{proof}

\begin{comment} % non abbiamo ancora gli strumenti per fare questi esercizi
\begin{exercise}
  Calcolare 
  \[
  \lim_{x\to 0^+} x^x.
  \]
\end{exercise}

\begin{exercise}
  Trovare un esempio di funzioni $f(x)$ e $g(x)$ tali che 
  \[
     \lim_{x\to 0} f(x) = 0, \qquad 
     \lim_{x\to 0} g(x) = 0
  \]
  ma 
  \[
    \lim_{x\to 0} f(x)^{g(x)} \neq 1.
  \]
\end{exercise}
\end{comment}

\section{proprietà frequenti e definitive}

\begin{definition}[proprietà frequenti e definitive]
Diremo che un predicato $P(x)$ definito su un insieme $A\subset \RR$ 
di cui $x_0$ è punto di accumulazione vale 
\myemph{definitivamente} per $x\to x_0$ se
vale in un intorno di $x_0$ ovvero:
\[
  \exists U\in \B_{x_0}\colon \forall x \in A\cap U\setminus\ENCLOSE{x_0}\colon P(x).
\]
Diremo che $P(x)$ vale \myemph{frequentemente} per $x\to x_0$
se in ogni intorno di $x_0$ c'è almeno un punto $x\neq x_0$ 
in cui vale:
\[
  \forall U\in \B_{x_0}\colon \exists x\in A\cap U\setminus\ENCLOSE{x_0}
  \colon P(x).
\]
\end{definition}

Chiaramente se una proprietà vale definitivamente vale anche frequentemente
infatti se c'è un intorno $U$ su cui vale la proprietà per ogni altro intorno 
$V$ la proprietà risulta valida su $U\cap V$ che non è mai vuoto.
Se una proprietà vale frequentemente significa in particolare che vale per 
infiniti valori diversi in quanto se vale in punto $x\neq x_0$ 
posso sempre trovare un intorno $V$ di $x_0$ che non contiene $x$
e in tale intorno trovo un ulteriore punto in cui vale la proprietà. 
Iterando il procedimento posso trovare infiniti punti diversi su cui 
la proprietà è valida.

Le due proprietà sono complementari nel senso che vale 
in base alle proprietà dei quantificatori universali vale 
la seguente relazione:
\[
  \text{non frequentemente $P(x)$} \iff
  \text{definitivamente non $P(x)$}
\]

Se due proprietà $P(x)$ e $Q(x)$ valgono definitivamente allora anche
$P(x)\land Q(x)$ vale definitivamente. Se invece valgono entrambe
frequentemente allora anche $P(x) \lor Q(x)$ vale frequentemente.

\begin{example}
La proprietà $x^3 - x > 1000 x^2 + 1$
vale definitivamente per $x\to +\infty$.
La proprietà $x>0$ vale frequentemente per $x\to 0$.
\end{example}


La definizione di limite $f(x) \to \ell$ per $x\to x_0$ 
potrebbe quindi enunciarsi così:
per ogni intorno $U$ di $\ell$ si ha $f(x)\in U$ definitivamente.
E la sua negazione è: esiste un intorno $U$ di $\ell$ per cui
frequentemente $f(x)\not\in U$.


\section{criteri di confronto}

\begin{theorem}[criteri di confronto]
\label{th:confronto}%
\index{teorema!del confronto}%
\index{limite!confronto}%
\index{teorema!dei due carabinieri}%
\mymark{***}%
Siano 
$f$, $g$, $h$ tre funzioni reali%
\footnote{Non è possibile fare confronti tra valori complessi, visto che
sui numeri complessi non abbiamo un ordinamento}
definite su uno stesso dominio $A\subset \RR$ 
con punto di accumulazione $x_0$
\mymargin{confronto tra limiti}
\begin{enumerate}
\item
Se per ogni $x\in A$ si ha
\[
f(x) \le g(x)
\]
e se entrambe le funzioni ammettono limite: $f(x) \to \ell_1$ 
e $g(x) \to \ell_2$ per $x\to x_0$
allora
\[
\ell_1 \le \ell_2.
\]

\item
Se per ogni $x\in A$ si ha:
\[
f(x) \le g(x)
\]
e se $f(x)\to +\infty$ allora anche $g(x) \to +\infty$ per $x\to x_0$.
Viceversa se $f(x) \le g(x)$ e $g(x) \to -\infty$ allora anche $f(x) \to -\infty$.

\item
(teorema dei carabinieri)
\mynote{teorema dei carabinieri}%
\index{teorema!dei carabinieri}%
Se per ogni $x\in A$ vale
\[
f(x) \le h(x) \le g(x)
\]
 e se le due
funzioni $f$ e $g$ hanno lo stesso limite: $f(x) \to \ell$ e $g(x)\to \ell$
per $x\to x_0$
allora anche $h(x) \to \ell$.
\end{enumerate}
\end{theorem}
%
\begin{proof}
\mymark{**}
\begin{enumerate}
\item
Se per assurdo fosse $\ell_1 > \ell_2$
la funzione $f(x)-g(x)$ avrebbe limite positivo e per il teorema
della permanenza del segno dovrebbe essere positiva in un intorno di 
$x_0$. Ma questo contraddice l'ipotesi $f(x)-g(x)\le 0$. 

\item Se non fosse $g(x)\to +\infty$ significa che 
esiste $\alpha \in \RR$ tale che $g(x)<\alpha$ frequentemente. 
Ma visto che definitivamente $f(x)>\alpha$ troviamo 
che frequentemente si ha $f(x)>g(x)$ in contrasto con l'ipotesi 
$f(x)\le g(x)$.

\item
Se $f$ e $g$ hanno lo stesso limite $\ell$ significa che per ogni
$U$ intorno di $\ell$ si ha definitivamente $f(x)\in U$ e $g(x)\in U$.
Visto che $U$ è un intervallo anche $g(x)$ è definitivamente in $U$ 
e quindi $g(x)\to \ell$.
\end{enumerate}
\end{proof}

\begin{example}
  Sapendo che $f(x)=x\to +\infty$ per $x\to +\infty$ dimostrare
  che anche la funzione $g(x) = \sqrt{x^3-3x+1}\to+\infty$.
  \end{example}
  %
  \begin{proof}
    Osserviamo che si ha 
    \[
    \sqrt{x^3-3x+1} 
    =\sqrt{x^3\enclose{1-\frac 3{x^2}+\frac 1 {x^3}}}
    = x\sqrt x\cdot \sqrt{1- \frac 3 {x^2}+\frac 1 {x^3}}  
    \]
e se $x\ge 4$ sarà quindi 
\[
  \sqrt{x^3-3x+1} 
    \ge x\sqrt 4 \cdot \sqrt{1-\frac 3{16}}
    \ge 2 \cdot \frac{13}{16} \cdot x \ge x \to +\infty
\] 
\end{proof}
  

