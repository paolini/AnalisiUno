\chapter{Listati}

Il seguente codice è scritto in \myemph{python 3},
un linguaggio di programmazione
molto semplice e pulito che permette, tra l'altro, di utilizzare diverse librerie utili per il calcolo numerico e scientifico.

\lstset{% general command to set parameter(s)
  language=python,
  basicstyle=\small\ttfamily, % print whole listing small
  keywordstyle=\color{black}\bfseries,
  % underlined bold black keywords
  identifierstyle=, % nothing happens
  commentstyle=\color{black!50}, % white comments
  stringstyle=\color{Maroon}\ttfamily, % typewriter type for strings
  showstringspaces=false} % no special string spaces

\section{bisection.py}

Vedi esempio~\ref{ex:75445}.
\myshortqrcode{bisection}{github}{bisection.py}
\label{code:bisection}
\lstinputlisting{code/bisection.py}

\section{series.py}

Vedi esempio~\ref{ex:52573}.
\myshortqrcode{series}{github}{series.py}
\label{code:series}
\lstinputlisting{code/series.py}

\section{compute\_e.py}

Vedi tabella~\ref{fig:cifre_e}.
\myshortqrcode{computee}{github}{compute_e.py}
\label{code:compute_e}
\lstinputlisting{code/compute_e.py}

\section{compute\_pi.py}

Vedi esercizio~\ref{ex:cifre_pi}.
\myshortqrcode{computepi}{github}{compute_pi.py}
\label{code:compute_pi}
\lstinputlisting{code/compute_pi.py}

\section{Mandelbrot.py}

Vedi figura~\ref{fig:mandelbrot}.
\myshortqrcode{Mandelbrot}{github}{Mandelbrot.py}
\label{code:Mandelbrot}
\lstinputlisting{code/Mandelbrot.py}

\section{Koch.py}

Vedi figura~\ref{fig:koch}.
\myshortqrcode{Koch}{github}{Koch.py}
\label{code:Koch}
\lstinputlisting{code/Koch.py}

\section{Fourier.py}

Vedi figura~\ref{fig:fourier}
\myshortqrcode{Fourier}{github}{Fourier.py}
\label{code:Fourier}
\lstinputlisting{code/fourier.py}
