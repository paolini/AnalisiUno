\section{funzioni continue}
\label{sec:continuita}

Intuitivamente una funzione continua ha la proprietà
che se in un punto $x_0$ assume un valore $y_0=f(x_0)$ allora
in punti abbastanza vicini ad $x_0$ i valori assunti
non saranno molto diversi dal valore $y_0$.
Nella definizione seguente questo viene formalizzato:
fissato il punto $x_0$ e scelto un errore $\eps>0$
che siamo disposti a commettere sui valori della funzione
possiamo trovare un errore $\delta>0$ per cui nei punti
che differiscono da $x_0$ per meno di $\delta$ il valore
della funzione differisce da $y_0$ per meno di $\eps$.

Se $x$ e $y$ sono due punti di $\RR$ la loro distanza 
è data da $\abs{x-y}$. 
Una disuguaglianza del tipo $\abs{x-y}<r$ significa 
che il punto $x$ si trova nell'intervallo $(x-r,x+r)$.
Nel seguito ci concentreremo sulle funzioni con dominio 
e codominio nei numeri reali. 
Ma potremo osservare che molte delle definizioni e molti 
dei teoremi possono essere enunciati e dimostrati in maniera 
identica anche per le funzioni con dominio e codominio 
nei numeri complessi%
\footnote{Ci sono in realtà quattro casi rilevanti:
dominio e codominio reale, dominio e codominio complesso, 
dominio reale e codominio complesso, 
dominio complesso e codominio reale. 
Per dire che il dominio è sottoinsieme dei numeri complessi 
diremo che la funzione $f$ è \emph{di variabile complessa},
mentre per dire che la funzione $f$ ha codominio nei numeri 
complessi diremo che $f$ è \emph{a valori complessi}
o più semplicemente che $f$ è complessa.
Il caso di funzioni complesse di variabile complessa 
è ovviamente il caso più generale. 
Ma a volte può essere utile restringersi alla variabile 
reale dove abbiamo la struttura lineare data dall'ordinamento.
Sempre per mantenere l'ordinamento il caso reale e complesso 
si distinguono anche per i punti all'infinito: nel caso reale 
ne introduciamo due, $+\infty$ e $-\infty$ mentre nel caso 
complesso si introduce un unico punto $\infty$.
} 
Ad una prima lettura si consiglia di trascurare 
il caso complesso.
Anche quando $x$ e $y$ sono punti di $\CC$ il modulo della 
differenza $\abs{x-y}$ rappresenta la distanza tra i due 
punti $x$ e $y$. 
Fissato $y\in \CC$ e $r>0$
\mynote{La condizione $r>0$ non ha senso se $r\in \CC$ 
quindi se si impone questa condizione è sottointeso 
che $r\in \RR$}
la condizione $\abs{x-y}<r$ identifica i punti $x$ 
che si trovano all'interno di un cerchio di raggio $r$
intorno a $y$.

\begin{definition}[funzione continua]
  \label{def:continua}%
  \index{funzione!continua}%
  \index{continuità}%
  Sia $f\colon A \subset \RR \to \RR$ 
  (oppure $f\colon A\subset \CC \to \CC$) una funzione. 
  Diremo che $f$ è \emph{continua nel punto}%
\mymargin{continua nel punto}%
\index{continuo!nel punto} $x_0\in A$ se
  \begin{equation}\label{eq:continuita}
  \forall \eps>0 \colon \exists \delta>0 \colon
  \forall x\in A\colon
  \abs{x-x_0} < \delta \implies \abs{f(x)-f(x_0)} < \eps.
  \end{equation}
  
  Diremo che $f$ è \emph{continua}%
\mymargin{continua}%
\index{continuo} se è continua in ogni punto $x_0 \in A$.
  \end{definition}
  
  %%%%%%%%%%%%%%%%%%%
  %%%%%%%%%%%%%%%%%%%
  %%%%%%%%%%%%%%%%%%%
  
  Attenzione: la funzione $f(x)=\frac{1}{x}$
  assume vicino a $x=0$ valori molto diversi tra loro (ad esempio $f(0.01)-f(-0.01)=200$)
  ma ciò non toglie che la funzione possa essere continua in quanto il punto
  $x=0$ non appartiene al dominio e quindi,
  in base alla definizione precedente, non ha senso e non ha importanza
  verificare se la funzione è continua in tale punto.
  
  \begin{theorem}[continuità del reciproco]
  \label{th:cont_reciproco}%
  La funzione $f\colon \RR\setminus\ENCLOSE{0}\to \RR$ definita
  da $f(x)=\frac 1 x$ è una funzione continua.
  \end{theorem}
  %
  \begin{proof}
  Siano $x,x_0\neq 0$.
  Se prendiamo $\delta < \frac{\abs{x_0}}2$
  e se $\abs{x-x_0} < \delta$ si avrà,
  per disuguaglianza triangolare inversa,
  $\abs{x} > \frac{\abs{x_0}}2$. Dunque
  \[
  \abs{\frac 1 x - \frac 1 {x_0}}
  = \frac{\abs{x-x_0}}{\abs{x\cdot x_0}}
  \le \frac{2\delta}{\abs{x_0}^2}.
  \]
  Si ottiene quindi la condizione $\abs{f(x)-f(x_0)}<\eps$
  se si sceglie $\delta$ in modo che
  risulti anche $\delta < \frac{\abs{x_0}^2\cdot \eps}{2}$.
  \end{proof}
  
  \begin{example}
  La \emph{funzione segno}%
\mymargin{funzione segno}%
\index{funzione!segno} $\sgn\colon \RR \to \RR$ definita da
  \[
    \sgn(x) = \begin{cases}
    1 & \text{se $x> 0$}\\
    0 & \text{se $x=0$}\\
    -1 & \text{se $x<0$}
    \end{cases}
  \]
  è un esempio di funzione non continua.
  \end{example}
  %
  \begin{proof}
  Verifichiamo che la funzione non è continua
  nel punto $x_0=0$.
  Infatti se $x\neq 0$ si ha
  \[
  \abs{\sgn(x)-\sgn(x_0)} = \abs{\sgn(x)} = 1
  \]
  e quindi se scegliamo $\eps<1$ non è possibile
  trovare $\delta>0$ per cui valga la condizione
  di continuità~\eqref{eq:continuita}
  nel punto $x_0=0$.
  \end{proof}
  
  \begin{exercise}
  Dimostrare che le funzioni $f(x) = x$ e $g(x)=\abs{x}$ sono continue.
  Dimostrare che le funzioni $h(x) = \lfloor x\rfloor$ e $k(x)=\lceil x \rceil$
  non sono continue (ma sono continue in ogni punto
  $x_0\in \RR \setminus \ZZ$).
  \end{exercise}
  
  \begin{definition}[operazioni sulle funzioni]
  Sia $A \subset \RR$ (oppure $A\subset \CC$) 
  e siano $f,g$ funzioni $A \to \RR$ (oppure $A\to \CC$).
  Possiamo allora definire
  $f+g$, $-f$, $f-g$, $f\cdot g$, $\abs f$, $f^n$ (con $n\in \NN$)
  e (se $g(x)\neq 0$ per ogni $x\in A$) anche $f/g$
  come funzioni $A \to \RR$ (o $A\to \CC$) mediante le seguenti ovvie
  definizioni
  \begin{gather*}
  (f+g)(x) = f(x) + g(x), \qquad
  (f-g)(x) = f(x) - g(x), \\
  (f\cdot g)(x) = f(x) \cdot g(x), \qquad
  (f/g)(x) = f(x) / g(x), \\
  (-f)(x) = -(f(x)), \qquad
  \abs f(x) = \abs{f(x)}, \qquad
  f^n(x) = (f(x))^n.
  \end{gather*}
  
  Se $f\colon A \to B$ e $g\colon B\to C$ ricordiamo
  che abbiamo definito la funzione composta
  $g\circ f\colon A \to C$:
  \[
    (g\circ f)(x) = g(f(x)).
  \]
  
  Se $c\in \RR$ (o $c\in \CC$) è un numero considereremo a volte 
  $c\colon A \to \RR$ come una funzione \emph{costante}.
  Risulta quindi inteso che $c\cdot f$ è la funzione definita da
  $(c\cdot f)(x) = c\cdot (f(x))$.
  La somma e il prodotto per costante rendono l'insieme $\RR^A$
  delle funzioni $A \to \RR$ uno spazio vettoriale sul campo $\RR$
  (lo stesso vale per $\CC^A$ con $A\subset \CC$)
  \end{definition}
  
  \begin{theorem}[composizione di funzioni continue]
  \label{th:continuita_composizione}%
  Se $f$ e $g$ sono funzioni (reali o complesse) definite e continue
  in uno stesso punto $x_0$ (reale o complesso)
  allora anche
  \[
    f+g, \qquad
    f\cdot g, \qquad
    f-g, \qquad
    \abs{f}, \qquad
    f^n\quad \text{(con $n\in \NN$)}
  \]
  sono funzioni definite e continue nel punto $x_0$.
  Se inoltre $g(x_0)\neq 0$ anche la funzione
  $f/g$
  è definita e continua nel punto $x_0$.
  
  Se $f\colon A\subset \RR \to \RR$ è una funzione continua
  nel punto $x_0\in A$ e
  $g\colon B\subset \RR \to \RR$ è una funzione
  continua nel punto $y_0=f(x_0)\in B$ allora la funzione $g\circ f$
  definita in $f^{-1}(B)$ è continua nel punto $x_0$
  (lo stesso vale sostituendo $\CC$ ad $\RR$)
  \end{theorem}
  %
  \begin{proof}
  Mostriamo prima di tutto la continuità
  della composizione $g\circ f$.
  Per la continuità di $f$ in $x_0$ e di $g$ in $y_0=f(x_0)$
  si ha che per ogni $\eps>0$ esiste un $\gamma>0$
  e per ogni $\gamma>0$ esiste un $\delta>0$ per cui
  \begin{align*}
   \abs{x-x_0}< \delta &\implies \abs{f(x)-f(x_0)}<\gamma,\\
   \abs{y-y_0}< \gamma & \implies \abs{g(y)-g(y_0)}<\eps
  \end{align*}
  da cui
  \[
  \abs{x-x_0}< \delta
  \implies \abs{f(x)-f(x_0)}< \gamma
  \implies \abs{g(f(x))-g(f(x_0))} < \eps
  \]
  che non è altro che la condizione di
  continuità~\eqref{eq:continuita} per $g\circ f$.
  
  Se $f$ e $g$ sono continue nel punto $x_0$
  allora per ogni $\eps'>0$ esistono $\delta_1$
  e $\delta_2$ tali che
  \begin{align*}
   \abs{x-x_0} < \delta_1 &\implies \abs{f(x)-f(x_0)} < \eps',
   \\
   \abs{x-x_0} < \delta_2 &\implies \abs{g(x)-g(x_0)} < \eps'.
  \end{align*}
  In particolare scegliendo $\delta = \min\ENCLOSE{\delta_1,\delta_2}$
  se $\abs{x-x_0} < \delta$ valgono contemporaneamente
  entrambe le stime:
  \[
   \abs{f(x)-f(x_0)}< \eps', \qquad
   \abs{g(x)-g(x_0)}< \eps'.
  \]
  
  Dunque per la somma osserviamo che se $\abs{x-x_0}<\delta$
  si ha
  \begin{align*}
   \abs{(f+g)(x) - (f+g)(x_0)}
    &= \abs{f(x)+g(x)-f(x_0)-g(x_0)}\\
    &\le \abs{f(x)-f(x_0)} + \abs{g(x)-g(x_0)}
    \le 2 \eps'.
  \end{align*}
  Dato $\eps>0$ basterà quindi scegliere $\eps' = \eps / 2$
  e $\delta$ come sopra per ottenere la condizione di continuità.
  
  Per il prodotto si ha
  \begin{align*}
    \MoveEqLeft \abs{f(x)g(x)-f(x_0)g(x_0)}\\
    &= \abs{f(x)g(x) - f(x_0)g(x) + f(x_0)g(x) - f(x_0)g(x_0)}\\
    &\le \abs{f(x)-f(x_0)}\cdot\abs{g(x)} + \abs{f(x_0)}\cdot \abs{g(x)-g(x_0)} \\
    &\le \eps' \abs{g(x)} + \abs{f(x_0)} \eps'.
  \end{align*}
  Osserviamo ora che $\abs{g(x)}\le \abs{g(x)-g(x_0)} + \abs{g(x_0)}$
  e quindi $\abs{g(x)}\le \abs{g(x_0)}+\eps'$ da cui:
  \begin{align*}
    \abs{f(x)g(x)-f(x_0)g(x_0)}
    &\le \eps' \enclose{\abs{g(x_0)}+\eps'} + \abs{f(x_0)} \eps' \\
    &= \eps' \cdot(\abs{f(x_0)} + \abs{g(x_0)} + \eps').
  \end{align*}
  Possiamo facilmente rendere questa quantità inferiore a
  qualunque $\eps>0$: basterà prendere $\eps'<1$ e
  \[
    \eps' < \frac{\eps}{\abs{f(x_0)} + \abs{g(x_0)} + 1}.
  \]
  
  La funzione $f^n$ è continua per induzione su $n$
  in quanto prodotto di funzioni
  continue: $f^{n+1} = f^{n} \cdot f$.
  
  Abbiamo già visto che la funzione $h(x) = \frac{1}{x}$ è continua,
  dunque se $g$ è continua anche la funzione $\frac{1}{g(x)} = h\circ g$
  è continua essendo composizione di funzioni continue. 
  Di conseguenza anche la funzione $\frac{f}{g} = f \cdot \frac{1}{g}$
  è continua, essendo il prodotto di funzioni continue.
  
  Analogamente la funzione $f-g$ è la somma di $f$ con $-g$ e
  la funzione $-g$ è la composizione di $g$ con $h(x)=-x$.
  E' immediato verificare che la funzione $h(x)=-x$ è continua
  e dunque anche la differenza $f-g$ è continua.
  Lo stesso vale per la funzione $\abs f$ che è la composizione
  di $f$ con la funzione $h(x) = \abs{x}$.
  \end{proof}
  
  Il precedente teorema è molto importante ed utile in quanto
  garantisce che ogni funzione definita tramite una espressione
  che coinvolge solamente le funzioni e le operazioni
  elencate nell'enunciato del teorema, risulta certamente
  essere una funzione continua. Come nel seguente.
  
  \begin{example}
  La funzione
  \[
  f(x) = \frac{(x-3)\cdot x -\frac{1}{x+x^2}}{\abs{x-\frac{1-x^3}{x}}}
  \]
  è continua.
  
  Si intende che tale funzione è definita sull'insieme degli $x\in \RR$
  per cui tutte le operazioni coinvolte sono definite ovvero
  $f\colon D \subset \RR \to \RR$
  con
  \[
    D = \ENCLOSE{x\in \RR \colon x+x^2 \neq 0,\ x\neq 0,\ \abs{x-\frac{1-x^3}{x}}\neq 0}.
  \]
  \end{example}
  \begin{proof}
  Per convincersi che questa funzione $f$ è continua
  si nota che le funzioni $x$ e le costanti $3$ e $1$ sono
  funzioni continue.
  Ma allora, per il teorema~\ref{th:continuita_composizione},
  anche le funzioni $x-3$ e $x^2=x\cdot x$ sono continue.
  Dunque anche $(x-3)\cdot x$, $x+x^2$ e $x^3$ e $1-x^3$ sono continue.
  Di conseguenza sono continue pure $\frac 1{1+x^2}$ e $\frac{1-x^3}{x}$.
  E poi saranno continue anche $(x-3)\cdot x - \frac 1{1+x^2}$ e $x-\frac{1-x^3}{x}$
  e quindi $x-\frac{1-x^3}{x}$ e pure $\abs{x-\frac{1-x^3}{x}}$. Infine sarà
  dunque continua $f(x)$.
  \end{proof}
  
  In particolare è chiaro che le funzioni lineari e quadratiche 
  che abbiamo introdotto nelle sezioni precedenti sono funzioni continue
  in quanto sono ottenute sommando e moltiplicando tra loro funzioni continue.
  
  %%%%%%%%%%%%%%%%%%%
  
  \begin{theorem}[continuità delle funzioni monotone]
    \label{th:monotona_continua}%
    \mymark{*}%
    \index{funzione!continua}%
    \index{funzione!monotòna}%
    \index{funzione!suriettiva}%
    Sia $A\subset \RR$ e sia
    $f\colon I \to \RR$ una funzione monotòna.
    Se $f(I)$ è un intervallo allora $f$ è continua.
%    \label{th:monotona_continua_reverse}%
%    Allora $f$ è continua se e solo se $f(I)$ è 
%    un intervallo.
\end{theorem}
  %
  \begin{proof}
  Senza perdita di generalità possiamo supporre che $f$ 
  sia crescente. 
  % Dimostriamo innanzitutto che se $f(I)$ è intervallo allora $f$ è continua.
  
  Prendiamo un punto $x_0\in I$ e sia dato $\eps>0$.
  Scegliamo qualunque $\eps'$ con $0<\eps'<\eps$.
  Vogliamo trovare $x_1<x_0$ tale che  
  per ogni $x\in [x_1,x_0]\cap I$ si abbia $f(x)\ge f(x_0)-\eps'$.
  Siccome $f(I)$ è un intervallo che contiene il punto $f(x_0)$ 
  ci sono due possibilità: o $f(x)\ge f(x_0)-\eps'$ per ogni $x\in I$
  e quindi possiamo scegliere $x_1<x_0$ a piacere
  oppure esiste $x_1\in I$ tale che $f(x_1)=f(x_0) - \eps'$.
  In questo secondo caso dovrà essere $x_1<x_0$ (in quanto $f$ è crescente)
  e per monotonia si avrà, come voluto $f(x)\ge f(x_1) \ge f(x_0)-\eps'$ 
  per ogni $x\in [x_1,x_0]$.
  
  In modo analogo possiamo trovare $x_2>x_0$ tale 
  che per ogni $x\in [x_0,x_2]\cap I$ si abbia $f(x) \le f(x_0)+\eps'$.
  Essendo $f$ crescente, se $x\ge x_0$ si avrà anche $f(x)\ge f(x_0)$ 
  e se $x\le x_0$ si avrà $f(x) \le f(x_0)$. 
  Dunque per ogni $x\in [x_1,x_2]$ si avrà $\abs{f(x)-f(x_0)}\le\eps'<\eps$.
  Basterà scegliere $\delta = \min\ENCLOSE{x_0-x_1,x_2-x_0}$ 
  per ottenere la continuità di $f$ in $x_0$:
  \[
  \forall x\in A\colon \abs{x-x_0}<\delta \implies \abs{f(x)-f(x_0)}<\eps.
  \]
\end{proof}
  %

Il teorema precedente (teorema~\ref{th:monotona_continua}) 
ci permette di affermare che per $a>0$, $a\neq 1$ 
le funzioni $\exp_a$ e $\log_a$ sono continue. 
Infatti tali funzioni sono bigezioni monotone tra 
gli intervalli $\RR$ e $\RR_+$.

Anche le funzioni trigonometriche $\sin$ e $\cos$ sono funzioni continue.
Sappiamo infatti che $\sin \colon [-\tau/4,\tau /4]\to [-1,1]$
e $\cos\colon [0,\tau/2]\to [-1,1]$ sono strettamente monotone 
e bigettive dunque sono continue su tali intervalli. 
Grazie alle proprietà di simmetria e periodicità
(oppure utilizzando le formule di addizione) è facile verificare 
che tali funzioni sono continue su tutto $\RR$.

Grazie al teorema~\ref{th:continuita_composizione} sappiamo 
che se $n\in \NN$ la funzione $x^n$ è continua su tutto il suo 
dominio $\RR$. 
La funzione inversa $\sqrt[n]{x}$ è anch'essa crescente e 
bigettiva come funzione $[0,+\infty)\to[0,+\infty)$ e dunque 
anch'essa risulta essere continua.
Per simmetria (si veda l'esercizio~\ref{ex:simmetrica_continua}) 
possiamo concludere 
%che sia la potenza $x^n$ 
che le radici $\sqrt[n]{x}$ sono 
funzioni continue su tutto il loro dominio.
Anche la funzione $f(x) = x^\alpha$, $f\colon [0,+\infty)\to [0,+\infty]$ 
con $\alpha>0$ è crescente ed è invertibile (l'inversa è $x^{\frac 1 \alpha}$)
e dunque è surgettiva e continua.
Se $\alpha<0$ la funzione $f(x)=x^\alpha$ è definita sull'intervallo 
$(0,+\infty)$ ed è anch'essa continua in quanto 
composizione di funzioni continue: $x^{\alpha} = \frac{1}{x^{-\alpha}}$.

%
\begin{exercise}\label{ex:simmetrica_continua}
  Sia $f\colon \RR\to \RR$ una funzione pari oppure dispari.
  Se la restrizione di $f$ all'intervallo $[0,+\infty)$ 
  è continua allora $f$ è continua su tutto $\RR$.
\end{exercise}

\subsection{limite}

Se una funzione $f$ non è definita in un punto $x_0$ non ha senso chiedersi
se in tale punto è continua. 
Possiamo però chiederci se è possibile definire la funzione anche nel punto 
$x_0$ dando un valore opportuno $\ell$ in modo da rendere $f$ continua 
in quel punto. 
Se ciò è possibile diremo che la funzione $f(x)$ ha limite $\ell$ 
per $x$ che tende a $x_0$ e scriveremo:
\mynote{Si legge: ``$f(x)$ tende ad $\ell$ per $x$ che tende a $x_0$''.}
\[
  f(x) \to \ell \qquad \text{per $x\to x_0$}
\]
(daremo tra poco una definizione con maggiore precisione e generalità).

Ad esempio la funzione $f(x) = \frac{x^2-1}{x-1}$ è una funzione definita per $x\neq 1$
e coincide, se $x\neq 1$ con la funzione lineare $\tilde f(x) = x+1$ definita 
su tutto $\RR$. 
Visto che $\tilde f$ è continua e $\tilde f(1)=2$
scriveremo:
\[
  \frac{x^2-1}{x-1} \to 2
  \qquad \text{per $x\to 1$.}
\]

Se estendiamo $f$ con un valore $\ell$ nel punto $x_0$ otteniamo 
in generale la funzione 
\[
  \tilde f(x) = \begin{cases}
    f(x) & \text{se $x\neq x_0$}\\
    \ell & \text{se $x=x_0$}
  \end{cases}  
\]
e la continuità di $\tilde f$ nel punto $x_0$ si scrive così:
\[
\forall\eps>0\colon \exists \delta>0\colon
\abs{x-x_0}<\delta \implies \abs{\tilde f(x)-\tilde f(x_0)}<\eps.  
\]
Osserviamo che se $x=x_0$ allora ovviamente $\abs{\tilde f(x)-\tilde f(x_0)}
= 0 < \eps$ dunque possiamo supporre, nella condizione precedente, 
che sia $x\neq x_0$ e dunque $\tilde f(x)=f(x)$. 
Inoltre visto che $\tilde f(x_0)=\ell$ 
e $\tilde f(x)=f(x)$ se $x\neq x_0$,
si ottiene 
la seguente condizione:
\begin{equation}\label{eq:55338}
\forall \eps>0\colon \exists \delta >0 \colon 
  \forall x\neq x_0\colon
  \abs{x-x_0}<\delta \implies \abs{f(x) - \ell} < \eps.
\end{equation}
La~\eqref{eq:55338} potrebbe dunque essere utilizzata per definire 
il concetto di limite $f(x)\to \ell$ per $x\to x_0$.
Sarà però molto utile estendere il concetto di limite $f(x)\to \ell$ 
per $x\to x_0$ anche nei casi in cui $\ell$ e/o $x_0$ possano 
essere infiniti (cioè elementi dei reali estesi $\bar \RR$).

Per fare ciò osserviamo che
se definiamo%
\mynote{%
la lettera $B$ sta per \emph{ball} in quanto 
più in generale (se fossimo in $\RR^3$ invece che in $\RR$)
l'insieme dei punti che distano meno di $\rho$ da un punto fissato 
è l'interno di una sfera piena. 
In geometria una sfera piena si chiama \emph{palla} 
se contiene solo i punti interni (e non la superficie sferica)
e si chiama \emph{disco} o \emph{palla chiusa} se contiene 
anche i punti della superficie.
}
$B_\rho(x_0) = \ENCLOSE{x\in \RR \colon \abs{x-x_0}<\rho}$
la condizione \eqref{eq:55338}
può essere scritta anche nella forma 
\[
  \forall \eps>0\colon \exists \delta >0 \colon 
  x \in B_\delta(x_0)\setminus\ENCLOSE{x_0} \implies f(x) \in B_\eps(\ell)  
\]
che a sua volta si può scrivere nella forma:
\[
  \forall \eps>0\colon \exists \delta >0 \colon 
  f(B_\delta(x_0)\setminus\ENCLOSE{x_0}) \subset  B_\eps(\ell).    
\]

L'idea è che gli insiemi del tipo
$B_\rho(x_0)$ 
rappresentano i punti \emph{vicini} al punto $x_0$. 
In analogia potremmo pensare che i punti \emph{vicini} 
a $+\infty$ siano i punti di una qualunque semiretta 
del tipo $(M,+\infty]$.
Si dà quindi la seguente.


\begin{definition}[intorno]
Per $x\in \RR$ definiamo la famiglia degli \emph{intorni}%
\mymargin{intorni}%
\index{intorno} (basilari) di $x$
come l'insieme di tutti gli intervallini aperti, simmetrici, centrati in $x$:
\[
  \B_x = \ENCLOSE{ (x-\eps, x+\eps) \colon \eps>0}.
\]
Definiamo poi le famiglie 
degli \emph{intorni destri} e \emph{intorni sinistri}
\mymargin{intorni destri/sinistri}%
\index{intorni!destri/sinistri}
\index{intorni}
di $x$ come
\[
  \B_{x^+} = \ENCLOSE{ [x, x+\eps) \colon \eps>0},
  \qquad
  \B_{x^-} = \ENCLOSE{ (x-\eps , x] \colon \eps>0}
\]
e le famiglie degli intorni di $+\infty$ e $-\infty$ come segue
\[
  \B_{+\infty} = \ENCLOSE{ (\alpha,+\infty], \colon \alpha \in \RR },\qquad
  \B_{-\infty} = \ENCLOSE{ [-\infty, -\beta), \colon \beta\in \RR}.
\]

Per ogni $x\in \bar \RR = [-\infty, +\infty]$
risultano quindi definiti gli intorni $\B_x$ e per
ogni $x\in \RR$ sono definiti gli intorni $\B_{x^+}$ e $\B_{x^-}$.

Più in generale se $x\in \CC$ o $x\in \RR^n$
si definiscono gli intorni di $x$ 
utilizzando la norma $\abs{\cdot}$.
Posto $B_\rho(x) = \ENCLOSE{y\colon \abs{y-x}<\rho}$
(palla di raggio $\rho$ centrata in $x$)
la famiglia di intorni di $x$ non è altro che
\[
  \B_x = \ENCLOSE{B_\rho(x)\colon \rho>0}.
\]
Nel caso $x\in \RR$ si riottiene la stessa definizione che 
abbiamo dato sopra.

In $\CC$ o in $\RR^n$ non c'è un ordinamento quindi 
tipicamente si aggiunge un solo punto all'infinito: $\infty$.
Gli intorni di $\infty$
sono dati dalla parte esterna di una palla:
\[
  \B_\infty = \ENCLOSE{\ENCLOSE{y\colon \abs{y}> R}\colon R>0}.
\]

Anche su $\RR$ si può considerare un unico punto 
all'infinito (che potremmo denotare con $\infty$)
invece che i due punti $+\infty$ e $-\infty$. 
Gli intorni di $\infty$ in $\RR$ saranno quindi della forma $[-\infty,-R) \cup (R,+\infty]$.
\end{definition}

\mynote{Per definire il concetto di limite abbiamo definito il concetto di intorno
\emph{basilare}. Formalmente un intorno (generico) di un punto $x_0$ è un qualunque insieme che 
contiene un intorno basilare di $x_0$. 
Nelle definizioni che useremo noi sarà sufficiente considerare 
gli intorni basilari, ma è bene sapere che la definizione
di intorno comprende una famiglia più grande di insiemi.
In generale una famiglia di insiemi $\B$ è detta una 
\emph{base di intorni} di $x_0$ se soddisfa le seguenti proprietà:
(1) ogni elemento di $\B$ contiene a sua volta $x$ come elemento,
(2) l'intersezione di due elementi di $\B$ contiene un elemento di $\B$.
}

\begin{definition}[funzione continua (con gli intorni)]
  \label{def:continua_intorni}
  Sia $f\colon A\to B$ con $A$ e $B$ sottoinsiemi di $\bar \RR$, di $\bar \CC$ 
  o di $\RR^n$. 
  Diremo che $f$ è continua nel punto $x_0$ se vale la seguente proprietà:
  \[
    \forall V \in \B_{f(x_0)} \colon \exists U \in \B_{x_0} \colon f(U) \subset V.
  \]

  E' chiaro, dalla discussione precedente, che questa definizione 
  estende la Definizione~\ref{def:continua}.
\end{definition}

Possiamo allora esplicitare la definizione di limite per funzioni reali 
di variabile reale. 
Ma la stessa identica definizione può essere data per funzioni definite 
tra gli spazi su cui abbiamo definito il concetto di \emph{intorno basilare}
e cioè $\bar \RR$, $\bar \CC$ e $\RR^n$.

\begin{definition}[limite di funzione]
\mymark{***}
Sia $A\subset \RR$ e $f\colon A \to \RR$.
Sia $x_0\in [-\infty,+\infty]$
% un punto di accumulazione di $A$ 
e sia $\ell \in [-\infty,+\infty]$.
Allora diremo che la funzione $f$ ha limite $\ell$ per $x$ che tende a $x_0$ 
e scriveremo\mymargin{limite di funzione}%
\index{limite!di funzione}
\[
  f(x) \to \ell \qquad \text{per $x\to x_0$}
\]
se per ogni intorno di $\ell$ esiste un intorno di $x_0$ tale che
la funzione valutata nell'intorno di $x_0$, tolto eventualmente $x_0$,
assume valori
nell'intorno di $\ell$:
\begin{equation}\label{eq:def_limite}
  \forall V \in \B_\ell \colon \exists U \in \B_{x_0} \colon f(U\setminus\ENCLOSE{x_0}) \subset V.
\end{equation}

La stessa definizione può essere data restringendosi agli intorni destri/sinistri del punto $x_0$ (nel caso $x_0 \in \RR$). Si otterranno quindi le definizioni
di \emph{limite destro} e \emph{limite sinistro}
\mymargin{limite destro/sinistro}%
\index{limite!destro/sinistro} %
semplicemente sostituendo $\B_{x_0^+}$ o $\B_{x_0^-}$ al posto di 
$\B_{x_0}$ nella definizione
precedente. 
In tal caso scriveremo $f(x)\to \ell$ per $x\to x_0^+$ per il limite 
destro e $f(x)\to \ell$ per $x\to x_0^-$ per il limite sinistro.

Infine anche il risultato del limite può essere $\ell^+$ o $\ell^-$, 
in tal caso useremo $\B_{\ell^+}$ o $\B_{\ell^-}$ 
al posto di $\B_{\ell}$.

Anche per le funzioni complesse $f\colon A\subset \CC \to \CC$ 
si applica la stessa identica definizione.
\end{definition}
  
\begin{example}
Si consideri la funzione segno:
\[
\sgn(x) =
\begin{cases}
  1 & \text{se $x>0$},\\
  0 & \text{se $x=0$},\\
  -1 & \text{se $x<0$}.
\end{cases}
\]
Si può verificare che
\[
\sgn(x) \to 1 \qquad \text{per $x\to 0^+$}
\]
e
\[
\sgn(x) \to -1 \qquad \text{per $x\to 0^-$}
\]
\end{example}

Abbiamo già visto che nel caso in cui $x_0\in \RR$ e $\ell\in \RR$ 
siano entrambi finiti 
la definizione di limite $f(x)\to \ell$ per $x\to x_0$
si traduce nella condizione~\eqref{eq:55338}.

Anche negli altri casi esplicitando le definizioni di intorno
si possono ottenere delle condizioni più esplicite.
Ad esempio la condizione $f(x)\to -\infty$ per $x\to x_0^+$
si traduce nel modo seguente:
\[
\forall \beta\in \RR\colon \exists \delta>0 \colon x_0 < x < x_0+\delta 
\implies f(x) < -\beta.  
\]
Visto che ci sono cinque diversi casi per il punto $x_0$:
$x_0\in \RR$, $x_0^+$, $x_0^-$, $+\infty$, $-\infty$ e altrettanti 
casi per $\ell$ (in effetti anche il risultato del limite 
può essere \emph{destro} o \emph{sinistro}), si ottengono 
in tutto 25 diverse definizioni di limite.

In tutte queste definizioni di limite è sottointeso che i punti 
$x$ che vengono presi in considerazione sono punti 
del dominio di $f\colon A \subset \RR \to \RR$.
Accade allora che se esiste un intorno $V\in \B_{x_0}$
per cui $A \cap V \setminus \ENCLOSE{x_0}$ è vuoto allora 
la condizione di limite è vuota ed è quindi sempre verificata 
qualunque sia il valore di $\ell$. 
E' quindi inutile fare il limite per $x\to x_0$ 
se $x_0$ non soddisfa la seguente.

\begin{definition}[punto di accumulazione]
  \mymark{*}
  Siano $A\subset  \RR$ un insieme e $x_0\in [-\infty, +\infty]$.
  Diremo che $x_0$ è un \emph{punto di accumulazione}%
\mymargin{punto di accumulazione}%
\index{punto!di accumulazione} di $A$
  se ogni intorno di $x_0$ contiene punti di $A$ diversi da $x_0$, ovvero:
  \[
   \forall U \in \B_{x_0}\colon (A\setminus \ENCLOSE{x_0}) \cap U \neq \emptyset.
  \]

  Stessa definizione si può dare per $x_0^+$ e $x_0^-$ 
  utilizzando gli intorni destri/sinistri di $x_0$. 

  Sempre la stessa definizione si può applicare 
  quando $A\subset \CC$ e $x_0\in \bar \CC$.
\end{definition}

\begin{theorem}[unicità del limite]
\mymark{*}
Sia $A\subset \RR$, $f\colon A \to \RR$, $x_0$
punto di accumulazione per $A$ e $\ell_1, \ell_2\in [-\infty,+\infty]$.
Se per $x\to x_0$ si ha
\[
  f(x) \to \ell_1 \qquad\text{e}\qquad f(x) \to \ell_2
\]
allora $\ell_1=\ell_2$.

Risultato analogo si ha per i limiti destro e sinistro: $x\to x_0^+$, 
$x\to x_0^-$.
Anche su $\CC$ vale lo stesso risultato
prendendo $\ell_1,\ell_2\in \bar \CC$.
\end{theorem}
%
\begin{proof}
\mymark{*}
Supponiamo per assurdo che $\ell_1\neq \ell_2$.
Allora esiste un intorno $V_1$ di $\ell_1$ ed un intorno $V_2$ di $\ell_2$
tali che $V_1\cap V_2 = \emptyset$ (basta prendere degli intorni abbastanza piccoli). 
Ma per le definizioni di limite $f(x)\to \ell_1$ e $f(x)\to \ell_2$ 
dovranno esistere $U_1$ e $U_2$ intorni di $x_0$ su cui si ha 
$f(U_1)\subset V_1$ e $f(U_2)\subset V_2$. 
Ma allora $f((A\setminus\ENCLOSE{x_0})\cap U_1)\cap f((A\setminus\ENCLOSE{x_0})\cap U_2)\subset V_1\cap V_2 = \emptyset$... 
e questo è assurdo perché certamente $U_1\cap U_2$ 
contiene punti di $A$ diversi da $x_0$ in quanto 
$U_1$ e $U_2$ sono uno contenuto nell'altro e $x_0$ 
è un punto di accumulazione per $A$.
\end{proof}

Il teorema precedente garantisce che se $x_0$ è un punto di accumulazione 
del dominio di $f$ allora il limite per $x\to x_0$ se esiste è unico. 
In tal caso possiamo dunque dare la seguente.
%
\begin{definition}[operatore di limite]
Sia $f\colon A\subset \RR \to \RR$ una funzione e $x_0$ 
un punto di accumulazione per $A$. 
Se esiste $\ell\in\closeinterval{-\infty}{+\infty}$ tale che
$f(x)\to \ell$ per $x\to x_0$ allora poniamo
\[
  \lim_{x\to x_0} f(x) = \ell.
\]
Lo stesso si può fare per il limite destro $x\to x_0^+$ 
e sinistro $x\to x_0^-$.
Anche per funzioni di variabile e/o di valore 
complesso si può dare la stessa definizione prendendo $x_0\in \bar \CC$ 
e/o $\ell\in \bar \CC$.
\end{definition}

Abbiamo visto che in un punto di accumulazione se il limite 
esiste allora è unico.
Sarà però utile osservare che in effetti il limite può non 
esistere, come si può verificare con un esempio.

\begin{example}[in generale il limite non esiste]
Il limite 
\[
  \lim_{x\to 0} \frac{x}{\abs{x}}
\]
non esiste. 
\end{example}
\begin{proof}
Osserviamo infatti che la funzione $f(x) = \frac{x}{\abs x}$ 
vale $1$ se $x>0$ e $-1$ se $x<0$.
Dunque se il limite esistesse e fosse $\ell\in \bar \RR$, 
scelto $\eps=1$, dovrebbe esistere un intervallo intorno di $0$ 
della forma $\openinterval{-\delta}{\delta}$ tale che 
$\abs{f(x)-\ell}<1$ per ogni $x$ in tale intervallo.
In particolare siccome in tale intervallo ci sono sicuramente 
sia numeri positivi che negativi si dovrà avere 
contemporaneamente
\[
  \abs{1-\ell}<1, \qquad 
  \abs{-1-\ell}<1  
\]
che è impossibile in quanto per disuguaglianza triangolare 
\[
 2 = \abs{1-(-1)} 
 \le \abs{1-\ell} + \abs{-1-\ell}
 < 1 + 1 = 2
\]
che è assurdo.
\end{proof}

Si noti che nell'esempio precedente il limite per $x\to 0$ 
non esiste ma in realtà esistono sia il limite per $x\to 0^+$ 
(vale $1$) che i limite per $x\to 0^-$ (vale $-1$).
Possiamo esibire un esempio ``patologico'' di funzione 
che non ammette limite (né da destra né da sinistra)
in nessun punto:
\[
  f(x) = 
  \begin{cases}
     1 & \text{se $x\in \QQ$}\\ 
     0 & \text{se $x\in \RR\setminus \QQ$}.
  \end{cases}
\]
In ogni intervallo $\openinterval a b$ con $a<b$ questa funzione 
assume sia il valore $0$ che il valore $1$ 
in quanto sia $\QQ$ che $\RR\setminus \QQ$ intersecano 
l'intervallo (sono insiemi densi).
Ma se $f$ avesse limite $\ell$ dovrebbe esistere un intervallo 
in cui i valori distano meno di $\eps$ da $\ell$ e quindi 
dovrebbe essere $\abs{1-\ell}<\eps$ e $\abs{0-\ell}<\eps$
che è impossibile se scegliamo $\eps < \frac 1 2$.

\begin{theorem}[collegamento tra limiti e continuità]%
\mymark{***}%
  Sia $A\subset \RR$, $f\colon A \to \RR$. 
  Se $x_0\in A$ è un punto di accumulazione di $A$
  allora $f$ è continua in $x_0$ se e solo se
  \[
    \lim_{x\to x_0}f(x) = f(x_0).
  \]
  Se $x_0\in A$ non è punto di accumulazione diremo 
  che $x_0$ è un \emph{punto isolato}%
\mymargin{punto isolato}%
\index{punto!isolato} di $A$.
  In tal caso la funzione $f$ è sempre continua nel punto $x_0$.

Risultato analogo vale per funzioni complesse e/o di variabile complessa.
\end{theorem}
  
  \begin{proof}
  In base alla definizione~\ref{def:continua} la funzione $f$ è continua nel
  punto $x_0\in \RR$ se
  \[
   \forall \eps>0 \colon \exists \delta >0 \colon
   \forall x \in A\colon
   \abs{x-x_0}<\delta \implies \abs{f(x)-f(x_0)} < \eps
  \]
  mentre la definizione di limite $f(x)\to f(x_0)$ per $x\to x_0$
  si espande in
  \[
  \forall \eps>0 \colon \exists \delta>0\colon
  \forall x \in A, x\neq x_0\colon
  \abs{x-x_0}<\delta \implies \abs{f(x)-f(x_0)} < \eps.
  \]
  L'unica differenza è che nella definizione di limite
  c'è la condizione $x\neq x_0$. Ma visto che per $x=x_0$
  si ha $\abs{f(x)-f(x_0)}=0$ tale condizione è in questo caso 
  inutile e quindi le due definizioni sono equivalenti.

  Se il punto $x_0$ è isolato la definizione di continuità
  è sempre verificata in quanto esiste un $\delta>0$ 
  tale che il punto $x=x_0$ è l'unico punto di $A$ 
  nell'intorno $(x_0-\delta,x_0+\delta)$.
  \end{proof}

\begin{example}
  Non è difficile convincersi che gli unici punti di accumulazione 
  per l'insieme $\ZZ$ sono $+\infty$ e $-\infty$.
  Dunque qualunque funzione $f\colon \ZZ \to \RR$ è continua in quanto 
  tutti i punti del suo dominio sono punti isolati.
\end{example}
  
\begin{theorem}[località del limite]%
\label{th:localita_limite}%
Il limite di una funzione per $x\to x_0$ dipende solamente dai valori di $f$
in un intorno di $x_0$ e non dipende dal valore di $f$ in $x_0$.

Più precisamente: se $A,B\subset \RR$, $f\colon A\to \RR$, $g\colon B\to \RR$, 
$x_0\in \RR$ sono tali che 
esiste un intorno $V$ di $x_0$ per cui 
$(A\setminus\ENCLOSE{x_0}) \cap  V = (B\setminus \ENCLOSE{x_0}) \cap V$ 
e $f(x)=g(x)$ per ogni $x\in(A\setminus\ENCLOSE{x_0}) \cap  V$ 
allora se $f(x)\to \ell$ per $x\to x_0$ anche $g(x)\to \ell$ 
per $x\to x_0$.

Lo stesso risultato vale per funzioni complesse e/o di variabile complessa.
\end{theorem}
%
\begin{proof}
  Basta osservare che nella definizione di limite 
  non è restrittivo supporre che l'intorno del punto $x_0$ 
  sia sempre preso all'interno dell'intorno $V$ su cui 
  le due funzioni coincidono.
\end{proof}

\begin{theorem}[restrizione del limite]
Se una funzione ha limite $\ell$ per $x\to x_0$ 
e se restringiamo l'insieme di definizione della funzione 
allora il limite della funzione non cambia. 
Più precisamente
se $f\colon A \to \RR$ è una funzione tale che $f(x)\to \ell$ per $x\to x_0$
e se $B \subset A$ e $g\colon B\to \RR$ è la restrizione di $f$ 
a $B$ allora anche $g(x)\to \ell$ per $x\to x_0$. 
\end{theorem}
%
\begin{proof}
Il teorema segue immediatamente dalla definizione di limite se si osserva
che restringendo il dominio la condizione di validità del limite si indebolisce
in quanto gli intorni di $x_0$ vengono intersecati con il dominio della funzione.
\end{proof}

Si osservi che a differenza del teorema sulla località del limite è
possibile che la funzione ristretta $g$ abbia limite quando la funzione
$f$ non aveva limite.
Si osservi anche che in entrambi questi teoremi sarà opportuno 
che $x_0$ sia un punto di accumulazione, altrimenti la condizione $f(x)\to \ell$ 
risulta essere vuota.

\begin{theorem}[legame tra limite, limite destro e limite sinistro]%
\mymark{*}%
Sia $A\subset \RR$, $f\colon A \to \RR$ una funzione e $x_0$ un punto di accumulazione
di $A$. Sia $A^+ = A \cap [x_0,+\infty)$ e $A^- = A \cap (-\infty, x_0]$.

Se $x_0$ è punto di accumulazione sia di $A^+$ che di $A^-$
allora si ha
\[
  \lim_{x\to x_0} f(x) = \ell
\]
se e solo se
\[
  \lim_{x\to x_0^+} f(x) = \lim_{x\to x_0^-} f(x) = \ell.
\]

Se $x_0$ è punto di accumulazione di $A^+$ ma non di $A^-$ allora
i limiti
\[
  \lim_{x\to x_0} f(x) \qquad \text{e}\qquad \lim_{x\to x_0^+} f(x)
\]
sono equivalenti. Analogamente se $x_0$ è punto di accumulazione
di $A^-$ ma non di $A^+$ risultano equivalenti
\[
  \lim_{x\to x_0} f(x) \qquad \text{e}\qquad \lim_{x\to x_0^-} f(x).
\]
\end{theorem}
%
\begin{proof}
Si tratta semplicemente di verificare le definizioni di limite sfruttando il fatto che intorni di un punto $x_0$ sono formati dall'unione di intorno destro e intorno sinistro.
\end{proof}

\begin{theorem}[limite della funzione composta/cambio di variabile]
\label{th:limite_composta}
Siano $A\subset \RR$, $B\subset \RR$,
$x_0$ un punto di accumulazione di $A$,
$y_0$ un punto di accumulazione di $B$,
$\ell\in [-\infty,+\infty]$.
Siano $f\colon A \to B$, $g\colon B\to \RR$
funzioni tali che $f(x)\neq y_0$ se $x\neq x_0$ e 
\[
  \lim_{x\to x_0} f(x) = y_0,
\qquad
  \lim_{y\to y_0} g(y) = \ell.
\]
Allora nel secondo limite si può porre $y=f(x)$ e al posto di $y\to y_0$ 
si può mettere $x\to x_0$ (in quanto il primo limite ci dice che se $x\to x_0$ 
allora $y\to y_0$) e dunque vale:
\[
 \lim_{x\to x_0} g(f(x)) = \ell.
\]
\end{theorem}
%
\begin{proof}
Visto che $g(y)\to \ell$
per ogni $U$ intorno di $\ell$ deve esistere un $V$ intorno di $y_0$
tale che $g((B\setminus\ENCLOSE{y_0})\cap V) \subset U$
e visto  che $f(x)\to y_0$ deve esistere un intorno $W$ di $x_0$
tale che $f((A\setminus\ENCLOSE{x_0})\cap W) \subset V$.
Ma visto che per ipotesi $f$ assume valori in $B\setminus\ENCLOSE{y_0}$
si ha anche $f((A\setminus\ENCLOSE{x_0})\cap W)\subset (B\setminus\ENCLOSE{y_0}) \cap V$
e quindi
\[
  g(f((A\setminus\ENCLOSE{x_0})\cap W)) \subset g((B\setminus \ENCLOSE{y_0}) \cap V)
  \subset U
\]
che significa che $g(f(x)) \to \ell$.
\end{proof}

\begin{exercise}
  Si faccia un esempio di una funzione $f\colon\RR\to \RR$ 
  e una funzione $g\colon\RR\to \RR$ tali che 
  \[
  \lim_{x\to 0} f(x) = 0, \qquad 
  \lim_{x\to 0} g(x) = 0
  \]
  ma
  \[
  \lim_{x\to 0} g(f(x)) = 1.
  \]
  Quale ipotesi nel teorema precedente viene a mancare?
\end{exercise}

\begin{theorem}[limite di funzioni monotòne]%
  \mymark{**}%
  \label{th:limite_monotona}%
Sia $f\colon A \subset \RR \to \RR$ una funzione crescente. 
Se $x_0$ è un punto di accumulazione sinistro per $A$ 
il seguente limite esiste e vale
\[
   \lim_{x\to x_0^-}f(x) = \sup f(\ENCLOSE{x\in A\colon x<x_0})
\]
e se $x_0$ è un punto di accumulazione destro per $A$ 
il seguente limite esiste e vale
\[
   \lim_{x\to x_0^+} f(x) = \inf f(\ENCLOSE{x\in A\colon x>x_0}).
\]
In particolare se $+\infty$ è punto di accumulazione per $A$ 
si ha 
\[
  \lim_{x\to +\infty} f(x) = \sup f(A)
\]
e se $-\infty$ è punto di accumulazione per $A$ si ha 
\[
  \lim_{x\to -\infty} f(x) = \inf f(A).
\]

Gli stessi risultati valgono per le funzioni decrescenti, 
scambiando $\sup$ e $\inf$.
\end{theorem}
%
\begin{proof}\mymark{**}
Supponiamo che $f$ sia crescente e consideriamo il limite 
sinistro $x\to x_0^-$. Può anche essere $x_0=+\infty$, in tal 
caso il limite sinistro $x\to x_0^-$ equivale a $x\to +\infty$.
Poniamo $B=\ENCLOSE{x\in A\colon x<x_0}$
e $\ell=\sup f(B)$ e ricordiamo le proprietà che caratterizzano 
l'estremo superiore
(sappiamo che $B$ non è vuoto in quanto $x_0^-$ per ipotesi 
è un punto di accumulazione per $A$):
\begin{gather*}
  \forall x \in B \colon f(x) \le \ell \\
  \forall y < \ell \colon \exists \alpha \in B \colon f(\alpha) > y.
\end{gather*}
Siccome $f$ è crescente, dalla seconda condizione 
si ottiene che per ogni $x>\alpha$ si ha $f(x)\ge f(\alpha)> y$
e mettendo insieme le due condizioni si ottiene che per ogni $y<\ell$
esiste $\alpha\in\RR$ tale che per ogni $x>\alpha$ si ha $y<f(x)\le \ell$.
Certamente si ha $\alpha < x_0$ perché $\alpha\in B$.   
Dunque la condizione $x>\alpha$ identifica un intorno sinistro 
del punto $x_0$ e si ha dunque $f(x)\to \ell$ per $x\to x_0^-$.

Ragionamento analogo si può fare per il limite destro  
e per le funzioni decrescenti.
\end{proof}

\begin{exercise}
  Utilizzando il teorema precedente si studino le \emph{discontinuità}
  delle funzioni monotone. 
  Si dimostri quindi che l'insieme dei punti in cui una funzione monotona 
  \emph{non} è continua può essere messo in corrispondenza biunivoca 
  con un sottoinsieme di $\QQ$ e dunque tale insieme è numerabile.
\end{exercise}

Il teorema precedente si applica in particolare alle funzioni 
esponenziali, potenze, radici e logaritmi negli estremi dei loro 
domini. Ad esempio è chiaro che se $x_0\in (0,+\infty)$ 
si ha 
\[
  \lim_{x\to x_0} \log_a x = \log_a x_0
\]
in quanto il logaritmo è una funzione continua. 
Se $a>1$ il logaritmo è crescente e la sua immagine è tutto $\RR$ 
dunque si ha 
\[
  \lim_{x\to 0^+} \log_a x = \inf \RR = -\infty
  \qquad
  \lim_{x\to +\infty} \log_a x = \sup \RR = +\infty.
\]
Per l'esponenziale avremo 
\[
 \lim_{x\to x_0} a^x = a^{x_0}
\]
se $x_0\in \RR$ per continuità. 
Se $a>1$ l'esponenziale è crescente 
ed ha immagine $(0,+\infty)$ dunque 
\[
  \lim_{x\to -\infty} a^x = 0, \qquad 
  \lim_{x\to +\infty} a^x = +\infty.
\]
Se $0<a<1$ i limiti all'infinito 
si scambiano in quanto $a^{x}= \frac{1}{a^{-x}}$.
Per le potenze avremo, se $x_0\in[0,+\infty)$ e $\alpha>0$
\[
  \lim_{x\to x_0} x^\alpha = x_0^\alpha
\]
per continuità. Invece essendo $x^\alpha$ crescente e bigettiva 
su $[0,+\infty)$ si avrà 
\[
  \lim_{x\to +\infty} x^\alpha = +\infty.
\]
Se $\alpha<0$ la funzione $x^\alpha$ è continua, decrescente e bigettiva 
su $(0,+\infty)$ e dunque in tal caso:
\[
  \lim_{x\to +\infty} x^\alpha = 0.
\]
Tutti questi limiti \emph{notevoli} possono essere facilmente 
ricordati se teniamo in mente i grafici delle funzioni elementari
Figura~\ref{fig:esponenziale_logaritmo} (a pagina \pageref{fig:esponenziale_logaritmo})
e~\ref{fig:potenza_intera_radice} (a pagina \pageref{fig:potenza_intera_radice}).

Anche la funzione valore assoluto $f(x)=\abs{x}$ è separatamente 
monotona sugli intervalli $[0,+\infty)$ e $(-\infty,0]$. 
Dunque sappiamo che 
\[
  \lim_{x\to +\infty} \abs{x} = \abs{+\infty} = +\infty, 
  \lim_{x\to -\infty} \abs{x} = \abs{-\infty} = +\infty.
\]
Se $x_0\in \RR$ ovviamente si ha pure
\[
  \lim_{x\to x_0} \abs{x_0} = \abs{x_0}
\]
in quanto è banale verificare che la funzione $\abs{x}$ è continua.
Lo stesso vale per la funzione \emph{modulo} $\abs{z}$ quando $z\in \CC$.

Ci sarà anche utile osservare che vale anche questa proprietà 
\[
\lim_{x\to x_0} \abs{f(x)} = 0 \iff 
\lim_{x\to x_0} f(x) = 0
\]
valida perché le definizioni di limite $f(x)\to 0$ e $\abs{f(x)}\to 0$ 
coincidono in quanto $\abs{f(x)-0} = \big\lvert{\abs{f(x)} - 0}\big\rvert$.

\begin{theorem}[permanenza del segno]%
\mymark{***}%
\index{permanenza del segno}%
\index{teorema!della permanenza del segno}%
\mymargin{permanenza del segno}%
Se
\[
  \lim_{x\to x_0} f(x) > 0
\]
allora esiste un intorno $U$ di $x_0$ tale che 
per ogni $x\in U$ si ha $f(x) > 0$.

Viceversa se esiste un intorno $U$ di $x_0$ tale che per ogni $x \in U$  
si ha $f(x)\ge 0$ e se esiste $\ell$ tale che 
\[
    \lim_{x\to x_0} = \ell
\]
allora $\ell\ge 0$.
\end{theorem}
%
Intuitivamente il teorema precedente ci dice che le disuguaglianze strette 
si preservano (permangono) togliendo l'operatore di limite mentre 
le disuguaglianze larghe si preservano passando al limite.
%
\begin{proof}
Sia $\ell\in [-\infty,+\infty]$ il valore del limite.
Se $\ell>0$ esiste certamente un intorno $V$ di $\ell$ 
tale che $V\subset (0,+\infty]$: se $\ell\in \RR$ basta prendere 
$V=(\ell/2,3 \ell/2)$, se $\ell=+\infty$ basta prendere $V=(1,+\infty]$.
Per la definizione di limite esiste $U$ intorno di $x_0$ 
tale che $f(U)\subset V$ e il risultato segue.

Per il viceversa si ragiona per assurdo. 
Se il limite $\ell$ di $f$ esistesse e fosse $\ell < 0$
allora per il punto precedente (applicato a $-f$) dovremmo 
concludere che c'è un intorno di $x_0$ in cui $f(x)<0$.
Ma allora non è possibile che ci sia un intorno in cui $f(x)\ge 0$.
\end{proof}

\begin{theorem}[operazioni con i limiti di funzione]
  \label{th:operazioni_limiti}%
  \index{limite!della somma}%
  \index{limite!del prodotto}%
  \index{limite!del rapporto}%
\mymark{***}%
Se
\[
  \lim_{x\to x_0}f(x) = \ell_1,\qquad
  \lim_{x\to x_0}g(x) = \ell_2
\]
allora si ha
\begin{gather*}
  \lim_{x\to x_0} \enclose{f(x) + g(x)} = \ell_1 + \ell_2, \qquad
  \lim_{x\to x_0} \enclose{f(x) - g(x)} = \ell_1 - \ell_2, \\
  \lim_{x\to x_0} f(x)\cdot g(x) = \ell_1 \cdot \ell_2, \qquad
  \lim_{x\to x_0} \frac{f(x)}{g(x)} = \frac{\ell_1}{\ell_2}
\end{gather*}
sempre che le operazioni utilizzate sul lato destro delle uguaglianze
siano state definite\mynote{%
Si veda la sezione~\ref{sec:reali_estesi}.
I casi in cui le operazioni non sono definite si chiamano 
usualmente \emph{forme indeterminate}
\index{forme indeterminate}
e sono: $+\infty - (+\infty)$, 
$-\infty - (-\infty)$, $+\infty + (-\infty)$, 
$-\infty + (+\infty)$, $0\cdot (+\infty)$, $+\infty \cdot 0$
$0\cdot (-\infty)$, $-\infty \cdot 0$, $\frac 0 0$,
$\frac{+\infty}{+\infty}$, $\frac{-\infty}{-\infty}$,
$\frac{+\infty}{-\infty}$, $\frac{-\infty}{+\infty}$. 
}.
Inoltre se $g(x)\to 0^+$ 
si ha
\[
    \lim_{x\to x_0} \frac{1}{g(x)} = +\infty
\]
(informalmente potremmo scrivere: $\frac{1}{0^+} = +\infty$).

Gli stessi risultati valgono per funzioni di variabile e/o valore complesso.
Anzi i limiti a $\infty$ si comportano meglio, visto che 
$g(z)\to 0$ risulta equivalente a $1/g(z)\to \infty$.

\end{theorem}
%
\begin{proof}
Consideriamo la somma dei limiti
e innanzitutto il caso in cui $\ell_1$ ed $\ell_2$ 
siano entrambi finiti. 
In tal caso
dato un qualunque intorno $U=(\ell-\eps,\ell+\eps)$ 
di $\ell=\ell_1+\ell_2$
se prendiamo gli intorni $U_1=(\ell_1-\eps/2,\ell_1+\eps/2)$ 
e $U_2 = (\ell_2-\eps/2,\ell_2+\eps/2)$ si possono trovare 
degli intorni $V_1$ e $V_2$ di $x_0$ per cui si ha 
$f(V_1\setminus\ENCLOSE{x_0}) \subset U_1$ e 
$f(V_2\setminus \ENCLOSE{x_0}) \subset U_2$
e a maggior ragione questo risulta se prendiamo $V=V_1\cap V_2$ 
al posto di $V_1$ e $V_2$. 
Visto che $U_1+U_2 = U$ si avrà allora 
$(f+g)(V\setminus\ENCLOSE{x_0})\subset U$.

Sempre nel caso della somma 
se $\ell_1=+\infty$ e $\ell_2\neq -\infty$
allora $\ell_1+\ell_2=+\infty$ e 
dato un intorno di $+\infty$ 
della forma $U=(\alpha,+\infty]$ con $\alpha>0$
possiamo prendere 
$U_1 = (2\alpha,+\infty)$ e $U_2 = (\ell_2-\alpha,\ell_2+\alpha)$ 
se $\ell_2\in \RR$ oppure $U_2 = (\alpha,+\infty)$ se $\ell_2=+\infty$.
In ogni caso si ha $U_1+U_2 = U$ e si procede quindi come nel caso precedente.
Il caso $\ell_1=-\infty$ si dimostra in maniera del tutto analoga.

Per la differenza basta osservare che $f(x)-g(x) = f(x) + (-g(x))$.
Se $\ell_2$ è finito la continuità della funzione $h(y)=-y$
ci garantisce che $-g(x)\to -\ell_2$.
Lo stesso si verifica facilmente quando $\ell_2$ è infinito 
(basta osservare che se $U=(\alpha,+\infty]$ è un intorno di $+\infty$
allora $-U = [-\infty,-\alpha)$ è un intorno di $-\infty$). 
Dunque il limite della differenza si riconduce al limite della somma.

Per quanto riguarda il prodotto ci ricordiamo che grazie 
al teorema~\ref{th:isomorfismo} il gruppo additivo totalmente 
ordinato $\RR = \openinterval{-\infty}{+\infty}$ 
è isomorfo al gruppo moltiplicativo $\RR_+ = \openinterval 0 {+\infty}$
e di conseguenza $\bar \RR = \closeinterval{-\infty}{+\infty}$
corrisponde a $\closeinterval {0^+}{+\infty}$
(l'isomorfismo $\RR_+\to \RR$ è dato dalla funzione $\log_a x$ con $a>1$ 
fissato,
tale funzione preserva gli intorni dei punti corrispondenti salvo il fatto 
che gli intorni di $-\infty$ si trasformano in intorni destri di $0$).

Dunque se $f > 0$ e $g > 0$ i risultati per il limite del prodotto 
sono conseguenza dei risultati analoghi per il limite della somma.
Cambiando segno ad una delle due funzioni o ad entrambe ci si può 
ricondurre al caso $f > 0$ e $g > 0$ quando le due funzioni assumono 
un segno costante in almeno un intorno del punto $x_0$.
Per il teorema della permanenza del segno questo è vero se 
$\ell_1\neq 0$ e $\ell_2\neq 0$: il limite del prodotto in questo caso 
è dunque $\ell_1\cdot \ell_2$.
Se invece $\ell_1=0$ e $\ell_2\in\RR$ sappiamo che $f(x)\to 0$ 
è equivalente a $\abs{f(x)}\to 0$ dunque in tal caso si può rimpiazzare $f(x)$ con $\abs{f(x)}\ge 0$
(gli eventuali punti in cui $f(x)=0$ sono irrilevanti e possono essere tolti 
dal dominio di $f$ per garantire $\abs{f}>0$) e ottenere che il limite 
del prodotto è $0$.
Lo stesso accade se $\ell_2=0$ con la funzione $g(x)$.

Per quanto riguarda il rapporto il teorema~\ref{th:isomorfismo}
ci dice che le proprietà additive dell'opposto diventano (tramite logaritmo)
proprietà moltiplicative del reciproco, almeno se siamo nell'ambito di numeri positivi.
Dunque se $f(x)\to \ell$ con $\ell>0$ 
allora $\frac{1}{f(x)} \to \frac{1}{\ell}$.
Se $\ell=0$ non possiamo concludere niente in quanto le proprietà 
additive a $-\infty$ si rispecchiano nelle proprietà moltiplicative 
in $0^+$, non in $0$. 
Dunque i risultati sul limite del rapporto discendono dai risultati 
sul limite del prodotto con il reciproco.
\end{proof}


Se dobbiamo calcolare un limite in cui compare un elevamento 
a potenza $f(x)^{g(x)}$ con base ed esponente variabile converrà 
scrivere
\[
 f(x)^{g(x)} = a^{g(x)\cdot \log_a f(x)}
\]
per poter applicare il teorema precedente al prodotto $g(x)\cdot \log_a f(x)$.

\begin{example}
  Si calcoli
  \[
  \lim_{x\to +\infty} \frac{2+x^2-x^3}{(x^2-x+2)^2}.
  \]
  \end{example}
  \begin{proof}[Svolgimento.]
  Per $x\to +\infty$ per il teorema sul limite del
  prodotto sappiamo che $x^2\to +\infty$ e
  $x^3\to +\infty$.
  Per il teorema sulla somma dei limiti
  sappiamo che $2+x^2\to +\infty$ ma
  a priori non possiamo applicare tale teorema
  al limite di $(2+x^2)-x^3$ in quanto
  $(+\infty)-(+\infty)$ è una forma indeterminata
  (non rientra nelle ipotesi di quel teorema).
  Bisogna allora intuire che le potenze di $x$
  con esponente maggiore sono preponderanti e vanno
  quindi messe in evidenza tramite
  manipolazioni algebriche. L'espressione
  di cui vogliamo calcolare il limite
  si può quindi riscrivere in questo modo:
  \begin{align*}
  \frac{2+x^2-x^3}{(x^2-x+2)^2}
  &= \frac{x^3 \enclose{\frac 2 {x^3}+\frac{1}{x}-1}}{x^4\enclose{1-\frac 1 x + \frac{2}{x^2}}^2}
  = \frac{1}{x}\cdot \frac{\frac 2 {x^3}+\frac{1}{x}-1}{\enclose{1-\frac 1 x + \frac{2}{x^2}}^2}.
  \end{align*}
  A questo punto l'espressione non presenta più forme indeterminate.
  Applicando i teoremi precedenti possiamo allora affermare che
  si ha
  \begin{align*}
  \lim_{x\to+\infty}\frac{1}{x}\cdot \frac{\frac 2 {x^3}+\frac{1}{x}-1}{\enclose{1-\frac 1 x + \frac{2}{x^2}}^2}
  &= \frac{1}{+\infty} \cdot \frac{\frac 2 {(+\infty)^3}+\frac{1}{+\infty}-1}{\enclose{1-\frac 1 {+\infty} + \frac{2}{(+\infty)^2}}^2}\\
  &= 0 \cdot \frac{0+0-1}{(1-0+0)^2} = 0\cdot (-1) = 0.
  \end{align*}
\end{proof}

L'aver definito le operazioni sulle quantità infinite
risulta in effetti comodo nello svolgimento dei limiti.
Bisogna però essere consapevoli che $\bar \RR$ non è un campo
e quindi le operazioni con i simboli $+\infty$ e $-\infty$
non rispettano molte delle regole che siamo abituati
ad avere sui numeri finiti.
Sarà quindi opportuno ricondursi immediatamente ad una espressione
che abbia senso in $\RR$, sulla quale potremo
applicare le manipolazioni algebriche con più tranquillità.
Per questo motivo in molti testi l'espressione intermedia in cui compaiono 
le operazioni eseguite sulle quantità $+\infty$ e $-\infty$ non è ritenuta valida
e si preferisce scrivere direttamente il risultato.
  
\begin{example}
Si calcoli il seguente limite nel campo complesso:
\[
\lim_{z\to 0} \enclose{\frac 1 {z} - \frac 1 {z^2-z}}. 
\]
\end{example}
\begin{proof}
  Anche in questo caso non possiamo applicare direttamente le regole 
  di calcolo del limite in quanto otterremmo la forma indeterminata 
  $\infty - \infty$.
  Possiamo però operare una semplice manipolazione algebrica
  per eliminare l'indeterminazione:
  \[
    \frac 1 {z} - \frac 1 {z^2-z}
    = \frac{z- 1 - 1}{z^2-z}
    = \frac{z-2}{z^2-z}
    \to \frac{0-2}{0^2-0} = \frac{-2}{0} = \infty 
    \qquad \text{per $z\to 0.$}
  \]
\end{proof}

\begin{comment} % non abbiamo ancora gli strumenti per fare questi esercizi
\begin{exercise}
  Calcolare 
  \[
  \lim_{x\to 0^+} x^x.
  \]
\end{exercise}

\begin{exercise}
  Trovare un esempio di funzioni $f(x)$ e $g(x)$ tali che 
  \[
     \lim_{x\to 0} f(x) = 0, \qquad 
     \lim_{x\to 0} g(x) = 0
  \]
  ma 
  \[
    \lim_{x\to 0} f(x)^{g(x)} \neq 1.
  \]
\end{exercise}
\end{comment}

\subsection{proprietà frequenti e definitive}

\begin{definition}[proprietà frequenti e definitive]
Diremo che un predicato $P(x)$ definito su un insieme $A\subset \RR$ 
di cui $x_0$ è punto di accumulazione vale 
\emph{definitivamente}%
\mymargin{definitivamente}%
\index{definitivamente} per $x\to x_0$ se
vale in un intorno di $x_0$ ovvero:
\[
  \exists U\in \B_{x_0}\colon \forall x \in A\cap U\setminus\ENCLOSE{x_0}\colon P(x).
\]
Diremo che $P(x)$ vale \emph{frequentemente}%
\mymargin{frequentemente}%
\index{frequentemente} per $x\to x_0$
se in ogni intorno di $x_0$ c'è almeno un punto $x\neq x_0$ 
in cui vale:
\[
  \forall U\in \B_{x_0}\colon \exists x\in A\cap U\setminus\ENCLOSE{x_0}
  \colon P(x).
\]
\end{definition}

Chiaramente se una proprietà vale definitivamente vale anche frequentemente
infatti se c'è un intorno $U$ su cui vale la proprietà per ogni altro intorno 
$V$ la proprietà risulta valida su $U\cap V$ che non è mai vuoto.
Se una proprietà vale frequentemente significa in particolare che vale per 
infiniti valori diversi in quanto se vale in punto $x\neq x_0$ 
posso sempre trovare un intorno $V$ di $x_0$ che non contiene $x$
e in tale intorno trovo un ulteriore punto in cui vale la proprietà. 
Iterando il procedimento posso trovare infiniti punti diversi su cui 
la proprietà è valida.

Le due proprietà sono complementari nel senso che
in base alle proprietà dei quantificatori logici vale 
la seguente relazione:
\[
  \text{non frequentemente $P(x)$} \iff
  \text{definitivamente non $P(x)$}
\]

Se due proprietà $P(x)$ e $Q(x)$ valgono definitivamente allora anche
$P(x)\land Q(x)$ vale definitivamente. Se invece valgono entrambe
frequentemente allora anche $P(x) \lor Q(x)$ vale frequentemente.

\begin{example}
La proprietà $x^3 - x > 1000 x^2 + 1$
vale definitivamente per $x\to +\infty$.
La proprietà $x>0$ vale frequentemente per $x\to 0$.
\end{example}


La definizione di limite $f(x) \to \ell$ per $x\to x_0$ 
potrebbe quindi enunciarsi così:
per ogni intorno $U$ di $\ell$ si ha $f(x)\in U$ definitivamente.
E la sua negazione è: esiste un intorno $U$ di $\ell$ per cui
frequentemente $f(x)\not\in U$.


\subsection{criteri di confronto}

\begin{theorem}[criteri di confronto]
\label{th:confronto}%
\index{teorema!del confronto}%
\index{limite!confronto}%
\index{teorema!dei due carabinieri}%
\index{carabineri!teorema dei}%
\mymark{***}%
Siano 
$f$, $g$, $h$ tre funzioni reali%
\mynote{Non è possibile fare confronti tra valori complessi, visto che
sui numeri complessi non abbiamo un ordinamento}
definite su uno stesso dominio $A\subset \RR$ 
con punto di accumulazione $x_0$
\mymargin{confronto tra limiti}%
\index{confronto tra limiti}
\begin{enumerate}
\item
Se per ogni $x\in A$ si ha
\[
f(x) \le g(x)
\]
e se entrambe le funzioni ammettono limite: $f(x) \to \ell_1$ 
e $g(x) \to \ell_2$ per $x\to x_0$
allora
\[
\ell_1 \le \ell_2.
\]

\item
Se per ogni $x\in A$ si ha:
\[
f(x) \le g(x)
\]
e se $f(x)\to +\infty$ allora anche $g(x) \to +\infty$ per $x\to x_0$.
Viceversa se $f(x) \le g(x)$ e $g(x) \to -\infty$ allora anche $f(x) \to -\infty$.

\item
(teorema dei carabinieri)
Se per ogni $x\in A$ vale
\mymargin{teorema dei carabinieri}%
\index{teorema!dei carabinieri}%
\[
f(x) \le h(x) \le g(x)
\]
 e se le due
funzioni $f$ e $g$ hanno lo stesso limite: $f(x) \to \ell$ e $g(x)\to \ell$
per $x\to x_0$
allora anche $h(x) \to \ell$.
\end{enumerate}
\end{theorem}
%
\begin{proof}
\mymark{**}
\begin{enumerate}
\item
Se per assurdo fosse $\ell_1 > \ell_2$
la funzione $f(x)-g(x)$ avrebbe limite positivo e per il teorema
della permanenza del segno dovrebbe essere definitivamente positiva. 
Ma questo contraddice l'ipotesi $f(x)-g(x)\le 0$. 

\item 
Se non fosse $g(x)\to +\infty$ significa che 
esiste $\alpha \in \RR$ tale che $g(x)<\alpha$ frequentemente. 
Ma visto che definitivamente $f(x)>\alpha$ troviamo 
che frequentemente si ha $f(x)>g(x)$ in contrasto con l'ipotesi 
$f(x)\le g(x)$.

\item
Se $f$ e $g$ hanno lo stesso limite $\ell$ significa che per ogni
$U$ intorno di $\ell$ si ha definitivamente $f(x)\in U$ e $g(x)\in U$.
Visto che $U$ è un intervallo anche $h(x)$ è definitivamente in $U$ 
e quindi $h(x)\to \ell$.
\end{enumerate}
\end{proof}

\begin{example}
  Dimostrare che 
  \[
    \lim_{x\to +\infty} \enclose{2x - \lfloor x\rfloor} = +\infty. 
  \]
\end{example}
\begin{proof}[Svolgimento.]
  Sappiamo che $\lfloor x\rfloor \le x$ dunque 
  \[
    2x - \lfloor x \rfloor \ge 2x -x = x \to +\infty.  
  \]
  Dunque per confronto si ottiene il risultato richiesto.
\end{proof}

\begin{corollary}[limitata per infinitesima]%
\label{cor:limitata_per_infinitesima}%
\mymark{**}%
  Se $f(x)$ è una funzione limitata e $g(x)\to 0$ per 
  $x\to x_0$ allora anche
  \[
    f(x)\cdot g(x) \to 0 \qquad \text{per $x\to x_0.$} 
  \]
\end{corollary}
%
\begin{proof}
  Se $f$ è limitata significa che esiste $R>0$ tale che $\abs{f(x)}\le R$
  per ogni $x$ nel dominio di $f$.
  Ma allora 
  \[
     0\le \abs{f(x)\cdot g(x)} = \abs{f(x)}\cdot \abs{g(x)} 
     \le R \abs{g(x)} \to R\cdot 0 = 0.
  \]
  Per confronto deduciamo che $\abs{f(x)\cdot g(x)}\to 0$ 
  che è equivalente alla tesi.
\end{proof}

\begin{exercise}
  Mostrare che 
  \[
   \lim_{x\to +\infty} \frac{x-\lfloor x \rfloor}{x} = 0.  
  \]
\end{exercise}
%\chapter{funzioni continue}

\begin{comment}
Lo scopo principale di questo corso è quello di studiare le
funzioni con dominio e codominio nei numeri reali:
\[
  f \colon A \subset \RR \to \RR.
\]
Ad esempio per ogni $n\in \ZZ$ abbiamo già definito la funzione
potenza:
\[
  f(x) = x^n.
\]
Se $n\ge 0$ si ha $f\colon \RR \to \RR$ (la funzione è definita
per ogni $x\in \RR$), se invece $n<0$ si ha
$f\colon \RR\setminus\ENCLOSE{0}\to \RR$ in quanto le potenze
con esponente negativo non sono definite se la base è $0$.
\end{comment}

\section{limiti}

Se una funzione non è definita in un punto potremmo comunque 
essere interessati a capire se, come per le funzioni continue,
intorno a quel punto i valori assunti dalla funzione si avvicinano 
ad un determinato valore.

\begin{definition}[limite finito]
  \label{def:limite_finito}%
  \index{limite!definizione}%
  Sia $f\colon A\subset \RR \to \RR$ una funzione, 
  $x_0\in \RR$ e $\ell \in \RR$.
  Diremo che $f(x)$ tende a $\ell$ (oppure: ha limite $\ell$) 
  quando $x$ tende a $x_0$ 
  e scriveremo
  \[
    f(x) \to \ell \qquad \text{per $x\to x_0$}
  \]
  se 
  \begin{equation}\label{eq:limite_finito}
    \forall \eps>0\colon \exists \delta>0 \colon 
    \forall x \in A\setminus\ENCLOSE{x_0}\colon
    \abs{x-x_0}<\delta \implies \abs{f(x)-\ell}< \eps.
  \end{equation}

  (Stessa identica definizione può essere data per funzioni 
  di variabile complessa e/o valori complessi.)
\end{definition}

Possiamo osservare che la definizione è molto simile alla 
definizione~\ref{def:continua}. 
In effetti confrontando le due definizioni si deduce che 
la condizione $f(x)\to \ell$ per $x\to x_0$ 
è equivalente a richiedere che l'estensione $\tilde f \colon A \cup \ENCLOSE{x_0}\to \RR$ 
definita da 
\[
 \tilde f(x) = \begin{cases}
    f(x) & \text{se $x\neq x_0$}\\ 
    \ell & \text{se $x=x_0$}
 \end{cases}
\]
sia continua nel punto $x_0$. 
Si osservi che non importa che la funzione $f$ sia definita nel punto 
$x_0$ e se è definita il valore $f(x_0)$ è irrilevante nella 
definizione di limite.

\begin{example}
  Si ha 
  \[
  \frac{\sqrt{x}-1}{x-1} \to \frac 1 2 \qquad \text{per $x\to 1$}. 
  \]
  Infatti, moltiplicando numeratore e denominatore 
  per $\sqrt x + 1$ si nota che se $x\ge 0$, $x\neq 1$ si ha:
  \[
      \frac{\sqrt x-1}{x-1}
      = \frac{x-1}{(x-1)(\sqrt x +1)}
      = \frac{1}{\sqrt x + 1}.
  \]
  Osserviamo che la funzione a lato sinistro è definita sull'insieme 
  $A = [0,1)\cup(1,+\infty)$ mentre la funzione a lato
  destro è definita su tutto l'intervallo $[0,+\infty)$.
  Inoltre quest'ultima funzione è continua 
  (in quanto la radice è una funzione continua e somma e rapporto di funzioni 
  continue è una funzione continua) che assume il valore $\frac 1 2$
  quando $x=1$. 
  Dunque la funzione sinistra pur non essendo definita per $x=1$ 
  ha limite $\frac 1 2$ per $x\to 1$.
\end{example}

Sarà molto utile estendere il concetto di limite 
ai punti all'infinito della retta reale estesa (o del piano complesso).
Per fare ciò la definizione di limite che abbiamo dato deve 
essere opportunamente modificata a seconda che la variabile o il valore 
del limite o entrambi siano finiti o infiniti (distinguendo poi $+\infty$ 
da $-\infty$). 
Molti altri casi devono essere poi presi in considerazione 
per introdurre i limiti destro e sinistro.
Per rendere la trattazione più compatta e omogenea useremo 
quindi l'approccio utilizzato nella \myemph{topologia} che è 
il campo della matematica che si occupa di limiti e continuità nella 
massima generalità possibile. 
Faremo questo senza definire cos'è la \emph{topologia} 
(cosa che ci porterebbe troppo lontano) ma semplicemente 
utilizzando la stessa terminologia in maniera trasparente.

\begin{definition}[intorno]
Sia $U\subset\bar \RR$. Andremo a definire cosa 
significa che $U$ è un \myemph{intorno} di $x_0$
per ogni $x_0\in \bar \RR$.

Se $x_0\in \RR$ diremo che $U$ è un intorno di $x_0$ 
quando vale questa condizione:
\[
\exists \eps>0 \colon \forall x\in \RR\colon 
\abs{x-x_0} < \eps \implies x\in U.
\]
Diremo invece che $U$ è un intorno di $+\infty$ 
se 
\[
 \exists M>0 \colon \forall x\in \bar \RR \colon 
 x > M \implies x\in U;
\]
e analogamente diremo che $U$ è un intorno di $-\infty$
se 
\[
 \exists M>0 \colon \forall x\in \bar \RR\colon 
 x < -M \implies x \in U.  
\]

Dato $x_0\in \bar \RR$ denotiamo con
\[
  \mathcal U_{x_0} = \ENCLOSE{U\in \mathcal P(\bar \RR)
  \colon \text{$U$ è un intorno di $x_0$}}
\]
la famiglia di tutti i suoi intorni.

(Definizioni analoghe possono essere date sul piano complesso: al finito 
la definizione è formalmente la stessa; gli intorni di $\infty\in \bar \CC$ 
sono gli insiemi $U\subset \bar \CC$ per cui esiste $M>0$ tale che per ogni $z\in \bar \CC$ 
se $\abs{z}>M$ si ha $x\in U$.)
\end{definition}

\begin{definition}[limite generale]
  Sia $f\colon A\subset \bar \RR \to \bar \RR$ una funzione, 
  sia $x_0\in \bar \RR$ e $\ell\in \bar \RR$. 
  Scriveremo 
  \[
    f(x)\to \ell\qquad\text{per $x\to x_0$}
  \]
  se vale la seguente condizione:
  \begin{equation}\label{def:limite}
  \forall U\in \mathcal U_\ell\colon 
  \exists V\in \mathcal U_{x_0}\colon 
  \forall x\in A\setminus\ENCLOSE{x_0}\colon 
  x\in V \implies f(x) \in U.
  \end{equation}
\end{definition}

Osserviamo innanzitutto che questa definizione è equivalente 
alla definizione~\ref{def:limite_finito}.
Infatti la proprietà~\eqref{eq:limite_finito} può essere riscritta 
così:
\[
\forall \eps>0\colon \exists \delta>0\colon 
\forall x\in A\setminus\ENCLOSE{x_0}\colon 
 x \in (x_0-\delta,x_0+\delta) \implies f(x) \in (\ell-\eps,\ell+\eps).
\]
Supponiamo che quest'ultima proprietà sia valida e dimostriamo che allora 
anche 


