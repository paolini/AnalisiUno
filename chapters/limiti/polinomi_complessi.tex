\section{polinomi complessi}
\label{ch:ancora_polinomi}

\subsection{il teorema fondamentale dell'algebra}

Il teorema fondamentale dell'algebra afferma che ogni polinomio non costante 
si annulla in almeno un punto del piano complesso.
\mynote{si vedano le note storiche a fine capitolo}
Per dimostrare il teorema dobbiamo estendere il teorema 
di Weierstrass alle funzioni di una variabile complessa.
Nel teorema di Weierstrass reale la funzione per ipotesi è definita su un intervallo
chiuso e limitato. 
Nel piano complesso non esiste il concetto di \emph{intervallo} in quanto non abbiamo 
un ordinamento ma vedremo che comunque il teorema di Weierstrass rimane valido per 
le funzioni continue definite su insiemi chiusi e limitati secondo le seguenti definizioni.

\begin{definition}[chiusura sequenziale]
Un insieme $A\subset \CC$ si dice
essere \emph{sequenzialmente chiuso}
\mymargin{sequenzialmente chiuso}%
\index{sequenzialmente!chiuso}%
se presa una qualunque successione
di punti $a_n\in A$ se $a_n \to a$ per qualche $a\in \CC$
allora $a\in A$.
\end{definition}

\begin{definition}[limitatezza]
Un insieme $A\subset \CC$ si dice essere \emph{limitato}%
\mymargin{limitato}%
\index{limitato}
se
\[
  \sup \ENCLOSE{ \abs{z}\colon z \in A} < +\infty.
\]
\end{definition}

Si noti che una successione $a_n\in \CC$ 
è limitata (si veda capitolo~\ref{sec:successione_limitata})
se e solo se la sua immagine $\ENCLOSE{a_n\colon n\in \NN}\subset \CC$
è un insieme limitato.

% \begin{theorem}[Bolzano-Weierstrass complesso]
% Se $z_n\in \CC$ è una successione limitata allora
% è possibile estrarre una sottosuccessione $z_{n_k}$ convergente:
% $z_{n_k} \to z$ con $z\in \CC$.
% \end{theorem}
% %
% \begin{proof}
% Siano $x_n$ e $y_n$ la parte reale ed immaginaria di $z_n$: $z_n = x_n + i y_n$. Visto che $\abs{x_n} =\sqrt{x_n^2}\le \sqrt{x_n^2+y_n^2} = \abs{z_n}$ e, allo stesso modo $\abs{y_n} \le \abs{z_n}$,
% possiamo affermare che entrambe le successioni $x_n$ e $y_n$ sono limitate (ma stavolta in $\RR$).
% Dunque possiamo applicare il teorema di Bolzano-Weierstrass (reale) alla successione $x_n$ per trovare una sottosuccessione $x_{n_j}\to x$ convergente. E possiamo applicare di nuovo il teorema di Bolzano-Weierstrass alla sottosuccessione $y_{n_j}$ per trovare una sotto-sottosuccessione $y_{n_{j_k}}\to y$ anch'essa convergente.
% Posto $n_k = n_{j_k}$ avremo dunque trovato una sottosuccessione $z_{n_k} = x_{n_k} + i y_{n_k} \to x+iy$ convergente.
% \end{proof}

\begin{theorem}[Weierstrass complesso]
Sia $A\subset \CC$ un insieme non vuoto, sequenzialmente chiuso e limitato e sia $f\colon A \to \RR$ una funzione continua.
Allora $f$ ha massimo e minimo su $A$.
\end{theorem}
%
\begin{proof}
Dimostriamo l'esistenza del minimo: per il massimo la dimostrazione è perfettamente analoga.
Sia $m=\inf f(A)$.
Essendo $A$ non vuoto, per il lemma sull'esistenza delle successioni minimizzanti sappiamo esistere una successione $z_n \in A$ tale che $f(z_n) \to m$.
Essendo $A$ limitato possiamo applicare il teorema di Bolzano-Weierstrass per trovare $z\in \CC$ e una sottosuccessione $z_{n_k} \to z$. Essendo $A$ sequenzialmente chiuso possiamo quindi affermare che $z\in A$. Essendo $f$ continua concludiamo che
\[
f(z) = \lim_{k\to+\infty} f(z_{n_k}) = m
\]
e dunque $z$ è un punto di minimo per $f$.
\end{proof}

\begin{theorem}[esistenza del minimo per funzioni coercive]
Sia $f\colon \CC \to \RR$ una funzione continua tale che per ogni
successione $z_n \to \infty$ (ovvero $\abs{z_n}\to +\infty$)
si abbia $f(z_n) \to +\infty$.
Allora $f$ ha minimo.
\end{theorem}
%
\begin{proof}
Consideriamo l'insieme
\[
  A = \ENCLOSE{z \in \CC \colon f(z) \le f(0)}.
\]
Chiaramente $0\in A$ e quindi $A$ non è vuoto.
L'insieme $A$ è anche sequenzialmente chiuso in quanto se $z_k\in A$ allora $f(0) - f(z_k)\ge 0$,
per continuità $f(0)-f(z_k)\to f(0)-f(z)$
e per il teorema della permanenza del segno si ottiene $f(0)-f(z) \ge 0$ cioè $z \in A$.
Dimostriamo ora che $A$ è anche limitato. 
Se non lo fosse esisterebbe, per assurdo, una successione $z_n \in A$ tale che $\abs{z_n}\to +\infty$ cioè $z_n \to \infty$. 
Ma allora, per ipotesi su $f$, si avrebbe $f(z_n)\to +\infty$ che contraddice la condizione $f(z_n) \le f(0)$. 
Essendo $A$ non vuoto, sequenzialmente chiuso e limitato ed essendo $f\colon A \to \RR$ continua, 
possiamo applicare il teorema di Weierstrass complesso per dedurre che $f$ ha minimo su $A$ in un punto $w \in A$. 
Ma essendo $0\in A$ si avrà sicuramente $f(w)\le f(0)$ e per ogni $z\in \CC \setminus A$ si ha invece $f(z) > f(0)$ per come è stato definito $A$. 
Dunque $w$ è minimo di $f$ su tutto $\CC$, non solo su $A$.
\end{proof}

\begin{theorem}[teorema fondamentale dell'algebra]
\label{th:fondamentale_algebra}
\mymargin{teorema fondamentale dell'algebra}%
\index{teorema!fondamentale dell'algebra}%
Sia $f(z)$ un polinomio di grado $N\ge 1$ a coefficienti complessi:
\[
  f(z) = \sum_{j=0}^N a_j \cdot z^j
\]
con $a_j\in \CC$ per $j=0,\dots,N$ e $a_N \neq 0$.
Allora esiste $w\in \CC$ tale che $f(w) = 0$.
\end{theorem}
%
\begin{proof}
  Osserviamo innanzitutto che $\abs{f(z)}$ è coerciva cioè che
  se $z_n \to \infty$ allora $\abs{f(z_n)}\to +\infty$.
  Infatti si ha
  \begin{align*}
    \abs{f(z_n)}
    &= \abs{\sum_{j=0}^N a_j z_n^j}
    = \abs{a_N z_n^N  + \sum_{j=0}^{N-1} a_j z_n^j}\\
    &= \abs{z_n}^N \cdot \abs{a_N + \sum_{j=0}^{N-1} \frac{a_j}{z_n^{N-j}}}
    \to +\infty
  \end{align*}
  se $z_n \to \infty$.
  
  Sappiamo che tutti i polinomi sono funzioni continue in quanto somme di 
  prodotti di funzioni continue e il modulo è anch'esso una funzione continua 
  dunque $\abs{f(z)}$ è certamente una funzione continua.
  
  Dunque possiamo applicare il teorema di esistenza del minimo per le funzioni 
  coercive: esiste $w\in \CC$ tale che $\abs{f(w)}$ è minimo.
  
  Per concludere il teorema basterà dimostrare che $f(w)=0$.
  L'idea che vogliamo sviluppare è che i polinomi complessi se assumono un valore
  $f(w)$ in un punto $w\in \CC$ allora assumono anche tutti i valori vicini
  ad esso in quanto \emph{localmente} il polinomio assomiglia ad una potenza $z^n$
  e l'equazione $z^n=c$ ha sempre soluzione, come abbiamo già visto.
  Dunque vicino a $w$ ci saranno dei punti in cui $f$ assume valori che in modulo 
  sono minori a $f(w)$: a meno che non sia proprio $f(w)=0$, nel qual caso 
  ovviamente non è possibile avere numeri con modulo inferiore a $0$.
  
  Possiamo scrivere il polinomio $f(z)$ nella forma
  \[
    f(z) = \sum_{j=0}^N b_j (z-w)^j
  \]
  in quanto la traslazione di un polinomio è ancora un polinomio dello stesso 
  grado.
  Dimostrare che $f(w)=0$ è quindi equivalente a dimostrare che $b_0=0$.
  Supponiamo allora per assurdo che sia $b_0\neq 0$. 
  L'andamento del polinomio vicino al punto $w$ è dominato dai termini di grado 
  più basso: alcuni di questi potrebbero essere nulli, ma possiamo considerare 
  il primo indice $k>0$ per cui si ha $b_k\neq 0$. Allora potremo scrivere:
  \[
      f(z) = b_0 + b_k(z-w)^k + \sum_{j=k+1}^N b_j (z-w)^j.
  \]
Ora scriviamo $z-w$, $b_0$ e $b_k$ in forma esponenziale:
$z - w = \rho e^{i\theta}$, $b_0 = r e^{i \alpha_0}$,
$b_k = r_k e^{i\alpha_k}$ 
e applichiamo la disuguaglianza triangolare
  \begin{align*}
    \abs{f(z)} 
    &= \abs{r e^{i\alpha} + r_k \rho^k e^{i(\theta k+\alpha_k)} 
    + \sum_{j=k+1}^N b_j \rho^j e^{i\theta j}} \\
    &\le \abs{r e^{i\alpha} + r_k \rho^k e^{i(\theta k + \alpha_k )}} 
    + \sum_{j=k+1}^N \abs{b_j} \rho^j.
  \end{align*}
Per raggiungere un assurdo vogliamo dimostrare che esiste $z$ (cioè esistono $\rho$ e $\theta$)
per cui il valore della funzione risulti in modulo minore di $r = \abs{b_0} = \abs{f(w)}$.
Innanzitutto se $\rho$ è sufficientemente piccolo possiamo rendere arbitrariamente piccole 
le potenze $\rho^j$ con $j>k$: basta scegliere $\rho\le 1$ 
e $\rho \le r_k/(2\sum \abs{b_j})$
cosicché si avrà 
\[
  \sum_{j=k+1}^N \abs{b_j} \rho^j
  \le \rho^{k+1} \sum_{j=k+1}^N \abs{b_j} \le r_k\frac{\rho^{k}}{2}.
\]
Per quanto riguarda il termine
$r e^{i\alpha} + r_k \rho^k e^{i(\theta k + \alpha_k)}$ 
sarà sufficiente scegliere $\theta = (\pi + \alpha - \alpha_k) /k$ 
in modo che i due addendi 
abbiano fase opposta e la somma sia distruttiva:
\[
  \abs{r e^{i\alpha} + r_k \rho^k e^{i(\theta k + \alpha_k)}}
  = \abs{e^{i\alpha}(r + r_k \rho^k e^{i\pi})}
  = r - r_k\rho^k.
\]
In definitiva abbiamo
\[
\abs{f(z)} = \abs{f(w+\rho e^{i\theta})} 
\le r - r_k\rho^k + r_k\frac{\rho^k}{2} 
< r = \abs{f(w)}
\]
che è assurdo in quanto $\abs{f(w)}$ doveva essere il valore minimo.
\end{proof}
  
\subsection{fattorizzazione dei polinomi}

\begin{theorem}[fattorizzazione dei polinomi complessi]
\index{decomposizione!dei polinomi complessi}%
\index{fattorizzazione!dei polinomi complessi}%
\index{polinomio!fattorizzazione complessa}%
Sia $p(z)$ un polinomio non nullo. Allora posto $n=\deg p$ esistono dei numeri complessi $z_1, z_2, \dots, z_n$ ed un numero complesso $c\neq 0$ tali che
\begin{equation}\label{eq:34985}
  p(z) = c \prod_{k=1}^n (z-z_k).
\end{equation}
Gli $z_k$ sono unici a meno dell'ordine e $c$ pure è univocamente determinato.
\end{theorem}
%
\begin{proof}
Dimostriamo il teorema per induzione su $n=\deg p$. Se $n=0$ il polinomio $p$ è costante: $p(z) = c$. Ricordando che un prodotto di $n=0$ fattori è uguale a $1$ si ottiene quindi il risultato voluto.

Sia ora $p(z)$ un qualunque polinomio di grado $n>0$. Per il teorema fondamentale dell'algebra sappiamo che esiste un numero complesso $z_n$ tale che $p(z_n)=0$. Per il teorema di Ruffini si ha allora
\[
  p(z) = (z-z_n) q(z)
\]
con $q$ un qualche polinomio di grado $n-1$. Per ipotesi induttiva possiamo dunque supporre che esistano $z_1, \dots, z_{n-1}$ e $c$ numeri complessi tali che
\[
   q(z) = c \prod_{k=1}^{n-1} (z-z_k)
\]
e la tesi segue.
\end{proof}

Nella fattorizzazione~\eqref{eq:34985} le \emph{radici} $z_k$ possono
anche ripetersi. Se mettiamo insieme i fattori corrispondenti alla stessa radice
si ottiene una decomposizione della forma
\begin{equation}\label{eq:358925}
P(z) = c \prod_{k=1}^m (z-z_k)^{p_k}
\end{equation}
con $z_1, \dots, z_m$ numeri complessi distinti (le radici del polinomio $P$)
e $p_k$ interi positivi.
L'esponente $p_k$ si chiama
\emph{molteplicità}%
\mymargin{molteplicità}%
\index{molteplicità} della radice $z_k$ e risulta
\[
  p_1 + \dots + p_m = n.
\]
Questa uguaglianza si esprime dicendo che un polinomio $P\in \CC[z]$
di grado $n$ ha sempre $n$ radici contate con la loro molteplicità.
Le radici distinte sono invece $m\le n$.

Se invece $P\in \RR[x]$ è un polinomio a coefficienti reali
non è detto che $P$ abbia radici: ad esempio il
polinomio $x^2+1$ non ha radici reali in quanto $x^2+1>0$
come succede in ogni campo ordinato.
Potrà allora essere utile pensare a $P$ come ad un polinomio
in $\CC[x]$ con coefficienti reali.

Se $P\in \RR[x]$ è un polinomio a coefficienti reali
\[
  P = \sum_{k=0}^n a_k x^k, \qquad a_k\in \RR
\]
possiamo pensare a $P$ anche come un polinomio in
$\CC[x]$, visto che $a_k\in \RR\subset \CC$.
Il seguente teorema ci dà un criterio per distinguere
i polinomi a coefficienti reali dentro a $\CC[x]$.

\begin{theorem}
\label{th:caratterizzazione_polinomi_reali}%
Se $P\in \CC[x]$ è un polinomio a coefficienti complessi
\[
  P = \sum_{k=0}^n a_k x^k,\qquad a_k\in \CC
\]
e se esiste un insieme infinito $A\subset \RR$
tale che per ogni $x\in A$ si abbia $P(x)\in \RR$
allora $a_k\in\RR$ per ogni $k=0,\dots,n$ e dunque $P\in \RR[x]$.

In particolare se la funzione polinomiale associata a
$P\in\CC[x]$ manda $\RR$ in $\RR$ allora $P$ è un polinomio
a coefficienti reali.
\end{theorem}
%
\begin{proof}
Consideriamo il polinomio:
\[
  Q = \sum_{k=0}^n (a_k-\bar a_k) x^k.
\]
Allora per ogni $x\in A$ si ha
\[
  Q(x) = \sum_{k=0}^n a_k x^k - \sum_{k=0}^n \bar a_k x^k
     = P(x) - \overline{P(x)} = P(x) - P(x) = 0
\]
in quanto se $x\in \RR$ risulta
$\overline{x^k} = {\bar x}^k = x^k$.
Visto che $A$ è infinito,
per il principio di annullamento dei polinomi
(teorema~\ref{th:annullamento_polinomi})
deduciamo che $Q=0$.
Ma allora tutti i coefficienti di $Q$ devono
essere nulli e cioè $\bar a_k = a_k$.
Ne consegue che $a_k\in \RR$ per ogni $k=0,\dots,n$.
\end{proof}

Rifacendoci alla fattorizzazione dei polinomi a coefficienti
complessi possiamo fattorizzare anche i polinomi a coefficienti
reali se ci accontentiamo di avere fattori quadratici
invece che lineari.

\begin{theorem}[fattorizzazione dei polinomi reali]
  \label{th:fattorizzazione_polinomio_reale}%
Se $P\in \RR[x]$ è un polinomio a coefficienti reali
potremo scrivere
\begin{equation}\label{eq:35549}
  P = a \cdot \prod_{k=1}^n (x-x_k)^{p_k} \cdot \prod_{j=1}^m (x^2 + \alpha_j x + \beta_j)^{q_j}
\end{equation}
dove $a\in \RR$, $x_k\in \RR$, $p_k\in \NN\setminus\ENCLOSE{0}$,
$\alpha_j,\beta_j\in \RR$, $q_j\in \NN\setminus\ENCLOSE{0}$
con $\alpha_j^2 - 4 \beta_j < 0$
e
\[
  \sum_{k=1}^n p_k + 2 \sum_{j=1}^m q_j = \deg P.
\]

La fattorizzazione~\eqref{eq:35549} è unica a meno
dell'ordine dei fattori.

I numeri $x_1,\dots,x_n$ sono tutte le radici reali distinte
del polinomio $P$ con rispettive molteplicità
$p_1,\dots,p_n$ mentre tutte le radici complesse (non reali)
distinte saranno $\mu_1,\dots, \mu_m$
e $\bar \mu_1, \dots, \bar \mu_m$ con
\[
  x^2 + \alpha_j x + \beta_j = (x-\mu)\cdot(x-\bar \mu)
\]
da cui
\[
  \alpha_j = -2\Re \mu_j, \qquad \beta_j=\abs{\mu_j}^2
\]
e $q_j$ saranno le molteplicità delle radici $\mu_j$
 e $\bar \mu_j$.
\end{theorem}
%
\begin{proof}
Osserviamo innanzitutto che se $P$ è a coefficienti reali
si ha
\[
  \overline{P(z)} = P(\bar z), \qquad \forall z\in \CC
\]
dunque se $\mu$ è una radice di $P$ anche $\bar \mu$
è una radice di $P$.
Ora se $\mu$ è una radice non reale di $P$ sappiamo
che in campo complesso $P$ risulta divisibile
per $x-\mu$ (teorema~\ref{th:Ruffini} di Ruffini):
\[
  P = (x-\mu) \cdot P_1.
\]
Ma anche $\bar \mu$ è radice di $P$ ed essendo
$\bar \mu \neq \mu$ si dovrà avere
\[
Q(\bar \mu) = \frac{P(\bar \mu)}{\bar \mu - \mu} = 0
\]
e dunque applicando nuovamente il teorema di Ruffini
\[
 P = (x-\mu)\cdot (x-\bar \mu)\cdot P_2.
\]
Ora osserviamo che
\[
(x-\mu)\cdot(x-\bar \mu) = x^2 - (\mu + \bar \mu) x + \mu \bar \mu
 = x^2 - 2 (\Re \mu)\cdot x + \abs{\mu}^2
\]
è un polinomio a coefficienti reali e
visto che $P_2$ si ottiene dividendo $P$ per tale polinomio,
anche $P_2$ è un polinomio a coefficienti reali.

Ripetendo lo stesso procedimento sul polinomio $P_2$
potremo fattorizzare $P$ diminuendo il grado a passi
di $2$ finché non si esauriscono tutte le radici complesse
non reali accoppiandole a due a due.
Dunque se $\mu$ è una radice complessa
del polinomio reale $P$ la molteplicità di $\mu$ è
uguale alla molteplicità di $\bar \mu$.

Dopodiché
si potrà completare la fattorizzazione dividendo
per i fattori lineari $x-x_k$ corrispondenti
alle radici reali del polinomio $P$.
Si otterrà quindi la fattorizzazione desiderata.
\end{proof}

