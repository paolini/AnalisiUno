\section{l'esponenziale complesso}
\label{sec:esponenziale_complesso}%

L'esponenziale reale e le funzioni trigonometriche possono essere pensate 
come strumenti intermedi per definire un isomorfismo naturale 
da $\CC$ come gruppo additivo in $\CC$ come gruppo moltiplicativo.

Possiamo infatti definire la funzione $\exp \colon \CC \to \CC$ mediante 
la \emph{formula di Eulero}
\[
 \exp(x+iy) = e^x \cdot (\cos y + i \sin y).
\]
Questa funzione ha le seguenti proprietà:
\begin{enumerate}
  \item $\exp(z+w) = \exp z \cdot \exp w$;
  \item $\exp\bar z = \overline{\exp z}$;
  \item $\abs{\exp z} = e^{\Re z}$;
  \item $\exp x = e^x$ se $x\in \RR$.
\end{enumerate}
Visto che $\exp$ estende la funzione esponenziale reale $e^x$
sarà anche naturale usare la stessa notazione ponendo:
\mynote{%
La notazione $e^z$ non ci deve far pensare che abbiamo definito 
una operazione di elevamento a potenza tra numeri complessi. 
In effetti non è possibile definire in maniera sensata e univoca 
la potenza $z^w$ se la base $z$ non è un numero reale positivo.
Questo è legato al fatto che la funzione esponenziale $e^z$
non è univocamente invertibile e quindi non si può definire il 
logaritmo $\ln z$ se non come funzione \emph{multivoca}.
}%
\[
  e^z = \exp z = e^x\cdot (\cos y + i\sin y), \qquad \text{se $z=x+iy$}.  
\]

Si noti che la funzione esponenziale complessa è $2\pi i$ periodica, infatti 
se $z=x+iy$ si ha 
\[
\exp(z+2\pi i) 
= e^x\cdot (\cos(y+2\pi) + i\sin(y+2\pi)) 
= e^x\cdot (\cos y + i \sin y) 
= \exp(z).
\]

% Nel capitolo precedente abbiamo introdotto l'esponenziale complesso ed
% abbiamo osservato che la funzione $f\colon \RR \to \CC$ definita da
% $f(t) = e^{it}$ ha valori sulla circonferenza unitaria in quanto
% $\abs{e^{it}}=1$. Tramite la definizione~\ref{def:sincos}
% abbiamo introdotto le funzioni seno e coseno in
% modo che risulti $f(t) = \cos t + i \sin t$.
% Sappiamo che $f(0) = e^0 = 1$ e, per come abbiamo definito $\pi$,
% sappiamo che $f(\pi/2) = i$.

\subsection{rappresentazione polare dei numeri complessi}

Per come li abbiamo definiti i numeri complessi $z\in \CC$ si possono 
rappresentare nella forma:
\[
  z = x + i y.
\]
Questa rappresentazione dei numeri complessi viene chiamata \emph{cartesiana}
perché fa corrispondere ogni numero $z$ alle sue coordinate $(x,y)$ nel
piano complesso (piano di Gauss). 

Grazie alla definizione di esponenziale complesso possiamo anche 
dare una rappresentazione \emph{polare} dei numeri complessi.
Se $z=x+iy$ è un qualunque numero complesso possiamo definire 
$\rho = \abs{z} = \sqrt{x^2+y^2}$ il suo \emph{modulo} 
ovvero la distanza geometrica tra il punto $z$ del piano complesso 
e l'origine $0\in \CC$.
Se $z\neq 0$ possiamo definire la misura dell'angolo individuato 
da $z$ con l'asse delle $x$ come quell'unico 
$\theta \in \closeopeninterval{0}{2\pi}$ tale che
\[
  \begin{cases}
    x = \rho \cos \theta,\\
    y = \rho \sin \theta.
  \end{cases}
\]
Si ha quindi 
\[
  z = \rho \cdot (\cos \theta + i \sin \theta). 
\]
La coppia di numeri $(\theta,\rho)$ con $\rho>0$ e $\theta\in\closeopeninterval{0}{2\pi}$ 
si chiamano \emph{coordinate polari} del numero complesso $z$ 
e identificano univocamente $z$. 
Per la periodicità delle funzioni $\sin$ e $\cos$ risulta chiaro 
che l'angolo $\theta$ può essere sostitutito con $\theta +2k\pi$ 
per qualunque $k\in \ZZ$ lasciando invariato il punto $z$.
Se $z=0$ allora $\rho=\abs{z}=0$ e la coordinata $\theta$ è irrilevante.

La rappresentazione \emph{esponenziale} di un numero complesso 
è sostanzialmente identica alla rappresentazione polare 
ma utilizza l'esponenziale complesso invece che 
le funzioni trigonometriche. 
Essendo $e^{i\theta} = \cos \theta + i\sin \theta$
se $z=\rho\cdot (\cos \theta + i \sin \theta)$ potremo scrivere:
\[
  z = \rho \cdot e^{i\theta}.  
\]

Se $z=\rho e^{i\theta}$
il numero $\theta$ viene usualmente chiamato \emph{argomento}
del numero complesso $z$ e si denota a volte 
in questo modo:
\[
  \theta = \arg z.  
\]
La definizione di argomento è intrinsecamente ambigua 
in quanto $\theta$ non è univocamente determinato
(al posto di $\theta$ possiamo scegliere $\theta+2k\pi)$ 
con qualunque $k\in \ZZ$).
Per avere una definizione univoca si può imporre 
la condizione $\theta\in\closeopeninterval{0}{2\pi}$.
\mynote{Ma la condizione 
$\theta\in\closeinterval{-\pi}{\pi}$ 
andrebbe ugualmente bene.}
Se $z=0$ possiamo definire, arbitrariamente, 
$\arg z = 0$. 
In formule si ha:
\[
  \arg z =
  \begin{cases}
%   \arctg \frac y x & \text{se $x>0$,} \\
   \frac \pi 2 - \arctg \frac x y & \text{se $y>0$,} \\
   \frac 3 2 \pi- \arctg \frac x y & \text{se $y<0$,} \\
   \pi & \text{se $y=0$ e $x<0$,} \\
   0 & \text{se $y=0$ e $x\ge 0$.}
   \end{cases}
\]

\subsection{radici complesse $n$-esime}

Sia $c\in \CC$ un numero
complesso $c\neq 0$.
Ci poniamo il problema di determinare le soluzioni complesse
dell'equazione
\[
  z^n = c.
\]
Tali soluzioni saranno chiamate \emph{radici $n$-esime}%
\mymargin{radici $n$-esime}%
\index{radice!$n$-esima} di $c$.

Scriviamo $c$ e $z$ in forma esponenziale:
\[
  c = r e^{i\alpha}, \qquad
  z = \rho e^{i\theta}.
\]
Si avrà allora
\[
  z^n = \rho^n (e^{i\theta})^n = \rho^n e^{i n \theta}.
\]
Affinche sia $z^n = c$ si dovrà avere l'uguaglianza dei moduli, cioè $\rho^n = r$ e l'uguaglianza a meno di multipli interi di $2\pi$ degli argomenti:
$n \theta = \alpha + 2 k \pi$ con $k\in \ZZ$.
Dunque si trova
\[
  \theta = \frac{\alpha}{n} + k\frac{2\pi}{n}
\qquad k \in \ZZ.
\]
Osserviamo ora che per $k=0,\dots, n-1$ il secondo addendo
$k 2\pi /n$ assume $n$ valori distinti compresi in $[0,2\pi)$.
Per gli altri valori di $k$ si ottengono degli angoli che differiscono
da questi di un multiplo di $2\pi$ e quindi non si trovano
altre soluzioni.

Dunque l'equazione $z^n = c$ per $c\neq 0$ ha $n$ soluzioni distinte date
da
\[
z_k = \sqrt[n]{r} \cdot e^{i\alpha/n + 2k\pi i /n},
\qquad k=0,1, \dots, n-1
\]
dove $\alpha = \arg(c)$ e $r = \abs{c}$.
Dal punto di vista geometrico si osserva che
$z_0$ è il numero complesso con modulo la radice $n$-esima del numero
dato $c$ e argomento pari ad un $n$-esimo dell'argomento di $c$.
Tutte le altre soluzioni si trovano sulla circonferenza centrata in $0$
e passante per $z_0$ e risultano essere, insieme ad $z_0$, i vertici
di un $n$-agono regolare.

In particolare nel caso $c=1$ si osserva che le radici $n$-esime dell'unità
si rappresentano geometricamente come i vertici dell'$n$-agono regolare iscritto
nella circonferenza unitaria e con un vertice in $z_0=1$.

\begin{exercise}
Si trovino le soluzioni $z \in \CC$ delle seguenti equazioni.
Scrivere le soluzioni in forma polare e cartesiana.
\begin{gather*}
   z^4 = -4 \\
   z^6 = i\\
   z^3 = -8i \\
   z^4 = z\\
   z^2 + 1 = i\sqrt{3} \\
   (z-i)^4 = 1\\
   1 + z + z^2 + z^3 = 0\\
   z^{14} - z^6 - z^8 + 1 = 0
\end{gather*}
\end{exercise}

