\section{il teorema di Weierstrass}

Ricordiamo che nella definizione~\ref{def:funzione_limitata}
abbiamo definito i concetti di massimo, minimo, estremo superiore
e inferiore di una funzione a valori reali.

\begin{lemma}[successioni minimizzanti/massimizzanti]%
\mymargin{successioni mi\-ni\-miz\-zan\-ti}%
\index{successione!minimizzanti}%
\index{successione!massimizzanti}%
Sia $A$ un insieme non vuoto e
sia $f\colon A \to \RR$ una funzione. Allora esistono
due successioni $a_n$ e $b_n$ di punti di $A$ tali che
\[
  \lim_{n\to +\infty} f(a_n) = \inf f(A), \qquad
  \lim_{n\to +\infty} f(b_n) = \sup f(A).
\]
\end{lemma}
\mymargin{successioni mi\-ni\-miz\-zan\-ti}
%
\begin{proof}
Ricordiamo che $f(A) = \ENCLOSE{f(x)\colon x \in A}$ è l'immagine
della funzione $f$. Facciamo la dimostrazione per l'estremo inferiore,
risultato analogo si potrà ottenere per l'estremo superiore.

Sia $m=\inf f(A)$.
Se $m=-\infty$ significa che $f(A)$ non è inferiormente limitato,
in particolare per ogni $n\in \RR$ esiste $a_n$ tale che
$f(a_n) < - n$.
Dunque (per confronto) $f(a_n) \to -\infty$
come volevamo dimostrare.

Se $m\in \RR$ per le proprietà caratterizzanti l'estremo inferiore
sappiamo che per ogni $\eps>0$ esiste $a\in A$ tale che
$f(a) < m + \eps$.
Per ogni $n\in\NN$ possiamo scegliere $\eps=1/n$ e ottenere quindi
una successione $a_n$ tale che $f(a_n) < m + 1/n$.
D'altra parte essendo $m$ un minorante di $f(A)$ sappiamo che
$m \le f(a_n)$.
Abbiamo dunque $m \le f(a_n) < m+ 1/n$ e per il teorema dei
carabinieri possiamo quindi concludere che $f(a_n) \to m$
per $n\to +\infty$.
\end{proof}

\begin{theorem}[Weierstrass]%
\label{th:weierstrass}%
\mymark{***}%
\mymargin{teorema di Weierstrass}%
\index{teorema!di Weierstrass}%
\index{Weierstrass!teorema di}%
\mynote{vedi note storiche a pag~\pageref{note:isoperimetrico}}%
Siano $a,b\in \RR$, $a\le b$ e $f\colon [a,b]\to \RR$ una funzione continua.
Allora esistono punti di massimo e di minimo per $f$ su $[a,b]$.
\end{theorem}
%
\begin{proof}
\mymark{***}%
Dimostriamo solamente che $f$ ha minimo, per il massimo la dimostrazione procede
infatti in maniera del tutto analoga.

Sia $m= \inf f([a,b])$.
Per il lemma precedente sappiamo che esiste una successione $a_n$ minimizzante ovvero tale che
$a_n \in A$ e $f(a_n)\to m$ per $n\to +\infty$.

Per il teorema di Bolzano-Weierstrass dalla successione $a_n$ possiamo estrarre una sottosuccessione 
$a_{n_k}$ convergente: $a_{n_k} \to x_0$.
Visto che $a_{n_k} \in [a,b]$ si avrà, per il teorema della permanenza del segno, anche 
$x_0 \in [a,b]$ (si applichi la permanenza del segno alle successioni $a_{n_k}-a$ e $b-a_{n_k}$).

Dunque abbiamo una successione $a_{n_k}\to x_0$ con $a_{n_k}\in [a,b]$ e
$x_0 \in [a,b]$. Essendo $f$ continua si avrà dunque $f(a_{n_k}) \to f(x_0)$.
Ma noi sapevamo che $f(a_n)\to m$ e dunque anche $f(a_{n_k}) \to m$.
Concludiamo quindi che $f(x_0) = m$ cioè $m$, l'estremo inferiore,
è un valore assunto dalla funzione ed è quindi un minimo.
Dal canto suo $x_0$ è un punto di minimo assoluto.
\end{proof}

\begin{corollary}[limitatezza delle funzioni continue]
Sia $f\colon [a,b]\to \RR$ una funzione continua. Allora $f$ è limitata.
\end{corollary}
\begin{proof}
Visto che $f$ ha massimo $M$ e minimo $m$ si ha $f(x)\in [m,M]$ per ogni $x\in[a,b]$.
Ovviamente $m>-\infty$ e $M<+\infty$ in quanto $m$ e $M$ sono valori della funzione $f$.
\end{proof}

\begin{exercise}
In ognuno dei seguenti casi decidere se esiste o meno una funzione 
con le caratteristiche indicate:
\begin{enumerate}
  \item $f\colon \RR\to (0,1)$ continua e bigettiva;
  \item $f\colon \RR\to [0,1]$ continua e surgettiva;
  \item $f\colon [0,1]\to \RR$ continua e surgettiva;
  \item $f\colon (0,1]\to \RR$ continua e surgettiva;
  \item $f\colon (0,1]\to \RR$ bigettiva. 
\end{enumerate}
\end{exercise}
%%%%%%%%%%%
%%%%%%%%%%%
