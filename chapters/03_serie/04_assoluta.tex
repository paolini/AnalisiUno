\section{convergenza assoluta}

Per le serie a termini positivi abbiamo molti criteri di convergenza
che invece, in generale, non si applicano alle serie di segno qualunque
o alle serie di numeri complessi.
La convergenza di queste ultime, però, può a volte ricondursi
facilmente
alla
convergenza delle serie a termini positivi, passando al modulo
ogni termine.

\begin{definition}[convergenza assoluta]
\mymark{***}
Diremo che una serie (a termini reali o complessi) $\sum a_n$
è \emph{assolutamente convergente}%
\mymargin{assolutamente convergente}%
\index{assolutamente!convergente} se la serie $\sum \abs{a_n}$
è convergente.
\end{definition}

\begin{theorem}[convergenza assoluta]\label{th:convergenza_assoluta}
\mymark{***}%
Se una serie $\sum a_n$ (reale o complessa)
è assolutamente convergente allora è convergente e vale
\[
  \abs{\sum_{k=0}^{+\infty} a_k} \le \sum_{k=0}^{+\infty} \abs{a_k}.
\]
\end{theorem}
%
\begin{proof}
\mymark{*}
Supponiamo inizialmente che gli $a_n$ siano numeri reali.
Definiamo $a_n^+ = \max\ENCLOSE{0, a_n}$ e $a_n^- = -\min \ENCLOSE{0, a_n}$.
Cioè se $a_n\ge 0$ si ha $a_n^+ = a_n$ e $a_n^-=0$ se invece $a_n\le 0$
si ha $a_n^+ =0$ e $a_n^- = -a_n$.
Dunque $a_n^+\ge 0$, $a_n^-\ge 0$,
\[
   a_n = a_n^+  - a_n^-
   \qquad\text{e}\qquad
   \abs{a_n} = a_n^+ + a_n^-.
\]
Allora se $\sum \abs{a_n}$ converge,
per confronto anche $\sum a_n^+$ e $\sum a_n^-$ convergono.
Dunque, per il teorema sulla somma dei limiti,
$\sum a_n = \sum a_n^+ - \sum a_n^-$
e quindi anche $\sum a_n$ converge.

Se abbiamo una successione di complessi $a_n = x_n + i y_n$
e se
$\sum \abs{a_n}$ converge allora, per confronto,
anche $\sum \abs{x_n}$ e $\sum\abs{y_n}$ convergono
(si osservi infatti che $\abs{x} \le \abs{x+iy}$ e $\abs{y}\le \abs{x+iy}$).
Dunque $\sum x_n$ e $\sum y_n$ convergono per quanto
già dimostrato sulle serie a termini reali.
Ma allora anche $\sum i y_n$ e $\sum a_n = \sum (x + iy_n)$ convergono.

Poniamo ora
\[
  S_n  = \sum_{k=0}^n a_k.
\]
Per la subadditività
del modulo sappiamo che per le somme finite si ha
\[
 \abs{S_n} =\abs{\sum_{k=0}^n a_k}
 \le \sum_{k=0}^n \abs{a_k} \le \sum_{k=0}^{+\infty} \abs{a_k}.
\]
E per continuità del modulo, posto $S= \lim S_n$ si ha
\[
  \abs{\sum_{k=0}^{+\infty} a_k}
  = \abs{S}
  = \lim_{n\to +\infty} \abs{S_n}
  \le \sum_{k=0}^{+\infty} \abs{a_k}.
\]
\end{proof}

\begin{theorem}[scambio delle serie]
\label{th:scambio_somma}
Sia $a_{k,j}\in \RR$ o $a_{k,j}\in \CC$ una successione a due indici $k\in \NN$, $j\in \NN$.
Se 
\begin{equation}\label{eq:499655}
  \sum_{k=0}^{+\infty}\sum_{j=0}^{+\infty}\abs{a_{k,j}}<+\infty  
  \qquad \text{oppure} \qquad
  \sum_{j=0}^{+\infty}\sum_{k=0}^{+\infty}\abs{a_{k,j}}<+\infty  
\end{equation}
allora
\begin{equation}\label{eq:scambio_somma}
  \sum_{k=0}^{+\infty} \sum_{j=0}^{+\infty} a_{k,j}
  = \sum_{j=0}^{+\infty} \sum_{k=0}^{+\infty} a_{k,j}.
\end{equation}
\end{theorem}
%
\begin{proof}
Visto che l'enunciato è simmetrico in $k$ e $j$ 
possiamo supporre che sia soddisfatta la prima 
delle due alternative nell'ipotesi~\eqref{eq:499655}:
$\sum_k \sum_j \abs{a_{k,j}}<+\infty$.

Si intende, ovviamente, che per ogni $k$ 
la serie $\sum_j \abs{a_{k,j}}$ è convergente
dunque $\sum_j a_{k,j}$ è assolutamente convergente.
% e, per il teorema~\ref{th:convergenza_assoluta}, 
% anche $\sum_k \abs{\sum_j a_{k,j}} 
% \le \sum_k \sum_j \abs{a_{k,j}}$
% è assolutamente convergente.
Ovviamente 
$\sum_k \abs{a_{k,j}} \le \sum_k \sum_j \abs{a_{k,j}}$
e dunque per ogni $j$ anche la serie 
$\sum_k a_{k,j}$ è assolutamente convergente.
Posto
\[
S = \sum_{k=0}^{+\infty}\sum_{j=0}^{+\infty} a_{k,j},
\qquad
S_n = \sum_{j=0}^{n-1}\sum_{k=0}^{+\infty} a_{k,j}  
\]
dobbiamo dimostrare che $S_n \to S$
per $n\to +\infty$. 
Applichiamo la definizione di limite:
sia $\eps>0$ fissato.
Per il teorema~\ref{th:coda} della coda 
esiste $K$ tale che 
\[
  \sum_{k=K}^{+\infty} \sum_{j=0}^{+\infty} \abs{a_{k,j}}<\frac \eps 2
\]
e per ogni $k<K$ esiste $J_k$ tale che 
\[
  \sum_{j=J_k}^{+\infty} \abs{a_{k,j}} < \frac{\eps}{2K}.  
\]
Dunque posto $J=\max\ENCLOSE{J_0,J_1, \dots, J_{K-1}}$
se $n>J$ si ha:
\begin{align*}
  \abs{S-S_n}
  &= \abs{\sum_{k=0}^{+\infty}\sum_{j=0}^{+\infty}a_{k,j} 
  - \sum_{j=0}^{n-1}\sum_{k=0}^{+\infty}a_{k,j}}
  = 
  \abs{\sum_{k=0}^{+\infty}\sum_{j=0}^{+\infty}a_{k,j} 
  - \sum_{k=0}^{+\infty}\sum_{j=0}^{n-1}a_{k,j}}\\
  &=
  \abs{\sum_{k=0}^{+\infty}\sum_{j=n}^{+\infty}a_{k,j} }
  = \abs{\sum_{k=0}^{K-1}\sum_{j=n}^{+\infty} a_{k,j}
   + \sum_{k=K}^{+\infty}\sum_{j=n}^{+\infty} a_{k,j}}\\
   &\le \sum_{k=0}^{K-1}\sum_{j=n}^{+\infty} \abs{a_{k,j}}
    + \sum_{k=K}^{+\infty}\sum_{j=0}^{+\infty} \abs{a_{k,j}}
   \le \sum_{k=0}^{K-1}\sum_{j=J_k}^{+\infty} \abs{a_{k,j}}
   + \frac \eps 2
  \le \frac \eps 2 + \frac \eps 2 = \eps
\end{align*}
come volevamo dimostrare.
\end{proof}

\begin{theorem}[convergenza incondizionata]%
\label{th:convergenza_incondizionata}%
\mymark{*}%
\index{convergenza!incondizionata}%
Se $\sum a_k$ è una serie assolutamente convergente e $\sigma\colon \NN \to \NN$
è una qualunque funzione biettiva (permutazione dei numeri naturali)
si ha
\[
  \sum_{k=0}^{+\infty} a_k
  = \sum_{j=0}^{+\infty} a_{\sigma(j)}.
\]
\end{theorem}
\begin{proof}
Definiamo la successione $b_{k,j}$ a due indici:
\[
  b_{k,j} = \begin{cases}
    a_k & \text{se $k=\sigma(j)$},\\
    0 & \text{altrimenti}.
  \end{cases}  
\] 
Chiaramente $\sum_j \abs{b_{k,j}} = \abs{a_k}$
e dunque $\sum_k \sum_j \abs{b_{k,j}} = \sum_k \abs{a_k} < +\infty$.
Dunque, per il teorema~\ref{th:scambio_somma}
\[
 \sum_{k=0}^{+\infty} a_k 
 = \sum_{k=0}^{+\infty} \sum_{j=0}^{+\infty}b_{k,j}
 = \sum_{j=0}^{+\infty} \sum_{k=0}^{+\infty}b_{k,j}
 = \sum_{j=0}^{+\infty} a_{\sigma(j)}.  
\]
\end{proof}

\begin{theorem}[somme alla Cauchy]
  \label{th:somma_Cauchy}%
Sia $a_{k,j}\in \RR$ o $a_{k,j}\in \CC$ una successione a due indici $k\in \NN$, $j\in\NN$.
Se 
\[
  \sum_{k=0}^{+\infty} \sum_{j=0}^{+\infty} \abs{a_{k,j}} < +\infty
  \qquad\text{oppure}\qquad 
  \sum_{n=0}^{+\infty} \sum_{k=0}^n \abs{a_{k,n-k}} < +\infty
\]  
allora 
\begin{equation}\label{eq:somma_Cauchy}
  \sum_{k=0}^{+\infty} \sum_{j=0}^{+\infty} a_{k,j}
   = \sum_{n=0}^{+\infty} \sum_{k=0}^{n} a_{k,n-k}.
\end{equation}
\end{theorem}
%
\begin{proof}
Poniamo 
\[
  b_{k,n} = \begin{cases}
    a_{k,n-k} & \text{se $k\le n$}\\
    0 & \text{se $k>n$}.
  \end{cases}  
\]
Osserviamo allora che 
\begin{align*}
  \sum_{k=0}^{+\infty} \sum_{n=0}^{+\infty} b_{k,n}
  &= \sum_{k=0}^{+\infty} \sum_{n=k}^{+\infty} a_{k,n-k} 
  = \sum_{k=0}^{+\infty} \sum_{j=0}^{+\infty} a_{k,j}
  \\
  \sum_{n=0}^{+\infty}\sum_{k=0}^{+\infty} b_{k,n} 
  &= \sum_{n=0}^{+\infty}\sum_{k=0}^{n} a_{k,n-k}
\end{align*}
e analogamente
\begin{align*}
  \sum_{k=0}^{+\infty} \sum_{n=0}^{+\infty} \abs{b_{k,n}}
  &= \sum_{k=0}^{+\infty} \sum_{j=0}^{+\infty} \abs{a_{k,j}}
  \\
  \sum_{n=0}^{+\infty} \sum_{k=0}^{+\infty} \abs{b_{k,n}}
  &= \sum_{n=0}^{+\infty} \sum_{k=0}^{n} \abs{a_{k,n-k}}.  
\end{align*}
Dunque si può applicare il teorema~\ref{th:scambio_somma}
per scambiare le somme delle serie con termini $b_{k,n}$
per ottenere il risultato:
\[
  \sum_{k=0}^{+\infty} \sum_{j=0}^{+\infty} a_{k,j}
  = \sum_{k=0}^{+\infty}\sum_{n=0}^{+\infty} b_{k,n} 
  = \sum_{n=0}^{+\infty} \sum_{k=0}^{+\infty} b_{k,n}
  = \sum_{n=0}^{+\infty} \sum_{k=0}^{n} a_{k,n-k}.
\]
%da cui~\eqref{eq:somma_Cauchy}.
\end{proof}

%%%%%%%%%%%
%%%%%%%%%%%
