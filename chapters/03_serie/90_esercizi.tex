\section{esercizi}

\begin{exercise}
  Utilizzando il teorema~\ref{th:approx_e} dimostrare che
  \[
  \lim_{n\to +\infty} n \cdot \sin(2\pi \cdot e\cdot n!) = 2\pi.
  \]
\end{exercise}

\begin{exercise}
Determinare il carattere delle seguenti serie
\[
    \sum_n \frac{n^2-n^3}{3^n}, \qquad
    \sum_n \frac{(n!)^2}{(2n)!}
\]
\[
\sum_n \frac{(-1)^n}{\ln\abs{n^7 - 10n^5 + 3}},  \qquad
\sum_n \frac{n-10}{n^2+10}
\]
\end{exercise}
  
\begin{exercise}
  Determinare il carattere delle seguenti serie
  \[
    \sum_n \enclose{\frac 1 n - \sin \frac 1 n},\qquad
    \sum_n \sin\enclose{\pi n + \frac 1 n}
  \]
\end{exercise}
  
\begin{exercise}
Calcolare la somma della serie:
\[
  \sum_{n=2}^{+\infty} \ln\enclose{1+\frac{(-1)^n}{n}}.
\]
(suggerimento: raggruppare i termini 2 a 2)
\end{exercise}
  
