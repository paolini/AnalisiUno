\section{prodotti infiniti}
%%%
\index{prodotti infiniti}

Così come abbiamo fatto la teoria per le somme infinite si potrebbe fare
la teoria dei prodotti infiniti ponendo
\[
  \prod_{k=0}^{+\infty} a_k = \lim_{n\to +\infty} \prod_{k=0}^n a_k
  = \lim_{n\to +\infty} a_0 \cdot a_1 \cdots a_n.
\]

Supporremo sempre $a_k>0$ altrimenti il segno del prodotto difficilmente
sarebbe definito.
Allora, utilizzando il logaritmo (che trasforma prodotti in somme) possiamo
ricondurre i prodotti infiniti
alle serie:
\[
  \prod_{k=0}^{+\infty} a_k = \lim_{n\to +\infty} e^{\sum_{k=0}^n \ln a_k}.
\]

Osserviamo che se la serie dei logaritmi diverge a $-\infty$ il prodotto infinito
ha limite $0$. Avendo richiesto che i termini $a_k$ siano tutti positivi il prodotto
non potrà mai essere minore di zero. Per mantenere l'analogia con le serie diremo
che il prodotto infinito converge se il limite dei prodotti parziali è finito e positivo.
Diremo che diverge se il limite è $+\infty$ oppure $0$.

Dunque potremo dire che il prodotto infinito converge se e solo se la serie dei logaritmi converge.

Osserviamo quindi che condizione necessaria affinché un prodotto infinito
$\prod a_k$ sia convergente
dovrà essere $\ln a_k\to 0$ ovvero $a_k \to 1$. In tal caso visto che
\[
  \ln a_k = \ln (1+(a_k-1)) \sim a_k - 1
\]
si osserva che se $a_k\to 1$ e $a_k\ge 1$ il prodotto infinito $\prod a_k$ converge
se e solo se converge la serie $\sum (a_k-1)$.

\begin{example}[somma dei reciproci dei primi]
\index{primi!somma dei reciproci}%
\index{somma!dei reciproci dei primi}%
Possiamo utilizzare i prodotti infiniti per dimostrare che la somma
dei reciproci dei numeri primi è divergente.
Sia $p_k$ la successione dei numeri
primi ($p_1=2$, $p_2=3$, $p_3=5$, $\dots$ stiamo dando per scontato che i numeri
primi sono infiniti). Allora vogliamo dimostrare che
\begin{equation}\label{eq:489467523}
  \sum_{k=1}^{+\infty} \frac{1}{p_k} = +\infty.
\end{equation}

Questo risultato ha una certa rilevanza nell'ambito della teoria dei numeri
in quanto ci dice che $p_k$ non può andare all'infinito come una
potenza $k^\alpha$ con $\alpha>1$ in quanto la serie $\sum 1/k^\alpha$
è convergente.

Mostriamo quindi che vale~\eqref{eq:489467523}.
Si noti che per ogni $n\in \NN$ il termine $\frac 1 n$
può essere decomposto come il prodotto di
potenze dei reciproci dei numeri primi. 
L'idea è quella di considerare il prodotto:
\begin{align*}
  \prod_{k=1}^{N} \sum_{j=0}^{M} \frac{1}{p_k^j}
  &= 
  (1 + \frac 1 2 + \frac 1 {2^2} + \dots + \frac{1}{2^M}) \cdot
  (1 + \frac 1 3 + \frac 1 {3^2} + \dots + \frac{1}{3^M}) \cdot \\
  &\quad \cdot(1 + \frac 1 5 + \frac 1 {5^2} + \dots + \frac{1}{5^M}) \cdots
  (1 + \frac 1 {p_N} + \frac 1 {p_N^2} + \dots + \frac{1}{p_N^M})
\end{align*}
osservando che nello sviluppare tale prodotto si ottengono 
come addendi i reciproci dei prodotti di tutte le potenze dei numeri primi
$p_1,\dots, p_N$ con esponenti non superiori a $M$.
Dunque se $n < p_{N+1}$ e $n \le 2^M$ 
allora $n$ è uno di tali prodotti e dunque $\frac 1 n$ 
compare nello sviluppo di cui sopra.
Dunque possiamo affermare che 
\begin{equation}\label{eq:8834884}
  \prod_{k=1}^{+\infty} \frac{1}{1-\frac{1}{p_k}}
  = \prod_{k=1}^{+\infty} \sum_{j=0}^{+\infty} \frac{1}{p_k^j}
  \ge \prod_{k=1}^{N} \sum_{j=0}^{M} \frac{1}{p_k^j}
  \ge \sum_{n=1}^{K-1} \frac 1 n
\end{equation}
dove $K$ è il più piccolo tra $p_{N+1}$ e $2^M$. 
Visto che $K$ può essere reso arbitrariamente grande 
scegliendo $N$ e $M$ sufficientemente grandi, 
e visto che $\sum \frac 1 n =+\infty$ significa che il prodotto
al lato sinistro di \eqref{eq:8834884} deve essere 
esso stesso divergente.
Ricordiamo allora che tale prodotto ha lo stesso carattere
della seguente serie
\[
  \sum_{k=1}^{+\infty} \enclose{\frac{1}{1-\frac 1 {p_k}}-1}
  = \sum_{k=1}^{+\infty} \frac{\frac 1 {p_k}}{1-\frac{1}{p_k}}
\]
ma visto che $1/p_k\to 0$ si ha
\[
  \frac{\frac 1 {p_k}}{1-\frac{1}{p_k}}
  \sim \frac 1 {p_k}
\]
e dunque, per il criterio del confronto asintotico, la serie precedente ha
lo stesso carattere della serie
\[
\sum_{k=1}^{+\infty} \frac{1}{p_k}
\]
che quindi è divergente.
\end{example}

%%%
