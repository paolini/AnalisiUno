\section{cenni storici sulla successione armonica}


Ci si può chiedere come mai la successione
\[
 1, \frac 1 2, \frac 1 3, \frac 1 4, \frac 1 5, \dots
\]
venga chiamata \emph{armonica}, c'entra qualcosa con l'armonia musicale?

Secondo la tradizione, Pitagora un giorno si affacciò all'officina di un fabbro, 
da cui provenivano i suoni dei martelli che battevano sulle incudini. 
Notò che alcuni martelli producevano suoni tra loro armonici, mentre altri no. 
Incuriosito, studiò il fenomeno cercando di produrre suoni di diverse frequenze 
utilizzando delle corde con tensioni e lunghezze diverse. 
Scoprì così che suoni percepiti come armonici corrispondevano a frequenze 
\emph{commensurabili}, cioè che potevano essere espresse come rapporti di numeri interi.


Naturalmente, la sensazione di "armonicità" è soggettiva e prodotta dal nostro cervello, 
ma possiamo ipotizzare il motivo per cui una frequenza che è multiplo di un'altra venga percepita come armonica.
Consideriamo uno strumento a corda: pizzicando la corda, si induce 
una vibrazione che può essere descritta come una sovrapposizione di 
onde sinusoidali (si veda la teoria delle serie di Fourier, 
nel capitolo~\ref{sec:convergenza_integrale}). 
Poiché la corda è fissata agli estremi, una possibile oscillazione è quella 
in cui la corda assume la forma di una sinusoide 
con semiperiodo pari alla lunghezza della corda stessa. 
Tuttavia, sono possibili anche altre oscillazioni sovrapposte: 
ad esempio, un'oscillazione con periodo uguale alla lunghezza 
della corda, che crea un nodo (punto fisso) al centro. 
Se tocchiamo la corda a un terzo della sua lunghezza, forziamo 
la creazione di un nodo in quel punto, impedendo la formazione 
delle due onde fondamentali e permettendo di sentire solo le 
\emph{armoniche} corrispondenti a oscillazioni con almeno 
un nodo in quel punto.
In sostanza, il suono prodotto è una sovrapposizione di onde sinusoidali pure, 
con frequenze che sono multipli interi della frequenza fondamentale. 
Man mano che le frequenze aumentano, la loro intensità diminuisce rapidamente, 
altrimenti la somma delle sinusoidi (la serie) divergerebbe. 
La diversa intensità delle armoniche determina il timbro dello strumento musicale. 
Ad esempio, in un flauto, all'estremità aperta del tubo si ha un \emph{ventre} 
invece che un nodo, e si ottengono solo i multipli dispari della frequenza fondamentale.
Poiché è molto difficile ottenere un suono puro, la presenza di armoniche 
è onnipresente nei suoni naturali. 
È quindi naturale che due suoni con frequenze una multiplo dell'altra 
siano percepiti come armonici, mentre suoni con frequenze in rapporti 
lontani da numeri interi piccoli risultino dissonanti.


Sulla base di queste considerazioni, possiamo capire come vengono definite 
le note musicali nel sistema temperato occidentale. 
Scegliamo come riferimento il suono prodotto da una corda di una certa lunghezza, 
chiamato \emph{tonica}. 
Se riduciamo a metà la lunghezza della corda, otteniamo un suono con frequenza fondamentale doppia: 
questa nota si trova un'\emph{ottava} sopra la tonica 
(il motivo del nome ``ottava'' sarà chiarito più avanti). 
La nostra percezione ci dice che questo suono è più acuto del precedente, 
ma talmente simile che tendiamo a dargli lo stesso nome. 
Lo stesso accade se raddoppiamo la lunghezza della corda (dimezzando la frequenza): 
otteniamo la stessa nota, ma più grave, cioè un’ottava sotto la tonica.

La nostra percezione delle frequenze è quindi \emph{logaritmica} e non lineare: 
il passaggio da una frequenza a quella doppia è percepito come lo stesso intervallo, 
indipendentemente dalla frequenza di partenza. 
Se triplichiamo la frequenza otteniamo una nota \emph{diversa} 
ma molto assonante con la tonica, chiamata \emph{dominante}. 
L’intervallo tra tonica e dominante è detto \emph{quinta} 
(vedremo più avanti il motivo del nome). 
Poiché raddoppiare o dimezzare la frequenza non cambia la percezione della nota, 
la dominante può essere rappresentata dal rapporto $\frac 3 2$, intermedio tra 1 e 2. 
La relazione armonica tra tonica e dominante può essere invertita dividendo la frequenza per tre: 
riportando la frequenza $\frac 1 3$ nell’intervallo tra $1$ e $2$ 
(moltiplicando per quattro, cioè salendo di due ottave), 
si ottiene il rapporto $\frac 4 3$, chiamato \emph{sottodominante} 
perché inferiore a $\frac 3 2$.


Se la nota di partenza la chiamiamo Do (tonica), 
le note con frequenza doppia o metà saranno chiamate sempre Do, 
ma sulle ottave superiori e inferiori rispetto al Do di partenza. 
Le frequenze triple e un terzo saranno chiamate Sol e Fa rispettivamente. 
Nell’intervallo tra due Do consecutivi con frequenze $f$ e $2f$ 
troveremo un Fa di frequenza $\frac 4 3 f$ e un Sol 
di frequenza $\frac 3 2 f$.

Per comprendere meglio questa struttura, è utile passare alla 
scala logaritmica: i logaritmi dei rapporti tra frequenze 
diventano differenze dei logaritmi. 
Calcolando i logaritmi in base $2$ dei rapporti 
con la frequenza della nota Do di riferimento, i Do 
corrispondono a numeri interi: il Do di riferimento a $0$, 
l’ottava sopra (frequenza doppia) a $1$, la seconda ottava sopra 
corrisponde a $2$, l’ottava inferiore a $-1$ e così via. 
La frequenza tripla dei Sol, riportata nell’intervallo $[0,1]$, 
corrisponde a $\log_2\frac 3 2 \approx 0.585$, 
mentre quella un terzo dei Fa a $\log_2 \frac 4 3 \approx 0.415$.

\begin{figure}
    \begin{center}
        \setlength{\tabcolsep}{1mm}% riduce lo spazio tra le colonne a 1mm
        \begin{tabular}{@{}l|ccccccccccccc@{}}
            note   &   Do   &   Reb           &   Re     &    Mib        &   Mi          &    Fa    &  Solb/Fa\#                       &   Sol    &  Lab           &   La          &   Sib         &   Si            \\
            quinte &   $0$  &  $-5$           &  $+2$    &    $-3$       &  $+4$         &   $-1$   &  $-6/+6$                         &  $+1$    &  $-4$          &  $+3$         &  $-2$         &  $+5$           \\
            freq   &   $1$  &$\frac{256}{243}$&$\frac 98$&$\frac{32}{27}$&$\frac{81}{64}$&$\frac 43$&$\frac{1024}{729}/\frac{729}{512}$&$\frac 32$&$\frac{128}{81}$&$\frac{27}{16}$&$\frac{16}9$   &$\frac{243}{128}$\\
        $\log_2$   &\small 0&\small .075 &\small  .170& \small  .245& \small  .340 & \small  .415& \small  .490 / .510         &\small  .585&\small  .660&\small  .755  &\small  .830  &\small  .925    \\
              temp &\small 0&\small .083 &\small  .167& \small  .250& \small  .333 & \small  .417& \small  .500                &\small  .583&\small  .667&\small  .750  &\small  .833  &\small  .917    \\
        \end{tabular}
    \end{center}
    \caption{Tabella delle note, dei rapporti di frequenza e dei logaritmi in base 2. 
    L’ultima riga mostra la suddivisione equabile (temperata) dell’ottava.}
\end{figure}
    

Possiamo ora completare la scala delle note musicali aggiungendo le frequenze 
triple e un terzo delle note Fa e Sol. 
Poiché l’intervallo tra Do e Sol e tra Fa e Do si chiama intervallo di quinta, 
la costruzione che stiamo descrivendo prende il nome di circolo delle quinte.
La frequenza tripla di una nota, nella scala logaritmica (a meno di ottave), 
corrisponde a sommare $0.585$ oppure sottrarre $0.415$. 
Salendo (di una quinta) da un Sol si ottiene una nota di frequenza $\frac 9 8$ 
rispetto al Do di riferimento, il cui logaritmo ha parte frazionaria $\approx 0.170$; 
chiamiamo $Re$ questa nota. Proseguendo in questo modo, moltiplicando e dividendo 
le frequenze per $3$, si ottengono note che si distribuiscono nell’intervallo 
delle frequenze tra $f$ e $2f$. 
Le frequenze multiple di $3$ partendo dal Do, danno, nell’ordine: 
Sol, Re, La, Mi, Si e Fa\#. 
Dividendo per 3 si ottengono: Fa, Sib, Mib, Lab, Reb e Solb.
Le frequenze delle note Fa\# e Solb sono molto vicine: 
$\frac{3^6}{2^9} = \frac{729}{512} \approx 1.42$ e 
$\frac{2^{10}}{3^6} = \frac{1024}{729} \approx 1.40$. 
I logaritmi in base due di queste frazioni sono $\log_2 \frac{729}{512} \approx 0.510$ 
e $\log_2 \frac{1024}{729} \approx 0.490$. 
Le due frequenze risultano praticamente indistinguibili e possiamo quindi 
identificare le due note come la stessa. 
La differenza tra queste frequenze si chiama \emph{comma musicale}.
Si ottengono così 12 note diverse, distribuite quasi uniformemente nell’intervallo 
tra la frequenza di base $f$ e la sua ottava $2f$. 
La correzione di questo piccolo errore può essere distribuita sulle diverse note 
in modo diverso, dando origine a diversi sistemi di temperamento.
La scala \emph{ben temperata} (che si è imposta dopo che Bach ne ha mostrato 
la versatilità nella composizione) distribuisce il comma in modo uniforme 
tra le note, ottenendo una scala in cui la frequenza tra due note consecutive 
(semitono) ha come rapporto esattamente la radice dodicesima di due: 
$\sqrt[12]{2}\approx 1.06$. 
In questo modo, i logaritmi delle frequenze suddividono l’intervallo $[0,1]$ 
in dodici parti uguali di ampiezza $\frac 1 {12} \approx 0.083$.


Abbiamo così spiegato il motivo per cui sono state scelte 12 diverse note musicali: 
in ultima analisi, 12 è il più piccolo numero per cui una potenza di 3 si avvicina 
molto ad una potenza di 2, dato che $3^{12}$ è molto vicino a $2^{19}$. 
La sequenza delle più piccole potenze di $3$ che meglio approssimano le potenze 
di $2$ (ordinate in base all'errore relativo) è: 
1, 2, 5, 12, 41, 53, 306, \dots. 
Dunque la scelta di 12 note è molto naturale 
(anche la scala pentatonica, con 5 note, è comune in molte culture musicali).

Abbiamo quindi visto come il rapporto $\frac 1 2$ determina il nome
delle note (note con frequenza doppia sono la stessa nota) e come il rapporto 
$\frac 1 3$ determina la scelta della scala musicale e il numero di note distinte.
Le armoniche successive sono legate all’\emph{armonia musicale}, 
cioè alla composizione di accordi e scale. 
Il rapporto $\frac 1 4$ non genera nuove note, poiché come $\frac 1 2$ 
rappresenta la stessa nota musicale. 
Il rapporto successivo è $\frac 1 5$, che approssimativamente corrisponde a un Mi 
(se, come sopra, abbiamo scelto il Do come nota di riferimento). 
Le frequenze $1,3,5$ generano l’accordo maggiore, formato dalle note Do, Sol, Mi. 
Componendo gli accordi maggiori sulla tonica (Do, Mi, Sol), 
sulla dominante (Sol, Si, Re) e sulla sottodominante (Fa, La, Do), 
si ottengono le sette note della scala maggiore: Do, Re, Mi, Fa, Sol, La, Si.
Per questo motivo l’intervallo tra Do e Sol si chiama \emph{quinta}, 
in quanto comprende cinque delle sette note, e l’intervallo tra Do e Fa 
si chiama \emph{quarta}. 
L’intervallo che comprende l’intera scala si chiama \emph{ottava}, 
perché comprende come ottava nota il Do superiore.

% ranking scale 
% sorted([(abs(1-2**round(log(3**n)/log(2))/3**n),n) for n in range(100)])[:10]
% [(0.0, 0), (0.0020859537426888286, 53), (0.009418849414340569, 94), (0.01152885180860852, 41), (0.013459631454855736, 12), (0.015517509028936116, 65), (0.023190618041241784, 82), (0.025329407756841116, 29), (0.026738101230810996, 24), (0.02876828053116498, 77)]
%
% 53 0.0020859537426888286
% 41 0.01152885180860852
% 12 0.013459631454855736
% 5 0.053497942386831365
% 2 0.11111111111111116
% 1 0.33333333333333326






