\section{somme su insiemi arbitrari}
\index{serie!su insiemi arbitrari}%
\index{somme!su insiemi arbitrari}%
%%%%%%%%%%

\begin{lemma}\label{lemma:12734}
  Sia $A$ un insieme numerabile e siano 
  $\alpha\colon \NN\to A$ e $\beta\colon \NN\to A$ 
  due funzioni bigettive.
  Sia $f\colon A\to \RR$ (o 
  $f\colon A \to \CC$) una funzione tale che 
  \begin{equation}\label{eq:12734}
    \sum_{k=0}^{+\infty} \abs{f(\alpha(k))} < +\infty,  
  \end{equation}
  oppure sia $f\colon A\to [0,+\infty)$ una funzione 
  qualunque. 
  Allora si ha
  \begin{equation}\label{eq:12735}
    \sum_{k=0}^{+\infty} f(\alpha(k)) =
    \sum_{j=0}^{+\infty} f(\beta(j)). 
  \end{equation}
\end{lemma}
%
\begin{proof}
Posto $a_k=f(\alpha (k))$ e $\sigma = \alpha^{-1}\circ \beta$ si ha 
$a_{\sigma(j)} = f(\beta (j))$.
Dunque se vale~\eqref{eq:12734} si può 
applicare il teorema~\ref{th:convergenza_incondizionata}
per ottenere~\eqref{eq:12735}.

Se invece $f(x)\ge 0$ si ha ovviamente $\abs{f(x)}=f(x)$
dunque se almeno una delle due somme in~\eqref{eq:12735}
è finita allora si applica il teorema~\ref{th:convergenza_incondizionata}
alla serie $\sum a_k$ e si ottiene dunque l'uguaglianza 
in~\eqref{eq:12735}.
Se invece entrambe le serie sono divergenti
(essendo serie a termini positivi non ci sono altre possibilità!)
l'uguaglianza~\eqref{eq:12735} è comunque valida
essendo $+\infty = +\infty$.
\end{proof}

\begin{definition}[somme arbitrarie]
Sia $f\colon A\to [0,+\infty]$ una funzione definita su un insieme 
$A$ numerabile (cioè $\#A \le \#\NN$).
Allora definiamo
  \[
    \sum_{x\in A}f(x)
    = \sum_{k=0}^{+\infty} f(\sigma(k)) 
  \]
dove $\sigma\colon \NN\to A$ è una qualunque funzione bigettiva
(il risultato non dipende da $\sigma$ grazie al lemma~\ref{lemma:12734}).

Se $f\colon A\to \CC$ o $f\colon A\to \RR$ (sempre con $A$ numerabile)
diremo che $f$ è sommabile se $\sum_{x\in A}\abs{f(x)}<+\infty$. 
In tal caso possiamo definire $\sum_{x\in A} f(x)$ come prima,
visto che anche in questo caso la somma non dipende
dall'ordine degli addendi.

Se $A=\ENCLOSE{a_0, \dots, a_{n-1}}$ è un insieme finito si può definire 
\[
  \sum_{x\in A} f(x) = \sum_{k=0}^{n-1} f(a_k)
\]
ed è chiaro che la somma non dipende dall'ordine degli addendi.
\end{definition}

\begin{theorem}[famiglie sommabili]
  Sia $A = \displaystyle\bigcup_{n\in \NN} A_n$ una unione di insiemi 
  disgiunti: $A_n\cap A_m=\emptyset$ se $n\neq m$.
  Supponiamo che $A$ sia numerabile (o finito).
  Sia $f\colon A \to \RR$ non negativa oppure $f\colon A \to \CC$ 
  una funzione sommabile (cioè $\displaystyle\sum_{x\in A}\abs{f(x)}< +\infty$).
  Allora 
  \[
    \sum_{x\in A}f(x) = \sum_{n\in \NN} \sum_{x\in A_n} f(x).  
  \]
\end{theorem}
%
\begin{proof}
Sia $x_k$ una numerazione degli elementi di $A$ e definiamo 
la successione a due indici:
\[
  a_{n,k} = \begin{cases}
    f(x_k) & \text{se $x_k\in A_n$}\\
    0 & \text{altrimenti.}
  \end{cases}  
\]
Per ogni $k\in \NN$ la seguente somma 
ha un solo addendo non nullo:
\[
 \sum_{n=0}^{+\infty}  a_{n,k} = f(x_k)
\]
mentre per ogni $n\in \NN$ si ha 
\[
 \sum_{k=0}^{+\infty} a_{n,k} = \sum_{x\in A_n} f(x)  
\]
in quanto i termini non nulli della successione 
$a_{n,k}$ sono uguali al valore di $f$ su una enumerazione 
degli elementi di $A_n$.

Allora si ha
\[
 \sum_{k=0}^{+\infty}\sum_{n=0}^{+\infty} \abs{a_{n,k}}
 = \sum_{k=0}^{+\infty} {f(x_k)} = \sum_{x\in A} \abs{f(x_k)}.
\]
Se quest'ultima somma è finita oppure se i termini sono tutti non negativi
possiamo allora scambiare le due somme grazie al teorema~\ref{th:scambio_somma}:
\[
  \sum_{x\in A} f(x) 
  = \sum_{k=0}^{+\infty} \sum_{n=0}^{+\infty} a_{n,k}
  = \sum_{n=0}^{+\infty} \sum_{k=0}^{+\infty} a_{n,k}
  = \sum_{n\in \NN} \sum_{x\in A_n}f(x).  
\]
\end{proof}

%%%
