Data una successione $a_n$ di numeri reali o complessi
possiamo considerare la successione
delle cosiddette \emph{somme parziali}%
\mymargin{somme parziali}%
\index{somme!parziali}
\[
  S_n = \sum_{k=0}^{n-1} a_k.
\]
\mynote{La somma parte da $k=0$ perché vogliamo utilizzare 
tutti i termini della successione, ma in molti casi 
la successione avrà come primo indice $k=1$ e quindi
sarà naturale partire da $k=1$. 
L'ultimo termine è $n-1$ in modo che $S_n$ sia la somma dei primi 
$n$ termini della successione.
Non cambierebbe niente se scegliessimo $n$ al posto di $n-1$}
Potremo scrivere più concisamente $S_n = \sum a_n$.
Intuitivamente si intende sommare i termini della successione $a_k$
per $k$ che parte da $0$ fino a $k=n-1$:
\[
  \sum_{k=0}^{n-1} a_k = a_0 + a_1 + a_2 + \dots + a_{n-1}.
\]
Formalmente la somma $S_n=\displaystyle \sum_{k=0}^n a_k$
è definita ricorsivamente
dalle seguenti relazioni:
\[
  \begin{cases}
    S_0 = 0, \\
    S_{n+1} = S_n + a_{n}.
  \end{cases}
\]

I numeri $a_n$ si chiamano \emph{termini}%
\mymargin{termini}%
\index{termine} della serie.
\index{termini!di una serie}\index{serie!termini}%
Se la successione delle somme parziali ammette limite il limite viene chiamato
\emph{somma}%
\mymargin{somma}%
\index{somma}
\index{somma!di una serie}\index{serie!somma}%
della serie e si indica con
\[
  \sum_{k=0}^{+\infty} a_k 
  = \lim_{n\to +\infty} S_n 
  = \lim_{n\to+\infty} \sum_{k=0}^{n-1} a_k.
\]

La terminologia già introdotta per le successioni si applica anche alle
serie che sono in effetti anch'esse delle successioni.
In particolare una serie può essere convergente, divergente o indeterminata.
Questo è il \emph{carattere della serie}.
\mymargin{carattere}%
\index{carattere!di una serie}%
\index{serie!carattere}%

Più in generale si potrà considerare la somma che parte da un certo
indice $m\in \ZZ$.
Fissato $m$ si potrà ad esempio considerare la serie:
\[
  S_n = \sum_{k=m}^{n-1} a_k
\]
che risulta definita per ogni $n\in\NN$, $n\ge m$
come
\[
  S_n = a_m + a_{m+1} + \dots + a_{n-1}.
\]

\begin{example}
Consideriamo la serie $S_n$ definita
come la somma dei numeri naturali da $1$ a $n$:
\[
  S_n 
  = \sum_{k=0}^{n-1} (k+1)
  = \sum_{k=1}^{n} k
  = 1 + 2 + \dots + n.
\]
Sappiamo che si ha (esercizio~\ref{ex:somma_lineare})
\[
  S_n = \frac{n(n+1)}{2}.
\]
E mediante la definizione di limite si può verificare che
risulta $S_n \to +\infty$.

Questo si esprime dicendo che la serie $\sum n$ è divergente:
\[
  \sum_{k=0}^{+\infty} (k+1)
  = \sum_{k=1}^{+\infty} k 
  = +\infty.
\]
\end{example}

\begin{example}[la serie geometrica]
Fissato $q \in \RR$ o $q\in \CC$ alla successione
$  a_n = q^n$
di termini
\[
  a_0 = 1,\qquad
  a_1 = q,\qquad
  a_2 = q^2,\qquad
  a_3 = q^3, \dots
\]
è associata la
\emph{serie geometrica}%
\mymargin{serie geometrica}%
\index{serie!geometrica}
$\sum q^n$
le cui somme parziali sono
\[
  S_0 = 0, \qquad
  S_1 = 1, \qquad
  S_2 = 1 + q, \qquad
  S_3 = 1 + q + q^2, \qquad
  \dots
\]
\end{example}

Il seguente teorema ci mostra come per diversi valori di $q$
la serie geometrica assume
tutti i possibili caratteri:
convergente, divergente, indeterminato.

\begin{theorem}[somma della serie geometrica]
\label{th:serie_geometrica}%
\mymark{***}%
\mymargin{somma della serie geometrica}%
\index{somma!della serie geometrica}%
Sia $q\in \CC$. Se $q\neq 1$ si ha
\[
 \sum_{k=0}^{n-1} q^k  = \frac{1-q^n}{1-q}.
\]
Se $\abs{q} < 1$ la serie geometrica converge:
\[
 \sum_{k=0}^{+\infty} q^k = \frac{1}{1-q}.
\]

Se $\abs{q}\ge 1$ la serie non converge
ma dobbiamo distinguere se la serie
si considera a valori reali o complessi
per determinarne il carattere.

Se consideriamo la serie a valori reali
(quindi $q\in \RR$) se $q\ge 1$ la serie
diverge a $+\infty$,
se $q\le -1$ la serie è indeterminata.

Se consideriamo la serie a valori complessi
$q\in \CC$, se $\abs{q}>1$ la serie diverge
a $\infty$, lo stesso succede se $q=1$.
Se $\abs{q}=1$ ma $q\neq 1$ la serie
è indeterminata.
\end{theorem}
%
\begin{proof}
Il primo risultato riguarda una somma finita.
Si ha
\[
  (1-q)\cdot \sum_{k=0}^{n-1} q^k
  = \sum_{k=0}^{n-1} q^k - q \cdot \sum_{k=0}^{n-1} q^k
  = \sum_{k=0}^{n-1} q^k - \sum_{k=1}^{n} q^k
  = 1 - q^n
\]
da cui si ottiene, se $q\neq 1$, il risultato voluto.

Passando al limite per $n\to +\infty$, se $\abs{q} < 1 $
si nota che $\abs{q^n} = \abs{q}^{n} \to 0$
e la serie converge dunque a $\frac 1{1-q}$.

Consideriamo ora la serie a valori reali.
Se $q>1$ osserviamo che
$q^n\to +\infty$ e quindi la serie diverge a $+\infty$
(infatti in questo caso $1-q$ è negativo).
Se $q=1$ si ha $q^k=1$ e quindi
$\displaystyle\sum_{k=0}^{n-1} q^k = n \to +\infty$.

Per gli altri casi osserviamo che si ha
\[
  \sum_{k=0}^{n-1} q^k = \frac{1-q^n}{1-q}
  = \frac{1}{1-q} - \frac{1}{1-q}\cdot q^k
\]
e dunque il carattere della serie coincide
con il carattere della successione $q^k$.
Se consideriamo la successione a valori
complessi e se $\abs{q}>1$ la serie è divergente
a $\infty$ in quanto
$\abs{q^k}=\abs{q}^k\to +\infty$.
Se invece consideriamo la serie a valori reali
e $q<-1$ (il caso $q\ge 1$ l'abbiamo già considerato)
la serie risulta indeterminata in quanto
in valore assoluto tende a $+\infty$ ma
il segno dei termini pari è opposto al segno dei termini
dispari.

Ci rimane da considerare il caso $\abs{q}=1$, $q\neq 1$.
Se $q$ è reale c'è solo il caso $q=-1$ e sappiamo
che $(-1)^k$ è indeterminata.
Anche se $q$ è complesso vogliamo dimostrare
che la successione $q^k$ è indeterminata:
se non lo fosse si avrebbe $q^k\to \ell$
con $\abs{\ell}=1$ in quanto $\abs{q^k}=1$
e passando al limite nell'uguaglianza
\[
  q^{k+2} = q\cdot q^{k+1}
\]
si avrebbe $\ell = q\cdot \ell$ da cui
(dividendo per $\ell$) otteniamo $q=1$.
Dunque se $q\neq 1$ la successione $q^k$
è indeterminata e così è la serie
geometrica.
\end{proof}

\begin{theorem}[linearità della somma]
\label{th:linearita_somma_serie}%
\mymargin{linearità della somma infinita}%
\index{linearità della somma infinita}
Se $\sum a_n$ e $\sum b_n$ sono convergenti
allora per ogni $\lambda,\mu\in \RR$ (o $\CC$)
anche $\sum (\lambda a_n + \mu b_n)$ è convergente
e si ha
\[
 \sum_{k=0}^{+\infty} (\lambda a_n + \mu b_n)
 = \lambda \sum_{k=0}^{+\infty} a_n  + \mu \sum_{k=0}^{+\infty} b_n.
\]
\end{theorem}
%
\begin{proof}
Se $S_n$ e $R_n$ sono le somme parziali delle serie $\sum a_n$ e $\sum b_n$
allora le somme parziali della serie $\sum (\lambda a_n + \mu b_n)$ sono
$\lambda S_n + \mu R_n$ (in quanto sulle somme finite vale la proprietà distributiva e commutativa). Ma se $S_n \to S$ e $R_n \to R$ allora
$\lambda S_n + \mu R_n \to \lambda S + \mu R$.
\end{proof}

Osserviamo che le serie (così come le successioni) formano uno spazio
vettoriale in cui le operazioni di somma e prodotto per scalare vengono
eseguite termine a termine: $\sum a_n + \sum b_n = \sum (a_n + b_n)$,
$\lambda \sum a_n = \sum (\lambda a_n)$.
Il teorema precedente ci dice allora che le serie (così come le successioni)
convergenti sono un sottospazio vettoriale e che la somma della serie (così come il limite della successione) è un'operatore lineare definito su tale sottospazio.

\begin{theorem}[condizione necessaria per la convergenza]
\mymark{***}
\mymargin{condizione necessaria}%
\index{condizione necessaria}
Se la serie $\sum a_n$ converge allora $a_n \to 0$.
\end{theorem}
%
\begin{proof}
\mymark{***}
Se la serie $\sum a_n$ converge significa che le somme parziali
$S_n = \displaystyle\sum_{k=0}^{n-1} a_k$ convergono: $S_n \to S$. Ma allora
\[
  a_n = S_n - S_{n-1} \to S - S = 0.
\]
\end{proof}

\begin{theorem}[stabilità del carattere]%
\index{carattere!di una serie}%
\index{stabilità!del carattere}%
\index{serie!stabilità del carattere}%
\index{serie!che differiscono su un numero finito di termini}%
Se le due successioni $a_n$ e $b_n$ differiscono solo su un numero finito
\mymargin{stabilità del carattere}%
di termini, allora le serie corrispondenti $\sum a_n$ e $\sum b_n$ hanno lo stesso carattere.
\end{theorem}
%
\begin{proof}
Se le successioni differiscono su un numero finito di termini significa
che esiste un $N\in \NN$ tale che per ogni $k>N$ si ha $a_k=b_k$.
Dunque se indichiamo con $S_n = \sum_{k=0}^{n-1} a_k$ e 
$R_n = \sum_{k=0}^{n-1} b_k$
le corrispondenti successioni delle somme parziali, si avrà per ogni $n>N$
\[
  S_n - R_n
    = \sum_{k=0}^{n-1} a_k - \sum_{k=0}^{n-1} b_k
    = \sum_{k=0}^{N-1} (a_k - b_k) 
    = C
\]
dove $C$ è una costante indipendente da $n$. 
Dunque, per $n>N$ si ha
\[
  S_n = R_n + C.
\]
Se il limite di $R_n$ non esiste allora non esiste neanche il limite
di $S_n$ (altrimenti essendo $R_n = S_n -C$ anche il limite di $R_n$ dovrebbe esistere). Se il limite di $R_n$ è infinito allora il limite di $S_n$ è uguale
al limite di $R_n$. E se il limite di $R_n$ è finito anche il limite di $S_n$ è finito.

Dunque il carattere della successione $S_n$ è lo stesso della successione $R_n$
cioè le due serie hanno lo stesso carattere.
\end{proof}

Se una serie ha
primo termine con un indice diverso da $0$
ci si potrà sempre ricondurre (con un cambio di variabile)
ad una serie il cui indice parte da zero. Ad esempio
(facendo il cambio di variabile $j=k-1$ da cui $j=0$ quando $k=1$
e ricordando che l'indice utilizzato nelle somme delle
serie è una variabile muta):
\[
 \sum_{k=1}^{+\infty} \frac{1}{2^k}
 = \sum_{j=0}^{+\infty} \frac{1}{2^{j+1}}
 = \sum_{k=0}^{+\infty} \frac{1}{2^{k+1}}.
\]

Si osservi inoltre che in base al teorema precedente quale sia il primo indice
da cui si comincia a sommare non è rilevante per quanto riguarda il carattere della serie.
Se però la serie è convergente la sua somma può variare, ad esempio:
\[
 \sum_{k=1}^{+\infty} \frac{1}{2^k}
 = \enclose{\sum_{k=0}^{+\infty} \frac{1}{2^k}} - 2^0.
\]

Nota bene: in molti libri si scrive $\infty$ al posto di $+\infty$.
Risulta quindi molto comune omettere il segno $+$ davanti a $\infty$
nella terminologia delle serie (e anche delle successioni) visto
che gli indici si intendono numeri naturali e quindi $-\infty$ non avrebbe
senso.

Ci sono però casi in cui può essere utile usare anche gli indici negativi,
ad esempio
se $a_k$ è definita per ogni $k\in \ZZ$
si potrebbe definire (ma non lo faremo):
\[
  \sum_{k=-\infty}^{+\infty} a_k
  = \sum_{k=0}^{+\infty} a_k +
  \sum_{k=1}^{+\infty} a_{(-k)}
\]
richiedendo che entrambe le serie al lato destro
dell'uguaglianza esistano e non abbiano somme infinite di segno opposto.

\begin{theorem}[coda di una serie convergente]
\label{th:coda}%
\mymark{*}%
Sia $\sum a_n$ una serie convergente. Allora
\[
  \lim_{n\to +\infty} \sum_{k=n}^{+\infty} a_k = 0.
\]
\end{theorem}
%
\begin{proof}
\mymark{*}
Posto
\[
  S_n = \sum_{k=0}^{n-1} a_k,
\]
per definizione di serie convergente sappiamo che esiste $S$ finito
tale che $S_n \to S$. Osserviamo allora che
\[
  \sum_{k=n}^{+\infty} a_k = \lim_{N\to+\infty} \sum_{k=n}^{N-1} a_k
   = \lim_{N\to +\infty} S_N - S_n = S - S_n
\]
e, per $n\to +\infty$ si ha ovviamente $S - S_n \to S - S = 0$.
\end{proof}

