\section{studio di funzione}

\begin{exercise}
\index{problema!della lattina}%
\index{lattina!problema della}%
Determinare raggio e altezza di una lattina cilindrica di volume $33 cl$
che a parità di volume ha la minima superficie totale.
\end{exercise}
\begin{proof}[Svolgimento.]
Sia $V$ il volume, $S$ l'area della superficie totale, $h$ l'altezza e $r$
il raggio di base del cilindro.
Sappiamo che
\[
  V = \pi h r^2, \qquad
  S = 2\pi r h + 2 \pi r^2.
\]
Ricavando $h$ dalla prima equazione e sostituendo nella seconda otteniamo
\[
  S = 2 \pi r \frac{V}{\pi r^2} + 2 \pi r^2
    = \frac{2V}{r} + 2 \pi r^2.
\]
La funzione $S(r)$ è definita e continua su $(0,+\infty)$
e si ha $S(r)\to +\infty$ per $r\to 0^+$
e anche per $r\to +\infty$. 
Dunque $S$ ammette minimo per il teorema di Weierstrass
generalizzato.
Per trovare il minimo basterà calcolare la derivata
\[
 \frac{dS}{dr} = -\frac{2V}{r^2} + 4 \pi r = \frac{4\pi r^3 - 2V}{r^2}
\]
e trovare i punti critici
\[
  4\pi r^3 = 2V
\]
da cui
\begin{align*}
 r = \sqrt[3]{\frac{V}{2\pi}} \approx 3.74 cm\\
 h = \frac{V}{\pi r^2} \approx 7.49 cm.
\end{align*}
\end{proof}

\begin{exercise}
Risolvere l'equazione
\begin{equation} \label{eq:4734521}
  e^x = x^3.
\end{equation}
\end{exercise}
%
\begin{proof}[Svolgimento.]
Il lato sinistro dell'equazione è un numero positivo e quindi
certamente possiamo supporre che sia $x>0$ altrimenti il lato destro non sarebbe anch'esso positivo.
Possiamo quindi prendere il logaritmo di ambo i membri
e ottenere l'equazione
\[
  x = 3 \ln x
\]
che è equivalente all'equazione data.

Consideriamo allora la funzione
\[
 f(x) = x - 3 \ln x
\]
cosicché le soluzioni cercate risultano essere gli zeri di $f$.
Si ha
\[
  f'(x) = 1 - \frac{3}{x}
\]
e possiamo quindi affermare che
\mynote{Il simbolo $\gtrless$ serve per indicare che questa
diseguazione e le seguenti possono essere scritte sia con il segno
$>$ che con il segno $<$ pur di prendere in tutte lo stesso segno.}%
$f'(x) \gtrless 0$
se e solo se
\[
  1 \gtrless \frac 3 x
\]
ovvero (ricordiamo che stiamo supponendo $x>0$)
\[
  x \gtrless 3.
\]
Cioè: se $x>3$ si ha $f'(x)>0$ e di conseguenza
(teorema~\ref{th:criteri_monotonia})
$f$ è strettamente crescente se ristretta all'intervallo
$[3,+\infty)$,
se invece $x<3$ si ha $f'(x)<0$ e di conseguenza
$f$ è strettamente decrescente se ristretta all'intervallo
$(0,3]$.
Per capire se la funzione $f$ si annulla in qualche punto
dobbiamo ora valutare $f$ negli estremi degli intervalli su cui risulta monotona. Si ha
\[
  \lim_{x\to 0^+} f(x) = +\infty, \qquad
  f(3) = 3 (1 - \ln 3) <0, \qquad
  \lim_{x\to +\infty} f(x) = +\infty.
\]
Osserviamo quindi che agli estremi degli intervalli
$(0,3]$ e $[3,+\infty)$ la funzione assume segni opposti e quindi, per il teorema degli zeri (teorema~\ref{th:zeri}), deve annullarsi almeno una volta in ognuno dei due intervalli. D'altra parte su ognuno dei due intervalli la funzione è strettamente monotona e quindi iniettiva, dunque non può annullarsi più di una volta. Deduciamo quindi che la funzione $f$ si annulla in esattamente due punti $x_1$, $x_2$ con
\[
  0 < x_1 < 3 < x_2.
\]

I punti $x_1$ e $x_2$ sono quindi univocamente determinati
e possono essere calcolati, con un errore piccolo a piacere,
utilizzando il metodo di bisezione, come abbiamo
visto nella dimostrazione del teorema degli zeri.
\end{proof}

\begin{example}
Si consideri la funzione
\[
  f(x) = \arctg x + \arctg\frac 1 x.
\]
Si ha
\[
  f'(x) = \frac{1}{1+ x^2} + \frac{1}{1 + \frac {1}{x^2}} \frac{-1}{x^2}
    = \frac {1}{1+x^2} - \frac{1}{x^2 + 1} = 0.
\]
Osserviamo che la funzione $f$ è definita su $\RR\setminus \ENCLOSE{0}$ che non è un intervallo ma è unione di due intervalli disgiunti: $(-\infty, 0) \cup (0, +\infty)$. Possiamo allora applicare i criteri di monotonia separatamente ai due intervalli ottenendo che $f(x)$ è costante su ognuno dei due intervalli. Dunque esisteranno $c_1$ e $c_2$ tali che
\[
  f(x) = \begin{cases} c_1 & \text{se $x>0$,} \\
  c_2 & \text{se $x<0$.}
  \end{cases}
\]
Possiamo determinare facilmente $c_1$ e $c_2$ osservando che
\begin{align*}
c_1 &= f(1) = \arctg 1 + \arctg 1 = \frac{\pi}{2} \\
c_2 &= f(-1) = \arctg (-1) + \arctg (-1) = - \frac{\pi}{2}.
\end{align*}
\end{example}
In effetti la funzione $f$ pur avendo derivata nulla non è costante ma solo \emph{localmente costante}.

\begin{example}
Consideriamo la funzione $f(x) = x^3$ la cui derivata è $f'(x) = 3x^2$.
Per ogni $x\in \RR$ si ha $f'(x)\ge 0$ dunque possiamo dedurre,
come già sappiamo, che $f$ è crescente.
Scelto invece $I = [0,+\infty)$ l'intervallo aperto corrispondente è $J=(0,+\infty)$. 
Osserviamo che su $J$ si ha $f'(x) > 0$ quindi possiamo concludere che $f$ 
è strettamente crescente su tutto $I$. 
Lo stesso vale per l'intervallo $(-\infty,0]$. 
Mettendo insieme le due cose possiamo concludere che $f(x) = x^3$ 
è strettamente crescente su tutto $\RR$
nonostante che sia $f'(0)=0$. 
Questo mostra che una funzione strettamente monotona può avere derivata 
nulla in un punto.
\end{example}

Più in generale è facile osservare che se $f$ è monotona ma non strettamente
monotona significa che ci sono due punti $a$ e $b$ per cui $f(a) = f(b)$.
Ma se $f$ è monotona allora per ogni $x\in [a,b]$ si deve avere
$f(x) = f(a) = f(b)$ (ad esempio: se $f$ è crescente si dovrebbe avere
$f(a) \le f(x) \le f(b)$ ma se $f(a)=f(b)$ necessariamente $f(x)=f(a)=f(b)$).
Dunque $f$ risulterebbe essere costante su $[a,b]$ e in particolare avremmo
una infinità più che numerabile di punti in cui la derivata si annulla.
Questo significa che se $f'(x)\ge 0$ su un intervallo e se $f'(x)=0$ su un
numero finito o anche numerabile di punti o, ancora, su un insieme di punti
con parte interna vuota, allora comunque $f$ è strettamente
crescente.
Ragionamento analogo vale naturalmente anche per le funzioni decrescenti.

\begin{exercise}
Dimostrare che
\[
  \cos x \ge 1- \frac{x^2}{2} \qquad \forall x \in \RR.
\]
\end{exercise}

\begin{exercise}
Si consideri la funzione $f\colon \RR \to \RR$ definita da
\[
  f(x) = 2e^{x-1} - x^2.
\]
Si mostri che $f$ è bigettiva e
che la funzione inversa $f^{-1}\colon \RR \to \RR$ è derivabile in
tutti i punti tranne il punto $1$ dove ha un flesso verticale.
Si calcoli $(f^{-1})'(2/e)$.
\end{exercise}
\begin{proof}[Svolgimento.]
Risulta
\[
  f'(x) = 2e^{x-1} - 2x, \qquad f''(x) = 2e^{x-1}-2.
\]
Dunque $f''(x) > 0$ per $x > 1$ e $f''(x)< 0$ per $x<1$
e per i criteri di monotonia
$f'$ è strettamente crescente su $[1,+\infty)$ e strettamente
decrescente su $(-\infty, 1]$. Visto che $f'(1)=0$ risulta quindi che
$f'(x)\ge 0$ per ogni $x\in \RR$ e $f'(x)=0$ solo per $x=1$.
Dunque $f$ è crescente per il criterio di monotonia ma anche
strettamente crescente perché se fosse crescente ma non strettamente
ci dovrebbe essere un intero intervallo in cui $f'$ si annulla.
Dunque $f$ è iniettiva. Visto che $f(x)\to \pm\infty$ per $x\to \pm\infty$
si ha $\sup f(\RR) = +\infty$, $\inf f(\RR) = -\infty$ e per il teorema dei valori intermedi
otteniamo che $f(\RR)=\RR$. 
Dunque $f$ è anche suriettiva ed è quindi invertibile.

La funzione inversa di $f$ è anch'essa monotona, 
è quindi continua per il Teorema~\ref{th:monotona_continua} ed 
è derivabile nei punti corrispondenti ai punti in cui $f$ ha derivata non nulla.
L'unico punto in cui $f$ ha derivata nulla è $x=1$ e visto che $f(1) = 1$ risulta
che $f^{-1}(y)$ è derivabile per ogni $y\neq 1$ e vale
\[
  (f^{-1})'(f(x)) = \frac{1}{f'(x)} = \frac{1}{2 e^{x-1}-2x}.
\]
Osservando che $f(0)=2/e$ si trova quindi
\[
  (f^{-1})'(2/e) = \frac{1}{f'(0)} = \frac{1}{2/e} = \frac e 2.
\]
\end{proof}

