\section{derivate parziali}

Capita a volte di avere funzioni che dipendono da più variabili o da parametri.
Ad esempio la funzione $f(x,y) = x^2y+y^2$ è una funzione definita 
sulle coppie di numeri reali: $f\colon \RR^2\to \RR$.
Se considero una delle due variabili, ad esempio la prima variabile $x$, 
come un parametro fissato, posso identificare la funzione $f$ con 
una funzione $g\colon \RR \to (\RR^\RR)$ 
che ad ogni valore del parametro $x$ restituisce una funzione 
della variabile $y$:
\[
  g(x)(y) = f(x,y).
\]
Dunque per ogni $x$ fissato, posso considerare la funzione $h=g(x)$
che è una funzione della sola variabile $y$: $h(y) = g(x)(y) = f(x,y)$.  
E posso quindi farne la derivata $h'(y)$ in qualunque punto $y$. 
Il valore che ottengo si chiama \emph{derivata parziale}%
\mymargin{derivata parziale}%
\index{derivata!parziale}
della funzione $f$ rispetto alla variabile $y$ calcolata nel punto $(x,y)$.
Per fare il calcolo della derivata possiamo applicare le usuali regole di 
derivazione facendo attenzione che se stiamo 
facendo la derivata rispetto a $y$ la variabile $x$ va trattata come se 
fosse costante, perché stiamo in effetti fissando $x$ prima di fare la derivata.
Se $f(x,y) = x^2y+y^2$ si ottiene dunque%
\mynote{Se non siamo abituati a pensare che $x$ possa essere un valore fissato 
si provi a sostituire la variabile $x$ con una lettera diversa come $c$ o $\pi$ che ci dà l'idea 
di una quantità costante.}:
$h'(y) = x^2 + 2y$.
Se ora consideriamo anche $x$ variabile otteniamo una nuova funzione di due 
variabili $(x,y)\mapsto (g(x))'(y) = h'(y)$ che può essere scritta 
con le seguenti notazioni% 
\mynote{%
Si osservi che per le funzioni di più variabili c'è una ambiguità di fondo nelle notazioni.
Infatti se ho una espressione in due variabili come $x^2 y + y^2$ è chiaro 
che questa rappresenta una funzione di due variabili ma non è del tutto ovvio dire quale 
è la prima variabile e quale è la seconda. 
Normalmente le variabili vengono indicate in ordine alfabetico a partire dalla $x$... 
ma è solo una convenzione. In principio l'espressione $x^2y + y^2$ 
potrebbe anche rappresentare una funzione di tre variabili $x,y,z$. 
Dunque se ho una espressione che utilizza certi nomi per le variabili è naturale 
usare gli stessi nomi nel simbolo utilizzato per la derivata parziale come 
nella notazione $D_y$.
Se invece le variabili non hanno un nome sarebbe più sensato indicare la variabile dandone 
la posizione numerica come nella notazione $D_2$.
Ci sono casi in cui la notazione può essere molto ambigua, come ad esempio 
se scrivessi $\frac{\partial f(y,x)}{\partial y}$: 
sto derivando $f$ rispetto alla prima variabile oppure derivo $f(x,y)$ rispetto alla seconda 
variabile e poi scambio le due variabili?
}:
\begin{align*}
\frac{\partial f(x,y)}{\partial y} 
&= \frac{\partial f}{\partial y}(x,y)
= D_y f(x,y)
= f_y (x,y) \\
&= D_2 f(x,y) = f_{,2} (x,y) \\
&= h'(y)
\end{align*}

Analogamente potremmo fare la derivata parziale di $f$ rispetto alla prima 
variabile $x$. Se $f(x,y) = x^2y+y^2$ si ottiene
\[
\frac{\partial f(x,y)}{\partial x} = 2xy.
\]

Stiamo qui trattando le funzioni di più variabili come se fossero funzioni di una sola 
variabile dipendenti da altri parametri. 
Questo perché stiamo facendo un corso di analisi sulle funzioni di \emph{una} variabile (reale).
\index{una!variabile}%
\index{funzione!di una variabile}%
\index{variabile!funzione di una}%
Per trattare tutte le variabili come una unica variabile multidimensionale 
bisogna fare diverse considerazioni 
aggiuntive che vengono trattate nei corsi di analisi sulle funzioni di 
\emph{più} 
\index{più!variabili}%
\index{funzione!di più variabili}%
\index{variabili!funzione di più}%
variabili (reali). 
Verranno in tal caso introdotti i concetti di \emph{gradiente} $\nabla f$, 
\emph{derivata} $Df$ e \emph{differenziale} $df$ che noi non affronteremo qui.

