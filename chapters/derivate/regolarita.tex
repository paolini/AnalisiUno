\section{classi di regolarità}

Sia $A\subset \RR$ e sia $f\colon A\to \RR$ una funzione definita su $A$.
Diremo che $f$ è di classe $C^0$ se $f$ è continua. 
Possiamo definire $C^0(A)$ come l'insieme delle funzioni continue definite 
su $A$ (a valori in $\RR$).
Se $f$ è derivabile (su tutto $A$) e la derivata di $f$ è anch'essa continua 
(su tutto $A$) diremo che 
$f$ è di classe $C^1$ e scriveremo $f\in C^1(A)$.
Andando avanti: se $f$ è derivabile e anche $f'$ è derivabile e se la derivata 
seconda $f''$ è continua, diremo che $f$ è di classe $C^2$. 
E così via.

E' importante osservare che $C^1(A)$ non coincide con l'insieme delle funzioni 
derivabili su $A$. 
Infatti abbiamo già visto nell'esempio~\ref{ex:derivata_non_continua} 
che esistono funzioni derivabili la cui derivata non è continua 
e quindi tali funzioni, pur essendo derivabili, non sono di classe $C^1$.


Per definire formalmente l'insieme delle funzioni di classe $C^k$ 
per ogni $k\in \NN$ dobbiamo dare una definizione per induzione.
Definiamo inoltre la classe $C^{\infty}(A)$ che è l'insieme delle funzioni 
che sono derivabili $k$ volte per ogni $k\in \NN$.
%
\begin{definition}[funzioni di classe $C^k$]
\mymark{**}
Dato $A\subset \RR$ denotiamo con $\RR^A$ l'insieme di tutte le funzioni $f\colon A \to \RR$.
Per ogni $k\in \NN$ definiamo
$C^k(A) \subset \RR^A$ nel modo seguente:
\mymargin{$C^k$}%
\index{$C^k$}
\begin{enumerate}
\item se $k=0$ poniamo
\[
  C^0(A) = \ENCLOSE{f\in \RR^A \colon \text{$f$ continua}}.
\]
\item
  se $k>0$ definiamo induttivamente
  \[
  C^{k}(A) = \ENCLOSE{f\in \RR^A \colon \text{$f$ derivabile e $f'\in C^{k-1}(A)$}}.
  \]
\end{enumerate}
Chiaramente se $j\ge k$ si ha $C^j(A) \subset C^k(A)$ dunque $C^k(A)$ è una famiglia decrescente (rispetto all'inclusione insiemistica) ed ha senso definire:
\mymargin{$C^\infty$}%
\index{$C^\infty$}
\[
  C^\infty(A) = \ENCLOSE{f\in \RR^A\colon \forall k \in \NN\colon f\in C^k(A)} = \bigcap_{k\in \NN} C^k(A).
\]

Le funzioni $f\in C^k(A)$ sono derivabili $k$ volte.
Utilizziamo la notazione $f^{(j)}$%
\mymargin{$f^{(j)}$}%
\index{$f^{(j)}$} per denotare la $j$-esima derivata
di una funzione $f$.
Dunque avremo
\[
  f^{(0)} = f, \qquad
  f^{(1)} = f', \qquad
  f^{(2)} = f'', \dots
\]
\end{definition}

Abbiamo già osservato che $\RR^A$ è uno spazio vettoriale reale.
Gli spazi $C^k(A)$ per $k=0, \dots, \infty$ sono una famiglia decrescente di sottospazi vettoriali di $\RR^A$.
Infatti sappiamo che la combinazione lineare di funzioni continue è continua e che la combinazione lineare di funzioni derivabili è derivabile.

\begin{theorem}
Le funzioni $x$, $e^x$, $\ln x$, $\sin x$, $\cos x$, $\tg x$, $\arctg x$ 
sono di classe $C^\infty$. 
Somma, differenza, prodotto, rapporto, e composizione di funzioni di classe $C^n$ 
sono anch'esse di classe $C^n$ per $n=0,1,2,\dots,\infty$.
\end{theorem}
\begin{proof}
Verifichiamo per induzione che se $f,g \in C^n$ allora $f+g \in C^n$.
Per $n=0$ sappiamo che la somma di funzioni continue è continua: $f,g\in C^0$ implica $f+g\in C^0$.
Per $n>1$ se $f,g\in C^n$ allora $f$ e $g$ sono derivabili e si ha 
$(f+g)'=f'+g'$ (teorema~\ref{th:derivate_operazioni}).
Ma per definizione di $C^n$ si ha $f',g'\in C^{n-1}$ e per ipotesi induttiva 
$f'+g'\in C^{n-1}$ dunque $f+g\in C^n$.

Dimostrazione analoga si può fare per il prodotto $f\cdot g$ osservando 
che $(f\cdot g)' = f'g + fg'$ e che se $f,g\in C^n$ allora $f',g'\in C^{n-1}$ 
dunque $f'g,fg'\in C^{n-1}$ e $f'g+fg'\in C^{n-1}$.

Stessa cosa per la composizione visto che $(f\circ g)' = (f'\circ g)\cdot g'$.

Dunque somma, prodotto e composizione di funzioni di classe $C^n$ sono di classe $C^n$
e lo stesso vale dunque per le funzioni di classe $C^\infty$.

Notiamo ora che le funzioni $-x$ e $\frac 1 x$ sono di classe $C^\infty$
(possiamo esplicitamente calcolarne le derivate di ordine qualunque) e quindi 
se $f\in C^n$ anche $-f$ e $\frac 1 f$ sono di classe $C^n$ per la proprietà della 
composizione. 
Ne segue che anche la differenza e il rapporto di funzioni di classe $C^n$   
è di classe $C^n$.

Possiamo quindi prendere in considerazione le funzioni elementari elencate nell'enunciato 
e osservare che sono tutte funzioni derivabili e che la loro derivata si scrive 
come somma, prodotto, rapporto o composizione delle stesse funzioni elementari.
Dunque se tutte queste funzioni sono di classe $C^n$ allora anche la loro derivata 
è di classe $C^n$. 
Ma se la derivata è di classe $C^n$ allora la funzione è di classe $C^{n+1}$
e dunque, per induzione, tutte le funzioni elementari sono di classe $C^\infty$.
\end{proof}

\begin{example}
  La funzione 
  \[
    \frac{e^{x-\arctg x^2}}{1+\ln x}
  \]
  è di classe $C^\infty$.
\end{example}

\begin{definition}[funzioni lipschitziane]
\mymark{**}
\mymargin{funzione lipschitziana}
\index{funzione!lipschitziana}
\index{Lipschitz}
Una funzione $f\colon A\subset \RR \to \RR$ si dice essere 
\emph{lipschitziana} (o anche $L$-lipschitziana se vogliamo mettere 
in evidenza la dipendenza da $L$) se esiste $L>0$ tale che
per ogni $x,y\in A$ si ha 
\[
  \abs{f(x) - f(y)} \le L\cdot  \abs{x-y}.
\]
La più piccola costante $L$ per la quale è soddisfatta la precedente relazione si chiama \emph{costante di Lipschitz}
\mymargin{costante di Lipschitz}
\index{costante!di Lipschitz}
di $f$.
\end{definition}

\begin{theorem}[criterio di Lipschitz]
\mymark{**}
Sia $f\colon A \subset \RR$ una funzione lipschitziana
con costante di Lipschitz $L$.
Se $f$ è derivabile in un punto $x\in A$ allora $\abs{f'(x)}\le L$.
Viceversa se $f\colon I \subset \RR \to \RR$ è una funzione derivabile 
definita su un intervallo $I$
e se esiste $L$ tale che per ogni $x\in I$ si ha $\abs{f'(x)}\le L$ 
allora $f$ è lipschitziana.
\end{theorem}
%
\begin{proof}
Se $f$ è Lipschitziana significa che il rapporto incrementale è limitato.
Cioè esiste $L>0$ tale che
\[
  \abs{\frac{f(x) - f(y)}{x-y}} \le L \qquad \forall x,y\in A.
\]
Dunque la derivata, che è il limite del rapporto incrementale, se esiste è anch'essa limitata
dalla stessa costante: $\abs{f'(x)}\le L$ per ogni $x \in A$.

Viceversa se la derivata è limitata $\abs{f'(z)}\le L$ per ogni $z \in I$ e se $x,y\in I$ sono punti qualunque,
allora, per il teorema di Lagrange, il rapporto incrementale di $f$ è uguale alla derivata in un punto $z\in(x,y)$:
\[
  \abs{\frac{f(x) - f(y)}{x-y}} = \abs{f'(z)} \le L.
\]
e dunque la funzione è $L$ lipschitziana:
\[
  \abs{f(x)- f(y)} \le L \abs{x-y}.
\]
\end{proof}

\begin{definition}[funzioni Hoelderiane]
Sia $\alpha>0$.
\mymargin{funzione hoelderiana}
\index{funzione!hoelderiana}
\index{Hoelder}
Una funzione $f\colon A \subset \RR \to \RR$ si dice essere $\alpha$-Hoelderiana se
esiste una costante $C>0$ tale che
\begin{equation}\label{eq:2964536}
  \abs{f(x) - f(y)} \le C \abs{x-y}^\alpha.
\end{equation}
\end{definition}

\begin{definition}[uniforme continuità]
\mymark{**}%
Una funzione $f\colon A \subset \RR \to \RR$ si dice essere
\emph{uniformemente continua}
\index{uniforme continuità}%
\index{continuità!uniforme}%
\mymargin{uniforme continuità}%
se
\[
 \forall \eps>0\colon \exists \delta > 0 \colon
 \forall x,y\in A \colon \abs{x-y} < \delta \implies \abs{f(x)-f(y)} < \eps.
\]
\end{definition}

\begin{theorem}
Ogni funzione lipschitziana è $1$-Hoelderiana (e viceversa).
Ogni funzione $\alpha$-Hoelderiana è uniformemente continua.
Ogni funzione uniformemente continua è continua.
Ogni funzione $\alpha$-Hoelderiana con $\alpha>1$ ha derivata nulla.
\end{theorem}
%
\begin{proof}
Le prime tre affermazioni seguono direttamente dalle definizioni.
Per l'ultima osservazione si noti che se $\alpha>1$ nella
disuguaglianza~\eqref{eq:2964536} si può dividere per $\abs{x-y}$ e
ottenere quindi che il rapporto incrementale tende a zero se $y\to x$.
\end{proof}

\begin{definition}[modulo di continuità]
Sia $f\colon A\subset \RR \to\RR$ una funzione.
Definiamo il \emph{modulo di continuità}%
\mymargin{modulo di continuità}%
\index{modulo!di continuità} di $f$ come la funzione
$M\colon$ $[0,+\infty) \to [0,+\infty]$ definita da
\[
  M(r) = \sup \ENCLOSE{\abs{f(x)-f(y)}\colon x,y \in A, \abs{x-y} \le r}.
\]
\end{definition}

\begin{theorem}[proprietà del modulo di continuità]
Sia $f\colon A\to \RR$ e sia $M\colon [0,+\infty)\to [0,+\infty)$ il suo
modulo di contintuità.
La funzione $M(r)$ è crescente.

La funzione $f$ è uniformemente continua se e solo se
\[
  \lim_{r\to 0} M(r) = 0.
\]

La funzione $f$ è lipschitziana se e solo se esiste $L$ tale che
\[
  M(r) \le Lr.
\]

La funzione $f$ è $\alpha$-Hoelderiana se e solo se esiste $C$ tale che
\[
  M(r) \le C r^\alpha.
\]
\end{theorem}
%
\begin{proof}
Osserviamo che la condizione
\[
   M(r) \le c
\]
è equivalente a
\[
\forall x,y\in A \colon \abs{x-y}\le r \implies  \abs{f(x)-f(y)} \le c.
\]
La condizione $M(r)\to 0$ per $r \to 0$ significa
\[
 \forall \eps>0\colon \exists \delta>0 \colon \forall r>0\colon r<\delta \implies M(r)<\eps
\]
e diventa quindi la condizione di uniforme continuità.

Le condizioni di lipschitzianità e di $\alpha$-hoelderianità risultano pure immediatamente.
\end{proof}

\begin{theorem}[restrizione / incollamento di funzioni uniformemente continue]
Sia $f\colon A \subset \RR \to \RR$ una funzione uniformemente continua. Se $B\subset A$ la restrizione $f_{|B}$ di $f$ a $B$ è anch'essa uniformemente continua.

Siano $I,J\subset \RR$ intervalli tali che $I\cap J \neq \emptyset$.
Sia $f\colon I \cup J \to \RR$ una funzione. Se $f_{|I}$ e $f_{|J}$ sono uniformemente continue allora $f$ è uniformemente continua.
\end{theorem}
\begin{proof}
La prima parte, sulla restrizione di una funzione uniformemente continua, deriva direttamente dalla definizione:
se una qualunque proprietà vale per ogni $x,y\in A$ allora a maggior ragione vale per ogni $x,y\in B$ quando $B\subset A$.

Vediamo la seconda parte dell'enunciato: supponiamo $f$ sia uniformemente continua su $I$ e su $J$.
Sia dato $\eps>0$ e siano $\delta_1$ e $\delta_2$ i valori di $\delta$ dati dalle condizioni di uniforme continuità di $f$ rispettivamente su $I$ e su $J$. Consideriamo $\delta = \min\ENCLOSE{\delta_1, \delta_2}$.
Siano ora $x,y$ punti qualunque di $I\cup J$ con $\abs{x-y}< \delta$.

Si possono allora avere due casi possibili:
$x$ e $y$ stanno nello stesso intervallo ($I$ o $J$) oppure stanno uno in $I$ e l'altro in $J$.
Nel primo caso essendo $\delta < \delta_1$ e $\delta< \delta_2$ l'uniforme continuità di $f$ su $I$ e su $J$ garantisce che valga in ogni caso $\abs{f(x)-f(y)} < \eps$. Nel secondo caso deve esistere un punto $z \in I\cap J$ che sia un punto intermedio tra $x$ e $y$.
Allora usando la disuguaglianza triangolare:
\[
  \abs{f(x) - f(y)} \le \abs{f(x) - f(z)} + \abs{f(z) - f(y)}
     \le \eps + \eps.
\]
si ottiene dunque (salvo rimpiazzare $\eps$ con $\eps/2$) anche in
questo caso la stima voluta.
\end{proof}

\begin{theorem}[Heine-Cantor]
  \label{th:heine_cantor}%
  \mymark{***}%
  \index{teorema!di Heine-Cantor}%
  \mymargin{Heine-Cantor}%
Siano $a,b\in \RR$, $a<b$.
Sia $f\colon [a,b]\to \RR$ una funzione continua.
Allora $f$ è uniformemente continua.
\end{theorem}
%
\begin{proof}
\mymark{***}%
Supponiamo per assurdo che $f$ non sia uniformemente continua. Allora
$f$ soddisfa la negazione della proprietà
\[
 \forall \eps>0\colon \exists \delta > 0 \colon
 \forall x,y\in [a,b] \colon \abs{x-y} < \delta \implies \abs{f(x)-f(y)} < \eps
\]
che è
\[
 \exists \eps>0\colon \forall \delta > 0 \colon
 \exists x,y \in [a,b] \colon \abs{x-y} < \delta \land \abs{f(x)-f(y)} \ge \eps.
\]
Dunque dato $\eps>0$ che soddisfa la precedente proprietà possiamo
prendere $\delta=1/k$
per ogni $k\in \NN$, ottenendo quindi due successioni $x_k$, $y_k$ tali che
\[
  \abs{x_k-y_k} < \frac 1 k \qquad\text{e}\qquad \abs{f(x_k) - f(y_k)} \ge \eps.
\]
Per il teorema di Bolzano-Weierstrass esisterà una sottosuccessione convergente $x_{k_j} \to c$.
Visto che $\abs{x_k-y_k}\to 0$ si dovrà avere anche $y_{k_j}\to c$.
Ma allora, per la continuità di $f$:
\[
 \abs{f(x_{k_j})-f(y_{k_j})} \to \abs{f(c) - f(c)} = 0
\]
in contraddizione con la condizione $\abs{f(x_k)-f(y_k)}\ge \eps$.
\end{proof}

\begin{theorem}[dell'asintoto]
\index{teorema!dell'asintoto}
\mymargin{teorema dell'asintoto}
Sia $f\colon [a,+\infty) \to \RR$ una funzione continua e sia $g\colon [a,+\infty) \to \RR$ una funzione uniformemente continua tale che
\[
  \lim_{x\to +\infty} f(x) - g(x) = 0.
\]
Allora anche $f$ è uniformemente continua.

In particolare se $f$ ha un \emph{asintoto obliquo}%
\mymargin{asintoto obliquo}%
\index{asintoto!obliquo} ovvero se esistono $m\in \RR$ e $q\in \RR$ tali che
\[
  \lim_{x\to +\infty} f(x) - (mx + q)  = 0
\]
allora $f$, se è continua, è uniformemente continua.

Risultato analogo vale per le funzioni definite su intervalli del tipo $(-\infty,a]$ (facendo i limiti a $-\infty$) e quindi per funzioni definite su tutto $\RR$ (facendo i limiti sia a $+\infty$ che a $-\infty$).
\end{theorem}

\begin{proof}
Per ogni $\eps>0$ esiste $M>a$ tale che $\abs{f(x)-g(x)} < \eps/3$ per ogni $x\ge M$. D'altra parte la funzione $f$, per il teorema di Heine-Cantor, è uniformemente continua su $[a,M+1]$ e dunque esiste $\delta_1>0$ tale che presi $x,y \in [a,M+1]$ con $\abs{x-y}< \delta_1$
si ha $\abs{f(x)-f(y)} < \eps$. D'altra parte $g$ è uniformemente continua su $[M,+\infty)$ e quindi esiste $\delta_2$ tale che dati $x,y\in [M,+\infty)$ con $\abs{x-y} < \delta_2$ si ha $\abs{g(x)-g(y)} < \eps/3$. Ma in quest'ultimo caso si ha:
\begin{align*}
  \abs{f(x)-f(y)} &\le \abs{f(x) - g(x)} + \abs{g(x) - g(y)} + \abs{g(y)-f(y)} \\
  & \le \frac{\eps}{3} + \frac{\eps}{3} + \frac{\eps}{3} = \eps.
\end{align*}
Posto dunque $\delta = \min\ENCLOSE{1, \delta_1, \delta_2}$
scelti comunque $x,y\in [a,+\infty)$ con $\abs{x-y}< \delta$ siamo certamente in uno dei due casi precedenti e quindi, in ogni caso, si ottiene $\abs{f(x)-f(y)} < \eps$, come dovevamo dimostrare.

Nel caso particolare $g(x) = mx +q$ si osserva semplicemente che $g$ è uniformemente continua in quanto è $L$-lipschitziana con $L=\abs{m}$.
\end{proof}

%%%%%%%%%%%%%%%%%%%%%%%%%%%%%%%%%%%%%%
%%%%%%%%%%%%%%%%%%%%%%%%%%%%%%%%%%%%%%
%%%%%%%%%%%%%%%%%%%%%%%%%%%%%%%%%%%%%%
%%%%%%%%%%%%%%%%%%%%%%%%%%%%%%%%%%%%%%

