\section{formula di Taylor}
\index{formula!di Taylor}


\begin{definition}[polinomio di Taylor]
\mymark{***}
Sia $A\subset \RR$, $f\colon A \to \RR$ una funzione
e sia $x_0\in A$ un punto in cui la derivata $n$-esima
(e quindi anche le derivate precedenti) sia definita.

Allora il polinomio:
\[
  P(x) = \sum_{k=0}^n \frac{f^{(k)}(x_0)}{k!}(x-x_0)^k
\]
viene detto \emph{polinomio di Taylor}%
\mymargin{polinomio di Taylor}%
\index{polinomio!di Taylor}
di ordine $n$
della funzione $f$,
centrato in $x_0$.
\end{definition}

Nel caso particolare in cui sia $x_0=0$ il polinomio di Taylor viene anche chiamato \emph{polinomio di MacLaurin}%
\mymargin{polinomio di MacLaurin}%
\index{polinomio!di MacLaurin}.

\begin{example}
  Il polinomio di Taylor di ordine $3$ per la funzione $f(x) = \sin x$ 
  centrato in $x_0=0$ è:
  \[
    P(x) = x - \frac{x^3}{6}.
  \]
  Infatti si ha $f(0) = \sin(0) = 0$, $f'(0) = \cos(0) = 1$, 
  $f''(0) = -\sin(0) = 0$ e $f'''(0) = -\cos(0) = -1$.
  Ovviamente $1!=1$ e $3!=6$. 
\end{example}

\begin{theorem}[caratterizzazione polinomio di Taylor]%
\label{th:caratterizzazioneTaylor}%
\mymark{***}%
Il polinomio di Taylor di una funzione $f$, di ordine $n$, centrato in $x_0$ è l'unico polinomio $P$ di grado non superiore ad $n$ tale che
\[
  P^{(k)}(x_0) = f^{(k)}(x_0), \qquad \forall k \in \ENCLOSE{0,\dots,n}.
\]
\end{theorem}
%
\begin{proof}
  Ogni polinomio di grado non superiore ad $n$ può essere scritto nella forma:
\[
  P(x) = \sum_{k=0}^{n} a_k (x-x_0)^k
\]
(basti notare che se $P(x)$ è un polinomio anche $Q(t) = P(x_0+t)$ è
un polinomio)
e le sue derivate sono:
\begin{align*}
   P^{(j)}(x)
   = \sum_{k=j}^n a_k \cdot \frac{k!}{(k-j)!}\cdot (x-x_0)^{k-j}.
\end{align*}
Per $x=x_0$ l'unico addendo non nullo è quello con $k=j$, dunque
\[
  P^{(k)}(x_0) = k! \cdot a_k.
\]
Dunque si ha $P^{(k)}(x_0) = f^{(k)}(x_0)$ se e solo se $a_k = f^{(k)}(x_0)/k!$ 
cioè se $P$ è il polinomio di Taylor di $f$.
\end{proof}

\begin{remark}[polinomio di Taylor della derivata]
  \label{rem:taylor_derivata}%
Se $P_n$ è il polinomio di Taylor centrato in $x_0$ di ordine $n$ per $f$, allora
si ha
\begin{align*}
P_n'(x) &= \sum_{k=1}^{n} \frac{f^{(k)}(x)}{k!}\cdot k(x-x_0)^{k-1} \\
  &= \sum_{k=1}^{n} \frac{f^{(k)}(x)}{(k-1)!} (x-x_0)^{k-1} \\
  &= \sum_{j=0}^{n-1} \frac{f^{(j+1)}(x)}{j!} (x-x_0)^j.
\end{align*}
Dunque si verifica che $P'_n$ è il polinomio di Taylor di ordine $n-1$ per $f'$.
In breve: il polinomio di Taylor di ordine $n-1$ della derivata è la derivata
del polinomio di Taylor di ordine $n$ della funzione.

Viceversa se
\[
   \sum_{k=0}^n a_k (x-x_0)^k
\]
è il polinomio di Taylor di ordine $n$ della derivata $f'(x)$, allora
il polinomio di Taylor di ordine $n+1$ di $f$ sarà%
\mynote{Questa è una \emph{antiderivata} (o \emph{primitiva}) del polinomio dato, 
si veda la definizione~\ref{def:primitiva}.}
\begin{equation}\label{eq:4993725}
  P_{n+1}(x) = f(x_0) + \sum_{k=0}^{n} \frac{a_{k}}{k+1}\cdot (x-x_0)^{k+1}
\end{equation}
in quanto
\[
  \frac{f^{(k+1)}(x_0)}{(k+1)!}
  =\frac{(f')^{(k)}(x_0)}{(k+1)!}
  = \frac{a_k \cdot k!}{ (k+1)!}
  = \frac{a_k}{k+1}.
\]
\end{remark}

\begin{theorem}[formula di Taylor con resto di Lagrange]
\label{th:taylor_lagrange}%
\mymark{**}%
\index{formula!di Taylor!con resto di Lagrange}%
\index{teorema!di Taylor con resto di Lagrange}%
\index{Taylor!resto di Lagrange}
\mymargin{Taylor con resto di Lagrange}
Sia $I\subset \RR$ un intervallo, $x_0\in I$
e $f\in C^{n+1}(I)$.
% \mynote{Si intende che $I$ non può essere vuoto, perché $x_0\in I$, 
% e $I$ non può essere un singoletto perché se $I=\ENCLOSE{x_0}$
% nessuna funzione può essere derivabile in $x_0$ 
% in quanto $x_0$ non è punto di accumulazione per $I$.
% }%
Sia $P$ il polinomio di Taylor di $f$ di ordine $n$ centrato in $x_0$.
Per ogni $x\in I\setminus\ENCLOSE{x_0}$ esiste $y\in (x_0,x)$
\mynote{%
in questa dimostrazione gli intervalli si considerano 
non orientati: $[a,b] = [b,a]$, $(a,b)=(b,a)$ se $a>b$.}
tale che
\[
  f(x) = P(x) + \frac{f^{(n+1)}(y)}{(n+1)!}(x-x_0)^{n+1}.
\]
\end{theorem}
%
\begin{proof}\mymark{*}%
La tesi si può scrivere nella forma 
\[
  \frac{f(x) - P(x)}{(x-x_0)^{n+1}} 
  = \frac{f^{(n+1)}(y)}{(n+1)!}.
\]

Procediamo con una dimostrazione per induzione su $n$. 
Per $n=0$ abbiamo che $f$ è derivabile (anzi è $C^1$, ma è una 
ipotesi eccessiva) e la tesi è
\begin{equation}\label{eq:655643}
  \frac{f(x) - f(x_0)}{x-x_0}  = f'(y)  
\end{equation}
che coincide con il teorema~\ref{th:lagrange} di Lagrange.

Supponiamo allora per induzione 
che il teorema sia vero con $n-1$ al posto 
di $n$ e consideriamo $f\in C^{n+1}$ con polinomio di Taylor $P$
di ordine $n$ centrato in $x_0$. 
La funzione $f'$ è in $C^{n}$ e il suo polinomio 
di Taylor di ordine $n-1$ è $P'$ (osservazione~\ref{rem:taylor_derivata}).
Per ottenere~\eqref{eq:655643} osserviamo ora che si ha,
essendo $P(x_0)=f(x_0)$:
\[
  \frac{f(x)-P(x)}{(x-x_0)^{n+1}}  
  = \frac{f(x) - P(x) - (f(x_0)-P(x_0))}{(x-x_0)^{n+1} - (x_0-x_0)^{n+1}}
\]
ed applicando il teorema~\ref{th:cauchy} di Cauchy
si trova che esiste $t\in (x_0,x)$ tale che
\[
  \frac{f(x)-P(x)}{(x-x_0)^{n+1}}  
  = \frac{f'(t)-P'(t)}{(n+1)(t-x_0)^n}. 
\]
Ora possiamo applicare l'ipotesi induttiva alla funzione 
$f'$ con polinomio di Taylor $P'$
e mettendo $t$ al posto di $x$ per ottenere 
l'esistenza di $y\in (x_0,t)\subset(x_0,x)$ tale che 
\[
  \frac{f'(t)-P'(t)}{(n+1)(t-x_0)^n}
  = \frac{(f')^{(n)}(y)}{(n+1)\cdot n!}.
\]
Osservando che $(f')^{(n)} = f^{(n+1)}$ e che $(n+1)\cdot n!=(n+1)!$ 
si ottiene la tesi.
\end{proof}

\begin{example}[somma della serie armonica a segni alterni]
Si ha:
  \[
    \sum_{k=1}^{+\infty} \frac{(-1)^{k-1}}{k} = \ln 2.
  \]
\end{example}
\begin{proof}
Consideriamo la funzione $f(x) = \ln x$.
Si osservi che
\begin{gather*}
  f'(x) = x^{-1},\qquad
  f''(x) = -x^{-2},\qquad
  f'''(x) = 2x^{-3}, \\
  f^{(4)}(x) = -3\cdot 2 x^{-4}, \qquad
  \dots. \qquad
  f^{(k)}(x) = \frac{(-1)^{k-1}(k-1)!}{x^k}
\end{gather*}
e si ha quindi, per ogni $k\ge 1$,
\[
  \frac{f^{(k)}(1)}{k!} = \frac{(-1)^{k-1}}{k}
\]
da cui si deduce che il polinomio di Taylor di ordine $n$
centrato in $x_0=1$ è
\[
P(x) = \sum_{k=1}^n (-1)^{k-1}\frac{(x-1)^k}{k}.
\]
La formula di Taylor con resto di Lagrange ci dice allora che per
ogni $x>1$ esiste $y\in(1,x)$ tale che
\[
  \ln(x) - \sum_{k=1}^n (-1)^{k-1}\frac{(x-1)^k}{k} 
  = \frac{(-1)^n(x-1)^{n+1}}{(n+1)y^{n+1}}.
\]
Se $x=2$ si ha $y>1$ e $(x-1)^{n+1}=1$ dunque
il lato destro tende a zero per $n\to +\infty$ 
(attenzione che $y$ dipende anch'esso da $n$). 
Ma per $x=2$ e $k\to +\infty$ il lato sinistro tende a 
\[
 \ln 2 - \sum_{k=1}^\infty \frac{(-1)^{k-1}}{k}
\]
e dunque si ottiene il risultato voluto.
\end{proof}

Nel teorema seguente utilizzeremo una nuova notazione. 
\begin{definition}[notazione $o$-piccolo]
Scriveremo
\mymargin{$o$-piccolo}%
\index{$o$-piccolo}%
\[
  f(x) = P(x) + o(g(x)), \qquad \text{per $x\to x_0$}
\]
(si legga: ``$f(x)$ è uguale a $P(x)$ più un $o$-piccolo di $g(x)$'')
se risulta
\[
  f(x) - P(x) \ll g(x)
  \qquad\text{ovvero}\qquad
  \lim_{x\to x_0} \frac{f(x)-P(x)}{g(x)} = 0.
\]
\end{definition}

La quantità $o(g(x))$ rappresenta una funzione incognita
che però sappiamo essere trascurabile (nel senso appena descritto)
rispetto alla funzione $g(x)$.
Questa notazione è molto comoda ma va utilizzata con cautela:
nel capitolo~\ref{sec:landau} vedremo come formalizzare rigorosamente 
questa notazione.

\begin{theorem}[formula di Taylor con resto di Peano]
\label{th:taylor_peano}%
\mymark{***}%
\index{formula!di Taylor!con resto di Peano}%
\index{teorema!di Taylor con resto di Peano}%
\index{Taylor!resto di Peano}%
Sia $I\subset \RR$ un intervallo, $x_0\in I$, 
e supponiamo che $f$ sia derivabile $n-1$ volte su tutto $I$
e che esista $f^{(n)}(x_0)$.
\mynote{In particolare le ipotesi sono soddisfatte se $f\in C^n(I)$.}%
Sia $P$ il polinomio di Taylor di $f$ di ordine $n>0$ centrato in $x_0$. 
Allora si ha
\begin{equation}\label{eq:taylor_peano}
  \lim_{x\to x_0}\frac{f(x) - P(x)}{(x-x_0)^n} = 0
\end{equation}
ovvero, scritto in maniera più espressiva:
\begin{equation}\label{eq:Taylor}
  f(x) = P(x) + o((x-x_0)^n).
\end{equation}

Viceversa se $P(x)$ è un polinomio qualunque, di grado minore o uguale a $n$, e vale la
formula~\eqref{eq:Taylor} allora $P$ è il polinomio di Taylor di $f$.
\end{theorem}
%
\begin{proof}
\mymark{**}%
\mynote{Anche qui usiamo la convenzione $[a,b] = [b,a]$ quando $b<a$.}
%
Si procede per induzione in modo simile alla dimostrazione 
del teorema~\ref{th:taylor_lagrange}.
%
Per $n=1$ si ha $P(x)=f(x_0) + f'(x_0)\cdot (x-x_0)$ e quindi 
\[
 \frac{f(x)-P(x)}{x-x_0} = \frac{f(x) - f(x_0)}{x-x_0} - f'(x_0) 
 \to f'(x_0) - f'(x_0) = 0  \qquad\text{per $x\to x_0$.}
\]

Supponiamo ora che sia $n>1$ e che il teorema sia vero
quando si pone $n-1$ al posto di $n$. Si ha 
\[
  \frac{f(x)-P(x)}{(x-x_0)^n} 
  = \frac{f(x)-P(x)-(f(x_0)-P(x_0))}{(x-x_0)^n - (x_0-x_0)^n}
\]
e, visto che $P(x_0) = f(x_0)$ possiamo applicare il teorema~\ref{th:cauchy} di Cauchy
alle funzioni $f(x)-P(x)$ e $(x-x_0)^n$,
per ottenere che esiste $t\in (x_0,x)$ tale che
\[
  \frac{f(x)-P(x)}{(x-x_0)^n} = \frac{f'(t)-P'(t)}{n\cdot (t-x_0)^{n-1}}.
\]
Applichiamo ora l'ipotesi induttiva alla funzione $f'(t)$ il cui 
polinomio di Taylor di ordine $n-1$ è proprio $P'(t)$. Si ottiene dunque 
che 
\[
 \lim_{t\to x_0} \frac{f'(t)-P'(t)}{(t-x_0)^{n-1}} = 0 
\]
e visto che per $x\to x_0$ essendo $t\in(x_0,x)$ si ha anche 
$t\to x_0$, si ottiene, come volevamo dimostare,
\[
 \lim_{x\to x_0} \frac{f(x)-P(x)}{(x-x_0)^n} 
 = \lim_{t\to x_0} \frac{f'(t)-P'(t)}{n\cdot (t-x_0)^{n-1}} = 0. 
\]

\emph{Implicazione inversa.}
Per completare la dimostrazione dobbiamo dimostrare che il polinomio di 
Taylor $P$ è l'unico polinomio di grado non superiore ad $n$ che soddisfa 
la condizione~\eqref{eq:taylor_peano}.
Se ci fosse un altro polinomio $Q$ con la stessa proprietà la differenza 
$R=(f-P)-(f-Q)=Q-P$ dovrebbe essere un polinomio di grado 
non superiore ad $n$ che soddisfa la condizione 
\begin{equation}\label{eq:496390}
  \lim_{x\to x_0}\frac{R(x)}{(x-x_0)^n}=0.  
\end{equation}
Posto 
\[
  R(x) = \sum_{k=0}^n a_k (x-x_0)^k  
\]
osserviamo che $R(x)\to R(x_0) = a_0$ per $x\to x_0$. 
Allora deve essere necessariamente $a_0 = 0$, 
altrimenti il limite~\eqref{eq:496390} sarebbe infinito.
Ma se $a_0=0$ possiamo raccogliere $x-x_0$ a numeratore 
e denominatore e da~\eqref{eq:496390} ottenere 
\[
  \lim_{x\to x_0}\frac{\displaystyle \sum_{k=1}^n a_k (x-x_0)^{k-1}}{(x-x_0)^{n-1}}=0.  
\]
Iterando il procedimento scopriamo che tutti i coefficienti $a_k$ 
devono essere nulli e dunque $Q=P$.
\end{proof}

\begin{exercise}
Calcolare:
\[
  \lim_{x\to 0} \frac{1-\cos(x^2)}{x^4}.
\]
\end{exercise}
\begin{proof}[Svolgimento.]
Il polinomio di Taylor di ordine $2$ per la funzione $f(x) = \cos x$ 
centrato in $x_0=0$ è 
\[ 
  P(x)
   = \cos 0 - \sin 0\cdot x -\frac{\cos 0}{2} \cdot x^2 
   = 1 - \frac{x^2}{2}.
\]
Il teorema~\ref{th:taylor_peano} (formula di Taylor con resto di Peano)
ci dice allora che si ha  
\begin{equation}\label{eq:498101}
   \lim_{x\to 0} \frac{\cos x - 1 + \frac {x^2}{2}}{x^2} = 0
\end{equation}
ovvero, utilizzando la notazione degli $o$-piccolo:
\begin{equation}\label{eq:498102}
  \cos x = 1 - \frac{x^2}{2} + o(x^2)
  \qquad \text{per $x\to 0.$}
\end{equation}
Sostituendo $x^2$ al posto di $x$ nel limite~\eqref{eq:498101} 
si ottiene di conseguenza:
\begin{equation}\label{eq:498103}
  \lim_{x\to 0}\frac{\cos x^2 -1 + \frac{x^4}{2}}{x^4} = 0
\end{equation}
che può essere scritta in maniera più espressiva come:
\[
  \cos (x^2) = 1 - \frac{x^4}{2} + o(x^4).
\]
Quest'ultima relazione può essere ottenuta direttamente 
da~\eqref{eq:498102} sostituendo $x^2$ al posto di $x$, 
come vedremo nel capitolo~\ref{sec:landau}.
Allora per $x\to 0$ si ha:
\[
  \frac{1-\cos(x^2)}{x^4} 
  = \frac{\frac{x^4}{2} - o(x^4)}{x^4} 
  = \frac 1 2 - \frac{o(x^4)}{x^4} \to \frac 1 2
\]
in quanto, per definizione di $o$-piccolo, si ha
\[
  \lim_{x\to 0 } \frac{o(x^4)}{x^4} = 0.
\]
\end{proof}

\begin{example}
Sia $f(x) = \cos(x^2)$. Calcolare $f''''(0)$.
\end{example}
\begin{proof}[Svolgimento.]
Abbiamo visto nell'esempio precedente che per $x\to 0$ si ha 
\[
  \cos(x^2) = 1 - \frac{x^4}{2} + o(x^4).
\]
Grazie alla seconda parte del teorema~\ref{th:taylor_peano} (formula di Taylor con resto di Peano)
sappiamo allora che il polinomio 
\[
  P(x) = 1 - \frac{x^4}{2}
\]
è il polinomio di Taylor di ordine $4$ centrato in $x_0=0$ per la funzione $f(x)$.
D'altra parte, per la definizione di polinomio di Taylor, si ha
\[
  P(x) = f(0) + f'(0) \cdot x + \frac{f''(0)}{2} \cdot x^2 
  + \frac{f'''(0)}{6} \cdot x^3 + \frac{f''''(0)}{24} \cdot x^4
\]
e quindi, per il principio di identità dei polinomi, si deduce
\[
  f(0) = 1,\qquad
  f'(0) = 0,\qquad
  f''(0) = 0,\qquad
  f'''(0) = 0,\qquad
  f''''(0) = -12.
\]
\end{proof}

\begin{definition}[coefficiente binomiale reale]
\label{def:binomiale_reale}%
\mymark{***}%
Dato $\alpha \in \RR$, $k \in \NN$ definiamo
\[
 {\alpha \choose k } 
 = \frac{\alpha \cdot (\alpha-1) \cdots (\alpha -k +1)}{k!}
 = \prod_{j=1}^k \frac{\alpha-j+1}{j}.
\]
Osserviamo che se $\alpha \in \NN$ questa definizione coincide
con la definizione data nel capitolo~\ref{ch:binomiale}.
\end{definition}

\begin{theorem}[sviluppi di Taylor di alcune funzioni elementari]
\label{th:sviluppi_taylor}%
\mymark{***}%
Per ogni $n\in \NN$ si hanno, per $x\to 0$,
le relazioni riportate nella tabella~\ref{tb:taylor}
a pagina~\pageref{tb:taylor}.
\end{theorem}
\begin{table}
\begin{align*}
e^x &= \sum_{k=0}^n \frac{x^k}{k!} + o(x^n) \\
  &= 1 + x + \frac{x^2}{2} + \frac {x^3}{6} + \dots + \frac{x^n}{n!} + o(x^n) \\
\sin x &= \sum_{k=0}^n (-1)^k\frac{x^{2k+1}}{(2k+1)!} + o(x^{2n+2}) \\
 &= x - \frac{x^3}{6} + \dots + (-1)^{n} \frac{x^{2n+1}}{(2n+1)!}  + o(x^{2n+2})\\
 \cos x &= \sum_{k=0}^n (-1)^k \frac{x^{2k}}{(2k)!} + o(x^{2n+1}) \\
   &= 1 - \frac{x^2}{2} + \frac{x^4}{24} - \dots + (-1)^n\frac{x^{2n}}{(2n)!} + o(x^{2n+1})\\
 (1+x)^\alpha &= \sum_{k=0}^n {\alpha \choose k} x^k + o(x^n)\\
    &= 1 + \alpha x + \frac{\alpha (\alpha-1)}{2} x^2 + \dots + {\alpha \choose n} x^n + o(x^n) \\
  \ln (1+x) &= \sum_{k=1}^n (-1)^{k-1} \frac{x^k}{k} + o(x^n) \\
         &= x - \frac{x^2}{2} + \frac{x^3}{3} - \dots + (-1)^{n-1}\frac{x^n}{n} + o(x^n) \\
  \arctg x &= \sum_{k=0}^n (-1)^k \frac{x^{2k+1}}{2k+1} + o(x^{2n+1})\\
    &= x - \frac{x^3}{3} + \frac{x^5}{5} - \dots + (-1)^n \frac{x^{2n+1}}{2n +1} + o(x^{2n+1})\\
  \arcsin x &= x + \frac{x^3}{6} + \frac{3}{40} x^5 + \dots + \frac{(2n-1)!!}{(2n)!!} \frac{x^{2n+1}}{2n+1} + o(x^{2n+1}) \\
  \arccos x &= \frac \pi 2 - \arcsin x\\
  \tg x &= x + \frac{x^3}{3} + \frac{2}{15} x^5 + o(x^5).
\end{align*}
\caption{sviluppi di Taylor, per $x\to 0$, di alcune funzioni elementari.
\index{Taylor!polinomi delle funzioni elementari}%
\index{sviluppo!polinomio di Taylor}%
Si veda il teorema~\ref{th:sviluppi_taylor}.}
\label{tb:taylor}%
\end{table}
%
\begin{proof}
\mymark{**}
Per definizione, il coefficiente del termine $x^k$
nel polinomio di Taylor di $f(x)$ non è altro che $a_k=f^{(k)}(0)/k!$
Se $f(x) = e^x$ allora $f^{(k)}(x) = e^x$ e dunque $f^{(k)}(0) = 1$. Si trovano quindi i coefficienti $a_k = 1/k!$.

Se $f(x) = \sin x$ si ha $f'(x)=\cos x$, $f''(x) =-\sin x$, $f'''(x) = -\cos x$ e $f^{(4)}(x) = \sin x$... e poi le derivate si ripetono ogni quattro iterazioni. Valutando le derivate in $x=0$ si ottiene dunque la sequenza $0, 1, 0, -1, \dots$ che si ripete indefinitamente. Si ottiene dunque lo sviluppo indicato. Discorso analogo si può fare per $f(x) = \cos x$.

Se $f(x) = (1+x)^\alpha$ si ottiene $f'(x) = \alpha (1+x)^{\alpha -1}$, $f''(x) = \alpha (\alpha -1) (1+x)^{\alpha -2}$ e così via...
Valutando le derivate in $x=0$ si ottiene la sequenza: $1$, $\alpha$, $\alpha(\alpha-1)$, $\alpha (\alpha-1)(\alpha -2)$\dots da cui, dividendo per $k!$, si ottiene che i coefficienti del polinomio di Taylor risultano essere i coefficienti binomiali ${\alpha \choose k}$.

Se $f(x) = \ln (1+x)$ osserviamo che $f(0)=0$ poi si ha $f'(x) = (1+x)^{-1}$ e le derivate successive coincidono dunque con le derivate di $(1+x)^\alpha$ con $\alpha=-1$: i coefficienti (a parte il primo che è nullo) concidono quindi con i coefficienti binomiali
\[
{-1 \choose k} = \frac{(-1)(-2)(-3) \dots (-k)}{k!} = (-1)^k.
\]

Similmente se $f(x) = \arctg x$ osserviamo che $f(0) = 0$ e $f'(x) = (1+x^2)^{-1}$. Per quanto già visto sappiamo che si ha
\[
 (1+y)^{-1} = \sum_{k=0}^n (-1)^k y^k + o(y^n)
\]
da cui sostituendo $y=x^2$ si ha
\[
 f'(x) = (1+x^2)^{-1} = \sum_{k=0}^n (-1)^k x^{2k} + o(x^{2n}).
\]
Utilizzando la seconda parte del Teorema~\ref{th:taylor_peano} (formula di Taylor con resto di Peano),
abbiamo verificato che vale l'equazione \eqref{eq:Taylor}
per il polinomio $P(x) = \sum_{k=0}^n (-1)^k x^{2k}$.
Dunque $P(x)$ è il polinomio di Taylor di grado $2n$ di $f'(x)$.
Ma il polinomio di Taylor della derivata è la derivata del polinomio
di Taylor, quindi possiamo utilizzare \eqref{eq:4993725} per ottenere
il polinomio di Taylor riportato in tabella.

Metodo analogo si usa per $f(x)=\arcsin x$, osservando che
\begin{align*}
  f'(x) &= \frac{1}{\sqrt{1-x^2}}
  = (1-x^2)^{-\frac 1 2} \\
  &= \sum_{k=0}^n {-\frac 1 2 \choose k} (-x^2)^k + o(x^{2n}) \\
  &= \sum_{k=0}^n \frac{\enclose{-\frac 1 2}\cdot \enclose{-\frac 1 2 - 1}\cdots \enclose{-\frac 1 2 - k+1}}{k!}(-1)^k x^{2k} + o(x^{2n}) \\
  &= \sum_{k=0}^n \frac{\frac{-1}{2} \cdot \frac{-3}{2}\cdots \frac{-(2k-1)}{2}}{k!}(-1)^k x^{2k} + o(x^{2n}) \\
  &= \sum_{k=0}^n \frac{(2k-1)!!}{2^k \cdot k!}x^{2k} + o(x^{2n}) \\
  &= \sum_{k=0}^n \frac{(2k-1)!!}{(2k)!!}x^{2k} + o(x^{2n})
 \end{align*}
abbiamo trovato il polinomio di Taylor di $f'(x)$ e utilizzando \eqref{eq:4993725}
si ottiene la formula riportata in tabella.

Per quanto riguarda la funzione $f(x) = \tg x$ ci limitiamo a calcolare esplicitamente i primi termini:
\begin{align*}
  f(x) &= \tg x
  & f(0)&=0\\
  f^{(1)} &= 1+ f^2
  & f^{(1)}(0) &= 1\\
  f^{(2)} &= 2ff'
  & f^{(2)}(0) &= 0 \\
  f^{(3)} &= 2(f')^2 + 2f f''
  & f^{(3)}(0) &= 2 \\
  f^{(4)} &= 4 f' f'' + 2 f' f'' + 2 f f''' = 6 f' f'' + 2 f f'''
   & f^{(4)}(0) &= 0 \\
   f^{(5)} &= 6 (f'')^2 + 6 f' f''' + 2 f' f''' + 2 f f^{(4)}&&\\
           &= 6 (f'')^2 + 8 f' f''' + 2 f f^{(4)}
    & f^{(5)}(0) &= 16.
\end{align*}
Si ottengono quindi i coefficienti:
 $a_0 = 0$, $a_1 = 1$, $a_2 = 0$, $a_3 = 2/3! = 1/3$, $a_4=0$, $a_5 = 16/ 5! = 2/15$.
\end{proof}

