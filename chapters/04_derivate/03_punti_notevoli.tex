\section{punti notevoli}

In questa sezione introdurremo una terminologia che è largamente utilizzata 
nello studio di funzione.
Cercheremo di dare delle definizioni anche per alcuni termini su cui potrebbe 
non esserci un consenso univoco.
Ricordiamoci di non usare queste definizioni in modo troppo formale:
sarà sempre meglio
verificare che il nostro interlocutore ci comprenda
perché spesso alcuni termini potrebbero essere utilizzati in maniera
impropria o con significati leggermente diversi.

Nel dubbio potremo sempre evitare di utilizzare questa terminologia 
riconducendoci ai concetti sottostanti.

\begin{definition}[punti notevoli]
Sia $f\colon A \subset \RR \to \RR$ una funzione. Se $f$ è derivabile in un
punto $x_0\in A$ e $f'(x_0) = 0$ diremo che $x_0$ è un \emph{punto critico}%
\mymargin{punto critico}%
\index{punto!critico}
o
\emph{punto stazionario}
\index{punto!stazionario}
di $f$.

Se $x_0\in A$ ed esiste un intorno $U$ di $x_0$ per cui $x_0$ risulta
essere un punto di minimo (rispettivamente di massimo) per $f$ ristretta ad $U$
diremo che $x_0$ è un punto di \emph{minimo relativo}%
\mymargin{minimo relativo}%
\index{minimo!relativo} o \emph{minimo locale}
(rispettivamente \emph{massimo relativo} o \emph{massimo locale}).
Per contrapposizione i punti di massimo e minimo su tutto il dominio $A$
vengono anche
chiamati massimo/minimo \emph{assoluto} di $f$.

Diremo che $x_0\in A$ è un \emph{punto di flesso}%
\mymargin{punto di flesso}%
\index{punto!di flesso} per $f$ se
$f$ è derivabile in un intorno di $x_0$ e $x_0$ è un punto di massimo
o minimo relativo per $f'$. Nel punto $x_0$ la retta tangente
ha equazione $y=r(x) = f'(x_0) (x-x_0) + f(x_0)$. Se $x_0$ è
minimo per $f'$ risulta che $f(x)-r(x)$ è crescente
quindi $f(x)\ge r(x)$ per $x\ge x_0$ e $f(x)\le r(x)$ per $x\le x_0$
(il grafico della funzione attraversa la retta tangente da sotto a sopra)
mentre se $x_0$ è massimo per $f'$ risulta che $f(x)\le r(x)$ per $x\ge x_0$
e $f(x) \ge r(x)$ per $x\le x_0$ (il grafico della funzione attraversa
la tangente da sopra a sotto).
Se la funzione $f$ non è derivabile in $x_0$ ma il limite del rapporto
incrementale esiste ed è infinito, diremo che $x_0$ è un
\emph{flesso verticale}%
\mymargin{flesso verticale}%
\index{flesso!verticale}. In tale punto la retta tangente è verticale
e il grafico della funzione attraversa tale retta.

Sia $x_0\in A$ un punto in cui la funzione $f$ è continua ed esistono
i limiti destro e sinistro del rapporto incrementale
(che si chiamano \emph{derivata destra} e \emph{derivata sinistra})
\[
  m^{\pm} = \lim_{h\to 0^\pm}\frac{f(x+h) - f(x)}{h}.
\]
Se $m^+ \neq m^-$ chiaramente $f$ non è derivabile in $x_0$.
Se entrambi $m^+$ ed $m^-$ sono finiti diremo che $x_0$ è un
\emph{punto angoloso}%
\mymargin{punto angoloso}%
\index{punto!angoloso} in quanto le due semirette tangenti
in $x_0$ (da destra e da sinistra) formano un angolo non piatto.
Se $m^+=-m^-=+\infty$ oppure se $m^+=-m^-=-\infty$
diremo che il punto $x_0$ è un \emph{punto di cuspide}%
\mymargin{punto di cuspide}%
\index{punto!di cuspide} (c'è una
semiretta tangente verticale).
\end{definition}

\begin{definition}[asintoti]
Diremo che la retta $y=mx+q$ è un asintoto per il grafico di $f$ 
per $x\to +\infty$ se risulta
\[
  \lim_{x\to +\infty} f(x) - (mx+q) = 0.
\]
Se $m=0$ diremo che il grafico di $f$ ha un \emph{asintoto orizzontale} $y=q$
\mymargin{asintoto orizzontale e obliquo}%
\index{asintoto orizzontale e obliquo}%
\index{asintoto!orizzontale}%
\index{asintoto!obliquo}%
altrimenti diremo che $y=mx+q$ è un \emph{asintoto obliquo}.
Stessa cosa si può dire per $x\to -\infty$.

Se $x_0\in \RR$ e si ha 
\[
  \lim_{x\to x_0} \abs{f(x)} = +\infty
\]
diremo che la retta $x=x_0$ è un \emph{asintoto verticale}%
\mymargin{asintoto verticale}%
\index{asintoto!verticale} per il 
grafico della funzione $f$.
\end{definition}

\begin{definition}[punti di discontinuità]
Se per $x_0\in \RR$ si ha 
\[
  \lim_{x\to x_0^-} f(x) = \ell_1, 
  \qquad 
  \lim_{x\to x_0^+} f(x) = \ell_2
\]
e se $\ell_1\neq \ell_2$ diremo che nel punto $x_0$ 
la funzione $f$ ha una \emph{discontinuità a salto}%
\mymargin{discontinuità a salto}%
\index{discontinuità!a salto}.
Se $\ell_1=\ell_2$ e se $f$ non è definita nel punto $x_0$ 
oppure se $f(x_0)\neq \ell_1$ diremo che 
nel punto $x_0$ la funzione $f$ 
ha una \emph{discontinuità eliminabile}%
\mymargin{discontinuità eliminabile}%
\index{discontinuità!eliminabile}.
\end{definition}

Si osservi che, nonostante la terminologia utilizzata,
una funzione continua può avere una discontinuità 
a salto (ad esempio: $f(x)=\frac{x}{\abs{x}}$) 
e può anche avere una discontinuità eliminabile
(ad esempio: $f(x) = \frac{x}{x}$).

