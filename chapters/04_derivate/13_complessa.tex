
\section{derivata complessa}
\label{sec:derivata_complessa}

Se abbiamo una funzione $f\colon A\subset \CC \to \CC$
possiamo definire la derivata (complessa) esattamente
come abbiamo fatto per la derivata reale:
\[
  f'(x) = \lim_{h\to 0} \frac{f(x+h)-f(x)}{h}
\]
(se il limite esiste finito).
Si osservi che in questo caso $h\in \CC$,
il rapporto incrementale è una divisione complessa
e il limite è un limite complesso.

Ad esempio la funzione $f(z)=z$ è derivabile
in senso complesso e la sua derivata è $f'(z) = 1$
in quanto
\[
  \lim_{h\to 0} \frac{z+h-z}{h} = \lim_{h\to 0} \frac{h}{h}=1.
\]
Le dimostrazioni che riguardano la derivata della somma
e la derivata del prodotto si ripetono formalmente identiche
a come le abbiamo fatte per la derivata reale.
Potremo quindi affermare che se $f$ e $g$ sono derivabili
in senso complesso in un punto $z_0\in \CC$ risulta
\begin{align*}
  (f+g)'(z_0) 
    &= f'(z_0) + g'(z_0), \\
  (f\cdot g)'(z_0) 
    &= f'(z_0)\cdot g(z_0) + f(z_0)\cdot g'(z_0).
\end{align*}
Dunque si avrà, se $n\in \NN\setminus\ENCLOSE{0}$,
\[
  \enclose{z^n}' = n z^{n-1}.
\]
Ovviamente la derivata di una costante è nulla.
La formula per la derivata della funzione composta è anch'essa valida 
e si dimostra ripetendo formalmente la stessa dimostrazione che
abbiamo già visto:
\[
  \enclose{f(g(z))}'  = f'(g(z))\cdot g'(z).
\]
Anche la formula per la derivata del reciproco si ottiene
con la stessa dimostrazione che abbiamo già visto nel caso
reale:
\[
  \enclose{\frac 1 z}' = - \frac{1}{z^2}
\]
e di conseguenza vale la formula per la derivata del rapporto
\[
\enclose{\frac{f(z)}{g(z)}}' = \frac{f'(z)\cdot g(z) - f(z)\cdot g'(z)}{g^2(z)}.
\]

Abbiamo quindi che ogni funzione polinomiale a coefficienti in $\CC$
è derivabile in senso complesso.
E lo stesso vale per le funzioni razionali ovvero i rapporti
di polinomi.

La funzione esponenziale è un'altro esempio importantissimo
di funzione complessa derivabile in quanto risulta
\[
  \lim_{h\to 0} \frac{e^{z+h}-e^z}{h}
  = \lim_{h\to 0} e^z \cdot \frac{e^h-1}{h} = e^z
\]
in virtù del noto limite notevole di cui
gode la funzione esponenziale
(teorema~\ref{th:exp_complesso}).
Più in generale è facile verificare che tutte le funzioni
analitiche sono derivabili in senso complesso.

Un risultato sorprendente dell'analisi complessa ci dice
che è vero anche il viceversa: ogni funzione
derivabile in senso complesso in tutti i punti del suo
dominio risulta essere analitica (e in particolare di classe
$C^\infty$).

In effetti la derivabilità in senso complesso è una proprietà
molto forte. Ad esempio la funzione $f(z) = \bar z$
non risulta essere derivabile in senso complesso in quanto
\[
\frac{\overline{z+h}-\overline z}{h} = \frac{\bar h}{h}
\]
e per $h\to 0$ il limite non esiste visto che se
$h$ è reale tale rapporto vale sempre $1$
ma se $h$ è immaginario puro tale rapporto vale sempre $-1$.
