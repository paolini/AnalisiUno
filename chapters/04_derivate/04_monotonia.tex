\section{criteri di monotonia}

\begin{theorem}[Fermat]
\mymark{***}
Sia $f\colon (a,b)\to \RR$ una funzione derivabile.
Se $x_0\in (a,b)$ è un punto di massimo o minimo per $f$ allora
$f'(x_0)=0$.
\end{theorem}
%
\begin{proof}
\mymark{***}
Senza perdere di generalità possiamo suppore che $x_0$ sia un punto di massimo per $f$.
Sappiamo che
\[
  f'(x_0) = \lim_{x\to x_0}\frac{f(x)-f(x_0)}{x-x_0}.
\]
Visto che $x_0$ è un punto dell'intervallo aperto $(a,b)$ la funzione $f$ è definita in un intorno destro di $x_0$ e quindi possiamo restingere il limite ai valori $x>x_0$ ottenendo:
\[
  f'(x_0) = \lim_{x\to x_0^+}\frac{f(x) - f(x_0)}{x-x_0}.
\]
Visto che $x_0$ è un punto di massimo per $f$ sappiamo che $f(x)-f(x_0)\le 0$. Essendo $x-x_0>0$ l'intero rapporto incrementale risulta essere non positivo.
Dunque, per il teorema della permanenza del segno,
possiamo concludere che $f'(x_0)\le 0$.

Ma possiamo anche restringere la funzione ad un intorno sinistro di $x_0$ e osservare che
\[
  f'(x_0) = \lim_{x\to x_0^-}\frac{f(x)-f(x_0)}{x-x_0}.
\]
Ma ora il numeratore è, come prima, non positivo mentre il denominatore $x-x_0$ è negativo. Dunque il rapporto incrementale stavolta è non negativo e quindi, per la permanenza del segno, $f'(x_0) \ge 0$.

Abbiamo scoperto quindi che $f'(x_0)\le 0$ e $f'(x_0)\ge 0$
da cui deduciamo $f'(x_0)=0$.
\end{proof}

Il teorema di Fermat si può
enunciare dicendo che ogni punto di massimo o minimo relativo interno
al dominio di una funzione in cui la funzione è derivabile
è necessariamente un punto critico.
In particolare per determinare massimi e minimi assoluti e relativi
di una funzione sarà sufficiente esaminare i punti di frontiera,
i punti di non derivabilità e i punti critici.


\begin{theorem}[Rolle]
\mymark{***}
\index{teorema!di Rolle}
\mymargin{Rolle}%
\index{Rolle}
Sia $f\colon [a,b]\to \RR$, $a,b\in \RR$, $a<b$, una funzione continua su tutto $[a,b]$ e derivabile su $(a,b)$.
\mynote{Se $b<a$ il teorema è ugualmente valido 
se si intende $[a,b]=[b,a]$}%
Se $f(a) = f(b)$ allora esiste $x_0 \in (a,b)$ tale che $f'(x_0)=0$.
\end{theorem}
%
\begin{proof}
\mymark{***}
Essendo $f$ una funzione continua
possiamo applicare il teorema di Weiestrass per dedurre che $f$ ha massimo e 
minimo sull'intervallo chiuso e limitato $[a,b]$. 
Se il punto di massimo o il punto di minimo sta nell'intervallo aperto 
$(a,b)$ possiamo applicare il teorema di Fermat per ottenere che la derivata 
di $f$ si annulla in tale punto.

In caso contrario sia il punto di massimo che il punto di minimo sono estremi 
dell'intervallo, cioè sono uguali ad $a$ o a $b$. Ma visto che $f(a)=f(b)$ 
i valori massimo e minimo coincidono e quindi la funzione è costante. 
Ma in tal caso $f'(x)=0$ per ogni $x\in [a,b]$.
\end{proof}

\begin{theorem}[Lagrange]\label{th:lagrange}%
\mymark{***}%
\index{teorema!di Lagrange}%
\mymargin{Lagrange}%
\index{Lagrange}%
Sia $f\colon [a,b]\to \RR$ una funzione continua su $[a,b]$ e derivabile su $(a,b)$
con $a,b\in \RR$, $a<b$.
\mynote{Se $b<a$ il teorema è ugualmente valido 
se si intende $[a,b]=[b,a]$}%
Allora esiste un punto $x_0\in (a,b)$ tale che
\[
  f'(x_0) = \frac{f(b) - f(a)}{b-a}
\]
\end{theorem}
%
\begin{proof}
\mymark{***}
Consideriamo la funzione ausiliaria:
\[
  g(x) = f(x) - \frac{f(b)-f(a)}{b-a} x.
\]
Per verifica diretta si osserva che
\[
  g(b) = g(a) = \frac{b f(a) - a f(b)}{b-a}.
\]
La funzione $g$ soddisfa quindi le ipotesi del teorema di Rolle e dunque esisterà $x_0\in (a,b)$ tale che $g'(x_0)=0$.
Ma si osserva che
\[
  g'(x) = f'(x) - \frac{f(b)-f(a)}{b-a}
\]
e dunque se $g'(x_0)=0$ si ottiene il risultato desiderato.
\end{proof}

\begin{theorem}[criteri di monotonia]%
\label{th:criteri_monotonia}
\mymark{***}%
\mymargin{criteri di monotonia}%
\index{criterio!di monotonia}%
Sia $f\colon I \to \RR$ una funzione definita su un intervallo $I\subset \RR$. Sia $J= (\inf I, \sup I)$ l'intervallo aperto con gli stessi estremi di $I$.
Supponiamo che $f$ sia continua su $I$ e derivabile su $J$. 
Allora valgono i seguenti criteri:
\begin{enumerate}
\item
$(\forall x \in J\colon f'(x)\ge 0)$
$\iff$
$f$ è crescente (su tutto $I$);
\item
$(\forall x \in J\colon f'(x)\le 0)$
$\iff$
$f$ è decrescente (su tutto $I$);
\item
$(\forall x \in J\colon f'(x)=0)$
$\iff$
$f$ è costante (su tutto $I$);
\item
$(\forall x \in J\colon f'(x)>0)$
$\implies$
$f$ è strettamente crescente (su tutto $I$);
\item
$(\forall x \in J\colon f'(x)<0)$
$\implies$
$f$ è strettamente decrescente (su tutto $I$).
\end{enumerate}
\end{theorem}
%
\begin{proof}
\mymark{***}
Dimostriamo innanzitutto le implicazioni da sinistra verso destra.

Per la prima, se $f$ non fosse crescente ci dovrebbero essere due punti $a, b \in I$ tali che $a < b$ ma $f(a) > f(b)$.
Dunque si avrebbe
\[
  \frac{f(b) - f(a)}{b - a} < 0.
\]
Applicando il teorema di Lagrange all'intervallo $[a,b]$ si troverebbe un punto $x\in (a,b)$ tale che $f'(x) < 0$. Chiaramente $(a,b)\subset J$ e quindi questo contraddice l'ipotesi $f'(x) \ge 0$.

La seconda implicazione (per le funzioni decrescenti) si dimostra in maniera analoga cambiando verso alle disuguaglianze.

Anche la terza implicazione si dimostra tramite il teorema di Lagrange in modo analogo alle precedenti. Oppure basta osservare che se $f'(x)=0$ allora valgono contemporaneamente $f'(x)\ge 0$ e $f'(x)\le 0$ quindi mettendo insieme le prime due implicazioni si ottiene che $f$ è contemporaneamente crescente e decrescente dunque è costante.

Per la quarta implicazione si procede come per la prima. Per assurdo si  avrebbero $a<b$ con $f(b) \le f(a)$. Ma allora
\[
  \frac{f(b) - f(a)}{b-a} \le 0
\]
e applicando il teorema di Lagrange si troverebbe un punto $x\in (a,b)$ con $f'(x) \le 0$, contro l'ipotesi $f'(x) > 0$.

La quinta implicazione si dimostra in maniera analoga cambiando verso alle disuguaglianze.

Vediamo ora le implicazioni da destra verso sinistra.
Per la prima, supponiamo che $f$ sia crescente e prendiamo $x\in J$. Allora è chiaro che per ogni $h>0$ si avrà $f(x+h) \ge f(x)$ e dunque
\[
  \frac{f(x+h)- f(x)}{h} \ge 0.
\]
Facendo il limite per $h \to 0^+$ si ottiene $f'(x)$ e, per la permanenza del segno, dovra essere $f'(x) \ge 0$.

In maniera analoga (invertendo le disuguaglianze) si dimostra la seconda implicazione.

La terza discende dalle prime due oppure, più semplicemente, dalle regole di derivazione, in quanto la derivata di una costante è zero.
\end{proof}

\begin{example}
La funzione $f(x) = 1/x$ è definita su $\RR \setminus \ENCLOSE{0}$, è derivabile
e la derivata $f'(x) = -1/x^2$ è ovunque negativa. La funzione $f$ è quindi strettamente
decrescente separatamente sui due intervalli $(0,+\infty)$ e $(-\infty,0)$ sui quali
possiamo applicare il criterio di monotonia. Ma non è
decrescente su tutto il suo dominio in quanto, ad esempio, $f(-1) = -1 < 1 = f(1)$.
Questo esempio mostra che nei criteri di monotonia l'ipotesi che il dominio sia un intervallo
è fondamentale.
\end{example}

