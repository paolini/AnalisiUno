\section{interpretazione geometrica dell'integrale}
\index{integrale!interpretazione geometrica}%
\index{interpretazione!geometrica dell'integrale}%
%%%%

Il seguente risultato mette in corrispondenza l'integrale di Riemann 
con la misura di Peano-Jordan ed è un modo formale per dare all'integrale 
di Riemann il significato di area (con segno) del sottografico della 
funzione integranda. 
Nel seguito non faremo mai uso di questo risultato che serve solo 
a soddisfare la nostra intuizione.

\begin{theorem}[interpretazione geometrica dell'integrale]
Sia $f\colon[a,b]\to \RR$, una funzione limitata, $a\leq b$. Posto 
\begin{align*}
  E^+ &= \ENCLOSE{(x,y)\in [a,b]\times\RR\colon 0\le y \le f(x)}, \\
  E^- &= \ENCLOSE{(x,y)\in [a,b]\times\RR\colon f(x)\le y \le 0}   
\end{align*}
si ha che $f$ è Riemann-integrabile su $[a,b]$ se e solo se 
gli insiemi $E^+$ ed $E^-$ sono misurabili secondo Peano-Jordan.
E in tal caso risulta:
\[
   \int_a^b f = m(E^+) - m(E^-)  
\]
dove $m$ è la misura di Peano-Jordan.
\end{theorem}
%
\begin{proof}
\emph{Passo 1:} supponiamo che sia $f(x)\ge 0$ per ogni $x\in [a,b]$.
Se $P=\ENCLOSE{x_0,x_1, \dots, x_N}$ è una suddivione di $[a,b]$ 
è chiaro che posto $I_k = [x_k,x_{k+1}]$ l'unione dei rettangoli 
$R_k = I_k \times [0,\inf f(I_k)]$ ci dà
un polirettangolo contenuto in $E^+$.
E $S_*(f,P)$ rappresenta proprio la misura di tale polirettangolo.
Dunque $m_*(E^+)\ge S_*(f,P)$ per ogni suddivisione $P$ 
e quindi $m_*(E^+)\ge I_*(f,P)$.
Ragionando in maniera analoga con i rettangoli la cui altezza 
è il $\sup$ di $f$, si osserva che $S^*(f,P)$ rappresenta 
la misura di un polirettangolo contenente $E^+$ e dunque 
$m^*(E^+)\le I^*(f,P)$.

D'altra parte preso un qualunque polirettangolo contenuto in $E^+$
possiamo considerare la suddivisione $P$ formata da $a$, $b$ e da tutte le coordinate $x$ 
dei lati verticali dei rettangoli che compongono il polirettangolo. 
E' chiaro che il polirettangolo sarà allora contenuto nell'unione
dei rettangoli determinati da $P$ con altezza l'estremo inferiore di $f$.
Dunque $m_*(E^+)\le I_*(f)$.
Analogamente se prendiamo un polirettangolo che contiene $E^+$ possiamo 
innanzitutto tagliare tutti i rettangoli con le rette $x=a$ e $x=b$
e rimuovere eventuali rettangoli che si trovino a sinistra di $x=a$ 
o a destra di $x=b$. 
Questo diminuisce la misura del polirettangolo e mantiene l'inclusione di $E^+$.
A questo punto consideriamo la suddivisione $P$ ottenuta prendendo i punti 
$a$, $b$ e tutte le coordinate $x$ dei lati verticali dei rettangoli che formano 
il polirettangolo.
E' chiaro allora che la misura del polirettangolo originario è maggiore o uguale 
a $S^*(f,P)$ e dunque $m^*(E^+)\ge I^*(f)$.

Essendo dunque $m_*(E^+)=I_*(f)$ e $m^*(E^+)=I_*(f)$ risulta 
che se $f\ge0$ allora $f$ è Riemann-integrabile se e solo se $E^+$ è Peano-Jordan 
misurabile e in tal caso si ha $\int_a^b f = m(E^+)$.

\emph{Passo 2:} sia $f\colon[a,b]\to \RR$ qualunque.
Allora si avrà $f = f^+ - f^-$. 
A $f^+$ e $f^-$ potremo applicare il passo precedente.
Osserviamo che l'insieme $E^+$ di $f^-$ non è altro che la riflessione 
(rispetto all'asse delle ascisse) dell'insieme $E^-$ di $f$
e questi insiemi hanno la stessa misura di Peano-Jordan.
Dunque, se $f$ è Riemann-integrabile anche $f^+$ e $f^-$ 
lo sono (Teorema~\ref{th:reticolo}) 
dunque $E^+$ ed $E^-$ sono Peano-Jordan misurabili per il passo precedente
e si ha 
\begin{equation}
  \label{47694}
\int_a^b f = \int_a^b f^+ - \int_a^b f^- = m(E^+) - m(E^-).  
\end{equation}
Viceversa se $E^+$ ed $E^-$ sono Peano-Jordan misurabili 
allora $f^+$ ed $f^-$ sono Riemann-integrabili (per il passo precedente)
e anche $f=f^+-f^-$ è Riemann-integrabile (Teorema~\ref{th:integrale_lineare})
e di nuovo deve valere~\eqref{47694}.
\end{proof}

\begin{example}[area del trapezio]
Sia $f(x) = mx + q$ una funzione lineare definita 
su un intervallo $[a,b]$ e supponiamo che $f(a)\ge 0$ 
e $f(b)\ge 0$.
L'insieme $E^+=\ENCLOSE{(x,y)\colon x\in [a,b], 0\le y \le f(x)}$
è dunque un trapezio di basi (verticali) $f(a)$ e $f(b)$
e altezza (orizzontale) $b-a$. La sua area sarà dunque 
\begin{align*}
  \int_a^b f 
  &= \int_a^b (mx+q)\, dx 
  = m\int_a^b x\, dx + \int_a^b q\, dx
  = \frac{m}{2}(b^2-a^2) + (b-a) q\\ 
  &= \frac 1 2 (b-a)(mb+ma+2q)
  = \frac 1 2 (b-a)(f(a) + f(b))
\end{align*}
che è effettivamente la formula elementare per 
calcolare l'area di un trapezio.
\end{example}

