\section{integrale di una funzione razionale}
%
In questa sezione presenteremo un metodo per
calcolare esplicitamente l'integrale di una qualunque funzione razionale.
\index{integrale!di una funzione razionale}%
\index{primitiva!di una funzione razionale}%

\begin{definition}[funzione razionale]
\mymargin{funzione razionale}%
\index{funzione!razionale}%
\index{razionale!funzione}%
Una funzione $f\colon A \subset \RR \to \RR$
si dice essere \emph{razionale} se si può scrivere nella forma
\[
  f(x) = \frac{P(x)}{Q(x)}
\]
con $P$ e $Q$ funzioni polinomiali a coefficienti reali.
\end{definition}

Il metodo che andremo a presentare può essere utilizzato ``alla cieca'' in
quanto una volta trovata la decomposizione non c'è bisogno di sapere
come mai il metodo utilizzato funziona (e perché funziona sempre).
La dimostrazione del perché questo metodo funzioni è un esercizio non
banale di algebra lineare a cui dedicheremo il resto di questo capitolo
perché, tutto sommato, pensiamo che sia sempre utile avere la consapevolezza degli strumenti che si utilizzano.

\subsection{esempi con denominatore di primo e secondo grado}

Prima ancora di enunciare il teorema e di farne la dimostrazione,
vogliamo presentare il metodo tramite degli esempi. 
Probabilmente questo è il modo più semplice per capire come bisogna operare.

\begin{example}[denominatore di grado $1$]
  Calcolare 
  \[
  \int \frac{x^3+1}{x-1}\, dx.  
  \]
\end{example}
\begin{proof}[Svolgimento.]
Siamo di fronte all'integrale di una funzione razionale con denominatore di grado 1 e numeratore di grado 3.
In generale quando il grado del numeratore non è inferiore a quello del denominatore bisogna innanzitutto 
svolgere la divisione tra polinomi (si veda~\ref{th:divisione_polinomi}). Si trova: 
\[
\frac{x^3+1}{x-1} = x^2 + x + 1 + \frac{2}{x-1}.
\]
Calcolare l'integrale del polinomio quoziente è sempre banale:
\[
\int \enclose{x^2+x+1}  \, dx \ni \frac{x^3}{3} + \frac{x^2}{2} + x.
\]
Visto che il resto nella divisione tra polinomi ha grado inferiore al denominatore, ci si riconduce 
sempre all'integrale di una funzione razionale in cui il numeratore ha grado inferiore del denominatore.
Se il denominatore ha grado $1$, come in questo esempio, il numeratore dovrà sempre avere grado $0$ e quindi
è costante. 
Dunque a meno di costanti moltiplicative ci si riconduce sempre all'integrale di una funzione del tipo $\frac{1}{x+c}$
la cui primitiva è un logaritmo:
\[
 \int \frac{2}{x-1}\, dx \ni 2 \ln \abs{x-1} = \ln (x-1)^2
\]
In conclusione una delle primitive cercate è 
\[
  \int \frac{x^3+1}{x-1}\, dx \ni \frac{x^3}{3} + \frac{x^2}{2} + x + \ln (x-1)^2.
\]
\end{proof}

\begin{example}[denominatore di grado 2 con radici reali distinte]
Calcolare 
\[
  \int \frac{x+1}{x^2-x}\, dx.
\]
\end{example}
\begin{proof}[Svolgimento.]
Il polinomio a denominatore si fattorizza $x^2 -x = x (x-1)$. 
Allora l'idea è quella di pensare che il denominatore di secondo 
grado si sia ottenuto come denominatore comune in una somma di 
funzioni razionali con denominatori di primo grado:
\begin{equation}\label{eq:541177}
  \frac{x+1}{x(x-1)} \stackrel?= \frac{A}{x} + \frac{B}{x-1}  
\end{equation}
con $A,B\in \RR$ costanti da determinare. 
Questa si chiama \emph{decomposizione in fratti semplici}.
\index{decomposizione!in fratti semplici}%
Le costanti possono essere determinate andando a ritroso:
\[
  \frac{A}{x} + \frac{B}{x-1}    
  = \frac{A(x-1) + Bx}{x(x-1)} \stackrel!= \frac{x+1}{x(x-1)}.
\]
Affinchè valga l'uguaglianza si deve avere 
\[
  A(x-1) + Bx = (A+B)x - A \stackrel!= x+1  
\]
e quindi $A+B = 1$ e $-A=1$ da cui $A=-1$ e $B=2$.
%
\mynote{\textbf{metodo dei residui.} 
\index{residui!metodo dei}%
\index{metodo!dei residui}%
I coefficienti $A$ e $B$ si possono trovare
anche nel modo seguente. 
Se moltiplico l'equazione~\eqref{eq:541177}
per $x$ e poi faccio tendere $x$ a $0$,
il termine $B$ sul lato destro si cancella 
e ottengo immediatamente $A=-1$. 
Se invece moltiplico l'equazione per $x-1$ e poi faccio tendere $x$ a $1$,
il termine $A$ sul lato destro si cancella
e ottengo immediatamente $B=2$.}
%
Dunque 
\[
  \frac{x+1}{x(x-1)} = \frac{-1}{x} + \frac{2}{x-1}  
\]
e la primitiva è una somma di logaritmi:
\[
  \int \frac{x+1}{x^2-x}\, dx
  \ni -\ln\abs{x} + 2 \ln \abs{x-1} 
  = \ln \frac{(x-1)^2}{\abs{x}}.
\]
\end{proof}

\begin{example}[denominatore di secondo grado con una radice doppia]
Calcolare 
\[
  \int \frac{x-1}{x^2+2x+1}\, dx
\]
\end{example}
\begin{proof}[Svolgimento.]
Il polinomio a denominatore è un quadrato perfetto $x^2+2x+1=(x+1)^2$.
In questo caso la decomposizione in fratti semplici non può avere 
i due denominatori uguali ma bisogna elevare un denominatore al 
quadrato:
\begin{equation}\label{eq:2888323}
\frac{x-1}{(x+1)^2} \stackrel?= \frac{A}{x+1} + \frac{B}{(x+1)^2}.  
\end{equation}
Perché valga l'uguaglianza si deve avere
\[
\frac{A}{x+1} + \frac{B}{(x+1)^2} = \frac{A(x+1) + B}{(x+1)^2}  
\stackrel!=\frac{x-1}{(x+1)^2}. 
\]
Da cui si trova $Ax+A+B=x-1$ ovvero $A=1$ e $B=-2$. 

\mynote{\textbf{metodo dei residui.} I coefficienti $A$ e $B$ si possono trovare
anche nel modo seguente.
Se moltiplico l'equazione~\eqref{eq:2888323}
per $(x+1)^2$ e poi faccio tendere $x$ a $-1$,
il coefficiente $A$ sul lato destro sparisce 
e si ottiene immediatamente $B=-2$.
Tenendo ora in considerazione il fatto che il termine 
con il coefficiente $B$ compensa il primo ordine di infinito 
del termine a lato sinistro (formalmente: si porti a sinistra 
dell'equazione il termine con coefficiente $B$)
se moltiplichiamo l'equazione per $x+1$ e poi facciamo tendere $x$ a $-1$
Sottraendo ora $-2/(x+1)^2$ ad ambo i membri si elimina 
il coefficiente $B$ e si ottiene immediatamente $A=1$.}

Dunque 
\[
  \int \frac{x-1}{(x+1)^2}\, dx
  = \int \frac{1}{x+1}\, dx - 2 \int \frac{1}{(x+1)^2}
  \ni \ln \abs{x+1} + \frac{2}{x+1}. 
\]
\end{proof}

\begin{example}[denominatore di secondo grado indecomponibile]
Calcolare 
\[
 \int \frac{x+2}{x^2+2x+3}\, dx
\]
\end{example}
\begin{proof}[Svolgimento.]
In questo caso il denominatore non ha radici reali (il discriminante è negativo) e quindi non può 
essere fattorizzato.
Bisogna allora svolgere un procedimento in due passi: 
\begin{enumerate}
  \item si elimina il termine con la $x$ al numeratore sottraendo 
  una opportuna derivata del denominatore (così da ottenere la derivata 
  del logaritmo del denominatore);
  \item si completa il quadrato a denominatore per ricondursi alla 
  derivata dell'arcotangente. 
\end{enumerate} 
Per il primo passo si osserva che $(x^2+2x+3)' = 2x+2$ e dividendo il numeratore 
per $2x+2$ si ottiene $x+2 = \frac 1 2 (2x+2) + 1$ da cui:
\[
\int \frac{x+2}{x^2+2x+3} \, dx
= \frac 1 2 \int \frac{2x + 2}{x^2+2x+3}\, dx + \int \frac{1}{x^2+2x+3}\, dx. 
\]
Ma il primo integrale è immediato 
\[
\int \frac{2x+2}{x^2+2x+3}\, dx \ni \ln (x^2+2x+3).
\]
Per il secondo integrale l'idea è di fare il completamento 
del quadrato per eliminare il monomio di grado uno e quindi 
ricondursi con un cambio di variabile 
$y=(x+1)/\sqrt 2$, $dx = \sqrt 2 dy$ 
a $\int \frac{1}{1+y^2}\, dy\ni \arctg y$:
\[
  \int \frac{1}{x^2+2x+3}\, dx 
  = \int \frac{1}{\enclose{x+1}^2 + 2}\, dx 
  = \frac 1 2 \int \frac{1}{\enclose{\frac{x+1}{\sqrt 2}}^2+1}\, dx 
  \ni \frac{\sqrt 2}{2} \arctg \frac{x+1}{\sqrt 2}.
\]
In conclusione l'integrale cercato è:
\[
  \frac 1 2 \ln (x^2+2x+3) + \frac{\sqrt 2}{2} \arctg \frac{x+1}{\sqrt 2}.
\]
\end{proof}

\subsection{esempio del metodo generale}

Il metodo generale può essere intuito con il seguente 
esempio (decisamente più complicato).

\begin{example}
  Calcolare
  \[
   \int \frac{x^{10}+x^8-x^6+1}{x^8-x^7+2x^6-2x^5+x^4-x^3}\, dx
  \]
\end{example}
\begin{proof}[Svolgimento.]
Per prima cosa vogliamo ricondurci ad una funzione razionale
in cui il grado del polinomio al numeratore sia inferiore
al grado del polinomio al denominatore. Per fare questo svolgiamo
la divisione tra polinomi
(teorema~\ref{th:divisione_polinomi}):
\[
  \frac{x^{10}+x^8-x^6+1}{x^8-x^7+2x^6-2x^5+x^4-x^3}
  = x^2 + x + \frac{x^4+1}{x^8-x^7+2x^6-2x^5+x^4-x^3}.
\]
L'integrale di $x^2+x$ è immediato e quindi ci possiamo concentrare
sulla funzione razionale con numeratore $x^4+1$.
Per procedere è ora necessario determinare tutte le radici, reali e complesse, del polinomio a denominatore in modo da poterlo fattorizzare
come prodotto di fattori lineari e quadratici
(teorema~\ref{th:fattorizzazione_polinomio_reale}).
Non c'è un metodo generale per trovare le radici di un polinomio.
In questo caso particolare si osserva
che il polinomio è divisibile per $x^3$. Poi si osserva che il polinomio
di quinto grado risultante si annulla per $x=1$ (caso fortunato!) ed
è quindi divisibile per $x-1$ (teorema~\ref{th:Ruffini} di Ruffini).
Il polinomio di quarto grado
risulta infine un quadrato perfetto e si ottiene quindi la fattorizzazione
cercata:
\[
x^8-x^7+2x^6-2x^5+x^4-x^3 = x^3(x-1)(x^2+1)^2.
\]
Ora il teorema di decomposizione di Hermite (teorema~\ref{th:Hermite} che andremo poi a dimostrare)
ci garantisce che ogni funzione razionale della forma
\[
\frac{P(x)}{x^3(x-1)(x^2+1)^2}
\]
con $\deg P < 8$ (il grado del denominatore) può essere
scritto nel modo seguente, con
opportune costanti $A,B,C,\dots,H$:
\begin{equation}\label{eq:438943}
  \frac{A}{x} + \frac{B}{x-1} + \frac{Cx+D}{x^2+1}
  + \enclose{\frac{Ex^3 + Fx^2 + G x + H}{x^2(x^2+1)}}'.
\end{equation}
I primi addendi hanno a denominatore i fattori in cui abbiamo
decomposto il denominatore della funzione originale. A numeratore
c'è sempre un polinomio di grado inferiore al grado del denominatore
e quindi c'è una costante sopra i fattori lineari e una funzione lineare
sopra i termini quadratici. Dopodiché si aggiunge la derivata di una funzione
razionale in cui il denominatore ha gli stessi fattori ma elevati ad una
potenza di uno inferiore a quella che c'era nel polinomio originale.
A numeratore, di nuovo, un generico polinomio di grado inferiore al denominatore.
Per trovare i coefficienti $A,B,C, \dots, H$ 
(il numero dei coefficienti è pari al grado del polinomio a denominatore) 
è sufficiente svolgere la derivata in~\eqref{eq:438943}, 
mettere a denominatore comune tutti gli addendi 
e uguagliare con la funzione razionale originale:
\mynote{%
Nello svolgere la derivata della funzione razionale è conveniente
considerare la funzione razionale come il prodotto del numeratore
per i reciproci dei fattori a denominatore ed utilizzare la formula
per la derivata di un prodotto (invece che la formula per la derivata del rapporto)%
}
\begin{align*}
  &\frac{A}{x} + \frac{B}{x-1} + \frac{Cx+D}{x^2+1}
  + \frac{3 E x^2 + 2 F x + G}{x^2(x^2+1)} \\
  & \quad + (E x^3 + F x^2 + G x + H)\cdot \left[\frac{-2}{x^3}\cdot \frac{1}{x^2+1}
  + \frac{1}{x^2}\cdot\frac{-2x}{(1+x^2)^2}\right]
\end{align*}
moltiplicando tutto per $x^3(x-1)(x^2+1)^2$ si ottiene
\begin{align*}
  &Ax^2(x-1)(x^2+1)^2 + Bx^3(x^2+1)^2 \\
  &\quad + (Cx+D)x^3(x-1)(x^2+1) + (3Ex^2+2Fx+G)x(x-1)(x^2+1)\\
  &\quad + (Ex^3+Fx^2+Gx+H)(-2(x-1)(x^2+1)-2x\cdot x(x-1))
\end{align*}
ovvero
\begin{align*}
  &(A+B+C)x^7+(-A-C+D-E)x^6 + (2A+2B+C-D+E-2F)x^5 \\
  &\quad  +(-2A-C+D+E+2F-3G)x^4 + (A+B-D-E+3G-4H)x^3 \\
  &\quad  + (-A-G+4H)x^2 + (G-2H) x + 2H.
\end{align*}
Ora imponiamo che quest'ultimo polinomio sia identicamente uguale
al polinomio originario $x^4+1$ ottenendo il seguente sistema lineare
\[
  \begin{dcases}
    \scriptstyle
    A+B+C=0 \\[-1ex]
    \scriptstyle
    -A-C+D-E=0 \\[-1ex]
    \scriptstyle
    2A + 2B + C - D + E - 2 F = 0\\[-1ex]
    \scriptstyle
    -2A-C+D+E+2F-3G = 1\\[-1ex]
    \scriptstyle
    A+B-D-E+3G-4H=0\\[-1ex]
    \scriptstyle
    -A-G+4H=0\\[-1ex]
    \scriptstyle
    G-2H=0\\[-1ex]
    \scriptstyle
    2H=1
  \end{dcases}
\]
che risolto ci dà: $A=1$, $B=\frac 1 2$, $C = -\frac 3 2$, $D=1$, $E=\frac 3 2$, $F=1$, $G=1$, $H=\frac 1 2$.
Abbiamo ottenuto, in definitiva:
\begin{align*}
\frac{x^4+1}{x^3(x-1)(x^2+1)^2}
&= \frac{1}{x} + \frac{\frac 1 2}{x-1} + \frac{1 - \frac 3 2 x}{x^2+1}
+ \enclose{\frac{\frac 3 2 x^3 + x^2 + x + \frac 1 2}{x^2(x^2+1)}}'
\end{align*}
e osservando che
\[
 \int \frac{1 - \frac 3 2 x}{x^2+1}\, dx
 = \int \frac{1}{x^2+1}\, dx - \frac 3 4 \int \frac{2x}{x^2+1}\, dx
 \ni \arctg x - \frac 3 4 \ln(x^2+1)
\]
si ottiene infine
\begin{align*}
\int \frac{x^4+1}{x^3(x-1)(x^2+1)^2}\, dx
 &\ni \ln \abs{x} + \frac 1 2 \ln\abs{x-1} + \arctg x - \frac 3 4 \ln(x^2+1) \\
 &\quad + \frac{\frac 3 2 x^3 + x^2+x+\frac 1 2}{x^2(x^2+1)}
\end{align*}
\end{proof}

Il metodo esposto nell'esempio precedente è un algoritmo 
valido in generale:
\index{algoritmo!per l'integrazione di funzioni razionali}%
\index{integrazione!di funzioni razionali!algoritmo}%
\index{funzione!razionale!integrazione}%
\index{decomposizione!in fratti semplici}%
\index{scomposizione!in fratti semplici}%
\begin{enumerate}
  \item si effettua la divisione con resto per ricondursi
  al caso in cui il numeratore della funzione razionale
  ha grado inferiore al denominatore
  \item si decompone il denominatore come prodotto di
  polinomi lineari e quadratici
  \item si utilizza la scomposizione di Hermite (teorema~\ref{th:Hermite})
  per scrivere la funzione razionale come somma di \emph{fratti semplici}
  più la derivata di un'altra funzione razionale. I coefficienti
  della scomposizione vanno trovati risolvendo un sistema lineare.
  \item si integrano i fratti semplici.
\end{enumerate}

Per concludere l'algoritmo precedente è necessario saper integrare i fratti semplici. Se il denominatore è lineare l'integrale è immediato (si ottiene un logaritmo).

Se il denominatore è invece quadratico l'integrale può essere leggermente più complicato ed è opportuno sapere come procedere. I termini quadratici
saranno della forma:
\begin{equation}\label{eq:0943784}
\frac{Ax + B}{x^2 + \alpha x + \beta}
\end{equation}
con $\alpha^2-4\beta<0$ (in caso contrario il polinomio quadratico
a denominatore può essere scomposto in due fattori lineari, utilizzando
la formula risolutiva per le equazioni di secondo grado~\eqref{eq:secondo_grado}). L'idea è quella di effettuare un cambio
di variabile lineare in modo da ricondursi ad un denominatore
della forma $y^2+1$. Per fare questo si utilizza il metodo del completamento del quadrato come abbiamo già fatto in~\eqref{eq:24589}:
\[
  x^2 + \alpha x + \beta
  = \enclose{x+\frac\alpha 2}^2  + \beta - \frac{\alpha^2}{4}
  = \dots
\]
a questo punto si raccoglie $\beta-\frac{\alpha^2}{4}$
(che per ipotesi è positivo)
e lo si porta dentro il quadrato:
\[
 \dots = \enclose{\beta-\frac{\alpha^2}{4}}\cdot\Enclose{\enclose{\frac{x+\frac \alpha 2}{\sqrt{\beta -\frac{\alpha^2}{4}}}}^2 + 1}
\]
e dunque tramite la sostituzione $y=\frac{x + \frac \alpha 2 }{\sqrt{\beta - \frac{\alpha^2}{4}}}$
abbiamo ricondotto la nostra funzione razionale~\eqref{eq:0943784}
alla forma
\[
  \frac{a y + b}{y^2 + 1} = \frac a 2 \frac{2y}{y^2+1} + b\frac{1}{y^2+1}
\]
che può essere integrata immediatamente.

\subsection{decomposizione in fratti semplici e decomposizione di Hermite}
\index{decomposizione!in fratti semplici}%
\index{scomposizione!in fratti semplici}%
\index{decomposizione!di Hermite}%
\index{scomposizione!di Hermite}%

Nel seguito proponiamo gli enunciati e le dimostrazioni che garantiscono
il funzionamento dell'algoritmo per l'integrazione delle funzioni razionali.
Difficilmente queste dimostrazioni possono essere portate a termine
senza utilizzare un minimo di proprietà delle funzioni complesse. In particolare è necessario sapere che le derivate delle funzioni elementari
complesse (somme, prodotti, rapporti, composizione) si svolgono esattamente
con le stesse regole che si applicano alle funzioni reali (si veda a riguardo il capitolo~\ref{sec:derivata_complessa}).

\begin{lemma}[indipendenza dei fratti semplici]
\label{lemma:72995}
Siano $\lambda_1,$ $\dots,$ $\lambda_n \in \CC$ numeri complessi
distinti e siano $p_1,\dots,p_n \in \NN\setminus\ENCLOSE{0}$.
Sia $\Omega = \CC \setminus \ENCLOSE{\lambda_1, \dots, \lambda_n}$.
Allora le funzioni
$u_{\lambda_k}^{j}\colon \Omega\subset \CC \to \CC$
\begin{equation}
\label{eq:483818}
  u_{\lambda_k}^{j}(z) = \frac{1}{(z-\lambda_k)^{j}}
  \qquad k=1,\dots,n, \quad j=1,\dots, p_k
\end{equation}
sono elementi linearmente indipendenti dello spazio
vettoriale complesso $\CC^\Omega$.
\end{lemma}
%
\begin{proof}
Procediamo per induzione sul numero di funzioni
$N=p_1+ \dots + p_n$.
Se $N=1$ abbiamo una unica funzione che non è identicamente
nulla (anzi: non si annulla mai)
e quindi costituisce un insieme linearmente indipendente.

Supponiamo allora di avere un insieme formato da più funzioni
nella forma~\eqref{eq:483818}.
e di avere
una combinazione lineare complessa identicamente nulla:
\[
\sum_{k=1}^n \sum_{j=1}^{p_k} \alpha_{k,j} u_{\lambda_k}^j = 0.
\]
Consideriamo la funzione $f\colon \Omega \to \CC$
definita da
\begin{equation}
\label{eq:567384}
  f(z)
  = (z-\lambda_n)^{p_n}\sum_{k=1}^n \sum_{j=1}^{p_k} \frac{\alpha_{k,j}}{(z-\lambda_k)^{j}}
  = \sum_{k=1}^n \sum_{j=1}^{p_k} \alpha_{k,j}\frac{(z-\lambda_n)^{p_n}}{(z-\lambda_k)^{j}}.
\end{equation}
Questa funzione è identicamente nulla
in $\Omega$ in quanto ottenuta moltiplicando la combinazione lineare identicamente
nulla per il fattore $(z-\lambda_n)^{p_n}$.
Ma si osserva che si ha
\[
  \lim_{z\to \lambda_n} f(z) = \alpha_{n,p_n}
\]
in quanto se $k\neq n$ si ha
\[
\lim_{z\to \lambda_n}\frac{(z-\lambda_n)^{p_n}}{(z-\lambda_k)^{p_k}} = 0
\]
in quanto $\lambda_k\neq \lambda_n$ e il numeratore
tende a zero, se invece $k=n$ e $j<p_n$ si ha comunque
\[
\lim_{z\to \lambda_n}\frac{(z-\lambda_n)^{p_n}}{(z-\lambda_k)^{j}}
= \lim_{z\to \lambda_n}(z-\lambda_n)^{p_n-j} = 0.
\]
Dunque $\alpha_{n,p_n}=0$.
Ma allora abbiamo una combinazione lineare nulla di $N-1$
funzioni e per ipotesi induttiva possiamo quindi concludere
che anche tutti gli altri coefficienti sono nulli.
\end{proof}

\begin{theorem}[scomposizione complessa in fratti semplici]
\label{th:fratti_semplici_complessi}%
\index{scomposizione!in fratti semplici complessi}%
\index{decomposizione!in fratti semplici complessi}%
\index{fratti semplici!scomposizione complessa}%
Se $\lambda_1, \dots, \lambda_n$ sono le radici complesse
distinte del polinomio $Q$ a coefficienti complessi,
e $p_1,\dots, p_n$ sono le rispettive molteplicità
con $p_1 + \dots + p_n = \deg Q$
e se $P$ è un qualunque polinomio a coefficienti complessi
con $\deg P < \deg Q$,
allora esistono, unici, dei polinomi $R_1, \dots, R_n$
a coefficienti complessi con $\deg R_k < p_k$ tali che
si abbia
\begin{equation}\label{eq:46772341}
  \frac{P(z)}{Q(z)}
  = \sum_{k=1}^n \frac{R_k(z)}{(z-\lambda_k)^{p_k}}
  \qquad \forall z \in \CC\setminus\ENCLOSE{\lambda_1,\dots,\lambda_n}.
\end{equation}
\end{theorem}
%
\begin{proof}
Sia $N=\deg Q$ e consideriamo
l'insieme dei polinomi a coefficienti complessi
di grado inferiore a $N$:
\[
  V_N = \ENCLOSE{P\in \CC[z]\colon \deg P < N}.
\]
Questo è un sottospazio vettoriale di
$V=\CC^\Omega$ ed ha dimensione
$N$ (una base è data dai monomi $1, z, z^2, \dots, z^{N-1}$).
Di conseguenza è facile verificare che anche lo spazio vettoriale
\[
  W_Q = V_N \cdot \frac{1}{Q} = \ENCLOSE{\frac{P}{Q}\colon \text{$\deg P < N$}}
\]
i cui elementi sono le funzioni razionali che stanno al lato sinistro
di~\eqref{eq:46772341},
ha la stessa dimensione $\dim W_Q = \dim V_N = N$.
D'altra parte il lato destro di~\eqref{eq:46772341}
è una combinazione lineare di funzioni
$u_{\lambda_k}^j = 1/(z-\lambda_k)^j$ che sono
anch'esse elementi di $W_Q$ in quanto $(z-\lambda_k)^j$ divide $Q(x)$
se $j\le p_k$.
Ma per il lemma~\ref{lemma:72995} i vettori $u_{\lambda_k}^j$ sono
$N$ vettori indipendenti e quindi
sono una base di $W$.
Significa allora che ogni elemento di $W$ può
essere scritto in modo unico come combinazione lineare degli
$u_{\lambda_k}^j$ cioè, per ogni
polinomio $P$ con $\deg P < \deg Q$ si ha
\begin{equation}\label{eq:458934}
 \frac{P(z)}{Q(z)}
 = \sum_{k=1}^n \sum_{j=1}^{p_k} \frac{A_{k,j}}{(z-\lambda_k)^j}
 = \sum_{k=1}^n \frac{\displaystyle \sum_{j=1}^{p_n} A_{k,j}\cdot (z-\lambda_k)^{p_n-j}}{(z-\lambda_k)^{p_n}}
\end{equation}
e posto
\[
  R_k(z) = \sum_{j=1}^{p_n} A_{k,j}\cdot(z-\lambda_k)^{p_n-j}
\]
è chiaro che $R_k$ è un polinomio di grado inferiore a $p_n$
univocamente determinato dai coefficienti $A_{k,j}$.
\end{proof}


\begin{theorem}[scomposizione di Hermite]
\label{th:Hermite}%
\index{Hermite!scomposizione in fratti semplici}%
\index{teorema!di Hermite}%
Sia $Q(x)$ un polinomio monico a coefficienti reali
con radici reali $\lambda_1, \dots, \lambda_n$ di molteplicità
$p_1, \dots, p_n$ e radici complesse coniugate non reali
$\mu_1, \bar \mu_1, \dots, \mu_m, \bar \mu_m$ di molteplicità
$q_1, \dots, q_m$. Posto $\alpha_k = 2\Re \mu_k$ e $\beta_k = \abs{\mu_k}^2$
potremo scrivere (grazie al teorema~\ref{th:fattorizzazione_polinomio_reale})
\begin{equation}\label{eq:8845638}
  Q(x) = \prod_{k=1}^n (x-\lambda_k)^{p_k} \cdot \prod_{k=1}^m (x^2+ \alpha_k x + \beta_k)^{q_k}.
\end{equation}
Sia $P(x)$ un qualunque polinomio a coefficienti reali
con $\deg P < \deg Q$.

Allora posto
\[
 \tilde Q(x) = \prod_{k=1}^n (x-\lambda_k)^{p_k-1} \cdot \prod_{k=1}^m (x^2+\alpha_k x+\beta_k)^{q_k-1}
\]
esistono (unici) dei coefficienti reali
$A_1,\dots, A_n$, $B_1, \dots, B_m$, $C_1,\dots, C_m$
e un polinomio a coefficienti reali $R(x)$ con $\deg R < \deg \tilde Q$
per cui risulta:
\begin{equation}\label{eq:scomposizione_hermite}
  \frac{P(x)}{Q(x)} = \sum_{k=1}^n \frac{A_k}{x-\lambda_k}
  + \sum_{k=1}^m \frac{B_k x + C_k}{x^2+\alpha_k x + \beta_k}
  + \enclose{\frac{R(x)}{\tilde Q(x)}}'.
\end{equation}
Svolgendo la derivata si potrà
riscrivere il lato destro con denominatore comune $Q(x)$
e uguagliando i numeratori di lato destro e lato sinistro
si otterrà un sistema lineare in $\deg Q$ incognite
che avrà una unica soluzione.
\end{theorem}
%
\begin{proof}
Consideriamo le funzioni
\begin{align*}
   u_\lambda^j(z) &= \frac{1}{(z-\lambda)^j}\\
   v_\mu(z) &= \frac{1}{z^2-2 \Re \mu\cdot z + \abs{\mu}^2},\\
   w_\mu(z) &= \frac{z}{z^2-2 \Re \mu\cdot z + \abs{\mu}^2},\\
   t_j(z) &= \frac{z^j}{\tilde Q(z)}.
\end{align*}
Siano $\lambda_1, \dots, \lambda_N$ tutte le radici
complesse del polinomio $Q(z)$ con $p_1, \dots, p_N$
le rispettive molteplicità.
Grazie al lemma~\ref{lemma:72995} possiamo affermare,
come abbiamo fatto nella dimostrazione del
teorema~\ref{th:fratti_semplici_complessi} che lo
spazio vettoriale $V_Q$ di tutte le funzioni
razionali della forma $\frac{P(z)}{Q(z)}$
con $Q$ fissato e $\deg P<\deg Q$
è generato dalle combinazioni lineari
a coefficienti complessi delle funzioni
\[
  u_{\lambda_k}^j, \qquad k=1, \dots, N, \quad j=1, \dots, p_k.
\]
Analogamente lo spazio vettoriale $V_{\tilde Q}$
è generato dalle funzioni
\[
  u_{\lambda_k}^j, \qquad k=1, \dots, N, \quad j=1, \dots, p_k-1.
\]
Osserviamo ora che per $j>1$ si ha
\[
 \frac {d}{dz} u_\lambda^{j-1} =
 \frac {d}{dz} \frac{1}{(z-\lambda)^{j-1}}
 = \frac{1-j}{(z-\lambda)^j} = (1-j) u_\lambda^j
\]
significa quindi che l'operatore lineare $\frac d {dz}$
manda $V_{\tilde Q}$ in $V_Q$, più precisamente
l'immagine di $\frac{d}{dz} V_{\tilde Q}$ è il sottospazio
di $V_Q$ generato dalle funzioni $u_{\lambda_k}^j$
con $j>1$. Tale operatore
è inoltre iniettivo in quanto manda vettori della
base di $V_{\tilde Q}$ in vettori della base di $V_Q$
che sono certamente indipendenti.
Abbiamo quindi una decomposizione in somma diretta:
\[
  V_Q = V_1 \oplus \frac{d}{dz} V_{\tilde Q}
\]
dove $V_1$ è lo spazio generato
dalle funzioni $u_{\lambda_1}^1, \dots, u_{\lambda_N}^1$.

Ma anche le funzioni $t_0, \dots, t_{M-1}$ con $M=\deg \tilde Q$
sono una base di $V_{\tilde Q}$ dunque una base
di $V_Q$ è data da
\[
  u_{\lambda_1}, \dots, u_{\lambda_N},
  t_0', \dots, t_{M-1}'
\]
dove $t_m'$ è la derivata di $t_m$.
Quindi, visto che ogni combinazione lineare
di $u_{\mu}$ e $u_{\bar \mu}$ può essere scritta
come combinazione lineare di $v_\mu$ e $w_\mu$
si ottiene che un'altra base di $V_Q$ è
\begin{equation}\label{eq:63928}
  u_{\lambda_1}, \dots, u_{\lambda_n},
  v_{\mu_1}, \dots, v_{\mu_m},
  w_{\mu_1}, \dots, w_{\mu_m},
  t_0', \dots, t_{M-1}'.
\end{equation}
Per brevità chiamiamo $e_1, \dots, e_N$
la base di $V_Q$ elencata in~\eqref{eq:63928}.
Ognuna di queste funzioni si scrive come
rapporto di polinomi a coefficienti reali
il cui denominatore è un divisore del polinomio $Q$
che è anch'esso un polinomio a coefficienti reali.
Dunque le funzioni $Q\cdot e_1, \dots, Q\cdot e_N$
sono in realtà dei polinomi a coefficienti reali
e devono necessariamente essere una base dello spazio
vettoriale complesso $Q\cdot V_Q$ che non è altro che l'insieme
di tutti i polinomi a coefficienti complessi
di grado inferiore ad $N$. In particolare questi polinomi
sono indipendenti.
Ma allora, visti come polinomi a
coefficienti reali sono pure indipendenti e sono quindi
una base dello spazio di tutti i polinomi a coefficienti reali
di grado inferiore a $N$.
Dunque devono esistere dei coefficienti reali $\alpha_1, \dots, \alpha_N$
per cui si ha
\[
  P(x) = \sum_{k=1}^N \alpha_k\cdot Q(x)\cdot e_k(x)
\]
da cui
\[
  \frac{P(x)}{Q(x)} = \sum_{k=1}^N \alpha_k \cdot e_k(x).
\]
Ricordando che le funzioni $e_k$ non sono altro che le funzioni
elencate in~\eqref{eq:63928} si ottiene finalmente la decomposizione
cercata.
\end{proof}

