\section{funzioni speciali}

E' certamente molto utile conoscere i metodi di integrazione 
per poter scrivere esplicitamente la primitiva di una 
generica funzione. 
Ma è anche molto utile sapere che esistono alcune funzioni 
elementari la cui primitiva non è una funzione elementare.
Un esempio su tutti è la 
\emph{campana di Gauss}%
\mymargin{campana di Gauss}%
\index{campana!di Gauss}
\index{Gauss!funzione di}%
\index{funzione!gaussiana}%
$f(x) = e^{-x^2}$ la cui funzione integrale 
\[
 F(x) = \int_0^x e^{-t^2}\, dt
\]
non può essere espressa mediante composizione di funzioni 
elementari.
Questo fatto non ci deve scoraggiare più di tanto,
daremo un nome alla funzione $F$ e ne studieremo le proprietà 
utilizzando la teoria che abbiamo sviluppato.
Ovviamente è probabile che prima di noi qualcun'altro 
si sia imbattutto in tali funzioni e gli abbia già dato 
un nome e ne abbia studiato le proprietà. 
Queste funzioni si chiamano usualmente \emph{funzioni speciali}
per distinguerle dalle \emph{funzioni elementari}.
\index{funzione!speciale}%
\index{speciale!funzione}%
\index{funzioni!elementari}%
\index{elementare!funzione}%
Nell'esempio specifico è stata definita la 
\emph{funzione di errore}%
\mymargin{funzione di errore}%
\index{funzione!di errore}
\[
  \erf x = \frac{2}{\sqrt \pi} \int_0^x e^{-t^2}\, dt.
\]

Altri esempi di funzioni la cui primitiva non si esprime mediante 
funzioni elementari sono i seguenti:
\[
  \frac{1}{\ln x}, \qquad 
  \frac{e^x}{x}, \qquad
  \frac{\sin x}{x}, \qquad \sin\frac{\pi x^2}{2}
\]
per le quali si definiscono le rispettive primitive:
\emph{logaritmo integrale}, \emph{integrale esponenziale},
\emph{seno integrale},
\emph{integrale di Fresnel}:
\[
\li x, \qquad 
\ei x, \qquad 
\Si x, \qquad
S x.
\]

Il teorema tramite il quale si dimostra che queste 
funzioni non ammettono una primitiva esprimibile come 
composizione di funzioni elementari (funzioni razionali, 
esponenziali, trigonometriche e loro inverse) si chiama 
Teorema di Liouville.%
\newsavebox{\qrDeLellis}\sbox{\qrDeLellis}{%
\myqrdoclink{http://cvgmt.sns.it/paper/3456/}{}{Il teorema di Liouville}}%
\mynote{%
Il teorema di Liouville è un teorema di tipo algebrico 
e quindi va ben lontano dagli scopi di questo corso.
Per chi fosse interessato a vederne la dimostrazione 
una lettura piacevole può essere l'articolo di Camillo De Lellis: 
\emph{Il teorema di Liouville ovvero perché ``non esiste'' la primitiva
di $\exp(x^2)$}.\\
\usebox{\qrDeLellis}
}

