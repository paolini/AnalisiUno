\section{studio di funzioni integrali}

\begin{theorem}[derivata di una funzione integrale]
Sia $I\subset \RR$ un intervallo e $f\colon I \to \RR$
una funzione localmente Riemann-integrabile.
Siano $a,b\colon A \to I$ funzioni definite
su un insieme $A \subset \RR$.
Allora è ben definita la funzione $F\colon A \to \RR$
\[
  F(x) = \int_{a(x)}^{b(x)} f(t)\, dt.
\]
Se inoltre $f$ è continua e se $a,b$ sono derivabili
allora anche $F$ è derivabile e si ha
\[
  F'(x) = f(b(x)) \cdot b'(x) - f(a(x)) \cdot a'(x).
\]
\end{theorem}
%
\begin{proof}
Per le ipotesi enunciate per ogni $x\in A$ risulta
che $f$ è integrabile sull'intervallo $[a(x),b(x)]$
e dunque la funzione $F$ è ben definita.
Fissato $x_0\in I$
possiamo considerare la funzione integrale
\[
  G(x) = \int_{x_0}^x f(t)\, dt
\]
e scrivere
\[
  F(x)
    = \Enclose{G(x)}_{a(x)}^{b(x)}
    = G(b(x)) - G(a(x)).
\]
Se $f$ è continua possiamo applicare il
teorema fondamentale del calcolo integrale
che garantisce che $G$ è derivabile e $G'(x) = f(x)$
per ogni $x\in I$. Dunque se anche $a$ e $b$
sono derivabili si ha, per la formula di derivazione
della funzione composta:
\[
F'(x)
= G'(b(x)) \cdot b'(x) - G'(a(x)) \cdot a'(x)
= f(b(x)) \cdot b'(x) - f(a(x)) \cdot a'(x).
\]
\end{proof}

\begin{example}
La funzione
\[
 F(x) = \int_{x^2}^{x^4} \frac{1}{\ln t}\, dt
\]
è definita per ogni $x\in \RR\setminus\ENCLOSE{0,1,-1}$ 
in quanto in tal caso l'intervallo
$[x^2,x^4]$ è contenuto nell'insieme di definizione
della funzione integranda.
Si ha inoltre
\[
F'(x)
= \frac{4x^3}{\ln (x^4)} - \frac{2x}{\ln (x^2)}
= \frac{x^3-x}{\ln \abs{x}}.
\]
\end{example}

\begin{theorem}[integrale dell'$o$-piccolo]
Sia $f(x)$ una funzione continua definita
su un intervallo $[0,b]$ e tale che
per $x\to 0^+$ si ha
$f(x) = o(x^\alpha)$ per un qualche $\alpha\ge 0$.
Allora posto
\[
  F(x) = \int_0^x f(t)\, dt
\]
risulta $F(x) = o(x^{\alpha+1})$.
In modo più conciso si può dunque scrivere
\[
  \int_0^x o(t^\alpha)\, dt = o (x^{\alpha+1}),
  \qquad \text{per $x\to 0^+$}
\]
se la funzione integranda è continua.
\end{theorem}
%
\begin{proof}
Per il teorema della media integrale (teorema~\ref{th:media_integrale})
per ogni $x\in [0,b]$ deve esistere $c(x)\in[0,x]$ tale che
\[
  F(x)
  = \int_0^x f(t)\, dt
  = x \cdot f(c(x)).
\]
Dunque, essendo $0\le c(x)\le x$ si ha
\[
\abs{\frac{F(x)}{x^{\alpha+1}}}
= \abs{\frac{f(c(x))}{x^\alpha}}
= \abs{\frac{f(c(x))}{c^\alpha(x)}}
\cdot \abs{\frac{c(x)}{x}}^\alpha
\le \abs{\frac{f(c(x))}{c^\alpha(x)}} \to 0
\]
visto che $f(x) = o(x^\alpha)$
e che per $x\to 0$ anche $c(x)\to 0$
si ha infatti
\[
  \lim_{x\to 0} \frac{f(c(x))}{c^\alpha(x)} = 0.
\]
\end{proof}

\begin{example}
Si voglia calcolare
\[
  \lim_{x\to 0^+} \frac{1}{x^4}\int_{\sin^2 x}^{\sin x} \frac{2- t\sin t - 2 \cos t}{e^t - 1}\, dt.
\]
Chiamata $f(x)$ la funzione integranda non è difficile
verificare, tramite le formule di Taylor, che risulta
\[
  f(x)
  = \frac{\frac{x^4}{12}-o(x^4)}{x+o(x)}
  = \frac{x^3}{12} + o(x^3), \qquad \text{per $x\to 0$}.
\]
In particolare la funzione $f$ può essere estesa per
continuità ponendo $f(0)=0$.
Si ha dunque, per il teorema precedente,
\[
  F(x) = \int_0^x f(t)\, dt
  = \int_0^x \enclose{\frac{t^3}{12}+o(t^3)}\, dt
  = \frac{x^4}{48} + o(x^4)
\]
da cui
\begin{align*}
 \frac{1}{x^4} \int_{\sin^2 x}^{\sin x}
 f(t) \, dt
 &= \frac{F(\sin x) - F(\sin^2 x)}{x^4}\\
 &= \frac{\frac{(\sin x)^4}{48} + o(\sin^4 x) - \frac{(\sin^2 x)^4}{48} + o(\sin^8 x)}{x^4} \\
 &= \frac{\frac{x^4}{48} + o(x^4)}{x^4} \to \frac{1}{48}.
\end{align*}
\end{example}

\begin{example}
Si voglia calcolare
\[
  \lim_{x\to 0^+} \int_{x}^{2x} \frac{1}{t+\sin t}\, dt.
\]
Posto
\[
  f(x) = \frac{1}{x+\sin x}
\]
si ha
\[
 f(x)
  = \frac{1}{2x+o(x^2)} = \frac{1}{2x(1+o(x))}
  = \frac{1+o(x)}{2x} = \frac{1}{2x} + o(1).
\]
La funzione $f(x)-\frac 1{2x} = o(1)$
può essere estesa per continuità anche in $x=0$
e dunque possiamo applicare il teorema precedente
per ottenere
\begin{align*}
  \int_x^{2x} f(t)\, dt
  &= \int_x^{2x} \enclose{\frac 1 {2t} + o(1)}\, dt \\
  &= \frac 1 2 \Enclose{\ln t}_x^{2x} +  \Enclose{o(t)}_x^{2x} \\
  &= \frac {\ln 2x - \ln x} 2 + o(2x) - o(x)
  = \frac {\ln 2} 2 + o(x) \to \frac{\ln 2}{2}.
\end{align*}
\end{example}

\begin{theorem}
Siano $a,b\colon J \to \RR$ due funzioni definite
su un intervallo $J$ a valori in un intervallo $I$.
Sia $x_0$ un punto di accumulazione di $J$ e sia $a$ un punto
di accumulazione di $I$. Supponiamo inoltre
che per $x\to x_0$ si abbia $a(x)\to a$, $b(x)\to a$.
Siano $f,g\colon I \to \RR$ funzioni continue.
Se $f(x)\sim g(x)$ per $x\to a$ allora si ha
\[
  \int_{a(x)}^{b(x)} f(t)\, dt
  \sim \int_{a(x)}^{b(x)} g(t)\, dt
  \qquad \text{per $x\to x_0$}.
\]
\end{theorem}
%
\begin{proof}
Sia $F$ una primitiva di $f$ e $G$ una primitiva di $g$.
Allora
\[
  \frac{\displaystyle\int_{a(x)}^{b(x)}f(t)\, dt}{\displaystyle\int_{a(x)}^{b(x)} g(t)\, dt}
  = \frac{F(b(x)) - F(a(x))}{G(b(x))-G(a(x))}.
\]
Applicando il teorema~\ref{th:cauchy} di Cauchy si ha
\[
\frac{F(b(x)) - F(a(x))}{G(b(x))-G(a(x))}
= \frac{F'(c(x))}{G'(c(x))}
= \frac{f(c(x))}{g(c(x))}
\]
per un certo $c(x)$ compreso tra $a(x)$ e $b(x)$.
Visto che $a(x)\to a$ e $b(x)\to a$ si avrà anche $c(x)\to a$
per $x\to x_0$. Essendo inoltre $f(x)\sim g(x)$ per $x\to a$
facendo un cambio di variabile nel limite possiamo dedurre che
\[
\frac{f(c(x))}{g(c(x))} \to 1 \qquad \text{per $x\to x_0$}.
\]
\end{proof}

\begin{example}
Si voglia calcolare
\[
  \lim_{x\to 0 }\int_{x^2}^{3x^2} \frac{\tg t}{t^2}\, dt.
\]
Visto che per $x\to 0$ si ha
\[
  \frac{\tg x}{x^2} \sim \frac 1 x
\]
e visto che
\[
  \int_{x^2}^{3x^2}\frac{1}{t}\, dt
  = \ln(3x^2) - \ln(x^2) = \ln 3
\]
grazie al teorema precedente
possiamo dedurre che il limite cercato è proprio $\ln 3$.
\end{example}

