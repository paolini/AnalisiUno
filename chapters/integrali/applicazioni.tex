\section{alcune applicazioni del calcolo integrale}

\begin{theorem}[irrazionalità di $\pi$]
\mymargin{$\pi$ è irrazionale}
\index{$\pi$!irrazionalità}
\index{irrazionalità!di $\pi$}
Il numero $\pi$ è irrazionale.
\end{theorem}
%
\begin{proof}
Il nostro primo obiettivo è quello di trovare delle formule
ricorsive per il calcolo del seguente integrale:
\[
  I_n = \int_0^\pi x^n (\pi-x)^n \sin x\, dx.
\]
Innanzitutto osserviamo che si ha
\[
  I_0 = \int_0^\pi \sin x\, dx = \Enclose{-\cos x}_0^\pi = 2
\]
e, utilizzando l'integrazione per parti:
\begin{align*}
  I_1 &= \int_0^\pi x(\pi-x)\sin x\, dx\\
      &= \Enclose{x(\pi-x)(-\cos x)}_0^\pi
       -\int_0^\pi (\pi-2x)(-\cos x)\, dx \\
      &= \int_0^\pi (\pi-2x)\cos x\, dx
      = \pi \int_0^\pi \cos x\, dx - 2 \int_0^\pi x\cos x\, dx \\
      &= - 2 \enclose{\Enclose{x\sin x}_0^\pi - \int_0^\pi \sin x\, dx}
      = 2 \Enclose{\cos x}_0^\pi = 4.
\end{align*}
Per $n\ge 2$ si procede in maniera simile,
la funzione $x^n (\pi-x)^n$ si
annulla agli estremi di integrazione quindi integrando per parti
 si ottiene:
\[
  I_n = \int_0^\pi \Enclose{n x^{n-1}(\pi-x)^n - n x^n (\pi-x)^{n-1}} \cos x\, dx
\]
anche la funzione tra parentesi quadre si annulla
agli estremi dell'intervallo di integrazione e quindi integrando nuovamente
per parti si ottiene:
\begin{align*}
  I_n &= -\int_0^\pi \big[n(n-1) x^{n-2}(\pi-x)^n
    - 2n^2 x^{n-1}(\pi-x)^{n-1} \\
  &\qquad\quad   + n(n-1) x^n(\pi-x)^{n-2}\big] \sin(x)\, dx \\
  &= 2n^2 I_{n-1} - (n^2-n) \int_0^\pi \Enclose{(\pi-x)^2 + x^2}x^{n-2}(\pi-x)^{n-2} \sin x\, dx \\
  &= 2n^2 I_{n-1} - (n^2-n) \int_0^\pi \Enclose{\pi^2 - 2x(\pi-x)}x^{n-2}(\pi-x)^{n-2}\sin x\, dx \\
  &= 2n^2 I_{n-1} - (n^2-n) \pi^2 I_{n-2} + 2(n^2-n) I_{n-1}
\end{align*}
da cui
\begin{align}\label{eq:9530978}
   I_n = (4n^2-2n)I_{n-1} -(n^2-n)\pi^2 I_{n-2}.
\end{align}

Supponiamo ora per assurdo che sia $\pi = \frac p q$ con $p,q\in \NN$ e consideriamo
la successione
\[
   a_n = \frac{q^{2n}}{n!} I_n.
\]
Dalla relazione~\eqref{eq:9530978} deduciamo
un relazione ricorsiva per $a_n$:
\begin{align*}
  a_n &= \frac{q^{2n}}{n!}\enclose{-(n^2-n)\frac{p^2}{q^2} \frac{(n-2)!}{q^{2n-4}}a_{n-2}
  + (4n^2-2n) \frac{(n-1)!}{q^{2n-2}}a_{n-1}} \\
  &= - q^2 p^2 a_{n-2} + (4n-2)q^2 a_{n-1}
\end{align*}
che ci dice, in particolare, che se $a_{n-2}$ e $a_{n-1}$
sono interi allora anche $a_n$ è intero (in quanto somma di prodotti
di numeri interi).
Visto che anche i primi due termini:
\[
  a_0 = I_0 = 2, \qquad
  a_1 = q^2 I_1 = 4 q^2
\]
sono interi possiamo dedurre, per induzione, che
tutti i termini della successione $a_n$ sono interi.

D'altra parte osservando che $y=x(\pi-x)$
è il grafico di una parabola con asse $x=\frac \pi 2$
sappiamo che per ogni $x$ si ha $x(\pi-x) \le \frac{\pi^2}{4}$
e quindi
\[
  0 \le x^n(\pi-x)^n \sin x
    \le x^n (\pi-x)^n \le \frac{\pi^{2n}}{4^n}.
\]
Dunque
\[
  0 \le I_n \le \frac{\pi^{2n+1}}{4^n}
\]
cioè
\[
  0 \le a_n \le \frac{q^{2n}}{n!}\frac{\pi^{2n+1}}{4^n}.
\]
Visto che $n! \gg C^n$, possiamo dunque affermare che $a_n\to 0$.
D'altra parte abbiamo visto che
$a_n \in \ZZ$ e certamente $a_n > 0$ in quanto $I_n \neq 0$ visto che $f_n$ è
una funzione continua, non negativa e non identicamente nulla.
Ma non è
possibile che una successione di numeri interi positivi converga a zero:
abbiamo quindi ottenuto l'assurdo.
\end{proof}

\begin{theorem}[prodotto di Wallis]
\label{th:wallis}%
\mymark{*}%
\mymargin{prodotto di Wallis}%
\index{$\pi$!prodotto di Wallis}%
\index{Wallis!approssimazione di $\pi$}%
\index{formula!di Wallis}%
Si ha
\[
  \frac{\pi}{2}
  = \prod_{k=1}^{+\infty} \frac{(2k)^2}{(2k-1)(2k+1)}
  = \frac{2}{1}
  \cdot \frac{2}{3}
  \cdot \frac{4}{3}
  \cdot \frac{4}{5}
  \cdot \frac{6}{5}
  \cdot \frac{6}{7}
  \cdot \frac{8}{7}
  \cdot \frac{8}{9}
  \dots
\]
E si ottiene di conseguenza la seguente stima asintotica per
il coefficiente \emph{binomiale centrale}%
\mymargin{binomiale centrale}%
\index{binomiale!centrale}
\[
 {2n \choose n} \sim \frac{4^n}{\sqrt{\pi n}}
 \qquad \text{per $n\to +\infty$.}
\]
\end{theorem}
%
\begin{proof}
Consideriamo la successione di integrali:
\[
  I_n = \int_0^\pi \sin^n(x)\, dx.
\]
Essendo $0\le \sin^n(x) \le 1$ per ogni $x\in [0,\pi]$ ed essendo
$\sin^{n+1}(x)\le \sin^n(x)$ per ogni $x\in[0,\pi]$ è chiaro che $I_n$ è una
successione decrescente di numeri positivi.

Da un calcolo diretto troviamo che
\[
  I_0 = \int_0^\pi 1\, dx = \pi, \qquad
  I_1 = \int_0^\pi \sin(x)\, dx = \Enclose {-\cos x}_0^\pi = 2.
\]
Se $n\ge 2$,
integrando per parti troviamo invece una relazione ricorsiva
\begin{align*}
 I_n &= \int_0^\pi \sin^{n-1}(x) \sin x\, dx \\
     &= \Enclose{-\sin^{n-1} x \cos x }_0^\pi + (n-1)\int_0^\pi \sin^{n-2} x \cos^2 x \, dx \\
     &= 0 + (n-1)\int_0^\pi\sin^{n-2}x\cdot (1-\sin^2 x)\, dx \\
     &= (n-1) I_{n-2} - (n-1) I_{n}
\end{align*}
da cui
\begin{equation}
\label{eq:187464}%
  I_n = \frac{n-1}{n} I_{n-2}.
\end{equation}
Questa formula ricorsiva ci permette di calcolare separatamente
i termini pari e dispari della successione
utilizzando la notazione del \emph{semi fattoriale}
(si veda l'osservazione~\ref{rem:doppio_fattoriale}):
\begin{align*}
  I_{2n}
  &= \frac{2n-1}{2n}\cdot \frac{2n-3}{2n-2} \cdots \frac {3}{4}\cdot \frac{1}{2} \cdot I_0
  = \frac{(2n-1)!!}{(2n)!!} \cdot \pi \\
  I_{2n+1} &= \frac{2n}{2n+1}\cdot \frac {2n-2}{2n-1} \cdots
  \frac{4}{5}\cdot \frac{2}{3} \cdot I_1
  = \frac{(2n)!!}{(2n+1)!!}\cdot 2.
\end{align*}

Ricordando che $I_n$ è decrescente si osserva
che vale
\[
  \frac{I_{2n+2}}{I_{2n}}
  \le \frac{I_{2n+1}}{I_{2n}}
  \le \frac{I_{2n}}{I_{2n}} = 1
\]
e grazie a~\eqref{eq:187464}
\[
 \frac{I_{2n+2}}{I_{2n}}
 = \frac{2n+1}{2n+2} \to 1
\]
da cui ottieniamo che $\frac{I_{2n+1}}{I_{2n}}\to 1$.

In conclusione, visto che $\pi$ compare nella formula dei termini pari, ma
non in quella dei termini dispari, e visto che le due espressioni sono
asintotiche possiamo ottenere
la formula per il calcolo di $\frac \pi 2$:
\begin{align*}
\frac{\pi}{2}
&= \frac{\pi}{2} \lim_{n\to +\infty} \frac{I_{2n+1}}{I_{2n}}
= \lim_{n\to +\infty} \frac{\frac{(2n)!!}{(2n+1)!!}}{\frac{(2n-1)!!}{(2n)!!}}
= \lim_{n\to +\infty} \frac{\enclose{(2n)!!}^2}{(2n+1)!!(2n-1)!!} \\
&= \lim_{n\to +\infty} \frac{(2n)(2n)(2n-2)(2n-2) \cdots 4\cdot 4 \cdot 2 \cdot 2}
  {(2n+1)(2n-1)(2n-1)(2n-3)(2n-3)\cdots 3 \cdot 3 \cdot 1}.
\end{align*}

Per ottenere una stima asintotica del coefficiente binomiale centrale
ricordiamo che si ha (osservazione~\ref{rem:doppio_fattoriale})
\begin{align*}
  (2n)!! &= 2^n n! \\
  (2n+1)!! &= \frac{(2n+1)!}{2^n n!}
\end{align*}
dunque la stima asintotica di Wallis si può scrivere come
\begin{align*}
 \sqrt{\frac{\pi}{2}}
 &\sim \frac{(2n)!!}{\sqrt{2n+1}(2n-1)!!}
  = \frac{2^n n!\sqrt{2n+1}}{\frac{(2n+1)!}{2^n n!}} \\
 &= \frac{\enclose{2^n (n!)}^2 \sqrt{2n+1}}{(2n+1)!} 
 \sim \frac{4^n (n!)^2}{\sqrt{2n}(2n)!}
\end{align*}
da cui
\[
{2n \choose n}
= \frac{(2n)!}{(n!)^2}
\sim \frac{4^n}{\sqrt{\pi n}}.
\]
\end{proof}

% \begin{remark}\label{rem:cifre_pi}
% Non è difficile stimare l'errore che si commette nell'approssimare
% $\pi$ tramite la formula di Wallis. Posto
% \[
%   P_n = 2 \prod_{k=1}^n \frac{(2k)^2}{(2k-1)(2k+1)}
% \]
% la successione $P_n$ converge crescendo a $\pi$ e si ha
% \begin{align*}
% \ln\frac{\pi}{P_n}
% &= \sum_{k=n+1}^{+\infty}\ln \frac{(2k)}{(2k-1)(2k+1)}
% = \sum_{k=n+1}^{+\infty}\ln \enclose{1+\frac{1}{4k^2-1}} \\
% &\le \sum_{k=n+1}^{+\infty}\frac{1}{4k^2-1}
% = \frac 1 2 \sum_{k=n+1}^{+\infty}\Enclose{\frac{1}{2k-1} - \frac{1}{2k+1}}\\
% &= \frac 1 2 \cdot \frac{1}{2(n+1)-1}
% = \frac{1}{4n-2}
% \end{align*}
% da cui, usando il fatto che per
% $x$ piccolo vale $e^{2x}-1 \le x$,
% si ha
% \[
%  0
%  \le \pi-P_n
%  = P_n\enclose{\frac{\pi}{P_n} - 1}
%  \le \pi \enclose{e^{\frac{1}{4n-2}}-1}
%  \le \frac{\pi}{2n} \le \frac{1}{n}.
% \]
% Si scopre dunque che la convergenza è piuttosto lenta,
% per calcolare $N$ cifre decimali bisogna moltiplicare
% tra loro $10^N$ termini.
% \end{remark}

\begin{theorem}[formula di Stirling]
\label{th:stirling}%
\mymark{*}%
\mymargin{formula di Stirling}%
\index{formula!di Stirling}%
\index{Stirling formula di}%
\index{fattoriale!formula di Stirling}%
Si ha
\[
  n! \sim \sqrt{2\pi n}\cdot \frac{n^n}{e^n}
  \qquad \text{per $n\to +\infty$.}
\]
\end{theorem}
%
\begin{proof}
Si osservi che la tesi:
\[
  \lim_{n\to +\infty}\frac{\sqrt{n} \cdot n^n}{n!\cdot e^n } = \frac{1}{\sqrt{2\pi}}
\]
è equivalente, passando ai logaritmi, a dimostrare che
\[
   \lim_{n\to +\infty} \enclose{\frac 1 2 \ln n + n \ln n - \ln(n!) - n}
   = -\frac 1 2 \ln (2\pi).
\]
La prima parte della dimostrazione sarà volta a dimostrare che il limite
precedente esiste ed è finito. Alla fine, per trovare il valore esatto,
si utilizzerà la formula di Wallis.
La quantità $n\ln n - n$ è, a meno di una costante additiva, l'integrale
del logaritmo sull'intervallo $[1,n]$.
Abbiamo già visto nell'esempio~\ref{ex:498124}
che $\ln(n!)$ si ottiene approssimando tale integrale con dei rettangoli.
Se procediamo invece con una approssimazione tramite trapezi, otteniamo
una stima più precisa che ci darà il termine aggiuntivo $\frac 1 2 \ln n$,
necessario per far convergere il limite.

Osserviamo in generale che se $f:[a,b]\to \RR$ è una funzione concava allora
il grafico di $f$ è compreso tra la retta passante per gli estremi $(a,f(a))$,
$(b,f(b))$ (retta secante) e una qualunque retta tangente, ad esempio la retta
tangente in $((a+b)/2,f((a+b)/2))$.
Di conseguenza l'area sotto il grafico, cioè $\int_a^b f(x)\, dx$ è compreso tra
le aree dei due corrispondenti trapezi rettangoli. L'altezza di entrambi i
trapezi è pari a $(b-a)$ e l'area si calcola moltiplicando l'altezza per la
media delle basi. Nel caso del trapezio con lato obliquo sulla secante, la media
delle basi è $(f(a)+f(b))/2$, nel caso del trapezio con lato obliquo sulla
retta tangente nel punto medio, la media delle basi è uguale alla sezione nel
punto medio: $f((a+b)/2)$. Si ottiene dunque,
per ogni $f$ concava:
\[
   (b-a)\cdot  \frac{f(a) + f(b)}{2}
   \le \int_a^b f(x) \, dx
   \le (b-a) \cdot f\enclose{\frac{a+b}{2}}.
\]

Applicando queste stime alla funzione $f(x) = \ln x$ nell'intervallo $[k, k+1]$
si ottiene:
\[
\frac{\ln(k) + \ln(k+1)}{2}
\le \int_k^{k+1} \ln x\, dx
\le \ln \enclose{k+\frac 1 2}.
\]

Chiamiamo $a_k$ la differenza tra le prime due quantità.
Chiaramente $a_k\ge 0$
e risulta
\begin{align*}
a_k
 &= \int_k^{k+1} \ln x\, dx - \frac{\ln (k) + \ln(k+1)}{2}\\
 & \le \ln\enclose{k+\frac 1 2} - \frac{\ln(k) + \ln(k+1)}{2} \\
 & = \frac 1 2 \ln\frac{\enclose{k+\frac 1 2}^2}{k(k+1)}
  = \frac 1 2 \ln\enclose{1+\frac{1}{4(k^2+k)}}\\
 & = \frac 1 {8k^2} + o\enclose{\frac 1 {k^2}}
 \qquad \text{per $k\to+\infty$.}
\end{align*}
La serie $\sum a_k$ è dunque convergente ovvero
la successione
\[
  A_n = \sum_{k=1}^n a_k
\]
è convergente. Chiamiamo $\ell$ il limite di $A_n$. Si ha:
\begin{align*}
A_n
&= \sum_{k=1}^{n-1} \int_{k}^{k+1} \ln x\, dx -
\sum_{k=1}^{n-1} \frac{\ln k + \ln(k+1)}{2} \\
&= \int_{1}^{n} \ln x\, dx - \sum_{k=2}^{n-1} \ln k - \frac{\ln 1}{2}
-\frac{\ln n}{2} \\
&= \Enclose{ x \ln x -x}_1^n
- \sum_{k=1}^n \ln k + \frac{\ln n}{2} \\
&= n \ln n - n + 1 - \ln(n!) + \frac {\ln n}{2} \to \ell
 \end{align*}
da cui
\[
e^{A_n} = \frac{n^n\cdot e \cdot \sqrt{n}}{e^n \cdot n!} \to e^\ell
\]
ovvero, posto $c=e/e^\ell$,
\[
  n! \sim c\cdot\frac{n^n \sqrt{n}}{e^n}.
\]
Per determinare la costante incognita $c$ possiamo sfruttare la formula di Wallis
sul coefficiente binomiale centrale:
\begin{align*}
\frac{4^n}{\sqrt{\pi n}}
 &\sim
{2n \choose n}
= \frac{(2n)!}{(n!)^2}
\sim \frac{\displaystyle c \frac{(2n)^{2n}\sqrt{2n}}{e^{2n}}}
{\displaystyle\enclose{c\frac{n^n\sqrt n}{e^n}}^2} \\
&= \frac{2^{2n}\sqrt{2}}{c\sqrt{n}}
= \frac{4^n \sqrt 2}{c\sqrt n}
\end{align*}
da cui si ottiene
\[
  \frac{1}{\sqrt \pi} \sim \frac{\sqrt 2}{c}
\]
e quindi $c=\sqrt{2\pi}$.
\end{proof}