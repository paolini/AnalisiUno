\section{ordini di infinito, equivalenza asintotica}
%%%%%
%%%%%

\begin{definition}[ordine di infinito/infinitesimo]%
  \label{def:ordine_infinito}%
  \mymark{***}%
  \mymargin{ordine di infinito/infinitesimo}%
\index{ordine di infinito/infinitesimo}%
  \index{ordine!di infinito}%
  \index{infinito}%
  \index{infinitesimo}%
  Sia $A\subset \RR$, $f,g\colon A \to (0,+\infty)$.
  Sia $x_0\in \bar \RR$ un punto di accumulazione di $A$.
  \begin{enumerate}
  \item
  Diremo che
  per $x\to x_0$ la funzione $f$ è \emph{molto più piccola}
  della funzione $g$ e scriveremo $f(x) \ll g(x)$ se vale
  \mymargin{$\ll$}%
\index{$\ll$}
  \[
  \frac{f(x)}{g(x)} \to 0, \qquad \text{per $x\to x_0$}
  \]
  diremo invece che $f$ è \emph{molto più grande}
  di $g$ e scriveremo $f(x) \gg g(x)$ se
  \mymargin{$\gg$}%
\index{$\gg$}
  \[
  \frac{f(x)}{g(x)} \to +\infty, \qquad \text{per $x\to x_0$.}
  \]
  \item
  Diremo infine che $f$ e $g$
  sono \emph{asintoticamente equivalenti}%
\mymargin{equivalenza asintotica}%
\index{asintoticamente equivalenti}
  \mymargin{equivalenza asintotica}%
\index{equivalenza!asintotica}%
  per $x\to x_0$
  e scriveremo $f(x) \sim g(x)$ se
  \mymargin{$\sim$}%
\index{$\sim$}
  \[
  \frac{f(x)}{g(x)} \to 1, \qquad \text{per $x\to x_0.$}
  \]
  \end{enumerate}
\end{definition}
  
Ad esempio è facile verificare che se $\alpha > \beta > 0$
allora $x^\alpha \gg x^\beta$ per $x\to +\infty$
mentre $x^\alpha \ll x^\beta$ per $x\to 0^+$.
Analogamente se $a>b>1$ allora $a^x\gg b^x$ per $x\to +\infty$
mentre $a^x \ll b^x$ per $x\to -\infty$.

E' molto facile verificare che le relazioni
$\ll$ e $\gg$ sono una l'inversa dell'altra
e soddisfano la proprietà transitiva
mentre la relazione $\sim$ soddisfa la proprietà simmetrica
e la proprietà transitiva.

\begin{theorem}[ordini di infinito]
\label{th:ordine_infinito}%
\mymargin{ordini di infinito}%
\index{ordini di infinito}%
\mymark{***}%
Siano $a>1$ e $\alpha>0$. Per $n\to +\infty$, $n\in\NN$
si ha
\[
  n^\alpha \ll a^n \ll n! \ll n^n
\]
e per $x\to +\infty$, $x\in \RR$ si ha 
\[
\log_a x \ll x^\alpha \ll a^x.
\]
\end{theorem}
%
\begin{proof}
\mymark{**}
Cominciamo col mostrare che $a^n \ll n!$
applicando il criterio del rapporto alla successione $\frac{a^n}{n!}$:
\[
\frac{\displaystyle \frac{a^{n+1}}{(n+1)!}}{\displaystyle \frac{a^n}{n!}}
= \frac{a^{n+1}}{a^n}\cdot \frac{n!}{(n+1)!}
= a \cdot \frac {1}{n + 1} \to 0 < 1.
\]
Dunque si ha, come richiesto, $a^n / n! \to 0$.
Si procede in modo analogo per mostrare che $n! \ll n^n$:
\begin{align*}
\frac{(n+1)!}{n!}\cdot \frac{n^n}{(n+1)^{n+1}}
&= (n+1) \cdot \enclose{\frac{n}{n+1}}^n \frac {1}{n+1}\\
&= \frac{1}{\enclose{1+\frac 1 n}^n} \to \frac 1 e < 1.
\end{align*}
  
Per dimostrare che
$n^\alpha \ll a^n$
si può procedere con il criterio del rapporto, come nei casi precedenti:
\[
\frac{(n+1)^\alpha}{n^\alpha}\cdot \frac{a^n}{a^{n+1}}
= \frac 1 a \cdot \enclose{\frac{n+1}{n}}^\alpha \to \frac 1 a \cdot 1^\alpha = \frac 1 a < 1
\]
da cui $n^\alpha / a^n \to 0$.

Per $x\in \RR$,
cerchiamo di ricondurci ad una successione a valori interi.
Osserviamo che si ha
\[
\lfloor x \rfloor
\le x
\le \lfloor x \rfloor + 1
\]
da cui, per monotonia,
\[
\lfloor x \rfloor^\alpha
\le x^\alpha
\le (\lfloor x \rfloor + 1)^\alpha
= \lfloor x \rfloor^\alpha \enclose{1+ \frac{1}{\lfloor x \rfloor}}^\alpha
\]
e
\[
a^{\lfloor x \rfloor}
\le a^{x}
\le a^{\lfloor x \rfloor + 1}
= a \cdot a^{\lfloor x \rfloor}.
\]
Dunque
\[
\frac{\lfloor x \rfloor^\alpha}{a \cdot a^{\lfloor x \rfloor}}
\le \frac{x^\alpha}{a^{x}}
\le \frac{\lfloor x \rfloor^\alpha \enclose{1+ \frac{1}{\lfloor x \rfloor}}^\alpha}
    {a^{\lfloor x \rfloor}}.
\]
Ma ora, se $x\to +\infty$ sapendo che $n = \lfloor x\rfloor \to +\infty$ 
possiamo effettuare un cambio di variabile nel limite
\[
\lim_{x\to +\infty} \frac{\lfloor x \rfloor^\alpha}{a^{\lfloor x \rfloor}} 
= \lim_{n\to+\infty} \frac{n^\alpha}{a^n} = 0
\]
da cui segue che $\frac{x^\alpha}{a^{x}}\to 0$.

Per dimostrare l'ultima relazione, $\log_a x\ll x^\alpha$,
operiamo il cambio di variabile $y = \alpha \cdot \log_a x$
cosicché $a^y = x^\alpha$.
Notiamo che se $x\to +\infty$
anche $y \to +\infty$.
Dunque, per le proprietà precedenti,
sappiamo che $y \ll a^y$ e dunque
\[
\frac{\log_a x}{x^\alpha}
= \frac{1}{\alpha}\cdot\frac{y}{a^{y}} \to 0.
\]
\end{proof}

Le notazioni e gli
ordini di infinito individuati nel teorema precedente
sono strumenti molto utili nel calcolo dei limiti.

L'equivalenza asintotica
si mantiene per prodotto e rapporto:
se $f\sim F$ e $g\sim G$ allora
\[
 f \cdot g \sim F \cdot G,
 \qquad
 \frac{f}{g} \sim \frac{F}{G}.
\]
Osserviamo inoltre che se
$f \sim g$ e se $f\to \ell$ allora
anche $g\to \ell$.
Se poi $\ell\in(0,+\infty)$
la relazione $f\sim \ell$ è equivalente ad $f\to \ell$.

Per quanto riguarda la somma
è facile verificare che se $f\ll g$ allora
$(f+g) \sim g$ in quanto
\[
  \frac{f + g}{g} = \frac{f}{g} + 1 \to 1.
\]

In un limite in cui compaiono somme di termini
di ordini diversi potremo allora raccogliere i termini di ordine
massimo per individuare il limite, come facciamo
nel seguente.

\begin{example}
Calcolare il limite
\[
\lim_{n\to+\infty}
\frac{2n^4 + 3^n - 3 \ln n}{n! - 3\sqrt n}.
\]
\end{example}
\begin{proof}[Svolgimento.]
Si ha
\[
\frac{2n^4 + 3^n - 3 \ln n}{n! - 3\sqrt{n}}
= \frac
{3^n \cdot \enclose{2\frac{n^4}{3^n}+ 1 - 3\frac{\ln n}{3^n}}}
{n!\cdot \enclose{1-3\frac{\sqrt n}{n!}}}
\]
e ricordando che risulta (teorema~\ref{th:ordine_infinito})
\[
n^4 \ll 3^n, \qquad
\ln n \ll 3^n, \qquad
\sqrt n \ll n!, \qquad
3^n \ll n!
\]
avremo
\[
\frac{2n^4 + 3^n - 3 \ln n}{n! - 3\sqrt{n}}
\sim \frac{3^n}{n!} \to 0.
\]
\end{proof}


\begin{exercise}
Calcolare i seguenti limiti
\begin{gather*}
  \lim_{n\to +\infty} \frac{\displaystyle \ln\sqrt{n^2+n^n}}
  {\displaystyle e^{1 + \ln n}\cdot \ln(n^2-n\sqrt n)}, \qquad
  \lim_{n\to +\infty} \frac{\sqrt{n! + 2^n}}{3^n}, \\
  \lim_{n\to +\infty} \frac{\sqrt{(2n)!}}{n^n}, \qquad
  \lim_{n\to +\infty} \sqrt[n]{e^n + \sqrt{10^n}}.
\end{gather*}
\end{exercise}

\begin{exercise}
  Dimostrare che $\sqrt[n]{n}\to 1$ per $n\to +\infty$.
\end{exercise}



%%%%%%%%%%%
%%%%%%%%%%%
