\section{scambio della derivata con il limite}

\begin{theorem}[scambio del limite con la derivata]
\mymark{***}
Sia $I\subset \RR$ un intervallo e siano $f_k\in C^1(I)$ funzioni tali che $f_k(x_0)$ converge per almeno un punto $x_0\in I$ e la successione delle derivate $f_k'$ converge
ad una funzione $g\colon I \to \RR$
uniformemente su ogni intervallo chiuso e limitato $[a,b]\subset I$. Allora esiste $f\in C^1(I)$ tale che $f'=g$ e $f_k$ converge a $f$ uniformemente su ogni intervallo chiuso e limitato $[a,b]\subset I$.
In queste ipotesi si può quindi scambiare la derivata con il limite:
\[
  \lim_{k\to +\infty}\enclose{\frac{d}{dx} f_k(x)}
  = f'(x)
  = \frac{d}{dx} \enclose{\lim_{k\to +\infty} f_k(x)},
  \qquad \forall x \in I.
\]
\end{theorem}
%
\begin{proof}
\mymark{***}
Per ipotesi esiste $y_0\in \RR$ tale che $f_k(x_0) \to y_0$.
Definiamo
\[
  f(x) = y_0 + \int_{x_0}^x g(t)\, dt.
\]
Per la continuità del limite uniforme sappiamo che $g$ è continua, dunque possiamo applicare il teorema fondamentale del calcolo per dedurre che $f'=g$. Mostriamo ora che su ogni intervallo $[a,b]\subset I$ si ha $f_k \To f$. Per la formula fondamentale del calcolo integrale si ha:
\[
  \int_{x_0}^x f_k'(t) dt = f_k(x) - f_k(x_0)
\]
dunque
\begin{align*}
  \sup_{x\in [a,b]}\abs{f_k(x) - f(x)}
  &= \sup_{x\in [a,b]} \abs{f_k(x_0) + \int_{x_0}^x f_k'(t) - y_0 - \int_{x_0}^x g(t)\, dt} \\
  &\le \abs{f_k(x_0) - y_0} + \sup_{x\in [a,b]}\abs{\int_{x_0}^x \abs{f_k'(t) - g(t)}\, dt} \\
  &\le \abs{f_k(x_0) - y_0} + (b-a)\Abs{f_k' - g} \to 0.
\end{align*}
\end{proof}

Lo spazio $C^1([a,b])$ è un sottospazio vettoriale di $C^0([a,b])$ ma non è chiuso, come si deduce dall'esempio~\ref{ex:466533} (si potrebbe anzi dimostrare che $C^1$ è denso in $C^0$) dunque $C^1$ non è completo rispetto alla norma uniforme.
Per trasformare lo spazio $C^1([a,b])$ in uno spazio di Banach
possiamo definire una norma più forte, come ad esempio
questa:
\[
  \Abs{f}_{C^1} = \Abs{f}_\infty + \Abs{f'}_\infty.
\]
\begin{theorem}[$C^1$ spazio di Banach]
\mymark{*}
Lo spazio vettoriale $C^1([a,b])$ dotato della norma $\Abs{\cdot}_{C^1}$ risulta essere uno spazio di Banach.
\end{theorem}
%
\begin{proof}
E' facile verificare che $\Abs{\cdot}_{C^1}$ è una norma su $C^1([a,b])$, dobbiamo solo verificare che lo spazio risulta completo. Sia dunque $f_k$ una successione di Cauchy rispetto alla norma $C^1$. Allora $f_k'$ e $f_k$ sono entrambe successioni di Cauchy in $C^0$ in quanto $\Abs{f_k}_\infty \le \Abs{f_k}_{C^1}$ e $\Abs{f_k'}_\infty \le \Abs{f_k}_{C^1}$.
Dunque, per la completezza di $C^0$, sappiamo che esistono $f,g\in C^0([a,b])$ tali che $f_k\To f$ e $f_k'\To g$.
In base al teorema di scambio del limite con la derivata possiamo affermare che $f\in C^1$ e $f'=g$, dunque
\[
  \Abs{f_k-f}_{C^1} = \Abs{f_k-f}_\infty + \Abs{f_k'-g}_\infty \to 0.
\]
\end{proof}

Il teorema di scambio del limite con l'integrale ci dice che
l'operatore integrale $S\colon C^0 \to C^1$ è continuo tra i due spazi di Banach. Anche l'operatore differenziale $D\colon C^1 \to C^0$ $f\mapsto Df = f'$ è ovviamente continuo.

\subsection{serie di funzioni}

Se $f_k\colon A \to \RR$ è una successione di funzioni
definite su uno stesso insieme $A$, possiamo considerare (come abbiamo già fatto per le successioni numeriche) la successione delle somme parziali:
\[
  S_n(x) = \sum_{k=0}^n f_k(x), \qquad x\in A.
\]
Tale successione si chiama \emph{serie} corrispondente alla successione di funzioni $f_k$
e si indica a volte come $\sum f_n$. Per ogni $x$ in cui la serie è convergente si può quindi definire la
\emph{somma}%
\mymargin{somma}\index{somma} della serie
\[
  S(x) = \sum_{k=0}^{+\infty} f_k(x) = \lim_{n\to +\infty} S_n(x).
\]
La somma $S$ è dunque il limite puntuale della successione delle somme parziali $S_n$.

I teoremi che abbiamo dimostrato per le successioni di funzioni sono quindi validi anche per le serie di funzioni. Basterà ricordare che la \emph{convergenza uniforme della serie}
\mymargin{convergenza uniforme di una serie}%
\index{serie!convergenza uniforme}%
\index{convergenza!uniforme di una serie}%
 è la convergenza uniforme delle somme parziali. Dunque $\sum f_k$ converge uniformemente a $S$ se $S_n \To S$ ovvero se
\[
  \Abs{S - S_n}_\infty = \Abs{\sum_{k=n+1}^{+\infty} f_k}_\infty \to 0
  \qquad \text{per $n\to +\infty$.}
\]



\begin{theorem}[integrale di una serie di funzioni]
\label{th:integrale_serie}
\mymark{**}%
\index{teorema!integrazione di una serie di funzioni}%
\index{serie!integrale}%
\mymargin{integrale di una serie}%
\index{integrale!di una serie}%
Sia $f_k\colon [a,b]\to\RR$ una successione di funzioni continue definite sull'intervallo $[a,b]\subset \RR$.
Se la serie $\sum f_k$ converge uniformemente
allora si può scambiare l'integrale con la somma della serie:
\[
  \int_a^b \enclose{\sum_{k=0}^{+\infty} f_k(t)}\, dt
  = \sum_{k=0}^{+\infty} \enclose{\int_a^b f_k(t)\, dt}
  \qquad \forall x \in I.
\]
\end{theorem}
\begin{proof}
\mymark{**}
La dimostrazione è una semplice conseguenza del fatto che lo scambio 
può essere fatto sulle somme finite e il passaggio al limite 
può essere fatto grazie al teorema di scambio del limite con l'integrale.

Sia $S_n = \sum f_n$ la successione delle somme parziali e sia $S$ 
il limite delle somme parziali. Per ipotesi $S_n\To S$. 
Applicando il teorema di scambio dell'integrale con il limite si ha
\[
  \lim_{n\to +\infty} \int_a^b S_n(t)\, dt = \int_a^b S(t)\, dt.
\]
Ma da un lato, sfruttando l'additività dell'integrale sulle somme finite:
\begin{align*}
  \lim_{n\to +\infty} \int_a^b S_n(t)\, dt
   &= \lim_{n\to+\infty}\int_a^b \enclose{\sum_{k=0}^n f_k(t)} \,  dt\\
   &= \lim_{n\to+\infty}\sum_{k=0}^n \enclose{\int_a^b f_k(t)\, dt}\\
   &= \sum_{k=0}^\infty \enclose{\int_a^b f_k(t)\, dt}
\end{align*}
e dall'altro lato:
\[
  \int_a^b S(t)\, dt = \int_a^b \enclose{\sum_{k=0}^{+\infty} f_k(t)}\, dt.
\]
\end{proof}

\begin{theorem}[derivata di una serie di funzioni]
\mymark{**}
\index{teorema!derivazione di una serie di funzioni}
\index{serie!derivata}
\mymargin{derivazione di una serie}
Sia $f_k\colon I\to\RR$ una successione di funzioni continue definite sull'intervallo $I$. 
Se le funzioni $f_k$ sono di classe $C^1$ e la serie delle derivate $\sum f_k'$ converge uniformemente
su ogni intervallo chiuso e limitato $[a,b]\subset I$
e se c'è almeno un punto $x_0\in I$ tale che la serie
$\sum f_k(x_0)$ converge, allora
\[
  \frac{d}{dx} \sum_{k=0}^{+\infty} f_k(x) = \sum_{k=0}^{+\infty} \frac{d}{dx}f_k(x)
  \qquad \forall x \in I.
\]
\end{theorem}

\begin{proof}
\mymark{**}
Sia $S_n$ la successione delle somme parziali. Per ipotesi sappiamo che esiste una funzione $T\colon I \to \RR$ tale che $S_n' \To T$ in ogni intervallo $[a,b]\subset I$.
Sappiamo inoltre che $S_n(x_0)$ converge.
Dunque possiamo applicare il teorema di scambio del limite con la derivata per ottenere che esiste $S\in C^1(I)$ tale che
 $S_n(x)\to S(x)$ per ogni $x\in I$ e
\[
   S'(x) = T(x) \qquad \forall x\in I.
\]
Ma da un lato
\begin{align*}
S'(x)
&= \frac{d}{dx} \lim_{n\to +\infty} S_n(x) \\
&= \frac{d}{dx} \sum_{k=0}^{+\infty} f_k(x)
\end{align*}
e dall'altro lato
\begin{align*}
T(x)
&= \lim_{n\to +\infty} S_n'(x)
 = \lim_{n\to +\infty} \frac{d}{dx} \sum_{k=0}^n f_k(x) \\
&= \lim_{n\to +\infty} \sum_{k=0}^n f_k'(x)
 = \sum_{k=0}^{+\infty} f_k'(x).
\end{align*}
\end{proof}

La convergenza uniforme di una serie non è molto semplice da verificare. Più semplice è la seguente condizione, che vedremo essere più forte.

\begin{definition}[convergenza totale di una serie di funzioni]
\mymark{***}
Siano $f_k\colon A \to \RR$ funzioni definite su un insieme $A\subset \RR$. Diremo che la serie di funzioni $\sum f_k$
\emph{converge totalmente}
\mymargin{convergenza totale}
\index{convergenza!totale}
se la serie numerica $\sum \Abs{f_k}_\infty$
è convergente.
\end{definition}

\begin{theorem}[convergenza totale]
\mymark{***}
Se la serie $\sum f_n$ converge totalmente allora converge uniformemente.
\end{theorem}
%
\begin{proof}
\mymark{***}
A $x$ fissato
la serie $\sum f_n(x)$ converge assolutamente in quanto
\[
  \sum_{k=0}^\infty \abs{f_n(x)}
  \le \sum_{k=0}^\infty \Abs{f_n}_\infty < +\infty.
\]
Dunque la serie converge e posto
\[
  S_n(x) = \sum_{k=0}^n f_k(x), \qquad
  S(x) = \sum_{k=0}^{+\infty} f_k(x)
\]
si ha che $S_n\to S$ puntualmente.
Per mostrare che $S_n \To S$ basta osservare che per
$n\to +\infty$ si ha:
\[
  \abs{S(x) - S_n(x)}
  = \abs{\sum_{k=n+1}^{+\infty} f_k(x)}
  \le \sum_{k=n+1}^{+\infty}\abs{f_k(x)}
  \le \sum_{k=n+1}^{+\infty}\Abs{f_k}_\infty \to 0.
\]
\end{proof}

\begin{theorem}[convergenza totale delle serie di potenze]
\label{th:convergenza_totale}
\mymark{***}%
Sia $\sum a_n z^n$ una serie di potenze e sia $R\in[0,+\infty]$ il suo raggio di convergenza. Allora la serie converge totalmente su ogni disco $D_r$ con $r<R$.
\end{theorem}
%
\begin{proof}
Ora osserviamo che sul disco di raggio $r$ si ha $\Abs{a_k z^k}_\infty = \abs{a_k} r^k$ e dunque
\[
\sum_{k=0}^{+\infty} \Abs{a_k z_k}_\infty
= \sum_{k=0}^{+\infty} \abs{a_k}r^k < +\infty
\]
in quanto la serie $\sum a_k z^k$ converge assolutamente per $z=r$ essendo $r<R$.
\end{proof}

\begin{corollary}
\label{cor:derivata_serie_potenze}
\mymark{**}
La serie di potenze
\[
  f(x) = \sum_{k=0}^{+\infty} a_k x^k
\]
ha lo stesso raggio di convergenza $R$ della serie delle derivate
\[
  g(x) = \sum_{k=1}^{+\infty} k a_k x^{k-1}
\]
e per $x\in (-R,R)$ si ha
\[
  f'(x) = g(x).
\]
\end{corollary}
%
\begin{proof}
Che le due serie abbiano lo stesso raggio di convergenza l'abbiamo già dimostrato nel Teorema~\ref{th:raggio_serie_derivate}. Nel teorema precedente abbiamo mostrato che su ogni intervallo $[-r,r]$ con $r<R$ la serie di potenze con somma $f$ converge totalmente. Dunque converge uniformemente e possiamo scambiare la derivata con la somma per ottenere $f'(x) = g(x)$.
\end{proof}

\begin{example}
Sappiamo che la serie di potenze
\[
f(x) = \sum_{k=1}^{+\infty} \frac{x^k}{k}
\]
ha raggio di convergenza $R=1$ (si usi ad esempio il criterio del rapporto). La serie delle derivate è
\[
 g(x) = \sum_{k=1}^{+\infty} x^{k-1} = \sum_{k=0}^{+\infty}x^k = \frac{1}{1-x}.
\]
Dunque per $\abs{x}<1$ si ha
\[
  f'(x) = g(x) = \frac{1}{1-x}
\]
da cui
\[
 f(x) = f(0) + \int_0^x f'(t)\, dt = \int_0^x \frac{1}{1-t}\, dt
  = \Enclose{-\ln(1-t)}_0^x = -\ln(1-x).
\]
Abbiamo quindi scoperto che vale
\[
  \sum_{k=1}^{+\infty} \frac{x^k}{k} = - \ln (1-x) \qquad\text{per ogni $x\in (-1,1)$.}
\]
Osserviamo ora che la serie con somma $f(x)$
non converge per $x=1$ (serie armonica) ma
converge per $x=-1$ (criterio di Leibniz).
Per il Teorema~\ref{th:lemma_abel} (lemma di Abel)
sappiamo che la funzione $f(x)$ è continua nel punto $x=-1$ e dunque possiamo concludere che
\[
  \sum_{k=1}^{+\infty} \frac{x^k}{k} = -\ln(1-x)
  \qquad \forall x \in [-1,1).
\]
In particolare abbiamo trovato la somma della serie armonica a segni alterni:
\[
  \sum_{k=1}^{+\infty} \frac{(-1)^{k+1}}{k} = - f(x) = \ln 2.
\]

Osserviamo che queste informazioni sono coerenti con lo sviluppo di Taylor di $\ln(1+x)$ che avevamo già determinato. Ma non sono conseguenza di esso, in quanto lo sviluppo di Taylor ci dà informazioni solamente per $x\to 0$ mentre ora abbiamo ottenuto informazioni per ogni $x$ in $[-1,1)$.
 \end{example}

\begin{example}
Applichiamo l'idea precedente alla funzione $\arctg$. Si ha
\[
  \arctg'(x) = \frac{1}{1+x^2} = \sum_{k=0}^{+\infty} (-x^2)^k
  = \sum_{k=0}^{+\infty} (-1)^k x^{2k}
  = \sum_{k=0}^{+\infty} \frac{(-1)^k}{2k+1}(x^{2k+1})'.
\]
La serie
\[
 f(x) = \sum_{k=0}^{+\infty}\frac{(-1)^k}{2k+1} x^{2k+1}
\]
ha raggio di convergenza $R=1$ e dunque per ogni $x\in(-1,1)$ sappiamo che la serie delle derivate converge alla derivata della serie da cui
\[
  f'(x) = \arctg' x.
\]
Visto che $f(0) = 0 = \arctg 0$ possiamo concludere che $f(x) =\arctg x$ per ogni $x\in (-1,1)$.
La serie è convergente anche per $x=1$, per il criterio di Leibniz. Per continuità (Lemma di Abel) si ottiene che $f(1) = \arctg 1$.
Dunque
\[
  \arctg x = \sum_{k=0}^{+\infty}\frac{(-1)^k}{2k+1}x^{2k+1}
  \qquad \forall x \in (-1,1].
\]
In particolare per $x=1$ si ottiene la formula di \emph{Gregory-Leibniz}%
\mymargin{Gregory-Leibniz}\index{Gregory-Leibniz}
\index{$\pi$!formula di Gregory-Leibniz}
\index{Gregory!approssimazione $\pi$}
\index{Leibniz!approssimazione $\pi$}
\index{formula!di Gregory-Lebniz per $\pi/4$}
\[
  \frac{\pi}{4} = \sum_{k=0}^{+\infty} \frac{(-1)^k}{2k+1} =
   1 - \frac{1}{3} + \frac{1}{5} - \frac{1}{7} + \dots
\]
\end{example}

