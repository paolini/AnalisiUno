\section{convergenza uniforme}

\begin{definition}[convergenza uniforme]
\mymark{***}
Sia $A$ un insieme non vuoto e
$f\colon A \to \RR$.
Definiamo la \emph{norma uniforme}%
\mymargin{norma uniforme}\index{norma!uniforme} (o norma del $\sup$)
di $f$ come
\[
  \Abs{f}_\infty = \sup_{x\in A} \abs{f(x)}
\]

Se anche $g\colon A \to \RR$
definiamo la \emph{distanza uniforme}
\mymargin{distanza uniforme}
\index{distanza!uniforme}
tra $f$ e $g$ come
\[
  d_\infty(f,g) = \sup_{x\in A} \abs{f(x)-g(x)}.
\]

Se $f_k$ è una successione di funzioni e $f$ è una funzione, diremo che $f_k$
\emph{converge uniformemente}
\mymargin{convergenza uniforme}%
\index{convergenza!uniforme}%
a $f$
e scriveremo
\[
f_k \To f
\] se
$d_\infty(f_k,f)\to 0$.
\end{definition}


\begin{example}
\label{ex:466533}
La successione
\[
f_k(x) = \sqrt{x^2 + \frac{1}{k}}
\]
converge uniformemente (su tutto $\RR$) alla funzione $f(x) = \abs{x}$. Infatti posto $g_k(x) = f_k(x) - f(x)$, 
la funzione $g_k$ risulta essere derivabile per $x\neq 0$ e per $x>0$ si ha
\[
  g'_k(x) = \frac{x}{\sqrt{x^2+\frac 1 k}} - 1 < 0.
\]
Dunque la funzione $g_k$ è decrescente su $[0,+\infty)$. Per simmetria (è una funzione pari) è crescente su $(-\infty, 0]$. 
Risulta quindi che il massimo di $g_k$ è in $x=0$. Chiaramente $g_k \ge 0$ quindi si ha:
\[
  \Abs{f_k - f}_\infty = \sup_{x\in \RR} g_k(x) = g_k(0) = \frac{1}{\sqrt k} \to 0.
\]
Dunque $f_k \To f$.
\end{example}

Osserviamo che in generale $\Abs{f}_\infty$ e $d_\infty(f,g)$ possono assumere il valore $+\infty$ (ad esempio se $A=\RR$, $f(x)=x$ e $g(x)=0$)
e quindi non è detto che siano effettivamente
una norma e una distanza.

\begin{theorem}[proprietà della norma uniforme]
La norma uniforme soddisfa tutte le proprietà di una norma
(Definizione~\ref{def:norma}), salvo il fatto che può assumere valori in $[0,+\infty]$ invece che in $[0,+\infty)$.
\end{theorem}
%
\begin{proof}
Chiaramente la norma uniforme non assume valori negativi in quanto estremo superiore di un insieme (non vuoto) di numeri reali non negativi. Inoltre se $\Abs{f}_\infty=0$ significa che $\abs{f(x)}=0$ per ogni $x$ e dunque $f=0$ (proprietà di separazione).

L'omogenità segue dall'omogeneità del valore assoluto, in quanto si ha
\[
  \sup_{x\in A} \abs{(\lambda \cdot f)(x)}
  = \sup_{x\in A}\abs{\lambda \cdot f(x)}
  = \sup_{x\in A}\abs{\lambda}\cdot \abs{f(x)}
  = \abs{\lambda} \cdot \sup_{x\in A}\abs{f(x)}.
\]

La disuguaglianza triangolare segue dalla disuguaglianza triangolare del valore assoluto, che viene preservata facendone l'estremo superiore:
\[
  \sup_{x\in A} \abs{f(x)+g(x)}
  \le \sup_{x\in A} \Enclose{\abs{f(x)} + \abs{g(x)}}
  \le \sup_{x\in A} \abs{f(x)} + \sup_{x\in A} \abs{g(x)}.
\]
\end{proof}

\begin{theorem}[completezza delle funzioni limitate]%
  \label{th:limitate_completo}%
Sia $A$ un insieme.
Lo spazio vettoriale
delle funzioni limitate $f\colon A \to \RR$
(cioè delle funzioni con norma uniforme finita)
\[
  \B(A) = \ENCLOSE{f\in \RR^A\colon \Abs{f}_\infty < +\infty }
\]
dotato della norma uniforme $\Abs{\cdot}_\infty$ risulta essere uno spazio di Banach (ovvero uno spazio vettoriale normato e completo).
Su tale spazio di Banach la distanza indotta dalla norma è la distanza uniforme $d_\infty$ e la convergenza indotta dalla distanza è la convergenza uniforme.
\end{theorem}
%
\begin{proof}
Per definizione risulta verificato che la norma uniforme $\Abs{\cdot}_\infty$ assume valori finiti su $\B(A)$.
Dunque, in base al teorema precedente, $\Abs{\cdot}_\infty$ è effettivamente una norma e $\B(A)$ risulta quindi essere uno spazio normato. Dimostriamo ora che esso è completo, cioè che le successioni di Cauchy convergono.

Sia $f_k$ una successione di Cauchy in $\B(A)$.
Allora per ogni $x\in A$ risulta che $f_k(x)$ è una successione di Cauchy in $\RR$ in quanto si ha (per definizione di $\sup$)
\[
  \abs{f_k(x) - f_j(x)} \le \Abs{f_k - f_j}_\infty
\]
e quindi se $\Abs{f_k- f_j}_\infty < \eps$
a maggior ragione per $x\in A$ fissato si ha $\abs{f_k(x)-f_j(x)} < \eps$.

Dunque per ogni $x\in A$ la successione numerica $f_k(x)$ converge in quanto $\RR$ è completo. Posto $f(x) = \lim f_k(x)$ abbiamo dunque trovato un candidato limite della successione.
Dovremo ora mostrare che $f\in \B(A)$ e che $f_k$ converge uniformemente a $f$.
Per ogni $\eps>0$ per la condizione di Cauchy dovrà esistere $N\in \NN$ tale che se $k,j>N$ allora
\[
  d_\infty(f_k,f_j) < \eps.
\]
Ma allora per ogni $x\in A$, per ogni $k>N$ e per ogni $j>N$ si avrà:
\[
  \abs{f_k(x) - f(x)} \le \abs{f_k(x) - f_j(x)} +
  \abs{f_j(x) - f(x)} < \eps + \abs{f_j(x)-f(x)}.
\]
Visto che per ogni $x$ si ha $f_j(x) \to f(x)$, per ogni $x$ esiste un $j$ tale che $\abs{f_j(x)-f(x)} < \eps$ e quindi possiamo concludere che
\[
  \abs{f_k(x)-f(x)} < 2\eps.
\]
Facendo il $\sup$ per $x\in A$ si ottiene dunque
\[
  \Abs{f_k -f}_\infty \le 2 \eps.
\]
Abbiamo quindi verificato la definizione di limite $\Abs{f_k -f}_\infty\to 0$. In particolare $\Abs{f}_\infty < +\infty$ in quanto vale la disuguaglianza triangolare
\[
  \Abs{f}_\infty \le \Abs{f-f_k}_\infty + \Abs{f_k}_\infty < +\infty
\]
essendo $\Abs{f-f_k}_\infty \to 0$ e $\Abs{f_k}_\infty < +\infty$.
\end{proof}

\begin{definition}[convergenza puntuale]
\mymark{***}
Sia $f_k\colon A \to \RR$ una successione di funzioni
e sia $f\colon A \to \RR$ una funzione.
Se per ogni $x\in A$ si ha $f_k(x)\to f(x)$ diremo che
la successione $f_k$
\emph{converge puntualmente}
\mymargin{convergenza puntuale}
\index{convergenza!puntuale}
ad $f$.
\end{definition}

\begin{theorem}[convergenza uniforme implica convergenza puntuale]
\mymark{***}
Sia $f_k\colon A \to \RR$ una successione di funzioni.
Se $f_k$ converge uniformemente ad una funzione $f$ allora $f_k$ converge puntualmente ad $f$.
\end{theorem}
%
\begin{proof}
E' sufficiente osservare che per ogni $x\in A$ si ha
\[
  \abs{f_k(x)-f(x)} \le \sup_{y\in A} \abs{f_k(y)-f(y)}
   = \Abs{f_k-f}_\infty \to 0.
\]
\end{proof}

\begin{example}[successione che converge puntualmente ma non uniformemente]
\mymark{***}%
\label{ex:puntuale_non_uniforme}%
Sia $f_k\colon [0,1]\to \RR$ la successione di funzioni definita da $f_k(x)=x^k$. Se $x\in[0,1)$ si ha $x^k \to 0$ mentre se $x=1$ si ha $x^k \to 1$. 
Dunque la successione $f_k$ converge puntualmente alla funzione
\[
f(x) =
 \begin{cases}
  0 & \text{se $x\in [0,1)$}\\
  1 & \text{se $x=1$}.
 \end{cases}
\]
Osserviamo però che
\[
  d_\infty(f_k,f) = \sup_{x\in [0,1]} \abs{f_k(x)-f(x)}
  \ge \lim_{x\to 1^-} \abs{f_k(x) - f(x)} = 1.
\]
dunque non ci può essere convergenza uniforme di $f_k$ verso $f$.
\end{example}

E' facile convincersi che la successione $f_k$ dell'esempio precedente, oltre a non convergere uniformemente non ammette nessuna estratta convergente uniformemente. Perciò tale successione non può essere contenuta in nessun compatto di $C^0([0,1])$. In particolare il disco unitario
\[
  D = \ENCLOSE{f\in C^0([0,1])\colon \Abs{f}_\infty \le 1}
\]
risulta essere un insieme chiuso e limitato che però non è compatto.

\begin{theorem}[continuità del limite uniforme]
\mymark{***}%
Sia $X$ uno spazio metrico e siano $f_k\colon X\to \RR$
funzioni continue che
convergono uniformemente ad una funzione $f\colon X \to \RR$. 
Allora anche $f$ è continua.
\end{theorem}
%
\begin{proof}
\mymark{***}
Fissato $x_0\in X$ basta dimostrare che per ogni $\eps>0$
esiste $\delta>0$ tale che se $d(x,x_0)< \delta$ allora $\abs{f(x)-f(x_0)} < 3 \eps$.
Per definizione di convergenza uniforme dato $\eps>0$
esiste un $N\in \NN$ (in realtà ne esistono infiniti) per cui
$d_\infty(f_N,f)< \eps$. Per la continuità di $f_N$ in corrispondenza dello stesso $\eps$ esiste $\delta>0$
tale che se $d(x,x_0) < \delta$ allora $\abs{f_N(x)-f_N(x_0)} < \eps$. Ma allora se $d(x,x_0)<\delta$ si ha
\begin{align*}
\abs{f(x)-f(x_0)}
&\le \abs{f(x) - f_N(x)}
 + \abs{f_N(x)-f_N(x_0)}
 + \abs{f_N(x_0) - f(x_0)} \\
 &\le \Abs{f-f_N}_\infty + \eps + \Abs{f-f_N}_\infty
  \le 3\eps.
\end{align*}
\end{proof}

\begin{theorem}[completezza di $C^0({[a,b]})$]%
\label{th:C0_completo}
\mymark{***}%
\mymargin{$C^0([a,b])$ è completo}%
\index{completezza!di $C^0([a,b])$}%
Lo spazio $C^0([a,b])$ delle funzioni continue definite su un intervallo chiuso e limitato, 
dotato della norma uniforme $\Abs{\cdot}_\infty$ risulta essere uno spazio di Banach (ovvero uno spazio vettoriale normato e completo).
\end{theorem}
%
\begin{proof}
Per il teorema di Weierstrass ogni funzione continua definita sul compatto $[a,b]$ è limitata. 
Dunque $C^0([a,b])$ è un sottospazio vettoriale di $\B([a,b])$. 
Inoltre il teorema precedente (continuità del limite) ci dice che $C^0([a,b])$ è un sottospazio chiuso di $\B([a,b])$.
Ma $\B([a,b])$ è completo e quindi anche $C^0([a,b])$ essendo chiuso in $\B([a,b])$ è completo.
\end{proof}

La norma uniforme è la norma naturale su $C^0([a,b])$ in quanto lo rende uno spazio completo. Per questo motivo la norma uniforme sulle funzioni continue
viene anche chiamata \emph{norma $C^0$} e si
può denotare nel modo seguente:
\index{$\Abs{\cdot}_{C^0}$}
\index{norma!$C^0$}
\[
  \Abs{f}_{C^0} = \Abs{f}_{C^0([a,b])} = \Abs{f}_\infty
  \qquad\text{per $f\in C^0([a,b])$.}
\]

