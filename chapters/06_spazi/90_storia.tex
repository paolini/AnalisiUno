\section{note storiche}

\label{note:Basilea}%
Il problema di Basilea è un problema posto da Mengoli nel 1650 
e risolto da Eulero nel 1734 dopo che i maggiori matematici del tempo 
(tra cui i famosi membri della famiglia Bernoulli che vivevano appunto a Basilea) 
avevano tentato invano di risolverlo. 
Si tratta di determinare la somma della serie \eqref{eq:basilea}.
La soluzione di Eulero (completamente diversa da quella che stiamo proponendo qui)
è stata poi ripresa da Riemann che ha definito la celebre funzione \emph{zeta}
\index{Riemann!funzione $\zeta$}%
\index{$\zeta$!di Riemann}%
\index{Basilea!problema di}%
\index{problema!di Basilea}%
\index{Eulero!problema di Basilea}%
\index{$\pi$!problema di Basilea}%
\[
\zeta(s) = \sum_{k=1}^{+\infty} \frac{1}{k^s}  
\]
di grandissima rilevanza nella teoria dei numeri.
