\chapter{richiami di algebra lineare}

\begin{definition}[autovalori e autovettori]
Sia $V$ uno spazio vettoriale sul campo $\KK$ (con $\KK = \RR$ o $\KK=\CC$)
e $A\colon V\to V$ un operatore lineare.
Diremo che $\lambda\in \KK$ è un \emph{autovalore}%
\mymargin{autovalore}\index{autovalore} di $A$
e esiste $v\in V$, $v\neq 0$ tale che
\[
  Av = \lambda v.
\]
In tal caso $v$ si dice essere un \emph{autovettore}%
\mymargin{autovettore}\index{autovettore} di $A$ relativo
all'autovalore $\lambda$.

Denotiamo con $A-\lambda$ l'operatore lineare $A-\lambda I$
dove $I\colon V\to V$ è l'identità.
Lo spazio vettoriale
\[
  \ker (A-\lambda I)
\]
si chiama \emph{autospazio}%
\mymargin{autospazio}\index{autospazio} relativo all'autovalore $\lambda$.
L'autospazio è composto dal vettore $0$ e da tutti gli autovalori
relativi all'autovalore $\lambda$.

Diremo che $v$ è un autovettore generalizzato di grado $m$ ($m\ge 1$ intero)
relativo all'autovalore $\lambda$
se
\[
 (D-\lambda)^m v = 0
 \qquad \text{ma} \qquad
  (D-\lambda)^{m-1} v \neq 0.
\]

Gli autovettori generalizzati di grado $1$ sono esattamente gli
autovettori.
\end{definition}

\begin{proposition}[proprietà degli autovettori]
Sia $A\colon V \to V$ un operatore lineare.
Denotiamo con
\[
  W_\lambda^m = \ker (A-\lambda)^m \setminus \ker (A-\lambda)^{m-1}
\]
l'insieme (non è uno spazio vettoriale!) di tutti gli autovettori
generalizzati di $A$ di grado $m$ relativi all'autovalore $\lambda$.
Allora se $\lambda \neq \mu$ si ha
\begin{enumerate}
\item autovettori relativi ad autovalori distinti sono distinti:
\[
W_\lambda^1 \cap W_\mu^1 = \emptyset;
\]
\item gli operatori $A-\lambda$ e $A-\mu$ commutano:
\[
  (A-\lambda)(A-\mu)v = (A-\mu)(A-\lambda);
\]
\item
l'operatore $A-\mu$ lascia invariati gli insiemi $W_\lambda^m$:
\[
  v\in W_\lambda^n
  \implies
  (D-\mu)v \in W_\lambda^n.
\]
\end{enumerate}
\end{proposition}
%
\begin{proof}
\begin{enumerate}
\item
Se esistesse $v\in W_\lambda^1 \cap W_\lambda^1$ si avrebbe
\[
(\lambda - \mu) v = \lambda v - \mu v = Av - Av = 0.
\]
Ma questo è impossibile se $v\neq 0$ e $\mu\neq \lambda$.

\item
Si ha
\[
  (A-\lambda)(A-\mu)
  = A(A-\mu I) - \lambda (A-\mu I)
  = A^2 - \mu A -\lambda A + \lambda \mu I
  = A^2 - (\mu+\lambda) A + \lambda\mu I.
\]
Visto che il lato destro dell'uguaglianza è invariante se
scambiamo $\lambda$ e $\mu$ anche il lato sinistro deve esserlo.

\item
Se $(A-\lambda)^m v = 0$ si ha
\[
(A-\lambda)^m (A-\mu) v = (A-\mu)(A-\lambda^m) v = 0.
\]
Vogliamo mostrare che se $(A-\lambda)^{m-1} v \neq 0$
allora anche $(A-\lambda)^{m-1} v \neq 0$.

\end{enumerate}
\end{proof}
