\section{completezza}

\begin{definition}[successioni di Cauchy]
\mymark{***}
\index{successione!di Cauchy}
\index{Cauchy!successione di}
Sia $(X,d)$ uno spazio metrico e $x_k$ una successione di punti di $X$.
Diremo che $x_k$ è una
\emph{successione di Cauchy}
\mymargin{successione di Cauchy}
se
\[
 \forall \eps>0\colon \exists n\in \NN\colon \forall j>n \colon \forall k > n \colon d(x_j,x_k) < \eps.
\]
\end{definition}

La proprietà che definisce le successioni di Cauchy
si potrebbe anche scrivere così:
\[
  \lim_{k \to +\infty} \sup_{j\ge k} d(x_j, x_k) = 0.
\]

\begin{theorem}[le successioni convergenti sono di Cauchy]
\mymark{**}%
\label{th:se_converge_cauchy}
Sia $x_k\to x$ una successione convergente in uno spazio metrico $(X,d)$. Allora $x_k$ è di Cauchy.
\end{theorem}
%
\begin{proof}
\mymark{**}
Per definizione se $x_k \to x$ si ha
\[
  \forall \eps>0\colon \exists n\in \NN \colon
  \forall k>n \colon d(x_k,x)< \eps.
\]
Applicando la disuguaglianza triangolare, per ogni $j,k>n$
si ottiene il risultato desiderato:
\[
  d(x_j, x_k) \le d(x_k,x) + d(x,x_j) \le 2\eps.
\]
\end{proof}

\begin{definition}[completezza]
\mymark{***}
\mymargin{completezza}
\index{completezza}
Uno spazio metrico $(X,d)$ si dice essere \emph{completo}%
\mymargin{completo}\index{completo}
se ogni successione di Cauchy è convergente.
\end{definition}

Il prototipo di spazio metrico completo è $\RR$, come vedremo nel teorema~\ref{th:R-completo}.
Un esempio di spazio metrico non completo è $\QQ$.
Infatti se prendiamo una
successione $q_n \in\QQ$ con $q_n\to \sqrt 2$ la successione $q_n$ è convergente
in $\RR$ e quindi è di Cauchy in $\RR$. Ma visto che $q_n\in \QQ$ risulta
che $q_n$ è di Cauchy anche in $\QQ$ (la condizione di Cauchy è la stessa).
Ma in $\QQ$ tale successione non converge in quanto $\sqrt 2\not \in \QQ$.

\begin{definition}[spazio di Banach]
Uno spazio vettoriale normato si dice essere uno
\emph{spazio di Banach}%
\mymargin{spazio di Banach}%
\index{spazio!di Banach}%
\index{Banach!spazio di}
se, come spazio metrico, risulta essere completo.
Se la norma è euclidea, cioè deriva da un prodotto scalare, lo spazio si dirà
\emph{spazio di Hilbert}.
\mymargin{spazio di Hilbert}%
\index{spazio!di Hilbert}
\end{definition}

\begin{lemma}
\label{lm:cauchy_limitata}
Ogni successione di Cauchy è limitata
(più precisamente: se $x_n$ è una successione di Cauchy in uno spazio metrico $X$ allora l'insieme $\ENCLOSE{x_n\colon n\in \NN}$
è un insieme limitato).
\end{lemma}
%
\begin{proof}
Sia $x_k \in \RR$ una successione di Cauchy.
Fissato $\eps>0$ sappiamo che esiste $N\in \NN$
per cui per ogni $k,j>N$ si ha
$d(x_k,x_j) < \eps$. 
In particolare per ogni $k>N$ si ha
\[
  d(x_k, x_{N+1}) < \eps.
\]
Dunque scelto
\[
  R > \max\ENCLOSE{d(x_0,x_1), d(x_0,x_2), \dots, d(x_0,x_N), d(x_0, x_{N+1}) + \eps}
\]
si osserva che per ogni $k\in \NN$ si ha $d(x_0, x_k)< R$ in quanto 
se $k \le N$ abbiamo scelto appositamente $R$ in modo che sia più grande di $d(x_0,x_k)$ e se $k > N$ allora
\[
  d(x_0,x_k) \le d(x_0,x_{N+1}) + d(x_{N+1},x_k)
    \le d(x_0,x_{N+1}) + \eps < R.
\]

Significa quindi che per ogni $k\in \NN$ si ha $x_k \in B_{R}(x_0)$ 
che è la definizione di limitatezza in uno
spazio metrico.
\end{proof}

\begin{lemma}
\label{lm:cauchy_estratta_convergente}
Se una successione di Cauchy ha una sottosuccessione convergente, allora l'intera successione è convergente.
\end{lemma}
%
\begin{proof}
Sia $x_k$ la successione di Cauchy e sia $x_{k_j}\to x$ una  sottosuccessione convergente.
Allora per ogni $\eps>0$
esiste $m$ tale che se $k,j>m$ allora $d(x_k ,x_j) < \eps$.
Visto che $x_{k_j} \to x$ possiamo trovare $j$ tale che $k_j > m$ e tale che $d(x_{k_j},x) < \eps$. Ma allora
\[
  d(x_k,x) \le d(x_k, x_{k_j}) + d(x_{k_j},x)
   \le 2 \eps.
\]
E questo è vero per ogni $k > m$ da cui risulta verificata la definizione di limite $x_k \to x$.
\end{proof}

\begin{theorem}[completezza di $\RR$]
\mymark{***}%
\mymargin{$\RR$ è completo}%
\index{completezza!di $\RR$}%
\label{th:R-completo}%
$\RR$ è completo.
\end{theorem}
%
\begin{proof}
\mymark{***}
Dobbiamo dimostrare che se $x_k$ è una successione di Cauchy in $\RR$ allora $x_k$ converge.
Per il lemma~\ref{lm:cauchy_limitata} sappiamo che $x_k$ è limitata.
Ma allora per il teorema di Bolzano-Weierstrass sappiamo che $x_k$ ha
una estratta convergente.
Grazie al lemma~\ref{lm:cauchy_estratta_convergente} possiamo quindi concludere
che la successione $x_k$ è essa stessa convergente.
\end{proof}

\begin{corollary}[completezza di $\RR^n$ e $\CC$]
Gli spazi $\RR^n$ e $\CC$ (con la usuale distanza euclidea)
sono completi.
\end{corollary}
%
\begin{proof}
Basta osservare che la convergenza (o la condizione di Cauchy) di una successione in $\RR^n$
si ha se e solo se ogni componente della successione è convergente (o di Cauchy) in $\RR$. 
Dunque essendo $\RR$ completo anche $\RR^n$ lo è. 
Come spazio metrico $\CC$ è isomorfo ad $\RR^2$ dunque anch'esso è completo.
\end{proof}

\begin{theorem}[completezza dei compatti]
Ogni spazio metrico compatto è completo.
\end{theorem}
%
\begin{proof}
Visto che lo spazio è compatto ogni successione di Cauchy
ammette una sottosuccessione convergente.
Ma allora, grazie al lemma~\ref{lm:cauchy_estratta_convergente},
l'intera successione converge e dunque lo spazio è completo.
\end{proof}

\begin{theorem}[chiusi in spazi compatti e in spazi completi]%
\label{th:chiuso_completo}%
\label{th:compatto_completo}%
Sia $(X,d)$ uno spazio metrico e sia $A\subset X$ un sottoinsieme chiuso in $X$. Se $X$ è compatto allora anche $A$ è compatto, se $X$ è completo allora anche $A$ è completo.
\end{theorem}
\begin{proof}
Se $X$ è compatto da ogni successione in $A$ si può estrarre una sottosuccessione convergente ad un punto di $X$. 
Ma siccome $A$ è chiuso il punto sta in $A$ e dunque la sottosuccessione è convergente in $A$.

Una successione di Cauchy in $A$ è di Cauchy anche in $X$. 
Se $X$ è completo tale successione converge ad un punto di $X$. 
Se $A$ è chiuso tale punto è in $A$ e dunque la successione converge in $A$.
\end{proof}

\begin{theorem}[estendibilità delle funzioni uniformemente continue]
\mymark{**}%
\label{th:estensione_uniformemente_continua}%
Sia $f\colon A \subset X \to Y$ una funzione definita su un sottoinsieme $A$
di uno spazio metrico $X$ a valori in uno spazio metrico completo $Y$
(ad esempio $X=\RR$ e $Y=\RR$). Se $f$ è uniformemente continua allora
esiste una unica funzione continua $\tilde f \colon \bar A \to Y$
tale che per ogni $x\in A$ si ha $\tilde f(x) = f(x)$.
Inoltre $\tilde f$ è anch'essa uniformemente continua.
\end{theorem}
%
\begin{proof}
La prima osservazione è
che se una funzione $f$ è uniformemente continua allora $f$
manda successioni di Cauchy in successioni di Cauchy.
Infatti se $f$ è uniformemente continua si ha
\[
 \forall \eps>0\colon \exists \delta>0 \colon d(x,y)<\delta \implies d(f(x),f(y))<\eps
\]
e se $a_n$ è di Cauchy si ha
\[
 \exists N\in\NN \colon k,j>N \implies d(a_k,a_j)< \delta
\]
dunque mettendo insieme le due proprietà
si ottiene la condizione
di Cauchy per $f(a_k)$:
\[
\forall \eps>0\colon \exists N\in \NN\colon k,j>N \implies d(f(a_k),f(a_j))< \eps.
\]

Preso un punto $x \in \bar A$ esiste certamente $a_n \in A$ con $a_n\to x$.
Vogliamo dimostrare che $f(a_n)$ ha limite in $Y$ e che tale limite non
dipende dalla successione $a_n$ scelta. Se $a_n \to a$ allora $a_n$ è di Cauchy in $X$
(teorema~\ref{th:se_converge_cauchy}).
Per quanto detto prima possiamo affermare che anche $f(a_n)$ è di Cauchy
e quindi converge in $Y$, in quanto $Y$ è completo per ipotesi.
Inoltre il $\lim f(a_n)$ non dipende dalla successione scelta perché
se $b_n\to x$ fosse un'altra successione potremmo costruire una successione
$c_n$ che alterna i punti di $a_n$ e $b_n$ (ponendo, ad esempio, $c_{2n} = a_n$ e
$c_{2n+1} = b_n$). Anche $c_n \to x$ ed è di Cauchy, dunque $f(c_n)$ converge.
Ma $f(a_n)$ e $f(b_n)$ sono sotto-successioni di $f(c_n)$ e dunque convergono
allo stesso limite.
Abbiamo quindi dimostrato che per ogni $x\in \bar A$ esiste $\ell \in Y$
tale che se $a_n\to x$, $a_n\in A$ allora $f(a_n) \to \ell$. Possiamo
quindi definire $\tilde f(x)=\ell$. Se esiste una estensione continua di
$f$ non può che essere definita in questo modo, in quanto le funzioni continue
mandano successioni convergenti in successioni convergenti.

D'altra parte possiamo dimostrare che $\tilde f$ è una funzione uniformemente continua.
Infatti per ogni $\eps>0$ possiamo scegliere $\delta>0$ dato dall'uniforme
continuità di $f$: dati $x,y\in \bar A$ con $d(x,y)<\delta/3$ esisteranno
$x_n,y_n \in A$ con $x_n\to x$, $y_n\to y$. Si potrà quindi trovare $n$
sufficientemente grande in modo che
$d(x_n,x)<\delta/3$, $d(y_n,y)<\delta/3$,
$d(f(x_n),\tilde f(x)) < \eps$, $d(f(y_n),\tilde f(y)) < \eps$.
Si avrà allora
$d(x_n,y_n)< d(x_n,x)+d(x,y)+d(y,y_n) < \delta$, dunque $d(f(x_n),f(y_n))<\eps$
e in conclusione
\[
d(f(x),f(y))
\le d(f(x),f(x_n)) + d(f(x_n),f(y_n)) + d(f(y_n),f(y))
\le 3\eps.
\]
Abbiamo quindi dimostrato l'uniforme continuità di $\tilde f$ con $3\eps$ al posto
di $\eps$ e $\delta/3$
al posto di $\delta$.
\end{proof}

\begin{definition}[lipschitz]
\mymark{***}
Sia $f\colon X \to Y$ una funzione definita tra due spazi metrici.
Dato $L\ge 0$
diremo che $f$ è $L$-lipschitziana se
\index{funzione!lipschitziana}
per ogni $x,y \in X$ si ha
\[
  d(f(x),f(y)) \le L \cdot d(x,y).
\]
Diremo che $f$ è lipschitziana se esiste $L\ge 0$ tale che $f$ sia $L$-lipschitziana.
\end{definition}

\begin{theorem}
\label{th:lipschitz_uniformemente_continua}%
\mymark{*}%
Se $f\colon X \to Y$ è lipschitziana allora
$f$ è sequenzialmente continua, cioè
\[
  x_k \to x \implies f(x_k)\to f(x).
\]
\end{theorem}
%
\begin{proof}
Se $x_k\to x$ significa che $d(x_k,x) \to 0$, quindi
\[
  d(f(x_k), f(x)) \le L \cdot d(x_k,x) \to 0.
\]
\end{proof}

Osserviamo che la distanza $d(x,y)$ di uno spazio metrico $X$ risulta sempre essere una funzione $1$-lip\-schit\-zia\-na rispetto ad ognuna delle due variabili $x$ e $y$. Infatti per la disuguaglianza triangolare inversa si ha
\[
  \abs{d(x_1,y) - d(x_2,y)} \le d(x_1, x_2).
\]
Di conseguenza la norma di uno spazio normato è anch'essa $1$-lip\-schit\-zia\-na. 
In particolare la distanza e la norma risultano essere funzioni continue.

\begin{theorem}[delle contrazioni o punto fisso di Banach-Caccioppoli]
\mymark{***}%
\mymargin{teor. contrazioni}%
\index{teorema!di Banach-Caccioppoli}%
\index{teorema!delle contrazioni}%
\index{punto!fisso}%
\index{contrazione}%
\index{Caccioppoli!teorema delle contrazioni}%
\index{Banach!teorema delle contrazioni}%
\label{th:banach-caccioppoli}%
Sia $X$ uno spazio metrico completo non vuoto e sia $f\colon X \to X$ una funzione $L$-lipschitziana con $L<1$ (diremo che $f$ è una \emph{contrazione}).
Allora esiste ed è unico un punto
$x\in X$ tale che $f(x) = x$.
\end{theorem}
%
\begin{proof}
\mymark{***}
Si consideri un qualunque punto $p \in X$ e si definisca
la successione $x_k\in X$ tramite la definizione ricorsiva
\[
\begin{cases}
  x_0 = p \\
  x_{k+1} = f(x_k).
\end{cases}
\]
Visto che $f$ è $L$-lipschitziana si avrà
\begin{align*}
  d(x_2, x_1) &= d(f(x_1),f(x_0)) \le L \cdot d(x_1,x_0) \\
  d(x_3, x_2) &= d(f(x_2),f(x_1)) \le L \cdot d(x_2,x_1)
  \le L^2 \cdot d(x_1, x_0) \\
  d(x_4, x_3) &= d(f(x_3),f(x_2)) \le L \cdot d(x_3,x_2)
  \le L^3 \cdot d(x_1, x_0) \\
  &\vdots
\end{align*}
possiamo quindi dimostrare induttivamente che
per ogni $m\in \NN$ si ha
\[
  d(x_{m+1}, x_m) \le L^m \cdot d(x_1, x_0).
\]
Ma allora per ogni $k\in \NN$ e per ogni $j>k$
utilizzando la disuguaglianza triangolare e facendo la somma della progressione geometrica
si ha
\[
  d(x_k,x_j) \le \sum_{m=k}^{j-1} d(x_m, x_{m+1})
   \le \sum_{m=k}^{j-1} L^m \cdot d(x_1, x_0)
   = \frac{L^k-L^j}{1-L} d(x_1,x_0).
\]
Visto che $L<1$ se $k\to +\infty$ e $j>k$ questa quantità tende a zero e quindi
risulta che $x_k$ è una successione di Cauchy. Essendo per ipotesi $X$ completo sappiamo che la successione converge $x_k \to x$ ad un punto $x\in X$.
Per la continuità di $f$, passando al limite nell'equazione
$x_{k+1} = f(x_k)$ si ottiene $x = f(x)$.
Abbiamo quindi trovato un punto fisso.
Se $y\in X$ fosse un altro punto fisso si avrebbe:
\[
  d(x,y) = d(f(x),f(y)) \le L \cdot d(x,y)
\]
che è assurdo se $L<1$ e $x\neq y$.
\end{proof}

