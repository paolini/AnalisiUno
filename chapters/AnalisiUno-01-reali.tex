\chapter{insiemi numerici}

Supponiamo esista un insieme $\RR$ su cui sono definite le \mymargin{$\RR$}
operazioni $+$ e $\cdot$
e la relazione d'ordine $\le$ che
lo rendono un \emph{campo ordinato continuo}
come specificato nelle seguenti definizioni.
Gli elementi di tale insieme verranno chiamati
\myemph{numeri reali}.

\begin{definition}[campo]\label{def:field}
\label{def:campo}%
Si dice \myemph{campo}
un insieme $X$ su cui sono definite le operazioni di somma $+$ e
prodotto $\cdot$ che soddisfano le proprietà
per ogni $x,y,z\in X$:
\begin{enumerate}
\item associativa: $(x+y)+z = x + (y+z)$, $(x\cdot y)\cdot z = x \cdot (y \cdot z)$;
\item commutativa: $x+y=y+x$, $x\cdot y = y \cdot x$;
\item distributiva: $x\cdot (y+z) = x\cdot y + x \cdot z$;
\item esistenza degli elementi neutri: $0,1\in X$,
$0\neq 1$, $0+x = x$, $1\cdot x = x$;
\item esistenza dell'opposto: per ogni $x$ esiste $y$ tale che $x+y = 0$;
\item esistenza del reciproco: per ogni $x\neq 0$ esiste $y$ tale che $x \cdot y = 1$.
\end{enumerate}
\end{definition}

\begin{definition}[ordine totale]
\label{def:order}
Si dice \myemph{ordine!totale}
una relazione $\le$ con le seguenti proprietà
\mymargin{$\le$}
\begin{enumerate}
\item dicotomica: $x \le y$ o $y \le x$;
\item riflessiva: $x \le x$;
\item antisimmetrica: se $ x\le y$ e $y \le x$ allora $x=y$;
\item transitiva: se $x\le y $ e $ y \le z$ allora $x\le z$.
\end{enumerate}
Un insieme su cui è definita una relazione di
ordine totale si dice essere \emph{totalmente ordinato}.
\end{definition}

Definiamo $x<y$ se $x\le y$ e $x \neq y$ e definiamo le relazioni
inverse $x \ge y$ se $y\le x$ e $x>y$ se $y<x$.
\mymargin{$\ge$ $<$ $>$}

\begin{definition}[campo ordinato]
\label{def:campo_ordinato}%
Un campo su cui è definito un ordinamento totale
si dice essere un \myemph{campo!ordinato}
se le operazioni e l'ordinamento sono compatibili
nel senso che
valgono le seguenti proprietà:
\begin{enumerate}
\item positività: se $x\ge 0$ e $y \ge 0$ allora $x+y \ge 0$ e $x\cdot y\ge 0$;
\item monotonia: se $x \ge y$ allora $x+z \ge y+z$.
\end{enumerate}
\end{definition}

\begin{definition}[proprietà di Dedekind o continuità]
\label{def:dedekind}
\index{ordinamento!continuo}
\mymark{***}
Diremo che un insieme $X$ totalmente ordinato è
\emph{continuo} o \myemph{Dedekind-completo}
se
dati $A$ e $B$ sottoinsiemi non vuoti di $X$ tali che $A \le B$
(cioè: per ogni $a \in A$ e per ogni $b\in B$ vale $a\le b$) allora esiste
$x\in X$ tale che $A\le x \le B$ (cioè per ogni $a\in A$ e per ogni $b\in B$
vale $a\le x \le b$).
\end{definition}

L'insieme $\RR$ dei numeri reali soddisfa dunque, per ipotesi,
tutte le definizioni precedenti.
La proprietà di continuità, come vedremo,
è quella che sta alla base
dell'analisi matematica, ma prima di esaminare in dettaglio
tale proprietà, vediamo le proprietà algebriche di $\RR$.

\begin{theorem}
In un campo ordinato
valgono le seguenti
familiari proprietà:
\begin{enumerate}
  \item l'opposto e il reciproco sono unici
  (denotiamo con $-x$ l'unico opposto di $x$ e con $x^{-1}$ l'unico inverso di $x\neq 0$);
  \item $-(-x) = x$, $\enclose{x^{-1}}^{-1}$;
  \item $x \cdot 0 = 0$;
  \item $x\ge 0 \iff -x \le 0$;
  \item $(-x)\cdot y = -(x\cdot y)$;
  \item $-x = (-1)\cdot x$;
  \item $(-1)\cdot(-1) = 1$;
  \item $x\cdot x \ge 0$;
  \item $1 > 0$;
  \item se $x\cdot y = 0$ allora $x = 0$ o $y = 0$;
  \item se $x>0$ e $y>0$  allora $x\cdot y > 0$.
\end{enumerate}
\end{theorem}
%
\begin{proof}
\begin{enumerate}
\item
Supponiamo $y$ e $z$ siano due opposti di $x$ cioè $x+y=0$, $x+z=0$.
Allora da un lato $x+y+z = 0+z = z$, dall'altro $x+y+z = y+x+z= y+ 0 = y$.
Dunque $y=z$. Dimostrazione analoga si può fare per il reciproco.

\item
Se $x$ è opposto di $y$ allora $y$ è opposto di $x$ in quanto la somma
è commutativa. Dunque l'opposto di $-x$ è $x$ cioè $-(-x)=x$. Lo stesso
vale per il reciproco.

\item
Si ha
\[
x\cdot 0 = x \cdot 0 + x + (-x) %= x\cdot 0 + x\cdot 1 + (-x)
=x\cdot(0+1) + (-x) = x + (-x) = 0.
\]

\item
Se $x\ge 0$ sommando ad ambo i membri $-x$ si ottiene $x+(-x) \ge 0 + (-x)$
cioè $0 \ge -x$. Sommando $x$ ad ambo i membri si riottiene $x\ge 0$.


\item
Osserviamo che $(-x)\cdot y + x\cdot y = ((-x)+x)\cdot y = 0$ dunque $(-x)\cdot y$ è l'opposto di $x\cdot y$.

\item
Dunque $(-1)\cdot x = - (1 \cdot x) = - x$

\item
e per $x=-1$ si ottiene $(-1)\cdot(-1) = -(-1) = 1$.

\item
Si ha
\[
(-x)\cdot(-x) = (-1)\cdot x \cdot (-1)\cdot x = x\cdot x.
\]
Dunque se $x\ge 0$ per assioma di positività
abbiamo $x\cdot x\ge 0$ e se $x\le 0$ abbiamo $-x\ge 0$ e quindi
$x\cdot x = (-x)\cdot(-x) \ge 0$.

\item
In particolare per $x=1$ otteniamo $1\ge 0$.
Essendo inoltre per assioma $0\neq 1$ otteniamo $1> 0$.

\item
Se fosse $x\cdot y = 0$ e $x\neq 0$ allora $x$ avrebbe inverso $x^{-1}$
e avremmo:
\[
  y = x^{-1} \cdot x \cdot y = x^{-1}\cdot 0 = 0.
\]
Dunque o $x=0$ oppure $y=0$.

\item
Se $x>0$ e $y>0$ allora $x\ge 0$ e $y\ge 0$ da cui $x\cdot y\ge 0$.
Se fosse $x\cdot y=0$ uno dei due fattori si dovrebbe annullare
cosa che abbiamo escluso per ipotesi.
\end{enumerate}
\end{proof}

\begin{definition}[valore assoluto]
\mymark{***}
Definiamo il \myemph{valore assoluto} $\abs{x}$ di un numero $x\in \RR$ nel seguente modo
\[
\abs{x} =
\begin{cases}
  x & \text{se $x\ge 0$}, \\
  -x & \text{se $x<0 $}.
\end{cases}
\]
\end{definition}

\begin{proposition}[proprietà del valore assoluto]
\mymark{**}
Si ha
\begin{enumerate}
\item $\abs{x}\ge 0$ (positività)
\item $\big\lvert\abs{x}\big\rvert = \abs{x}$ (idempotenza)
\item $\abs{-x} = \abs{x}$ (simmetria)
\item $\abs{x\cdot y} = \abs{x}\cdot \abs{y}$ (omogenità)
\item $\abs{x+y} \le \abs{x} + \abs{y}$ (convessità)
\item $\abs{x-y} \le \abs{x-z} + \abs{z-y}$ (disuguaglianza triangolare)
\item $\big\lvert\abs{x}-\abs{y}\big\rvert \le \abs{x-y}$ (disuguaglianza triangolare inversa)
\end{enumerate}
Useremo inoltre spesso la seguente equivalenza (valida
anche con $<$ al posto di $\le$). Se $r\ge 0$ allora
\[
 \abs{x-y} \le r
 \iff
 y - r \le x \le y + r.
\]
\end{proposition}
%
\begin{proof}
\mymark{*}
Le prime quattro proprietà sono immediate conseguenze della definizione.

Dimostriamo innanzitutto l'ultima osservazione.
Se $x\ge y$ allora $x-y\ge 0$ e quindi $\abs{x-y} \le r$ è
equivalente a $x-y\le r$ cioè $x\le y+r$.
Se $x<y$ allora $x-y<0$ e quindi $\abs{x-y} \le r$ è
equivalente a $y-x \le r$ cioè $x\ge y-r$.
Viceversa se $y-r \le x \le y+r$ allora vale sia $x-y \le r$ che $y-x \le r$ e dunque $\abs{x-y}\le r$.

Osserviamo allora che per la precedente osservazione applicata
a $\abs{x-0} \le \abs{x}$ si ottiene
\[
  -\abs{x} \le x \le \abs{x}
\]
e sommando la stessa disuguaglianza con $y$ al posto di $x$ si
ottiene
\[
  -(\abs{x} + \abs{y}) \le x + y \le \abs{x} + \abs{y}
\]
che è equivalente alla proprietà di convessità:
\[
  \abs{x+y} \le \big\lvert\abs{x} + \abs{y}\big\rvert = \abs{x} + \abs{y}.
\]

Ponendo $y=z-x$ nella disuguaglianza precedente, si ottiene
\[
  \abs{z} \le \abs{x} + \abs{z-x}
\]
da cui
\[
  \abs{z} - \abs{x} \le \abs{z-x}.
\]
Scambiando $z$ con $x$ si ottiene la disuguaglianza opposta
e mettendole assieme si ottiene
la disuguaglianza triangolare inversa:
\[
\big\lvert \abs{z}-\abs{x} \big\rvert  \le \abs{z-x}.
\]

La disuguaglianza triangolare segue dalla convessità:
\[
 \abs{x-y} = \abs{x-z + z-y} \le \abs{x-z} + \abs{z-y}.
\]
\end{proof}

Osserviamo che dal punto di vista geometrico
$\abs{x-y}$ rappresenta la \emph{distanza} tra i punti
$x$ e $y$ sulla retta reale.

\begin{comment}
%% si potrebbe dimostrare la continuità delle quattro
%% operazioni per semplificare la definizione di
%% radice quadrata
Avendo definito una distanza tra i punti i $\RR$
vogliamo ora osservare che le quattro operazioni
$+$, $-$, $\cdot$, $/$ sono continue nel senso
che il risultato dell'operazione cambia di poco
se modifichiamo abbastanza poco gli operandi.
Formalmente possiamo dare la seguente.

\begin{definition}[operazione continua]
\index{operazione!continua}
Una operazione\footnote{
Per \emph{operazione} su un insieme $X$
si intende una funzione $*\colon A\subset X\times X \to X$
definita su un sottoinsieme $A$ delle coppie
di elementi di $X$.}
$*$ su $\RR$ si dice essere \emph{continua}
se per ogni $\eps>0$ esiste $\delta>0$ tale
che se $\abs{x-x'}<\delta$ e $\abs{y-y'<\delta}$
allora $\abs{x*y - x'*y} < \eps$.
\end{definition}

\begin{theorem}[continuità delle quattro operazioni]
In $\RR$ le operazioni $+$, $-$, $\cdot$, $/$ sono continue.
\end{theorem}
%
\begin{proof}
Per la somma la questione è semplice.
Se $\abs{x-x'}<\delta$ e $\abs{y-y'}<\delta$
si ha infatti
\[
 \abs{(x+y) - (x'-y')}
 = \abs{x-x' + y-y'}
 \le \abs{x-x'} + \abs{y-y'} \le 2\delta.
\]
Basterà quindi scegliere $\delta=\eps/2$ per
ottenere la continuità.

Per il prodotto si ha:
\begin{align*}
\abs{x\cdot y - x' \cdot y'}
& = \abs{x\cdot y - x \cdot y' + x \cdot y' - x'\cdot y'}
= \abs{x(y-y') + (x-x')\cdot y'}
\le \abs{x}\cdot\abs{y-y'} + \abs{x-x'}\cdot\abs{y'} \\
& \le \abs{x}\delta + \delta\abs{y'}
....
\end{align*}
\end{proof}
\end{comment}

Come applicazione dell'assioma di continuità possiamo mostrare l'esistenza
della \myemph{radice quadrata}.
Più avanti, con qualche strumento in più, rivedremo più in generale
la costruzione della radice $n$-esima.

\begin{theorem}[radice quadrata]
\label{th:radice_quadrata}
\mymark{***}
Dato $y\ge 0$ esiste un unico $x\ge 0$ tale che $x^2=y$.
Tale $x$ verrà denotato con $\sqrt y$, \emph{radice quadrata} di $y$.
\mymargin{$\sqrt{\cdot}$}
\end{theorem}
\begin{proof}
\mymark{*}
Se $y=0$ allora è facile verificare che $x^2=y$ ha come unica soluzione $x=0$.
Supponiamo allora $y>0$ e
consideriamo i seguenti due insiemi
\[
  A = \{x\ge 0 \colon x^2 \le y\},\qquad
  B = \{x\ge 0 \colon x^2 \ge y\}
\]
e verifichiamo che soddisfino le ipotesi dell'assioma di continuità.
Innanzitutto $0\in A$ e quindi $A$ non è vuoto.
Neanche $B$ è vuoto in quanto $y+1\in B$,
infatti essendo $y+1\ge 1$ si ha
$(y+1)^2 \ge y+1$. Verifichiamo inoltre che $A \le B$.
Preso $a\in A$ e $b\in B$ si ha $a^2 \le y \le b^2$.
Se fosse $a>b$ dovremmo avere $a^2>b^2$, dunque $a \le b$.

Dunque possiamo applicare l'assioma di continuità
che ci garantisce l'esistenza di $z\in \RR$ tale che $A \le z \le B$.
Vogliamo ora verificare che $z^2 = y$.

Ci servirà innanzitutto sapere che $z>0$. Se $y\ge 1$ si avrebbe $1\in A$
e dunque $z\ge 1$ essendo $z\ge A$. Se $y<1$ allora $y^2 < y$ e dunque $y^2 \in A$
da cui si ottiene $z\ge y^2 > 0$.

Se fosse $z^2 < y$ vorremmo dimostrare che esiste $\eps>0$ tale che
$(z+\eps)^2 \le y$.
Questo succede se $(z+\eps)^2 = z^2 + 2 \eps z + \eps^2 \le y$
e si può ottenere, ad esempio,
imponendo che sia $2\eps z \le (y-z^2)/2$ e $\eps^2 \le (y-z^2)/2$.
Cioè (ricordiamo che $z>0$) se $\eps \le (y-z^2)/(4z)$ e $\eps \le 1$
(in modo che $\eps^2 \le \eps$)
e $\eps \le (y-z^2)/2$. Dunque possiamo
trovare $\eps>0$
tale che
\[
\eps \le 1, \qquad
\eps \le \frac{y-z^2}{4z}, \qquad
\eps \le \frac{y-z^2}2
\]
(basta prendere il più piccolo dei tre numeri)
si osserva allora
che vale $(z+\eps)^2\le y$.
Dunque $z+\eps \in A$ e dunque non può essere $z\ge A$.

Se fosse $z^2 > y$ vorremmo dimostrare che esiste $\eps>0$ tale che
$(z-\eps)^2 \ge y$.
Questo succede se $(z-\eps)^2 = z^2 - 2\eps z + \eps^2 \ge y$.
E' quindi sufficiente che sia $z^2 - 2 \eps z \ge y$ ovvero basta scegliere
\[
  \eps = \frac{z^2-y}{2z}.
\]
Ma allora se $(z-\eps)^2\ge y$ si ha $z-\eps \in B$ e dunque non può
essere $z \le B$.

Rimane dunque l'unica possibilità che sia $z^2 = y$, come volevamo dimostrare.

Se ci fosse un altro $w\ge 0$ tale che $w^2 = y$ si avrebbe $w^2 - z^2=0$ ovvero
$(w-z)(w+z)=0$ da cui (ricordando che $z>0$ e quindi $w+z\neq 0$)
si ottiene $w-z=0$. Dunque $w=z$.
\end{proof}


\section{i numeri naturali, interi, razionali}

\begin{definition}[numeri naturali]
\label{def:naturali}
\mymargin{numeri naturali}
Un sottoinsieme $A\subset \RR$ si dice essere \emph{induttivo}
se valgono le due proprietà:
\begin{gather*}
  0 \in A, \\
  n\in A \implies n+1 \in A.
\end{gather*}
La famiglia di tutti i sottoinsiemi induttivi di $\RR$ non è vuota
in quanto $\RR$ stesso è induttivo. Definiamo $\NN$ come l'intersezione
di tutti i sottoinsiemi induttivi di $\RR$ (ovvero: il più piccolo sottoinsieme induttivo di $\RR$).
\mymargin{$\NN$}
\end{definition}

Risulta immediato dimostrare che l'insieme $\NN$ così definito
soddisfa gli assiomi di Peano (si veda \cite{appunti_logica})
in particolare vale il seguente.

\begin{theorem}[principio di induzione]
Se $P(n)$ è un predicato nella variabile $n\in \NN$ tale che:
\begin{enumerate}
\item $P(0)$ è vero;
\item per ogni $n\in \NN$ se $P(n)$ è vero allora anche $P(n+1)$ è vero.
\end{enumerate}
Allora $P(n)$ è vero per ogni $n\in \NN$.
\end{theorem}
%
\begin{proof}
Si consideri l'insieme $I = \{n \in \NN\colon P(n) \text{ è vera}\}$.
Le ipotesi del principio di induzione garantiscono che $I$ sia induttivo
(definizione~\ref{def:naturali})
e quindi che $\NN \subset I$. Dunque $P(n)$ è vera per ogni $n\in \NN$.
\end{proof}

\begin{theorem}[definizione per induzione]
Sia $X$ un insieme, sia $\alpha\in X$ e sia $g\colon X\to X$ una funzione.
Allora esiste una unica funzione $f\colon \NN \to X$ tale che
\begin{equation}\label{eq:4835628}
  \begin{cases}
    f(0) = \alpha, \\
    f(n+1) = g(f(n)).
  \end{cases}
\end{equation}

Più in generale se abbiamo $\alpha\in X$ e una funzione $g\colon X\times \NN \to X$
esisterà una unica funzione $f\colon \NN \to X$ tale che
%
\begin{equation}
  \begin{cases}
    f(0) = \alpha, \\
    f(n+1) = g(n, f(n)).
  \end{cases}
\end{equation}
\end{theorem}
%
\begin{proof}
Si rimanda agli appunti di logica \cite{appunti_logica}
per una dimostrazione formale.
Informalmente sarà possibile definire una
funzione $f_n$ sull'insieme dei primi $n+1$ numeri naturali
che soddisfi le proprietà richieste. Bisognerà poi
fare l'unione delle funzioni $f_n$ per ottenere una funzione
definita su tutto $\NN$.
\end{proof}


%\section{potenza intera, fattoriale, coefficiente binomiale}

A partire da $\NN$ si può costruire l'insieme $\ZZ$ dei
\myemph{numeri interi}
e l'insieme $\QQ$ dei \myemph{numeri razionali}:
\begin{align*}
  \ZZ
  \mymargin{$\ZZ$}
    &= \NN \cup (-\NN)
    = \{x\in \RR\colon \exists n\in\NN\colon (x=n) \lor (x=-n)\}, \\
  \QQ
  \mymargin{$\QQ$}
    &= \frac{\ZZ}{\NN\setminus\{0\}}
    = \left\{x \in \RR\colon \exists p\in \ZZ\colon \exists q \in \NN\setminus\{0\}\colon x = \frac{p}{q}\right\}.
\end{align*}

Si avrà dunque $\NN \subset \ZZ \subset \QQ \subset \RR$.

\begin{definition}[potenza intera]
\mymargin{potenza intera}
Dato $x \in \RR$ possiamo definire la potenza $x^n$ per ogni
$n\in \NN$ come l'unica funzione che soddisfa
\mymargin{$x^n$}
la seguente definizione per induzione
\[
\begin{cases}
  x^0 = 1\\
  x^{n+1} = x\cdot x^n.
\end{cases}
\]
Per $x\in \RR\setminus\{0\}$ possiamo anche definire $x^{-n}$ con $n\in \NN$
come
\[
x^{-n} = \frac{1}{x^n}.
\]
Risulta quindi che $x^n$ è definito per ogni $n\in \ZZ$ se $x\neq 0$.
\end{definition}

In base alla definizione si ha $x^0 = 1$, $x^1=x$, $x^2=x\cdot x$,
$x^3=x\cdot x \cdot x$ e così via. Dunque in generale
$x^n$ è il prodotto di $n$ fattori tutti uguali a $x$.
Si osservi in particolare che abbiamo definito $0^0=1$.
In alcuni testi si preferisce lasciare indefinito $0^0$
ma vedremo che questa definizione risulterà
molto utile e naturale.

Si osservi che la notazione
$x^{-1}$ appena introdotta coincide
con l'inverso moltiplicativo che, per questo motivo, avevamo già denotato
con $x^{-1}$.

\begin{theorem}[proprietà delle potenze intere]
Per ogni $x,y\in \RR$, e per ogni $n,m \in \NN$
valgono le seguenti proprietà:
\begin{enumerate}
  \item  $x^{n+m} = x^n \cdot x^m$;
  \item $(x^n)^m = x^{n\cdot m}$;
  \item $(x\cdot y)^n = x^n \cdot y^n$;
  \item $\displaystyle \enclose{\frac{x}{y}}^n = \frac{x^n}{y^n}$ se $y\neq 0$.
\end{enumerate}
Le stesse proprietà valgono per $n,m \in\ZZ$ se $x\neq 0$, $y\neq 0$.
\end{theorem}
%
\begin{proof}
Dimostriamo, come esempio, solamente la prima proprietà: $x^{n+m} = x^n \cdot x^m$.
Fissato $m\in \NN$ procediamo per induzione su $n$.
Se $n=0$ si ha $x^{0+m} = x^m = x^m \cdot 1 = x^m \cdot x^0$.
Supponendo la proprietà sia stata verificata per $n$, verifichiamo
che vale anche con $n+1$ al posto di $n$. Si ha infatti
\[
 x^{(n+1)+m} = x^{n+m+1} = x \cdot x^{n+m} = x \cdot x^n \cdot x^m
  = x^{n+1} \cdot x^m.
\]
\end{proof}

\begin{exercise}
Utilizzando il principio di induzione
si dimostri che $2^n > n$ per ogni $n\in \NN$.
\end{exercise}

\begin{definition}[fattoriale]
Possiamo definire il \myemph{fattoriale} di un numero $n$, indicato con $n!$, come il prodotto
dei numeri naturali da $1$ a $n$:
\[
  n!  = 1 \cdot 2 \cdot 3 \cdots n.
\]
Se $n=0$ il prodotto è vuoto e quindi $0!=1$, l'elemento neutro del prodotto.
La definizione formale si ottiene mediante una definizione per induzione:
\[
  \begin{cases}
    0! = 1 \\
    (n+1)! = (n+1) \cdot n!
  \end{cases}
\]

A volte sarà utile considerare anche i prodotti di solamente i numeri
pari o i numeri dispari fino ad un certo numero $n$. Questo
si chiama \myemph{doppio fattoriale} e si indica con $n!!$:
\begin{align*}
  (2n)!! &= 2 \cdot 4 \cdot 6 \cdots (2n) \\
  (2n+1)!! &= 1 \cdot 3 \cdot 5 \cdots (2n+1).
\end{align*}
\end{definition}

\begin{remark}
Si osservi che risulta
\[
  (2n)!! = (2\cdot 1) \cdot (2\cdot 2) \cdot (2\cdot 3) \cdots (2\cdot n)
        = 2^n \cdot n!
\]
mentre
\[
  (2n+1)!! = 1 \cdot \frac{2}{2} \cdot 3 \cdot \frac{4}{4}
\cdot 5 \cdot \frac{6}{6} \dots \frac{2n}{2n} \cdot (2n+1)
= \frac{(2n+1)!}{(2n)!!}
 = \frac{(2n+1)!}{2^n \cdot n!}.
\]
Queste formule permettono di esprimere il doppio fattoriale utilizzando
il fattoriale (singolo) e le potenze.
\end{remark}

\begin{exercise}
Si dia una definizione per induzione del doppio fattoriale
e si dimostrino, per induzione, le formule nell'osservazione precedente.
\end{exercise}

In generale è possibile introdurre una notazione per ripetere le operazioni
di somma e prodotto un numero arbitrario di volte:
\begin{align*}
  \sum_{k=1}^n f(k) &= f(1) + f(2) + \dots + f(n) \\
  \prod_{k=1}^n f(k) &= f(1) \cdot f(2) \cdots f(n).
\end{align*}
Formalmente questo può essere fatto tramite una definizione per induzione,
osservando che ogni somma (o prodotto) di $n$ numeri si ottiene mediante
la somma dei primi $n-1$ a cui si aggiunge (o si moltiplica) l'ultimo numero.

\begin{definition}[somme e prodotti]
Se $a,b\in \ZZ$, $a\le b$ e $f\colon \{a, a+1, a+2, \dots, b\}\to X$
è una funzione a valori in un insieme $X$ in cui è definita una
operazione $+$ di addizione (ad esempio $\RR$)
si definisce
\[
  \sum_a^b f
  \qquad \text{ovvero} \qquad
  \sum_{k=a}^b f(k).
\]
tramite la definizione per induzione:
\[
    \displaystyle \sum_a^a f = f(a), \qquad
    \displaystyle \sum_{a}^{n+1} f = \enclose{\sum_a^n f} + f(n+1).
\]
Analogamente si definisce
\[
  \prod_a^b f
  \qquad\text{ovvero}\qquad
  \prod_{k=a}^b f(k)
\]
quando su $X$ è definita una operazione di moltiplicazione,
ponendo:
\[
    \displaystyle \prod_{a}^a f = f(a), \qquad
    \displaystyle \prod_{a}^{n+1} f = \enclose{\prod_a^n f} \cdot f(n+1).
\]
\end{definition}

La definizione precedente ci permette, ad esempio, di definire le potenze e il fattoriale
come prodotti ripetuti, almeno quando $n\ge 1$:
\[
  x^n = \prod_{k=1}^n x, \qquad
  n! = \prod_{k=1}^n k.
\]

\begin{definition}[coefficiente binomiale]
\label{def:binomiale}
\mymark{***}
Definiamo per ogni $n\in \NN$ e per ogni $k\in \NN$, $k\le n$
il \myemph{coefficiente binomiale}
\[
{n \choose k}
=\frac{n!}{k!(n-k)!}.
\]
Per convenzione può essere utile porre
${n \choose k}=0$ se $k\in \ZZ$ con $k< 0$ o $k>n$.
Questa definizione verrà estesa nella definizione~\ref{def:binomiale_reale}.
\end{definition}
%
Si osservi che risulta
\begin{align*}
\frac{n!}{(n-k)!} &= \frac{n (n-1) \cdots (n-k+1)(n-k) \cdots 2\cdot 1}{(n-k) \cdots 2\cdot 1}\\
 &= n (n-1) \cdots (n-k+1) = \prod_{j=1}^k (n+1-k)
\end{align*}
e questo permette in pratica di calcolare il coefficiente binomiale ${n \choose k}$
svolgendo solamente $k$ prodotti a numeratore e a denominatore, indipendentemente
da quanto è grande $n$.

\begin{theorem}[triangolo di Tartaglia]
\mymark{*}
Per ogni $n\in \NN$ e $k \in \NN$ con $1 \le k \le n$ si ha
\[
  {n+1 \choose k} =
      {n \choose k-1} + {n \choose k}
\]
mentre
\[
  {n+1 \choose 0} = 1 = {n+1 \choose n+1}
\]
\end{theorem}
%
\begin{proof}
La seconda parte si ottiene direttamente dalla definizione.
Per la prima parte si ha:
\begin{align*}
{n \choose k-1} + {n \choose k}
&= \frac{n!}{(k-1)!(n-k+1)!} + \frac{n!}{k!(n-k)!} \\
&= \frac{k\cdot n! + (n-k+1)\cdot n!}{k!(n-k+1)!} \\
&= \frac{(n+1)\cdot n!}{k!(n+1-k)!}
= {n+1 \choose k}.
\end{align*}
\end{proof}

In base al teorema precedente i coefficienti binomiali si possono
elencare come nella tabella~\ref{tab:binomiali}
partendo dalla prima riga e scrivendo in ogni riga successiva
la somma dei due termini nella riga precedente sopra e
a sinistra del numero considerato, immaginando che negli spazi
vuoti ci siano degli zeri.

\begin{table}
\begin{tabular}{c|ccccccccc}
$\displaystyle{n \choose k}$& 0 & 1 & 2 & 3 & 4 & 5 & 6 & $k$ &\\ \hline
  0 & 1 &   &   &   &   &   &   & &\\
  1 & 1 & 1 &   &   &   &   &   & &\\
  2 & 1 & 2 & 1 &   &   &   &   & &\\
  3 & 1 & 3 & 3 & 1 &   &   &   & &\\
  4 & 1 & 4 & 6 & 4 & 1 &   &   & &\\
  5 & 1 & 5 & 10& 10& 5 & 1 &   & &\\
  6 & 1 & 6 & 15& 20& 15& 6 & 1 & &\\
$n$ &$\vdots$&&   &   &   &   &   & $\ddots$ &
\end{tabular}
\caption{Il triangolo di Tartaglia (o di Pascal).}
\label{tab:binomiali}
\end{table}

\begin{theorem}[sviluppo binomiale]
\mymark{***}
Se $a,b\in \RR$ e $n\in \NN$ si ha:
\[
(a+b)^n = \sum_{k=0}^n {n \choose k} a^k \cdot b^{n-k}.
\]
\end{theorem}
%
\begin{proof}
Lo dimostriamo per induzione su $n$.
Per $n=0$ l'uguaglianza è soddisfatta per verifica diretta (ambo i membri sono uguali ad $1$).

Supponendo valida l'uguaglianza per un certo $n\in \NN$ proviamo a verificarla
per $n+1$:
\begin{align*}
(a+b)^{n+1}
&= (a+b)\cdot (a+b)^n
 = (a+b) \sum_{k=0}^n {n \choose k} a^k \cdot b^{n-k}\\
&= \sum_{k=0}^n {n \choose k} a^{k+1}\cdot b^{n-k}
   + \sum_{k=0}^n {n \choose k} a^k \cdot b^{n-k+1} \\
&= \sum_{k=1}^{n+1} {n \choose k-1} a^k \cdot b^{n-k+1}
   + \sum_{k=0}^n {n \choose k}a^k \cdot b^{n-k+1} \\
&= \sum_{k=0}^{n+1} \enclose{{n \choose k-1} + {n \choose k}} a^k \cdot b^{n+1-k}.
\end{align*}
Nell'ultimo passaggio abbiamo sfruttato il fatto che per $k<0$ e per $k>n$ il coefficiente binomiale è nullo.
Sfruttando la relazione del triangolo di Tartaglia si ottiene infine,
come volevamo dimostrare
\[
  = \sum_{k=0}^{n+1}{n+1 \choose k} a^k \cdot b^{n+1-k}.
\]
\end{proof}

\begin{exercise}[interpretazione combinatoria del coefficiente binomiale]
Il numero di sottoinsiemi di $k$ elementi di un insieme con $n$ elementi
è ${n\choose k}$.
\end{exercise}

\begin{exercise}
Provare che
\[
 \sum_{k=0}^n {n \choose k} = 2^n.
\]
\end{exercise}

\section{estremo superiore}

Osserviamo che $\QQ$ è un campo ordinato che però non soddisfa l'assioma
di continuità.
Infatti se consideriamo i due insiemi:
\[
 A = \{x\in \QQ \colon x\ge 0, x^2 \le 2\},
 B = \{x\in \QQ \colon x\ge 0, x^2 \ge 2\}
\]
si può verificare che $A$ e $B$ sono non vuoti, $A \le B$, ma l'elemento
di separazione $\sqrt{2}\in \RR$ non è elemento di $\QQ$, in base al seguente
classico risultato.

\begin{theorem}[Pitagora]
\mymark{**}
L'equazione $x^2=2$ non ha soluzioni in $\QQ$.
\end{theorem}
%
\begin{proof}
\mymark{*}
Supponiamo $x\in \QQ$ sia una soluzione di $x^2=2$.
Allora si potrà scrivere $x=p/q$ con $p\in \ZZ$ e $q\in \NN$, $q\neq 0$.
Possiamo anche supporre che la frazione $p/q$ sia ridotta ai minimi
termini cioè che $p$ e $q$ non abbiano fattori in comune.
Moltiplicando l'equazione
$(p/q)^2=2$ per $q^2$ si ottiene $p^2 = 2 q^2$.
Risulta quindi che $p^2$ è pari.
Ma allora anche $p$ è pari (perché il quadrato di un dispari è dispari).
Ma se $p$ è pari allora $p^2$ è multiplo di quattro.
Ma allora anche $2q^2$ è multiplo di quattro e quindi $q^2$ è pari.
Dunque anche $q$ è pari. Ma avevamo supposto che $p$ e $q$ non avessero
fattori in comune quindi questo non può accadere.
\end{proof}

Abbiamo quindi dimostrato che $\sqrt{2} \in \RR \setminus \QQ$
(diremo che $\sqrt 2$ è \myemph{irrazionale}) e dunque $\RR \neq \QQ$.

\begin{definition}
\mymark{***}
Siano $x \in \RR$ e $A \subset \RR$.
Se $A\le x$ (ovvero $a\le x$ per ogni $a\in A$)
diremo che $x$ è un \myemph{maggiorante} di $A$.
Se $x \le A$ diremo invece che $x$ è un \myemph{minorante} di $A$.
Se $A$ ammette un maggiorante diremo che $A$ è \emph{superiormente limitato},
se $A$ ammette un minorante diremo che $A$ è \emph{inferiormente limitato},
\index{superiormente limitato}
\index{inferiormente limitato}
\index{limitato!superiormente}
\index{limitato!inferiormente}
se $A$ ammette sia maggiorante che minorante diremo che $A$ è \myemph{limitato}.

Se $A \le x$ e $x\in A$ diremo che $x$ è il massimo di $A$,
se $x\le A$ e $x\in A$ diremo che $x$ è il minimo di $A$

Se $x$ è minimo dei maggioranti di $A$ diremo che $x$ è
\myemph[estremo superiore e inferiore]{estremo superiore}
di $A$ se invece $x$ è massimo dei minoranti diremo che $x$ è
\myemph[]{estremo inferiore} di $A$.
\end{definition}

Massimo e minimo di un insieme $A$, se esistono, sono unici.
Infatti se $x$ e $y$ fossero due minimi di $A$ si avrebbe $x\le y$ in
quanto $x\le A$ e $y\in A$. Analogamente si avrebbe $y\le x$ e
quindi $x=y$. Ragionamento analogo se $x$ e $y$ fossero due massimi.
Anche l'estremo superiore e l'estremo inferiore se esistono sono
unici in quanto sono essi stessi un minimo ed un massimo
(rispettivamente dei maggioranti e dei minoranti).

Useremo quindi le notazioni:
\mymargin{$\max$ $\min$ $\sup$ $\inf$}
\[
 \max A, \qquad
 \min A, \qquad
 \sup A, \qquad
 \inf A
\]
per denotare univocamente (quando esistono) il massimo, il minimo,
l'estremo superiore e l'estremo inferiore di un insieme $A$.

Se l'insieme $A$ è finito e non vuoto, il massimo e il minimo esistono
sempre. Se, ad esempio, non esiste il massimo di $A$ significa che scelto
$x_k\in A$ esiste sempre $x_{k+1}\in A$ con $x_{k+1} > x$ e quindi l'insieme
$A$ deve contenere infiniti punti $x_0,x_1, \dots, x_k,\dots $
Ad esempio l'insieme $A = \{x\in \RR\colon 0<x<1\}$ non ha né massimo né
minimo perché per ogni $x\in A$ si ha $\frac x 2<x$, $\frac{1+x}{2}>x$
con $\frac x 2\in A$ e $\frac{1+x}{2}\in A$, dunque nessun $x\in A$ può essere
massimo o minimo.

\begin{theorem}[esistenza del $\sup$]
\label{th:sup}
\mymark{**}
Se $A$ è un insieme non vuoto,
superiormente limitato, allora esiste l'estremo superiore di $A$.
Tale numero $x=\sup A$ è caratterizzato dalle seguenti proprietà
\begin{enumerate}
\item $\forall a\in A\colon x \ge a$;
\item $\forall \eps>0\colon \exists a\in A \colon x < a + \eps$.
\end{enumerate}

Risultato analogo vale per l'estremo inferiore.
\end{theorem}
%
\begin{proof}
\mymark{*}
Consideriamo l'insieme dei maggioranti
$B = \{ b\in \RR \colon b \ge A\}$.
Per ipotesi $B$ è non vuoto e per come è definito risulta $A\le B$.
Dunque dall'assioma di continuità deduciamo l'esistenza di un numero $x\in \RR$
tale che $A\le x \le B$. La prima disuguaglianza $A\le x$ ci dice che $x$ è un
maggiorante e quindi $x\in B$, la seconda $x\le B$ ci dice che $x$ è il minimo
di $B$ e quindi concludiamo che $x$ è l'estremo superiore di $A$.

La prima delle due proprietà caratterizzanti il $\sup$ traduce la condizione
che $x$ sia un maggiorante di $A$. La seconda delle due proprietà esprime il
fatto che $x$ sia il minimo dei maggioranti, infatti se $x$ è il minimo
dei maggioranti significa che nessun numero minore di $x$ è un maggiorante, ovvero
che ogni $x-\eps$ con $\eps>0$ non è un maggiorante, ovvero
che esiste $a\in A$ tale che $a > x-\eps$.
\end{proof}

La seguente proprietà dei numeri reali esprime il fatto
che non esistono gli \emph{infinitesimi} ovvero numeri reali positivi
che siano più piccoli di ogni $1/n$ con $n\in \NN$.

\begin{theorem}[proprietà archimedea dei numeri reali]
\mymark{**}
\mymargin{proprietà!archimedea}
Dato $x\in \RR$ esiste $n\in \NN$ tale che $n > x$.
E se $x>0$ esiste $m\in \NN$ tale che $1/m < x$.
\end{theorem}
%
\begin{proof}
\mymark{*}
Se esistesse $x\in \RR$ tale che $x \ge \NN$
allora $\NN$ sarebbe superiormente limitato.
Dunque avrebbe un estremo superiore $m= \sup \NN$.
Siccome $m$ è il minimo dei maggioranti di $\NN$
e $m-1$ è più piccolo di $m$, allora $m-1$ non è un maggiorante
di $\NN$. Dunque deve esistere $n\in \NN$ tale che $n>m-1$.
Ma allora $n+1 > m$ ed essendo $n+1\in \NN$ troviamo che $m$
non poteva essere un maggiorante di $\NN$.

Dunque per ogni $y\in \RR$ esiste $n\in \NN$ tale che $n>y$.
Se $x\in \RR$ e $x>0$ allora posto $y=1/x$ possiamo trovare
$n\in \NN$ con $n>y = 1/x$ da cui $x > 1/n$.
\end{proof}

\begin{theorem}[parte intera]
\mymark{*}
Dato $x\in \RR$ esiste un unico $m\in \ZZ$ tale che $m-1 \le x < m$.
\end{theorem}
%
\begin{proof}
Supponiamo per un attimo che sia $x\ge 1$.
In tal caso consideriamo l'insieme $A=\{n\in \NN\colon n > x\}$.
Per la proprietà archimedea tale insieme non può essere vuoto e,
per il buon ordinamento di $\NN$ (si vedano gli appunti di logica),
deve avere un minimo $m$.
Dunque $m>x$ (in quanto $m\in A$) e $m\ge 1$ (in quanto $x\ge 1$).
Quindi necessariamente $m-1 \le x$ altrimenti avremmo che $m-1\in A$ e $m$
non poteva essere il minimo. Si ottiene dunque $m-1\le x < m$ come volevamo
dimostrare.

Nel caso fosse $x<1$ possiamo trovare un $N\in \NN$ (sempre per la proprietà archimedea) per cui $x+N \ge 1$. Applicando il ragionamento precedente a $x+N$ si trova comunque il risultato desiderato.
\end{proof}

\begin{definition}[parte intera]
\mymark{**}
\mymargin{parte!intera}
Dato $x\in \RR$ denotiamo con $\lfloor x\rfloor$ l'unico intero
che soddisfa
\mynote{$\lfloor\cdot\rfloor$} %% *** non viene bene nell'indice!
\[
  x - 1 < \lfloor x \rfloor \le x
\]
e denotiamo con $\lceil x \rceil = - \lfloor -x \rfloor$ l'unico intero che soddisfa (verificare!)
\mynote{$\lceil\cdot\rceil$} %% *** non viene bene nell'indice!
\[
  x \le \lceil x \rceil < x + 1.
\]
Si ha dunque
\[
  \lfloor x \rfloor \le x \le \lceil x \rceil
\]
con entrambe le uguaglianze che si realizzano quando $x\in \ZZ$.
I due interi $\lfloor x \rfloor$ e $\lceil x \rceil$
sono la migliore approssimazione intera di $x$ rispettivamente
per difetto e per eccesso.
L'intero più vicino ad $x$ (approssimazione per \myemph{arrotondamento})
è
\[
  \left\lfloor x + \frac 1 2 \right\rfloor
\quad \text{ossia} \quad
  \left\lceil x-\frac 1 2 \right\rceil
\]
(le due espressioni differiscono solamente quando $x$ si trova nel punto medio tra due interi consecutivi, nel qual caso la prima approssima per eccesso e la seconda per difetto).
\end{definition}
In alcuni testi si usa la notazione $[x]$ per denotare la parte intera $\lfloor x \rfloor$ e si definisce
anche la \emph{parte frazionaria}
\[
   \{x\} = x - [x]
\]
per evitare ambiguità con il normale utilizzo delle parentesi
noi preferiremo evitare queste notazioni.

\begin{theorem}[densità di $\QQ$ in $\RR$]
\mymark{*}
\mymargin{densità di $\QQ$}
Dati $x,y \in \RR$ con $x<y$ esiste $q\in \QQ$ tale che $x<q<y$.
\end{theorem}
%
\begin{proof}
Per la proprietà archimedea dei numeri reali essendo $y-x>0$
deve esistere $n\in \NN$ tale che $y-x > 1/n$ così si avrà
\[
    nx + 1 < ny.
\]
Prendiamo allora $m=\lfloor nx + 1\rfloor$ cosicché si abbia
\[
  nx < m \le nx + 1.
\]
Mettendo insieme le due disuguaglianze e dividendo per n si ottiene,
come volevamo dimostrare,
\[
 x < \frac{m}{n} < y.
\]
\end{proof}


\section{reali estesi}

%%%%%%%%%%%%%%%%%%%
%%%%%%%%%%%%%%%%%%%
%%%%%%%%%%%%%%%%%%%

\begin{definition}[reali estesi]
\mymargin{$\bar{\RR}$}
Denotiamo con $\bar \RR=\RR \cup \{+\infty, -\infty\}$ l'insieme dei numeri reali
\mymargin{$+\infty$, $-\infty$}
a cui vengono aggiunti due ulteriori \emph{quantità} che chiameremo
\emph{infinite} e che denotiamo con $+\infty$ e $-\infty$.
Diremo che $x\in \bar \RR$ è \emph{finito} se $x\in \RR$.
\end{definition}


Estendiamo la relazione d'ordine imponendo che valga
\[
  -\infty \le x \le +\infty, \qquad \forall x \in \bar\RR.
\]

Estendiamo anche la addizione e moltiplicazione
tra reali estesi imponendo che valga per ogni $x\in \bar \RR$
\begin{gather*}
  x + (+\infty) = +\infty, \qquad \text{se $x\neq -\infty$}\\
  x + (-\infty) = -\infty, \qquad \text{se $x\neq +\infty$}\\
  x \cdot (+\infty) = +\infty, \qquad
  x \cdot (-\infty) = -\infty, \qquad \text{se $x>0$} \\
  x \cdot (+\infty) = -\infty, \qquad
  x \cdot (-\infty) = +\infty, \qquad \text{se $x<0$}.
\end{gather*}

Si definiscono anche:
\[
 -(+\infty) = -\infty, \qquad
 -(-\infty) = +\infty, \qquad
 \frac{1}{+\infty} = \frac{1}{-\infty}=0
\]
facendo però attenzione che
questi formalmente non sono \emph{opposto}
e \emph{reciproco} in quanto
su $\bar \RR$ non sono più garantite
le regole: $x + (-x) = 0$ e $x \cdot (1/x) = 1$.
Infatti
le operazioni $(+\infty) + (-\infty)$ e $+\infty \cdot 0$ vengono
lasciate indefinite.

Definiamo anche il valore assoluto: $\abs{+\infty} = \abs{-\infty} = +\infty$.

Possiamo infine definire la sottrazione e la divisione tramite
addizione e moltiplicazione:
\[
  x - y = x + (-y), \qquad \frac{x}{y} = x \cdot \frac{1}{y}.
\]

Possiamo definire gli operatori $\sup$ e $\inf$
anche sugli insiemi illimitati ponendo:
\begin{align*}
  \sup A = +\infty \qquad \text{se $A$ non è superiormente limitato}\\
  \inf A = -\infty \qquad \text{se $A$ non è inferiormente limitato}.
\end{align*}
Osserviamo infatti che su $\bar \RR$ la quantità $+\infty$
è maggiorante di qualunque insieme e $-\infty$ è minorante, dunque
queste definizioni mantengono su $\bar \RR$ le proprietà caratterizzanti:
l'estremo superiore è il minimo dei maggioranti e
l'estremo inferiore è il massimo dei minoranti.
Definiamo infine
\begin{align*}
  \sup \emptyset = -\infty\\
  \inf \emptyset = +\infty.
\end{align*}
Queste ultime definizioni possono essere comprese da un punto di vista
strettamente logico: ogni numero reale è sia maggiorante che minorante
dell'insieme vuoto, dunque il minimo dei maggioranti non esiste in $\RR$
ma in $\bar \RR$ è $-\infty$
e il massimo dei minoranti è $+\infty$.

\section{intervalli}

\begin{definition}[intervallo]
\mymargin{intervallo}
Un insieme $I\subset \RR$ si dice essere un \emph{intervallo}
se soddisfa la \emph{proprietà dei valori intermedi}:
\[
  \text{se $x, y \in I$ e $x<z<y$ allora $z \in I$.}
\]
\end{definition}
\begin{theorem}[caratterizzazione intervalli]
Sia $I$ un intervallo e sia $a=\inf I$, $b=\sup I$. Allora
$z\in I$ se $a < z < b$.
\end{theorem}
%
\begin{proof}
Se $I=\emptyset$ si ha $a>b$ e quindi nessun $z$ verifica $a<z<b$.
Supponiamo $I\neq \emptyset$ e
sia $a < z < b$.
Visto che $a$ è il massimo dei minoranti di $I$ deve esistere $x \in I$ tale
che $a \le x < z$ altrimenti ogni $x$ tra $a$ e $z$ sarebbe un minorante di $I$
e $a$ non sarebbe il minimo. Analogamente dovrebbe esistere $y\in I$ con $z<y\le b$.
Ma allora, per la proprietà dei valori intermedi anche $z\in I$.
\end{proof}

Il teorema precedente ci dice che una volta identificati i due estermi
di un intervallo, tutti i punti intermedi devono stare nell'intervallo.
Gli estremi, invece, possono essere o non essere inclusi nell'intervallo.
Punti esterni agli estremi non possono invece esserci.
Possiamo quindi caratterizzare tutti gli intervalli di $\bar \RR$
introducendo le seguenti notazioni. Dati $a,b\in \bar \RR$ con $a\le b$
poniamo
\begin{align*}
[a,b] &= \{x\in \bar \RR\colon a \le x \le b\} \\
[a,b) &= \{x\in \bar \RR\colon a \le x < b\} \\
(a,b] &= \{x\in \bar \RR\colon a < x \le b\}\\
(a,b) &= \{x\in \bar \RR\colon a < x < b\}.
\end{align*}
Abbiamo quindi utilizzato le parentesi quadre per indicare che gli estremi
sono inclusi e le parentesi tonde per indicare che gli estremi sono esclusi.
Osserviamo che in alcuni testi si usano le parentesi quadre rovesciate al posto
delle parentesi tonde.

Noi considereremo per lo più intervalli di $\RR$ (non di $\bar \RR$): in tal
caso gli estremi infiniti non potranno mai essere inclusi nell'intervallo.

%%%%%%%%%%%%%%%%%%%

\section{i numeri complessi}

Dal punto di vista geometrico l'insieme $\CC$ dei \myemph{numeri!complessi}
\index{$\CC$}
può essere visto come un modello del piano euclideo.
Il piano euclideo è uno spazio affine reale di dimensione 2.
Possiamo mettere delle coordinate sul piano se fissiamo un punto $O$ (origine)
e due vettori ortonormali $e_1$, $e_2$. Chiamiamo $0$ il vettore
nullo $OO$ e chiamiamo $1$ il vettore $e_1$.
La retta passante per $O$ con direzione $e_1$ rappresenta l'insieme
dei numeri reali $\RR$.
Chiamiamo $i$ il vettore $e_2$.
La retta passante per $O$ con direzione $e_2$ verrà chiamata
\emph{retta dei numeri immaginari}.

Un generico punto del piano $z$ potrà essere scritto in maniera univoca
nella base scelta: $z = x e_1 + y e_2$ ovvero, per come abbiamo definito $e_1$ ed
$e_2$:
\[
z = x + i y.
\]
Tale $z$ viene chiamato
\emph{numero complesso} con parte reale $x$ e parte immaginaria $y$.
Questa rappresentazione del numero complesso $z$ viene
chiamata \myemph{rappresentazione cartesiana} in quanto definisce
il punto $z$ del piano complesso tramite le sue coordinate cartesiane
$x$ e $y$.
I numeri reali sono \emph{immersi} nei complessi, nel senso che se
$x\in \RR$ allora $z= x + i\cdot 0 = x$ è anche un numero complesso.
Il numero complesso $i = 0 + i\cdot 1$ viene chiamata \myemph{unità immaginaria}
e i numeri complessi della forma $iy$ sono chiamati \emph{immaginari}.
\index{numeri!immaginari}
\index{immaginario}
Un numero
complesso $z = x+iy$ è quindi una somma tra un numero reale ed un numero
immaginario. Il numero reale $x$ viene chiamato \emph{parte reale}
\index{parte!reale}
di $z$ e
si denota con $x=\Re z$.
\mymargin{$\Re z$}
Il numero reale $y$ viene chiamato
\emph{parte immaginaria}
\index{parte!immaginaria}
di $z$ e si denota con $y=\Im z$
\mymargin{$\Im z$}
(osserviamo che la parte immaginaria di un numero complesso è un numero
reale, non immaginario). Dunque $z= \Re z + i \Im z$.

L'insieme dei numeri complessi viene denotato con $\CC$.
Lo spazio $\CC$, per come
è stato costruito, è uno spazio vettoriale reale di dimensione $2$.
Abbiamo quindi già definite la \myemph{addizione}
\index{complessi!addizione}
tra elementi di $\CC$ e la moltiplicazione
tra elementi di $\CC$ ed elementi di $\RR$.
Se $a,b,c,d,t\in \RR$ si ha:
\begin{gather*}
 (a+ib) + (c+id) = (a+c) + i (b+d), \\
 t(a+ib) = ta + itb.
\end{gather*}

Vogliamo estendere la \myemph{moltiplicazione} a tutte le coppie di numeri complessi.
\index{complessi!moltiplicazione}
Imponendo (arbitrariamente) che valga $i\cdot i = -1$ e che rimanga
valida la proprietà distributiva, si ottiene
questa definizione:
\[
   (a+ib) \cdot (c+id) = (ac-bd) + i(ad+bc).
\]

Si può verificare che questa moltiplicazione estende quella "scalare" definita
in precedenza.
E' facile verificare che addizione e moltiplicazione soddisfano le proprietà commutativa associativa e distributiva, $0$ è elemento neutro per la addizione, $1$ è elemento neutro della moltiplicazione.
Si osservi che se $z=x+iy$ non è nullo, allora
\[
  (x+iy) \cdot \frac{x-iy}{x^2+y^2} = 1.
\]
Significa che ogni $z\neq 0$ ammette inverso moltiplicativo e quindi $\CC$ risulta essere un campo.

Osserviamo che su $\CC$ non si definisce una relazione d'ordine perché
in effetti non è possibile definire un ordine "compatibile" con le operazioni
appena definite.

Su $\CC$ definiamo delle ulteriori operazioni.
Il \myemph{coniugato}
\index{complessi!coniugio}
di un numero complesso $z=x+iy$ è il numero
$\bar z = x - iy$. Geometricamente l'operazione di coniugio è una simmetria
rispetto alla retta reale. I numeri reali sono in effetti punti fissi del
coniugio (il coniugato di un numero reale è il numero stesso).
E' un semplice esercizio verificare che il coniugio ``attraversa''
somma e prodotto:
\[
\overline{z+w} = \bar z + \bar w, \qquad
\overline{z\cdot w} = \bar z \cdot \bar w.
\]
Ovviamente risulta $\overline {\bar z} = z$.
E' anche utile osservare che si ha:
\begin{equation}\label{eq:re_im}
  \Re z = \frac{z+\bar z}{2}, \qquad
  \Im z = \frac{z-\bar z}{2i}
\end{equation}
e
\[
z \cdot \bar z = (x+iy)(x-iy) = x^2-i^2y^2 = x^2+y^2.
\]

Possiamo allora definire il
\myemph{modulo}
\index{complessi!modulo}
 di un numero complesso $z=x+iy$
come il numero reale
\[
\abs{z} = \sqrt{z\cdot\bar z} = \sqrt{x^2+y^2}.
\]
Geometricamente tale quantità rappresenta la distanza del punto $z$
dal punto $0$ e quindi la distanza tra due numeri complessi $z$ e
$w$ si potrà rappresentare con $\abs{z-w}$.

Osserviamo che se $z = x \in \RR \subset \CC$ il modulo di $z$ coincide
con il valore assoluto: $\abs{z} = \sqrt{x^2} = \abs{x}$ e per questo
motivo non distinguiamo, nelle notazioni, il modulo dal valore assoluto.
Più in generale risulta per ogni $z\in \CC$ (la verifica è immediata):
\[
  \abs{\Re z} \le \abs{z}, \qquad
  \abs{\Im z} \le \abs{z}.
\]

Possiamo a questo punto trovare una utile formula per calcolare
il reciproco di un numero complesso. Essendo infatti
$z\cdot \bar z = \abs{z}^2$ si osserva che
\[
  \frac{1}{z}
  = \frac{\bar z}{ \bar z \cdot z}
  = \frac{\bar z}{\abs{z}^2}.
\]

\begin{theorem}
Il modulo di un numero complesso soddisfa (come il valore assoluto)
le seguenti proprietà
\begin{enumerate}
\item $\big\lvert\abs{z}\big\rvert = \abs{z}$,
\item $\abs{-z} = \abs{z}$ = $\abs{\bar z}$,
\item $\abs{z\cdot w} = \abs{z}\cdot\abs{w}$.
\item $\abs{z+w} \le \abs{z}+\abs{w}$ (convessità),
\item $\abs{z-w} \le \abs{z-v} + \abs{v-w}$ (disuguaglianza triangolare),
\end{enumerate}
\end{theorem}
%
\begin{proof}
La prima proprietà è ovvia in quanto il valore assoluto di un numero reale
non negativo è il numero stesso.

La seconda proprietà viene immediatamente dalla definizione.

Per la terza proprietà sia $z=x+iy$, $w=a+ib$.
Allora:
\begin{align*}
\abs{z\cdot w}
&= \abs{(x+iy)\cdot(a+ib)}
=\abs{xa - y b+ i(xb + ay)} \\
&= \sqrt{(xa-yb)^2 + (xb+ay)^2}\\
&=\sqrt{x^2 a^2 + y^2b^2 - 2xayb + x^2b^2+a^2y^2+2xbay} \\
&=\sqrt{x^2 a^2 + y^2 b^2 + x^2 b^2 + a^2 y^2}\\
&=\sqrt{x^2(a^2+b^2) + y^2(a^2+b^2)}\\
&=\sqrt{(x^2+y^2)(a^2+b^2)}
=\abs{x+iy} \cdot \abs{a+ib}\\
&=\abs{z}\cdot\abs{w}.
\end{align*}

Per la quarta disuguaglianza osserviamo che si ha
\[
  \abs{z+w}^2 = (z+w)\cdot(\bar z + \bar w)
  = \abs{z}^2 + \abs{w}^2 + z\cdot \bar w + \bar z \cdot w
\]
e visto che
\[
  z\cdot \bar w + \bar z \cdot w
  = z \cdot \bar w + \overline{z \cdot w}
  = 2 \Re(z\bar w)
  \le 2 \abs {z\bar w}
  = 2 \abs{z}\cdot\abs{\bar w}
  = 2 \abs{z}\cdot\abs{w}
\]
otteniamo
\[
 \abs{z+w}^2 \le \abs{z}^2+\abs{w}^2 + 2 \abs{z}\cdot\abs{w}
 =\enclose{\abs z + \abs w}^2
\]
che è equivalente alla disuguaglianza di convessità.

La disuguaglianza triangolare è conseguenza immediata della convessità, infatti
\[
  \abs{z-w} = \abs{(z-v) + (v-w)}
  \le \abs{z-v} + \abs{v-w}.
\]
\end{proof}

Anche lo spazio dei numeri complessi può essere esteso aggiungendoci
un punto all'\myemph{infinito}.
A differenza dei reali, su cui era presente un ordinamento che era utile conservare,
nel caso dei numeri complessi è più usuale utilizzare un unico punto infinito
che si denota con \myemph{$\infty$}.
Definiamo lo spazio dei complessi estesi $\bar \CC$ come
\[
\bar \CC = \CC \cup \{\infty\}.
\]
Definiamo
\begin{align*}
  z + \infty &= \infty \qquad \forall z \in \bar\CC\\
  z - \infty &= \infty \qquad \forall z \in \bar\CC\\
   z\cdot \infty &= \infty \qquad \forall z \in \bar\CC\setminus\{0\} \\
   z / \infty &= 0 \qquad \forall z \in \CC \\
   z / 0 &= \infty \qquad \forall z \in \bar \CC \setminus\{0\}\\
   \abs{\infty} &= +\infty \in \bar \RR.
\end{align*}
Si noti che abbiamo definito la divisione per zero di numeri complessi
(e quindi anche reali) diversi da zero. Il risultato è $\infty$ e quindi
rimane confermato che la divisione per zero non è una operazione valida
se vogliamo un risultato finito.
Una quantità $z\in \bar \CC$ sarà detta \emph{finita} se $z\in \CC$.



%%%%%%%%%%%%%%%%%%%
%%%%%%%%%%%%%%%%%%%
%%%%%%%%%%%%%%%%%%%
