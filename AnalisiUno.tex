\documentclass[italian,a4paper,oneside,headinclude]{scrbook}
\usepackage{amsmath,amssymb,amsthm,thmtools}
\usepackage{babel,a4}
\usepackage[nochapters,pdfspacing]{classicthesis}
\usepackage[utf8]{inputenc}
\usepackage{comment} % for the comment environment
\usepackage{graphicx}
\usepackage{tikz}
\usepackage{pst-node}
\usetikzlibrary{cd} % commutative diagrams
\usepackage{eucal}
\AfterPreamble{\hypersetup{hidelinks=true,}}

%\usepackage{showkeys}

\usepackage{makeidx} % glossario

\usetikzlibrary{arrows}

\newcommand{\mymargin}[1]{\marginpar{#1}\index{#1}}
\newcommand{\myemph}[1]{\emph{#1}\marginpar{#1}\index{#1}}
\renewcommand{\subset}{\subseteq}

\newcommand{\eps}{\varepsilon}
\renewcommand{\phi}{\varphi}
\newcommand{\loc}{\mathit{loc}}
\newcommand{\weakto}{\rightharpoonup}

% calligraphic letters
\newcommand{\A}{\mathcal A}
\newcommand{\B}{\mathcal B}
\newcommand{\C}{\mathcal C}
\newcommand{\D}{\mathcal D}
\newcommand{\E}{\mathcal E}
\newcommand{\F}{\mathcal F}
\newcommand{\FL}{\mathcal F\!\mathcal L}
\renewcommand{\L}{\mathcal L}
\newcommand{\M}{\mathcal M}
\renewcommand{\P}{\mathcal P}
\renewcommand{\S}{\mathcal S}
\newcommand{\U}{\mathcal U}

% blackboard letters
\newcommand{\CC}{\mathbb C}
\newcommand{\HH}{\mathbb H}
\newcommand{\KK}{\mathbb K}
\newcommand{\NN}{\mathbb N}
\newcommand{\QQ}{\mathbb Q}
\newcommand{\RR}{\mathbb R}
\newcommand{\TT}{\mathbb T}
\newcommand{\ZZ}{\mathbb Z}

\newcommand{\abs}[1]{{\left|#1\right|}}
\newcommand{\Abs}[1]{{\left\Vert #1\right\Vert}}
\newcommand{\enclose}[1]{{\left( #1 \right)}}

\renewcommand{\vec}[1]{\boldsymbol #1}
\newcommand{\defeq}{:=}
\DeclareMathOperator{\divergence}{div}
\renewcommand{\div}{\divergence}
% \DeclareMathOperator{\ker}{ker}  %% already defined
\DeclareMathOperator{\Imaginarypart}{Im}
\renewcommand{\Im}{\Imaginarypart}
\DeclareMathOperator{\Realpart}{Re}
\renewcommand{\Re}{\Realpart}
%\DeclareMathOperator{\arg}{arg}

\newtheorem{theorem}{Teorema}
\newtheorem{lemma}[theorem]{Lemma}
\newtheorem{exercise}[theorem]{Esercizio}
\newtheorem{proposition}[theorem]{Proposizione}
\newtheorem{corollary}[theorem]{Corollario}
\newtheorem{example}[theorem]{Esempio}
\newtheorem{definition}[theorem]{Definizione}
\newtheorem{axiom}[theorem]{Assioma}

\title{Appunti di Analisi Matematica I%
\thanks{%
Puoi scaricare o contribuire a questi appunti su
\url{https://github.com/paolini/appunti/}}}
\author{E. Paolini}

\makeindex

\begin{document}
\maketitle

\tableofcontents

\chapter{I numeri reali}

Supponiamo esista un insieme $\RR$ su cui sono definite le operazioni $+$ e $\cdot$
e la relazione d'ordine $\le$ che soddisfano i seguenti assiomi.
\mymargin{numeri reali}
\mymargin{$\RR$}

\begin{axiom}[campo]\label{axiom_field}
\mymargin{campo}
Sull'insieme $\RR$ dei numeri reali sono definite le operazioni di somma $+$ e
prodotto $\cdot$ che soddisfano le proprietà:
\begin{enumerate}
\item associativa: $(x+y)+z = x + (y+z)$, $(x\cdot y)\cdot z = x \cdot (x \cdot z)$);
\item commutativa: $x+y=y+x$, $x\cdot y = y \cdot x$;
\item distributiva: $x\cdot (y+z) = x\cdot y + x \cdot z$;
\item esistenza degli elementi neutri: $0,1\in \RR$,
$0\neq 1$, $0+x = x$, $1\cdot x = x$;
\item esistenza di opposto: per ogni $x$ esiste $y$ tale che $x+y = 0$;
\item esistenza del reciproco: per ogni $x\neq 0$ esiste $y$ tale che $x \cdot y = 1$.
\end{enumerate}
\end{axiom}

Denotiamo con $-x$ l'\myemph{opposto} di $x$ e definiamo $x-y = x+(-y)$.
Denotiamo con $y^{-1}$ il \myemph{reciproco} di $y\neq 0$ e
definiamo $x / y = x\cdot y^{-1}$.

\begin{axiom}[totalmente ordinato]\label{axiom_order}
\mymargin{ordine}
\mymargin{$\le$}
Su $\RR$ è definita una relazione $\le$ con le seguenti proprietà
\begin{enumerate}
\item dicotomia: $x \le y$ o $y \le x$;
\item riflessiva: $x \le x$;
\item antisimmetrica: se $ x\le y$ e $y \le x$ allora $x=y$;
\item transitiva: se $x\le y $ e $ y \le z$ allora $x\le z$.
\end{enumerate}
\end{axiom}

Definiamo $x<y$ se $x\le y$ e $x \neq y$ e definiamo le relazioni
inverse $x \ge y$ se $y\le x$ e $x>y$ se $y<x$.
\mymargin{$\ge$ $<$ $>$}

\begin{axiom}[campo ordinato]
\mymargin{campo ordinato}
Le operazioni di campo e l'ordinamento sono compatibili nel senso che
valgono le seguenti proprietà:
\begin{enumerate}
\item positività: se $x\ge 0$ e $y \ge 0$ allora $x+y \ge 0$ e $x\cdot y\ge 0$;
\item monotonia: se $x \ge y$ allora $x+z \ge y+z$.
\end{enumerate}
\end{axiom}

\begin{axiom}[completezza]\label{axiom_complete}
\mymargin{completezza}
Se $A$ e $B$ sono sottoinsiemi non vuoti di $\RR$ tali che $A \le B$
(cioè: per ogni $a \in A$ e per ogni $b\in B$ vale $a\le b$) allora esiste
$x\in \RR$ tale che $A\le x \le B$ (cioè per ogni $a\in A$ e per ogni $b\in B$
vale $a\le x \le b$).
\end{axiom}


\begin{theorem}
In un campo ordinato valgono le seguenti
familiari proprietà:
\begin{enumerate}
  \item l'opposto e il reciproco sono unici (denotiamo con $-x$ l'unico opposto di $x$ e con $x^{-1}$ l'unico inverso di $x\neq 0$)
  \item $-(-x) = x$, $\enclose{x^{-1}}^{-1}$
  \item $x \cdot 0 = 0$
  \item $x\ge 0 \iff -x \le 0$
  \item $(-x)\cdot y = -(x\cdot y)$
  \item $-x = (-1)\cdot x$
  \item $(-1)\cdot(-1) = 1$
  \item $x\cdot x \ge 0$
  \item $1 > 0$
  \item se $x\cdot y = 0$ allora $x = 0$ o $y = 0$
  \item se $x>0$ e $y>0$  allora $x\cdot y > 0$
\end{enumerate}
\end{theorem}
%
\begin{proof}
\begin{enumerate}
\item
Supponiamo $y$ e $z$ siano due opposti di $x$ cioè $x+y=0$, $x+z=0$.
Allora da un lato $x+y+z = 0+z = z$, dall'altro $x+y+z = y+x+z= y+ 0 = y$.
Dunque $y=z$. Dimostrazione analoga si può fare per il reciproco.

\item
Se $x$ è opposto di $y$ allora $y$ è opposto di $x$ in quanto la somma
è commutativa. Dunque l'opposto di $-x$ è $x$ cioè $-(-x)=x$. Lo stesso
vale per il reciproco.

\item
Si ha
\[
x\cdot 0 = x \cdot 0 + x + (-x) %= x\cdot 0 + x\cdot 1 + (-x)
=x\cdot(0+1) + (-x) = x + (-x) = 0.
\]

\item
Se $x\ge 0$ sommando ad ambo i membri $-x$ si ottiene $x+(-x) \ge 0 + (-x)$
cioè $0 \ge -x$. Sommando $x$ ad ambo i membri si riottiene $x\ge 0$.


\item
Osserviamo che $(-x)\cdot y + x\cdot y = ((-x)+x)\cdot y = 0$ dunque $(-x)\cdot y$ è l'opposto di $x\cdot y$.

\item
Dunque $(-1)\cdot x = - (1 \cdot x) = - x$

\item
e per $x=-1$ si ottiene $(-1)\cdot(-1) = -(-1) = 1$.

\item
Si ha
\[
(-x)\cdot(-x) = (-1)\cdot x \cdot (-1)\cdot x = x\cdot x.
\]
Dunque se $x\ge 0$ per assioma di positività
abbiamo $x\cdot x\ge 0$ e se $x\le 0$ abbiamo $-x\ge 0$ e quindi
$x\cdot x = (-x)\cdot(-x) \ge 0$.

\item
In particolare per $x=1$ otteniamo $1\ge 0$.
Essendo inoltre per assioma $0\neq 1$ otteniamo $1> 0$.

\item
Se fosse $x\cdot y = 0$ e $x\neq 0$ allora $x$ avrebbe inverso $x^{-1}$
e avremmo:
\[
  y = x^{-1} \cdot x \cdot y = x^{-1}\cdot 0 = 0.
\]
Dunque o $x=0$ oppure $y=0$.

\item
Se $x>0$ e $y>0$ allora $x\ge 0$ e $y\ge 0$ da cui $x\cdot y\ge 0$.
Se fosse $x\cdot y=0$ uno dei due fattori si dovrebbe annullare
cosa che abbiamo escluso per ipotesi.
\end{enumerate}
\end{proof}

\begin{definition}[valore assoluto]
Definitamo il \myemph{valore assoluto} $\abs{x}$ di un numero $x\in \RR$ nel seguente modo
\[
\abs{x} =
\begin{cases}
  x & \text{se $x\ge 0$}, \\
  -x & \text{se $x<0 $}.
\end{cases}
\]
\end{definition}

\begin{proposition}[proprietà del valore assoluto]
Si ha
\begin{enumerate}
\item $\abs{x}\ge 0$ (positività)
\item $\big\lvert\abs{x}\big\rvert = \abs{x}$ (idempotenza)
\item $\abs{-x} = \abs{x}$ (simmetria)
\item $\abs{x\cdot y} = \abs{x}\cdot \abs{y}$ (omogenità)
\item $\abs{x+y} \le \abs{x} + \abs{y}$ (convessità)
\item $\abs{x-y} \le \abs{x-z} + \abs{z-y}$ (disuguaglianza triangolare)
\item $\big\lvert\abs{x}-\abs{y}\big\rvert \le \abs{x-y}$ (disuguaglianza triangolare inversa)
\end{enumerate}
Useremo inoltre spesso la seguente equivalenza (valida
anche con $<$ al posto di $\le$). Se $r\ge 0$ allora
\[
 \abs{x-y} \le r
 \iff
 y - r \le x \le y + r.
\]
\end{proposition}
%
\begin{proof}
Le prime quattro proprietà sono immediate conseguenze della definizione.

Dimostriamo innanzitutto l'ultima osservazione.
Se $x\ge y$ allora $x-y\ge 0$ e quindi $\abs{x-y} \le r$ è
equivalente a $x-y\le r$ cioè $x\le y+r$.
Se $x<y$ allora $x-y<0$ e quindi $\abs{x-y} \le r$ è
equivalente a $y-x \le r$ cioè $x\ge y-r$.
Viceversa se $y-r \le x \le y+r$ allora vale sia $x-y \le r$ che $y-x \le r$ e dunque $\abs{x-y}\le r$.

Osserviamo allora che per la precedente osservazione applicata
a $\abs{x-0} \le \abs{x}$ si ottiene
\[
  -\abs{x} \le x \le \abs{x}
\]
e sommando la stessa disuguaglianza con $y$ al posto di $x$ si
ottiene
\[
  -(\abs{x} + \abs{y}) \le x + y \le \abs{x} + \abs{y}
\]
che è equivalente alla proprietà di convessità:
\[
  \abs{x+y} \le \big\lvert\abs{x} + \abs{y}\big\rvert = \abs{x} + \abs{y}.
\]

Ponendo $y=z-x$ nella disuguaglianza precedente, si ottiene
\[
  \abs{z} \le \abs{x} + \abs{z-x}
\]
da cui
\[
  \abs{z} - \abs{x} \le \abs{z-x}.
\]
Scambiando $z$ con $x$ si ottiene la disuguaglianza opposta
e mettendole assieme si ottiene
la disuguaglianza triangolare inversa:
\[
\big\lvert \abs{z}-\abs{x} \big\rvert  \le \abs{z-x}.
\]

La disuguaglianza triangolare segue dalla convessità:
\[
 \abs{x-y} = \abs{x-z + z-y} \le \abs{x-z} + \abs{z-y}.
\]



\end{proof}

Osserviamo che dal punto di vista geometrico
$\abs{x-y}$ rappresenta la \emph{distanza} tra i punti
$x$ e $y$ sulla retta reale.

Come applicazione dell'assioma di completezza possiamo mostrare l'esistenza
della \myemph{radice quadrata}.
Più avanti, con qualche strumento in più, rivedremo più in generale
la costruzione della radice $n$-esima.

\begin{theorem}[radice quadrata]
Dato $y\ge 0$ esiste un unico $x\ge 0$ tale che $x^2=y$.
Tale $x$ verrà denotato con $\sqrt y$, \emph{radice quadrata} di $y$.
\mymargin{$\sqrt{\cdot}$}
\end{theorem}
\begin{proof}
Se $y=0$ allora è facile verificare che $x^2=y$ ha come unica soluzione $x=0$.
Supponiamo allora $y>0$ e
consideriamo i seguenti due insiemi
\[
  A = \{x\ge 0 \colon x^2 \le y\},\qquad
  B = \{x\ge 0 \colon x^2 \ge y\}
\]
e verifichiamo che soddisfino le ipotesi dell'assioma di completezza.
Innanzitutto $0\in A$ e quindi $A$ non è vuoto.
Neanche $B$ è vuoto in quanto $y+1\in B$,
infatti essendo $y+1\ge 1$ si ha
$(y+1)^2 \ge y+1$. Verifichiamo inoltre che $A \le B$.
Preso $a\in A$ e $b\in B$ si ha $a^2 \le y \le b^2$.
Se fosse $a>b$ dovremmo avere $a^2>b^2$, dunque $a \le b$.

Dunque possiamo applicare l'assioma di completezza
che ci garantisce l'esistenza di $z\in \RR$ tale che $A \le z \le B$.
Vogliamo ora verificare che $z^2 = y$.

Ci servirà innanzitutto sapere che $z>0$. Se $y\ge 1$ si avrebbe $1\in A$
e dunque $z\ge 1$ essendo $z\ge A$. Se $y<1$ allora $y^2 < y$ e dunque $y^2 \in A$
da cui si ottiene $z\ge y^2 > 0$.

Se fosse $z^2 < y$ vorremmo dimostrare che esiste $\eps>0$ tale che
$(z+\eps)^2 \le y$.
Questo succede se $(z+\eps)^2 = z^2 + 2 \eps z + \eps^2 \le y$.
Questo si può ottenere, ad esempio,
imponendo che sia $2\eps z \le (y-z^2)/2$ e $\eps^2 \le (y-z^2)/2$.
Cioè (ricordiamo che $z>0$) se $\eps \le (y-z^2)/(4z)$ e $\eps \le 1$
(in modo che $\eps^2 \le \eps$)
e $\eps \le (y-z^2)/2$. Dunque scegliendo
\[
\eps = \min\left\{ \frac{y-z^2}{4z}, 1, \frac{y-z^2}/2\right\}
\]
si osserva che $\eps>0$ e vale $(z+\eps)^2\le y$.
Dunque $z+\eps \in A$ e dunque non può essere $z\ge A$.

Se fosse $z^2 > y$ vorremmo dimostrare che esiste $\eps>0$ tale che
$(z-\eps)^2 \ge y$.
Questo succede se $(z-\eps)^2 = z^2 - 2\eps z + \eps^2 \le y$.
E' quindi sufficiente che sia $z^2 - 2 \eps z \le y$ ovvero basta scegliere
\[
  \eps = \frac{y-z^2}{2z}.
\]
Ma allora se $(z-\eps)^2\ge y$ si ha $z-\eps \in B$ e dunque non può
essere $z \le B$.

Rimane dunque l'unica possibilità che sia $z^2 = y$, come volevamo dimostrare.

Se ci fosse un altro $w\ge 0$ tale che $w^2 = y$ si avrebbe $w^2 - z^2=0$ ovvero
$(w-z)(w+z)=0$ da cui (ricordando che $z>0$ e quindi $w+z\neq 0$)
si ottiene $w-z=0$. Dunque $w=z$.
\end{proof}


\section{i numeri naturali, interi e razionali}

\begin{definition}[numeri naturali]
\mymargin{numeri naturali}
Un sottoinsieme $A\subset \RR$ si dice essere \emph{induttivo}
se $0\in A$ e $n\in A \implies n+1 \in A$.
La famiglia di tutti i sottoinsiemi induttivi di $\RR$ non è vuota
in quanto $\RR$ stesso è induttivo. Definiamo $\NN$ come l'intersezione
di tutti i sottoinsiemi induttivi di $\RR$ (ovvero: il più piccolo sottoinsieme induttivo di $\RR$).
\mymargin{$\NN$}
\end{definition}

Non è difficile dimostrare che l'insieme $\NN$ così definito
soddisfa gli assiomi di Peano (si vedano gli appunti di logica).

A partire da $\NN$ si può costruire l'insieme $\ZZ$ dei
\myemph{numeri interi}
\mymargin{$\ZZ$}
e l'insieme $\QQ$ dei \myemph{numeri razionali}:
\mymargin{$\QQ$}
\begin{align*}
  \ZZ
    &= \NN \cup (-\NN)
    = \{x\in \RR\colon \exists n\in\NN\colon (x=n) \lor (x=-n)\}, \\
  \QQ
    &= \frac{\ZZ}{\NN\setminus\{0\}}
    = \left\{x \in \RR\colon \exists p\in \ZZ\colon \exists q \in \NN\setminus\{0\}\colon x = \frac{p}{q}\right\}
\end{align*}

Si avrà dunque $\NN \subset \ZZ \subset \QQ \subset \RR$.
Si può verificare che $\QQ$ è un campo ordinato che però non soddisfa l'assioma
di completezza.
Infatti se consideriamo i due insiemi:
\[
 A = \{x\in \QQ: \colon x\ge 0, x^2 \le 2\},
 B = \{x\in \QQ: \colon x\ge 0, x^2 \ge 2\}
\]
si può verificare che $A$ e $B$ sono non vuoti, $A \le B$, ma l'elemento
di separazione $\sqrt{2}\in \RR$ non è elemento di $\QQ$, in base al seguente
classico risultato.

\begin{theorem}[Pitagora]
L'equazione $x^2=2$ non ha soluzioni in $\QQ$.
\end{theorem}
%
\begin{proof}
Supponiamo $x\in \QQ$ sia una soluzione di $x^2=2$.
Allora si potrà scrivere $x=p/q$ con $p\in \ZZ$ e $q\in \NN$, $q\neq 0$.
Possiamo anche supporre che la frazione $p/q$ sia ridotta ai minimi
termini cioè che $p$ e $q$ non abbiano fattori in comune.
Moltiplicando l'equazione
$(p/q)^2=2$ per $q^2$ si ottiene $p^2 = 2 q^2$.
Risulta quindi che $p^2$ è pari.
Ma allora anche $p$ è pari (perché il quadrato di un dispari è dispari).
Ma se $p$ è pari allora $p^2$ è multiplo di quattro.
Ma allora anche $2q^2$ è multiplo di quattro e quindi $q^2$ è pari.
Dunque anche $q$ è pari. Ma avevamo supposto che $p$ e $q$ non avessero
fattori in comune quindi questo non può accadere.
\end{proof}

Abbiamo quindi dimostrato che $\sqrt{2} \in \RR \setminus \QQ$
(diremo che $\sqrt 2$ è \myemph{irrazionale}) e dunque $\RR \neq \QQ$.

\section{estremo superiore}

\begin{definition}
Siano $x \in \RR$ e $A \subset \RR$.
Se $A\le x$ (ovvero $a\le x$ per ogni $a\in A$)
diremo che $x$ è un \myemph{maggiorante} di $A$.
Se $x \le A$ diremo invece che $x$ è un \myemph{minorante} di $A$.
Se $A$ ammette un maggiorante diremo che $A$ è \myemph{superiormente limitato},
se $A$ ammette un minorante diremo che $A$ è \myemph{inferiormente limitato},
se $A$ ammette sia maggiorante che minorante diremo che $A$ è \myemph{limitato}.

Se $A \le x$ e $x\in A$ diremo che $x$ è il massimo di $A$,
se $x\le A$ e $x\in A$ diremo che $x$ è il minimo di $A$

Se $x$ è minimo dei maggioranti di $A$ diremo che $x$ è
\myemph{estremo superiore}
di $A$ se invece $x$ è massimo dei minoranti diremo che $x$ è
\myemph{estremo inferiore} di $A$.
\end{definition}

Massimo e minimo di un insieme $A$, se esistono, sono unici.
Infatti se $x$ e $y$ fossero due massimi di $A$ si avrebbe $x\le y$ in
quanto $x\le A$ e $y\in A$. Analogamente si avrebbe $y\le x$ e
quindi $x=y$.
Anche l'estremo superiore e l'estremo inferiore se esistono sono
unici in quanto sono essi stessi un minimo ed un massimo
(rispettivamente dei maggioranti e dei minoranti).

Useremo quindi le notazioni:
\mymargin{$\max$ $\min$ $\sup$ $\inf$}
\[
 \max A, \qquad
 \min A, \qquad
 \sup A, \qquad
 \inf A
\]
per denotare univocamente (quando esistono) il massimo, il minimo,
l'estremo superiore e l'estremo inferiore di un insieme $A$.

\begin{theorem}[esistenza del $\sup$]
Se $A$ è un insieme non vuoto,
superiormente limitato, allora esiste l'estremo superiore di $A$.
Tale numero $x=\sup A$ è caratterizzato dalle seguenti proprietà
\begin{enumerate}
\item $\forall a\in A\colon x \ge a$;
\item $\forall \eps>0\colon \exists a\in A \colon x < a + \eps$.
\end{enumerate}

Risultato analogo vale per l'estremo inferiore.
\end{theorem}
%
\begin{proof}
Consideriamo l'insieme dei maggioranti
$B = \{ b\in \RR \colon b \ge A\}$.
Per ipotesi $B$ è non vuoto e per come è definito risulta $A\le B$.
Dunque dall'assioma di completezza deduciamo l'esistenza di un numero $x\in \RR$
tale che $A\le x \le B$. La prima disuguaglianza $A\le x$ ci dice che $x$ è un
maggiorante e quindi $x\in B$, la seconda $x\le B$ ci dice che $x$ è il minimo
di $B$ e quindi concludiamo che $x$ è l'estremo superiore di $A$.
La prima delle due proprietà caratterizzanti il $\sup$ traduce la condizione
che $x$ sia un maggiorante di $A$. La seconda delle due proprietà esprime il
fatto che $x$ sia il minimo dei maggioranti, infatti se $x$ è il minimo
dei maggioranti significa che nessun numero minore di $x$ è un maggiorante, ovvero
che ogni $x-\eps$ con $\eps>0$ non è un maggiorante, ovvero
che esiste $a\in A$ tale che $a > x-\eps$.
\end{proof}

La seguente proprietà dei numeri reali esprime il fatto
che non esistono gli \emph{infinitesimi} ovvero numeri reali positivi
che siano più piccoli di ogni $1/n$ con $n\in \NN$.

\begin{theorem}[proprietà archimedea dei numeri reali]
\mymargin{proprietà archimedea}
Dato $x\in \RR$ esiste $n\in \NN$ tale che $n > x$.
E se $x>0$ esiste $m\in \NN$ tale che $1/m < x$.
\end{theorem}
%
\begin{proof}
Se esistesse $x\in \RR$ tale che $x \ge \NN$
allora $\NN$ sarebbe superiormente limitato.
Dunque avrebbe un estremo superiore $m= \sup \NN$.
Siccome $m$ è il minimo dei maggioranti di $\NN$
e $m-1$ è più piccolo di $m$, allora $m-1$ non è un maggiorante
di $\NN$. Dunque deve esistere $n\in \NN$ tale che $n>m-1$.
Ma allora $n+1 > m$ ed essendo $n+1\in \NN$ troviamo che $m$
non poteva essere un maggiorante di $\NN$.

Dunque per ogni $y\in \RR$ esiste $n\in \NN$ tale che $n>y$.
Se $x\in \RR$ e $x>0$ allora posto $y=1/x$ possiamo trovare
$n\in \NN$ con $n>y = 1/x$ da cui $x > 1/n$.
\end{proof}

\begin{theorem}[parte intera]
Dato $x\in \RR$ esiste un unico $m\in \ZZ$ tale che $m-1 \le x < m$.
\end{theorem}
%
\begin{proof}
Supponiamo per un attimo che sia $x\ge 1$.
In tal caso consideriamo l'insieme $A=\{n\in \NN\colon n > x\}$.
Per la proprietà archimedea tale insieme non può essere vuoto e,
per il buon ordinamento di $\NN$ (si vedano gli appunti di logica),
deve avere un minimo $m$.
Dunque $m>x$ (in quanto $m\in a$) e $m\ge 1$ (in quanto $x\ge 1$).
Quindi necessariamente $m-1 \le x$ altrimenti avremmo che $m-1\in A$ e $m$
non poteva essere il minimo. Si ottiene dunque $m-1\le x < m$ come volevamo
dimostrare.

Nel caso fosse $x<1$ possiamo trovare un $N\in \NN$ (sempre per la proprietà archimedea) per cui $x+N \ge 1$. Applicando il ragionamento precedente a $x+N$ si trova comunque il risultato desiderato.
\end{proof}

\begin{definition}[parte intera]
\mymargin{parte intera}
Dato $x\in \RR$ denotiamo con $\lfloor x\rfloor$ l'unico intero
che soddisfa
\marginpar{$\lfloor\cdot\rfloor$} %% *** non viene bene nell'indice!
\[
  x - 1 < \lfloor x \rfloor \le x
\]
e denotiamo con $\lceil x \rceil = - \lfloor -x \rfloor$ l'unico intero che soddisfa (verificare!)
\marginpar{$\lceil\cdot\rceil$} %% *** non viene bene nell'indice!
\[
  x \le \lceil x \rceil < x + 1.
\]
Si ha dunque
\[
  \lfloor x \rfloor \le x \le \lceil x \rceil
\]
con entrambe le uguaglianze che si realizzano quando $x\in \ZZ$.
I due interi $\lfloor x \rfloor$ e $\lceil x \rceil$
sono la migliore approssimazione intera di $x$ rispettivamente
per difetto e per eccesso.
L'intero più vicino ad $x$ (approssimazione per arrotondamento)
è
\[
  \left\lfloor x + \frac 1 2 \right\rfloor
\quad \text{ossia} \quad
  \left\lceil x-\frac 1 2 \right\rceil
\]
(le due espressioni differiscono solamente quando $x$ si trova nel punto medio tra due interi consecutivi, nel qual caso la prima approssima per eccesso e la seconda per difetto).
\end{definition}
In alcuni testi si usa la notazione $[x]$ per denotare la parte intera $\lfloor x \rfloor$ e si definisce
anche la \emph{parte frazionaria}
\[
   \{x\} = x - [x]
\]
noi preferiamo evitare queste notazioni che possono risultare ambigue.

\begin{theorem}[densità di $\QQ$ in $\RR$]
\emph{densità di $\QQ$}
Dati $x,y \in \RR$ con $x<y$ esiste $q\in \QQ$ tale che $x<q<y$.
\end{theorem}
%
\begin{proof}
Per la proprietà archimedea dei numeri reali essendo $y-x>0$
deve esistere $n\in \NN$ tale che $y-x > 1/n$ così si avrà
\[
    nx + 1 < ny.
\]
Prendiamo allora $m=\lfloor nx + 1\rfloor$ cosicché si abbia
\[
  nx < m \le nx + 1.
\]
Mettendo insieme le due disuguaglianze e dividendo per n si ottiene,
come volevamo dimostrare,
\[
 x < \frac{m}{n} < y.
\]
\end{proof}


\begin{definition}[potenza intera]
\mymargin{potenza intera}
\mymargin{$x^n$}
Dato $x \in \RR$ possiamo definire la potenza $x^n$ per ogni
$n\in \NN$ come l'unica funzione che soddisfa
la seguente definizione per induzione (si vedano appunti di logica)
\[
\begin{cases}
  x^0 = 1\\
  x^{n+1} = x\cdot x^n.
\end{cases}
\]
Per $x\in \RR\setminus\{0\}$ possiamo anche definire $x^{-n}$ con $n\in \NN$
come
\[
x^{-n} = \frac{1}{x^n}.
\]
Risulta quindi che $x^n$ è definito per ogni $n\in \ZZ$ se $x\neq 0$.
\end{definition}

In base alla definizione si ha $x^0 = 1$, $x^1=x$, $x^2=x\cdot x$,
$x^3=x\cdot x \cdot x$ e così via. Dunque in generale
$x^n$ è il prodotto di $n$ fattori tutti uguali a $x$.

\begin{theorem}[proprietà delle potenze intere]
Per ogni $x,y\in \RR$, e per ogni $n,m \in \NN$
valgono le seguenti proprietà:
\begin{enumerate}
  \item  $x^{n+m} = x^n \cdot x^m$;
  \item $(x^n)^m = x^{n\cdot m}$;
  \item $(x\cdot y)^n = x^n \cdot y^n$;
  \item $\displaystyle \enclose{\frac{x}{y}}^n = \frac{x^n}{y^n}$ se $y\neq 0$.
\end{enumerate}
Le stesse proprietà valgono per $n,m \in\ZZ$ se $x\neq 0$, $y\neq 0$.
\end{theorem}
%
\begin{proof}
Dimostriamo, come esempio, solamente la prima proprietà: $x^{n+m} = x^n \cdot x^m$.
Fissato $m\in \NN$ procediamo per induzione su $n$.
Se $n=0$ si ha $x^{0+m} = x^m = x^m \cdot 1 = x^m \cdot x^0$.
Supponendo la proprietà sia stata verificata per $n$, verifichiamo
che vale anche con $n+1$ al posto di $n$. Si ha infatti
\[
 x^{(n+1)+m} = x^{n+m+1} = x \cdot x^{n+m} = x \cdot x^n \cdot x^m
  = x^{n+1} \cdot x^m.
\]
\end{proof}

\begin{exercise}
Si dimostri che $2^n > n$ per ogni $n\in \NN$.
\end{exercise}

%%%%%%%%%%%%%%%%%%%
%%%%%%%%%%%%%%%%%%%
%%%%%%%%%%%%%%%%%%%

\section{reali estesi}

%%%%%%%%%%%%%%%%%%%
%%%%%%%%%%%%%%%%%%%
%%%%%%%%%%%%%%%%%%%

\begin{definition}[reali estesi]
\mymargin{$\bar{\RR}$}
\mymargin{$+\infty$, $-\infty$}
Denotiamo con $\bar \RR=\RR \cup \{+\infty, -\infty\}$ l'insieme dei numeri reali
a cui vengono aggiunti due ulteriori \emph{quantità} che chiameremo
\emph{infinite} e che denotiamo con $+\infty$ e $-\infty$.
\end{definition}


Estendiamo la relazione d'ordine imponendo che valga
\[
  -\infty \le x \le +\infty, \qquad \forall x \in \bar\RR.
\]

Estendiamo anche la addizione e moltiplicazione
tra reali estesi imponendo che valga per ogni $x\in \bar \RR$
\begin{gather*}
  x + (+\infty) = +\infty, \qquad \text{se $x\neq -\infty$}\\
  x + (-\infty) = -\infty, \qquad \text{se $x\neq +\infty$}\\
  x \cdot (+\infty) = +\infty, \qquad
  x \cdot (-\infty) = -\infty, \qquad \text{se $x>0$} \\
  x \cdot (+\infty) = -\infty, \qquad
  x \cdot (-\infty) = +\infty, \qquad \text{se $x<0$}.
\end{gather*}

Si definiscono anche:
\[
 -(+\infty) = -\infty, \qquad
 -(-\infty) = +\infty, \qquad
 \frac{1}{+\infty} = \frac{1}{-\infty}=0
\]
facendo però attenzione che
questi formalmente non sono \emph{opposto}
e \emph{reciproco} in quanto
su $\bar R$ non sono più garantite
le regole: $x + (-x) = 0$ e $x \cdot (1/x) = 1$.
Infatti
le operazioni $(+\infty) + (-\infty)$ e $+\infty \cdot 0$ vengono
lasciate indefinite.

Definiamo anche il valore assoluto: $\abs{+\infty} = \abs{-\infty} = +\infty$.

Possiamo infine definire la sottrazione e la divisione tramite
addizione e moltiplicazione:
\[
  x - y = x + (-y), \qquad \frac{x}{y} = x \cdot \frac{1}{y}.
\]

Possiamo definire gli operatori $\sup$ e $\inf$
anche sugli insiemi illimitati ponendo:
\begin{align*}
  \sup A = +\infty \qquad \text{se $A$ non è superiormente limitato}\\
  \inf A = -\infty \qquad \text{se $A$ non è inferiormente limitato}
\end{align*}
e infine
\begin{align*}
  \sup \emptyset = -\infty\\
  \inf \emptyset = +\infty.
\end{align*}
Osserviamo infatti che su $\bar \RR$ la quantità $+\infty$
è maggiorante di qualunque insieme e $-\infty$ è minorante, dunque
queste definizioni mantengono su $\bar \RR$ le proprietà caratterizzanti:
l'estremo superiore è il minimo dei maggioranti e
l'estremo inferiore è il massimo dei minoranti.

\section{intervalli}

\begin{definition}[intervallo]
\mymargin{intervallo}
Un insieme $I\subset \RR$ si dice essere un \emph{intervallo}
se soddisfa la \emph{proprietà dei valori intermedi}:
\[
  \text{se $x, y \in I$ e $x<z<y$ allora $z \in I$.}
\]
\end{definition}
\begin{theorem}[caratterizzazione intervalli]
Sia $I$ un intervallo e sia $a=\inf I$, $b=\sup I$. Allora
$z\in I$ se $a < z < b$.
\end{theorem}
%
\begin{proof}
Se $I=\emptyset$ si ha $a>b$ e quindi nessun $z$ verifica $a<z<b$.
Supponiamo $I\neq \emptyset$ e
sia $a < z < b$.
Visto che $a$ è il massimo dei minoranti di $I$ deve esistere $x \in I$ tale
che $a \le x < z$ altrimenti ogni $x$ tra $a$ e $z$ sarebbe un minorante di $I$
e $a$ non sarebbe il minimo. Analogamente dovrebbe esistere $y\in I$ con $z<y\le b$.
Ma allora, per la proprietà dei valori intermedi anche $z\in I$.
\end{proof}

Il teorema precedente ci dice che una volta identificati i due estermi
di un intervallo, tutti i punti intermedi devono stare nell'intervallo.
Gli estremi, invece, possono essere o non essere inclusi nell'intervallo.
Punti esterni agli estremi non possono invece esserci.
Possiamo quindi caratterizzare tutti gli intervalli di $\bar \RR$
introducendo le seguenti notazioni. Dati $a,b\in \bar \RR$ con $a\le b$
poniamo
\begin{align*}
[a,b] &= \{x\in \bar \RR\colon a \le x \le b\} \\
[a,b) &= \{x\in \bar \RR\colon a \le x < b\} \\
(a,b] &= \{x\in \bar \RR\colon a < x \le b\}\\
(a,b) &= \{x\in \bar \RR\colon a < x < b\}.
\end{align*}
Abbiamo quindi utilizzato le parentesi quadre per indicare che gli estremi
sono inclusi e le parentesi tonde per indicare che gli estremi sono esclusi.
Osserviamo che in alcuni testi si usano le parentesi quadre rovesciate al posto
delle parentesi tonde.

Noi considereremo per lo più intervalli di $\RR$ (non di $\bar \RR$) e in tal
caso gli estremi infiniti non potranno mai essere inclusi.

%
%
%
%
\chapter{successioni}


Una \myemph{successione} numerica reale è una
funzione\footnote{%
Nella scrittura a mano si
sottolineano i caratteri invece di utilizzare il grassetto:
scriveremo $\underline a$ invece che $\vec a$.}
$\vec a\colon \NN \to \RR$.
L'insieme delle funzioni $\NN \to \RR$ viene usualmente indicato
con $\RR^\NN$. In effetti una successione $\vec a$ può essere interpretata
come un vettore con infinite componenti:
\[
  \vec a = (a_0, a_1, a_2, \dots)
\]
dove si intende
\[
   a_n = \vec a(n).
\]
Le componenti $a_n$ si chiamano \emph{termini} della successione.
I valori di $n$ si chiamano, invece, \emph{indici}.
L'intera
successione $\vec a$ può essere indicata con $(a_n)_{n=0}^\infty$
oppure $(a_n)_n$ oppure,
più semplicemente, con $a_n$ quando sia chiaro che si intende l'intera
successione $\vec a$ e non un singolo termine della stessa.

Le successioni vengono usualmente
considerate nei procedimenti di approssimazione.
Spesso infatti siamo interessati a capire qual è il numero (se esiste) a cui
la successione si avvicina al crescere di $n$.

\begin{definition}[successione convergente]
Diremo che una successione $a_n$ converge
\mymargin{convergenza}
\index{successione convergente}
ad un numero $\ell \in \RR$
ovvero ha limite finito
e scriveremo:
\mymargin{$a_n\to \ell$}
\[
  a_n \to \ell \qquad\text{(per $n\to +\infty$)}
\]
se scelto comunque un errore $\eps>0$ ogni termine della successione,
da un certo punto in poi, si trova a distanza inferiore di $\eps$
dal punto limite $\ell$. Formalmente:
\[
\forall \eps>0\colon \exists N\in\NN \colon \forall n\in \NN\colon
n>N \implies \abs{a_n - \ell}
< \eps.
\]
\end{definition}

\begin{example}
La successione $a_n = \frac{n}{n+1}$ converge a $\ell=1$:
\[
  \frac{n}{n+1}\to 1.
\]
\end{example}
\begin{proof}
Dato $\eps>0$ ci chiediamo quali siano gli indici $n$
per i quali risulta $\abs{a_n -1}<\eps$ e troviamo
che devono valere due disequazioni:
\[
  1- \eps < \frac{n}{n+1} < 1+\eps.
\]
Facilmente possiamo osservare che $n/(n+1)<1$ per ogni $n$, dunque
la seconda disequazione è sempre verificata. La prima disequazione
si riconduce a
\[
 n + 1 > \frac{1}{\eps}.
\]
Dunque qualunque sia $\eps>0$, scelto $N = \lceil 1 / \eps\rceil$
sappiamo che per ogni $n>N$
si ha $n+1 > n > N \ge \frac{1}{\eps}$ e quindi $1-\eps < a_n$.
Inoltre la disuguaglianza $a_n < 1 < 1+\eps$ è verificata per ogni $n$.
Abbiamo quindi verificato che vale la condizione che definisce
la convergenza.
\end{proof}


Le successioni possono anche avere limite infinito,
 in tal
caso si dice che divergono.
\begin{definition}[successione divergente]
\mymargin{divergente}
\index{successione divergente}
Una successione $a_n$ si dice avere limite $+\infty$
o divergere a $+\infty$
\mymargin{$a_n\to +\infty$}
\[
  a_n \to +\infty \qquad\text{(per $n\to +\infty$)}
\]
se comunque si scelga un numero reale, anche molto grande,
ogni termine della successione, da un certo punto in poi,
risulta essere maggiore di tale numero scelto. Formalmente
\[
  \forall M\in \RR\colon \exists N\in \NN \colon \forall n\in \NN\colon
  n>N \implies a_n >M.
\]
Definizione analoga si ha per il limite $-\infty$. Scriveremo
\mymargin{$a_n\to -\infty$}
\[
  a_n \to -\infty \qquad \text{(per $n\to +\infty$)}
\]
se
\[
  \forall M\in \RR\colon \exists N\in \NN \colon \forall n\in \NN\colon
  n>N\implies a_n < -M.
\]
\end{definition}

\begin{example}
Si ha
\[
  1000-n^2 \to -\infty.
\]
\end{example}
%
\begin{proof}
Per dimostrare che $1000-n^2\to -\infty$ sarà
necessario trovare per ogni $M\in \RR$
dei valori di $n$ per i quali si abbia $1000-n^2 < -M$.
Questo avviene se $n^2 > 1000 + M$. Visto che per ogni $n\in \NN$
si ha $n^2 \ge n$ (verificare!) sappiamo che se $n> 1000+M$ allora
anche $n^2 > 1000+M$. Dunque per ogni $M$ sarà sufficiente considerare
un numero intero $N \ge 1000 + M$
(ad esempio si potrebbe scegliere $N = \max\{0, \lceil 1000 + M\rceil\}$)
cosicché per ogni $n>N$ si avrebbe:
\[
 a_n = 1000 - n^2 \le 1000 - n < 1000 - N \le 1000 - (1000  + M) = -M
\]
come richiesto dalla definizione di limite $-\infty$.
\end{proof}

Volendo esprimere il concetto di limite in maniera uniforme
(senza dover distinguere limiti finiti e infiniti) possiamo
rendere la definizione un poco più astratta introducendo il concetto
di \emph{intorno}.
La condizione $\abs{a_n - \ell}< \eps$ può essere scritta in
modo equivalente come
$a_n \in (\ell-\eps, \ell+\eps)$. L'insieme $B_\eps(\ell) = (\ell-\eps, \ell+\eps)$
si chiama \myemph{intorno} di raggio $\eps$ centrato in $\ell$.
Possiamo quindi considerare la famiglia
di tutti questi intorni del punto $\ell$:
\[
 \U_\ell = \{B_\eps(\ell)\colon \eps>0 \}.
\]
Questa famiglia di insiemi si chiama \myemph{base di intorni} del punto $\ell$.
La definizione
di limite finito si può dunque riscrivere così:
\[
  a_n\to \ell
  \qquad \iff \qquad
  \forall U \in \U_\ell\colon \exists N\in \NN\colon \forall n>N\colon a_n\in U.
\]
Il vantaggio di questa \emph{astrazione} è che ora possiamo definire
gli intorni di $+\infty$ e di $-\infty$ nel modo seguente:
\begin{align*}
  \U_{+\infty} &= \{ (M,+\infty]\colon M\in \RR\}\\
  \U_{-\infty} &= \{ [-\infty, -M)\colon M \in \RR\}
\end{align*}
e la definizione data per il caso $\ell$ finito risulta valida anche nel
caso in cui $\ell$ sia infinito. Anzi, possiamo utilizzare gli intorni di
$+\infty$ anche per l'indice $n$, ottenendo questa definizione
generale
valida per ogni $\ell\in \bar \RR$:

\begin{definition}[definizione topologica di limite]
\mymargin{definizione topologica di limite}
\index{limite definizione topologica}
Sia $a_n$ una successione e sia $\ell \in \bar \RR$.
Diremo che $a_n$ ha limite $\ell$ e scriveremo
\[
  a_n \to \ell
\]
se
\[
  \forall U \in \U_\ell\colon \exists V \in \U_{+\infty} \colon \forall n\in \NN\colon
  n \in V \implies a_n\in U.
\]
\end{definition}

\begin{theorem}[unicità del limite]
\mymargin{unicità del limite}
Sia $a_n$ una successione e siano $\ell_1, \ell_2 \in \bar \RR$
tali che $a_n \to \ell_1$ e $a_n \to \ell_2$. Allora $\ell_1 = \ell_2$.
\end{theorem}
%
\begin{proof}
La dimostrazione si basa sul fatto che se $\ell_1 \neq \ell_2$ allora
esistono due intorni $U_1\in \U_{\ell_1}$ e $U_2 \in \U_{\ell_2}$
che sono tra loro disgiunti: $U_1 \cap U_2 = \emptyset$.

La verifica di questo fatto è piuttosto elementare.
Se $\ell_1$ ed $\ell_2$ sono entrambi finiti, è sufficiente
considerare $\eps = \abs{\ell_2-\ell_1}/2$ e si osserva immediatamente
che gli intervalli $(\ell_1-\eps, \ell_1+\eps)$ e $(\ell_2-\eps, \ell_2+\eps)$
sono disgiunti. Se $\ell_1$ ed $\ell_2$ sono entrambi finiti e sono diversi,
possiamo supporre $\ell_1=-\infty$ e $\ell_2=+\infty$. In tal caso gli
intorni $U_1= [-\infty,0)$ e $U_2=(0,+\infty]$ sono disgiunti.
Se $\ell_1$ è finito e $\ell_2 = +\infty$, basterà prendere
$U_1 = (\ell_1-1, \ell_1+1)$ ed $U_2 = (\ell_1+1,+\infty]$ per avere due
intorni disgiunti. Analogamente si procederà nel caso $\ell_2=-\infty$.

Una volta appurato che punti diversi $\ell_1\neq \ell_2$
ammettono intorni disgiunti $U_1$, $U_2$ sarà sufficiente
applicare la definizione di limite con entrambi i valori $\ell_1$
e $\ell_2$
per ottenere che da un lato esiste $N_1$ tale che per ogni
$n>N_1$ si ha $a_n \in U_1$, dall'altro esiste $N_2$ tale che per ogni
$n>N_2$ si ha $a_n \in U_2$. Ma allora per ogni $N> \max\{ N_1, N_2\}$
si dovrebbe avere $a_n \in U_1 \cap U_2 = \emptyset$ il che è impossibile.
\end{proof}

Osserviamo che non è detto che un limite esista, come si vede dal seguente
esempio.

\begin{example}
Sia $a_n = (-1)^n$. Non esiste $\ell \in \bar \RR$
tale che $a_n \to \ell$.
\end{example}
\begin{proof}
La successione $a_n$ ha come valori solamente i numeri $1$ e $-1$,
infatti se $n$ è pari si ha $(-1)^n=1$ e se $n$ è dispari $(-1)^n=-1$.
Supponiamo per assurdo che la successione abbia limite $\ell$
e consideriamo l'intorno $U = (\ell - 1, \ell + 1)$.
Se $\ell\ge 0$ certamente $-1 \not\in U$ in quanto $\ell-1\ge -1$.
Se invece $\ell \le 0$ certamente $1\not\in U$ in quanto $\ell+1\le 1$.
Dunque, in ogni caso, comunque si scelga $N\in \NN$ esisteranno degli
$n>N$ tali che $a_n \not \in U$.
\end{proof}

Abbiamo dunque osservato che in generale il limite di una successione può
non esistere ma se esiste è unico. Questo ci permette di definire
l'operatore di limite
che associa ad ogni successione che ammette limite
il suo (unico) limite in $\bar \RR$:
\mymargin{$\lim a_n$}
\[
   \lim a_n = \ell \qquad \text{se}\qquad a_n \to \ell.
\]
Per esplicitare il fatto che $a_n$ rappresenta
una intera successione e non un singolo valore,
(cioè $n$ è una variabile muta) si potrà anche scrivere
\[
  \lim_n a_n \qquad \text{oppure} \qquad \lim_{n\to+\infty} a_n.
\]

\begin{definition}[carattere di una successione]
Sia $a_n$ una successione. Se $a_n$ non ammette limite
si dice anche che $a_n$ è \emph{indeterminata}
\mymargin{successione indeterminata}
(si intende che è indeterminato il suo limite!).
Abbiamo dunque le seguenti alternative
\begin{enumerate}
 \item la successione è convergente (ha limite finito);
 \item la successione è divergente (ha limite infinito);
 \item la successione è indeterminata (non ha limite).
\end{enumerate}
Determinare il \emph{carattere}
\mymargin{carattere di una successione}
di una successione
significa specificare a quale delle tre categorie appartiene.
\end{definition}

\begin{theorem}[criteri di confronto]
\mymargin{confronto tra limiti}
\begin{enumerate}
\item
Se abbiamo due successioni $a_n$ e $b_n$, se per ogni $n$ si ha
\[
a_n \le b_n
\]
e se entrambe le successioni ammettono limite: $a_n \to a$ e $b_n \to b$
allora
\[
a \le b.
\]

\item
Siano $a_n$ e $b_n$ due successioni tali che per ogni $n$
si abbia
\[
a_n \le b_n\]
Se $a_n \to +\infty$ allora $b_n \to +\infty$.
Se $b_n \to -\infty$ allora $a_n \to -\infty$.

\item
(teorema dei carabinieri)
\mymargin{teorema dei carabinieri}
Se abbiamo tre successioni $a_n$, $b_n$ e $c_n$,
se per ogni $n$ vale
\[
a_n \le b_n \le c_n
\]
 e se le due
successioni $a_n$ e $c_n$ hanno lo stesso limite: $a_n \to \ell$ e $c_n\to \ell$
allora anche $b_n \to \ell$.
\end{enumerate}


\end{theorem}
%
\begin{proof}
\begin{enumerate}
\item
Se per assurdo fosse $a > b$ esisterebbero degli intorni disgiunti $U_a\in \U_a$
e $U_b \in \U_b$.
Inoltre si avrebbe $U_a > U_b$ (cioè: ogni punto di $U_a$ è maggiore
di ogni punto di $U_b$) visto che $a>b$.
Ma, dalla definizione di limite,
dovrebbero anche esistere $N_a$ e $N_b$ tali che per ogni $n>N_a$ si abbia
$a_n \in U_a$ e per ogni $n>N_b$ si abbia $b_n \in U_b$.
Ma allora per ogni $n> \max\{N_a, N_b\}$ si avrebbe
contemporaneamente $a_n \in U_a$ e $b_n \in U_b$ da cui $a_n > b_n$
contro l'ipotesi.

\item
Se $a_n \to +\infty$ per ogni $M\in \RR$ esiste $N\in \NN$ tale che
per ogni $n>N$ si ha $a_n > M$. Ma se $b_n>a_n$ si ha anche $b_n >M$
dunque anche $b_n \to +\infty$.
Dimostrazione analoga si ottiene nel caso $b_n \to -\infty$.

\item
Se $a_n$ e $c_n$ hanno lo stesso limite $\ell$ significa che per ogni
$U \in \U_\ell$ esistono $N_a$ ed $N_c$ tali che per ogni $n>N_a$ si ha
$a_n\in U$ e per ogni $n > N_c$ si ha $c_n\in U$.
Prendendo allora (al solito!) $N=\max\{N_a, N_b\}$ si trova che per
ogni $n>N$ si ha $a_n\in U$ e $c_n \in U$.
Visto che $U$ è un intervallo e visto che $a_n \le b_n  \le c_n$,
deduciamo che anche $b_n \in U$. Dunque la definizione di limite è verificata.
\end{enumerate}
\end{proof}

\begin{corollary}[permanenza del segno]
\mymargin{permanenza del segno}
Sia $a_n$ una successione e $c\in \RR$.
Se per ogni $n\in \NN$ si ha $a_n \ge c$ e se $a_n$ ha limite $\ell \in \bar \RR$
allora $\ell \ge c$.
In particolare se una successione ha valori non negativi
ed ammette limite, allora
il limite è non negativo.

Viceversa, se una successione $a_n$ ha limite positivo:
$a_n \to a >0$ allora $a_n>0$ per ogni $n$ tranne
al più un numero finito di termini.
 Risultato analogo vale se $a_n \to a <0$: a parte un numero
 finito di termini, i valori della successione sono negativi.
\end{corollary}
%
\begin{proof}
Prima parte.
E' sufficiente considerare la successione costante $c_n = c$
cosicché si ha $c_n \le a_n$ per ogni $n$. Ma ovviamente
la successione $c_n$ ha limite $c$ e dunque, per confronto,
deve essere $c\le \ell$.

Seconda parte. Se $a>0$ e $a\in \RR$
per definizione di limite esiste $N$
tale che per ogni $n>N$ si ha $a_n> a/2 > 0$. Dunque solo
un numero finito di termini (quelli con indice $n\le N$)
possono essere negativi. Se $a=+\infty$ il ragionamento
si ripete a maggior ragione sapendo che esiste $N$
tale che $a_n>1$ per $n>N$.
Cambiando segno alla successione si ottiene il caso $a<0$.
\end{proof}

Si presti molta attenzione al fatto che se $a_n$ è a termini positivi
non è detto che il limite sia positivo, possiamo solo affermare
che non è negativo. Ad esempio $1/(n+1)>0$ ma $1/(n+1) \to 0$ (verificare!).
In generale se una successione ha valori in un intervallo il suo limite,
se esiste, deve essere un punto del corrispondente intervallo chiuso.

Se una successione ha limite $0$ può avere infiniti termini
positivi e infiniti termini negativi, come nel caso
della successione $a_n = 1/(-2)^n$.

\begin{theorem}[successioni che differiscono su un numero finito di termini]
\mymargin{successioni che differiscono su un numero finito di termini}
Se $a_n$ e $b_n$ sono due successioni tali che l'insieme degli indici
$n$ su cui differiscono sia finito
(cioè $\#\{n\in \NN\colon a_n \neq b_n\} < \# \NN$)
allora se il limite di $a_n$ esiste e vale $\ell$ allora
anche il limite di $b_n$ esiste e vale $\ell$.
Viceversa se il limite di $a_n$ non esiste, non esiste neanche il limite
di $b_n$.
\end{theorem}
%
\begin{proof}
Un insieme di numeri naturali è finito se e solo se è limitato.
Dunque esiste $K\in \NN$ tale che per ogni $n>K$ si ha $a_n = b_n$.
Se $a_n$ ha limite $\ell$ soddisfa la proprietà
\[
 \forall U \in \U_\ell \colon
 \exists N\in \NN\colon \forall n > N \colon a_n \in U.
\]
Se al posto di $N$ mettiamo $N' = \max\{N,K\}$ sappiamo che per $n>N$
si ha anche $n>K$ e quindi $a_n=b_n$. Dunque anche la successione $b_n$
soddisfa la stessa proprietà.

Se la successione $a_n$ non ha limite allora pure $b_n$ non può avere limite
perché, se ce lo avesse, potremmo scambiare i ruoli di $a_n$ e $b_n$
e applicare il teorema già dimostrato concludendo che anche $a_n$ avrebbe limite.
\end{proof}

Il teorema precedente è molto utile perché ci permette di applicare
i criteri di confronto anche nel caso in cui le ipotesi siano violate
su un numero finito di termini, come nel seguente esempio.

\begin{example}
Sapendo che la successione $a_n = n$ tende a $+\infty$ dimostrare
che anche la successione $b_n = n^2-10$ tende a $+\infty$.
\end{example}
%
\begin{proof}
E' sufficiente osservare che $n^2-10 > n$ se $n\ge 4$ infatti
se $n\ge 4$ si ha
\[
n^2 - 10 \ge 4 n -10 = n + 3n - 10 \ge n + 12 -10 \ge n+2 > n.
\]
Dunque se consideriamo la successione $c_n$ ottenuta da $b_n$
modificando i primi quattro termini (ponendo ad esempio $c_0 = a_0$,
$c_1=a_1$, $c_2=a_2$ e $c_3=a_3$) si ottiene $c_n\ge a_n$ per ogni $n$.
Ma allora $c_n \to +\infty$ (per confronto con $a_n$) ma visto che
$b_n$ differisce da $c_n$ solo nei primi 4 termini anche $b_n\to +\infty$.
\end{proof}

Il teorema precedente garantisce inoltre
che per quanto riguarda lo studio del limite possiamo considerare
successioni che siano definite solamente da un certo indice in poi.
Ad esempio è molto frequente considerare successioni il cui primo indice sia
$n=1$ invece che $n=0$. Questo non cambia nulla per quanto riguarda il limite
della successione.

\begin{example}
La successione $a_n = 1/n$ è definita per $n\in \NN$ ma $n\neq 0$.
Ciò non toglie che possiamo studiarne il limite come qualunque altra
successione. Per evidenziare il fatto che il primo indice è $n=1$
si potrà usare la notazione $(a_n)_{n=1}^\infty$.
Per la cronaca: $1/n \to 0$.
\end{example}

Data una successione $a_n$ potremo considerare l'\emph{insieme
dei suoi valori}: $\{a_n\colon n\in \NN\}$.
Si tratta dell'immagine della funzione $n\mapsto a_n$
e a volte si chiama \myemph{supporto} della successione.
Si faccia attenzione al fatto che l'insieme dei valori
non descrive completamente la successione perché
viene persa l'informazione sull'ordine in cui vengono elencati
i termini della successione e sulla loro molteplicità (ogni valore
potrebbe essere assunti su molti indici diversi). Ad esempio
l'insieme dei valori della successione $a_n = (-1)^n$ è l'insieme $\{-1, 1\}$. Ma anche la successione
\[
 b_n = \begin{cases}
   -1 & \text{se $n\le 42$}\\
   1  & \text{altrimenti}
 \end{cases}
\]
ha lo stesso insieme dei valori. Si osservi però
che la prima successione non ammette limite mentre la seconda
è convergente (verificare!).

Le operazioni $\sup$, $\inf$, $\max$ e $\min$ che abbiamo
definito sugli insiemi, si intenderanno definite anche
sulle successioni, considerando l'insieme dei valori della successione.
Queste operazioni potrebbero essere sensibili anche ai primi termini della
successione (a differenza dell'operazione di limite) dunque potrebbe
essere necessario, per chiarezza, specificare qual è il primo indice
da cui si intende cominciare a considerare i valori. Ad esempio
se la successione $a_n$ è definita sui naturali tranne lo zero
si avrà:
\[
  \sup a_n = \sup_{n=1}^\infty a_n = \sup\{a_n \colon n \in \NN, n \ge 1\}.
\]

\begin{example}
Si consideri $a_n = \frac{1}{n+1}$ definita per $n\in \NN$.
Allora
\[
  \sup a_n = \max a_n = 1, \qquad
  \inf a_n = 0, \qquad \text{non esiste }\min a_n.
\]
\end{example}
\begin{proof}
Per ogni $n \in \NN$ si ha $n+1\ge 1$ e quindi $a_n = 1/(n+1) \le 1$.
Visto poi che $a_0 = 1$ si ottiene immediatamente che $\max a_n = 1$
e di conseguenza $\sup a_n = 1$.

Per verificare che $\inf a_n = 0$ dobbiamo verificare innanzitutto
che $0$ è minorante, e questo è vero in quanto $a_n = 1/(n+1)> 0$ essendo $n+1\ge 1 \ge 0$.
Inoltre dobbiamo verificare che per ogni $\eps >0$ esiste $n\in \NN$ tale
che $a_n < 0 + \eps = \eps$. Questo succede se $1/(n+1) < \eps$ ovvero
se $n > 1/\eps -1$ ad esempio per $n=\lceil 1/\eps\rceil$.
Abbiamo dunque verificato che $\inf a_n = 0$.
Il minimo di $a_n$ non esiste perché se esistesse dovrebbe essere uguale
all'estremo inferiore cioè dovrebbe essere $0$. Ma questo è impossibile
perché per ogni $n\in \NN$ si ha $a_n = 1/(n+1)\neq 0$.
\end{proof}

Diremo che una successione è limitata (superiormente / inferiormente) se
\mymargin{successioni limitate}
 l'insieme dei suoi valori
è un insieme limitato (superiormente / inferiormente).
Una successione $a_n$ è superiormente limitata se ammette un maggiorante:
\[
  \exists M\in \RR\colon \forall n \in \NN\colon a_n \le M
\]
ovvero il minimo dei maggioranti è finito:
\[
  \sup a_n < +\infty.
\]
Caratterizzazioni analoghe valgono per la limitatezza inferiore.

Una successione $a_n$
è limitata se la successione dei suoi valori assoluti $\abs{a_n}$ è superioremente limitata, ovvero:
\[
  \sup \abs{a_n} < +\infty.
\]
Questo discende dalla proprietà $-\abs{x} \le x \le \abs{x}$ valida
per ogni $x\in \RR$.

\begin{theorem}[limitatezza delle successioni convergenti]
\mymargin{limitatezza delle successioni convergenti}
Se $a_n$ è convergente allora $a_n$ è limitata.
Se $a_n\to +\infty$ allora $a_n$ è inferiormente limitata.
Se $a_n\to -\infty$ allora $a_n$ è superiormente limitata.
\end{theorem}
%
\begin{proof}
Sia $\ell \in \RR$ il limite di $a_n$.
Se $\ell$ è finito,
dalla definizione di limite (ponendo $\eps=1$) sappiamo che esiste $N\in \NN$
tale che per ogni $n> N$ si ha $\ell -1 < a_n < \ell+ 1$.
Prendiamo allora $M=\max\{a_0, a_1, \dots, a_N, \ell +1\}$
e $m =\min \{a_0, a_1, \dots, a_N, \ell-1\}$. Si avrà allora
che per ogni $n\in \NN$ vale
\[
  m \le a_n \le M
\]
e dunque $a_n$ è limitata.

Se $a_n \to +\infty$ allora, per definizione di limite, deve
esistere un $N$ tale che per ogni $n>N$ si abbia $a_n \ge 0$ (abbiamo
scelto arbitrariamente $M=0$ nella definizione). Ma allora
ponendo $K=\min\{a_0, a_1, \dots, a_N, 0\}$ si avrà che $a_n\ge K$ per ogni $n\in \NN$ dunque $a_n$ è inferiormente limitata.

Dimostrazione analoga si fa nel caso $a_n \to -\infty$.
\end{proof}


\begin{theorem}[limite della somma]
\mymargin{limite della somma}
Siano $a_n \to a$ e $b_n \to b$ con $a,b\in \bar \RR$.
Se $a$ e $b$ non sono
infiniti di segno opposto (nel qual caso $a+b$ non è stato definito),
allora
\[
    a_n + b_n \to a+b.
\]
Se $a$ e $b$ non sono infiniti con lo stesso segno allora
\[
   a_n - b_n \to a-b.
\]
\end{theorem}
%
\begin{proof}
Consideriamo inizialmente il caso in cui $a,b$ siano entrambi limiti finiti.
Allora per ogni $\eps>0$ esiste un $N$ (che, al solito, sarà il massimo tra un $N_a$ ed un $N_b$) tale per cui per ogni $n> N$ si ha
$\abs{a_n -a} < \eps$ e $\abs{b_n - b} < \eps/2$.

Risulta allora che per ogni $n> N$ si ha
\[
  \abs{(a_n + b_n) - (a+b)} \le \abs{a_n -a} + \abs{b_n -b} < \eps/2 + \eps/2 = \eps.
\]
Cioè $a_n+b_n \to a+b$, come volevamo dimostrare.

Se $a =+\infty$ e $b\neq -\infty$ allora la successione $b_n$ è inferiormente limitata cioè esiste $K\in \RR$ tale che $b_n \ge K$ per ogni $n$ e quindi $a_n+b_n \ge a_n + K$.
Visto che $a_n \to +\infty$ sappiamo che per ogni $M\in \RR$ esiste
$N\in \NN$ tal che per $n>N$ si ha $a_n > K - M$.
Ma allora $a_n + b_n >M$ da cui si ottiene la validità della definizione
di limite $a_n + b_n \to +\infty$.

Per completare la dimostrazione osserviamo che se $a_n\to \ell$ con $\ell \in \bar \RR$, allora $-a_n \to -\ell$. Si tratta semplicemente di cambiare
i segni nella definizione di limite.

Dunque se $a=-\infty$ e $b\neq +\infty$ possiamo cambiare segno a entrambe
le successioni e ricondurci al caso precedente. Questo completa la prima
parte della dimostrazione.

Per dimostrare la seconda parte (il limite della differenza è uguale alla differenza dei limiti) ci si riconduce alla prima parte, ricordando che la
differenza è la somma con l'opposto.
\end{proof}

\begin{theorem}[limite del valore assoluto]
\mymargin{limite del valore assoluto}
Se $a_n \to a$, $a \in \bar \RR$ allora
\[
  \abs{a_n} \to \abs{a}.
\]

Se $\abs{a_n} \to 0$ allora $a_n \to 0$.
\end{theorem}
%
\begin{proof}
Per la prima parte è sufficiente osservare che
(disuguaglianza triangolare inversa)
\[
  \big\lvert\abs{a_n} - \abs{a}\big\rvert \le \abs{a_n -a}
\]
\end{proof}

\begin{theorem}[prodotto di limitata per infinitesima]
\mymargin{prodotto limitata per infinitesima}
Se $a_n$ è una successione limitata e $b_n\to 0$ allora
$a_n\cdot b_n \to 0$.
\end{theorem}
%
\begin{proof}
Se $a_n$ è limitata esiste $M>0$ tale che $\abs{a_n}\le M$ per ogni $n\in\NN$.
Per la definizione di limite applicata a $b_n$, per ogni $\eps>0$
esiste $N\in \NN$ tale che per ogni $n>N$ si ha $\abs{b_n -0}=\abs{b_n} < \eps / M$. Allora per ogni $n>N$ si ha
\[
  \abs{a_n\cdot b_n - 0} = \abs{a_n\cdot b_n} \le M \abs{b_n} < M \eps /M = \eps.
\]
Dunque è verificata la definizione di limite $a_n \cdot b_n \to 0$.
\end{proof}

\begin{theorem}[limite del prodotto]
\mymargin{limite del prodotto}
Se $a_n \to a$ e $b_n \to b$
e se escludiamo il caso in cui uno dei due limiti
è zero e l'altro è infinito
allora
\[
  a_n \cdot b_n \to a\cdot b.
\]
\end{theorem}
\begin{proof}
Se $a$ e $b$ sono entrambi finiti si osserva che
\begin{align*}
  a_n \cdot b_n - a\cdot b
  &= a_n \cdot b_n - a_n \cdot b + a_n \cdot b - a\cdot b\\
  &\le a_n \cdot(b_n - b) + (a_n -a) \cdot b.
\end{align*}
Sappiamo che $b_n - b \to 0$ e $a_n - a \to 0$ (limite della differenza)
e sappiamo che $a_n$ è limitata e ovviamente la successione costante $b$ è anch'essa limitata.
Dunque (prodotto di limitata per infinitesima) si ha $a_n(b_n-b)\to 0$ e $(a_n-a)b\to 0$. E ancora applicando il limite della somma si ottiene
infine che $a_n b_n - ab\to 0$ il che è equivalente (sommo $ab$) ad $a_n b_n\to ab$, come volevamo dimostrare.

Se $a = +\infty$ e $b>0$ allora per la definizione di limite applicata a $b_n$
esiste $N_b\in \NN$ tale che per ogni $n>N$ si ha $b_n > b/2$. La definizione
di limite applicata ad $a_n$ ci dice invece che per ogni $M\in \RR$
esiste $N_a\in \NN$ tale che per ogni $n>N_a$ si ha $a_n > 2M/b$.
Deduciamo che per ogni $n> N=\max\{N_a, N_b\}$ si ha
\[
  a_n\cdot b_n > \frac{2M}{b}\frac{b}{2} = M.
\]
Si ottiene dunque la validità della definizione di limite $a_n b_n\to +\infty$.
Se $a= +\infty$ e $b<0$ oppure $a=-\infty$ e $b>0$ si può cambiare segno
ad una delle due successioni e ricondursi al caso precedente.
\end{proof}

\begin{theorem}[limite del reciproco]
\mymargin{limite del reciproco}
Sia $a_n \to a$.
Se $a\neq 0$ allora
\[
  \frac{1}{a_n} \to \frac{1}{a}.
\]
Se $a = 0$ ma $a_n>0$ per ogni $n\in \NN$ allora
\[
  \frac{1}{a_n} \to +\infty.
\]
Se $a=0$ ma $a_n<0$ per ogni $n\in \NN$ allora
\[
  \frac{1}{a_n} \to -\infty.
\]
\end{theorem}
%
\begin{proof}
Se $a$ è infinito allora
(che sia $a=+\infty$ o $a=-\infty$)
per ogni $\eps > 0$ esiste $N\in \NN$ tale che per ogni $n>N$ si ha
$\abs{a_n} > 1/\eps$. E dunque per ogni $n>N$
\[
\abs{\frac{1}{a_n}} < \eps
\]
che significa che $1/a_n \to 0$, come volevamo dimostrare.

Nel caso $0 < a < +\infty$ si ha
\begin{equation}\label{eq:lim_reciproco}
 \abs{\frac{1}{a_n} - \frac{1}{a}}
 = \abs{\frac{a - a_n}{a_n \cdot a}}
 = \abs{a_n -a} \cdot \frac{1}{\abs{a_n}\cdot a}
\end{equation}
Essendo $a_n \to a > 0$ esiste $N_a\in \NN$ tale
che per ogni $n>N_a$ si ha $a_n > a/2$. Dunque si ha
per ogni $n>N_a$:
\[
\frac{1}{\abs{a_n}\cdot a} \le \frac{2}{a^2}.
\]
Modificando un numero finito di termini possiamo dunque supporre
che la stima precedente sia valida per ogni $n\in \NN$
e dunque sul lato destro di \eqref{eq:lim_reciproco} abbiamo
il prodotto di una successione infinitesima per una successone limitata e dunque il limite è zero. Questo significa che $1/a_n \to 1/a$.

Il caso $-\infty < a < 0$ che può essere ricondotto al precedente cambiando segno ad $a_n$.

Consideriamo il caso $a=0$ con $a_n>0$ per ogni $n\in \NN$. In tal caso
dalla definizione di limite $a_n \to 0$ sappiamo che
per ogni $M>0$ esiste $N\in \NN$ tale che per $n>N$ si ha
$a_n = \abs{a_n} < 1/M$. Dunque per $n>N$ si ha $a_n > M$ ed
abbiamo ottenuto la validità della definizione di limite $a_n \to +\infty$.

Il caso $a=0$ con $a_n<0$ si riconduce al precedente cambiando
segno ad $a_n$.
\end{proof}

\begin{theorem}[limite del rapporto]
\mymargin{limite del rapporto}
Se $a_n \to a$ e $b_n\to b$ allora
\mymargin{limite del rapporto}
\[
  \frac{a_n}{b_n} \to \frac{a}{b}
\]
escludendo il caso in cui
$a$ e $b$ siano entrambi infiniti o entrambi nulli
e, nel caso in cui $b=0$ e $a\neq 0$, richiedendo che $b_n$
abbia sempre lo stesso segno (a meno di un numero finito di termini).
\end{theorem}
%
\begin{proof}
Possiamo ricondurci ai teoremi precedenti osservando che
\[
   \frac{a_n}{b_n} = a_n \cdot \frac{1}{b_n}.
\]
\end{proof}

\begin{definition}[funzioni monotòne]
\mymargin{funzioni monotòne}
\index{monotonia}
Una funzione $f\colon A \to \RR$ con $A\subset \RR$, si
dice essere
\begin{enumerate}
\item \myemph{crescente} se $x < y \implies f(x) \le f(y)$;
\item \myemph{decrescente} se $x < y \implies f(x) \ge f(y)$;
\item \myemph{monotòna} se crescente o decrescente;
\item \myemph{costante} se crescente e decrescente;
\item \myemph{strettamente crescente} se $x<y \implies f(x) < f(y)$;
\item \myemph{strettamente decrescente} se $x<y \implies f(x) > f(y)$;
\item \myemph{strettamente monotòna} se strettamente crescente o strettamente decrescente.
\end{enumerate}
\end{definition}

Si osservi che se $f\colon A \to \RR$ è costante allora esiste $c$ tale che
$f(x)=c$ per ogni $x\in A$. Si osservi anche che ogni funzione strettamente monotona è anche iniettiva. Si osservi infine (fare un esempio!) che esistono funzioni che non rientrano in nessuna delle categorie sopra elencate (cioè che non sono né crescenti né decrescenti).

Si faccia attenzione alla terminologia. In alcuni testi (in particolare nei testi anglosassoni) si utilizza il termine \emph{crescente} con il significato di \emph{strettamente crescente} e si usa la dizione \emph{non decrescente} per indicare il concetto che noi abbiamo definito con \emph{crescente}. In
effetti con le nostre definizioni una funzione crescente può essere costante
e quindi non crescere affatto! E' questo uno dei casi in cui il termine utilizzato nelle definizioni non corrisponde esattamente alla dizione utilizzata
nel linguaggio comune.

Le successioni sono funzioni $\NN \to \RR$ dunque le precedenti definizioni
si applicano anche alle successioni.
Per le successioni, tuttavia, si possono dare delle definizioni
equivalenti utilizzando il principio di induzione, come nel seguente
teorema la cui dimostrazione è immediata.

\begin{theorem}[successioni monotòne]
Una successione $a_n$ è
\begin{enumerate}
\item \emph{crescente}: se per ogni $n\in \NN$ si ha $a_{n+1} \ge a_n$;
\item \emph{decrescente}: se per ogni $n\in \NN$ si ha $a_{n+1} \le a_n$;
\item \emph{monotòna}: se crescente o decrescente;
\item \emph{costante}: se crescente e decrescente;
\item \emph{strettamente crescente}: se per ogni $n\in \NN$ si ha $a_{n+1}>a_n$;
\item \emph{strettamente decrescente}: se per ogni $n\in \NN$ si ha
$a_{n+1}<a_n$;
\item \emph{strettamente monotòna}: se strettamente crescente o strettamente decrescente.
\end{enumerate}
\end{theorem}

In una successione crescente la condizione $a_{n+1} \ge a_n$,
se utilizzata per induzione, ci permette
di ottenere che per ogni $m\ge n$ si ha $a_m \ge a_n$.
Lo stesso vale (mutando le relazioni) per le altre condizioni sopra definite.

\begin{theorem}[limite di successioni monotòne]
\mymargin{limite di successioni monotòne}
Ogni successione monotòna ammette limite.
Più precisamente: se $a_n$ è crescente allora $\lim a_n = \sup a_n$,
se $a_n$ è decrescente allora $\lim a_n = \inf a_n$.
\end{theorem}
%
\begin{proof}
Supponiamo sia $a_n$ crescente e sia $\ell = \sup a_n$.
Se $\ell$ è finito sappiamo che (caratterizzazione del $\sup$)
per ogni $\eps>0$ esiste $N\in \NN$ tale che $a_N> \ell -\eps$.
Ma siccome $a_n$ è crescente si avrà che per ogni $n>N$ vale
$a_n \ge a_N > \ell-\eps$.
D'altra parte sappiamo anche che $\ell\ge a_n$ per ogni $n\in \NN$
e dunque, mettendo insieme le due cose, si ottiene
\[
  \forall \eps>0 \colon \exists N\in \NN \colon \forall n\in \NN\colon
   n>N \implies \ell-\eps < a_n \le \ell < \ell + \eps.
\]
Abbiamo dunque verificato la definizione di limite $a_n \to \ell$.

Se  $\ell=+\infty$ sappiamo che $a_n$ non è superiormente limitata, cioè per ogni $M\in \RR$ esiste $N\in \NN$ tale che $a_N \ge M+1 > M$.
Essendo però $a_n$ crescente otteniamo anche che per ogni $n>N$ si
 ha $a_n \ge A_N > M$. Dunque si ottiene
 \[
 \forall M\in \RR\colon\forall N\in \NN\colon \forall n \in \NN\colon
  n>N \implies a_n > M
 \]
 che è la definizione di limite $a_n \to +\infty$.

Non può essere $\ell = -\infty$ in quanto il $\sup a_n \ge a_0 > -\infty$.
\end{proof}

\begin{definition}[sottosuccessione]
Se $a_n$ è una successione e $n_k$ è una successione strettamente crescente i cui valori sono numeri naturali, allora la successione
$b_k = a_{n_k}$ si dice essere una \myemph{sottosuccessione} di $a_n$
(o anche \emph{successione estratta} da $a_n$).
\end{definition}

Ricordando che una successione $a_n$ non è altro che una funzione
$a\colon \NN \to \NN$, la successione $n_k$ corrisponde ad una funzione
$n\colon \NN \to \NN$ e la sottosuccessione $a_{n_k}$ corrisponde alla funzione composta $a \circ n$.

Si osservi che nella definizione precedente la variabile $n$ rappresenta
una variabile muta quando scriviamo la successione $a_n$, ma
rappresenta anche il nome della successione fissata $n_k$.
Questo sovraccarico
di significato è voluto e se usato correttamente rende più semplice
le notazioni, in quanto la successione $n_k$ viene sostituita alla
variabile $n$, con lo stesso nome, nella successione $a_n$.
La sottosuccessione $a_{n_k}$ risulta essere una successione nella variabile $k$, non nella variabile $n$.

\begin{example}
Sia $a_n = n^2$ la successione dei quadrati perfetti:
\[
  a_0 = 0,\
  a_1 = 1,\
  a_2 = 4,\
  a_3 =9, \dots
\]
Consideriamo la successione dei numeri pari $n_k = 2k$:
\[
 n_0 = 0,\
 n_1 = 2,\
 n_2=4,\
 n_3=6, \dots
\]
la corrispondente sottosuccessione dei quadrati perfetti $b_k = a_{n_k}$
rappresenta la successione dei quadrati dei numeri pari:
\[
b_0 = a_0 = 0,\
b_1 = a_2 = 4,\
b_2 = a_4 = 16,\
b_3 = a_6 = 36, \dots
\]

Abbiamo in effetti \emph{estratto} alcuni dei termini della successione
originaria.
\end{example}

\begin{example}
Se $a_n = (-1)^n$ e $n_k=2k$ allora $a_{n_k} = 1$.
Vediamo quindi che una successione che non ammette limite
può contenere una sottosuccessione che invece ha limite.
\end{example}

\begin{theorem}[cambio di variabile nei limiti]
\mymargin{cambio di variabile nei limiti}
Se $n_k$ è una successione di numeri naturali con $n_k\to +\infty$
e se $a_n$ è una qualunque successione che ammette limite
\[
  \lim_k a_{n_k} = \lim_n a_n.
\]

Lo stesso vale se $a_{n_k}$ è una sottosuccessione
di $a_n$ (cioè nel caso in cui $n_k$ sia strettamente crescente).
\end{theorem}
%
\begin{proof}
Le definizioni di $a_n \to \ell$ e $n_k \to +\infty$ sono le seguenti
(usiamo la notazione con gli intorni per non dover distinguere
i casi di $\ell$ finito / infinito ma si potrebbe ugualmente
procedere con le definizioni usuali):
\begin{gather*}
 \forall U \in \U_\ell \colon \exists V \in \U_{+\infty} \colon
  a(V) \subset U,\\
 \forall V \in \U_{+\infty} \colon \exists W \in \U_{+\infty} \colon
  n(W) \subset V.
\end{gather*}
Mettendo insieme le due definizioni si ottiene
\[
  \forall U \in \U_\ell \colon
  \exists V \in \U_{+\infty} \colon
  \exists W \in \U_{+\infty} \colon
  n(W) \subset V \text{ e }
  a(V) \subset U
\]
e dunque
\[
  \forall U \in \U_\ell \colon \exists W \in \U_{+\infty} \colon
  a(n(W)) \subset U
\]
che corrisponde esattamente alla definizione di limite $a_{n_k} = a(n(k)) \to \ell$.

Per la seconda parte è sufficiente verificare che se $n_k$
è strettamente crescente
e a valori in $\NN$, deve necessariamente essere $n_k\to +\infty$.
Notiamo infatti che $n_k$ vista come funzione $n\colon \NN\to \NN$
è iniettiva (per la stretta monotonia) e dunque rappresenta una
corrispondenza biunivoca tra il suo dominio $\NN$ e l'insieme dei
suoi valori $\{n_k \colon k \in \NN\}$. Significa dunque che l'insieme
dei valori è un insieme infinito di numeri naturali che quindi non può che
essere illimitato. Dunque $\sup n_k = +\infty$ e dalla monotonia
otteniamo $\lim n_k = \sup n_k = +\infty$.
\end{proof}

\begin{theorem}[Bolzano-Weierstrass]
\mymargin{Bolzano-Weierstrass}
\index{teorema di Bolzano-Weierstrass}
Se $a_n$ è limitata allora esiste una sottosuccessione
$a_{n_k}$ convergente.
\end{theorem}
%
\begin{proof}
Sia $A_0=\inf a_n$ e $B_0=\sup a_n$. Essendo $a_n$ limitata sia $A_0$
che $B_0$ sono finiti e ogni termine della successione sta
nell'intervallo $[A_0,B_0]$. Definiamo $n_0$: ovviamente si avrà $a_{n_0} = a_0 \in [A_0, B_0]$.

Consideriamo il punto medio $M_0 = (A_0+B_0)/2$ dell'intervallo $[A_0,B_0]$ e consideriamo i due mezzi intervalli $[A_0,M_0]$ e $[M_0,B_0]$. Tutti i termini della successione stanno in almeno
uno di questi due intervalli.
Se consideriamo gli indici $n\in \NN$ della successione $a_n$, uno dei due sotto-intervalli deve contenere termini
della successione per infiniti indici.
Chiamiamo $[A_1, B_1]$
tale sottointervallo,
chiamiamo $n_1$ il più piccolo naturale maggiore di $n_0=0$
per cui $a_{n_1} \in [A_1,  B_1]$.

Ripetiamo il procedimento.
Consideriamo il punto medio $M_1$ dell'intervallo $[A_1,B_1]$.
Per costruzione l'intervallo contiene termini della successione
per infiniti indici dunque uno dei due sotto-intervalli $[A_1,M_1]$
o $[M_1,A_2]$ deve anche lui
contenere termini della successione per infiniti indici. Chiamiamo
$[A_2, B_2]$ tale intervallo e definiamo $n_2$ come il più piccolo
naturale maggiore di $n_1$ per cui $a_{n_2}\in [A_2, B_2]$.

Si può procedere così all'infinito (formalmente: tramite una definizione per induzione)
e ottenere quindi le successioni $A_k$, $B_k$ e $n_k$ che soddisfano le seguenti proprietà (da verificare con il principio di induzione):
\begin{enumerate}
\item $A_k$ è crescente, $B_k$ è decrescente, $A_k \le B_k$;
\item $B_k - A_k = (B_0-A_0)/2^k$;
\item $a_{n_k} \in [A_k, B_k]$.
\end{enumerate}

Essendo $A_k$ monotona sappiamo che esiste $\ell = \lim A_k$. Essendo poi $A_0 \le A_k \le B_k \le B_0$ sappiamo che $A_k$ è limitata, quindi $\ell$ è finito.
Inoltre
\[
 \lim B_k = \lim A_k + \frac{B_0-A_0}{2^k} = \lim A_k = \ell
 \]
 e dunque
passando al limite nelle disuguaglianze
\[
   A_k \le a_{n_k} \le B_k
\]
si ottiene (teorema dei carabinieri)
\[
  a_{n_k} \to \ell.
\]
\end{proof}

\begin{theorem}[Cantor: secondo metodo diagonale]
\mymargin{non numerabilità dei reali}
\index{secondo metodo diagonale di Cantor}
\index{Cantor}
\index{teorema di Cantor}
L'insieme dei numeri reali non è numerabile: $\#\RR > \#\NN$.
\end{theorem}
%
\begin{proof}
(si vedano gli appunti di logica per i concetti sulla cardinalità).
E' sufficiente dimostrare che $\#[0,1] > \#\NN$ in quanto
chiaramente $\#\RR \ge \#[0,1]$.
Supponiamo per assurdo che esista una funzione biettiva $a\colon \NN \to [0,1]$.
Questa funzione corrisponde dunque ad una successione $a_n$.
Consideriamo l'intervallo $[0,1]$ e dividiamolo in tre intervalli  di lunghezza $1/3$: $[0,1/3]$, $[1/3,2/3]$, $[2/3,1]$. Il punto $a_0$ non può stare in tutti e tre questi intervalli. Sia $[A_0,B_0]$ un intervallo (dei tre) che non contiene $a_0$: $a_0 \not \in [A_0,B_0]$.
Dividiamo anche $[A_0,B_0]$ in tre intervalli di lunghezza $1/9$.
Almeno uno di questi tre intervalli, che chiamiamo $[A_1,B_1]$,
non contiene $a_1$: $a_1 \not \in [A_1,B_1]$.
Procediamo così all'infinito in maniera simile al teorema precedente.
Otterremo due successioni $A_n$, $B_n$ che soddisfano queste proprietà:
\begin{enumerate}
\item $A_n$ crescente, $B_n$ decrescente;
\item $0\le A_n \le B_n \le 1$;
\item $a_n \not \in [A_n, B_n]$;
\item $B_n - A_n = 1/3^{n+1}$.
\end{enumerate}

Essendo $A_n$ monotona e limitata, essa ha limite finito $\lim A_n = \ell$.
Fissato $n$ osserviamo che per ogni $k\ge n$ si ha
\[
  A_n \le A_k \le B_n
\]
passando al limite in $k$ (con $n$ fissato) si ottiene
\[
  A_n \le \ell \le B_n
\]
che significa che $\ell \in [A_n, B_n]$ per ogni $n\in \NN$
(in particolare $\ell \in [0,1]$).
Visto che invece $a_n \not \in [A_n, B_n]$ risulta che per
ogni $n\in \NN$ si ha $\ell \neq a_n$.
Dunque il numero $\ell$ non è un termine della successione $a_n$
ovvero la funzione $a\colon \NN \to [0,1]$ non è suriettiva.
\end{proof}


\begin{definition}[continuità sequenziale]
\mymargin{continuità sequenziale}
Una funzione $f\colon A \subset \RR \to \RR$ si dice essere
\emph{(sequenzialmente) continua} se per ogni successione
convergente $a_n \to a$ con $a_n\in A$, $a \in A$ si ha
$f(a_n)\to f(a)$.
\end{definition}

Nel seguito potremo omettere l'avverbio "sequenzialmente" e parleremo
più semplicemente di \emph{funzioni continue}. Più avanti daremo
infatti una definizione ``più naturale'' di funzione continua
e dimostreremo che nel nostro ambito
la contintuità sequenziale è equivalente alla continuità.

\begin{example}
La funzione $f\colon \RR \to \RR$, $f(x) = x^2$ è sequenzialmente continua.
Infatti se $x_n\to x$ allora per il teorema sul limite del prodotto si ha
\[
  x_n^2 = x_n \cdot x_n \to x\cdot x = x^2.
\]
\end{example}

\begin{theorem}[degli zeri]
\mymargin{teorema degli zeri}
Sia $I\subset \RR$ un intervallo, $f\colon I \to \RR$ una funzione
continua, $a,b\in I$ tali che $f(a)\le 0$ e $f(b)\ge 0$.
Allora esiste $c\in I$ tale che $f(c)=0$.
\end{theorem}

\begin{proof}
La dimostrazione che adottiamo è di particolare rilevanza in quanto
non solo permette di dimostrare l'esistenza del punto $c$ che risolve
$f(x)=0$
ma ci presenta
un algoritmo, il \myemph{metodo di bisezione},
che può essere effettivamente utilizzato per approssimare
tale soluzione.

Possiamo supporre senza perdere di  generalità che sia $a<b$.
Poniamo $A_0 = a$, $B_0= b$ e consideriamo il punto medio $C_0 = (A_0+B_0)/2$.
Scegliamo tra i due intervalli $[A_0, C_0]$ e $[C_0,B_0]$ quello per cui
il segno ai due estremi è discorde (o, caso fortunato, nullo).
Più precisamente se $f(C_0)\ge 0$ poniamo $[A_1,B_1] = [A_0,C_0]$ altrimenti
scegliamo $[A_1,B_1] = [C_0,B_0]$ così si ha, in ogni caso,
$f(A_1)\le 0$, $f(B_1)\ge 0$.

Consideriamo il punto medio $C_1$ del nuovo intervallo $[A_1,B_1]$ e ripetiamo
il procedimento indefinitamente. Quello che otteniamo sono due successioni
$A_n$, $B_n$ con queste proprietà (che potrebbero essere dimostrate per induzione):
\begin{enumerate}
\item $A_n < B_n$, $B_n - A_n = (b-a)/2^n$;
\item $A_n$ è crescente, $B_n$ è decrescente;
\item $f(A_n)\le 0$, $f(B_n)\ge 0$.
\end{enumerate}

Essendo $A_n$ monotòna sappiamo che $A_n$ converge $A_n\to c$.
Inoltre visto che $A_n \in [a,b]$ anche $c\in [a,b]$ (per la permanenza del
segno delle successione $A_n-a$ e $b-A_n$).
Passando al limite nell'uguaglianza $B_n = A_n + (b-a)/2^n$
si ottiene che anche $B_n \to c$. Essendo $f$ continua
avremo
\[
f(A_n) \to f(c), \qquad
f(B_n) \to f(c).
\]
Ma $f(A_n)\le 0$ e quindi per la permanenza del segno anche $f(c)\le 0$.
D'altra parte $f(B_n) \ge 0$ e quindi $f(c)\ge 0$. Per dicotomia si
ottiene dunque $f(c) = 0$, come volevamo dimostrare.
\end{proof}

\begin{corollary}[proprietà dei valori intermedi]
\mymargin{proprietà dei valori intermedi}
Sia $I\subset \RR$ un intervallo e $f\colon I \to \RR$ una
funzione continua.
Allora se $f$ assume due valori $y_1$ e $y_2$ allora $f$
assume anche tutti i valori intermedi tra $y_1$ e $y_2$.
Detto altrimenti: una funzione continua
manda intervalli in intervalli.
\end{corollary}
%
\begin{proof}
Se $y_1$ e $y_2$ sono valori assunti da $f$ significa
che esistono $x_1,x_2 \in I$ tali che $f(x_1)= y_1$ e $f(x_2)=y_2$.
Allora scelto $y$ si consideri la funzione $g(x) = f(x)-y$.
Se $y$ è intermedio tra $y_1$ e $y_2$ la funzione $g$ assumerà
segni opposti in $x_1$ e $x_2$ e dunque, per il teorema degli zeri,
dovrà esserci un punto $x$ in cui $g$ si annulla. In tale punto
si avrà dunque $f(x)=y$, come volevamo dimostrare.
\end{proof}

\begin{exercise}
Si dimostri che se $I$ è un intervallo di $\RR$ ogni funzione $f\colon I \to \RR$
iniettiva e continua è strettamente monotona.
\end{exercise}

\begin{lemma}[caratterizzazione delle funzioni crescenti continue]
Sia $I\subset\RR$ un intervallo e sia $f\colon I \to \RR$ una
funzione crescente. Allora sono equivalenti
\begin{enumerate}
\item $f$ è continua;
\item per ogni $x\in I$
  \begin{enumerate}
    \item se $x\neq \inf I$ allora $f(x) = \sup f(\{y \in I\colon y<x\})$,
    \item e se $x\neq \sup I$ allora $f(x) = \inf f(\{y \in I \colon y>x\})$.
  \end{enumerate}
\end{enumerate}
\end{lemma}
%
\begin{proof}
Sia $A_x = \{y \in I\colon y<x \}$ e $B_x = \{y\in I \colon y>x\}$.
Se $f$ è crescente risulta sempre
\[
f(A_x) \le f(x) \le f(B_x)
\]
in quanto se $a\in A_x$ e $b\in B_x$ allora $a<x<b$ e quindi $f(a)\le f(x) \le f(b)$. Dunque
\[
\sup f(A_x) \le f(x) \le f(B_x).
\]

Supponiamo che $f$ sia continua e consideriamo un
qualunque $x\in I$.
Se $x\neq \inf I$ dobbiamo mostrare che $f(A_x)\ge f(x)$.
La successione $x_n = x - 1/n$ sta in $A_x$ per $n$
sufficientemente grande, dunque $\sup A_x \ge f(x_n)$.
Ma visto che $f(x_n)\to f(x)$ per confronto si ottiene $\sup A_x \ge f(x)$.
Analogamente se $x\neq \sup I$ prendendo la successione $x_n = x+1/n$
si trova che $\inf B_x \le f(x)$. Dunque $\sup f(A_x) = f(x) = \inf f(B_x)$ come volevamo dimostrare.

Supponiamo ora di sapere che
per ogni $x \in I$, tolti gli estremi di $I$, si abbia
$\sup f(A_x) = f(x) = \inf f(B_x)$.
Per ogni $\eps>0$, per la caratterizzazione di $\sup$ e $\inf$
dovranno allora esistere $y\in A_x$ e $z\in B_x$ tali che
$f(x) = \sup f(A_x) < f(y) +\eps$ e $f(x) = \inf f(B_x) > f(z)-\eps$.
Se $x_n\in I$ e $x_n \to x$ per $n$ abbastanza grande si dovrà avere
$y < x_n < z$ e quindi, per la monotonia di $f$: $f(y)\le f(x_n)\le f(z)$
da cui
\[
  f(x)-\eps < f(x_n) < f(x)+\eps.
\]
Significa allora che $f(x_n)\to f(x)$.

Se $x$ fosse un estremo di $I$, ad esempio se $x=\inf I$, si ripete
lo stesso ragionamento ma solo sul lato destro di $x$: per ogni $\eps>0$
esisterà $z\in B_x$ tale che $f(x) = \inf f(B_x) \ge f(z)-\eps$.
Ma se $x_n\in I$ dovrà essere $x_n\ge x$ (in quanto $x$ è l'estremo inferiore
di $I$) e quindi si avrà comunque
\[
  f(x) \le f(x_n) < f(x) + \eps
\]
da cui segue, per l'arbitrarietà di $\eps$, $f(x_n)\to f(x)$.
\end{proof}

\begin{theorem}[continuità della funzione inversa]
\mymargin{continuità della funzione inversa}
Sia $I\subset \RR$ un intervallo e sia $f\colon I \to \RR$ una
funzione continua strettamente crescente.
Allora posto $J=f(I)$ anche $J$ è un intervallo, $f\colon I\to J$ è
invertibile e $f^{-1}\colon J\to I$ è anch'essa
continua e strettamente crescente.
\end{theorem}
%
\begin{proof}
Che $J=f(I)$ sia un intervallo segue direttamente dal teorema dei valori intermedi, essendo $f$ continua. Essendo $f$ strettamente
crescente $f$ risulta essere iniettiva e quindi $f\colon I \to J$ è biettiva.
Esiste dunque la funzione inversa $f^{-1}\colon J \to I$.

Mostriamo ora che la stretta monotonia di $f^{-1}$ segue dalla stretta
monotonia di $f$. Presi $y_1 < y_2$ in $J$ poniamo $x_1=f^{-1}(y_1)$ e $x_2=f^{-1}(y_2)$ cosicché $y_1=f(x_1)$ e $y_2=f(x_2)$.
Se per assurdo fosse $f^{-1}(y_1)\ge f^{-1}(y_2)$ si avrebbe $x_1 \ge x_2$
e quindi $f(x_1)\ge f(x_2)$ cioè $y_1\ge y_2$, contro l'ipotesi $y_1 < y_2$.

Per mostrare che $f^{-1}$ è continua utilizziamo
il lemma precedente.
Dato $y \in J$ con $y\neq \inf J$, consideriamo l'insieme
$A'_y=\{t \in J\colon t<y\}$.
Essendo $f$ monotona e invertibile,
posto $x=f^{-1}(y)$ e $A_x=\{ s\in I\colon s<x\}$
si ha $f^{-1}(A'_y) = A_x$ e chiaramente $\sup A_x = x$.
Dunque $\sup f^{-1}(A'_y) = x = f^{-1}(y)$. In maniera analoga
si dimostra che $\inf f^{-1}(\{t\in J \colon t>y\}) = f^{-1}(y)$
quando $y\neq \sup J$. Dunque $f^{-1}$ è continua,
come volevamo dimostrare.
\end{proof}

%%%%%%%%%%%%%
%%%%%%%%%%%%%
%%%%%%%%%%%%%
\section{potenze e radici $n$-esime}
%%%%%%%%%%%%%
%%%%%%%%%%%%%
%%%%%%%%%%%%%
\index{potenza}
\index{radice}

\begin{theorem}[invertibilità della funzione potenza]
\mymargin{invertibilità della funzione potenza}
Per ogni $n\in \NN\setminus\{0\}$, la funzione
\begin{align*}
  f\colon [0,+\infty) &\to [0,+\infty)\\
   x &\mapsto x^n
\end{align*}
è strettamente crescente e biettiva.
Inoltre, se $n$ è dispari, la funzione
\begin{align*}
  f\colon \RR & \to \RR\\
    x &\mapsto x^n
\end{align*}
è strettamente crescente e biettiva.
\end{theorem}
%
\begin{proof}
La funzione $f$ è strettamente crescente perché
per l'assioma di monotonia del prodotto di numeri reali si
osserva che
se $x>y\ge 0$ allora $x^2 > y^2$, $x^3 > y^3$ e così via
(la dimostrazione andrebbe formalizzata, al solito, utilizzando il principio di
induzione).
Essendo strettamente crescente $f$ è anche iniettiva.
Per mostrare che $f\colon [0,+\infty) \to [0,+\infty)$ è suriettiva, consideriamo
un qualunque $y\in[0,+\infty)$ e prendiamo la funzione $g(x)=f(x)-y$.
Chiaramente $g(0) = 0^n - y = -y \le 0$. Se $y\le 1$ allora $g(1) = 1^n - y \ge 0$
altrimenti, se $y>1$, si ha $y^n>y$ e quindi $g(y) = y^n-y \ge 0$.
In ogni caso abbiamo verificato che esistono $a,b$ tali che $g(a)\le 0$ e $g(b)\ge 0$

Osserviamo che, per il teorema sul limite del prodotto,
se $x_k \to x$ allora $x_k^2 = x_k\cdot x_k \to x\cdot x = x^2$.
Questo dimostra che la funzione $x^2$ è (sequenzialmente) continua.
Per induzione si dimostra che $x^n$ è continua per ogni $n\in \NN$.
Per il teorema sul limite della differenza risulta che anche $g$ è
continua.
Dunque possiamo applicare il teorema degli zeri
per determinare l'esistenza di un $x\in[0,+\infty)$ tale che $g(x)=0$.
Visto che $x$ risolve l'equazione $f(x)=y$ e la surgettività è dimostrata.

Se $n$ è dispari si ha $(-x)^n = -(x^n)$.
In particolare per $x<0$ si ha $x^n<0$. Dunque se $x> y \ge 0$ si
ha $-x < y \le 0$ e
\[
  (-x)^n = -x^n < -y^n \le 0.
\]
E' quindi facile verificare che la funzione $x^n$
risulta strettamente crescente e bigettiva su tutto $\RR$.
\end{proof}

\begin{definition}[radice $n$-esima]
\mymargin{radice $n$-esima}
Se $n\in \NN$ è pari e non nullo
chiamiamo $\sqrt[n]{x}$ la funzione
inversa di $x^n$ come funzione $[0,+\infty)\to [0,+\infty)$.
Se $n\in \NN$ è dispari chiamiamo $\sqrt[n]{x}$
la funzione inversa di $x^n$ come funzione $\RR \to \RR$.

Si pone inoltre $\sqrt{x} = \sqrt[2]{x}$.
\end{definition}

\begin{theorem}[proprietà della radice $n$-esima]
\mymargin{proprietà della radice}
Per ogni $n,m \in \NN\setminus\{0\}$, $a\ge 0$, $b>0$ si ha:
\begin{enumerate}
\item $\sqrt[n]{a^n} = a$;
\item $\displaystyle \sqrt[nm]{a} = \sqrt[n]{\sqrt[m]{a}}$;
\item $\displaystyle \sqrt[n]{a\cdot b} = \sqrt[n]{a}\sqrt[n]{b}$;
\item $\displaystyle \sqrt[n]{\frac a b} = \frac{\sqrt[n]{a}}{\sqrt[n]{b}}$;
\item la funzione $x\mapsto \sqrt[n]{x}$ è continua e strettamente crescente.
\end{enumerate}
\end{theorem}

\begin{proof}
Le proprietà della radice vengono dedotte dalle corrispondenti proprietà
della potenza intera, sfruttando il fatto che la radice è la funzione inversa.
\end{proof}

\begin{theorem}[disuguaglianza di Bernoulli]
\mymargin{disuguaglianza di Bernoulli}
\index{Bernoulli}
Se $x > -1$ e $n\in \NN$ si ha
\[
(1+x)^n \ge 1 + nx.
\]
\end{theorem}
%
\begin{proof}
Lo dimostriamo per induzione su $n$. Per $n=0$, sostituendo si ottiene $1\ge 1$.
Supponendo che sia verificata la disuguaglianza per un certo $n$:
\[
(1+x)^n \ge 1 + nx
\]
moltiplicando ambo i membri per $1+x > 0$ si ottiene
\[
(1+x)^{n+1} \ge (1+x) (1+nx) = 1 + (n+1)x + n x^2
\ge 1 + (n+1)x
\]
che è proprio quello che volevamo dimostrare.
\end{proof}

\begin{theorem}[limite della radice $n$-esima]
\mymargin{limite della radice $n$-esima}
Se $a>0$
\[
   \lim_{n\to +\infty} \sqrt[n]{a} = 1
\]
\end{theorem}
%
\begin{proof}
Consideriamo innanzitutto il caso $a\ge 1$.
Posto $x=\sqrt[n]{a}-1$ nella disuguaglianza di Bernoulli, si ha
\[
a
= (1+x)^n
\ge 1 + nx
= 1 + n (\sqrt[n]{a}-1)
\]
da cui
\[
 \sqrt[n]{a} \le 1 + \frac{a-1}{n} \to 1 + 0 = 1.
\]
D'altra parte se $a\ge 1$ si ha $\sqrt[n]{a} \ge 1$
e dal confronto tra limiti si ottiene la tesi.

Se $a<1$ basta osservare che
\[
\sqrt[n]{a} = \frac{1}{\sqrt[n]{\frac 1 a}} \to 1
\]
per il caso precedente applicato con $1/a$ al posto di $a$.
\end{proof}

Siamo ora intenzionati a definire le potenze $x^y$ con esponente $y\in \RR$.
La funzione $f(x) = a^x$ si ottiene dal seguente teorema.

\begin{theorem}[funzione esponenziale]
\mymargin{funzione esponenziale}
Per ogni $a>1$ esiste una unica funzione $f\colon \RR \to \RR$
con le seguenti proprietà:
\begin{enumerate}
\item $f$ è crescente;
\item $f(1) = a$;
\item per ogni $x,y\in \RR\colon f(x+y) = f(x)\cdot f(y)$.
\end{enumerate}
Tale funzione risulta inoltre essere positiva,
strettamente crescente e
continua, soddisfa la relazione $f(-x)= 1/f(x)$ e
se $p\in \ZZ$ e $q\in \NN\setminus\{0\}$ risulta
\[
  f\enclose{\frac p q} = \sqrt[q]{a^p}.
\]
\end{theorem}
%
\begin{proof}
Dimostriamo innanzitutto che $f$ non si può annullare mai.
Infatti per ogni $x\in \RR$
\[
   a = f(1) = f(x+1-x) = f(x)\cdot f(1-x)
\]
e dunque se fosse $f(x)=0$ si avrebbe $a=0$ che abbiamo escluso per ipotesi.
Possiamo anzi dire che $f$ non è mai negativa in quanto
\[
  f(x) = f(x/2 + x/2) = f(x/2)^2 \ge 0.
\]
Osserviamo anche che
\[
  f(0) = f(0+0) = f(0)\cdot f(0)
\]
da cui, dividendo per $f(0)$ si ottiene $f(0)=1$.
E' anche facile verificare, per induzione, che
per ogni $n\in \NN$ si ha
\[
 f(nx) = (f(x))^n
\]
infatti $f((n+1)x) = f(nx+x)=f(nx)\cdot f(x)$.
In particolare se $n\in \NN$ si ha
\[
  f(n) = f(1)^n = a^n.
\]
Ma poi
\[
1=f(0) = f(x-x) = f(x)\cdot f(-x)
\]
e quindi $f(-x)=1/f(x)$ e quindi $f(nx) = (f(x))^n$ per ogni $n\in \ZZ$.
In particolare se $p\in \ZZ$ e $q \in \NN\setminus\{0\}$ si ha
\[
 a^p = f(p) = f(q\cdot p/q) = (f(p/q))^q
\]
da cui
\begin{equation}\label{eq:def_pow_0}
 f(p/q) = \sqrt[q]{a^p}.
\end{equation}

Dunque la funzione $f(x)$ è univocamente determinata per ogni $x\in \QQ$.
Osserviamo che, per le proprietà delle radici, se $p/q = n/m$ allora
\[
 \sqrt[m]{a^n}
 = \sqrt[mp]{a^{np}}
 = \sqrt[nq]{a^{np}}
 = \sqrt[q]{a^p}
\]
e questo significa che $f$ può effettivamente essere definita
coerentemente su tutto $\QQ$
tramite
la \eqref{eq:def_pow_0}.

Verifichiamo ora che $f$ deve essere strettamente crescente su $\QQ$.
Se $x,y \in \QQ$
con $x<y$ si avra $x=p/q$, $y=n/m$ con $pm < nq$ allora
$a^{pm} < a^{nq}$ e quindi
\[
f(x)
= \sqrt[q]{a^p}
= \sqrt[qm]{a^{pm}}
< \sqrt[qm]{a^{nq}}
= \sqrt[m]{a^n}
= f(y).
\]

Definiamo ora
\begin{align*}
  A_x &= \{t\in \QQ\colon t<x\} = \QQ \cap (-\infty,x),
  \\
  B_x &= \{t\in \QQ\colon t>x\} = \QQ \cap (x,+\infty).
\end{align*}
Ovviamente $A_x < x < B_x$ (nel senso che per ogni $\alpha \in A$
e per ogni $\beta \in B$ si ha $\alpha < x < \beta$).
Se vogliamo che $f$ sia crescente si dovrà quindi avere
$f(A_x) \le f(x) \le f(B_x)$ da cui in particolare
$\sup f(A_x) \le f(x) \le \inf f(B_x)$.

Vogliamo ora mostrare che deve essere
$\sup f(A_x) = \inf f(B_x)$
cosicché $f(x)$ sarà univocamente determinata
per ogni $x\in \RR$ da:
\begin{equation}\label{eq:def_pow_2}
  f(x) = \sup f(A_x) = \inf f(B_x).
\end{equation}
(ricordiamo infatti che su $\QQ$ $f$ è già stata univocamente determinata
e $A_x$ e $B_x$ sono sottoinsiemi di $\QQ$).

Dato qualunque $x\in \RR$ esiste $k\in \NN$ tale che $x < k$.
Scelto comunque $n \in \NN$, per la densità dei razionali
esistono $y,z\in \QQ$ tali che
\[
  y < x < z < k
\]
e tali che $z-y< \frac 1 n$. Allora si avrà
\begin{align*}
  0 \le f(z) - f(y)
  &= f(y)\cdot\enclose{\frac{f(z)}{f(y)}-1}
  \le f(k)\cdot \enclose{f(z-y)-1}\\
  &\le f(k)\cdot \enclose{f(1/n)-1}.
\end{align*}
Chiaramente $y\in A_x$ e $z\in B_x$ dunque
\[
\inf B_x - \sup A_x \le f(z) - f(y) \le f(k) \cdot (f(1/n)-1).
\]
Ricordiamo ora che $f(1/n) = \sqrt[n]{a} \to 1$ (teorema precedente)
dunque il lato destro della precedente disuguaglianza può essere reso
minore di qualunque $\eps>0$
e quindi, come voluto,
dovrà essere $\inf B_x = \sup A_x$.

Mostriamo ora che $f(x)$ deve essere strettamente crescente su tutto $\RR$,
ricordando
che abbiamo già verificato che $f$ è strettamente crescente su $\QQ$.
Ma presi $x,y\in \RR$ con $x<y$ esistono $z,w\in \QQ$ tale che $x<z<w<y$.
Allora $z\in B_x$ e $w\in A_y$, dunque
\[
  f(x) = \inf f(B_x) \le f(z) < f(w) \le \sup A_y = f(y).
\]

Il teorema di caratterizzazione delle funzioni monotone e continue
ci assicura che le condizione $f(x) = \sup A_x$ e $f(x)= \inf B_x$
garantiscono la continuità di $f$.

Infine, la relazione
\[
  f(x+y) = f(x) \cdot f(y)
\]
è verificata se $x,y \in \QQ$ (abbiamo costruito $f$ su $\QQ$ in modo
che questa relazione fosse valida).
Ma se $x,y \in \RR$ per densità esisteranno $x_n, y_n\in \QQ$ tali
che $x_n\to x$ e $y_n\to y$. Allora passando al limite
nella relazione
\[
  f(x_n+y_n) = f(x_n)\cdot f(y_n)
\]
sfruttando la continuità di $f$ si ottiene il risultato voluto.
\end{proof}

\begin{definition}[potenze con esponente reale]
\mymargin{potenze con esponente reale}
Sia $a\in \RR$ e $x\in \RR$.
\mymargin{$a^x$}
Se $a>1$ definiamo $a^x$ come
l'unica funzione definita dal teorema precedente.
Per $a=1$ definiamo $a^x = 1^x = 1$ per ogni $x\in \RR$.
Per $0<a<1$ definiamo $a^x = (1/a)^{-x}$.
\end{definition}

\begin{exercise}
Si dimostri che $a^{xy} = (a^x)^y$
e $(a\cdot b)^x = a^x \cdot b^x$.
(sfruttare l'unicità data dal teorema della funzione esponenziale).
\end{exercise}

Osserviamo che abbiamo dato due diverse definizioni di $x^y$.
La prima è quella delle potenze intere, valida se $x\in \RR$ e $y\in \ZZ$
e se $x\neq 0$ quando $y<0$. La seconda, quella delle potenze con esponente
reale è valida per $x>0$ e $y\in \RR$. Il teorema precedente ci garantisce
che le due definizioni coincidono quando $x>0$ e $y\in \ZZ$.

\begin{theorem}[limite dell'esponenziale]
\mymargin{limite dell'esponenziale}
Sia $a>1$ e $b>0$.
\begin{enumerate}
\item se $b_n \to +\infty$ allora $a^{b_n} \to +\infty$;
\item se $b_n \to -\infty$ allora $a^{b_n} \to 0$;
% \item se $a_n \to +\infty$ allora $(a_n)^b \to +\infty$.
\end{enumerate}
\end{theorem}
%
\begin{proof}
La disuguaglianza di Bernoulli garantisce che
\[
  a^n = (1+(a-1))^n \ge 1 + n(a-1).
\]
Ma allora se $x_n \to +\infty$
\[
  a^{x_n} \ge a^{\lfloor x_n \rfloor} \ge 1 + \lfloor x_n \rfloor (a-1)
  \ge 1 + (x_n-1)\cdot(a-1)
  \to +\infty.
\]
Se $x_n \to -\infty$ allora $-x_n \to +\infty$ ed essendo
$a^{-x_n} = 1/ a^{x_n}$ si ottiene il risultato voluto.
\end{proof}

%%%%%%%%%%%%%%%
%%%%%%%%%%%%%%%
%%%%%%%%%%%%%%%
\section{il logaritmo}
%%%%%%%%%%%%%%%
%%%%%%%%%%%%%%%
%%%%%%%%%%%%%%%

Fissato $a>1$ consideriamo la funzione esponenziale
$f\colon \RR \to \RR$, $f(x) = a^x$.
Sappiamo che $f$, essendo strettamente crescente, è iniettiva.
Il teorema sulla funzione esponenziale ci dice che $f(x)>0$ e quindi
$f(\RR) \subset (0,+\infty)$.
Visto però che $a^n\to +\infty$ e $a^{-n} \to 0$ scopriamo che
$\sup f(\RR)=+\infty$ e $\inf f(\RR)=0$.
Per la continuità di $f$ risulta però che $f(\RR)$ sia un intervallo
e dunque necessariamente $f(\RR) = (0,+\infty)$.
Dunque $f\colon \RR \to (0,+\infty)$ è biettiva.
Lo stesso vale nel caso $0<a<1$ (semplicemente per il fatto che $a^x = (1/a)^{-x}$).

\begin{definition}[logaritmo]
Per $a>0$, $a\neq 1$ definiamo $\log_a\colon (0,+\infty)\to \RR$,
chiamato \emph{logaritmo in base $a$}\mymargin{logaritmo},
la funzione inversa di $a^x$.
Si ha dunque
\[
  \log_a x = y \iff a^y = x.
\]
\end{definition}

\begin{theorem}[proprietà del logaritmo]
\mymargin{proprietà del logaritmo}
Se $a>0$, $a\neq 1$
\begin{enumerate}
  \item $\log_a(a^x) = x$;
  \item $\log_a(x\cdot y)= \log_a x + \log_a y$;
  \item $\log_a(x^y) = y \log_a (x)$;
  \item $\log_a(1) = 0$, $\log_a(a)=1$;
  \item  $\displaystyle  \log_a x = \frac{\log_b x}{\log_b a}$;
  \item se $a>1$ la funzione $\log_a(x)$ è strettamente crescente,
  se $0<a<1$ è strettamente decrescente;
  \item $\log_a(x)$ è una funzione continua;
  \item se $x_n \to +\infty$ allora $\log_a(x_n)\to +\infty$;
  \item se $x_n \to 0$, $x_n>0$ allora $\log_a(x_n)\to -\infty$.
\end{enumerate}
%
\begin{proof}
Tutte queste proprietà si ricavano direttamente dalle analoghe proprietà
dell'esponenziale.
\end{proof}
\end{theorem}

Per completezza enunciamo il seguente teorema che però
probabilmente non conviene memorizzare.
L'idea veramente rilevante, che si usa nella dimostrazione,
è il fatto che quando in una potenza variano sia la base
che l'esponente conviene riscrivere la potenza fissando
una base $c>1$ qualunque
tramite la seguente identità:
\[
  a^b = c^{b \log_c a}.
\]
In questo modo la base rimane fissata e quello
che varia è solo l'esponente.

\begin{theorem}[limite della potenza]
\mymargin{limite della potenza}
Siano $a_n$ e $b_n$ successioni
Se $a_n\to a$ e $b_n \to b$
sono successioni convergenti e $a>0$
allora
\[
  (a_n)^{b_n} \to a^b.
\]
Inoltre se $a_n \to a$, $a_n>0$, $b_n \to b$:
\begin{enumerate}
\item se $a<1$ e $b=+\infty$ allora ${a_n}^{b_n} \to 0$;
\item se $a<1$ e $b=-\infty$ allora ${a_n}^{b_n} \to +\infty$;
\item se $a>1$ e $b=+\infty$ allora ${a_n}^{b_n} \to +\infty$;
\item se $a>1$ e $b=-\infty$ allora ${a_n}^{b_n} \to 0$;
\item se $a=+\infty$ e $b>0$ allora ${a_n}^{b_n} \to +\infty$.
\end{enumerate}

\end{theorem}
%
\begin{proof}
Scelto un qualunque numero $c>1$ (ad esempio $c=2$)
si ha
\[
  a_n^{b_n} = c^{\log_c a_n^{b_n}}
   = c^{b_n \cdot \log_c a_n}.
\]

Per la continuità del logaritmo si ha $\log_c a_n \to \log_c a$
(oppure $\log_c a_n \to +\infty$ se $a=+\infty$),
per il teorema sul limite del prodotto si ha $b_n\cdot \log_c a_n \to b\cdot \log_c a$
e infine, per la continuità dell'esponenziale si ottiene
\[
  c^{b_n \cdot \log_c a_n} \to c^{b\log_c a} = a^b.
\]

Negli altri casi si procede con la stessa dimostrazione
e si osserva che il prodotto $b_n \cdot \log a_n$ non risulta essere
una forma indeterminata.
\end{proof}

Rimangono esclusi i seguenti casi (forme indeterminate):
\begin{enumerate}
\item $a=0$, $b=0$ forma indeterminata ``$0^0$'';
\item $a=1$, $b=+\infty$ forma indeterminata ``$1^{+\infty}$'';
\item $a=1$, $b=-\infty$ forma indeterminata ``$1^{-\infty}$'';
\item $a=+\infty$, $b=0$ forma indeterminata ``$(+\infty)^0$''.
\end{enumerate}

%%%%%%%%%%%%%%%%%
%%%%%%%%%%%%%%%%%
%%%%%%%%%%%%%%%%%
%%%%%%%%%%%%%%%%%
\section{La costante di Nepero}

La funzione esponenziale è legata ad un modello di crescita che si trova spesso
in natura: la \myemph{crescita esponenziale}.
Prendiamo come esempio una popolazione di batteri che cresce senza
limitazioni di spazio e di nutrimento\footnote{Esempi analoghi si potrebbero fare
con la crescita di un capitale tramite una rendita di interesse
o con il decadimento radioattivo.}.

Se $q(t)$ indica la numerosità della popolazione al tempo $t$ al tempo
$t+s$ ho una popolazione $q(t+s)$ che, fissato $s$, è proporzionale a $q(t)$,
perché ogni batterio avrà avuto la sua discendenza moltiplicata per un fattore
$k(s)$ indipendentemente dalla numerosità dei batteri.
Si avrà dunque
\[
  q(t+s) = q(t) \cdot k(s).
\]
In particolare fissato $q_0=q(0)$ si avrà
\[
 q(t) = q(0+t) = q_0 \cdot k(t)
\]
e ponendo $t=0$ si trova $k(0)=1$.
Inoltre da un lato
$  q(x+y) = q_0\cdot k(x+y) $
ma dall'altro
$q(x+y) = q(x) \cdot k(y) = q_0 \cdot k(x)\cdot k(y) $
da cui
\[
k(x+y) = k(x)\cdot k(y).
\]
Posto $a=k(1)$ e supponendo (cosa naturale)
che $q(t)$ sia crescente e quindi anche $k(t)$ sia crescente, dal
teorema di unicità della funzione esponenziale si ottiene che
\[
  k(t) = a^t
  \qquad\text{e
   quindi}
   \qquad q(t) = q_0 \cdot a^t.
\]

La costante $a$ nella formula $q_0\cdot a^t$ rappresenta
il coefficiente di crescita
di $q$ nel tempo unitario: $a = q(1) / q(0)$: questa costante dipende
però dall'unità di tempo scelta.
Sarebbe invece più naturale considerare la velocità di crescita relativa
\emph{istantanea} al tempo t:
\[
  c = \frac{q(t+\Delta t) - q(t) }{q(t) \cdot \Delta t}.
\]
Supponendo di conoscere $c$ possiamo cercare di calcolare $a=k(1)=q(1)/q_0$.
La definizione di $c$ può essere rivoltata per ottenere
\[
  q(t+\Delta t)
   = q(t) + c \cdot q(t) \cdot \Delta t
   = q(t)\enclose{1 + c \cdot \Delta t}.
\]
Scegliendo $t=0$ e $\Delta t = 1/n$ si ha
\[
    q(\Delta t)
    = q_0 + c \cdot q_0 \cdot \Delta t
    = q_0(1 + c \cdot \Delta t)
\]
poi
\[
  q(\Delta t + \Delta t) = q(\Delta t) (1+ c\cdot \Delta t) = q_0 (1+c\cdot \Delta t)^2
\]
e iterando
\[
  q\enclose{m \cdot \Delta t} = q_0 \enclose{1+ c \cdot \Delta t}^m.
\]
Dunque per calcolare $q(1)$ prendiamo $n\in \NN$ molto grande e poniamo
$\Delta t = 1/n$ cosicché:
\[
  q(1) = q( n\cdot \Delta t) = q_0 \enclose{1+{\frac c n}}^n.
\]

Nel caso $c=1$ quello che si ottiene per il coefficiente $a$ è la costante $e$
di Nepero, introdotta
nella definizione seguente. Vedremo che per $c$ generico ci si riconduce
comunque alla costante $e$ ottenendo $a=e^c$.

\begin{definition}[costante di Nepero]
Definiamo la \myemph{costante di Nepero}
\[
  e = \lim_{n\to +\infty} \enclose{1+\frac 1 n}^n.
\]
\end{definition}
La precedente definizione è giustificata dal seguente teorema.


\begin{theorem}[costante di Nepero]
La successione
\[
  a_n = \enclose{1+\frac 1 n}^n
\]
è crescente e limitata, dunque è convergente.
\end{theorem}
%
\begin{proof}
Dimostriamo innanzitutto che $a_n$ è crescente, cioè che
per ogni $n\ge 2$ si ha $a_n \ge a_{n-1}$.
E' chiaro che $a_n>0$ per ogni $n$,
quindi ci riconduciamo a
verificare che $\frac{a_n}{a_{n-1}} \ge 1$.

Si ha
\begin{align*}
\frac{a_n}{a_{n-1}}
&= \frac{\enclose{1+\frac 1 n}^n}{\enclose{1+\frac 1 {n-1}}^{n-1}}
= \frac{\enclose{\frac{n+1}{n}}^n}{\enclose{\frac{n}{n-1}}^{n-1}}\\
&= \enclose{\frac{n+1}{n}\cdot\frac{n-1}{n}}^n \cdot \frac{n}{n-1}
= \enclose{\frac{n^2- 1}{n^2}}^n \cdot \frac{n}{n-1}
\end{align*}
Osserviamo ora che la disuguaglianza di Bernoulli garantisce
\[
  \enclose{\frac{n^2 -1}{n^2}}^n
  = \enclose{1-\frac{1}{n^2}}^n
  \ge 1 - \frac{n}{n^2} = 1 - \frac{1}{n} = \frac{n-1}{n}
\]
da cui si ottiene, come volevamo, $a_n / a_{n-1} \ge 1$ cioè
$a_n$ è crescente.

Se ora consideriamo la successione
\[
  b_n = \enclose{1+\frac 1 n}^{n+1}
\]
osserviamo che si ha
\[
  b_n = \enclose{1+\frac 1 n}^n \cdot \enclose{1+\frac 1 n}
   = a_n\cdot \enclose{1+\frac 1 n} > a_n.
\]
Per dimostrare che $a_n$ è limitata sarà quindi sufficiente dimostrare
che $b_n$ è superiormente limitata. Vedremo ora che $b_n$ è decrescente (e quindi $a_n \le b_n \le b_1$ è superiormente limitata).

Procediamo in maniera analoga a quanto fatto per $a_n$:
\begin{align*}
\frac{b_{n-1}}{b_n}
& = \frac{\enclose{1+\frac{1}{n-1}}^n}{\enclose{1+\frac{1}{n}}^{n+1}}
  = \frac{\enclose{\frac{n}{n-1}}^n}{\enclose{\frac{n+1}{n}}^{n+1}}
  = \enclose{\frac{n}{n-1}\cdot\frac{n}{n+1}}^n\cdot\frac{n}{n+1} \\
& = \enclose{\frac{n^2}{n^2-1}}^n \frac{n}{n+1}
  = \enclose{1 + \frac{1}{n^2-1}}^n \frac{n}{n+1}.
\end{align*}
In base alla disuguaglianza di Bernoulli otteniamo
\[
  \enclose{1 + \frac{1}{n^2-1}}^n
  \ge 1 + n \frac{1}{n^2-1}
  \ge 1 + n\frac{1}{n^2}  = 1+\frac{1}{n} = \frac{n+1}{n}.
\]
Mettendo insieme le due stime si ottiene dunque $b_{n-1}/b_n \ge 1$
che è quanto ci rimaneva da dimostrare.
\end{proof}

\begin{exercise}
Posto $a_n = n! / n^n$ mostrare che $a_{n+1} / a_n \to e$.
\end{exercise}

\begin{theorem}[limiti che si riconducono al numero $e$]
\mymargin{limiti che si riconducono al numero $e$}
Se $a_k \to 0$, $a_k\neq 0$ allora, per $k\to +\infty$,
\[
  \enclose{1+a_k}^{\frac 1 {a_k}} \to e.
\]
\end{theorem}
%
\begin{proof}
Caso 1. Se $a_k > 0$ allora consideriamo la successione di naturali
$n_k = \lfloor 1/a_k \rfloor$ cosicché
\[
  n_k \le \frac{1}{a_k} \le n_k + 1
\qquad
\text{e}
\qquad
 \frac{1}{n_k+1} \le a_k \le \frac 1 {n_k}.
\]
dunque
\[
\enclose{1+\frac 1 {n_k+1} }^{n_k}
  \le \enclose{1+a_k}^{\frac 1 {a_k}}
  \le \enclose{1+\frac 1 {n_k}}^{n_k + 1}.
\]
Osserviamo ora che essendo che per $k\to +\infty$ anche
$n_k \to +\infty$ si deve
avere (in base al teorema di cambio di variabile nei limiti)
\[
 \lim_k \enclose{1 + \frac{1} {n_k}}^{n_k+1}\!\!\!
 = \lim_n \enclose{1+ \frac 1 n}^{n+1}\!\!\!
 = \lim_n \enclose{1+ \frac 1 n}^n\!\cdot\enclose{1+\frac 1 n}
 = e
\]
ed essendo anche $n_k + 1 \to \infty$
\[
  \lim_k \enclose{1 + \frac 1 {n_k+1}}^{n_k}
  = \lim_n \enclose{1+\frac 1 n}^{n-1}
  = \lim_n \frac{\enclose{1+\frac 1 n}^{n}}{1+\frac 1 n}
  = e.
\]
Dunque, per confronto tra i limiti, si ottiene $(1+a_k)^{\frac 1 {a_k}}\to e$.

Caso 2. Se $a_k<0$ possiamo scrivere $a_k = -\abs{a_k}$ da cui
\[
  \enclose{1+a_k}^{\frac 1 {a_k}}
  = \frac{1}{\enclose{1-a_k}^{\frac 1{a_k}}}
\]
e, procedendo come nel caso precedente, ci si riconduce al limite
\[
  \lim_n \frac{1}{\enclose{1-\frac 1 n}^n}.
\]
Per quest'ultimo osserviamo che si ha
\[
 \frac{1}{\enclose{1-\frac 1 n}^n}
 = \frac{1}{\enclose{\frac{n-1}{n}}^n}
 = \enclose{\frac{n}{n-1}}^n
 = \enclose{1 + \frac{1}{n-1}}^n
\]
che per quanto visto in precedenza (mettendo $n+1$ al posto di $n$)
ha anch'esso limite $e$.

Caso generale. Se $a_k$ ha segno variabile posso considerare
le due sottosuccessione dei termini di segno positivo e dei termini di segno
negativo. Per quanto visto nei casi precedenti entrambe le successioni
convergono ad $e$ e quindi è immediato verificare che l'intera
successione converge ad $e$.
\end{proof}

\begin{corollary}
Per ogni $x\in \RR$ si ha
\[
  \lim_{n\to +\infty} \enclose{1+ \frac x n}^n = e^x.
\]
\end{corollary}
%
\begin{proof}
Infatti, per il teorema precedente, posto $a_n = x/n$ si ha
\[
\lim_{n\to +\infty}\enclose{1+\frac x n}^{\frac n x} = e.
\]
Ma allora
\[
\enclose{1+ \frac x n}^n = \enclose{\enclose{1+\frac x n}^{\frac n x}}^x
\to e^x
\]
\end{proof}

\begin{corollary}
Se $a_n \to 0$, $a_n>0$ allora
\[
 \lim_{n\to +\infty} \frac{\ln \enclose{1+ a_n}}{a_n} = 1.
\]
\end{corollary}
%
\begin{proof}
Per ricondursi al teorema precedente basta osservare che
\[
  \frac{\ln(1+a_n)}{a_n}
  = \ln \enclose{(1+a_n)^{\frac 1 {a_n}}}.
\]
\end{proof}

\begin{exercise}
Mostrare che
\[
  \lim n^n\cdot \enclose{\frac{n+1}{n^2+1}}^n = e.
\]
\end{exercise}

\begin{exercise}
Mostrare che
\[
\lim n\cdot \ln\enclose{1 + \frac 1 n} = 1.
\]
\end{exercise}

\begin{definition}[logaritmi naturali]
Vedremo che il numero $e$ risulta essere una base naturale per la funzione
esponenziale e di conseguenza per il logaritmo. Il logaritmo in base
$e$ viene chiamato \myemph{logaritmo naturale} e viene indicato con $\ln = \log_e$.
\end{definition}

In alcuni testi si utilizza l'operatore $\log$, indicato senza una base esplicita,
ma la definizione non è completamente condivisa.
In certi testi (per lo più in ambito matematico)
si definisce $\log  = \ln = \log_e$,
in altri testi si considera $\log = \log_{10}$.

%%%%%%%%%%%%%%%%%%%
%%%%%%%%%%%%%%%%%%%
%%%%%%%%%%%%%%%%%%%
%%%%%%%%%%%%%%%%%%%
\section{ordini di infinito}

\begin{theorem}[criterio del rapporto alla Cesàro]
\mymargin{criterio del rapporto alla Cesàro}
\index{Cesàro}
Sia $a_n$ una successione a termini positivi.
Se
\[
  \frac{a_{n+1}}{a_n} \to \ell \in \RR
\]
allora
\[
 \sqrt[n]{a_n}\to \ell.
\]
\end{theorem}
%
\begin{proof}
Chiaramente deve essere $\ell\ge 0$. Supponiamo inizialmente
che sia $\ell>0$ e scegliamo qualunque $\eps>0$, con $\eps < \min\{1,\ell\}$.
Per la definizione di limite $a_{n+1}/a_n \to \ell$ esisterà $N$ tale che per $n>N-1$ si abbia
\[
  \ell -\eps < \frac{a_{n+1}}{a_n} < \ell + \eps.
\]
Osserviamo che in generale, per ogni $n\ge N$ si ha
\[
 a_n  = \frac{a_n}{a_{n-1}}\cdot\frac{a_{n-1}}{a_{n-2}}
 \cdot\cdots\cdot
     \frac{a_{N+2}}{a_{N+1}}\cdot\frac{a_{N+1}}{a_N}
     \cdot a_N.
\]
Dunque potendo stimare,
dall'alto e dal basso,
ognuno degli $n-N$ fattori di questo
prodotto di rapporti, si ottiene
\[
(\ell -\eps)^{n-N} \cdot a_N < a_n < (\ell +\eps)^{n-N} \cdot a_N
\]
da cui
\[
(\ell -\eps)^{\frac{n-N}{n}} \cdot \sqrt[n]{a_N}
< \sqrt[n]{a_n}
< (\ell +\eps)^{\frac{n-N}{n}} \cdot \sqrt[n]{a_N}
\]
ovvero
\[
(\ell -\eps) \cdot \sqrt[n]{\frac{a_N}{(\ell -\eps)^N}}
< \sqrt[n]{a_n}
< (\ell +\eps) \cdot \sqrt[n]{\frac{a_N}{(\ell -\eps)^N}}.
\]
Ricordiamo ora che per $n\to +\infty$
\[
\sqrt[n]{\frac{a_N}{(\ell -\eps)^N}} \to 1
\]
e quindi esisterà un $K>N$ tale che per ogni $n>K$ si ha
\[
 1-\eps < \sqrt[n]{\frac{a_N}{(\ell -\eps)^N}} < 1+\eps.
\]
Mettendo insieme le cose abbiamo ottenuto che
\[
 (\ell - \eps)(1-\eps)
 < \sqrt[n]{a_n}
 < (\ell + \eps)(1+\eps)
 \]
che garantisce la validità del limite $\sqrt[n]{a_n}\to \ell$.

Nel caso $\ell = 0$
non potremo richiedere $\eps < \ell$
e quindi le stime dal basso andranno rimpiazzate
dalla stima ovvia
$\sqrt[n]{a_n}\ge 0$.
\end{proof}

\begin{exercise}
Si applichi il risultato precedente per
verificare che
\[
   \lim \sqrt[n]{n} = 1
\]
e (più difficile)
\[
  \lim \frac{n}{\sqrt[n]{n!}} = e.
\]
\end{exercise}

\begin{corollary}[criterio del rapporto]
\mymargin{criterio del rapporto}
Sia $a_n$ è una successione a termini positivi tale che
il rapporto dei termini consecutivi abbia limite $\ell\in [0,+\infty]$:
\[
  \frac{a_{n+1}}{a_n} \to \ell.
\]
Se $\ell < 1$ allora $a_n \to 0$. Se $\ell>1$ allora $a_n \to +\infty$.

\end{corollary}
%
\begin{proof}
Applicando il teorema precedente sappiamo che $b_n = \sqrt[n]{a_n}\to \ell$. Ma allora
\[
  a_n = \enclose{b_n}^n \to
  \begin{cases}
    0 & \text{se $\ell < 1$}\\
    +\infty & \text{se $\ell > 1$}
  \end{cases}
\]
in quanto $b_n = \to \ell$ e $\ell^n \to 0$ se $\ell<1$ mentre $\ell^n \to +\infty$
se $\ell>1$.
\end{proof}

Osserviamo che, nel teorema precedente, non si può concludere niente nel
caso in cui sia $\ell = 1$. Infatti le due successioni $a_n = 1/n$ e $b_n = n$
hanno limiti diversi ($a_n \to 0$, $b_n\to +\infty$) ma per entrambe
il limite del rapporto di termini consecutivi tende ad $\ell=1$.


\begin{comment}
\begin{proof}
Per la definizione di limite $a_{n+1}/a_n \to \ell$
scelto $q$ con $\ell < q < 1$ e posto $\eps = q-\ell$
esiste $N\in \NN$ tale che per ogni $n>N-1$
si ha
\[
  \frac{a_{n+1}}{a_n} \le \ell+\eps = q
\]
cosicché $a_{n+1} \le q \cdot a_n$. Alla seconda
iterazione
\[
  a_{n+2} \le q a_{n+1} \le q^2 a_n
\]
e per induzione si può mostrare quindi che
per ogni $n\ge N$ si ha
\[
 a_{n+k} \le q^k a_n
\]
ovvero
\[
 a_{N+k} \le q^k a_N.
\]
Ponendo
\[
  c = \max \enclose{\{1\}\cup \left\{\frac{a_n}{q^n}\colon n=0,\dots,N\right\}}
\]
si ottiene che vale (per forza) per ogni $n\in \NN$
\[
  a_n \le c \cdot q^n.
\]
In particolare essendo $a_n \ge 0$ e $c\cdot q^n \to 0$ per $n\to +\infty$
per confronto  si ottiene
che $a_n \to 0$.
\end{proof}
\end{comment}

\begin{definition}[ordine di infinito/infinitesimo]
\mymargin{ordine di infinito/infinitesimo}
Se $a_n$ e $b_n$ sono successioni a termini positivi, diremo che
per $n\to +\infty$ la successione $a_n$ è \emph{molto più piccola}
della successione $b_n$ e scriveremo $a_n \ll b_n$ se vale
\mymargin{$\ll$}
\[
\frac{a_n}{b_n} \to 0.
\]
Se $a_n \ll b_n$ diremo anche che $b_n$ è \myemph{molto più grande}
di $a_n$ e scriveremo $b_n \gg a_n$.
\mymargin{$\gg$}
\end{definition}

\begin{theorem}[ordini di infinito]
\mymargin{ordini di infinito}
Per ogni $a>1$ si ha, per $n\to +\infty$
\[
a^n \ll n! \ll n^n.
\]
Se $a>1$, $\alpha>0$ e $x_n \to +\infty$ si ha,
per $n\to+\infty$
\[
\log_a(x_n) \ll (x_n)^\alpha \ll a^{x_n}.
\]
\end{theorem}
%
\begin{proof}
Cominciamo col mostrare che $a^n \ll n!$
applicando il criterio del rapporto alla successione $\frac{a^n}{n!}$:
\[
\frac{\displaystyle \frac{a^{n+1}}{(n+1)!}}{\displaystyle \frac{a^n}{n!}}
= \frac{a^{n+1}}{a^n}\cdot \frac{n!}{(n+1)!}
= a \cdot \frac {1}{n + 1} \to 0 < 1.
\]
Dunque si ha, come richiesto $a^n / n! \to 0$.
Si procede in modo analogo per mostrare che $n! \ll n^n$:
\begin{align*}
\frac{(n+1)!}{n!}\cdot \frac{n^n}{(n+1)^{n+1}}
&= (n+1) \cdot \enclose{\frac{n}{n+1}}^n \frac {1}{n+1}\\
&= \frac{1}{\enclose{1+\frac 1 n}^n} \to \frac 1 e < 1.
\end{align*}

Per la seconda parte del teorema cominciamo col dimostrare un caso particolare
e cioè
\[
  n^\alpha \ll a^n.
\]
Si può procedere con il criterio del rapporto, come nei casi precedenti:
\[
\frac{(n+1)^\alpha}{n^\alpha}\cdot \frac{a^n}{a^{n+1}}
= \frac 1 a \cdot \enclose{\frac{n+1}{n}}^\alpha \to \frac 1 a \cdot 1^\alpha = \frac 1 a < 1
\]
da cui $n^\alpha / a^n \to 0$.

Se ora $x_n\to +\infty$ è qualunque
cerchiamo di ricondurci ad una successione a valori interi.
Osserviamo che si ha
\[
\lfloor x_n \rfloor
\le x_n
\le \lfloor x_n \rfloor + 1
\]
da cui, per monotonia,
\[
\lfloor x_n \rfloor^\alpha
\le x_n^\alpha
\le (\lfloor x_n \rfloor + 1)^\alpha
= \lfloor x_n \rfloor^\alpha \enclose{1+ \frac{1}{\lfloor x_n \rfloor}}^\alpha
\]
e
\[
a^{\lfloor x_n \rfloor}
\le a^{x_n}
\le a^{\lfloor x_n \rfloor + 1}
= a \cdot a^{\lfloor x_n \rfloor}.
\]
Dunque
\[
\frac{\lfloor x_n \rfloor^\alpha}{a \cdot a^{\lfloor x_n \rfloor}}
\le \frac{x_n^\alpha}{a^{x_n}}
\le \frac{\lfloor x_n \rfloor^\alpha \enclose{1+ \frac{1}{\lfloor x_n \rfloor}}^\alpha}
    {a^{\lfloor x_n \rfloor}}.
\]
Ma ora, se $n\to +\infty$ sapendo che $\lfloor x_n\rfloor \to +\infty$ si ha
che (per il teorema di sostituzione del limite)
\[
\lim \frac{\lfloor x_n \rfloor^\alpha}{a^{\lfloor x_n \rfloor}} = 0
\qquad
\text{e}
\qquad
\lim \frac{\lfloor x_n \rfloor^\alpha }
    {a^{\lfloor x_n \rfloor}} = 0
\]
da cui segue che $\frac{x_n^\alpha}{a^{x_n}}\to 0$.

Per dimostrare l'ultima relazione, $\log_a(x_n)\ll (x_n)^\alpha$,
consideriamo la successione $y_n = \alpha \cdot \log_a x_n$
cosicché $a^{y_n} = x_n^\alpha$.
Notiamo che se $x_n\to +\infty$
anche $y_n \to +\infty$.
Dunque, per le proprietà precedenti,
sappiamo che $y_n \ll a^{y_n}$ e dunque
\[
\frac{\log_a x_n}{x_n^\alpha}
= \frac{1}{\alpha}\cdot\frac{y_n}{a^{y_n}} \to 0.
\]
\end{proof}

\begin{exercise}
Calcolare
\[
  \lim_{n\to +\infty} \frac{\displaystyle \ln\sqrt{n^2+n^n}}{\displaystyle e^{1 + \ln n}\cdot \ln(n^2-n\sqrt n)}.
\]
\end{exercise}

%%%%%%%%%%%%%%%%%%%%%%%%
%%%%%%%%%%%%%%%%%%%%%%%%
%%%%%%%%%%%%%%%%%%%%%%%%
%%%%%%%%%%%%%%%%%%%%%%%%
\chapter{i numeri complessi}
%%%%%%%%%%%%%%%%%%%%%%%%
%%%%%%%%%%%%%%%%%%%%%%%%
%%%%%%%%%%%%%%%%%%%%%%%%
%%%%%%%%%%%%%%%%%%%%%%%%

Dal punto di vista geometrico l'insieme $\CC$ dei \myemph{numeri complessi}
può essere visto come un modello del piano euclideo.
\mymargin{$\CC$}
Il piano euclideo è uno spazio affine reale di dimensione 2.
Possiamo mettere delle coordinate sul piano se fissiamo un punto $O$ (origine)
e due vettori ortonormali $e_1$, $e_2$. Chiamiamo $0$ il vettore
nullo $\vec{OO}$ e chiamiamo $1$ il vettore $e_1$.
La retta passante per $O$ con direzione $e_1$ rappresenta i numeri reali
$\RR$ come abbiamo già visto. Chiamiamo $i$ il vettore $e_2$.
La retta passante per $O$ con direzione $e_2$ verrà chiamata
\emph{retta dei numeri immaginari}.

Un generico punto del piano $z$ potrà essere scritto in maniera univoca
nella base scelta: $z = x e_1 + y e_2$ ovvero, per come abbiamo definito $e_1$ ed
$e_2$:
\[
z = x + i y.
\]
Tale $z$ viene chiamato
\emph{numero complesso} con parte reale $x$ e parte immaginaria $y$.
Questa rappresentazione del numero complesso $z$ viene
chiamata \myemph{rappresentazione cartesiana} in quanto definisce
il punto $z$ del piano complesso tramite le sue coordinate cartesiane
$x$ e $y$.
I numeri reali sono \emph{immersi} nei complessi, nel senso che se
$x\in \RR$ allora $z= x + i\cdot 0 = x$ è anche un numero complesso.
Il numero complesso $i = 0 + i\cdot 1$ viene chiamata \myemph{unità immaginaria}
e i numeri complessi della forma $iy$ sono chiamati \emph{immaginari}.
\mymargin{numeri immaginari}
Un numero
complesso $z = x+iy$ è quindi una somma tra un numero reale ed un numero
immaginario. Il numero reale $x$ viene chiamato \myemph{parte reale} di $z$ e
si denota a volte con $x=\Re z$.
\mymargin{$\Re z$}
Il numero reale $y$ viene chiamato
\myemph{parte immaginaria} di $z$ e si denota con $y=\Im z$
\mymargin{$\Im z$}
(osserviamo che la parte immaginaria di un numero complesso è un numero
reale, non immaginario). Dunque $z= \Re z + i \Im z$.

L'insieme dei numeri complessi viene denotato con $\CC$.
Lo spazio $\CC$, per come
è stato costruito, è uno spazio vettoriale reale di dimensione $2$.
Abbiamo quindi già definite la \myemph{addizione}
tra elementi di $\CC$ e la moltiplicazione
tra elementi di $\CC$ ed elementi di $\RR$.
Se $a,b,c,d,t\in \RR$ si ha:
\begin{gather*}
 (a+ib) + (c+id) = (a+c) + i (b+d), \\
 t(a+ib) = ta + itb.
\end{gather*}

Vogliamo estendere la \myemph{moltiplicazione} a tutte le coppie di numeri complessi.
Imponendo (arbitrariamente) che valga $i\cdot i = -1$ e che rimanga
valida la proprietà distributiva si ottiene
questa definizione:
\[
   (a+ib) \cdot (c+id) = (ac-bd) + i(ad+bc).
\]

Si può verificare che questa moltiplicazione estende quella "scalare" definita
in precedenza. Inoltre l'insieme $\CC$ equipaggiato delle due operazioni di
addizione e moltiplicazione risulta essere un campo.

Osserviamo che su $\CC$ non si definisce una operazione d'ordine perché
in effetti non è possibile definire un ordine "compatibile" con le operazioni
appena definite.

Su $\CC$ definiamo delle ulteriori operazioni.
Il \myemph{coniugato} di un numero complesso $z=x+iy$ è il numero
$\bar z = x - iy$. Geometricamente l'operazione di coniugio è una simmetria
rispetto alla retta reale. I numeri reali sono in effetti punti fissi del
coniugio (il coniugato di un numero reale è il numero stesso).
Osserviamo che si ha
\[
z \cdot \bar z = (x+iy)(x-iy) = x^2-i^2y^2 = x^2+y^2.
\]

Il \myemph{modulo} di un numero complesso $z=x+iy$
è il numero reale $ \abs{z} = \sqrt{z\cdot\bar z} = \sqrt{x^2+y^2}$.
Geometricamente tale quantità rappresenta la distanza del punto $z$
dal punto $0$ e quindi la distanza tra due numeri complessi $z$ e
$w$ si potrà rappresentare con $\abs{z-w}$.

Osserviamo che se $z = x \in \RR \subset \CC$ il modulo di $z$ coincide
con il valore assoluto: $\abs{z} = \sqrt{x^2} = \abs{x}$ e per questo
motivo non distinguiamo, nelle notazioni, il modulo dal valore assoluto.

Il modulo di un numero complesso soddisfa (come il valore assoluto)
le seguenti proprietà
\begin{enumerate}
\item $\big\lvert\abs{z}\big\rvert = \abs{z}$ (idempotenza),
\item $\abs{-z} = \abs{z}$, $\abs{\bar z}$ (simmetria),
\item $\abs{z+w} \le \abs{z}+\abs{w}$ (convessità),
\item $\abs{z-w} \le \abs{z-v} + \abs{v-w}$ (disuguaglianza triangolare),
\item $\abs{z\cdot w} = \abs{z}\cdot\abs{w}$ (omogenità).
\end{enumerate}

Possiamo a questo punto trovare una utile formula per calcolare
il reciproco di un numero complesso. Essendo infatti
$z\cdot \bar z = \abs{z}^2$ si osserva che
\[
  \frac{1}{z}
  = \frac{\bar z}{ \bar z \cdot z}
  = \frac{\bar z}{\abs{z}^2}.
\]

\section{rappresentazione polare dei numeri complessi}

I numeri complessi di modulo uno vengono chiamati \emph{unitari}.
\mymargin{complessi unitari}
\index{unitario}
Geometricamente i numeri complessi unitari sono i punti della circonferenza
unitaria centrata nell'origine del piano complesso.
Se $z=x+iy$ è unitario si ha $x^2+y^2=1$.
I prodotti e i
reciproci dei numeri complessi unitari sono anch'essi unitari,
risulta quindi che tali numeri formano un \emph{sottogruppo moltiplicativo}%
\footnote{
Un \emph{gruppo} è un insieme su cui è definita una operazione
(spesso denotata con il simbolo della moltiplicazione) che sia associativa,
che abbia elemento neutro e tale che ogni elemento abbia un inverso.
}
del gruppo dei numeri complessi.


I numeri complessi unitari possono essere utilizzati per rappresentare gli
angoli geometrici. Se $\theta$ è unitario, cioè $\abs{\theta}=1$, possiamo
pensare che $\theta$ rappresenti l'angolo con vertice nell'origine, delimitato
dall'asse dei reali positivi e dalla semiretta
uscente da $0$ e passante per $\theta$.

Ad esempio i numeri complessi unitari
$1, i, -1$ rappresentano, rispettivamente, gli angoli:
nullo, retto e piatto.

Ogni numero complesso $z$ potrà essere scritto nella forma
\[
  z = \rho \theta
\]
con $\rho>0$ (sottointeso $\rho \in \RR$ visto che il confronto $<$ non
ha senso sui complessi) e $\theta$ unitario.
Basta infatti
definire $\rho = \abs{z}$ e $\theta = z / \abs{z}$ (se $z\neq 0$, altrimenti
si potrà scegliere arbitrariamente $\theta=0$).
L'angolo $\alpha$ corrispondente al numero complesso unitario $\theta$
viene chiamato \myemph{argomento}
del numero complesso $z$ e si denota a volte con $\alpha = \arg z = \arg \theta$.
Se $\theta = x+ iy$ è unitario e $\alpha = \arg \theta$ allora le funzioni
che associano all'angolo geometrico $\alpha$ le due coordinate $(x,y)$
di $z$ si chiamano funzioni trigonometriche
\myemph{coseno} e \myemph{seno}: $x= \cos \alpha$,  $y=\sin \alpha$.

La rappresentazione di un numero complesso tramite modulo e argomento
si chiama \myemph{rappresentazione polare}:
\[
    z = \rho (\cos \alpha + i \sin \alpha).
\]
% Si contrappone alla
% \myemph{rappresentazione cartesiana} $z=x+iy$ in cui
% vengono evidenziate le coordinate cartesiane del punto $z$
% rispetto ai due assi reale e immaginario.

Fissato $\theta = \cos \alpha + i \sin \alpha$ unitario
consideriamo la funzione $R_\theta\colon \CC \to \CC$,
$R_\theta(z) = \theta\cdot z$.
Dalla proprietà distributiva della moltiplicazione complessa risulta
immediato che tale funzione è lineare (stiamo pensando a $\CC$ come
spazio vettoriale bidimensionale su $\RR$).
Osserviamo che $R_\theta(1) = \theta$ cioe $R_\theta$ agisce sul punto $1$
ruotandolo di un angolo $\alpha=\arg \theta$ sul piano complesso.
Se $\theta=x+ i y$ allora $R_\theta(i) = \theta\cdot i = y - ix$ e si osserva
che il punto di coordinate $(-x, y)$ di nuovo
non è altro che il ruotato di un angolo
$\alpha$ del punto $i$.
La matrice $M_\alpha$ associata alla applicazione lineare $R_\theta$ ha come
colonne le coordinate di $R_\theta(1) = \theta = \cos \alpha + i \sin \alpha$
e le coordinate di $R_\theta(i) = i\theta = -\sin \alpha + i \cos \alpha$:
\[
  M_\alpha =
  \begin{pmatrix}
  \cos \alpha & -\sin \alpha \\
  \sin \alpha & \cos \alpha
  \end{pmatrix}.
\]

Visto che la rotazione è una funzione lineare e coincide con $R_\theta$
su una base dello spazio $\CC$ scopriamo che $R_\theta$ deve coincidere con
la rotazione di un angolo $\theta$ su tutti i punti di $\CC$.

In particolare se $\theta$ e $\psi$ sono unitari, $\alpha= \arg \theta$ e
$\beta=\arg \psi$ il loro prodotto
$\theta\psi$ corrisponde alla rotazione di $\psi$ di un angolo $\alpha$
ovvero ad una rotazione pari alla somma degli angoli $\alpha+\beta$.
Dunque la moltiplicazione complessa sui numeri unitari rappresenta la somma
degli angoli corrispondenti: $\arg(\theta\psi) = \arg(\theta) + \arg(\psi)$.
Si può usare questa interpretazione
per dare significato geometrico alle identità algebriche: $i^2=-1$, $(-1)^2 = 1$,
$(-i)^2 = -1$ tramite composizione di rotazioni.

Possiamo allora più in generale intepretare il prodotto di due numeri complessi
$z\cdot w$. Se $z\neq 0$ possiamo scrivere $z = \abs{z} \cdot \theta$ con
$\theta= z/\abs{z}$ unitario, cosicché:
\[
  z \cdot w = \abs{z} \cdot R_\theta(w).
\]
Si capisce quindi che il numero complesso $z\cdot w$ si ottiene ruotando
$w$ dell'angolo identificato da $z$ con l'asse dei reali positivi, e quindi
riscalando il punto ottenuto di un fattore $\abs{z}$.
Dunque $\arg(z\cdot w) = \arg z + \arg w$ e $\abs{z\cdot w} = \abs{z}\cdot\abs{w}$:
il prodotto $z\cdot w$ dei numeri complessi $z$ e $w$ si ottiene moltiplicando
i moduli e sommando gli argomenti.

In particolare moltiplicando un numero complesso unitario $\cos \alpha + i \sin \alpha$ per un altro numero complesso unitario
$\cos \beta + i \sin \beta$ si deve ottenere il numero complesso unitario
 $\cos(\alpha+\beta) + i \sin(\alpha + \beta)$.
 Dunque
 \begin{gather*}
 \cos(\alpha+\beta) + i \sin(\alpha + \beta)
 = (\cos \alpha + i \sin \alpha) \cdot (\cos \beta + i \sin \beta)\\
 = \cos \alpha \cos \beta - \sin \alpha \sin \beta
  + i (\sin \alpha \cos \beta + \cos \alpha \sin \beta)
 \end{gather*}
 da cui seguono le \myemph{formule di addizione}:
 \begin{align*}
  \cos(\alpha + \beta) &= \cos\alpha \cos \beta - \sin \alpha \sin \beta\\
  \sin(\alpha + \beta) &= \sin \alpha \cos \beta + \cos \alpha \sin \beta
 \end{align*}
 e, nel caso particolare $\alpha=\beta$
 le \myemph{formule di duplicazione}:
\begin{align*}
  \cos(2\alpha) &= \cos^2 \alpha - \sin^2 \alpha \\
  \sin(2\alpha) &= 2 \sin \alpha \cos \alpha.
\end{align*}

\section{successioni di numeri complessi}

Se abbiamo una successione $z_n$ di numeri complessi
(ovvero una funzione $\NN \to \CC$) e un numero $z\in \CC$
diremo che $z_n$ converge
\mymargin{convergenza}
a $z$ e scriveremo $z_n \to z$ se la successione di numeri reali $\abs{z_n-z}$
è infinitesima ovvero $\abs{z_n - z} \to 0$.

In coordinate cartesiane, se $z_n = x_n + i y_n$ e $z = x + iy$ si verifica
facilmente che $\abs{z_n -z} = \sqrt{(x_n -x)^2 + (y_n-y)^2}$ tende a zero
se e solo se $x_n \to x$ e $y_n \to y$.
Dunque $z_n \to z$ se e solo se $\Re z_n \to \Re z$ e $\Im z_n \to \Im z$.

Anche lo spazio dei numeri complessi può essere esteso aggiungendoci
un punto all'\myemph{infinito}.
A differenza dei reali, su cui era presente un ordinamento che era utile conservare,
nel caso dei numeri complessi è più usuale utilizzare un unico punto infinito
che si denota con \myemph{$\infty$}.
Definiamo lo spazio dei complessi estesi $\bar \CC$ come
\[
\bar \CC = \CC \cup \{\infty\}.
\]
Definiamo $\abs{\infty} = +\infty \in \bar \RR$. Inoltre
definiamo
\begin{align*}
   z + \infty &= \infty \qquad \forall z \in \CC\\
   z\cdot \infty &= \infty \qquad \forall z \in \bar\CC\setminus\{0\} \\
   -\infty &= \infty \\
   1 / \infty &= 0.
\end{align*}

Diremo che una successione $z_n$ di numeri complessi diverge
\mymargin{divergenza}
e scriveremo
$z_n \to \infty$ se $\abs{z_n} \to +\infty$.

Se $z_n \to z$ con $z \in \bar \CC$ diremo che $z_n$ ha \myemph{limite} $z$ e
scriveremo
\[
  \lim_{n\to +\infty} z_n = z.
\]

Il limite in $\bar \CC$ può essere definito con il linguaggio
della  \myemph{topologia}
mediante l'introduzione degli intorni dei punti di $\bar \CC$.
Gli intorni dei punti $z\in \CC$ sono le \emph{palle} di raggio $r>0$ centrate
nel punto:
\begin{align*}
  B_r(z) &= \{w\in \CC \colon \abs{w-z} < r\} \\
  \U_z &= \{B_r(z) \colon r>0\}
\end{align*}
Gli intorni di $\infty$ sono i complementari delle palle centrate in $0$:
\[
  \U_\infty = \{\CC \setminus B_r(0)\colon r>0\}.
\]
In questo modo la definizione di limite rimane sempre la stessa.
Se $z_n\in \CC$ e $z\in \bar C$ si ha
\[
 \text{$z_n\to z$ per $n\to+\infty$}
\]
se e solo se
\[
 \forall U\in \U_z\colon \exists V \in \U_{+\infty}\colon n\in V \implies z_n \in U.
\]

\section{esponenziale complesso}

\begin{definition}[esponenziale complesso]
\mymargin{esponenziale complesso}
Per ogni $z \in \CC$ definiamo
\mymargin{$\exp(z)$}
\[
  \exp(z) = \lim_{n\to +\infty}\enclose{1+\frac z n}^n.
\]
Si userà anche la notazione $e^z = \exp(z)$.
\end{definition}

Che il limite nella definizione precedente esista non è affatto ovvio:
lo dimostreremo (si spera!) nel prossimo capitolo. Per ora supponiamo
che tale limite esista e supponiamo anche che l'esponenziale complesso
soddisfi l'usuale proprietà:
\[
  \exp(z+w) = \exp z \cdot \exp w.
\]

Dunque se $z=x+iy$ con $x,y \in \RR$, dovrà essere $e^z = e^x \cdot e^{iy}$.
Il fattore $e^x$ è l'usuale esponenziale reale, vogliamo ora dare
un significato geometrico all'esponenziale della parte immaginaria
$e^{iy} = \lim (1+iy/n)^n$.

Possiamo osservare che se $y\in \RR$
i punti $(1+iy/n)^k$ per $k=1\dots n$
sono i vertici di una spezzata
formata da $n$ segmenti
di lunghezza
\begin{align*}
 \abs{\enclose{1+\frac{iy}n}^{k+1}\!\!\! - \enclose{1+ \frac{iy}n}^k}
 &= \abs{\enclose{1+\frac{iy}n}^k\cdot \enclose{1+\frac{iy}n -1}}\\
 &= \enclose{\sqrt{1+\frac{y^2}{n^2}}}^{\!\!k} \cdot \frac{\abs y}{n}
 \le \enclose{1+\frac{y^2}{n^2}}^{\frac n 2}\cdot \frac{\abs{y}}{n}.
\end{align*}
In particolare la lunghezza totale della spezzata $\ell_n$ può essere stimata
come segue
\[
  \abs{y}
  \le \ell_n
  \le \enclose{1+\frac{y^2}{n^2}}^{\frac n 2} \cdot \abs{y}
\]
da cui osservando che
\[
 \enclose{1+\frac{y^2}{n^2}}^{\frac n 2} \to 1
\]
e utilizzando il criterio del confronto
si ottiene $\ell_n \to \abs{y}$.

Si osserva anche che i punti di tale spezzata si avvicinano
sempre di più alla circonferenza unitaria, infatti:
\[
  1
  \le \abs{\enclose{1 + \frac i n}^k}
  \le \abs{1+\frac i n}^n
  = \enclose{1 + \frac 1 {n^2}}^{\frac n 2}
  \to 1.
\]

E' dunque sensato pensare che il punto $e^{iy}$ sia il punto
della circonferenza unitaria che identifica un arco di lunghezza $\abs y$
a partire dal punto $1$ sull'asse reale.
Se $y>0$ l'arco è misurato in senso antiorario, altrimenti in senso orario.
Avremo dunque $\arg(\exp(iy)) = y$ essendo $y$ la lunghezza dell'arco
ovvero la misura in radianti dell'angolo corrispondente.
Supponendo d'ora in poi che le funzioni $\cos$ e $\sin$
abbiano come argomento la misura in radianti dell'angolo
si avrà dunque la \myemph{formula di Eulero}:
\[
  e^{iy} = \cos y + i \sin y.
\]
Potremo quindi definire la lunghezza
della semicirconferenza unitaria, come il più piccolo numero reale positivo
\myemph{$\pi$} tale che
\[
  e^{i \pi} = -1
\]
Ovvero il primo zero positivo della funzione $y \mapsto \sin y$
(vedremo più avanti che tale funzione è continua e cambia segno).

Osserviamo ora che si ha $e^{2i\pi} = (e^{i\pi})^2 = (-1)^2 = 1$ e quindi
\[
  e^{i{y+2k\pi}} = e^{iy}\cdot \enclose{e^{2i\pi}}^k = e^{iy}
\]
da cui in particolare
\[
  \cos(y+2 k\pi) = \cos y,
  \qquad
  \sin(y+2 k \pi) = \sin y.
\]
Ovvero le funzioni $e^{iy}$, $\cos y$ e $\sin y$ sono
\myemph{funzioni periodiche} di \myemph{periodo} $2\pi$.

Possiamo utilizzare l'esponenziale complesso per esprimere in forma
più compatta la rappresentazione polare dei numeri complessi.
Infatti se $\rho = \abs{z}$ e $\alpha = \arg z$ si avrà
\[
  z = \rho (\cos \alpha + i \sin \alpha) = \rho e^{i\alpha}.
\]


\section{radici complesse $n$-esime}

Sia $c\in \CC$ un numero
complesso $c\neq 0$.
Ci poniamo il problema di determinare le soluzioni complesse
dell'equazione
\[
  z^n = c.
\]
Tali soluzioni saranno chiamate \myemph{radici $n$-esime} di $c$.

Scriviamo $c$ e $z$ in forma esponenziale:
\[
  c = r e^{i\alpha}, \qquad
  z = \rho e^{i\theta}.
\]
Si avrà allora
\[
  z^n = \rho^n (e^{i\theta})^n = \rho^n e^{i n \theta}.
\]
Affinche sia $z^n = c$ si dovrà avere l'uguaglianza dei moduli, cioè $\rho^n = r$ e l'uguaglianza a meno di multipli interi di $2\pi$ degli argomenti:
$n \theta = \alpha + 2 k \pi$ con $k\in \ZZ$.
Dunque si trova
\[
  \theta = \frac{\alpha}{n} + k\frac{2\pi}{n}
\qquad k \in \ZZ.
\]
Osserviamo ora che per $k=0,\dots, n-1$ il secondo addendo
$k 2\pi /n$ assume $n$ valori distinti compresi in $[0,2\pi)$.
Per gli altri valori di $k$ si ottengono degli angoli che differiscono
da questi di un multiplo di $2\pi$ e quindi non si trovano
altre soluzioni.

Dunque l'equazione $z^n = c$ per $c\neq 0$ ha $n$ soluzioni distinte date
da
\[
z_k = \sqrt[n]{r} \cdot \exp(i\alpha/n + 2k\pi i /n),
\qquad k=0,1, \dots, n-1
\]
dove $\alpha = \arg(c)$ e $r = \abs{c}$.
Dal punto di vista geometrico si osserva che
$z_0$ è il numero complesso con modulo la radice $n$-esima del numero
dato $c$ e argomento pari ad un $n$-esimo dell'argomento di $c$.
Tutte le altre soluzioni si trovano sulla circonferenza centrata in $0$
e passante per $z_0$ e risultano essere, insieme ad $z_0$, i vertici
di un $n$-agono regolare.

In particolare nel caso $c=1$ si osserva che le radici $n$-esime dell'unità
si rappresentano geometricamente come i vertici dell'$n$-agono regolare iscritto
nella circonferenza unitaria e con un vertice in $z_0=1$.

\begin{exercise}
Si trovino le soluzioni $z \in \CC$ delle seguenti equazioni.
Scrivere le soluzioni in forma polare e cartesiana.
\begin{gather*}
   z^4 = -4 \\
   z^6 = i\\
   z^3 = -8i \\
   z^4 = z\\
   z^2 + 1 = i\sqrt{3} \\
   (z-i)^4 = 1\\
   1 + z + z^2 + z^3 = 0\\
   z^{14} - z^6 - z^8 + 1 = 0
\end{gather*}
\end{exercise}







%%%%%%%%%%%%%%%%%%%
%%%%%%%%%%%%%%%%%%%
%%%%%%%%%%%%%%%%%%%
%%%%%%%%%%%%%%%%%%%
\chapter{serie}

Data una successione $a_n$ di numeri reali o complessi
possiamo considerare la successione
delle cosiddette \myemph{somme parziali}
\[
  S_n = \sum_{k=0}^{n} a_k.
\]
Potremo scrivere più concisamente $S_n = \sum a_n$.
I numeri $a_n$ si chiamano \myemph{termini della serie}
La serie si dirà convergente, divergente o indeterminata se la
successione delle somme parziale è convergente, divergente o indeterminata.
Se la successione delle somme parziali ammette limite il limite viene chiamato
\myemph{somma della serie} e si indica con
\[
  \sum_{k=0}^{+\infty} a_k = \lim_{n\to +\infty} S_n = \lim_{n\to+\infty} \sum_{k=0}^n a_k.
\]

La terminologia già introdotta per le successioni si applica anche alle
serie che sono in effetti anch'esse delle successioni.
In particolare una serie può essere convergente, divergente o indeterminata.
Questo si chiama il \myemph{carattere della serie}.

\begin{example}[la serie geometrica]
Fissato $q \in \RR$ alla successione (cosiddetta geometrica)
\[
  a_n = q^n
\]
di termini
\[
  a_0 = 1,\qquad
  a_1 = q,\qquad
  a_2 = q^2,\qquad
  a_3 = q^3, \dots
\]
è associata la \myemph{serie geometrica}
\[
  S_n = \sum_{k=0}^{n} q^n
\]
le cui somme parziali sono
\begin{align*}
  S_0 &= 1, \\
  S_1 &= 1 + q, \\
  S_2 &= 1 + q + q^2, \\
  &\quad\vdots
\end{align*}
\end{example}

Il seguente teorema ci mostra come per diversi valori di $q$
la serie geometrica assume
tutti i caratteri: convergente, divergente, indeterminato.

\begin{theorem}[somma della serie geometrica]
\mymargin{somma della serie geometrica}
Sia $q\in \RR$. Se $q\neq 1$ si ha
\[
 \sum_{k=0}^n q^k  = \frac{1-q^{n+1}\!\!\!\!\!\!}{1-q}.
\]
Se $\abs{q} < 1$ la serie geometrica converge:
\[
 \sum_{k=0}^{+\infty} q^n = \frac{1}{1-q}
\]
se $q\ge 1$ diverge:
\[
 \sum_{k=0}^{+\infty} q^n = +\infty
\]
e se $q\le -1$ la serie geometrica è indeterminata.
\end{theorem}
%
\begin{proof}
Il primo risultato riguarda una somma finita.
Si ha
\[
  (1-q)\cdot \sum_{k=0}^n q^k
  = \sum_{k=0}^n q^k - q \cdot \sum_{k=0}^n q^k
  = \sum_{k=0}^n q^k - \sum_{k=1}^{n+1} q^k
  = 1 - q^{n+1}
\]
da cui si ottiene, se $q\neq 1$, il primo risultato.

Passando al limite per $n\to +\infty$, se $\abs{q} < 1 $
si nota che $q^{n+1} \to 0$ e la serie converge a $1/(1-q)$
mentre se $q>1$ osserviamo che
$q^n\to +\infty$ e quindi la serie diverge a $+\infty$ (infatti in questo
caso $1-q$ è negativo).
Se $q=1$ si ha $q^k=1$ e quindi $\sum_{k=0}^n q^k = n+1 \to +\infty$.

Se $q<0$ si ha $q= -\abs{q}$ da cui
\[
 \sum_{k=0}^n q^k
 = \frac{1-q^{n+1}\!\!\!\!\!}{1-q}
 = \frac{1-(-1)^{n+1}\abs{q}^{n+1}}{1+\abs{q}}.
 \]
 Se $q=-1$ si ha
 \[
   \sum_{k=0}^n q^k = \sum_{k=0}^n (-1)^k
   =
     \begin{cases}
      1 & \text{se $n$ è pari}\\
      0 & \text{se $n$ è dispari}
     \end{cases}
 \]
 e quindi la serie è indeterminata.
Se $q<-1$ si osserva che sui termini dispari
si ha $(-1)^{n+1}\abs{q}^{n+1}\to +\infty$ mentre sui
termini pari tale quantità tende a $-\infty$.
Lo stesso vale per le somme parziali della serie che quindi
è, anche in questo caso, indeterminata.
\end{proof}

\begin{theorem}[linearità della somma]
\mymargin{linearità della somma infinita}
Se $\sum a_n$ e $\sum b_n$ sono convergenti
allora per ogni $\lambda,\mu\in \CC$
anche $\sum (\lambda a_n + \mu b_n)$ è convergente
e si ha
\[
 \sum_{k=0}^{+\infty} (\lambda a_n + \mu b_n)
 = \lambda \sum_{k=0}^{+\infty} a_n  + \mu \sum_{k=0}^{+\infty} b_n.
\]
\end{theorem}
%
\begin{proof}
Se $S_n$ e $R_n$ sono le somme parziali delle serie $\sum a_n$ e $\sum b_n$
allora le somme parziali della serie $\sum (\lambda a_n + \mu b_n)$ sono
$\lambda S_n + \mu R_n$ (in quanto sulle somme finite vale la proprietà distributiva e commutativa). Ma se $S_n \to S$ e $R_n \to R$ allora
$\lambda S_n + \mu R_n \to \lambda S + \mu R$.
\end{proof}

Osserviamo che le serie (così come le successioni) formano uno spazio
vettoriale in cui le operazioni di somma e prodotto per scalare vengono
eseguite termine a termine: $\sum a_n + \sum b_n = \sum (a_n + b_n)$,
$\lambda \sum a_n = \sum (\lambda a_n)$.
Il teorema precedente ci dice allora che le serie (così come le successioni)
convergenti sono un sottospazio vettoriale e che la somma della serie (così come il limite della successione) è un'operatore lineare definito su tale sottospazio.

\begin{theorem}[condizione necessaria per la convergenza]
\mymargin{condizione necessaria per la convergenza}
Se la serie $\sum a_n$ converge allora $a_n \to 0$.
\end{theorem}
%
\begin{proof}
Se la serie $\sum a_n$ converge significa che le somme parziali
$S_n = \sum_{k=0}^n a_k$ convergono: $S_n \to S$. Ma allora
\[
  a_n = S_n - S_{n-1} \to S - S = 0.
\]
\end{proof}

\begin{theorem}[serie che differiscono su un numero finito di termini]
\mymargin{serie che differiscono su un numero finito di termini}
Se le due successioni $a_n$ e $b_n$ differiscono solo su un numero finito
di termini, allora le serie corrispondenti $\sum a_n$ e $\sum b_n$ hanno lo stesso carattere.
\end{theorem}
%
\begin{proof}
Se le successioni differiscono su un numero finito di termini significa
che esiste un $N\in \NN$ tale che per ogni $k>N$ si ha $a_k=b_k$.
Dunque se indichiamo con $S_n = \sum_{k=0}^n a_k$ e $R_n = \sum_{k=0}^n b_k$
le corrispondenti successioni delle somme parziali, si avrà per ogni $n>N$
\[
  S_n - R_n
    = \sum_{k=0}^n a_k - \sum_{k=0}^n b_k
    = \sum_{k=0}^N (a_k - b_k) = C
\]
dove $C$ è una costante indipendente da $n$. Dunque
\[
  S_n = R_n + C.
\]
Se il limite di $R_n$ non esiste allora non esiste neanche il limite
di $S_n$ (altrimenti essendo $R_n = S_n -C$ anche il limite di $R_n$ dovrebbe esistere). Se il limite di $R_n$ è infinito allora il limite di $S_n$ è uguale
al limite di $R_n$. E se il limite di $R_n$ è finito anche il limite di $S_n$ è finito.

Dunque il carattere della successione $S_n$ è lo stesso della successione $R_n$
cioè le due serie hanno lo stesso carattere.
\end{proof}

Come per le successioni potremo considerare serie il cui primo termine ha un indice diverso da $0$. Ci si potrà sempre ricondurre (con un cambio di variabile)
ad una serie il cui indice parte da zero. Ad esempio
(facendo il cambio di variabile $j=k-1$ da cui $j=0$ quando $k=1$
e ricordando che l'indice utilizzato nelle somme delle
serie è una variabile muta):
\[
 \sum_{k=1}^{+\infty} \frac{1}{2^k}
 = \sum_{j=0}^{+\infty} \frac{1}{2^{j+1}}
 = \sum_{k=0}^{+\infty} \frac{1}{2^{k+1}}.
\]

Si osservi inoltre che in base al teorema precedente quale sia il primo indice
da cui si comincia a sommare non è rilevante per quanto riguarda il carattere della serie.
Se però la serie è convergente la sua somma può variare, ad esempio:
\[
 \sum_{k=1}^{+\infty} \frac{1}{2^k} = \enclose{\sum_{k=0}^{+\infty} \frac{1}{2^k}} - 2^0.
\]

Nota bene: in molti libri si scrive $\infty$ al posto di $+\infty$.
Risulta quindi molto comune omettere il segno $+$ davanti a $\infty$
nella terminologia delle serie (e anche delle successioni) visto
che gli indici si intendono numeri naturali e quindi $-\infty$ non avrebbe
senso.

Ci sono però casi in cui può essere utile usare anche gli indici negativi,
ad esempio si potrebbe definire (ma non lo faremo):
\[
  \sum_{k=-\infty}^{+\infty} a_k
  = \sum_{k=0}^{+\infty} a_k +
  \sum_{k=-1}^{-\infty} a_k
  = \sum_{k=0}^{+\infty} a_k +
  \sum_{k=1}^{+\infty} a_{-k}
\]
quando $a_k$ è definita per ogni $k\in \ZZ$ e le due serie al lato destro
dell'uguaglianza esistono entrambe e non hanno somme infinite di segno opposto.

\section{serie telescopiche}

Una serie scritta nella forma
% $\Sigma \Delta \vec a$ cioè
% del tipo:
\[
  \sum (a_{k} - a_{k+1})
\]
viene detta \emph{telescopica}\mymargin{serie telescopica}
in quanto i singoli termini della somma, come i tubi di un cannocchiale,
si semplificano uno con l'altro permettendo al cannocchiale di chiudersi:
\[
  \sum_{k=0}^n (a_{k} - a_{k+1})
  = \sum_{k=0}^{n} a_k - \sum_{k=1}^{n+1} a_k
  = a_0 - a_{n+1}.
\]

Ad esempio per calcolare la somma della serie geometrica abbiamo
sfruttato questa relazione
\[
  \frac{1-q^{n+1}}{1-q} - \frac{1-q^n}{1-q}
  = \frac{q^n - q^{n+1}}{1-q} = q^n \cdot \frac{1-q}{1-q} = q^n
\]

In generale data una qualunque successione $a_n$ possiamo dunque trovare una
serie di cui $a_n$ risultano essere le somme parziali.
Nel seguente esempio si considera la successione $a_n = 1/(n+1)$.

\begin{example}[serie di Mengoli]
Si ha
\[
  \sum_{n=1}^{+\infty} \frac{1}{n(n+1)} = 1.
\]
\end{example}
%
\begin{proof}
Infatti
\[
  \sum_{k=1}^n \frac{1}{k(k+1)}
  = \sum_{k=1}^n \enclose{\frac{1}{k} - \frac{1}{k+1}}
  = \sum_{k=1}^n \frac{1}{k} - \sum_{k=2}^{n+1} \frac{1}{k}
  = 1 - \frac{1}{n+2} \to 1.
\]
\end{proof}

\section{serie a termini positivi}

\index{serie a termini positivi}
Nel seguito considereremo serie i cui termini sono numeri reali
positivi (o almeno non negativi).
Quando scriveremo $a_n >0$ (o $a_n \ge 0$) sarà sempre
sottointeso che $a_n\in \RR$ visto che per i numeri complessi non
reali non abbiamo definito la relazione d'ordine.

\begin{theorem}[carattere delle serie a termini positivi]
\mymargin{carattere delle serie a termini positivi}
Se $a_n\ge 0$
la serie $\sum a_n$ non può essere indeterminata:
o converge oppure diverge a $+\infty$.
\end{theorem}
%
\begin{proof}
Se $a_n \ge 0$ essendo $a_n = S_n - S_{n-1}$ significa che
la successione $S_n$ delle somme parziali è crescente.
Dunque il limite delle $S_n$ esiste ed è in $(-\infty, +\infty]$.
\end{proof}

\begin{theorem}[criterio del confronto]
\mymargin{criterio del confronto}
Siano $\sum a_n$ e $\sum b_n$ serie a
termini positivi che si confrontano: $0\le a_n\le b_n$.
Allora
\[
  \sum_{k=0}^\infty a_n \le \sum_{k=0}^\infty b_n.
\]
In particolare se $\sum b_n$ converge anche $\sum a_n$ converge
e se $\sum a_n$ diverge anche $\sum b_n$ diverge.
\end{theorem}
%
\begin{proof}
Se $S_n$ sono le somme parziali di $\sum a_n$ e $R_n$ sono le somme
parziali di $\sum b_n$ si ha $S_n \le R_n$ e il risultato
si riconduce al confronto tra successioni.
\end{proof}

\begin{example}
La serie
\[
 \sum \frac{1}{k^2}
\]
è convergente.
Infatti osservando che si ha per ogni $n>0$
\[
  \frac{1}{(n+1)^2} \le \frac{1}{n(n+1)}
\]
possiamo affermare che
\[
  \sum_{k=1}^\infty \frac{1}{k^2}
  = 1 + \sum_{k=1}^\infty \frac{1}{(k+1)^2}
  \le 1+ \sum_{k=1}^\infty \frac{1}{k(k+1)}
  = 2
\]
in quanto ci siamo ricondotti alla
serie telescopica di Mengoli che ha somma pari a $1$.
\end{example}

\begin{definition}[equivalenza asintotica]
\index{equivalenza asintotica}
Due successioni a termini positivi $a_n$ e $b_n$ si dicono essere
\myemph{asintoticamente equivalenti} e si scrive $a_n \sim b_n$
se, per $n\to +\infty$
\[
  \frac{a_n}{b_n} \to 1.
\]
\end{definition}

\begin{corollary}[criterio del confronto asintotico]
\mymargin{criterio del confronto asintotico}
Se $a_n$ e $b_n$ sono successioni positive
asintoticamente equivalenti
allora le serie corrispondenti $\sum a_n$ e $\sum b_n$
hanno lo stesso carattere.
\end{corollary}
%
\begin{proof}
Le serie a termini positivi non possono essere indeterminate
quindi è sufficiente verificare che se una serie converge, converge anche l'altra.
Essendo $a_n / b_n$ convergente tale rapporto deve anche essere
limitato, quindi esiste $C\in \RR$ tale che
\[
   a_n \le C \cdot b_n.
\]
Se la serie $\sum b_n$ converge anche $\sum C \cdot b_n$ converge e, per confronto,
converge anche $\sum a_n$.

Viceversa, scambiando il ruolo di $a_n$ e $b_n$ si verifica che se $a_n$
converge, converge anche $b_n$.
\end{proof}

\begin{example}
La serie
\[
\sum_n \frac{n^2+2n+3}{2n^4-n^3+n+1}
\]
è convergente. Infatti si può facilmente verificare che
\[
   \frac{n^2+2n+3}{2n^4-n^3+n+1} \sim \frac{1}{2n^2}.
\]
Ma sappiamo che la serie $\sum 1/n^2$ è convergente, di conseguenza
anche la serie $\sum 1/(2n^2)$ lo è (per linearità della somma)
e quindi, per confronto
asintotico, anche la serie data è convergente.
\end{example}

\begin{theorem}[criterio della radice]
\mymargin{criterio della radice}
Sia $\sum a_n$ una serie a termini non negativi
(cioè $a_n\ge 0$) tale che
$\sqrt[n]{a_n} \to \ell \in [0,+\infty]$.
Se $\ell<1$ allora la serie converge.
Se $\ell>1$ allora la serie diverge.
\end{theorem}
%
\begin{proof}
Nel caso $\ell < 1$
prendiamo $q$ con $\ell < q < 1$ e poniamo $\eps = q-\ell$.
Per la definizione di limite $\sqrt[n]{a_n}\to \ell$ sappiamo
esistere $N$ tale che per ogni $n > N$ si abbia
\[
  \sqrt[n]{a_n} < \ell + \eps = q
\]
cioè
\[
   a_n < q^n.
\]
Sapendo che $\sum q^n$ converge, sapendo anche che il carattere
della serie non cambia modificando un numero finito di termini,
per confronto possiamo concludere che anche la serie $\sum a_n$ converge.

Il caso $\ell > 1$ si tratta in maniera analoga.
\end{proof}

\begin{example}
La serie
\[
  \sum_k 2^{(\ln k) - k}
\]
è convergente. Infatti si ha
\[
 \sqrt[k]{2^{\ln k - k}}
 = 2^{\frac{\ln k - k}{k}}
 = 2^{\frac{\ln k }k - 1}
 \to 2^{-1}
 = \frac{1}{2}
 < 1.
\]
\end{example}

\begin{theorem}[criterio del rapporto]
\mymargin{criterio del rapporto}
Sia $\sum a_n$ una serie a termini positivi
tale che $a_{n+1} / a_n \to \ell \in [0,+\infty]$.
Se $\ell <1$ allora la serie converge.
Se $\ell > 1$ la serie diverge.
\end{theorem}
%
\begin{proof}
Per il criterio del rapporto alla Cesàro si ha $\sqrt[n]{a_n} \to \ell$
quindi ci riconduciamo al criterio della radice.
\end{proof}

\begin{example}
Per ogni $x\ge 0$ la serie
\[
  \sum \frac{x^n}{n!}
\]
converge.
\end{example}
%
\begin{proof}
Applichiamo il criterio del rapporto. Posto $a_n = x^n / n!$ si ha
\[
\frac{a_{n+1}}{a_n}
= \frac{x^{n+1}}{(n+1)!}\cdot \frac{n!}{x^n}
= \frac{x}{n+1} \to 0 < 1.
\]
Dunque la serie converge.
\end{proof}

Osserviamo invece che il criterio del rapporto non si applica alla
\myemph{serie armonica}
\[
  \sum_k \frac{1}{k}
\]
in quanto
\[
 \frac{\frac{1}{k+1}}{\frac{1}{k}}
 = \frac{k}{k+1} \to 1.
\]

Per capire se la serie armonica converge o diverge presentiamo il metodo
di \emph{condensazione} che verrà enunciato in generale nel prossimo teorema
ma che può essere meglio compreso se applicato al caso particolare
della serie armonica.

Mostreremo che la serie armonica diverge.
L'idea è semplicemente quella di raggruppare gli addendi della serie armonica
in gruppi di lunghezza potenze di due e stimare ogni gruppo dal basso
con il termine più piccolo (cioè l'ultimo) di ogni gruppo:
\begin{align*}
 \sum_{k=1}^\infty \frac{1}{k}
 & = 1 + \frac 1 2
     + \enclose{\frac 1 3 + \frac 1 4}
     + \enclose{\frac 1 5 + \frac 1 6 + \frac 1 7 + \frac 1 8}
     + \dots\\
 & > 1 + \frac 1 2 + 2 \cdot \frac 1 4 + 4 \cdot \frac 1 8 + \dots \\
   & = 1 + \frac 1 2 + \frac 1 2 + \frac 1 2 + \dots
    = +\infty.
\end{align*}

\begin{theorem}[criterio di condensazione di Cauchy]
\mymargin{criterio di condensazione di Cauchy}
Sia $a_n$ una successione decrescente di numeri reali non negativi:
$a_n \ge 0$.
Allora la serie $\sum a_k$ converge se e solo se converge
la serie
\[
  \sum 2^k a_{2^k}.
\]
\end{theorem}
%
\begin{proof}
Supponiamo per comodità che le somme partano da $k=1$.
Si tratta di raggruppare i termini $a_k$ in gruppi di potenze di due:
\begin{align*}
  &a_1, \\
  &a_2,\ a_3, \\
  &a_4,\ a_5,\ a_6,\ a_7, \\
  &a_8,\ a_9,\ a_{10},\ \dots,\ a_{15}, \\
  &\vdots\\
  &a_{2^n},\ a_{2^n+1},\ \dots,a_{2^{n+1}-1},\\
  &\vdots
\end{align*}
Posto $S_n = \sum_{k=1}^n a_k$,
sommando i termini delle prime $N$ righe si osserva quindi che:
\[
  S_{2^N-1}
  = \sum_{k=1}^{2^N-1} a_k
  = \sum_{n=0}^{N-1}\,\, \sum_{j=0}^{2^n-1} a_{2^k+j}
\]
Visto che la successione $a_k$ è decrescente i termini di ogni gruppo si
possono stimare dall'alto e dal basso con il primo e l'ultimo termine:
\[
  a_{2^n} \ge a_{2^n+1} \ge \dots \ge a_{2^{n+1}-1} \ge a_{2^{n+1}}
\]
e quindi
\[
\sum_{n=0}^{N-1} 2^n a_{2^{n+1}}
\le S_{2^N-1}
\le \sum_{n=0}^{N-1} 2^n a_{2^n}.
\]

Dunque se la serie $\sum 2^n a_{2^n}$ converge allora la sottosuccessione
di somme parziali $S_{2^N-1}$ è superiormente limitata: $S_{2^N-1}\le C$.
Essendo $a_k \ge 0$ la successione $S_k$ è crescente.
Ma allora l'intera
successione $S_k$ è limitata
perché per ogni $k\in \NN$ esiste $N\in \NN$ tale che $k\le 2^N-1$ per cui
$S_k \le S_{2^N-1} \le C$.
Visto che $S_k$ è crescente $S_k$ ha limite e visto che abbiamo
appena verificato che $S_k$ è limitata allora il limite è finito
e la serie è convergente.

Viceversa se la serie $\sum a_k$ converge significa che
$S_k$ converge e dunque anche la sottosuccessione
$S_{2^N}$ converge e di conseguenza esiste
$C\in \RR$ tale che per ogni $N \in \NN$:
 $S_{2^N-1}\le C$.
Ma allora
\[
  \sum_{n=0}^{N-1} 2^n a_{2^n}
  = \frac 1 2 \sum_{n=0}^{N-1} 2^{n+1} a_{2^n}
  \le \frac 1 2 S_{2^N-1}
  \le \frac 1 2 C.
\]
e dunque anche la serie $\sum 2^n a_{2^n}$ risulta essere limitata e
di conseguenza (essendo una serie a termini positivi) è convergente.
\end{proof}



\begin{corollary}[serie armonica generalizzata]
\mymargin{serie armonica generalizzata}
La serie
\[
 \sum_n \frac{1}{n^\alpha}
\]
converge se $\alpha>1$,
diverge se $0\le \alpha\le 1$.
\end{corollary}
%
\begin{proof}
Applichiamo il criterio di condensazione. Posto $a_n = 1/n^\alpha$ Si ha
\[
  \sum_n 2^n a_{2^n} = \sum_n 2^n \frac{1}{(2^n)^\alpha}
  = \sum_n 2^{n(1-\alpha)}
  = \sum_n \enclose{2^{(1-\alpha)}}^n
\]
che è una serie geometrica di ragione $q=2^{1-\alpha}$.
Se $\alpha>1$ allora $q<1$ e la serie armonica è convergente
se invece $\alpha \le $ allora $q\ge 1$ e la serie
armonica è divergente.
\end{proof}

\section{serie a termini complessi o di segno variabile}

Per le serie a termini positivi abbiamo molti criteri di convergenza
che invece, in generale, non si applicano alle serie di segno qualunque
o alle serie di numeri complessi.
La convergenza di queste ultime, però, può a volte ricondursi
facilmente
alla
convergenza delle serie a termini positivi, passando al modulo
ogni termine.

\begin{definition}[convergenza assoluta]
Diremo che una serie (a termini reali o complessi) $\sum a_n$
è \myemph{assolutamente convergente} se la serie $\sum \abs{a_n}$
è convergente.
\end{definition}

\begin{theorem}[convergenza assoluta]
Se una serie (reale o complessa)
è assolutamente convergente allora è convergente.
\end{theorem}
%
\begin{proof}
Supponiamo inizialmente che gli $a_n$ siano numeri reali.
Definiamo $a_n^+ = \max\{0, a_n\}$ e $a_n^- = -\min \{0, a_n\}$.
Cioè se $a_n\ge 0$ si ha $a_n^+ = a_n$ e $a_n^-=0$ se invece $a_n\le 0$
si ha $a_n^+ =0$ e $a_n^- = -a_n$.
Dunque $a_n^+\ge 0$, $a_n^-\ge 0$ e
\[
   a_n = a_n^+  - a_n^-
   \qquad\text{e}\qquad
   \abs{a_n} = a_n^+ + a_n^-.
\]
Allora se $\sum \abs{a_n}$ converge,
per confronto anche $\sum a_n^+$ e $\sum a_n^-$ convergono.
Dunque, per il teorema sulla somma dei limiti,
$\sum a_n = \sum a_n^+ - \sum a_n^-$
e quindi anche $\sum a_n$ converge.

Se abbiamo una successione di complessi $a_n = x_n + i y_n$
e se
$\sum \abs{a_n}$ converge allora, per confronto,
anche $\sum \abs{x_n}$ e $\sum\abs{y_n}$ convergono
(si osservi infatti che $\abs{x} \le \abs{x+iy}$ e $\abs{y}\le \abs{x+iy}$).
Dunque $\sum x_n$ e $\sum y_n$ convergono per quanto
già dimostrato sulle serie a termini reali.
Ma allora anche $\sum i y_n$ e $\sum a_n = \sum (x + iy_n)$ convergono.
\end{proof}

\section{la serie esponenziale}

Utilizzando il  criterio del rapporto
abbiamo già osservato che
per ogni $z \in \CC$
la serie
\[
\sum_{k=0}^\infty \frac{z^k}{k!}
\]
è assolutamente convergente.
Vogliamo mostrare che la somma di questa
serie coincide con la funzione esponenziale $e^z$
cioè
che vale il seguente.

\begin{theorem}[serie esponenziale]
Per ogni $z\in \CC$ si ha
\[
  \lim_{n \to +\infty} \enclose{1+\frac z n}^n = \sum_{k=0}^\infty \frac{z^k}{k!}.
\]
\end{theorem}
%
\begin{proof}
Utilizzando lo sviluppo del binomio osserviamo che si ha
\[
 \enclose{1+\frac z n}^n
 = \sum_{k=0}^n \binom{n}{k} \frac{z}{n^k}
 = \sum_{k=0}^n \frac{z}{k!} \cdot \frac{n!}{n^k\cdot (n-k)!}.
\]
Posto per ogni $n\ge k$
\begin{align*}
 c(n,k)
  &= \frac{n!}{n^k\cdot (n-k)!}
  = \frac{n \cdot (n-1) \cdot \ldots \cdot(n-k)}{n^k} \\
  &= \frac{n}{n}\cdot {\frac {n-1} n} \cdot \frac {n-2} {n} \cdot \ldots \cdot \frac{n-k}{n}
\end{align*}
osserviamo che $c(n,k)\le 1$ in quanto
prodotto di numeri minori o uguali ad $1$.
Inoltre, fissato $k$, si ha $c(n,k)  \to 1$ per $n\to +\infty$
in quanto ogni fattore tende
a $1$ per $n\to +\infty$ (si noti che a $k$ fissato il numero di fattori è
fissato ed è $k-1$).
Dunque da un lato abbiamo una stima facile
\[
  \enclose{1+\frac 1 n}^n
  = \sum_{k=0}^n \frac{c(n,k)}{k!}
  \le \sum_{k=0}^n \frac{1}{k!}
  < e.
\]
Per ottenere la stima dal lato opposto consideriamo un qualunque $\eps>0$.
Consideriamo $M\in \NN$ tale che
\[
\sum_{k=0}^M \frac 1 {k!} > e - \eps.
\]
Fissato $k$ visto che $c(n,k)\to 1$ esiste $N_k$ tale che per ogni $n>N_k$
si ha $c(n,k) > 1- \eps$. Prendiamo $\displaystyle N=\max_{k=0}^M N_k$
cosicchè per ogni $n>N$ e per ogni $k\le M$ si avrà $c(n,k)> 1-\eps$.
Allora, per ogni $n>N$, si avrà
\begin{align*}
\enclose{1+\frac 1 n}^n
&= \sum_{k=0}^n  \frac{c(n,k)}{k!}
\ge \sum_{k=0}^{M} \frac{c(n,k) }{k!}
\ge  (1-\eps) \sum_{k=0}^M \frac{1}{k!}\\
&\ge (1-\eps)(e-\eps)
= e - \eps (e-1) + \eps^2 \ge e - \eps(e-1).
\end{align*}
Visto che $\eps(e-1)$ può essere reso arbitrariamente piccolo
abbiamo dimostrato che $\enclose{1+\frac 1 n}^n \to e$.
\end{proof}


\begin{comment}
\section{operatori $\Sigma$ e $\Delta$}

Sullo spazio vettoriale $V = \RR^\NN$ delle successioni
reali (o più in generale $V=\CC^\NN$ delle successioni complesse)
possiamo definire due
operatori lineari $\Sigma\colon V \to V$ e $\Delta\colon V \to V$
che sono gli operatori
che mandano una successione nella serie corrispondente (cioè nella successione
delle somme parziali) e l'operatore inverso che manda la successione
delle somme parziali nella successione dei corrispondenti termini
della serie.

Formalmente, per ogni successione $\vec a \in V$
definiamo $\Sigma \vec a \in V$ e $\Delta \vec a \in V$
come segue:
\begin{align*}
  \enclose{\Sigma \vec a}_n &= \sum_{k=0}^n a_k\\
  \enclose{\Delta \vec a}_n &=
  \begin{cases}
    a_n - a_{n-1} & \text{se $n>0$,}\\
    a_0 & \text{se $n=0$.}
  \end{cases}
\end{align*}
Verifichiamo che $\Delta \Sigma \vec a = \vec a$.
Se $n>0$
\[ \enclose{\Delta \Sigma \vec a}_n
 = \enclose{\Sigma \vec a}_n
 - \enclose{\Sigma \vec a}_{n-1}
 = \sum_{k=0}^n a_k - \sum_{k=0}^{n-1} a_k
 = a_n
\]
e se $n=0$:
\[
\enclose{\Delta \Sigma \vec a}_0
= \enclose{\Sigma \vec a}_0
= \sum_{k=0}^0 a_k
= a_0.
\]

Viceversa si ha anche $\Sigma \Delta \vec a = \vec a$
infatti:
\begin{align*}
\enclose{\Sigma \Delta \vec a}_n
&= \sum_{k=0}^n \enclose{\Delta \vec a}_k
= a_0 + \sum_{k=1}^n (a_k - a_{k-1}) \\
&= a_0 + \sum_{k=1}^n a_k - \sum_{k=1}^n a_{k-1}
= \sum_{k=0}^n a_k - \sum_{k=0}^{n-1} a_k
= a_n.
\end{align*}

Introduciamo anche l'operatore di traslazione $\tau \colon V\to V$
\[
(\tau \vec a)_n = a_{n+1}.
\]

\begin{theorem}[differenza del prodotto]
Se $\vec a$ e $\vec b$ sono successioni allora
\[
  \Delta (\vec a \cdot \vec b) = a \cdot \Delta b + \Delta a\cdot \tau b.
\]
\end{theorem}
%
\begin{proof}
Infatti per $n>0$ si ha
\[
\enclose{\Delta (\vec a \cdot \vec b)}_n
= a_n\cdot b_n - a_{n-1}\cdot b_{n-1}
= a_n \cdot b_n - a_n \cdot b_{n-1} + a_n \cdot b_{n-1} - a_{n-1}\cdot b_{n-1}
= a_n \cdot (\Delta \vec b)_n + (\Delta \vec a \cdot)_n \cdot b_{n-1}
\]
\end{proof}

\end{comment}


\backmatter
\chapter{Appendici}


\section{Contributi}

Hanno segnalato errori e correzioni:
Alessandro Canzonieri,
Luca Casagrande,
Maria Antonella Secondo,
Bianca Turini,
Piero Viscone.
% \bibliographystyle{plain}
% \bibliography{biblio}

\printindex

\end{document}
