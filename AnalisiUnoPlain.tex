\documentclass[
	fontsize=10pt, % Base font size
]{scrbook}
\usepackage[italian]{babel} % Load characters and hyphenation
\usepackage[dvipsnames]{xcolor}
\usepackage{amsmath,amsthm,amssymb,thmtools}
\usepackage{mathtools} % MoveEqLeft
\usepackage{comment}
\usepackage{qrcode}
\usepackage[type={CC},modifier={by-nc-sa},version={4.0},lang={en}]{doclicense}
\usepackage{eucal}
\usepackage{tcolorbox}
\usepackage{parnotes}
\usepackage{marginnote}
\usepackage{marginfix}
\usepackage{caption,subcaption}
\usepackage{tikz}
\usepackage{pgfplots} % per disegnare i grafici di funzione
\usetikzlibrary{cd,calc,backgrounds} % commutative diagrams
\usepackage{cite}
\usepackage{listings}
\usepackage{imakeidx}
\usepackage{eucal}
\usepackage{array}
\usepackage{hyphenat}
\usepackage{hyperref}

\hypersetup{
    colorlinks=true,
    linkcolor=black,
    filecolor=black,      
    urlcolor=black,
    pdftitle={Paolini: Appunti di Analisi Matematica Uno},
	}
	
\makeindex

\newcommand{\eps}{\varepsilon}
\renewcommand{\phi}{\varphi}
\newcommand{\loc}{\mathit{loc}}
\newcommand{\weakto}{\rightharpoonup}
\newcommand{\implied}{\Longleftarrow}
\let\subsetstrict\subset
\renewcommand{\subset}{\subseteq}
\renewcommand{\supset}{\supseteq}

% calligraphic letters
\newcommand{\A}{\mathcal A}
\newcommand{\B}{\mathcal B}
\renewcommand{\C}{\mathcal C}
\newcommand{\D}{\mathcal D}
\newcommand{\E}{\mathcal E}
\newcommand{\F}{\mathcal F}
\newcommand{\FL}{\mathcal F\!\mathcal L}
\renewcommand{\H}{\mathcal H}
\newcommand{\K}{\mathcal K}
\renewcommand{\L}{\mathcal L}
\newcommand{\M}{\mathcal M}
\renewcommand{\P}{\mathcal P}
\renewcommand{\S}{\mathcal S}
\renewcommand{\U}{\mathcal U} %% intorni

% blackboard letters
\newcommand{\CC}{\mathbb C}
\newcommand{\HH}{\mathbb H}
\newcommand{\KK}{\mathbb K}
\newcommand{\NN}{\mathbb N}
\newcommand{\QQ}{\mathbb Q}
\newcommand{\RR}{\mathbb R}
\newcommand{\TT}{\mathbb T}
\newcommand{\ZZ}{\mathbb Z}

\newcommand{\abs}[1]{{\left|#1\right|}}
\newcommand{\Abs}[1]{{\left\Vert #1\right\Vert}}
\newcommand{\enclose}[1]{{\left( #1 \right)}}
\newcommand{\Enclose}[1]{{\left[ #1 \right]}}
\newcommand{\ENCLOSE}[1]{{\left\{ #1 \right\}}}
\newcommand{\floor}[1]{\left\lfloor #1 \right\rfloor}
\newcommand{\ceil}[1]{\left\lceil #1 \right\rceil}
\newcommand{\openinterval}[2]{\left(#1,#2\right)}
\newcommand{\closeinterval}[2]{\left[#1,#2\right]}
\newcommand{\closeopeninterval}[2]{\left[#1,#2\right)}
\newcommand{\opencloseinterval}[2]{\left(#1,#2\right]}

\newcommand{\To}{\rightrightarrows}
\renewcommand{\vec}[1]{\boldsymbol #1}
\newcommand{\defeq}{:=}
\DeclareMathOperator{\divergence}{div}
\renewcommand{\div}{\divergence}
% \DeclareMathOperator{\ker}{ker}  %% already defined
\DeclareMathOperator{\Imaginarypart}{Im}
\renewcommand{\Im}{\Imaginarypart}
\DeclareMathOperator{\Realpart}{Re}
\renewcommand{\Re}{\Realpart}
%\DeclareMathOperator{\arg}{arg}
\DeclareMathOperator{\tg}{tg}
\DeclareMathOperator{\arctg}{arctg}
\DeclareMathOperator{\tgh}{tgh}
\DeclareMathOperator{\settsinh}{settsinh}
\DeclareMathOperator{\settcosh}{settcosh}
\DeclareMathOperator{\setttgh}{setttgh}
\DeclareMathOperator{\erf}{erf}
\DeclareMathOperator{\li}{li}
\DeclareMathOperator{\ei}{ei}
\DeclareMathOperator{\Si}{Si}
\DeclareMathOperator{\FresnelS}{S}
\DeclareMathOperator{\tr}{tr}
\DeclareMathOperator{\im}{im}
\DeclareMathOperator{\sgn}{sgn}
\DeclareMathOperator{\diag}{diag}

\declaretheoremstyle[
spaceabove=6pt, spacebelow=6pt,
headfont=\normalfont\bfseries\itshape,
notefont=\mdseries, notebraces={(}{)},
bodyfont=\normalfont,
postheadspace=1em,
qed=,
%shaded={rulecolor=pink!30,rulewidth=1pt,bgcolor=pink!10}
]{exercise_style}

\declaretheoremstyle[
spaceabove=6pt, spacebelow=6pt,
postheadspace=1em,
qed=,
%shaded={rulecolor=blue!20,rulewidth=1pt,bgcolor=blue!5}
]{theorem_style}

\declaretheoremstyle[
spaceabove=6pt, spacebelow=6pt,
postheadspace=1em,
qed=,
%shaded={rulecolor=yellow!50,rulewidth=1pt,bgcolor=yellow!5}
]{axiom_style}

\numberwithin{equation}{chapter}
%\declaretheorem[name=Teorema,numberwithin=chapter]{theorem}
%\declaretheorem[name=Lemma,sibling=theorem]{lemma}
%\declaretheorem[name=Proposizione,sibling=theorem]{proposition}
%\declaretheorem[name=Corollario,sibling=theorem]{corollary}
%\declaretheorem[name=Paradosso,sibling=theorem]{paradox}
%\declaretheorem[%style=axiom_style,
%name=Assioma,sibling=theorem]{axiom}
%\declaretheorem[%name=Definizione,
%sibling=theorem]{definition}
%\declaretheorem[%style=exercise_style,
%name=Esempio,sibling=theorem]{example}
%\declaretheorem[%style=exercise_style,
%name=Esercizio,sibling=theorem]{exercise}
%\declaretheorem[%style=exercise_style,
%name=Osservazione,sibling=theorem]{remark}


%\newtheorem{theorem}{Teorema}[chapter]
%\newtheorem{lemma}[theorem]{Lemma}
%\newtheorem{exercise}[theorem]{Esercizio}
%\newtheorem{proposition}[theorem]{Proposizione}
%\newtheorem{corollary}[theorem]{Corollario}
%\newtheorem{example}[theorem]{Esempio}
%\newtheorem{definition}[theorem]{Definizione}
%\newtheorem{axiom}[theorem]{Assioma}

%% average integral, see https://tex.stackexchange.com/questions/759/average-integral-symbol
\def\Xint#1{\mathchoice
{\XXint\displaystyle\textstyle{#1}}%
{\XXint\textstyle\scriptstyle{#1}}%
{\XXint\scriptstyle\scriptscriptstyle{#1}}%
{\XXint\scriptscriptstyle\scriptscriptstyle{#1}}%
\!\int}
\def\XXint#1#2#3{{\setbox0=\hbox{$#1{#2#3}{\int}$ }
\vcenter{\hbox{$#2#3$ }}\kern-.6\wd0}}
\def\ddashint{\Xint=}
\def\dashint{\Xint-}

\ExplSyntaxOn
\newcommand\stripExclamation[1]{
\def\tmp{#1}
\regex_replace_all:nnN { "! } { 91848243 }\tmp
\regex_replace_all:nnN { ! } { \  }\tmp
\regex_replace_all:nnN { 91848243 } { ! }\tmp
\tmp}
\ExplSyntaxOff

\newcommand{\mymark}[1]{\reversemarginpar\marginnote{#1}\normalmarginpar}

\newcommand{\mynote}[1]{\marginnote{{\footnotesize\stripExclamation{#1}}}}
\newcommand{\mymargin}[1]{\mynote{#1}\index{#1}}
% utilizzo di myemph:
% \myemph[margin note]{emph and index}
\newcommand{\myemph}[2][dummy_not_defined]{%
  \emph{\stripExclamation{#2}}%
  \ifthenelse{\equal{#1}{dummy_not_defined}}%
    {\ifthenelse{\isempty{#2}}{}{\mynote{#2}}%
    \index{#2}}%
    {\mynote{#1}%
    \index{#2}}}%

\newwrite\myauxfile
\immediate\openout\myauxfile=AnalisiUno.myaux
\newcommand{\myaux}[1]{\write\myauxfile{#1}}

% #1 shorten url della pagina: l'url completo va messo in make-docs.sh
% #2 testo HTML da scrivere nel link della documentazione
\newcommand{\myurl}[2]{%
\marginnote{\qrcode[height=1.3cm]{http://paolini.github.io/AnalisiUno?#1}\\%
\scriptsize{(interagisci)}~}%
\write\myauxfile{<!--MYURL--><li><a href="?#1">#2</a> (pagina \thepage, capitolo \thechapter)</li>}%
}
% come \myurl ma con un offset verticale #3
\newcommand{\myurloff}[3]{%
\marginnote{\qrcode[height=1.3cm]{http://paolini.github.io/AnalisiUno?#1}\\%
\scriptsize{(interagisci)}~}[#3]%
\write\myauxfile{<!--MYURL--><li><a href="?#1">#2</a> (pagina \thepage, capitolo \thechapter)</li>}%
}
%
% #1: url della pagina
% #2: testo da scrivere sotto il qrcode
% #3: testo HTML da scrivere nel link della documentazione
%% negli URL i caratteri % vanno sostituiti con \%
\newcommand{\myqrcode}[3]{%
  \marginnote{\qrcode[height=1.7cm]{#1}\\%
  {\footnotesize #2}}%
  \myaux{<!--MYQRCODE--><li><a href="#1">#2 #3</a> (pagina \thepage)</li>}%
  }

\widemarginfalse

\graphicspath{{figures/}}

%% per compilare un solo capitolo scommenta
%% una delle righe seguenti:
%\includeonly{chapters/chapter-00-introduzione}
%\includeonly{chapters/chapter-01-reali}
%\includeonly{chapters/chapter-02-successioni}
%\includeonly{chapters/chapter-03-serie}
%\includeonly{chapters/chapter-04-complessi}
%\includeonly{chapters/chapter-05-derivate}
%\includeonly{chapters/chapter-06-integrali}
%\includeonly{chapters/chapter-07-spazi}
%\includeonly{chapters/chapter-08-ricorrenza}
%\includeonly{chapters/chapter-09-edo}
%\includeonly{chapters/chapter-98-algebra}


\begin{document}
\title{Appunti di\\Analisi Matematica Uno}
\author{Emanuele Paolini}
\date{\today}
\publishers{manu-fatto}

\frontmatter % Denotes the start of the pre-document content, uses roman numerals

\maketitle

\thispagestyle{empty}
\mbox{}
\vfill
\doclicenseThis

\chapter*{Introduzione}

Queste note sono nate come appunti per il corso di Analisi Matematica %% README
del corso di studi in Fisica dell'Università %% README
di Pisa negli anni accademici 2017/18 e 2018/19. %% README
 %% README
Queste note sono estensive, non c'è alcun tentativo di concisione. %% README
L'obiettivo è quello di raccogliere tutti quei risultati che non sempre è %% README
possibile esporre in maniera dettagliata e rigorosa a lezione. %% README
Troveremo, ad esempio, %% README
definizioni equivalenti della funzione esponenziale e una definizione %% README
analitica (tramite serie di potenze) %% README
delle funzioni trigonometriche (e di $\pi$). %% README
Proponiamo la dimostrazione del teorema fondamentale dell'algebra, %% README
della formula di Stirling e di Wallis, %% README
e dell'irrazionalità di $e$ e di $\pi$. %% README
Viene proposta una definizione formale dei simboli di Landau %% README
$o$-piccolo e $O$-grande con i relativi teoremi per trattare queste espressioni. %% README
Lo stesso viene fatto per il simbolo di integrale indefinito. %% README
Vengono trattati quei risultati algebrici che permettono di %% README
giustificare gli algoritmi per il calcolo delle primitive %% README
delle funzioni razionali e per risolvere le equazioni differenziali %% README
lineari con il metodo di similarità. %% README
 %% README
Le note (come il corso a cui fanno riferimento) %% README
riguardano l'analisi delle funzioni di una variabile %% README
reale. %% README
Gli argomenti trattati sono serie e successioni numeriche, %% README
il calcolo differenziale e il calcolo integrale. %% README
Viene anche introdotta la convergenza uniforme allo scopo di considerare, %% README
come ultimo argomento, lo studio delle equazioni differenziali ordinarie. %% README
Da subito vengono introdotti i numeri complessi che vengono utilizzati %% README
laddove possono aiutare a dare una visione più unitaria e concettualmente %% README
più semplice degli argomenti trattati (in particolare nello studio delle serie %% README
di potenze e nella definizione delle funzioni trigonometriche). %% README

\myqrcode{https://paolini.github.io/AnalisiUno/}{}{questa pagina web}
Queste note sono rese disponibili liberamente sia in formato PDF che %% README
in forma di sorgente %% README
\begin{comment}
LaTeX %% README
\end{comment}
\LaTeX{}
(puoi \emph{cliccare} o \emph{scansionare} il \emph{QR-code}
a margine).
Il materiale è costantemente in evoluzione %% README
e certamente contiene errori e incoerenze. Ogni suggerimento o commento è %% README
benvenuto! %% README


\section*{contributi}

Ringrazio gli studenti:
%
Valerio Amico,
Rico Bellani,
Fabio Bensch,
Davide Campanella Galanti,
Alessandro Canzonieri,
Luca Casagrande,
Alessandro Casini,
Tommaso Ceccotti,
Luca Ciucci,
Martino Dimartino,
Luigina Mazzone,
Michele Monti,
Ruben Pariente,
Paolo Pennoni,
Davide Perrone,
Lorenzo Pierfederici,
Mattia Ripepe,
Maria Antonella Secondo,
Antonio Tagliente,
Laura Toni,
Giacomo Trupiano,
Bianca Turini,
Francesco Vaselli,
Antoine Venturini,
Matteo Vilucchio,
Piero Viscone
%
che hanno segnalato errori e correzioni.

Ringrazio i colleghi Vincenzo Tortorelli e Pietro Majer
che mi hanno dato molti suggerimenti preziosi.


\tableofcontents

\mainmatter 

\chapter{fondamenti}
\section{logica}

Se, come diceva Galileo, la matematica è il linguaggio della natura,
è importante che la matematica sia essa stessa espressa in un linguaggio
che risulti essere il più possibile oggettivo e non ambiguo.
La \emph{logica} è la disciplina matematica che si occupa
dello studio e della formalizzazione del linguaggio matematico.
In questo capitolo riassumiamo in maniera sintetica ed intuitiva
alcuni concetti e contemporaneamente
fissiamo le notazioni che verranno utilizzate nel seguito.
%Per eventuali approfondimenti rimandiamo a \cite{appunti_logica}.

La logica studia i \emph{sistemi formali}%
\mymargin{sistemi formali}%
\index{sistemi formali} (anche detti \emph{sistemi logico-deduttivi}
o \emph{sistemi assiomatici)} che sono delle descrizioni meccaniche, non ambigue, 
di un linguaggio formale. 
Parliamo al plurale di \emph{sistemi formali} in quanto ogni ambito della matematica 
(o di altre scienze) potrebbe sviluppare un proprio sistema formale specializzato per quell'ambito. 
Il primo sistema formale è stato sviluppato da Euclide per descrivere le proprietà 
di punti, rette e circonferenze del piano (la \emph{geometria euclidea}).
Al tempo di Euclide la formalizzazione era ancora intuitiva e incompleta, 
la formalizzazione moderna 
di tale sistema è stata completata da Hilbert, dopo duemila anni.
Peano ha introdotto un sistema assiomatico per descrivere le proprietà dei numeri naturali.
Dedekind ha individuato gli assiomi per descrivere i numeri reali.
Cantor ha introdotto la \emph{teoria degli insiemi} all'interno della quale è stato 
possibile includere tutte le altre teorie matematiche. 
Tale teoria è stata formalizzata da Zermelo e Fraenkel
ed è questa la teoria che useremo nel nostro corso.

Tutti sistemi formali descrivono un linguaggio.
Per descrivere un linguaggio dobbiamo dire innanzitutto quali sono i \emph{simboli}%
\mymargin{simboli}%
\index{simboli}
di quel linguaggio. In generale possiamo pensare ai simboli come alle lettere 
(o caratteri, nel linguaggio dell'informatica) che possono essere utilizzati per comporre le frasi.
Simboli tipici delle teorie matematiche sono ad esempio: 
\texttt{x}, \texttt{y}, \texttt{5}, \texttt{7}, 
\texttt{+}, \texttt{=}, \texttt{)}, \texttt{:} etc.
Nei linguaggi informatici i simboli corrispondono ai caratteri presenti sulla tastiera 
di un computer. 
Nelle lingue naturali si prenderebbero come simboli le singole lettere dell'alfabeto
a cui aggiungere eventualmente i caratteri per la punteggiatura.
Una sequenza finita di simboli si chiama \emph{formula}%
\mymargin{formula}%
\index{formula} (potremmo anche chiamarle \emph{frasi}).
Ad esempio: \texttt{x7)+} potrebbe essere una formula formata da quattro simboli.
Tra tutte le formule un sistema formale deve individuare quelle che si potrebbero chiamare 
\emph{formule ben formate} ovvero le formule a cui effettivamente vogliamo dare significato.
Ad esempio la formula $x+5=7$ potrebbe essere una formula ben formata perché siamo in grado 
di dargli un significato.
Il sistema formale non dà un significato alle formule ben formate 
(il significato è una estrapolazione della nostra mente) ma semplicemente deve dare delle regole 
per determinare quali siano le formule ben formate e quali no.
Visto che ogni simbolo che utilizziamo può essere rappresentato al computer, possiamo pensare 
alle formule come alle stringhe dei linguaggi di programmazione e possiamo pensare che il sistema formale 
deve descrivere un algoritmo (cioè un procedimento meccanico) in grado di determinare 
se una formula è ben formata oppure no.
Tra le \emph{formule ben formate} il sistema formale deve infine specificare quali 
siano i \emph{teoremi}. 
Di nuovo questo deve essere fatto mediante un algoritmo puramente meccanico 
in modo da garantire che i teoremi risultino oggettivi e universali: 
non ci può essere disaccordo sulla validità di un teorema, eventualmente 
ci può essere disaccordo 
sulla interpretazione di tale teorema.
Tipicamente i sistemi formali sono \emph{deduttivi}.
Nei sistemi \emph{deduttivi} si identificano alcune formule che vengono 
chiamate
 \emph{assiomi} e che vengono immediatamente riconosciuti come teoremi.
Ad esempio vedremo che il primo assioma della teoria degli insiemi è 
\[
  \exists X\colon \not \exists y\colon y\in X
\]
che si potrebbe leggere 
\begin{displayquote}
esiste un insieme $X$ per il quale nessun $y$ è elemento di $X$
\end{displayquote}
ed esprime l'esistenza dell'insieme vuoto.
Oltre agli assiomi in un sistema deduttivo vengono specificate delle 
\emph{regole di inferenza} cioè dei modi in cui 
le formule possono essere modificate o composte in modo tale che se le formule 
di partenza sono teoremi anche la formula ottenuta lo è.
I \emph{sillogismi} di Aristotele possono essere utilizzati come esempi di regole 
di inferenza. 
Supponiamo che le formule 
\texttt{Socrate è un uomo} 
e \texttt{l'uomo è un animale}
siano entrambe teoremi. Allora possiamo pensare di definire una regola di inferenza 
che mi dice che se \texttt{X è un Y} è un teorema 
e \texttt{X è uno Z}
è un teorema allora anche \texttt{X è uno Z} è un teorema. 
Con questa regola di inferenza è quindi possibile dedurre che 
\texttt{Socrate è un animale}
è un teorema.
E' chiaro che se le regole formali possono essere definite meccanicamente 
tramite un algoritmo allora anche i teoremi possono essere determinati meccanicamente 
mediante un algoritmo.
La ricerca in matematica consiste nell'esplorare lo spazio delle formule ben formate 
per determinare quali siano effettivamente teoremi. 
Dare la \emph{dimostrazione}%
\mymargin{dimostrazione}%
\index{dimostrazione} di un teorema significa esibire tutta la catena delle formule 
e delle regole di inferenza che permettono di ottenere il teorema a partire dagli 
assiomi.

\subsection{proposizioni, operatori logici}

Tutto questo è un processo meccanico, ma in realtà è ovvio che i sistemi formali 
vengono definiti in modo tale da aver per noi un qualche tipo di significato intuitivo.
In tal modo la ricerca delle dimostrazioni non è un processo puramente meccanico 
ma segue delle linee di pensiero che possono richiedere intuizione, inventiva e anche 
senso estetico. 
Tipicamente a livello intuitivo vogliamo assegnare un valore di verità alle formule 
ben formate: vorremmo cioè dire che alcune formule sono 
\emph{vere}%
\mymargin{vero}%
\index{vere} 
ed altre sono 
\emph{false}%
\mymargin{falso}%
\index{false}. 
In tal caso le formule ben formate vengono usualmente chiamate \emph{proposizioni}%
\mymargin{proposizioni}%
\index{proposizione}
(nel linguaggio naturale diremmo: \emph{affermazioni}).
Un esempio di proposizione (falsa) potrebbe essere: 
\texttt{2+2=5}.

E' possibile combinare più proposizioni mediante
gli operatori logici. Se $P$ e $Q$ sono proposizioni
si può costruire la proposizione $P \land Q$
chiamata \emph{congiunzione logica}%
\mymargin{congiunzione}%
\index{congiunzione logica}.
Tale proposizione
si può leggere ``$P$ e $Q$'' ed è una proposizione
che risulta essere vera solamente nel caso in cui sia
$P$ che $Q$ siano vere
(si veda la tabella~\ref{tab:verita_operatori_logici}
per un riassunto schematico).
Spesso la congiunzione logica è sottointesa:
se si fa un elenco di proposizioni $P,Q,R$ 
si intende usualmente la loro congiunzione $P \land Q \land R$
cioè si intende che devono essere tutte vere.
La \emph{disgiunzione logica}%
\mymargin{disgiunzione}%
\index{disgiunzione logica} denotata
con $P \lor Q$
si può leggere ``$P$ o $Q$'' ed è una proposizione che
è vera se almeno una tra $P$ e $Q$ è vera.
La \emph{negazione logica}%
\mymargin{negazione}%
\index{negazione logica} denotata con $\lnot P$ è una
proposizione che si può leggere ``non $P$'' che
è vera quando $P$ è falsa ed è falsa quando $P$ è vera.

Operatori logici molto utilizzati sono le \emph{implicazioni}%
\mymargin{implicazioni}%
\index{implicazione}.
La proposizione $P\Rightarrow Q$ si può leggere ``$P$ implica $Q$''
e significa che $Q$ è vera se $P$ è vera. Non si confonda
il valore di verità di $P\Rightarrow Q$ con il valore di verità
di $Q$. Se $P$ è vera allora $P\Rightarrow Q$ è vera o falsa
a seconda che $Q$ sia vera o falsa. Ma se $P$ è falsa allora
l'implicazione $P\Rightarrow Q$ è vera indipendentemente dal
valore di $Q$. In effetti $P\Rightarrow Q$ è equivalente a
$Q \lor \lnot P$ perché per la verità di $P\Rightarrow Q$
basta che $Q$ sia vera (quando $P$ è vera) oppure che $P$ sia falsa.

La freccia inversa $P\Leftarrow Q$ si può utilizzare per
invertire l'implicazione: è equivalente a $Q \Rightarrow P$.
Se valgono entrambe le implicazioni
$(P \Leftarrow Q) \land (P\Rightarrow Q)$
è facile convincersi che $P$ e $Q$ devono avere lo stesso
valore di verità: diremo quindi che sono equivalenti e
scriveremo $P \Leftrightarrow Q$.

Nella tabella~\ref{tab:verita_operatori_logici} sono riportati
tutti i valori di verità che si possono ottenere combinando
tra loro due proposizioni. Nella tabella~\ref{tab:operatori_logici}
sono riportate alcune proprietà di tali operatori: queste
proprietà possono essere comprese interpretando il loro significato
ma possono anche essere dedotte meccanicamente tramite la tabella
di verità delle operazioni.

\begin{table}
\begin{center}
  \begin{tabular}{cc|cccccc}
    $P$ & $Q$ & $\neg P$ & $P\land Q$ & $P\lor Q$ & $P\Rightarrow Q$ &
    $P\Leftarrow Q$ & $P\Leftrightarrow Q$ \\\hline
    \texttt{F} & \texttt{F} & \texttt{V} & \texttt{F} & \texttt{F} & \texttt{V} & \texttt{V} & \texttt{V} \\
    \texttt{F} & \texttt{V} & \texttt{V} & \texttt{F} & \texttt{V} & \texttt{V} & \texttt{F} & \texttt{F} \\
    \texttt{V} & \texttt{F} & \texttt{F} & \texttt{F} & \texttt{V} & \texttt{F} & \texttt{V} & \texttt{F} \\
    \texttt{V} & \texttt{V} & \texttt{F} & \texttt{V} & \texttt{V} & \texttt{V} & \texttt{V} & \texttt{V} \\
    \end{tabular}
\end{center}
\caption{La tabella di verità degli operatori logici. 
$\texttt{F}$ significa \emph{falso}, $\texttt{V}$ significa \emph{vero}.}
\label{tab:verita_operatori_logici}
\end{table}

\begin{table}
\begin{tabular}{rcll}
                          &$\neg (P \land \neg P)$&                              & non contraddizione \\
                         &$P \lor \neg P$&                                       & terzo escluso \\
                         $\neg \neg P$ & $\iff$ & $ P$                           & doppia negazione\\
                                    $P \land Q$ & $\iff$ & $ Q \land P$                   & simmetria\\
                                     $P \lor Q$ & $\iff$ & $ Q \lor P$                    & \\
                              $\neg (P\land Q)$ & $\iff$ & $ (\neg P) \lor (\neg Q)$      & formule di De Morgan\\
                               $\neg (P\lor Q)$ & $\iff$ & $ (\neg P) \land (\neg Q)$     & \\
                            $(P\land Q) \lor R$ & $\iff$ & $ (P\lor R) \land (Q \lor R)$  & proprietà distributiva\\
                            $(P\lor Q) \land R$ & $\iff$ & $ (P\land R) \lor (Q \land R)$ & \\
                            $(P \Rightarrow Q)$ & $\iff$ & $ (Q \Leftarrow P)$            & antisimmetria\\
                            $(P\Rightarrow Q)$ & $\iff$ & $ (\neg Q\Rightarrow\neg P)$   & implicazione contropositiva\\
                        $\neg (P\Rightarrow Q)$ & $\iff$ & $ P \land (\neg Q)$            & controesempio\\
                             $P\Rightarrow Q$ & $\iff$ & $ \lnot(P \land (\neg Q))$     & dimostrazione per assurdo
\end{tabular}
\caption{Alcune proprietà degli operatori logici. Queste proposizioni sono tutte vere qualunque siano i valori di verità delle proposizioni $P$ e $Q$. In particolare le equivalenze logiche ci dicono che le proposizioni ai due lati dell'equivalenza assumono sempre lo stesso valore di verità e quindi sono interscambiabili.}
\label{tab:operatori_logici}
\end{table}

\subsection{predicati, quantificatori}

Un \emph{predicato}%
\mymargin{predicato}%
\index{predicato} è una proposizione che contiene
una o più variabili il cui valore di verità, quindi,
può dipendere dal valore assegnato alle variabili.
Un esempio di predicato è $x+2=5$ che risulta essere vero se $x=3$
e falso altrimenti.

Se un predicato dipende da una o più variabili queste
si chiamano \emph{variabili libere}%
\mymargin{variabili libere}%
\index{variabili libere}. E' possibile
\emph{chiudere} una variabile libera mediante un quantificatore.
Il \emph{quantificatore universale}%
\mymargin{quantificatore universale}%
\index{quantificatore universale} denotato col simbolo
$\forall$ serve ad affermare che il predicato è vero
per ogni possibile valore della variabile quantificata.
Ad esempio la proposizione $\forall x\colon x+2=5$ significa:
``per ogni $x$ si ha $x+2=5$'' ed è una proposizione falsa.
Il quantificatore $\forall x$ rende \emph{muta} la variabile
$x$ del predicato $x+2=5$ nel senso che il valore di verità 
della proposizione
non dipende più dal valore di quella variabile, che non è
più una variabile libera ma funge solo da segnaposto.
In effetti se cambio nome ad una variabile muta il valore 
di verità non cambia. Ad esempio scrivere $\forall x\colon x+1=1+x$
è equivalente a $\forall z\colon z+1=1+z$.

Il \emph{quantificatore esistenziale}%
\mymargin{quantificatore esistenziale}%
\index{quantificatore esistenziale} denotato col simbolo
$\exists$ serve ad affermare che il predicato è vero per
almeno un valore della variabile quantificata.
Ad esempio la proposizione $\exists x\colon x+2=5$ significa:
``esiste almeno un $x$ per cui risulta $x+2=5$''.
Quello che si ottiene è una proposizione in cui la variabile
$x$ è muta. In questo caso la proposizione è vera in quanto
per $x=3$ il predicato è vero.

Un predicato può dipendere da più variabili ed è quindi
possibile inserire più quantificatori. In tal caso l'ordine
dei quantificatori può essere rilevante.
Delle seguenti proposizioni la prima è vera, la seconda
invece è falsa:
\begin{gather*}
\forall x\colon \exists y\colon x+2=y \\
\exists y\colon \forall x\colon x+2=y.
\end{gather*}
Intuitivamente la prima affermazione ci dice che per ogni numero $x$ 
esiste un numero $y$ che differisce di $2$ da $x$.
La seconda ci dice che c'è un numero $y$ che differisce di $2$ 
da qualunque numero $x$.

\begin{table}
\begin{center}
\begin{tabular}{lcl}
$\lnot \forall x \colon P(x)$ & $\iff$ & $\exists x \colon \lnot P(x)$\\
$\lnot \exists x \colon P(x)$ & $\iff$ & $\forall x \colon \lnot P(x)$\\
$\forall x \colon P(x)$ & $\Longrightarrow$ & $P(c)$\\
$\exists x \colon P(x)$ & $\Longleftarrow$ & $P(c)$\\
$P \land \forall x \colon Q(x)$ & $\iff$ & $\forall x\colon P\land Q(x)$\\
$P \land \exists x \colon Q(x)$ & $\iff$ & $\exists x \colon P \land Q(x)$
\end{tabular}
\end{center}
\caption{Alcune proprietà dei quantificatori logici ($c$ è una qualunque costante)}
\label{tab:proprieta_quantificatori}
\end{table}

Alcune proprietà dei quantificatori logici sono riportate
nella tabella~\ref{tab:proprieta_quantificatori}.

Spesso il quantificatore universale viene sottointeso.
Ad esempio se si scrive $x+y=x+y$
si intende che tutte le variabili libere sono quantificate 
tramite quantificatore universale: 
$\forall x\colon \forall y\colon x+y=y+x$.
Il segno di interpunzione $\colon$ può essere omesso
$\forall x \forall y\ x+y=y+x$
e una quantificazione può essere fatta contemporaneamente 
su più variabili $\forall x,y\colon x+y = y+x$.
A volte, se non crea ambiguità, si potranno scrivere 
i quantificatori anche a fine frase $x+y=y+x, \forall x\forall y$.

\section{teoria degli insiemi}

Nei paragrafi precedenti abbiamo visto che i predicati possono essere
combinati tra loro e quantificati.
Ma quali sono i predicati elementari che possiamo considerare per fondare 
tutta la matematica?
La teoria degli insiemi ci fornisce la formalizzazione del concetto
di \emph{appartenenza}: il predicato di base (l'unico che ci servirà) è
un predicato della forma $x \in A$ che significa ``l'oggetto $x$ è un elemento
dell'insieme $A$''.
La teoria degli insiemi non spiega (né tantomento definisce)
cosa siano gli oggetti e cosa siano gli insiemi perché questi sono concetti
primitivi (come lo sono i \emph{punti} e le \emph{rette} per la geometria euclidea).
La teoria degli insiemi ci fornisce, semplicemente, le regole formali che
possono essere utilizzate per trattare i predicati della forma $x\in A$.
Ad esempio un \emph{assioma} 
della teoria degli insiemi è il seguente.
\begin{axiom}[insieme vuoto]
\[
  \exists A \colon \lnot \exists x\colon x \in A
\]
\end{axiom}
L'assioma significa: ``esiste un insieme $A$ che non contiene alcun elemento $x$''
ovvero stiamo dicendo che esiste l'\emph{insieme vuoto}%
\mymargin{insieme vuoto}%
\index{insieme vuoto} che normalmente viene
denotato con il simbolo $\emptyset$.
Altri opportuni assiomi della teoria degli insiemi garantiscono l'esistenza
dell'unione $A\cup B$, intersezione $A\cap B$ e differenza $A\setminus B$
di due insiemi qualunque $A$ e $B$. Tali operazioni
tra insiemi possono essere ricondotte alla relazione di appartenenza
e possono quindi essere definite come segue.
\begin{axiom}[operazioni tra insiemi]
Se $A$ e $B$ sono insiemi allora esistono gli insiemi  
$A\cup B$, $A\cap B$ e $A\setminus B$ tali che:
\begin{align*}
    x \in A \cup B &\iff (x\in A) \lor (x\in B),\\
    x \in A \cap B &\iff (x\in A) \land (x\in B),\\
    x \in A \setminus B &\iff (x\in A) \land \lnot (x \in B).
\end{align*}
\end{axiom}

La negazione dell'appartenenza $\lnot (x \in A)$ viene usualmente
abbreviata con $x \not \in A$.

Sempre utilizzando la semplice relazione di appartenenza possiamo definire
le relazioni di inclusione e uguaglianza tra insiemi:
$A \subset B$ si legge ``$A$ è un sottoinsieme di $B$'',
$A \supset B$ si legge ``$A$ è un sovrainsieme di $B$''
e $A=B$ si legge ``$A$ è uguale a $B$''. 
Queste relazioni sono definite dalle seguenti proprietà:
\begin{align*}
  A \subset B &\iff \forall x\colon (x\in A \implies x\in B)\\
  A \supset B &\iff \forall x\colon (x\in A \implied x\in B)\\
  A = B &\iff \forall x\colon (x\in A \iff x \in B).
\end{align*}
A parole diremo che $A$ è un sottoinsieme di $B$, 
$A \subset B$, se ogni elemento di $A$ è anche elemento di $B$.
Diremo che $A$ e $B$ sono uguali, $A=B$, 
se $A$ e $B$ hanno gli stessi elementi.
Ovviamente la relazione $\supset$ non è altro 
che la relazione inversa di $\subset$ cioè risulta 
$A\supset B$ se e solo se $B\subset A$.
Scriveremo anche $A \neq B$ per indicare
la relazione opposta dell'uguaglianza ovvero: $\lnot(A=B)$.
Si noti che $A=B$ è equivalente a $(A\subset B) \land (B\subset A)$.

La relazione $A \subset B$ significa che ogni elemento di $A$ 
è anche elemento di $B$ e quindi non ci sono elementi di $A$ che non stiano 
in $B$ cioè $A$ non è più grande di $B$. 
Come statagemma mnemonico si osservi che il simbolo 
$\subset$ è orientato in modo che l'insieme più piccolo 
stia dal lato più stretto della relazione così come 
nella disuguaglianza $3 \le 5$ il numero più piccolo 
sta dal lato più stretto del simbolo $\le$.
Per dimostrare che la relazione $A \subset B$ è vera 
bisognerà prendere un qualunque elemento di $A$ e dimostrare 
che tale elemento è anche elemento di $B$.
Per dimostrare che value l'uguaglianza $A=B$ tipicamente si procede 
dimostrando separatamente le due inclusioni $A\subset B$ e $B\subset A$.

Risulta molto utile la possibilità di costruire insiemi di insiemi.
Per questo motivo la teoria degli insiemi usualmente non fa distinzione
tra oggetti e insiemi di oggetti. Nella relazione primitiva $x\in A$ anche
$x$ può essere un insieme. Possiamo allora immaginare che ogni oggetto del
nostro universo sia un insieme. In questo modo la relazione di uguaglianza $A=B$
che abbiamo definito sopra risulta ben definita per ogni coppia di oggetti
(o insiemi, che è lo stesso) $A$ e $B$.

Visto che gli elementi di un insieme $\mathcal A$ sono a loro volta insiemi,
è possibile considerare l'unione $\bigcup \mathcal A$
e, se\mynote{%
L'intersezione di una famiglia vuota di insiemi darebbe l'insieme 
universo, che vedremo non può essere definito.
} 
$\mathcal A \neq \emptyset$, l'intersezione $\bigcap \mathcal A$ di tutti gli elementi
di $\mathcal A$.
Questo estende il concetto di unione e intersezione anche a famiglie
eventualmente infinite.

\begin{axiom}[unione e intersezione arbitraria]
Se $\mathcal A$ è un insieme qualunque esiste l'insieme $\bigcup \mathcal A$
e se $\mathcal A \neq \emptyset$ esiste l'insieme $\bigcap \mathcal A$
con le seguenti proprietà:
\begin{align*}
  x \in \bigcup \mathcal A & \iff \exists A \in \mathcal A \colon x\in A, \\
  x \in \bigcap \mathcal A & \iff \forall A \in \mathcal A \colon x\in A.
\end{align*}
\end{axiom}
Una notazione alternativa è la seguente:
\[
 \bigcup \mathcal A = \bigcup_{A\in \mathcal A} A, \qquad 
 \bigcap \mathcal A = \bigcap_{A\in \mathcal A} A.  
\]
Questa notazione mette in evidenza il fatto che l'insieme 
$\bigcup \mathcal A$ è l'unione di tutti gli elementi $A$ dell'insieme 
$\mathcal A$ mentre $\bigcap \mathcal A$ è l'intersezione 
di tutti gli elementi di $\mathcal A$.
L'insieme $\mathcal A$ viene spesso chiamata una \emph{famiglia}
di insiemi perché i suoi elementi vengono trattati come insiemi 
più che come oggetti.

Un altro assioma della teoria degli insiemi garantisce che per ogni
$x$ esiste un insieme il cui unico elemento è $x$. 
\begin{axiom}[singoletto]
  Se $x$ è un insieme esiste l'insieme $\ENCLOSE{x}$, 
  chiamato \emph{singoletto}%
\mymargin{singoletto}%
\index{singoletto}
  tale che:
  \[
    y \in \ENCLOSE{x} \iff y=x.
  \]
\end{axiom}
Facendo l'unione di singoletti possiamo definire (per elencazione) insiemi che contengono
un numero finito di oggetti:
\begin{align*}
  \ENCLOSE{a,b} &= \ENCLOSE{a} \cup \ENCLOSE{b} \\
  \ENCLOSE{a, b, c} &= \ENCLOSE{a,b} \cup \ENCLOSE{c}\\
  \ENCLOSE{a, b, c, d} &= \ENCLOSE{a,b,c} \cup \ENCLOSE{d}\\
  &\quad\vdots
\end{align*}

Si faccia però attenzione: l'insieme $\ENCLOSE{a,b}$ contiene due elementi
solamente se $a\neq b$, infatti se $a=b$ si potrà facilmente verificare
che
\begin{equation}\label{eq:4775523}
\ENCLOSE{a,a} = \ENCLOSE{a}.
\end{equation}

\begin{exercise}
  Verificare \eqref{eq:4775523} utilizzando le definizioni formali date in precedenza.
\end{exercise}
\begin{comment}
\begin{proof}[Svolgimento.]
Utilizzando le definizioni di unione, di singoletto e di disgiunzione logica
si ha l'equivalenza dei seguenti
predicati:
\begin{gather*}
  x \in \ENCLOSE{a,a}  \\
  x \in \ENCLOSE{a} \cup \ENCLOSE{a}\\
  (x \in \ENCLOSE{a}) \lor (x \in \ENCLOSE{a})\\
  (x = a) \lor (x = a) \\
  x = a \\
  x \in \ENCLOSE{a}
\end{gather*}
e dunque $\ENCLOSE{a,a}=\ENCLOSE{a}$ per la definizione di uguaglianza tra insiemi.
\end{proof}
\end{comment}

Se $P(x)$ è un predicato in una sola variabile $x$ vorremmo poter
definire l'insieme di tutti gli oggetti $x$ 
che rendono vero il predicato $P$.
Sorprendentemente se aggiungessimo questo assioma 
(chiamato assioma di specificazione ingenua) nella teoria degli insiemi
avremmo un paradosso%
\index{teoria!ingenua degli insiemi}%
\index{insiemi!teoria ingenua}%
\index{Cantor, Georg}%
\index{Frege}%
\index{Russell}%
\mynote{Georg Cantor (1845--1918), Bertrand Russell (1872--1970) 
vedi note storiche a pag.~\pageref{nota:Cantor}}.

\begin{theorem}[paradosso di Russell]
\label{th:Russell}%
Supponiamo che per ogni predicato $P(x)$ esista un insieme 
$\ENCLOSE{x\colon P(x)}$ formato da tutti gli $x$ 
che soddisfano il predicato $P(x)$:
\[
  a \in \ENCLOSE{x\colon P(x)} \iff P(a).
\]
Allora si ottiene un assurdo.
\end{theorem}
%
\begin{proof}
  Consideriamo il predicato $x \not \in x$
  per il quale dovrebbe esistere l'insieme:
  \[
    R = \ENCLOSE{x\colon x\not \in x}.  
  \]
  Ma allora avremmo, per definizione:
  \[
    R \in R 
    \iff R\not \in R
  \]
  in contrasto con il principio di non contraddizione
  (un predicato e la sua negazione non possono essere entrambi veri).
\end{proof}

Per evitare l'incoerenza è necessario limitare l'\emph{assioma di specificazione}
alla costruzione di sottoinsiemi di insiemi già costruiti.

\begin{axiom}[specificazione]
  Se $P$ è un qualunque predicato con una variabile libera $x$
  e se $B$ è un insieme, allora esiste l'insieme 
  $\ENCLOSE {x\in B\colon P(x)}$ formato 
  da tutti gli elementi di $B$ che soddisfano il predicato $P$:
\[
  a \in \ENCLOSE{x\in B\colon P(x)} \iff (a \in B \land P(a)).
\]
\end{axiom}

Se ora proviamo a ripetere il ragionamento di Russell dato un insieme $A$ 
possiamo costruire l'insieme $R=\ENCLOSE{x\in A\colon x\not \in x}$
e scopriremmo che non può essere $R\in R$ 
(altrimenti dovrebbe essere $R\not \in R$ assurdo).
Ma potrebbe benissimo essere $R\not \in R$ se inoltre $R\not \in A$.
Dunque si deduce che dato qualunque insieme $A$ c'è un insieme 
che non è elemento di $A$: dobbiamo quindi rassegnarci al fatto che non
esiste un insieme \emph{universo} contenente qualunque altro insieme,
ma almeno non otteniamo una contraddizione%
\mynote{%
D'altra parte non possiamo comunque escludere che una contraddizione 
esista. 
Infatti Goedel ha dimostrato che non è possibile dimostrare che 
non ci siano contraddizioni nel sistema assiomatico da noi considerato.
}.


Per completare la teoria degli insiemi introduciamo anche il concetto di
\emph{insieme delle parti}%
\mymargin{insieme delle parti}%
\index{insieme!delle parti}
\index{$\mathcal P(\cdot)$}% 
cioè l'insieme di tutti i sottoinsiemi di un insieme dato.
\begin{axiom}[insieme delle parti]
\label{def:insieme_parti}%
Se $A$ è un insieme esiste l'insieme $\P(A)$ delle parti di $A$
che è l'insieme di tutti i sottoinsiemi di $A$:
\begin{equation}\label{eq:insieme_delle_parti}
  x \in \mathcal P(A) \iff x \subset A.
\end{equation}
\end{axiom}

\section{relazioni}

Grazie agli assiomi precedenti è possibile definire il prodotto cartesiano
di due insiemi: sarà questo un concetto molto importante nel seguito.
Innanzitutto dobbiamo definire il concetto di \emph{coppia}%
\mymargin{coppia}%
\index{coppia}: dati
$a,b$ definiamo la coppia $(a,b)$ con primo elemento $a$ e secondo elemento $b$
come un oggetto che ha questa proprietà:%
\mynote{%
Un modo formale per definire la coppia tramite l'utilizzo di insiemi 
è dovuto a Kuratowski:
\[
   (a,b) = \ENCLOSE{\ENCLOSE{a},\ENCLOSE{a,b}}.
\]
Non è difficile verificare che questa definizione soddisfa la 
proprietà~\eqref{eq:coppia}.
Così possiamo definire l'insieme prodotto
\begin{align*}
  A\times B = \big\{&C\in \mathcal P(A\cup B)\colon \\ 
  & \exists a\colon \exists b\colon a\in A, b\in B, \\
  & C=\ENCLOSE{\ENCLOSE{a},\ENCLOSE{a,b}}\big\}
\end{align*}
}%
\begin{equation}\label{eq:coppia}
  (a, b) = (a', b') \iff (a=a') \land (b=b').
\end{equation}
Stiamo cioè richiedendo che una coppia venga identificata dai due
elementi che la compongono in cui, però, è importante anche l'ordine in
cui vengono elencati (a differenza dell'insieme $\ENCLOSE{a,b}$, in cui
l'ordine degli elementi è irrilevante).
Nel seguito useremo anche la notazione $a \mapsto b$ per indicare 
la coppia $(a,b)$ e la chiameremo \emph{freccia}%
\mymargin{freccia}%
\index{freccia} da $a$ in $b$.

Il \emph{prodotto cartesiano} $A\times B$ di due insiemi $A$ e $B$
è l'insieme di tutte le coppie
il cui primo elemento sta in $A$ e
il secondo elemento sta in $B$:
\[
  (a, b) \in A \times B \iff (a\in A \land b\in B).
\]

Se rappresentiamo gli elementi di $A$ come dei punti su una retta
orizzontale (asse delle $x$) e gli elementi di $B$ come dei punti
su una retta verticale (asse delle $y$) gli elementi di $A\times B$
possono essere rappresentati come i punti del piano che proiettati sull'asse
delle $x$ vanno in $A$ e proiettati sull'asse delle $y$ vanno in $B$
(si veda la figura~\ref{fig:funzione}). 
  
Una \emph{relazione}%
\mymargin{relazione}%
\index{relazione} $R$ tra gli elementi di un insieme $A$ e gli elementi
di un insieme $B$ non è altro che un sottoinsieme di $A\times B$
ovvero $R\in \mathcal P(A\times B)$.
Per una relazione $R\subset A\times B$ si userà la notazione infissa
$aRb$ per indicare $(a,b)\in R$.
Oppure con la notazione delle freccie potremo 
scrivere $a \stackrel R \mapsto b$
per indicare che la freccia $a\mapsto b$ è un elemento di $R$. 
Ad esempio se prendo $A=B=\ENCLOSE{1,2,3}$ e considero la relazione 
$R=\ENCLOSE{(1,2),(1,3),(2,3)}$
il predicato $aRb$ rappresenta l'usuale relazione d'ordine $a<b$ sui
tre numeri considerati.

\begin{definition}[relazione di equivalenza]
\label{def:equivalenza}%
Sia $R\subset A\times A$ una relazione sull'insieme $A$. Diremo che 
$R$ è una \emph{relazione di equivalenza}%
\mymargin{relazione di equivalenza}%
\index{relazione!di equivalenza} se valgono le seguenti proprietà
(tipiche dell'uguaglianza)
per ogni $x,y,z\in A$:
\begin{enumerate}
  \item riflessiva: $x R x$;
  \item simmetrica: se $x R y$ allora $y R x$;
  \item transitiva: se $x R y$ e $yRz$ allora $x R z$.
\end{enumerate}
Se $R$ è una relazione di equivalenza diremo che $x$ è equivalente a $y$ 
(tramite $R$) quando $xRy$.
Dato $x \in A$ l'insieme di tutti gli elementi di $A$ che sono equivalenti 
a $y$ si chiama \emph{classe di equivalenza}%
\mymargin{classe di equivalenza}%
\index{classe!di equivalenza} di $x$. 
Si definisce in questo modo:
\[
  [x]_R = \ENCLOSE{y\in A \colon xRy}.  
\]
Se $R$ è una relazione di equivalenza su $A$ definiamo 
l'\emph{insieme quoziente}%
\mymargin{insieme quoziente}%
\index{insieme!quoziente}%
$A/R$
come l'insieme di tutte le classi di equivalenza:
\[
 A/R 
 = \ENCLOSE{[x]_R\colon x\in A} 
 = \ENCLOSE{B\in \mathcal P(A)\colon \exists x\in A\colon B=[x]}.  
\]
\end{definition}

Se $R$ è una relazione di equivalenza su $A$ 
l'insieme quoziente $A/R$ rappresenta l'insieme 
degli oggetti di $A$ in cui vengono identificati tra loro gli oggetti tra loro 
equivalenti.
Spesso diremo che $A/R$ rappresenta $A$ a meno della equivalenza $R$.

Ad esempio se $A=\ENCLOSE{1,2,3,4,5}$ e prendiamo $D=\ENCLOSE{1,3,5}$ (i numeri dispari)
e $P=\ENCLOSE{2,4}=A\setminus D$ 
(i numeri pari) possiamo definire una relazione $R$ su $A$ mediante la proprietà:
\[
 x R  y \iff (x\in P \land y\in P) \lor (x\in D \land y\in D).
\]
La relazione $R$ rappresenta la proprietà degli elementi di $A$ 
di avere la stessa ``parità''. 
Si avrà $[1]_R = [3]_R=[5]_R= D$ e $[2]_R=[4]_R=P$.
Dunque 
\[
   A/R = \ENCLOSE{P,D} = \ENCLOSE{\ENCLOSE{2,4},\ENCLOSE{1,3,5}}
\]
e, in effetti, questo insieme ha due elementi $P$ e $D$ che è quello 
che si ottiene identificando i numeri di $A$ in base alla proprietà 
di essere pari o dispari. 

L'insieme quoziente $A/R$ è un \emph{partizione}%
\mymargin{partizione}%
\index{partizione} di $A$ in quanto 
è formato da insiemi disgiunti la cui unione è tutto $A$. In effetti 
dare una partizione di $A$ è equivalente da dare una relazione 
di equivalenza.

\section{funzioni}

Sia $f$ una relazione tra due insiemi $A$ e $B$. 
Useremo la notazione delle frecce quindi scriveremo $a\stackrel f\mapsto b$ 
se $(a,b)\in f$. 
Diremo che $f$ è una funzione da $A$ in $B$ e scriveremo 
$f\colon A\to B$ se per ogni $a\in A$ esiste un unico $b\in B$ 
tale che $a \stackrel f \mapsto b$.
Potremmo dire che $f$ è definita su $A$ 
(perché per ogni punto di $A$ c'è una freccia)
ed è univoca (perché tale freccia è unica).

Dato $a\in A$ esiste dunque un unico $b\in B$ tale che
$a\stackrel f \mapsto b$: chiameremo $fa$
o $f(a)$ 
\mymargin{$f(x)$}%
tale oggetto $b$
e diremo che la funzione $f$ manda $a$ in $b$. 
Si avrà dunque
\[
 b=f(a) \iff a\stackrel f \mapsto b.
\]
Gli elementi del dominio spesso vengono chiamati \emph{punti}
mentre gli elementi del codominio, se sono numeri, vengono 
chiamati \emph{valori}. 
La funzione $f$ avrà quindi valore $f(a)$ nel punto $a$.
L'insieme di tutte le funzioni $f\colon A\to B$ viene usualmente denotato $B^A$
oppure $A\to B$.
\mymargin{$B^A$}%
\index{$A^B$}%
\index{$A\to B$}%
\index{potenza!insiemi}%

\begin{figure}
  \mbox{}
  \hfill
  \begin{tikzpicture}[x=0.8cm]
  	\draw[->] (0,0) -- (4,0);
  	\draw[->] (0,0) -- (0,3.0);
  	\node at (4.3,0) {$A$};
  	\node at (0,3.4) {$B$};

      \draw (0.8,0) -- +(0,0.1) node [label=below:$a_1$]{};
      \draw (1.6,0) -- +(0,0.1) node [label=below:$a_2$]{};
      \draw (2.4,0) -- +(0,0.1) node [label=below:$a_3$]{};
      \draw (3.2,0) -- +(0,0.1) node [label=below:$a_4$]{};

      \draw (0,0.7) -- +(0.1,0) node [label=left:$b_1$]{};
      \draw (0,1.4) -- +(0.1,0) node [label=left:$b_2$]{};
      \draw (0,2.1) -- +(0.1,0) node [label=left:$b_3$]{};
      \draw (0,2.8) -- +(0.1,0) node [label=left:$b_4$]{};

  	\draw[dotted] (0.8,0) -- (0.8,0.7) -- (0,0.7);
  	\draw[dotted] (1.6,0) -- (1.6,2.1) -- (0,2.1);
  	\draw[dotted] (2.4,0) -- (2.4,1.4) -- (0,1.4);
  	\draw[dotted] (3.2,0) -- (3.2,2.1) -- (0,2.1);

  	\fill (0.8,0.7) circle (2pt) node[above] {$\scriptstyle (a_1,b_1)$};
  	\fill (1.6,2.1) circle (2pt) node[above] {$\scriptstyle (a_2,b_3)$};
  	\fill (2.4,1.4) circle (2pt) node[above] {$\scriptstyle (a_3,b_2)$};
  	\fill (3.2,2.1) circle (2pt) node[above] {$\scriptstyle (a_4,b_3)$};
  \end{tikzpicture}
%
\hfill
%
\begin{tikzpicture}[x=1cm,y=0.8cm]
\fill (1,4) circle (2pt) node (a1) [label=left:$a_1$]{};
\fill (0.8,3) circle (2pt) node (a2) [label=left:$a_2$]{};
\fill (1.2,2) circle (2pt) node (a3) [label=left:$a_3$]{};
\fill (1,1) circle (2pt) node (a4) [label=left:$a_4$]{};
\fill (4,3.5) circle (2pt) node (b1) [label=right:$b_1$] {};
\fill (3.5,2.5) circle (2pt) node (b2) [label=right:$b_2$] {};
\fill (4,1.5) circle (2pt) node (b3) [label=right:$b_3$] {};
\fill (4.5,2.5) circle (2pt) node (b4) [label=right:$b_4$] {};
\draw[->] (a1) edge (b1);
\draw[->] (a2) edge (b3);
\draw[line width=3pt,white] (a3) edge (b2);
\draw[->] (a3) edge (b2);
\draw[->] (a4) edge (b3);
\draw[draw=gray](1,2.5) ellipse (1cm and 2cm);
\draw[draw=gray](4.2,2.5) ellipse (1.2cm and 1.5cm);
\node at (1,5.5) {$A$};
\node at (4,5) {$B$};
\node at (2.5,4.5) {$f$};
\end{tikzpicture}
\hfill
\mbox{}
\caption[]{La funzione $f\colon A\to B$,
$f=\{a_1\mapsto b_1,$ $a_2 \mapsto b_3,$ $a_3 \mapsto b_2,$ $a_4 \mapsto b_3\}$
definita sull'insieme $A=\ENCLOSE{a_1, a_2, a_3, a_4}$
a valori nell'insieme $B=\ENCLOSE{b_1, b_2, b_3, b_4}$
rappresentata tramite grafico e
tramite diagrammi di Venn.}
\label{fig:funzione}
\end{figure}


L'insieme di partenza $A$ viene chiamato \emph{dominio}%
\mymargin{dominio}%
\index{dominio} 
della funzione $f\colon A\to B$,
mentre l'insieme di arrivo $B$ viene chiamato \emph{codominio}%
\mymargin{codominio}%
\index{codominio}.
La funzione $f$ rappresenta quindi un modo di assegnare in maniera univoca
ad ogni elemento del dominio un elemento del codominio.
Da un punto di vista informatico potremmo dire che $A$ è l'insieme
dei possibili \emph{input} e $B$ è l'insieme dei possibili \emph{output}
della funzione $f$.

Per come l'abbiamo definita, una funzione è dunque un insieme.
In generale le funzioni potrebbero essere definite in altri modi oppure 
potrebbero essere un concetto primitivo: dunque nei capitoli seguenti useremo le funzioni 
senza assumere che esse siano a loro volta degli insiemi.
Ad esempio invece di scrivere $(a,b)\in f$ scriveremo sempre $f(a)=b$ 
oppure $a\stackrel f \mapsto b$.
Sarà comunque molto importante
considerare l'insieme che rappresenta $f$ ma questo verrà chiamato
\emph{grafico}%
\mymargin{grafico}%
\index{grafico} di $f$, $G_f$ e potrà essere definito in questo modo:
\[
  G_f = \ENCLOSE{(x,y)\in A \times B\colon f(x)=y}.
\]
Nella nostra costruzione risulta effettivamente $G_f = f$ ma, come abbiamo detto,
in generale è opportuno distinguere la funzione dal suo grafico.
Uno degli argomenti principali di questo corso è lo studio 
del grafico delle funzioni reali, cioè le funzioni 
con dominio e codominio nell'insieme dei numeri reali.

\subsection{invertibilità}

Capita molto spesso che un fenomeno possa essere modellizzato matematicamente
tramite una funzione: si sa che ad un certo \emph{input} $a$ corrisponde
un \emph{output} $b=f(a)$. Molto spesso il problema da risolvere è
quello di determinare l'\emph{input} giusto $a$ per ottenere l'\emph{output}
voluto $b$. Questo problema corrisponde ad \emph{invertire} la funzione $f$:
dato $b\in B$ determinare $x\in A$ tale che $f(x) = b$.

Una funzione $f\colon A \to B$ si dice essere \emph{surgettiva}%
\mymargin{surgettiva}%
\index{surgettiva} (o \emph{suriettiva})
se per ogni $b\in B$ esiste almeno un $x\in A$ per cui $f(x)=b$. Questo
significa che il problema dell'inversione ha almeno una soluzione, qualunque
sia $b\in B$.
Una funzione $f\colon A \to B$ si dice essere \emph{iniettiva}%
\mymargin{iniettiva}%
\index{iniettiva}
se non esistono due punti distinti $a,a' \in A$, $a\neq a'$ tali
che $f(a) = f(a')$. Questo significa che il problema dell'inversione
$f(x)=b$ ha al più una soluzione (la soluzione, se esiste, è unica).
Una funzione $f\colon A \to B$ si dice essere \emph{bigettiva}%
\mymargin{bigettiva}%
\index{bigettiva}
(o \emph{biettiva})
\index{biettiva}%
\index{bigettiva}%
\index{funzione!bigettiva}%
\index{invertibile}%
\index{funzione!invertibile}%
\index{biettiva}%
o
\emph{invertibile}%
\mymargin{invertibile}%
\index{invertibile}%
\mynote{\textbf{Attenzione:} in alcuni testi (tra cui~\cite{Giusti}) si considerano
invertibili le funzioni iniettive, anche se non surgettive.}
se è sia iniettiva che surgettiva. Questo significa
che il problema dell'inversione $f(x)=b$ ha una unica soluzione $x\in A$
qualunque sia $b\in B$. In particolare, se $f$ è invertibile, per ogni $b\in B$ esiste
un unico $a\in A$ per cui $f(a)=b$.
Se $f$ è una funzione invertibile allora la relazione inversa $g$
cioè la relazione tale che $b\stackrel g \mapsto a$ quando $a \stackrel f \mapsto b$
risulta essere una funzione $g\colon B\to A$. 
Infatti $g$ è definita su tutto $B$ in quanto $f$ è surgettiva 
e $g$ è univoca in quanto $f$ è iniettiva.
Le proprietà caratteristiche della \emph{funzione inversa}%
\mymargin{funzione inversa}%
\index{funzione!inversa} $g$ sono:
\begin{equation}\label{eq:572098}
  \forall a\in A\colon g(f(a)) = a, \qquad
  \forall b\in B\colon f(g(b)) = b.
\end{equation}
La funzione inversa $g$ viene usualmente denotata con il simbolo $f^{-1}$.

Se $f\colon A\to B$ è bigettiva diremo che $A$ 
e $B$ sono in \emph{corrispondenza biunivoca}%
\mymargin{corrispondenza biunivoca}%
\index{corrispondenza!biunivoca} tramite $f$.
In effetti $f$ è una corrispondenza \emph{univoca} da $A$ in $B$
(manda in modo univoco ogni punto di $A$ in un punto di $B$)
e $f^{-1}$ è una corrispondenza univoca da $B$ in $A$.

Introduciamo ora delle notazioni che sarà comodo utilizzare nel seguito.
Se $f\colon A \to B$ è una funzione e se $C\subset A$ definiamo
\[
  f(C) 
  = \ENCLOSE{f(a)\colon a \in C} 
  = \ENCLOSE{b\in B\colon \exists a\in C\colon b=f(a)}.
\]
L'insieme $f(C)\subset B$ si chiama \emph{immagine}%
\mymargin{insieme immagine}%
\index{immagine}
\index{immagine!insieme}%
di $C$ (tramite $f$) ed è formato
da tutti i punti che si ottengono applicando $f$ agli elementi di $C$.
L'immagine $f(A)$ dell'intero dominio $A$ si chiama immagine di $f$
e si denota a volte con il simbolo $\Im f$.
Si noti che $f$ è surgettiva se e solo se $f(A)=B$ (cioè se l'immagine coincide
col codominio).
Ad esempio la funzione definita in Figura~\ref{fig:funzione}
ha immagine $f(A) = \ENCLOSE{f(a_1),f(a_2),f(a_3),f(a_4)} 
 = \ENCLOSE{b_1, b_2, b_3}$.

Anche se $f\colon A \to B$ non fosse iniettiva,
per ogni $C\subset B$ possiamo definire
\[
  f^{-1}(C) = \ENCLOSE{a\in A\colon f(a) \in C}.
\]
L'insieme $f^{-1}(C)\subset A$ si chiama \emph{preimmagine}
o \emph{controimmagine}
\mymargin{preimmagine}%
\index{preimmagine}%
\index{preimmagine!insieme}%
\index{controimmagine!insieme}%
\index{insieme!controimmagine}%
di $C$ (tramite $f$)
ed è formato da tutti i punti di $A$ che applicando $f$ vanno in $C$.
Si noti che se $b\in B$ l'insieme $f^{-1}(\ENCLOSE{b})$ non è altro che
l'insieme delle soluzioni dell'equazione $f(x)=b$. Come abbiamo
già visto tale insieme contiene almeno un elemento se $f$ è suriettiva,
contiene al più un elemento se $f$ è iniettiva e contiene esattamente
un elemento $f^{-1}(\ENCLOSE{b}) = \ENCLOSE{f^{-1}(b)}$ se $f$ è bigettiva.
Ad esempio se $f$ è la funzione definita in figura~\ref{fig:funzione}
si ha $f^{-1}(\ENCLOSE{b_3,b_4}) = f^{-1}(\ENCLOSE{b_3}) = \ENCLOSE{a_2,a_3}$,
$f^{-1}(\ENCLOSE{b_4}) = \emptyset$.


La notazione $f(C)$ appena introdotta è formalmente ambigua in quanto
potrebbe non essere chiaro se $C$ è un elemento oppure un sottoinsieme
del dominio di $f$.
In pratica il contesto dovrebbe rendere chiaro cosa si intende.

Più in generale ci capiterà di estendere questo abuso di notazione non solo
alle funzioni, ma anche alle relazioni e alle operazioni.
Ad esempio se $A$ e $B$ sono insiemi di numeri ci capiterà di scrivere $A\le B$
per intendere che ogni elemento di $A$ è minore o uguale ad ogni elemento di $B$
oppure $A+B$ per intendere l'insieme di tutti i numeri che si ottengono sommando
un numero di $A$ ad un numero di $B$.

\begin{exercise}
  Sia $f\colon A \to B$ una funzione qualunque. 
  Si verifichi che se $C\subset A$ e $D\subset B$ si ha 
  \[
    f^{-1}(f(C))\supset C,
    \qquad 
    f(f^{-1}(D)) \subset D. 
  \]
  
  Che ipotesi possiamo fare su $f$ per avere l'uguaglianza
  $f(f^{-1}(C)) = C$?
  E per avere $f^{-1}(f(D)) = D$?
\end{exercise}

Si osservi che se $f\colon A \to B$ è una funzione e 
se $C\supset f(A)$ allora possiamo scrivere $f\colon A \to C$ in quanto 
i valori di $f$ sono elementi di $C$. Dunque il \emph{codominio} di una 
funzione può essere esteso o ristretto con l'accortezza di mantenere 
tutti i valori dell'immagine. 
In particolare $f\colon A \to f(A)$ è certamente surgettiva.

Per rendere iniettiva una funzione dobbiamo invece restringere il dominio.

\begin{definition}[restrizione]
  Se $f\colon A\to B$ è una funzione e $C\subset A$, possiamo 
  \emph{restringere} il dominio di $f$ all'insieme $C$.
  \mynote{La notazione $f\llcorner C$ non è del tutto standard,
  probabilmente è più comune la notazione $f_{|C}$.}%
  Si ottiene una nuova funzione $f\llcorner C$ che coincide con $f$
  ma che è definita solo su $C$: $f\llcorner C\colon C\to B$:
  \[
  f\llcorner C(x) = f(x)\qquad 
  \forall x \in C.
  \]
\end{definition}


\subsection{funzione composta}
\index{composizione!di funzioni}%

Se $f\colon A\to B$ e $g\colon B\to C$ allora un punto $a\in A$ 
viene mandato tramite $f$ in un punto $b\in B$ e 
a sua volta il punto $b$ viene mandato da $g$ in un punto 
$c \in C$. 
La funzione che manda $a$ in $c$ viene chiamata 
\emph{funzione composta}%
\mymargin{funzione composta}%
\index{funzione!composta} se denota con $g\circ f$ 
e si può definire così:
\[
g\circ f \colon A \to C, \qquad 
(g\circ f)(x) = g(f(x)).  
\]

Se abbiamo tre funzioni 
$f\colon A\to B$, $g\colon B\to C$ e $h\colon C\to D$ 
allora è facile verificare che:
\[
   h \circ (g\circ f) = (h\circ g) \circ f
\]
in quanto per ogni $x\in A$ si ha
\[
(h \circ (g\circ f)) (x) =
h(g(f(x))) = (h\circ g)(f(x)) = ((h\circ g) \circ f) (x).  
\]
Significa che l'operatore di composizione $\circ$
soddisfa la proprietà associativa.

Se $f\colon A\to B$ è bigettiva allora 
per ogni $x\in A$ e per ogni $y\in B$ si ha,
grazie a~\eqref{eq:572098},
\[
  f^{-1} (f(x)) = x,
  \qquad f(f^{-1}(y)) = y.
\]
Significa che 
\[
  f^{-1}\circ f = \id_A, 
  \qquad
  f\circ f^{-1} = \id_B
\] 
dove $\id_X\colon X\to X$
è la funzione 
\emph{identità}%
\mymargin{identità}%
\index{identità} 
\index{funzione!identità}%
cioè la funzione 
che lascia fisso ogni punto di $X$:
\mymargin{$\id_X$}%
\index{$\id$}%
\[
\id_X(x) = x \qquad \text{per ogni $x\in X$}.
\]

\begin{theorem}
Se $f\colon A\to B$ e $g\colon B\to C$ sono entrambe invertibili
anche $g\circ f\colon A\to C$ è invertibile e si ha 
\begin{equation}\label{eq:inversa_composta}
  (g\circ f)^{-1} = f^{-1}\circ g^{-1}.
\end{equation}
\end{theorem}
\begin{proof}    
Per ogni $c\in C$ esiste un unico $b\in B$ tale che $g(b)=c$ 
ed un unico $a\in A$ tale che $f(a)=b$. 
Tale $a$ è l'unico punto per cui $g(f(a))=c$ e questo dimostra che 
$g\circ f$ è bigettiva. Inoltre $x=f^{-1}(b)$ e $b=g^{-1}(c)$ 
dunque $(g\circ f)^{-1}(c) = a = f^{-1}(g^{-1}(c))$ da cui si ottiene 
\eqref{eq:inversa_composta}.
\end{proof}

\begin{exercise}
  Si verifichi che la composizione di funzioni iniettive è iniettiva e la composizione 
  di funzione suriettive è suriettiva.
\end{exercise}

\section{cardinalità}

\begin{definition}[cardinalità]
  Diremo che due insiemi $A$ e $B$ hanno la stessa \emph{cardinalità}%
\mymargin{cardinalità}%
\index{cardinalità} 
  (oppure sono \emph{equipotenti}) 
  \index{equipotenza}%
  \index{insiemi!equipotenti}%
  \index{insieme!equipotente}%
  se esiste una funzione bigettiva $f\colon A \to B$.
  Scriveremo in tal caso:
  \[
    \# A = \# B.  
  \] 
  Se esiste una funzione iniettiva $f\colon A\to B$ significa che 
  $A$ ha la stessa cardinalità di un sottoinsieme di $B$ (in quanto $f\colon A \to f(A)$
  risulta essere bigettiva). Scriveremo in tal caso:
  \[
    \# A \le \#B.
  \]
\end{definition}

Intuitivamente se due insiemi hanno la stessa cardinalità 
significa che hanno lo stesso numero di elementi.
Ma la precedente definizione riesce a catturare tale concetto senza dover 
ricorrere al concetto di numero. 
Questo ha il vantaggio di rendere questa definizione applicabile 
a qualunque insieme, anche con \emph{infiniti} elementi.

Si osservi che non abbiamo dato una definizione di $\#A$ e quindi non stiamo 
definendo cos'è la cardinalità di un insieme ma stiamo soltanto definendo 
una relazione tra insiemi che 
denotiamo, impropriamente, utilizzando una uguaglianza: $\#A = \#B$.
E' ovvio che $\#A = \#A$ in quanto l'identità $\id_A$ è bigettiva.
E' anche chiaro che se $\#A = \#B$ e $\#B = \#C$ allora $\#A = \#C$ in quanto 
componendo tra loro due funzioni bigettive si ottiene ancora una funzione 
bigettiva. 
Infine se $\#A = \#B$ allora $\#A \le \#B$ (in quanto le funzioni bigettive 
sono iniettive) ed è chiaro che se $\#A \le \#B$ e $\#B \le \#C$ allora 
$\#A\le \#C$ (in quanto composizione di funzioni iniettive è iniettiva)

Si potrebbe anche definire $\#A \ge \#B$ quando esiste $f\colon A \to B$
surgettiva. Vorremmo verificare se questo coincide con $\#B \le \#A$, 
cioè vogliamo avere il seguente.
%
\begin{theorem}\label{th:95444}
  Esiste $f\colon A\to B$ surgettiva 
  se e solo se esiste $g\colon B\to A$ iniettiva.
\end{theorem}
% 
\begin{proof}
Da un lato se esiste $g\colon B\to A$ iniettiva, allora $g$ è una bigezione 
tra $B$ e $f(B)$. La funzione inversa $f\colon f(B) \to B$ 
\mynote{Questa funzione $f$ si chiama 
\emph{inversa sinistra}
\index{inversa!sinistra}%
di $g$ in quanto si ha $f(g(y))=y$ per ogni $y\in B$}
può essere estesa a tutto $A$ fissando un valore qualunque 
(questo si può fare se $B$ non è vuoto) nei punti di $A\setminus g(B)$
ottenendo quindi una funzione surgettiva da $A$ in $B$.
D'altro lato se esiste una funzione $f\colon A\to B$ surgettiva,
per ogni $b\in B$ l'insieme $f^{-1}(\ENCLOSE{b})$ non è mai 
vuoto e quindi è chiaro che debba esistere 
una funzione $g\colon B\to A$ tale che $g(b)$ è un qualunque elemento 
di tale insieme
\mynote{Questa funzione $g$ si chiama 
\emph{inversa destra} 
\index{inversa!destra}%
di $f$ 
in quanto $f(g(y))=y$ per ogni $y\in B$}
(qui si utilizza l'assioma~\ref{axiom:AC} discusso più sotto). 
Chiaramente $g$ è iniettiva.
\end{proof}

Nella dimostrazione del teorema precedente abbiamo utilizzato il seguente assioma 
della teoria degli insiemi che, sorprendentemente, non è conseguenza degli assiomi 
che abbiamo già introdotto finora.

\begin{axiom}[della scelta]%
  \label{axiom:AC}%
  \index{assioma!della scelta}%
  \index{scelta!assioma della}%
  \index{AC}%
  Sia $F\colon A \to \mathcal P(B)$
  una funzione tale che $F(a)\neq \emptyset$ 
  per ogni $a\in A$. Allora esiste una funzione
  $f\colon A \to B$ tale che $f(a)\in F(a)$
  per ogni $a\in A$.
\end{axiom}

L'assioma della scelta (denotato spesso con \emph{$AC$}%
\mymargin{$AC$}%
\index{AC}, \emph{axiom of choice})
è un assioma per certi versi controverso
e alcuni matematici preferiscono non utilizzarlo nelle loro dimostrazioni.
Il sistema formale che definisce la teoria degli insiemi senza 
introdurre l'assioma della scelta 
si chiama \emph{$ZF$}%
\mymargin{$ZF$}%
\index{ZF} (Zermelo-Fraenkel) mentre 
se si aggiunge l'assioma della scelta (choice) la teoria si chiama 
\emph{$ZFC$}%
\mymargin{$ZFC$}%
\index{ZFC}.

Grazie all'assioma della scelta è possibile dimostrare 
che dati due insiemi $A$ e $B$ le loro cardinalità
si possono confrontare: $\#A\le \#B$ oppure 
$\#B\le \#A$. 
La dimostrazione però diventa complicata e va oltre i nostri scopi.

Il seguente teorema dimostra invece la proprietà antisimmetrica 
della relazione tra cardinalità. 
E' probabilmente il primo vero teorema 
(con dimostrazione decisamente non banale) 
che andiamo a dimostrare.

\begin{theorem}[Cantor-Bernstein]%
  \label{th:cantor_bernstein}%
  \index{teorema!di Cantor-Bernstein}%
  \index{Cantor-Bernstein!teorema di}%
  Se $\#A \le \#B$ e $\#B \le \#A$ allora $\#A = \#B$.
\end{theorem}
%
\begin{figure}
  \centering
  \begin{tikzpicture}
    \foreach \x in {1,0.5,0.25,0.25*0.5,0.25*0.25,0.25*0.25*0.5} {
      \path[draw,fill=black!20] (2*\x,2*\x)--(-2*\x,2*\x)--(-2*\x,-2*\x)--(2*\x,-2*\x)--cycle;
      \draw[fill=white] (0,0) circle (1.6*\x);
    };
    \node at (-1.0,2.2) {$A$};
    \node at (-1.5,1.5) {$A\setminus B$};
    \node at (0,1.3) {$B$};
%    \node at (-0.75,0.75) {$$};
  \end{tikzpicture}
  \caption{
  Nella dimostrazione del teorema di Cantor-Bernstein
  $A$ è rappresentato da un quadrato e $B$ da un cerchio contenuto
  in $A$. L'immagine di $A$ in $B$ è rappresentata da un quadrato contenuto
  in $B$ e così via. La parte ombreggiata è l'insieme $D$.
  }
  \label{fig:omotetia}
\end{figure}
%
\begin{proof}
Per ipotesi sappiamo che esiste
$g\colon B\to A$ iniettiva. 
Posto $B'=g(B)$ risulta che $g\colon B\to B'$ è bigettiva.
Ma allora dimostrare che esiste una bigezione tra $A$ e $B$ 
è equivalente a dimostrare che esiste una bigezione tra 
$A$ e $B'$. 
Dunque senza perdere di generalità, sostituendo $B'$ a $B$ 
possiamo supporre che sia $B\subset A$.

Essendo per ipotesi $\#A \le \#B$ esiste $f\colon A \to B$ iniettiva.
Intuitivamente l'idea è quella di definire l'insieme
\[
 D = (A\setminus B)  
 \cup f(A\setminus B) 
 \cup f(f(A\setminus B)) 
 \cup \dots
\]
e di definire la bigezione $\phi \colon A \to B$ 
utilizzando $f$ sull'insieme $D$ e lasciando fisso
il resto.

Per farlo in maniera rigorosa
consideriamo la famiglia di insiemi 
$\F = \{X \subset A \colon X \supset A \setminus B, f(X) \subset X\}$ 
e definiamo $D = \bigcap \F$.
Osserviamo che $A \in \F$ quindi $\F\neq \emptyset$.
Abbiamo in pratica definito $D$ come il più piccolo 
sottoinsieme di $A$ che viene mandato in se stesso da $f$.

E' facile verificare che $f(D) \subset D$ infatti dato $x\in D$ per ogni $X\in \F$ deve essere $x\in X$ ma allora $f(x) \in X$ (per come è definito $\F$), dunque $f(x) \in D$. 
In modo analogo si dimostra che $D\supset A\setminus B$ e dunque concludiamo che $D\in \F$.

Verifichiamo ora che $f(D)=D\cap B$. Da un lato se $x\in D$ allora
$f(x) \in f(D)\subset D$ e $f(x)\subset f(A)\subset B$ da cui $f(x) \in D\cap B$.
Dall'altro lato se $y\in D \cap B$ e non fosse $y \in f(D)$
allora potremmo considerare l'insieme $X=D\setminus\{y\}$
e osservare che $X\in \F$.
Infatti in primo luogo $X \supset A \setminus B$ in quanto $D$ ha questa proprietà e $y \in B$.
Inoltre dato qualunque $x \in X$ visto che $X\subset D$ allora
$f(x) \in f(D)$ e, per ipotesi,
$y\not \in f(D)$ dunque $f(x)\neq y$ da cui $f(x) \in X$.
Dunque $X\in \F$ ma allora dovrebbe essere $D\subset X$ mentre
per costruzione abbiamo $y\in D$ ma non in $X$.

% Avendo visto che $f(D)=D\cap B$ possiamo facilmente
% osservare che $f(A\setminus D) \subset A \setminus D$.
% Infatti $f(A\setminus D)\subset f(A)=B$ e quindi
% $f(A\setminus D)\cap D \subset D \cap B = f(D)$
% ma essendo $f$ iniettiva si deve avere
% $f(A\setminus D) \cap f(D)=\emptyset$.

Possiamo allora definire $\phi \colon A \to B$
\[
\phi(x) =
\begin{cases}
   f(x) & \text{se $x\in D$}, \\
   x & \text{altrimenti.}
\end{cases}
\]
Chiaramente $\phi$ è iniettiva in quanto $f$ è iniettiva e manda $D$ in $D$
e l'identità è iniettiva e manda $A\setminus D$ in $A\setminus D$.

Per dimostrare che $\phi$ è suriettiva consideriamo qualunque $y \in B$.
Se $y\not \in D$ allora $\phi(y)=y$.
Se invece $y\in D$ essendo $y\in D\cap B = f(D)$ esisterà $x\in D$ tale
che $\phi(x) = f(x) = y$.
\end{proof}

Possiamo ora determinare se un insieme $X$ è finito o infinito.
Gli insiemi infiniti sono quelli che possono essere messi in corrispondenza 
biunivoca con un loro sottoinsieme proprio: esiste $f\colon X\to X$ 
tale che $f$ è iniettiva ma non suriettiva. 
Invece se un insieme è finito togliendo anche un solo punto
se ne diminuisce la cardinalità: se $f\colon X\to X$ è iniettiva
allora è anche suriettiva.

\begin{definition}[infinito]
  \label{def:infinito}%
  Diremo che un insieme $X$ è \emph{finito}
  \index{insieme!finito}%
  \index{finito!insieme}%
  \index{Dedekind!finito}%
  (più precisamente: Dedekind-finito)
  se ogni funzione iniettiva $f\colon X\to X$ è anche suriettiva.
  \mymargin{insieme finito}%
\index{insieme finito}%

  Diremo che un insieme $X$ è \emph{infinito} 
  (più precisamente Dedekind-infinito)
  \index{insieme!infinito}%
  \index{infinito!insieme}%
  \index{Dedekind!infinito}%
  se non è finito ovvero
  se esiste $f\colon X\to X$ iniettiva ma non suriettiva.
  \mymargin{insieme infinito}%
\index{insieme infinito}%
\end{definition}

Con gli assiomi che abbiamo introdotto finora non è possibile dimostrare 
che esistono insiemi infiniti. 
Ma tutti gli insiemi numerici che andremo a considerare devono essere 
insiemi infiniti... dobbiamo quindi assumere anche il seguente. 

\begin{axiom}[infinito]
  \label{axiom:infinito}%
  Esiste un insieme infinito. 
\end{axiom}


\section{i numeri naturali}

I numeri naturali $0,1,2,\dots$ sono i numeri che utilizziamo per contare o per 
fare le iterazioni. C'è un primo numero naturale, che per noi sarà $0$ e poi
per ogni numero naturale $n$ ce n'è uno successivo che chiameremo $\sigma(n)$. 
Partendo da $0$ e passando al successivo si raggiungono tutti i numeri naturali.

\begin{theorem}[assiomi di Peano]
  \label{th:assiomi_peano}%
  \mynote{Giuseppe Peano (1858--1932) vedi note storiche a pag.~\pageref{nota:Peano}} 
    \label{def:naturali}%
  \index{numeri!naturali}%
  \index{$\NN$}%
  \index{Peano}%
  \index{assiomi!di Peano}%
  \index{insieme!induttivo}%
  \index{induttivo}%
Esistono un insieme $\NN$, 
un elemento $0\in \NN$ 
e una funzione $\sigma\colon \NN\to\NN$ tali che
\begin{enumerate}
  \item $\sigma$ è iniettiva (due numeri diversi non possono avere lo stesso 
  successore);
  \item $\sigma(\NN) = \NN \setminus\ENCLOSE{0}$ 
  (il numero $0$ è l'unico numero che non è successore di nessun'altro numero);
  \item se $A\subset \NN$ e 
  \begin{enumerate} 
    \item[(i)] $0\in A$ 
    \item[(ii)] $n\in A \implies \sigma(n)\in A$
  \end{enumerate}
  allora $A=\NN$.
\end{enumerate}
\end{theorem}
%
\begin{proof}
  Lo si dimostra grazie all'assioma~\ref{axiom:infinito} di infinito.
  Si veda il teorema~\ref{th:esistenza_naturali}.
\end{proof}

Si può dimostrare inoltre (si veda il teorema~\ref{th:unicitaN}) che le proprietà 
enunciate nel teorema precedente caratterizzano univocamente l'insieme 
$\NN$ dei numeri naturali.

Dato $n\in \NN$ il numero $\sigma(n)$ si chiama il \emph{successore}%
\mymargin{successore}%
\index{successore} di $n$. 
Il numero $0\in \NN$ che per assioma non è il successore di nessun'altro 
numero naturale si chiama \emph{zero}%
\mymargin{zero}%
\index{zero}. 
Il successore di $0$ viene chiamato \emph{uno} e si definisce 
come $1=\sigma(0)$. 
Allora scriveremo $n+1$ al posto di $\sigma(n)$ per denotare 
il successore del numero $n$.
Per comodità diamo un nome amche alle altre \emph{cifre decimali}%
\mymargin{cifre decimali}%
\index{cifre!decimali} 
ovvero ai primi numeri naturali:
\begin{equation}\label{eq:cifre}
\begin{gathered}
% 1 \defeq \sigma(0),\quad  
 2 \defeq \sigma(1),\quad
 3 \defeq \sigma(2),\quad 
 4 \defeq \sigma(3),\quad
 5 \defeq \sigma(4),\\ 
 6 \defeq \sigma(5),\quad 
 7 \defeq \sigma(6),\quad 
 8 \defeq \sigma(7),\quad 
 9 \defeq \sigma(8).
\end{gathered}
\end{equation}
Dunque $\sigma$ è l'usuale operazione del contare:
 \[
 0 \stackrel\sigma\mapsto 1 \stackrel\sigma\mapsto 2 \stackrel\sigma\mapsto 
 3 \stackrel\sigma\mapsto 4 \stackrel\sigma\mapsto 5 \stackrel\sigma\mapsto 
 6 \stackrel\sigma\mapsto \dots  
 \]
Il primi due assiomi servono a garantire che in questo processo del \emph{contare}%
\mymargin{contare}%
\index{contare}
troviamo sempre numeri diversi (non si torna mai indietro) in quanto nessun numero 
può avere come successore $0$ o un numero già incontrato in precedenza (che 
se non è zero è il successore di un altro numero).

Il fatto che $\sigma\colon \NN\to \NN$ sia iniettiva ma non suriettiva 
significa che $\NN$ può essere messo in corrispondenza biunivoca con 
$\sigma(\NN)$ che è un sottoinsieme proprio di $\NN$.
Dunque $\NN$ è necessariamente un insieme infinito 
(definizione~\ref{def:infinito}).
In particolare $\#\NN = \#\enclose{\NN \setminus\ENCLOSE{0}}$.
Questa proprietà è per certi versi paradossale
(paradosso di Galileo)
\index{paradosso!di Galileo}%
\index{paradosso!degli insiemi infiniti}%
e caratterizza gli 
insiemi infiniti, nel senso di Dedekind.%
\mynote{Galileo Galilei (1564--1642) vedi note storiche a pag.~\pageref{nota:Galileo}}%

L'ultimo assioma ci dice che se un sottoinsieme dei numeri naturali contiene lo 
zero e contiene il successore di ogni suo elemento, allora contiene tutti i 
numeri naturali.
Serve a garantire che il processo del contare esaurisca tutti 
i numeri naturali, e che quindi non ci siano dei naturali \emph{irraggiungibili}
partendo da zero.
Questa proprietà viene usualmente utilizzata mediante il seguente.

\mynote{%
Il principio di induzione non è un teorema della teoria degli insiemi 
in quanto si riferisce il concetto di \emph{predicato} che è esterno 
al sistema stesso.}

\index{principio!di induzione}%
\index{induzione matematica}%
\begin{theorem}[principio di induzione]
  Sia $P(n)$ un predicato.
  Se 
  \begin{enumerate}
    \item vale $P(0)$
    \item $\forall n\in \NN\colon P(n)\implies P(n+1)$
  \end{enumerate} 
  allora $\forall n\in \NN\colon P(n)$.
\end{theorem}
%
\begin{proof}
  Consideriamo l'insieme $A=\{n\in \NN\colon P(n)\}$.
  Grazie alle ipotesi del teorema possiamo applicare la terza
  proprietà dei numeri naturali per dedurre che $A=\NN$.
  Dunque $P(n)$ è soddisfatta per ogni $n\in\NN$.
\end{proof}

Tramite il principio di induzione è anche possibile 
definire funzioni (o operazioni) per induzione.
Se ad esempio vogliamo definire una funzione $f\colon \NN\to X$
basterà dichiarare il valore di $f(0)$ e definire $f(n+1)$ 
a partire da $f(n)$.
Il principio di induzione (o meglio il teorema~\ref{th:induzione})
garantisce che $f$ sia univocamente definita su tutto $\NN$.
Questo metodo per definire una funzione si chiama 
\emph{definizione ricorsiva} in quanto definisce 
il valore della funzione sul termine $n$-esimo rifacendosi
al valore assegnato sui termini precedenti.
In particolare è possibile definire
su $\NN$ le operazioni di addizione, 
moltiplicazione ed elevamento a potenza
tramite le seguenti definizioni ricorsive:
\begin{gather*}
  \begin{cases}
    n + 0 = n,\\
    n + (m+1) = (n+m)+1;
  \end{cases}
  \qquad
  \begin{cases}
    n \cdot 0 = 0,\\
    n \cdot (m+1) = n\cdot m + n;
  \end{cases}
  \qquad
  \begin{cases}
    n^0 = 1,\\
    n^{m+1} = n\cdot n^m.
  \end{cases}
\end{gather*}

Se ad esempio volessimo sapere quanto vale $2+3$ possiamo utilizzare 
la proprietà $n+(m+1) = (n+m)+1$ ponendo $n=2$ e $m=2$.
Si ottiene quindi $2+3 = 2+(2+1) = (2+2)+1$.
Per sapere quanto fa $2+2$ usiamo nuovamente la stessa proprietà 
ma stavolta con $m=1$. 
Si ottiene infatti $2+2=2+(1+1)=(2+1)+1$.
Per definizione $2+1=3$, $3+1=4$ e $4+1=5$ dunque si ottiene $2+3=5$.

Definiamo infine una relazione di ordinamento su $\NN$ 
con l'idea che quando sommo tra loro due numeri naturali 
quello che ottengo è un numero \emph{più grande}:
\[
   n\le m \iff \exists k\in \NN \colon n+k = m.
\]

\begin{theorem}[operazioni su $\NN$]
Le operazioni di addizione e moltiplicazione 
soddisfano le seguenti proprietà:
  \begin{enumerate}
    \item elemento neutro:
      $n + 0 = 0 + n = n$,
      $n\cdot 1 = 1\cdot n = n$;
    \item elemento assorbente:
      $n\cdot 0 = 0$;
    \item proprietà associativa: 
      $(n+m)+k = n+(m+k)$, $(n\cdot m)\cdot k = n \cdot (m\cdot k)$
    \item proprietà commutativa: 
      $n+m = m+n$, $n\cdot m = m\cdot n$;
    \item proprietà distributiva:
     $k\cdot (n+m) = k\cdot n + k\cdot m$;
    \item proprietà invariantiva:
     se $m+k = n+k$ allora $m=n$;
  \end{enumerate}
l'elevamento a potenza soddisfa le seguenti proprietà:
\begin{enumerate}
  \item $n^0 = 1$;
  \item $n^1 = n$;
  \item $n^{m+k} = n^m \cdot n^k$;
  \item $(n^m)^k = n^{m\cdot k}$;
  \item $(n\cdot m)^k = n^k \cdot m^k$;
\end{enumerate}
l'ordinamento soddisfa infine queste proprietà:
\begin{enumerate}
  \item $n\le n$ (proprietà riflessiva);
  \item se $n\le m$ e $m\le k$ allora $n\le k$ (proprietà transitiva);
  \item se $n\le m$ e $m\le n$ allora $n=m$ (proprietà antisimmetrica);
  \item o $n\le m$ oppure $m\le n$ (dicotomia);
  \item monotonia:
   se $m\le n$ allora 
   $m+k\le n+k$, $n\cdot k \le m\cdot k$,
   $k^m \le k^n$ e $m^k \le n^k$. 
\end{enumerate}
\end{theorem}
\begin{proof}
  Si vedano i teoremi~\ref{th:proprieta_addizione},
  \ref{th:proprieta_moltiplicazione},
  \ref{th:proprieta_potenza} e~\ref{th:proprieta_ordine_N}.
\end{proof}

Si osservi che abbiamo definito $n^0=1$ per ogni $n\in \NN$,
compreso $n=0$. 
Dunque abbiamo consapevolmente definito $0^0=1$:
questa definizione (controversa) risulterà essere utile.
Si noti invece che $0^n=0$ solo se $n\neq 0$.
Infatti qualunque numero moltiplicato per $0$ dà zero, 
quindi una moltiplicazione ripetuta da zero 
se c'è almeno un fattore nullo. 
Ma nel prodotto $0^0$ ci sono $0$ fattori $0$ quindi non c'è in effetti nessuna 
moltiplicazione per $0$. 
E' dunque naturale che il risultato sia $1$, 
l'elemento neutro della moltiplicazione.

\begin{exercise}
  Dimostrare per induzione che per ogni $n\in \NN$, $n\ge 4$ si ha 
  \[  
    2^n \ge n^2.
  \]
  Ovvero, dimostrare che per ogni $n\in \NN$ si ha 
  \[
    2^{n+4} \ge (n+4)^2.  
  \]
\end{exercise}

\subsection{rappresentazione posizionale dei numeri naturali}

Per rappresentare in modo efficiente qualunque numero naturale utilizziamo la 
\emph{notazione posizionale decimale}%
\mymargin{notazione posizionale decimale}%
\index{notazione!posizionale decimale}.
Ricordiamo che le cifre decimali $0,1,2,\dots, 9$ sono state definite in~\eqref{eq:cifre}
a pag~\pageref{eq:cifre}.
Il numero rappresentato da una sequenza finita di cifre decimali può essere 
definito ricorsivamente: una sequenza di una sola cifra rappresenta il numero 
corrispondente alla cifra stessa, una sequenza di $n+1$ cifre rappresenta il 
numero rappresentato dalle prime $n$ cifre, moltiplicato per $9+1$ (cioè dieci),
e sommato alla ultima cifra. 
Ad esempio la sequenza di cifre $4701$ (leggi: quattromilasettecentouno)
è definita così:
\[ 
  4701 = ((4\cdot(9+1)+7)\cdot(9+1)+0)\cdot(9+1)+1.
\]
Osservando che $10 = 9+1$ si scriverà
\begin{align*}
  4701 
  & = ((4\cdot 10 + 7)\cdot 10 +0)\cdot 10 + 1 \\
  & = \mathbf 4\cdot 10^3 + \mathbf 7\cdot 10^2 + \mathbf 0\cdot 10^1 + \mathbf 1 \cdot 10^0. 
\end{align*}

\subsection{fattoriale e semi-fattoriale}

Definiamo il 
\emph{fattoriale}%
\mymargin{fattoriale}%
\index{fattoriale} 
\index{"!}%
\index{$n$"!}%
denotato con $n!$ (leggi: $n$ fattoriale) 
tramite la seguente definizione ricorsiva
\[
  \begin{cases}
    0! = 1 \\
    (n+1)! = (n+1) \cdot n!
  \end{cases}
\]
in tal modo risulta che $n!$ sia il prodotto dei primi $n$ numeri naturali positivi:
\[
  n! = 1 \cdot 2 \cdot 3 \cdots n.  
\]

\begin{exercise}
  \label{ex:6734098}%
  Utilizzando il principio di induzione
  si dimostri che per ogni $n\in \NN$:
  \[
    (n+1)! \ge 2^n, \qquad
    n^n \ge n!
  \]
\end{exercise}

\begin{table}
  \begin{center}
  \begin{tabular}{r|>{\small}r>{\small}r>{\small}r>{\small}r>{\small}r}
  $n$       & 0 & 1 & 2 & 3 & 4 \\
  \footnotesize $n+5$     & 5 & 6 & 7 & 8 & 9 \\ \hline
  $2^n$     & 1 & 2 & 4 & 8 & 16 \\
  \footnotesize $2^{5+n}$ & 32 & 64 & 128 & 256 & 512 \\
  \footnotesize $2^{10+n}$ & 1024 & 2048 & 4096 & 8192 & 16384 \\
  \footnotesize $2^{15+n}$ & 32768 & 65536 & 131072 & 262144 & 524288 \\
  \footnotesize $2^{20+n}$ & 1048576 & 2097152 & 4194304 & 8388608 & 16777216 \\  \hline
  $3^n$                    & 1 & 3 & 9 & 27 & 81 \\
  \footnotesize $3^{5+n}$  & 243 & 729 & 2187 & 6561 & 19683 \\  \hline
  $n!$      & 1 & 1 & 2 & 6 & 24 \\
  \footnotesize $(5+n)!$  & 120 & 720 & 5040 & 40320 & 362880 \\
  \footnotesize $(10+n)!$  & 3628800 & 39916800 & 479001600 & 6227020800 & 87178291200 \\ \hline
  \footnotesize $10^{3n}$  &  & K (chilo) & M (mega) & G (giga) & T (tera) \\ 
  \footnotesize $10^{3(n+5)}$  & P (peta) & E (exa) & Z (zetta) & Y (yotta) \\ \hline
  \footnotesize $2^{10n}$  &  & Ki (chibi) & Mi (mebi) & Gi (gibi) & Ti (tebi) \\
  \footnotesize $2^{10(n+5)}$ & Pi (pebi)& Ei (exbi) & Zi (zebi) & Yi (yobi)
  \end{tabular}
  \end{center}
  \caption{I primi valori (e nomi) di alcune delle sequenze che abbiamo definito.
  Sarà utile in particolare ricordare che  $2^{10}=1024$ 
  è molto vicino a $10^3=1000$: questo giustifica i nomi simili utilizzati
  per le potenze di $2^{10}$ e per le potenze di $10^3$.
  }
  \end{table}
  
  A volte sarà utile considerare anche i prodotti di solamente i numeri
  pari o i numeri dispari fino ad un certo numero $n$. Questo
  si chiama \emph{semi-fattoriale}%
\mymargin{semi-fattoriale}%
\index{semi-fattoriale} e si indica con $n!!$
  Lo possiamo definire separatamente sui numeri pari (cioè 
  i numeri che si possono scrivere nella forma $2n$ con $n\in \NN$)
  e i numeri dispari (che scriviamo nella forma $2n+1$):
  \index{"!"!}
  \index{$n$"!"!}
  \index{doppio fattoriale}%
  \index{fattoriale!doppio}%
  \index{fattoriale!semi-fattoriale}%
  \index{semi-fattoriale}%
  \begin{align*}
    (2n)!! &= 2 \cdot 4 \cdot 6 \cdots (2n) \\
    (2n+1)!! &= 1 \cdot 3 \cdot 5 \cdots (2n+1).
  \end{align*}
  
  \begin{remark}
  \label{rem:doppio_fattoriale}%
  Si osservi che risulta
  \[
    (2n)!! = (2\cdot 1) \cdot (2\cdot 2) \cdot (2\cdot 3) \cdots (2\cdot n)
          = 2^n \cdot n!
  \]
  mentre
  \[
    2^n \cdot n! \cdot (2n+1)!! 
    = (2n)!! \cdot (2n+1)!!
    = (2n+1)!
  \]
  Queste formule permettono di esprimere il semi-fattoriale utilizzando
  il fattoriale intero e le potenze.
  \end{remark}
  
\begin{exercise}
  Si dia una definizione per induzione del semi-fattoriale
  (separatamente per i pari e per i dispari)
  e si dimostrino, per induzione, le formule nell'osservazione precedente.
\end{exercise}

\subsection{sommatoria e produttoria}
\index{sommatoria}%
\index{produttoria}%

Se $A$ è un insieme su cui è definita un'operazione di addizione 
(o di moltiplicazione) e se $\vec a \in A^n$ 
è una $n$-upla di elementi di $A$, 
possiamo definire la somma (o il prodotto)
\begin{equation}
  \label{eq:6122112}
  \begin{aligned}
  \sum_{k=0}^{n-1} a_k &= a_0 + a_1 + \dots + a_{n-1} \\
  \prod_{k=0}^{n-1} a_k &= a_0 \cdot a_1 \cdots a_{n-1}.
  \end{aligned}
\end{equation}
delle componenti del vettore $\vec a = (a_0, \dots, a_{n-1})$.
La definizione formale deve essere data per induzione:
\[
  \begin{cases}
    \displaystyle\sum_{k=0}^{-1} a_k = 0, \\
    \displaystyle\sum_{k=0}^{n} a_k = \enclose{\sum_{k=0}^{n-1} a_k} + a_n;
  \end{cases}  \qquad
  \begin{cases}
    \displaystyle\prod_{k=0}^{-1} a_k = 1, \\
    \displaystyle\prod_{k=0}^{n} a_k = \enclose{\prod_{k=0}^{n-1} a_k} \cdot a_n.
  \end{cases}
\]
La variabile $k$ che si trova nelle formule~\eqref{eq:6122112} è \emph{muta}: 
al suo posto si può utilizzare qualunque altra variabile che non
compaia altrove.

E' naturalmente possibile anche fare una somma a partire da un indice 
diverso da $0$. Se $m,n\in \ZZ$, $m\le n+1$, porremo:
\[
  \sum_{k=m}^n a_k = \sum_{j=0}^{n-m} a_{m+j}.
\]
Dal punto di vista mnemonico abbiamo fatto 
un \emph{cambio di variabile} $j=k-m$:
per $k=m$ si trova $j=0$ e per $k=n$ si trova $j=n-m$.

\begin{exercise}
  Se l'operazione di addizione è associativa e commutativa
  si può dimostrare per induzione che 
  la somma è \emph{lineare}:
  \[
  \sum_{k=m}^n  \Enclose{a_k + b_k} 
  = \sum_{k=m}^n a_k + \sum_{k=m}^n b_k, 
  \qquad
  \sum_{k=m}^n c\cdot a_k 
  =c\cdot \sum_{k=m}^n a_k
  \]
  e \emph{additiva}:
  \[
    \sum_{k=m+1}^n a_k = \sum_{k=m+1}^j a_k + \sum_{k=j+1}^n a_k.  
  \]
\end{exercise}

\begin{exercise}
  Dimostrare che 
  \[
    \sum_{k=1}^n k = \frac{n\cdot (n+1)}{2}, \qquad
    \sum_{k=1}^n k^2 = \frac{n\cdot (n+1)\cdot (2n+1)}{6}
  \]
\end{exercise}
\begin{proof}[Svolgimento]
Il risultato della prima somma può essere interpretato nel modo seguente:
la media di una progressione aritmetica $\frac{1+2+ \dots + n}{n}$ 
è uguale alla media tra il primo 
e l'ultimo termine della progressione: $\frac{1+n}{2}$.

Una volta identificata la formula la possiamo dimostrare per induzione.
Per $n=0$ la somma è pari a $0$ per definizione.
Se la formula è vera per un certo $n$, si ha 
\[
  \sum_{k=1}^{n+1} k = \enclose{\sum_{k=1}^n k} + (n+1)
   = \frac{n\cdot(n+1)}{2} + (n+1) 
   = \frac{(n+2)(n+1)}{2}
\]
che è quanto volevamo dimostrare.

La formula per la somma dei quadrati si può dimostrare facilmente per 
induzione (lo si faccia per esercizio) se già conosciamo la formula da 
dimostrare.

Un metodo per trovare la formula è quello di osservare che la differenza 
di due termini cubici consecutivi risulta essere quadratico:
\[
k^3 - (k-1)^3 = 3 k^2 - 3k + 1.  
\]
Sommando ambo i lati della precedente equazione si ottiene:
\begin{equation}\label{eq:309838}
\sum_{k=1}^n k^3 - \sum_{k=1}^n (k-1)^3 = 3\sum_{k=1}^n k^2-3\sum_{k=1}^n k+\sum_{k=1}^n 1.
\end{equation}
Al lato destro compare la somma di cui vogliamo calcolare il valore 
insieme ad altre due somme di cui sappiamo già il valore. 
Basterà allora determinare il valore del lato sinistro dove 
osserviamo che i termini delle due somme si cancellano 
a vicenda,
\mynote{%
si chiama \emph{somma telescopica}
\index{somma!telescopica}%
\index{telescopico}%
in quanto i termini delle due sommatorie si chiudono uno nell'altro 
come i tubi di un cannocchiale.} %
tranne l'ultimo della prima somma 
e il primo della seconda: dunque si ottiene $n^3 - 0^3=n^3$.
Moltiplichiamo per $2$ l'equazione~\eqref{eq:309838},
mettiamo in evidenza la somma dei quadrati e 
utilizzando la formula $2\sum_{k=1}^n k = n(n+1)$
per ottenere:
\begin{align*}
  6 \sum_{k=1}^n k^2 
  &=  2n^3 + 3 n(n+1) - 2n
  = n \cdot \Enclose{2(n^2-1) + 3(n+1)}\\
  &= n \cdot (n+1)(2n+1)
\end{align*}
che è quanto volevamo dimostrare.

Con lo stesso metodo si può trovare una formula per il calcolo 
di $\sum_{k=1}^n k^3$ e così via\dots
\end{proof}

Se $\sigma\colon \Enclose{n} \to \Enclose{n}$
è bigettiva (una tale funzione si chiama \emph{permutazione}%
\mymargin{permutazione}%
\index{permutazione})
e se l'addizione è associativa e commutativa
allora si può fare il cambio di variabile $k=g(j)$:
\[
    \sum_{k=0}^{n-1} a_k = \sum_{j=0}^{n-1} a_{\sigma(j)}.
\]
Questa uguaglianza può essere dimostrata facilmente nel caso 
in cui $\sigma$ scambi due soli indici lasciando fissi tutti gli altri 
(trasposizione) e poi può essere estesa a tutte le permutazioni
osservando che ogni permutazione si scrivere come composizione 
di trasposizioni.

Questa proprietà è sostanzialmente la proprietà commutativa della somma 
estesa ad un numero qualunque di addendi.
Grazie a questa proprietà possiamo definire la somma di una funzione 
definita su qualunque insieme finito. 
Se $f\colon X \to A$ 
(su $A$ è definita una operazione $+$ associativa e commutativa)
e $\sigma\colon \Enclose{n} \to X$ è una bigezione (dunque $\#X = n$)
allora si può definire 
\[
  \sum_{x\in X} f(x) = \sum_{j=0}^{n-1} f(\sigma(j))  
\]
in quanto la somma sul lato destro non dipende dalla bigezione $\sigma$ che 
abbiamo scelto.

\section{gruppi ordinati}

Intuitivamente si può ottenere $\RR$ a partire da una retta geometrica 
con lo stesso procedimento con cui si costruisce un righello.
Iniziamo col segnare
sulla retta un punto di riferimento che chiamiamo $0$, 
dopodiché osserviamo che
il punto $0$ (come ogni altro punto) divide la retta in due parti. In modo arbitrario
chiamiamo positivi i punti che si trovano da una parte e negativi i punti che
si trovano dall'altra parte. Sulla semiretta dei numeri positivi scegliamo, arbitrariamente,
un punto $1$. Il segmento compreso tra i punti $0$ e $1$ sarà la nostra unità di
misura.
L'addizione potrebbe essere definita utilizzando i movimenti rigidi.
Traslando il punto $1$ di una unità alla volta si ottengono, per iterazione, tutti i numeri naturali.
Traslando all'indietro si ottengono i numeri interi negativi. 
Assumendo che tra due punti qualunque esistano sempre punti intermedi (divisibilità) e
assumendo che suddividendo la retta in due parti esista sempre un punto di 
suddivisione (continuità) mostreremo come si può suddividere 
ogni segmento in $n$ parti uguali per ogni numero naturale $n$. 
In tal modo potremo ritrovare i numeri razionali e la moltiplicazione per un numero razionale.
Con una estensione crescente riusciremo quindi a definire la moltiplicazione 
tra numeri reali. 
Osserveremo poi che la struttura moltiplicativa dei numeri reali positivi soddisfa 
gli stessi assiomi della struttura additiva dei reali e che quindi la costruzione fatta 
per costruire la moltiplicazione si potrà ripetere identica per costruire l'elevamento a potenza.

Vedremo in particolare che $\RR$ si caratterizza 
con l'essere un \emph{gruppo totalmente ordinato denso e continuo}
in base alle seguenti definizioni.

\begin{definition}[relazione d'ordine]
  \label{def:ordine}%
  Una relazione
  $\le$ su un insieme $R$ viene detta
  \emph{relazione d'ordine}%
\mymargin{relazione d'ordine}%
\index{relazione!d'ordine}
  se soddisfa le seguenti proprietà (per ogni $x,y,z\in R$):
  \begin{enumerate}
    \item[1.] riflessiva: $x\le x$;
    \item[2.] antisimmetrica: $x\le y \land y\le x \implies x=y$;
    \item[3.] transitiva: $x\le y \land y\le z \implies x\le z$.
  \end{enumerate}
  Si dice inoltre che la relazione d'ordine $\le$
  è una relazione d'\emph{ordine totale}
\mymargin{ordine totale}%
  (o \emph{lineare}) 
  \index{ordinamento!totale}%
  \index{ordinamento!lineare}%
  \index{totale!ordine}%
  \index{lineare!ordine}%
  e si dice che $R$ è \emph{totalmente ordinato} se vale
  \index{totalmente ordinato}%
  \begin{enumerate}
    \item[4.] dicotomia: $x\le y \lor y\le x$.
  \end{enumerate}
\end{definition}

In generale, se $\le$ è una relazione d'ordine su $R$ si definisce la 
relazione inversa $\ge$ ponendo $x\ge y$ quando $y\le x$.
La proprietà riflessiva identifica gli ordinamenti \emph{larghi}.
Si definisce il corrispondente ordinamento stretto $<$
ponendo $x < y$ quando $x\le y \land x\neq y$
e la relazione inversa $>$ ponendo $x>y$ quando $y<x$.

\begin{definition}[ordinamento continuo]
  \label{def:ordinamento_continuo}%
  Un ordinamento $\le$ su un insieme $R$ si dice essere
  \emph{continuo}%
\mymargin{ordinamento continuo}%
\index{continuo}
  \index{continuità!ordinamento}%
  \index{ordine!continuo}%
  (o \emph{Dedekind-completo})
  \index{Dedekind-completo}%
  \index{completo!Dedekind}%
  se dati comunque $A,B$ sottoinsiemi non vuoti di $R$
    tali che per ogni $a\in A$ e per ogni $b\in B$ risulta $a\le b$
    (concisamente scriveremo $A\le B$)
    allora esiste $c\in R$ tale che per ogni $a\in A$ e ogni $b\in B$ 
    si ha $a\le c \le b$ (concisamente $A\le c \le B$).
\end{definition}

A parole la condizione $A\le B$ 
si può esprimere dicendo che $A$ e $B$ sono separati, 
mentre la condizione $A\le c \le B$ 
si può esprimere dicendo che $c$ è un \emph{elemento di separazione}
tra $A$ e $B$. 
Allora la precedente definizione ci dice che un insieme ordinato
è \emph{continuo} se ogni coppia di sottoinsiemi separati 
ammette un elemento di separazione.
In molti testi questa condizione viene chiamata \emph{completezza}
o, più precisamente, \emph{completezza di Dedekind}.
\index{completezza!di Dedekind}%
\index{Dedekind!completo}%

\begin{definition}[ordinamento denso]
  \label{def:ordinamento_denso}%
  Un ordinamento si dice essere
  \emph{denso}%
\mymargin{ordinamento denso}%
\index{denso} o \emph{divisibile} 
  se tra due punti distinti esiste sempre un punto intermedio:
  \[
   x < y \implies \exists c \colon x < c < y.
  \]
\end{definition}


\begin{definition}[gruppo]
  Un insieme $R$ su cui è definita una \emph{operazione} $*$ 
%  \mynote{%
%  una operazione $*$ su $R$ è una funzione $*\colon R\times R \to R$ 
%  che si indica utilizzando la notazione \emph{infissa}: $x*y = *(x,y)$.}
%% già detto
  si dice essere un \emph{gruppo}%
\mymargin{gruppo}%
\index{gruppo} se l'operazione
  ha le seguenti proprietà:
  \begin{enumerate}
    \item associativa: $\forall x,y,z\in R\colon (x*y)*z = x*(y*z)$;
    \index{proprietà!associativa}%
    \index{associatività}%
    \item esistenza elemento neutro: 
    \index{elemento!neutro}%
    \index{neutro}%
    $\exists e\in R\colon \forall x\in R \colon e*x=x*e = x$;
    \item esistenza inverso: 
    \index{elemento!inverso}%
    \index{inverso}%
    $\forall x\in R\colon \exists y\in R\colon x*y=y*x=e$.
  \end{enumerate}
  Inoltre il gruppo si dice essere \emph{abeliano}%
\mymargin{gruppo abeliano}%
\index{abeliano} o \emph{commutativo}
  se vale la proprietà:
  \begin{enumerate}
    \item[4.] commutativa: $\forall x,y\in R\colon x*y = y*x$.
    \index{proprietà!commutativa}%
    \index{commutatività}% 
  \end{enumerate}
  
  Quando l'operazione viene denotata con il simbolo $+$ (addizione)
  diremo che il gruppo è additivo, denoteremo con $0$ 
  \index{zero}%
  (zero) l'elemento neutro e l'inverso di $x$ verrà chiamato \emph{opposto}
  e si denota con $-x$.
  \index{opposto}%
  Se invece si usa il simbolo $\cdot$ (moltiplicazione)
  diremo che il gruppo è moltiplicativo, l'elemento neutro potrà 
  essere denotato con il simbolo $1$ (uno o unità) e 
  \index{uno}\index{unità}%
  l'inverso di $x$ potrà essere chiamato \emph{reciproco}
  e si denota usualmente con $x^{-1}$.
  \index{reciproco}%
  \end{definition}
  
  L'elemento neutro di un gruppo è unico. 
  Se infatti $x$ e $y$ fossero due elementi neutri 
  si avrebbe $x = x*y = y$. 
  Anche l'inverso è unico: infatti se $y$ e $z$ fossero 
  due inversi di $x$ si avrebbe $y = y * x * z = z$.
  
  \begin{definition}[gruppo ordinato]
    Diremo che $R$ è un \emph{gruppo ordinato}%
\mymargin{gruppo ordinato}%
\index{gruppo!ordinato} se $R$ è un gruppo
    (con operazione $*$ ed elemento neutro $e$) 
    se è anche un insieme ordinato (con relazione $\le$)
    e se vale la proprietà:
    \begin{enumerate}
      \item[1.] monotonia: se $x\le y$ allora per ogni $z$ si ha:
       \[
       x*z \le y*z \qquad\text{e}\qquad z*x \le z*y.
       \] 
    \end{enumerate}
  \end{definition}


Su un gruppo additivo si può definire la moltiplicazione per un numero 
naturale $n\in\NN$, come addizione ripetuta:
\[
\begin{cases}
  0\cdot x  = 0 \\
  (n+1) \cdot x = n\cdot x + x.
\end{cases}  
\]
Ad esempio $3\cdot x = 2\cdot x + x = 1\cdot x + x + x = 0 + x + x + x$.
Se il gruppo è moltiplicativo l'analoga definizione permette di definire 
la potenza $n$-esima: $x^0 = 1$, $x^{n+1}=x^n\cdot x$.

Per induzione si possono dimostrare facilmente le seguenti proprietà:
\begin{enumerate}
  \item $(n+m)\cdot x = n\cdot x + m\cdot x$; 
  \item $(n\cdot m)\cdot x = n\cdot (m\cdot x)$;
  %\item $n\cdot(x+y) = n\cdot x + n\cdot y$; % serve commutatività?
  \item $-(n\cdot x) = n\cdot(-x)$;
  \item se $n\cdot x = 0$ allora o $n=0$ o $x=0$;
  \item se $0\le x \le y$ allora $n\cdot x \le n\cdot y$;
  \item se $0\le x < y$ e $n > 0$ allora $n\cdot x < n\cdot y$;
  \item se $x\ge 0$ e $n\ge m$ allora $nx\ge mx$.
\end{enumerate}

Se $n\cdot x = y$ e $n\neq 0$ scriveremo $x=\frac y n$. 
Affinché questa definizione sia univoca bisogna però verificare che 
se $n\cdot x = n\cdot x'$ allora $x=x'$.
Infatti si ha 
\[
  0 = n\cdot x - n\cdot x' = n\cdot (x-x')
\]
e se $n\neq 0$ si conclude $x-x'=0$ ovvero $x=x'$.

Il seguente teorema ci permette di dimostrare che la divisione per $n\neq 0$ 
si può sempre fare.
Si faccia presente che se interpretiamo questo teorema su un 
gruppo moltiplicativo, invece che additivo, il risultato
non è del tutto banale perché 
stiamo dimostrando l'esistenza della radice $n$-esima.

\begin{theorem}[divisibilità]
\label{th:divisibile}%
Se $R$ è un gruppo additivo totalmente ordinato, denso e continuo
allora per ogni $y\in R$, e per ogni $n\in \NN$, $n\neq 0$ 
esiste $x\in R$ tale che $n\cdot x = y$.
\end{theorem}
%
Per dimostrare il teorema ci servirà il seguente.
%
\begin{lemma}[esistenza di numeri piccoli]
  \label{lm:numeri_piccoli}%
Sia $R$ un gruppo additivo totalmente ordinato e denso.
Per ogni $y\in R$, $y>0$ ed ogni $n\in \NN$ esiste $x\in R$, 
$x>0$ tale che $nx\le y$.
\end{lemma}
%
\begin{proof}
Vogliamo innanzitutto dimostrare che 
\begin{equation}\label{eq:41095633}
  \forall y>0 \colon \exists x>0 \colon x+x \le y.
\end{equation}
Per la proprietà di densità sappiamo che esiste $z$ tale che 
$0<z<y$. 
Prendiamo $w=y-z$. Se $w\le z$ allora $w+w\le w+z=y$ e possiamo prendere $x=w$.
Altrimenti $z+z < w+z = y$ e possiamo prendere $x=z$.
%% Attenzione: la dimostrazione è delicata se il gruppo non è commutativo: bisogna 
%% fare attenzione all'ordine degli addendi! 

Consideriamo ora l'insieme $M$ dei numeri naturali che 
soddisfano il teorema:
\[
  M \defeq \ENCLOSE{m\in \NN\colon \exists x>0\colon mx\le y}.
\]
Basterà dimostrare che $M=\NN$, lo possiamo fare per induzione.
Ovviamente $0\in M$ e $1\in M$ perché basta scegliere $x=y$.
Supponendo ora che $m\in M$, $m\ge 1$, basterà dimostrare che anche $m+1\in M$.
Se $m\in \NN$ esiste $z>0$ tale che $m\cdot z\le y$.
Per la proprietà~\eqref{eq:41095633} esiste $x>0$ tale che $2\cdot x\le z$.
Allora 
\[
  (m+1)\cdot x 
  \le (m+m)\cdot x 
  = m\cdot(2\cdot x)
  \le m\cdot z \le y
\]
e dunque $m+1\in M$ come volevamo dimostrare.
\end{proof}
%
\begin{proof}[Dimostrazione del teorema~\ref{th:divisibile}]
  Dobbiamo dimostrare che per ogni $y\in R$ e per ogni $n\in \NN$, $n\neq 0$
  esiste $x\in R$ tale che $n\cdot x=y$.
  Per fissare le idee supponiamo che sia $y>0$: il caso $y=0$ è banale 
  e se $y<0$ basterà applicare il risultato all'opposto $-y$.
  
  Fissati $n\in \NN$, $n\neq 0$ e $y\in R$, $y>0$ 
  consideriamo allora gli insiemi:
  \begin{align*}
    A &= \ENCLOSE{a\in R\colon n\cdot a\le y},\\
    B &= \ENCLOSE{b\in R\colon n\cdot b\ge y}.
  \end{align*}
  Chiaramente $0\in A$ e $y\in B$ quindi $A$ e $B$ non sono vuoti.
  Inoltre si ha $a\le b$ per ogni $a\in A$ e $b\in B$ 
  perché se fosse $b < a$ si avrebbe $nb < na$ 
  (necessariamente $b>0$) che 
  è assurdo essendo $na \le y \le nb$.
  
  Dunque $A\le B$ sono separati e non vuoti e per l'ipotesi di continuità di $R$ 
  possiamo dedurre che esiste $x$ elemento di separazione: $A\le x \le B$.
  Il nostro obiettivo è dimostrare che $n\cdot x=y$.

  Se fosse $n\cdot x > y$ per il lemma~\ref{lm:numeri_piccoli} dovrebbe 
  esistere $\eps>0$ tale che $n\cdot \eps < n\cdot x - y$.
  Per $k$ abbastanza grande si avrà $(k+1)\eps > x$ e quindi prendendo 
  il minimo $k\in \NN$ con tale proprietà si avrà
  \mynote{Il teorema~\ref{th:buon_ordinamento} garantisce 
  che ogni insieme non vuoto di numeri naturali ha minimo} 
  $k\cdot \eps < x \le (k+1)\cdot \eps$.
  Ma allora 
  \mynote{Non abbiamo supposto che l'addizione sia commutativa,
  ma sfruttiamo il fatto che i multipli di uno stesso numero 
  $\eps$ commutano tra loro.}
  \[
    n\cdot \eps + n\cdot k\cdot \eps
    = n(k+1)\cdot \eps 
    \ge n\cdot x 
    > n\cdot \eps + y 
  \]
  cioè $n\cdot k\cdot \eps > y$ da cui
  $k\cdot \eps \in B$. 
  Ma $k\cdot\eps < x \le B$ e quindi abbiamo un assurdo.

  D'altra parte se fosse $n\cdot x < y$ per il lemma~\ref{lm:numeri_piccoli}
  dovrebbe esistere $\eps>0$ tale che $n\cdot \eps < y-n\cdot x$.
  Allora, analogamente a prima, esiste $k\in\NN$ tale che 
  $k\cdot \eps \le x < (k+1)\cdot \eps$ e quindi
  \[
    n\cdot (k+1) \cdot \eps 
    = n\cdot \eps + n \cdot k \cdot \eps 
    < y - n\cdot x + n\cdot x 
    =y.  
  \]
  Dunque $(k+1)\cdot \eps \in A$ e questo è assurdo perché 
  $(k+1)\cdot \eps > x \ge A$.

  Possiamo quindi concludere che $n\cdot x = y$, come volevamo dimostrare.
\end{proof}

Il seguente risultato ci dice, informalmente, che non 
esistono quantità infinite (ma neanche infinitesime) in $\RR$.
%
\begin{theorem}[proprietà archimedea]
  \label{th:archimede}%
  \mymark{**}%
  \mymargin{proprietà archimedea}%
  \index{proprietà!archimedea}%
  \mynote{Questo teorema ci dice, che il gruppo $R$ 
  non contiene né elementi \emph{infiniti} (cioè più grandi di qualunque 
  multiplo di un numero positivo $x$) 
  né elementi \emph{infinitesimi} (cioè più piccoli 
  di qualunque frazione $\frac{x}{n}$ con $x>0$ e $n\in \NN$).
  Dunque le quantità infinitesime
  utilizzate da Newton e Leibniz per definire derivate e 
  integrali, non sono giustificabili all'interno 
  di un gruppo totalmente ordinato denso e continuo.
  E' però possibile definire un gruppo totalmente ordinato 
  denso ma non continuo in cui esistono quantità infinite e infinitesime:
  è quello che viene fatto nella \emph{analisi non-standard}.
  \index{analisi non standard}%
  }
    Sia $R$ un gruppo totalmente ordinato, denso e continuo.
  Allora per ogni $x,y\in R$, $x>0$, $y>0$ 
  esiste $n\in \NN$ tale che $n\cdot x > y$.
\end{theorem}
%
\begin{proof}
\mymark{*}%
Fissato $x>0$ consideriamo gli insiemi:
\[
A = \NN\cdot x = \ENCLOSE{n\cdot x\colon n\in \NN},
B = \ENCLOSE{b\in R\colon \forall n\in\NN\colon n \cdot x \le b }.
\]
Per definizione $A\le B$ e $A$ è diverso dal vuoto.
Se per assurdo il teorema fosse falso, si avrebbe $y\in B$ 
e quindi anche $B$ sarebbe non vuoto.
Dunque per la continuità dell'ordinamento dovrebbe esistere 
un elemento di separazione $s\in R$ tale che $A\le s\le B$.
Visto che $s-x<s$ e $s\le B$,
sappiamo che $s-x \not \in B$.
Dunque esiste $n\in \NN$ tale che $n\cdot x > s-x$
ma allora $(n+1)\cdot x > s-x+x=s$. 
Ma questo è assurdo perché $(n+1)\cdot x\in A$ e $s\ge A$.
%
%%% dimostrazione col SUP
%  Fissati $x,y\in R$, $x,y>0$ consideriamo l'insieme 
%  \[
%     A = \NN \cdot y = \ENCLOSE{n\cdot y\colon n\in \NN}.
%  \]
%  Basta dimostrare che $A$ non è superiormente limitato perché 
%  in tal caso dato $x\in \RR$ esisterebbe $n\in \NN$ per cui $n\cdot y> x$.
%
%  Chiaramente $A$ non è vuoto in quanto $0\in A$ dunque se $A$ 
%  per assurdo fosse superiormente limitato esisterebbe 
%  $m=\sup A$, $m\in R$. 
%  Siccome $m$ è il minimo dei maggioranti di $A$
%  e $m-y$ è più piccolo di $m$, allora $m-y$ non è un maggiorante. 
%  Dunque deve esistere $n\in \NN$ tale che $ny>m-y$.
%  Ma allora $(n+1)y = ny + y > m$ ed essendo $n+1\in \NN$ troviamo che $m$
%  non poteva essere un maggiorante di $A$: assurdo.
\end{proof}

\section{isomorfismi lineari}

% \begin{definition}[omomorfismo]
% Siano $R$ ed $S$ due gruppi 
% e sia $f\colon R\to S$.
% 
% Diremo che $f$ è \emph{additiva} se mantiene l'operazione 
% di gurppo cioè per ogni $x,y\in R$ si ha:
% \begin{equation}\label{eq:additivita}
%    f(x+y) = f(x) + f(y).
% \end{equation}
% In generale l'operazione su $R$ si potrebbe denotare 
% con un simbolo diverso, ad esempio $*$ (asterisco)
% e su $S$ potremmo avere $\circ$ (circoletto) come operazione.
% In tal caso\eqref{eq:additivita} diventa 
% \[
%   f(x * y) = f(x) \circ f(y)
% \]
% e si dirà che $f$ è un \emph{omomorfismo}.
% \end{definition}
% 
% \begin{definition}[funzioni monotòne]
% Siano $R$ e $S$ insiemi ordinati 
% e sia $f\colon R\to S$.
% \begin{enumerate}
%   \item \emph{crescente} quando mantiene l'ordinamento: 
%   per ogni $x,y\in R$ se $x\le y$ allora $f(x) \le f(y)$
%   \item \emph{decrescente} quando inverte l'ordinamento: 
%   per ogni $x,y\in R$ se $x\le y$ allora $f(y) \le f(x)$
%   \item \emph{strettamente crescente} quando mantiene l'ordinamento stretto: 
%   per ogni $x,y\in R$ se $x < y$ allora $f(x) < f(y)$
%   \item \emph{strettamente decrescente} quando inverte l'ordinamento stretto: 
%   per ogni $x,y\in R$ se $x < y$ allora $f(y) > f(x)$
%   \item \emph{monotòna} se è crescente o decrescente
%   \item \emph{strettamente monotòna} se è strettamente crescente 
%   o strettamente decrescente
% \end{enumerate}
% \end{definition}

\begin{theorem}[isomorfismi di gruppi ordinati]%
  \label{th:isomorfismo}%  
  Supponiamo che $R$ e $S$ siano gruppi totalmente ordinati, densi e continui.
  Denotiamo con $\stackrel R*$ l'operazione di gruppo su $R$ 
  e con $\stackrel S*$ quella su $S$, con $e_R$ ed $e_S$ denotiamo i corrispondenti 
  elementi neutri e con $\stackrel R\le$ e $\stackrel S\le$ 
  denotiamo le relazioni d'ordine.

  Fissato $u\in R$ con $u > e_R$ e fissato qualunque $v \in S$,
  $v \ge e_S$ esiste una unica funzione $\phi\colon R\to S$
  tale che:
  \begin{enumerate}
    \item $\phi(u)=v$
    \item proprietà di omomorfismo: 
    $\phi(x \stackrel R* y) = \phi(x) \stackrel S* \phi(y)$,
    \item positività:
    se $x\ge 0$ allora $\phi(x) \ge 0$. 
  \end{enumerate}
  Inoltre risulta che 
  \begin{enumerate}
    \item monotonia: 
    $x\stackrel R\le y \implies \phi(x) \stackrel S\le \phi(y)$.
    \item $\phi(e_R)=e_S$;
    \item $\phi\colon R\to S$ è bigettiva;
    \item $\phi(x) \stackrel S* \phi(y) = \phi(y)\stackrel S*\phi(x)$ 
    per ogni $x,y\in R$.
  \end{enumerate}

  Se si sceglie $v\le e_S$ valgono gli stessi risultati salvo 
  che le disuguaglianza di positività e monotonia si invertono:
  $x\stackrel R\le y \implies \phi(y) \stackrel S\le \phi(x)$.
\end{theorem}
    
\begin{proof}
Per rendere la notazione più semplice denotiamo con $0$, $+$ e $\le$ 
rispettivamente gli elementi neutri, le operazioni di gruppo 
e le relazioni d'ordine di entrambi i gruppi. 
Sarà il contesto a rendere chiaro se stiamo operando sul gruppo $R$ 
o sul gruppo $S$.

Osserviamo innanzitutto che se vale la proprietà di omomorfismo
si deve avere
$\phi(0) = \phi(0+0) = \phi(0)+\phi(0)$
e quindi deve essere $\phi(0)=0$.
Di conseguenza $0=\phi(x-x) = \phi(x)+\phi(-x)$
da cui $\phi(-x) = -\phi(x)$.
Dunque basterà definire $\phi(x)$ per $x>0$.

\emph{Passo 1: definizione sugli interi.}
Per induzione si osserva che per garantire la proprietà 
di omomorfismo per ogni $n\in \NN$ e per ogni $x\in R$ 
dovrà essere $\phi(n\cdot x) = n\cdot \phi(x)$. 
Il passo induttivo è il seguente:
\[
  \phi((n+1)\cdot x) 
  = \phi(n\cdot x + x)
  = \phi(n\cdot x) + \phi(x)
  = n\cdot \phi(x) + \phi(x)
  = (n+1)\cdot \phi(x).
\]
In particolare avendo imposto $\phi(u)=v$ deduciamo che deve essere 
necessariamente $\phi(n\cdot u) = n\cdot \phi(v)$.

\emph{Passo 2: definizione sulle frazioni.}
Per il teorema~\ref{th:divisibile} (divisibilità), per ogni $n\in \NN$, $n\neq 0$ 
ed ogni $x\in R$ esiste $y=\frac{x}{n}$ tale che $n\cdot y=x$.
Allora $\phi(x) = \phi(ny)=n\cdot \phi(y)$ da cui 
si scopre che deve essere $\phi(\frac x n) = \frac{\phi(x)}{n}$.
Dunque per ogni $k\in \NN$ ed ogni $n\in \NN\setminus\ENCLOSE{0}$
si deve avere 
\begin{equation}\label{eq:610954}
  \phi\enclose{k\cdot \frac u n} = k \cdot \frac v n.
\end{equation}
Effettivamente \eqref{eq:610954} definisce univocamente 
$\phi$ sui numeri della forma $k\cdot \frac u n$ perché 
se $k\cdot \frac u n = k'\cdot \frac u {n'}$ allora 
$k \cdot n'\cdot  u = k' \cdot n\cdot u$
e $k\cdot n' = k'\cdot n$ dunque risulta anche 
$k\cdot \frac v n = k'\cdot \frac v {n'}$.

\emph{Passo 3: estensione per monotonia.}
Rimane ora da definire $\phi(x)$ per ogni $x\in R$.
L'univocità della definizione sarà garantita dalla positività:
$\phi(x)\ge 0$ se $x\ge 0$.

Osserviamo innanzitutto che essendo valida la proprietà di omomorfismo
la positività implica la monotonia. 
Infatti se $y\le x$ si ha $x-y\ge 0$
e dunque essendo $\phi(x-y) = \phi(x) - \phi(y)$
se $\phi(x-y)\ge 0$ allora risulta $\phi(x)\ge \phi(y)$. 

Nei punti precedenti il valore di $\phi$ è già univocamente assegnato sugli $a$ e $b$ 
che si scrivono in forma di frazione: $k\cdot \frac u n$.
Dunque fissato $x\in R$, $x>0$ 
se $a \le x \le b$ con $a,b$ frazioni, 
per la monotonia 
dovremo necessariamente avere che $\phi(x)$ è 
compreso tra $\phi(a)$ e $\phi(b)$.
Dunque $\phi(x)$ deve essere elemento di separazione 
dei due insiemi:
\[
A = \ENCLOSE{k \cdot \frac v n \colon k,n\in \NN, n\neq 0, k\cdot \frac u n \le x },\qquad
B = \ENCLOSE{k \cdot \frac v n \colon k,n\in \NN, n\neq 0, k\cdot \frac u n \ge x}.
\]
Effettivamente questi insiemi sono separati 
perché se $k\cdot \frac u n \le k'\cdot \frac u {n'}$ allora $kn'\le k'n$ 
e di conseguenza $k\cdot \frac v n \le k' \cdot \frac v {n'}$.
Dunque per l'ipotesi di continuità di $S$ 
possiamo trovare almeno un elemento di separazione $s\in R$ 
tale che $A\le s \le B$.

E tale elemento è unico in quanto se ci fossero due elementi di separazione, 
$y_1<y_2$, per la proprietà archimedea (teorema~\ref{th:archimede}).
dovrebbe esistere $n\in \NN$ tale che $\frac v n < y_2-y_1$
e allora, prendendo i multipli di $\frac v n$ si troverebbe un $k\in\NN$ 
tale che $y_1 < k\cdot \frac v n < y_2$. 
Ma allora ci chiediamo se $k\cdot \frac u n$ è maggiore o minore di $x$ e 
in entrambi i casi otteniamo un assurdo: se fosse minore o uguale a $x$ 
allora $k\cdot \frac v n$ dovrebbe essere elemento di $A$ ma non può esserlo 
perché $k\cdot \frac v n>y_1\ge A$, 
analogamente se fosse $k\frac u n \ge x$ 
si avrebbe un elemento di $B$ che è strettamente più piccolo di $y_2\le B$.

Dunque $\phi(x)$ è univocamente determinata dall'essere elemento di 
separazione degli insiemi $A$ e $B$.
Ponendo poi $\phi(-x) = -\phi(x)$ abbiamo univocamente definito $\phi$ 
su tutto $R$. 
Osserviamo che se $x=k\cdot \frac u n$ (con $k,n\in \NN$, $n\neq 0$)
allora $k\cdot \frac v n \in A\cap B$ e quindi abbiamo effettivamente 
definito $\phi\enclose{k\cdot \frac u n} = k\cdot \frac v n$ estendendo la definizione 
già data nei passi precedenti.
Dobbiamo ora verificare che effettivamente $\phi$ 
verifica le proprietà richieste.

Per prima cosa dimostriamo che $\phi$ risulta essere strettamente crescente. 
Prendiamo $x,y\in R$ con $0<x<y$.
Per la proprietà archimedea (teorema~\ref{th:archimede})
esiste $n\in \NN$ tale che $n\cdot(y-x) < \frac u 2$ e quindi esisterà un $k\in \NN$ 
tale che $x \le k \cdot \frac u n < (k+1)\cdot \frac u n \le y$.
Per definizione sappiamo che $\phi(x)$ è minore o uguale ad ogni 
elemento dell'insieme $B$ e quindi $\phi(x) \le k\cdot \frac v n$.
Analogamente $\phi(y) \ge (k+1)\cdot \frac v n$. 
Ma $(k+1)\cdot \frac v n > k\cdot \frac v n$ in quanto $\frac v n>0$ e quindi possiamo 
concludere che $\phi(x) < \phi(y)$.
Passando agli opposti la verifica si estende facilmente al caso in cui $x<y<0$.
Il caso $x= 0$ o $y=0$ è infine banale.

Dimostriamo ora che $\phi(x+y)=\phi(x)+\phi(y)$.
Al solito lo facciamo nel caso $x>0$ e $y>0$, 
gli altri casi verranno di conseguenza.
Per ogni $n\in \NN$ possiamo trovare $k,j\in \NN$ tali che 
\[
  k\cdot \frac u n \le x \le (k+1)\cdot \frac u n, \qquad 
  j\cdot \frac u n \le y \le (j+1)\cdot \frac u n
\]
e per come è stata definita $\phi$ dovrà essere 
\[
  k\cdot \frac v n \le \phi(x) \le (k+1)\cdot\frac v n, \qquad  
  j\cdot \frac v n \le \phi(y) \le (j+1)\cdot\frac v n.
\]
Allora sommando le disuguaglianze troviamo 
\[
  (k+j) \cdot \frac u n \le x + y \le (k+j+2) \cdot \frac u n  
\]
e per come è stata definita $\phi(x+y)$ dovrà essere 
\[
  (k+j)\cdot \frac v n \le \phi(x+y) \le (k+j+2)\cdot \frac v n.
\]
Per differenza si ottiene:
\[
 -2\cdot \frac v n \le \phi(x+y) - \phi(x) - \phi(y) \le 2\cdot \frac v n
\]
ma anche 
\[
 -2\cdot \frac v n \le \phi(x+y) - \phi(y) - \phi(x) \le 2\cdot \frac v n
\]
e questo è vero per ogni $n\in \NN$.
Ma la proprietà archimedea (teorema~\ref{th:archimede}) 
ci dice che nessun numero positivo 
può essere minore di $2\frac v n$ per ogni $n\in \NN$ ed, equivalentemente,
nessun numero negativo può essere maggiore $-2\frac v n$ per ogni $n\in \NN$.
Se ne deduce che $\phi(x+y)-\phi(x)-\phi(y)=0$ 
(ma anche $\phi(x+y) - \phi(y)-\phi(x)=0$)
ovvero vale la proprietà di omomorfismo $\phi(x+y)=\phi(x) + \phi(y)$
e anche $\phi(x+y) = \phi(y) + \phi(x)$.

L'ultima proprietà che ci resta da dimostrare è che $\phi\colon R\to S$ 
sia iniettiva e suriettiva.
Se $x\neq y$ si avrà $x<y$ o $y<x$.
Nel primo caso abbiamo già mostrato che $\phi(x)<\phi(y)$ 
e nel secondo caso $\phi(y)<\phi(x)$.
In ogni caso $\phi(x) \neq \phi(y)$ e quindi 
$\phi$ è iniettiva.

Dimostriamo infine che $\phi$ è suriettiva. 
Posto $T=\phi(R)$ osserviamo che per le proprietà di $\phi$ 
anche $T\subset S$ dev'essere 
un gruppo totalmente ordinato, denso e continuo come tutto $S$
e con la stessa operazione e lo stesso ordinamento di $S$.
E $\phi^{-1}\colon T\to R$ è un omomorfismo crescente che manda $v$ in $u$.
Ma invertendo i ruoli di $R$ ed $S$ sappiamo anche esistere un omomorfismo 
iniettivo $\psi\colon S\to R$ che manda $v$ in $u$ e la sua 
restrizione a $T$, per unicità, deve coincidere con $\phi^{-1}$.
Questo significa che $T=S$, perché altrimenti non sarebbe possibile 
estendere la bigezione $\phi^{-1}\colon T \to R$ a tutto $S$.
\end{proof}

\begin{theorem}[commutatività]
  Sia $R$ un gruppo totalmente ordinato, denso e continuo. 
  Allora $R$ è abeliano, cioè: $x+y=y+x$ per ogni $x,y\in R$.
\end{theorem}
%
\begin{proof}
Se $R$ ha un solo elemento, $R=\ENCLOSE{0}$, allora non c'è niente da dimostrare.
Altrimenti scegliamo $u\in R$, $u>0$ e applichiamo il teorema precedente 
con $S=R$ e $v=u$. 
Otteniamo che esiste una unica $\phi\colon R\to R$ con le proprietà 
enunciate nel teorema. 
Ma anche l'identità $\id(x)=x$ ha tali proprietà, quindi $\phi =\id$.
Ma il teorema ci dice anche che $\phi(x+y) = \phi(y)+\phi(x)$ da cui si ottiene
$x+y=y+x$, come volevamo dimostrare.
\end{proof}
    
\section{i numeri reali}

L'insieme dei numeri reali $\RR$ non è altro che un gruppo additivo 
totalmente ordinato, denso e continuo su cui abbiamo fissato una unità 
$1\in \RR$, con $1>0$.

Per il teorema di isomorfismo~\ref{th:isomorfismo}
i gruppi totalmente ordinati, densi e continui 
possono essere messi in corrispondenza gli uni con gli altri 
una volta che si sia fissata una unità $u>0$. 
In questo senso possiamo dire che l'insieme dei numeri reali 
$\RR$, se esiste, è unico (a meno di isomorfismi).

L'esistenza di $\RR$, vista la sua rilevanza, potrebbe anche essere presa 
per assioma. 
Ma in realtà tale gruppo può essere costruito a partire dall'insieme $\NN$ 
dei numeri naturali (passando per gli interi $\ZZ$ e i razionali $\QQ$). 
Per maggiori dettagli si veda il capitolo~\ref{sec:costruzione_reali}.

Per ora su $\RR$ abbiamo una unica operazione: l'addizione 
(e la sua operazione inversa, ovvero la sottrazione).
Abbiamo visto come si possa definire la moltiplicazione per un numero naturale 
come addizione ripetuta (e la divisione per un numero naturale non nullo 
come operazione inversa). 

\subsection{moltiplicazione su $\RR$}

Vogliamo ora utilizzare il teorema di isomorfismo per definire 
la moltiplicazione tra numeri reali.
\mynote{
  Si noti che la moltiplicazione tra numeri reali dipende dalla scelta 
  dell'unità $1\in \RR$, mentre l'addizione è indipendente da essa.
  In effetti l'operazione di moltiplicazione, come operazione interna, 
  non è affatto naturale. 
  Se ad esempio utilizziamo i numeri reali per misurare la quantità 
  di corrente che attraversa un filo ha senso definire la corrente 
  nulla ha senso scegliere un verso (arbitrario) e considerare 
  correnti positive e negative, ed ha senso sommare tra loro due 
  misure di corrente. 
  Tutto questo può essere fatto senza introdurre 
  una unità di misura.
  La moltiplicazione di due correnti non ha invece senso.
  Se fissiamo una unità di misura (ad esempio l'Ampère)
  è possibile moltiplicare tra loro le 
  due correnti ma non è corretto pensare che il risultato sia una corrente.
  Nei modelli fisici quasi sempre la moltiplicazione è una operazione 
  esterna: se faccio il prodotto di due correnti misurate in Ampère 
  ottengo un risultato che vive in uno spazio diverso, la cui unità 
  si chiama Ampère-quadro.
}
\begin{theorem}[moltiplicazione sui reali]
  Su $\RR$ è definita una unica operazione $\cdot$ (moltiplicazione) 
  che a due numeri $x,y\in \RR$ associa il loro prodotto $x\cdot y\in \RR$ 
  con le seguenti proprietà:
  \begin{enumerate}
    \item proprietà distributiva: $x\cdot (y+z) = x\cdot y + x\cdot z$;
    \item elemento neutro: $x\cdot 1 = x$;
    \item positività: $x\cdot y\ge 0$ se $x\ge 0$ e $y\ge 0$.
  \end{enumerate}

  Inoltre tale operazione ha anche le seguenti proprietà:
  \begin{enumerate}
    \item elemento assorbente: $x\cdot 0 = 0$;
    \item regola del segno: $(-y)\cdot x = -(y\cdot x)$.
    \item proprietà commutativa: $x\cdot y = y\cdot x$;
    \item proprietà associativa: $(x\cdot y)\cdot z = x\cdot (y\cdot z)$;
    \item esistenza del reciproco: se $x\neq 0$ esiste $y$ tale 
    che $x\cdot y = 1$;
    \item annullamento del prodotto: se $x\cdot y=0$ allora $x=0$ oppure $y=0$;
    \item monotonia: se $x\ge 0$ e $y\ge z$ allora $x\cdot y\ge x\cdot z$;
    \item stretta monotonia: se $x>0$ e $y>z$ allora $x\cdot y > x\cdot z$.
  \end{enumerate}
\end{theorem}
%
\begin{proof}
Fissato $x\in \RR$
applichiamo il teorema di isomorfismo~\ref{th:isomorfismo}
prendendo $R=S=\RR$, $u=1$ e $v=m$.
Si ottiene allora una unica funzione $\phi_x\colon \RR\to \RR$ 
tale che $\phi_x(1)=x$, $\phi_x(y+z)=\phi_x(y)+\phi_x(z)$ 
e infine se $x\ge 0$ e $y\ge 0$ 
allora $\phi_x(y)\ge 0$ se 
invece $x\le 0$ e $y\ge 0$ si ha $\phi_x(y)\le 0$.

Se definiamo $x\cdot y = \phi_x(y)$ 
le proprietà di $\phi_x$ diventano 
\mymargin{$x\cdot 1 = x$}
$x\cdot 1 = x$ (elemento neutro), 
\mymargin{$x\cdot (y+z) = x\cdot y + x\cdot z$}
$x\cdot (y+z) = x\cdot y + x\cdot z$ (proprietà distributiva),
\mymargin{$x\cdot y\ge 0$}
e infine $x\cdot y\ge 0$ se $x\ge 0$ e $y\ge 0$ (positività).
Queste proprietà definiscono dunque il prodotto 
in modo univoco. 
Inoltre essendo $\phi_x(0)=0$ si ottiene immediatamente 
la proprietà assorbente: $x\cdot 0 = 0$.
Dobbiamo ora dimostriamo che valgono anche tutte le 
altre proprietà. 

\mymargin{$(-x)\cdot y = -(x\cdot y) = x\cdot(-y)$}
Cominciamo col dimostrare la regola del segno.
Ovviamente $x\cdot(-y) = -(x\cdot y)$ perché dal teorema 
di isomorfismo sappiamo che $\phi_x(-y) = -\phi_x(y)$.
Per dimostrare che $(-x)\cdot y=-(x\cdot y)$ dobbiamo 
mostrare che $\phi_{-x}(y)=-\phi_x(y)$. 
Fissato $x$ consideriamo quindi la funzione 
$f(y) = -\phi_x(y)$. 
Chiaramente $f$ è additiva ed è negativa se $x\ge 0$
o positiva (se $x\le 0$). 
Inoltre $f(1) = -x$ dunque, per l'unicità degli isomorfismi,
deve essere $f(y) = \phi_{-x}(y)$ che è quanto dovevamo dimostrare.

\mymargin{$1\cdot x=x$}
Sappiamo già che $x\cdot 1=x$, vogliamo però dimostrare 
che anche $1\cdot x=x$. 
Basta osservare che la funzione identità $f(x)=x$ 
è additiva e positiva e risulta $f(1)=1$.
Dunque dall'unicità dell'isomorfismo si deduce che $f = \phi_1$
e dunque $x=1\cdot x$.

\mymargin{$(x+y)\cdot z = x\cdot z + y\cdot z$}
Dimostriamo ora che $(x+y)\cdot z = x\cdot z + y\cdot z$.
Ciò equivale a dimostrare che $\phi_{x+y} = \phi_x + \phi_y$.
Consideriamo allora la funzione $f(z)=\phi_x(z) + \phi_y(z)$
e supponiamo inizialmente che siano $x,y\ge 0$.
Chiaramente $f$ è additiva e positiva
perché $\phi_x$ e $\phi_y$ lo sono.
Inoltre $f(1)=\phi_x(1) + \phi_y(1) = x+y$. 
Dunque per l'unicità dell'isomorfismo 
deve essere $f = \phi_{x+y}$ come volevamo dimostrare.

\mymargin{$x\cdot y = y\cdot x$}
Sempre supponendo che $x,y\ge 0$ 
dimostriamo ora la proprietà commutativa cioè
$\phi_x(y) = \phi_y(x)$. 
Fissato $y$ consideriamo allora la funzione $f(x) = \phi_x(y)$
e osserviamo che per quanto visto al passo precedente 
$f$ è additiva. 
Se $x\ge 0$ e $y\ge 0$ risulta anche che $f$ è positiva.
Inoltre $f(1) = y$ e dunque, per l'unicità dell'isomorfismo,
deve essere $f(x) = \phi_y(x)$ che è quanto volevamo dimostrare
con $x\ge 0$ e $y\ge 0$.
Utilizzando la regola del segno possiamo estendere 
la proprietà commutativa anche al caso $x\le 0$ e/o $y\le 0$.

\mymargin{$x\cdot(y\cdot z) = (x\cdot y)\cdot z$}
Per la proprietà associativa dobbiamo dimostrare che 
$\phi_x(y\cdot z) = \phi_{x\cdot y}(z)$.
Fissati $x,y\ge 0$ consideriamo 
la funzione $f(z) = \phi_x(y\cdot z)$.
Per le proprietà già dimostrate è chiaro 
che $f$, come al solito, è additiva e positiva
se $x,y\ge 0$.
Inoltre $f(1) = \phi_x(y) = x\cdot y$ 
dunque per l'unicità dell'isomorfismo si 
trova $\phi_x(y\cdot z) = \phi_{x\cdot y}(z)$
che è quanto dovevamo dimostrare se $x,y\ge 0$.
Usando la regola del segno la proprietà si estende 
anche al caso $x<0$ e/o $y<0$.

\mymargin{$\exists y\colon x\cdot y=1$}
Per l'esistenza del reciproco basta osservare 
che dato $x\neq 0$ la funzione $\phi_x$ 
(sempre per il teorema di isomorfismo) è 
bigettiva. Dunque esiste $y$ tale che $\phi_x(y)=1$
cioè $x\cdot y = 1$.

\mymargin{$x\cdot y = 0$}
Per la regola di annullamento del prodotto osserviamo che 
se $x\neq 0$ la funzione $\phi_x$ è bigettiva 
e dunque $\phi_x(y)=0$ deve essere $y=0$.
Di conseguenza si ottiene anche la stretta monotonia.
\end{proof}

Su $\RR$ abbiamo definito due operazioni: l'addizione e la moltiplicazione.
Le proprietà di queste operazioni ci dicono che $\RR$ 
è un campo ordinato in base alla seguente

\begin{definition}[campo]
Sia $R$ un insieme su cui sono definite due operazioni: l'addizione $+$ e la moltiplicazione $\cdot$.
Diremo che $R$ è un \emph{campo}
se valgono le seguenti proprietà:
\begin{enumerate}
  \item $R$ è un gruppo abeliano rispetto alla addizione e denotiamo con $0$ il suo elemento neutro;
  \item $R\setminus\ENCLOSE{0}$ è un gruppo abeliano rispetto alla moltiplicazione e,
  se denotiamo con $1$ l'elemento neutro si ha $1\neq 0$;
  \mynote{se non richiediamo la proprietà commutativa della moltiplicazione si dirà 
  che $R$ è un \emph{corpo}.}
  \item proprietà distributiva: $x\cdot(y+z) = x\cdot y + x\cdot z$.
\end{enumerate}

Diremo inoltre che $R$ 
è un \emph{campo ordinato}
se è definita una relazione $\le$ 
con le seguenti proprietà:
\begin{enumerate}
  \item[4.] la relazione d'ordine è totale;
  \item[5.] positività: se $x\ge 0$ e $y\ge 0$ allora $x+y\ge 0$ e $x\cdot y \ge 0$.  
\end{enumerate} 

Diremo infine che $R$ è un \emph{campo ordinato continuo} se è un campo ordinato 
e l'ordinamento è continuo.
\end{definition}

Ovviamente in un campo sono define, oltre all'addizione e la moltiplicazione anche 
la sottrazione e la divisione. 
Infatti la proprietà di essere un gruppo additivo garantisce che 
per ogni $x\in R$ esista l'opposto $-x$ tale che $x+(-x)=0$.
Si potrà quindi definire la sottrazione come la somma con l'opposto: 
$z-x \defeq z + (-x)$.
Analogamente essendo $R\setminus\ENCLOSE{0}$ un gruppo moltiplicativo,
se $x\neq 0$ esiste anche il reciproco $y=1/x$, 
ovvero un elemento tale che $x\cdot y = 1$.
Se $x\neq 0$ si può quindi definire la divisione per $x$ come 
il prodotto col reciproco: $z/x = \frac{z}{x} = z\cdot (1/x)$.

Se $n \in \NN$ è un numero naturale possiamo anche definire 
l'elevamento a potenza $x^n$ induttivamente: $x^0 = 1$,
$x^1 = x\cdot x^0 = x$, 
$x^2=x\cdot x^1 = x\cdot x$, 
$x^3 = x\cdot x^2 = x\cdot x \cdot x$\dots
Se $x\neq 0$ possiamo inoltre definire $x^{-n} \defeq 1/x^n$.
Vedremo nel prossimo paragrafo come l'elevamento a potenza $x^y$ può 
essere esteso in maniera naturale a tutti gli esponenti $y$ reali 
se la base $x$ è positiva. 

Le proprietà delle operazioni di un campo dovrebbero essere ben note 
a chi ha seguito un qualunque programma scolastico di matematica.
Tipicamente se abbiamo una equazione su un campo, o una disequazione 
su un campo ordinato, 
è utile sapere come l'equazione (o disequazione) può essere modificata per ottenere 
una equazione (o disequazione) equivalente. 
Ad esempio:
%
\begin{enumerate}
  \item possiamo aggiungere o sottrarre lo stesso valore ai due lati 
  di una uguaglianza o disuguaglianza;
  \item possiamo moltiplicare o dividere i due lati di una uguaglianza per qualunque
  valore diverso da zero; possiamo moltiplicare ambo i lati di 
  una disuguaglianza per un numero positivo; 
  possiamo moltiplicare ambo i lati di una disuguaglianza 
  per un numero negativo se poi invertiamo il verso della disuguaglianza.
\end{enumerate}

Queste regole sono casi particolari di una regola più generale. 
Per quanto riguarda le uguaglianze se abbiamo una qualunque 
funzione iniettiva 
$f\colon A\to B$ sappiamo che 
\[
  x = y \iff f(x) = f(y).   
\]
Si nota allora che la funzione $f(x) = x + c$, 
$f\colon R\to R$ 
è iniettiva se $R$ è un gruppo additivo.
Allo stesso modo se $c\neq 0$ la funzione $f(x)=c\cdot x$ 
$f\colon R\to R$,
è bigettiva 
se $R$ è un campo. 
Per questo motivo si ottengono le regole enunciate in precedenza.

Su un insieme totalmente ordinato, 
le funzioni che mantengono l'ordinamento:
\[
 x < y \iff f(x) < f(y), \qquad x\le y \iff f(x)\le f(x)
\]
sono, per definizione, le funzioni \emph{strettamente crescenti}.
Tali funzioni mantengono sia l'ordine stretto che l'ordine largo.
  
\begin{exercise}[punto medio]
Sia $R$ un campo ordinato e siano $x,y\in R$ con $x<y$.
Dimostrare che posto $z=\frac{x+y}{2}$ si ha $x<z<y$. 
\end{exercise}

\begin{exercise}
  Si dimostri che su un campo risulta
  \[
   x = y \iff x^3= y^3
  \]
  e su un campo ordinato
  \[
  x < y \iff x\cdot x \cdot x^3 < y^3.
  \]
  Si mostri, con un esempio, che invece
  \[
  \text{non valgono:} \qquad 
  x = y \iff x^2 = y^2,
  x < y \iff x^2 < y^2.
  \]
\end{exercise}

L'esercizio precedente ci dice che un campo ordinato è sempre \emph{denso}.
Dunque un campo ordinato continuo è in particolare un gruppo totalmente 
ordinato denso e continuo.
Dunque per il teorema di isomorfismo ogni campo ordinato e continuo è
isomorfo ad $\RR$.

\subsection{elevamento a potenza su $\RR$}

Il teorema di isomorfismo ci ha permesso di definire la moltiplicazione sui 
numeri reali. 
Lo stesso identico metodo ci permetterà di definire l'elevamento a potenza.
Notiamo infatti che l'insieme $\RR_+$ dei reali positivi
\[
  \RR_+ = \ENCLOSE{x\in \RR\colon x>0}
\]
risulta essere un gruppo moltiplicativo in quanto il prodotto di reali positivi
è positivo, il reciproco di un numero positivo è positivo 
e l'elemento neutro $1$ è anch'esso positivo.
Non solo, l'ordinamento di $\RR$ è ovviamente un ordinamento anche su $\RR_+$
e mantiene le proprietà di essere un ordinamento totale 
denso e continuo che non dipendono dalla struttura di gruppo.
Verifichiamo infine che l'ordinamento è compatibile con la moltiplicazione
cioè che se $x\ge 1$ e $y\ge 1$ allora $x\cdot y\ge 1$. 
Questa è una banale conseguenza della monotonia:
\[
 x\cdot y \ge x\cdot 1 \ge 1\cdot 1 = 1.  
\]

Possiamo quindi concludere che $\RR_+$ è un gruppo moltiplicativo denso e continuo.
Dunque fissato $a>0$ possiamo applicare il teorema di isomorfismo 
con $R=\RR$ gruppo additivo e $S=\RR_+$ gruppo moltiplicativo.
Si ottiene l'esistenza di una funzione $\phi_a\colon \RR \to \RR_+$ con le seguenti proprietà:
$\phi_a(1)=a$, $\phi_a(x+y) = \phi_a(x)\cdot \phi_a(y)$, $\phi_a$ è crescente se $a\ge 1$, 
è decrescente se $a\le 1$ ed è bigettiva se $a\neq 1$.

Definiamo $a^x = \phi_a(x)$ e lo chiamiamo \emph{elevamento a potenza}
con \emph{base} $a>0$ ed esponente $x\in \RR$.

\begin{theorem}[elevamento a potenza]
Dato $a\in \RR$, $a>0$ per ogni $x\in \RR$ si può definire in modo unico 
l'operazione di elevamento a potenza $a^x$ che abbia le seguenti proprietà:
\begin{enumerate}
  \item $a^1=a$;
  \item $a^{x+y} = a^x \cdot a^y$;
  \item $x\mapsto a^x$ è crescente se $a\ge 1$ 
  ed è decrescente se $a\le 1$.
\end{enumerate}

Inoltre l'esponenziale ha le seguenti proprietà, valide per $a,b>0$, $x,y\in \RR$:
\begin{enumerate}
  \item $a^0=1$;
  \item $a^{-1} = \frac{1}{a}$;
  \item $(a\cdot b)^x = a^x\cdot b^x$;
  \item $(a^x)^y = a^{x\cdot y}$;
\end{enumerate}
\end{theorem}
\begin{proof}
Abbiamo già visto come le prime tre proprietà definiscono in maniera 
univoca l'omomorfismo monotono $\phi_a\colon \RR\to \RR$ grazie 
al teorema di isomorfismo (teorema~\ref{th:isomorfismo}).
Inoltre l'omomorfismo manda sempre l'elemento neutro nell'elemento 
neutro dunque $a^0 = 1$.
\mynote{Ricordiamo che in partenza abbiamo il gruppo additivo $\RR$
con elemento neutro $0$ mentre in arrivo 
abbiamo il gruppo moltiplicativo $\RR_+$ con elemento neutro $1$.}

\mymargin{$a^{-1}=\frac 1 a$}
Per la proprietà di omomorfismo si ha $a^{x-x} = a^x \cdot a^{-x}$
da cui $a^{-x}$ risulta essere il reciproco di $a^x$ cioè 
$a^{-x}= 1/a^x$.

\mymargin{$(a\cdot b)^x = a^x\cdot b^x$}
Per la potenza del prodotto fissati $a,b\ge 1$ basta considerare la 
funzione $f(x) = a^x\cdot b^x$. 
Chiaramente $f$ è un omomorfismo in quanto $a^x$ e $a^y$ lo sono 
(e il prodotto è commutativo). 
Essendo $f(1) = a^1\cdot b^1 = a\cdot b$ 
per l'unicità dell'omomorfismo concludiamo che $f = \phi_{a\cdot b}$
che significa $a^x\cdot b^x = (a\cdot b)^x$.
Usando la regola del reciproco possiamo estendere questa proprietà 
quando $a<1$ e/o $b<1$ (ma sempre $a,b\ge 0$).

\mymargin{$(a^x)^y = a^{x\cdot y}$}
Infine per la potenza di potenza fissato $a\ge 1$ e $x\ge 0$ 
consideriamo la funzione $f(y) = a^{x\cdot y}$.
Per le proprietà precedenti si verifica facilmente che  
$f(y+z) = f(x)\cdot f(z)$. 
Inoltre se $a\ge 1$, $x\ge 0$ e $y\ge 0$ si ha 
$f(y)\ge 1$ che è la positività. 
Dunque per il teorema di isomorfismo,
essendo $f(1)=a^x$, si ottiene $f(y) = (a^x)^y$
che è quanto volevamo dimostrare.
La regola del reciproco estende questa proprietà 
ai casi $x\le 0$ e $a\le 1$ (sempre con $a\ge 0$).
\end{proof}

\subsection{l'insieme dei numeri naturali $\NN \subset \RR$}
\index{naturali!$\NN$ dentro $\RR$}%
\index{numeri!naturali $\NN$ dentro $\RR$}%
\index{$\NN$}%

Fin'ora abbiamo considerato gli insiemi $\NN$ 
e $\RR$ come insiemi scorrelati. 
I numeri naturali $0\in \NN$ e $1=\sigma(0)\in \NN$ 
sono in generale diversi dai numeri reali 
$0\in \RR$ (elemento neutro dell'addizione) 
e $1\in \RR$ (elemento neutro della moltiplicazione).

Sarebbe anche sensato tenere distinti questi due insiemi numerici 
ma in pratica risulta comodo identificare i numeri naturali 
come sottoinsieme dei numeri reali.
Su $\RR$ possiamo definire la funzione (traslazione a destra di una unità)
$\sigma\colon \RR\to\RR$, $\sigma(x) = x+1$.
Vorremmo quindi considerare l'insieme che si ottiene 
partendo da $0\in \RR$ e considerando tutti gli elementi che si
ottengono iterando la funzione $\sigma$: $1=\sigma(0)$, $2=\sigma(1)$, \dots

Formalmente questo si ottiene mediante una definizione:
diremo che un sottoinsieme $A\subset \RR$ è \emph{induttivo}
se soddisfa le seguenti proprietà:
\index{insieme!induttivo}%
\mymargin{insieme induttivo}%
\begin{enumerate}
  \item $0\in A$;
  \item $x\in A \implies x+1 \in A$.
\end{enumerate}

Possiamo allora definire i numeri naturali (dentro $\RR$) come 
il \emph{più piccolo} sottoinsieme induttivo di $\RR$:
\[
 \NN = \bigcap\ENCLOSE{A\subset \RR \colon \text{$A$ induttivo}}.  
\]
Ovviamente l'insieme $\RR$ stesso è un insieme induttivo, 
dunque la famiglia di insiemi su cui stiamo facendo l'intersezione,
non è vuota e dunque l'intersezione è ben definita.
Inoltre l'insieme risultante è a sua volta induttivo:
visto che $0$ è elemento di tutti gli insiemi induttivi, 
certamente $0$ deve stare nella loro intersezione; 
e se $x$ sta nell'intersezione di tutti gli insiemi induttivi 
allora anche $x+1$ sta in ogni insieme induttivo e quindi 
nella loro intersezione. 
La funzione $\sigma\colon \RR\to\RR$ è iniettiva e dunque 
possiamo affermare che le prime due proprietà degli assiomi 
di Peano (definizione~\ref{def:naturali}) sono soddisfatte.
Ma anche l'assioma di induzione è soddisfatto in quanto 
se prendiamo un qualunque sottoinsieme induttivo di $\NN$ 
questo non può che essere uguale ad $\NN$ perché $\NN$, 
per definizione di intersezione, è contenuto in ogni 
sottoinsieme induttivo di $\RR$ (e quindi in ogni 
sottoinsieme induttivo di se stesso).

\subsection{l'insieme dei numeri interi $\ZZ \subset \RR$}
\index{interi!$\ZZ$ dentro $\RR$}%
\index{numeri!interi $\ZZ$ dentro $\RR$}%
\index{$\ZZ$}%

Avendo posto $\NN\subset \RR$ possiamo definire facilmente i numeri interi
$\ZZ \subset \RR$ 
ponendo $\ZZ = \NN - \NN$
cioè l'insieme di tutte le possibili somme tra i numeri naturali 
e i loro opposti. Si può facilmente verificare che 
\[
 \ZZ = \NN \cup (-\NN)  
\]
perché ogni differenza $n-m$ di numeri naturali 
è un numero naturale se $n\ge m$ ed è l'opposto 
di un numero naturale $n-m = -(m-n)$ se $n\le m$.
Dunque $\ZZ$ si ottiene aggiungendo ai numeri naturali 
$n\in \NN\subset \RR$ i loro opposti $-n\in \RR$.

In questo modo $\ZZ$ risulta essere un gruppo additivo
rispetto alla operazione di addizione ereditata da $\RR$.
Infatti la somma di due numeri in $\ZZ$ è ancora 
un numero in $\ZZ$:
\[
  (n-m) + (n'-m') = (n+n') - (m+m')
  \qquad\text{($n,m,n,m'\in \NN)$.}
\]

L'insieme numerico $\ZZ$ eredita da $\RR$ anche 
l'ordinamento e tale ordinamento è compatibile con l'addizione 
perché lo era in $\RR$. Dunque $\ZZ$ risulta essere un gruppo 
totalmente ordinato.
Però $\ZZ$ non è denso perché si può dimostrare che 
non esistono numeri interi strettamente compresi tra $0$ e $1$.
\mynote{
Se $0<x<1$ e per assurdo fosse $x\in \ZZ$ 
allora sarebbe $x\in \NN\setminus\ENCLOSE{0}$ in quanto $x>0$.
Ma $\NN\setminus\ENCLOSE{0} = \NN + 1$ quindi 
si avrebbe $x-1\in \NN$ che è assurdo in quanto $x-1<0$
se $x<1$.}
Si potrebbe anche dimostrare che $\ZZ$ è continuo (esercizio!),
fornendo dunque un esempio di gruppo totalmente ordinato, 
denso ma non continuo.

\subsection{l'insieme dei numeri razionali $\QQ\subset \RR$}
\index{razionali!$\QQ$ dentro $\RR$}%
\index{numeri!razionali $\QQ$ dentro $\RR$}%
\index{$\QQ$}%

Per ogni $m\in \ZZ$ ed ogni $n\in \ZZ$ se $q\neq 0$ è ben
definito il quoziente $\frac m n\in \RR$.
Definiamo allora l'insieme $\QQ\subset \RR$ di tutte le frazioni:
\[
    \QQ = \frac{\ZZ}{\ZZ\setminus\ENCLOSE{0}}.
\]
Si può verificare che $\QQ$ è un gruppo additivo in quanto 
la somma di frazioni è ancora una frazione:
\[
  \frac{m}{n} + \frac{m'}{n'}
  = \frac{mn'}{nn'} + \frac{m'n}{nn'} 
  = \frac{mn'+nm'}{n n'}
\]
e, ovviamente, l'opposto di una frazione è una frazione: 
$- \frac m n = \frac{-m}{n}$. 
Inoltre $\QQ\setminus \ENCLOSE{0}$ è un gruppo 
additivo in quanto anche il prodotto di due frazioni 
è ovviamente una frazione, e lo stesso vale per il reciproco di frazioni 
non nulle:
\[
  \frac{m}{n}\cdot \frac{m'}{n'} = \frac{m\cdot m'}{n\cdot n'},
  \qquad
  \frac{1}{\frac m n} = \frac n m.  
\]
L'ordinamento di $\RR$ viene ereditato da $\QQ$ ed è quindi ovviamente 
compatibile con le operazioni di addizione e moltiplicazione. 
Risulta quindi che $\QQ$ è un esempio di campo ordinato, come $\RR$.
Ovviamente $\QQ$ è denso come ogni campo ordinato.

Se l'ordinamento di $\QQ$ fosse continuo il teorema 
di isomorfismo ci direbbe 
che c'è un unico omomorfismo positivo $f\colon \QQ \to \RR$ 
tale che $f(1)=1$. 
Ma l'inclusione $x\mapsto x$ è anch'esso un omomorfismo positivo 
di $\QQ \to \RR$ e dunque dovrebbe coincidere con $f$.
Ma il teorema ci dice anche che tale $f$ è bigettiva dunque 
si avrebbe $\RR = f(\QQ) = \QQ$.
E' rilevante osservare che questo non accade e cioè 
$\QQ\neq \RR$ perché $\QQ$ non è continuo. 
Il primo ad accorgersene fu, per quanto ne sappiamo,
Pitagora che dimostrò che $\sqrt 2 = 2^{\frac 1 2}$
non è un numero razionale, dunque $\sqrt 2 \in \RR \setminus \QQ$.
Ovviamente $(\sqrt 2)^2 = (2^\frac 1 2)^2 = 2^1 = 2$ 
e dunque $\sqrt 2$ è una soluzione dell'equazione $x^2=2$.
Possiamo dunque enunciare l'irrazionalità di $\sqrt 2$ 
senza tirare in ballo i numeri reali, 
dimostrando che l'equazione $x^2=2$ non ha soluzioni in $\QQ$.

\begin{theorem}[Pitagora, irrazionalità di $\sqrt 2$]
  \mymark{**}%
  \label{th:pitagora}%
  L'equazione $x^2=2$ non ha soluzioni in $\QQ$.
  \end{theorem}
  %
  \begin{proof}
  \mymark{*}%
  Supponiamo $x\in \QQ$ sia una soluzione di $x^2=2$.
  Allora si potrà scrivere $x=p/q$ con $p\in \ZZ$ e $q\in \NN$, $q\neq 0$.
  Possiamo anche supporre che la frazione $p/q$ sia ridotta ai minimi
  termini cioè che $p$ e $q$ non abbiano fattori in comune.
  Moltiplicando l'equazione
  $(p/q)^2=2$ per $q^2$ si ottiene $p^2 = 2 q^2$.
  Risulta quindi che $p^2$ è pari.
  Ma allora anche $p$ è pari (perché il quadrato di un dispari è dispari).
  Ma se $p$ è pari allora $p^2$ è multiplo di quattro.
  Ma allora anche $2q^2$ è multiplo di quattro e quindi $q^2$ è pari.
  Dunque anche $q$ è pari. Ma avevamo supposto che $p$ e $q$ non avessero
  fattori in comune quindi questo non può accadere.
\end{proof}

Abbiamo quindi dimostrato che $\QQ$ non può essere continuo (altrimenti 
sarebbe isomorfo a $\RR$ dove l'equazione $x^2=2$ ha soluzione). 
Possiamo anche esplicitare un esempio di sottoinsiemi di $\QQ$ che sono 
\emph{separati} (nel senso della definizione~\ref{def:ordinamento_continuo})
ma non hanno elemento di separazione in $\QQ$:
\[
A= \ENCLOSE{x\in\QQ\colon x^2< 2},
\qquad
B = \ENCLOSE{x\in \QQ\colon x<0, x^2>2}.  
\]
Chiaramente $A\le B$ ma in $\RR$ questi due insiemi 
hanno come unico elemento di separazione $\sqrt 2$ che non è elemento di
$\QQ$.

\subsection{estremo superiore, estremo inferiore}

\begin{definition}[maggiorante, minorante, massimo, minimo, limitato]
  \label{def:minorante}%
  \label{def:maggiorante}%
  \label{def:minimo}%
  \label{def:limitato}%
  % Sia $\le$ una relazione d'ordine su un insieme $X$ e siano 
  Siano $x\in \RR$ e $A\subset \RR$.

  Diremo che $x$ è un \emph{minorante}%
\mymargin{minorante}%
\index{minorante} di $A$ e scriveremo:
  \[
    x \le A
  \]
  se per ogni $a\in A$ risulta $x\le a$. 
  Se esiste $x$ per cui $x\le A$ diremo che $A$ 
  è \emph{inferiormente limitato}.
  Se $x$ è un minorante e inoltre $x\in A$ diremo 
  che $x$ è il \emph{minimo}%
\mymargin{minimo}%
\index{minimo} di $A$ e scriveremo:
  \[
    x = \min A.  
  \] 

  Analogamente diremo che $x$ è un \emph{maggiorante}%
\mymargin{maggiorante}%
\index{maggiorante}
  di $A$ e scriveremo $x \ge A$ se $x\ge a$ per ogni $a\in A$,
  diremo che $A$ è superiormente limitato 
  se esiste $x$ per cui $x \ge A$ infine
  diremo che $x$ è il \emph{massimo}%
\mymargin{massimo}%
\index{massimo} di $A$ 
  e scriveremo $x=\max A$ se $x\ge A$ e $x\in A$.

  Se $A$ è superiormente limitato e inferiormente limitato
  diremo che $A$ è \emph{limitato}.

  Se $A$ non è limitato diremo che $A$ è \emph{illimitato}.
\end{definition}

Massimo e minimo di un insieme $A$, se esistono, sono unici.
Infatti se $x$ e $y$ fossero due minimi di $A$ si avrebbe $x\le y$ in
quanto $x\le A$ e $y\in A$. Analogamente si avrebbe $y\le x$ e
quindi $x=y$. Ragionamento analogo se $x$ e $y$ fossero due massimi.

\mynote{
  Se $A$ è un insieme finito il massimo e il minimo 
  esistono sempre. 
  Se $A$ non avesse minimo potrei definire per induzione 
  una funzione $f\colon \NN\to A$ tale che $f(0)\in A$ 
  è qualunque e $f(n+1) < f(n)$ (visto che $f(n)$ non è un minimo 
  di $A$ esiste un numero più piccolo). 
  Chiaramente $f$ è iniettiva quindi $\#A\ge \#\NN$
  e dunque $A$ è infinito.
}

\mynote{
  Non si confonda il concetto di \emph{insieme infinito} 
  con quello di \emph{insieme illimitato}.
  Un insieme infinito può essere illimitato: si 
  prenda ad esempio l'insieme $A=\ENCLOSE{x\in \RR\colon 0\le x \le 1}$
  che è chiaramente limitato ma è infinito perché contiene,
  ad esempio, tutti i numeri $\frac{1}{n}$ con $n\in\NN$.
}

\begin{definition}%
  \mymark{***}%
  Sia $A \subset \RR$.
  Se $A$ ha un maggiorante (cioè esiste $x\in \RR$ tale che 
  $x\ge A$, definizione~\ref{def:minorante})
  diremo che $A$ è \emph{superiormente limitato},
  se $A$ ammette un minorante diremo che $A$ è \emph{inferiormente limitato},
  \index{superiormente!limitato}%
  \index{inferiormente!limitato}%
  \index{limitato!superiormente}%
  \index{limitato!inferiormente}%
  se $A$ ammette sia maggiorante che minorante diremo che $A$ è 
  \emph{limitato}%
\mymargin{limitato}%
\index{limitato}.
  
  Se $x$ è minimo dei maggioranti di $A$ diremo che $x$ è
  \emph{estremo superiore}%
\mymargin{estremo superiore}%
\index{estremo!superiore}
  di $A$ se invece $x$ è massimo dei minoranti diremo che $x$ è
  \emph{estremo inferiore}%
\mymargin{estremo inferiore}%
\index{estremo!inferiore} di $A$:
  \begin{align*}
  \sup A &= \min \ENCLOSE{x\in \RR \colon x\ge A}, \\
  \inf A &= \max \ENCLOSE{x\in \RR \colon x \le A}.
  \end{align*}
  \index{$\sup$}%
  \index{$\inf$}%
  \index{sup}%
  \index{inf}%
\end{definition}

Osserviamo che se l'insieme $A$ è finito e non vuoto,
il massimo e il minimo esistono sempre.
Se, ad esempio, non esistesse il massimo di $A$ significa che scelto
$x_k\in A$ esisterebbe sempre $x_{k+1}\in A$ con $x_{k+1} > x_k$ e quindi l'insieme
$A$ dovrebbe contenere infiniti punti $x_0,x_1, \dots, x_k,\dots $
Insiemi infiniti, invece, potrebbero non avere massimo/minimo.
Un esempio di insieme che non ha né massimo né minimo è
l'insieme $A = \ENCLOSE{x\in \RR\colon 0<x<1}$ infatti per ogni
$x\in A$ si ha $\frac x 2<x$, $\frac{1+x}{2}>x$
con $\frac x 2\in A$ e $\frac{1+x}{2}\in A$,
dunque nessun $x\in A$ può essere
massimo o minimo. L'insieme $B=\ENCLOSE{x\in \RR \colon 0\le x \le 1}$
ha invece massimo $\max B= 1$ e minimo $\min B=0$.

\begin{theorem}[esistenza del $\sup$]%
  \label{th:sup}%
  \mymark{**}%
  Se $A\subset \RR$ è un insieme non vuoto
  e superiormente limitato, allora esiste l'estremo superiore di $A$.
  Se $A\subset \RR$ è un insieme non vuoto e inferiormente limitato 
  allora esiste l'estremo inferiore di $A$.
  \end{theorem}
  %
  \begin{proof}
  \mymark{*}
  Consideriamo l'insieme dei maggioranti
  \[
  B = \ENCLOSE{ b\in \RR \colon b \ge A}.
  \]
  Per ipotesi $B$ è non vuoto e per come è definito risulta $A\le B$.
  Dunque dall'assioma di continuità (definizione~\ref{def:ordinamento_continuo}) 
  deduciamo l'esistenza di un numero $x\in \RR$
  tale che $A\le x \le B$. La prima disuguaglianza $A\le x$ ci dice che $x$ è un
  maggiorante e quindi $x\in B$, la seconda $x\le B$ ci dice che $x$ è il minimo
  di $B$ e quindi concludiamo che $x$ è il minimo dei maggioranti 
  ovvero l'estremo superiore di $A$.

  Dimostrazione analoga vale per l'estremo inferiore.
\end{proof}

L'esistenza del $\sup$ (o dell'$\inf$)
è in effetti una condizione equivalente all'assioma di continuità.
Infatti se $A\le B$ certamente $\sup A$, se esiste, è elemento 
di separazione tra $A$ e $B$.

\subsection{parte intera}

Se $R$ è un gruppo con operazione $+$ ed elemento neutro $0$, 
allora per ogni $x\in R$ si può definire 
il prodotto $n\cdot x$ con $n\in \NN$ come somma ripetuta:
Si pone $0\cdot x=0$ e per induzione $(n+1)\cdot x= n\cdot x + x$.
 
\begin{theorem}[parte intera]
\mymark{*}%
  Dato $x\in \RR$ esiste un unico $m\in \ZZ$ tale che $m-1 < x \le m$.
\end{theorem}
%
\begin{proof}
  Supponiamo per un attimo che sia $x > 0$
  e consideriamo l'insieme $A=\ENCLOSE{n \in \NN \colon n\ge x}$.
  Tale insieme non è vuoto per la proprietà archimedea 
  e dunque ammette minimo per il principio del buon ordinamento.
  Se $m=\min A$ risulta quindi $m\in \NN$ e $m-1< x \le m$.

  Se $x\le 0$ per la proprietà archimedea esiste $k\in \NN$ tale che 
  $k>-x$. Allora applichiamo il risultato precedente a $x+k$ e consideriamo 
  $m-k$ al posto di $m$.
\end{proof}

\begin{definition}[parte intera]
  \mymark{**}%
  \mymargin{parte intera}%
\index{parte!intera}%
  Dato $x\in \RR$ denotiamo con $\lfloor x\rfloor$ l'unico intero
  che soddisfa
  \mymargin{$\lfloor\cdot\rfloor$} %% *** non viene bene nell'indice!
  \[
    x - 1 < \lfloor x \rfloor \le x
  \]
  e denotiamo con $\lceil x \rceil = - \lfloor -x \rfloor$ l'unico intero che soddisfa (verificare!)
  \mymargin{$\lceil\cdot\rceil$} %% *** non viene bene nell'indice!
  \[
    x \le \lceil x \rceil < x + 1.
  \]
  Si ha dunque
  \[
    \lfloor x \rfloor \le x \le \lceil x \rceil
  \]
  con entrambe le uguaglianze che si realizzano quando $x\in \ZZ$.
  I due interi $\lfloor x \rfloor$ e $\lceil x \rceil$
  sono la migliore approssimazione intera di $x$ rispettivamente
  per difetto e per eccesso.
  L'intero più vicino ad $x$ (approssimazione per \emph{arrotondamento}%
\mymargin{arrotondamento}%
\index{arrotondamento})
  è
  \[
    \left\lfloor x + \frac 1 2 \right\rfloor
  \quad \text{ossia} \quad
    \left\lceil x-\frac 1 2 \right\rceil
  \]
  (le due espressioni differiscono solamente quando $x$ si trova nel punto medio tra 
  due interi consecutivi, nel qual caso la prima approssima per eccesso e la seconda 
  per difetto).
\end{definition}

In alcuni testi si usa la notazione $[x]$ per denotare la parte intera $\lfloor x \rfloor$ e si definisce
anche la \emph{parte frazionaria}
\[
  \ENCLOSE{x} = x - [x].
\]
Per evitare ambiguità con il normale utilizzo delle parentesi
non useremo queste notazioni.

\subsection{numeri decimali}
%
Le frazioni il cui denominatore è una potenza
di $10$ si chiamano frazioni decimali:
\[
  x = \frac{p}{10^d}, \qquad p\in \ZZ, d\in \NN.
\]
Tali frazioni si possono rappresentare
scrivendo il numero
intero $p$ e segnando un punto
\mynote{%
in Italia si preferisce utilizzare la virgola, ma
ci rassegnamo alla notazione anglosassone che ormai è
ubiqua in tutta la strumentazione elettronica.
}%
di separazione
prima della $d$-esima cifra a partire da destra.
Ad esempio scriveremo:
\[
  1.4142 = \frac{14142}{10^4}.
\]
In generale una frazione $\frac{p}{q}\in \QQ$
può essere scritta in forma decimale solamente
se, quando ridotta ai minimi termini,
risulta che $q$ non ha fattori primi diversi
da $2$ e $5$ (in quanto le potenze di dieci
hanno solo questi fattori).
In ogni caso le frazioni decimali sono dense in $\RR$
in quanto dato $x\in \RR$ per ogni $\eps>0$ esiste
$d\in \NN$ tale che $10^{-d}\le\eps$ e quindi
posto $p=\lfloor 10^d\cdot x + \frac 1 2\rfloor$
si avrà
\[
    \abs{\frac{p}{10^d} - x} \le \frac{1}{2\cdot 10^d} < \eps.
\]
In tal caso scriveremo
\mymargin{$\approx$}%
\index{$\approx$}
\[
  x \approx \frac{p}{10^d}.
\]
Ad esempio scriveremo
\[
  \frac 2 3 \approx 0.6667 = \frac{6667}{10^4}
\]
per intendere%
\mynote{%
Si osservi che in base alla definizione data sarebbe anche corretto 
scrivere $\frac 2 3 \approx 0.6666$ che però sarebbe una approssimazione 
peggiore. 
Questa ambiguità è necessaria se vogliamo evitare i casi 
limite in cui bisogna conoscere molte più cifre decimali di quelle richieste 
per capire qual è la migliore approssimazione.
}%
\begin{equation}\label{eq:approx_23}
\abs{\frac 2 3 - \frac{6667}{10^4}}<\frac{1}{10^4}.
\end{equation}
Nel calcolo scientifico ogni uguaglianza numerica è intesa nel 
senso precedente, se non specificato diversamente. 

Le frazioni non decimali si possono scrivere con uno sviluppo
decimale \emph{periodico}. 
Non useremo mai questa notazione
che ricordiamo solamente con un esempio.
Il numero
\[
  x = 12.34\overline{567}
    = 12.34567\overline{567}
\]
è la frazione che risolve l'equazione
\[
  \frac{100x - 1234}{1000}
  = 100x-1234.567
  \qquad
\enclose
{\frac{0.\overline{567}}{1000}
= 0.000\overline{567} }
\]
ovvero
\[
  1234567 - 1234 = 99900 \cdot x,
  \qquad x = \frac{1234567-1234}{99900}.
\]

\begin{exercise}
Dimostrare che vale~\eqref{eq:approx_23}.
\end{exercise}  

\subsection{valore assoluto}

\begin{definition}[valore assoluto]
\mymark{***}
Se $\KK$ è un campo totalmente ordinato (ad esempio $\KK=\QQ$ o, come vedremo, $\KK=\RR$)
definiamo il \emph{valore assoluto}%
\mymargin{valore assoluto}%
\index{valore!assoluto} $\abs{x}$ di un numero $x\in \KK$ nel seguente modo:
\[
\abs{x} =
\begin{cases}
  x & \text{se $x\ge 0$}, \\
  -x & \text{se $x<0 $}.
\end{cases}
\]
\end{definition}
  
\begin{proposition}[proprietà del valore assoluto]
\mymark{**}
Si ha
\begin{enumerate}
\item $\abs{x}\ge 0$ (positività)
\item $\big\lvert\abs{x}\big\rvert = \abs{x}$ (idempotenza)
\item $\abs{-x} = \abs{x}$ (simmetria)
\item $\abs{x\cdot y} = \abs{x}\cdot \abs{y}$ (omogenità)
\item $\abs{x+y} \le \abs{x} + \abs{y}$ (convessità)
\item $\abs{x-y} \le \abs{x-z} + \abs{z-y}$ (disuguaglianza triangolare)
\item $\big\lvert\abs{x}-\abs{y}\big\rvert \le \abs{x-y}$ (disuguaglianza triangolare inversa)
\end{enumerate}
Useremo inoltre spesso la seguente equivalenza (valida
anche con $<$ al posto di $\le$). Se $r\ge 0$ allora
\[
  \abs{x-y} \le r
  \iff
  y - r \le x \le y + r.
\]
\end{proposition}
%
\begin{proof}
\mymark{*}
Positività, idempotenza, simmetria e omogenità sono immediate conseguenze della definizione.

Dimostriamo ora l'ultima osservazione.
Se $x\ge y$ allora $x-y\ge 0$ e quindi $\abs{x-y} \le r$ è
equivalente a $x-y\le r$ cioè $x\le y+r$.
Se $x<y$ allora $x-y<0$ e quindi $\abs{x-y} \le r$ è
equivalente a $y-x \le r$ cioè $x\ge y-r$.
Viceversa se $y-r \le x \le y+r$ allora vale sia $x-y \le r$ che $y-x \le r$ e 
dunque $\abs{x-y}\le r$.

Osserviamo allora che per la precedente osservazione applicata
a $\abs{x-0} \le \abs{x}$ si ottiene
\[
  -\abs{x} \le x \le \abs{x}
\]
e sommando la stessa disuguaglianza con $y$ al posto di $x$ si
ottiene
\[
  -(\abs{x} + \abs{y}) \le x + y \le \abs{x} + \abs{y}
\]
che è equivalente alla proprietà di convessità:
\[
  \abs{x+y} \le \big\lvert\abs{x} + \abs{y}\big\rvert = \abs{x} + \abs{y}.
\]

Ponendo $y=z-x$ nella disuguaglianza precedente, si ottiene
\[
  \abs{z} \le \abs{x} + \abs{z-x}
\]
da cui
\[
  \abs{z} - \abs{x} \le \abs{z-x}.
\]
Scambiando $z$ con $x$ si ottiene la disuguaglianza opposta
e mettendole assieme si ottiene
la disuguaglianza triangolare inversa:
\[
\big\lvert \abs{z}-\abs{x} \big\rvert  \le \abs{z-x}.
\]

La disuguaglianza triangolare segue dalla convessità:
\[
  \abs{x-y} = \abs{x-z + z-y} \le \abs{x-z} + \abs{z-y}.
\]
\end{proof}

Osserviamo che dal punto di vista geometrico
$\abs{x-y}$ rappresenta la \emph{distanza} tra i punti
$x$ e $y$.

\section{costruzione dei numeri naturali}

\begin{theorem}[esistenza dei numeri naturali]
  \label{th:esistenza_naturali}%
Se $X$ è un qualunque insieme infinito (definizione\ref{def:infinito})
esiste $\NN\subset X$, $0\in \NN$ e $\sigma\colon \NN\to\NN$ 
che soddisfano gli assiomi di Peano~\ref{def:naturali}.

Dunque dall'assioma~\ref{axiom:infinito} possiamo dedurre l'esistenza 
di un insieme $\NN$ di numeri naturali.
\end{theorem}
%
\begin{proof}
Se $X$ è infinito esiste $f\colon X\to X$ iniettiva ma non suriettiva. 
Scegliamo arbitrariamente $0\in X\setminus f(X)$. 
Preso un sottoinsieme $I\subset X$ diremo che $I$ è \emph{induttivo}%
\mymargin{induttivo}%
\index{induttivo}%
se $0\in I$ e se $n\in I\implies f(n)\in I$. 
Possiamo quindi definire:
\[
  \NN = \bigcap \ENCLOSE{I\subset X\colon \text{$I$ induttivo}}.
\]
Si verifica facilmente che $\NN$ è anch'esso un sottoinsieme induttivo di $X$.
Dunque per ogni $n\in \NN$ si ha $f(n)\in \NN$ e quindi possiamo 
definire $\sigma\colon \NN\to \NN$ come la restrizione di $f$ 
ad $\NN$: $\sigma(n)=f(n)$.

Osserviamo quindi che $\sigma$ soddisfa gli assiomi di Peano.
Il primo assioma è conseguenza dell'iniettività di $f$.
Il secondo è verificato per come abbiamo scelto $0$.
Per verificare il terzo assioma consideriamo un qualunque insieme 
$A\subset \NN$ tale che $0\in A$ e tale che se $n\in A$ anche $n+1\in A$.
Per definizione $A$ è induttivo e quindi certamente 
$\NN\subset A$ visto che $\NN$, per come è definito,
è sottoinsieme di ogni insieme induttivo.
\end{proof}

La costruzione precedente ci dice che l'insieme $\NN$ è il più piccolo insieme infinito
nel senso che: se $X$ è infinito allora $\# X \ge \# \NN$. 
\mynote{Questa osservazione ci dice che se $\NN$ e $\NN'$ sono due diversi insiemi di 
numeri naturali allora $\#\NN\le \#\NN'$ e $\#\NN'\le\#\NN$ per cui, 
grazie al teorema~\ref{th:cantor_bernstein}, $\# \NN = \# \NN'$.
Nel teorema~\ref{th:unicitaN} vedremo che non solo esiste una corrispondenza biunivoca 
tra $\NN$ e $\NN'$ ma che esiste una corrispondenza che preserva la struttura (cioè l'operazione 
$\sigma$). 
}


%% \begin{comment} % DOVE SI USA?
%%   \begin{lemma}
%%     Per ogni $n\in \NN$ si ha $\sigma(n)\neq n$.
%%     \end{lemma}
%%     \begin{proof}
%%       Lo si dimostra per induzione. Per $n=0$ sappiamo che $\sigma(0)\neq 0$ 
%%       in quanto zero non è successore di nessun numero naturale.
%%       Se ora supponiamo di sapere che $\sigma(n)\neq n$ sapendo che 
%%       $\sigma$ è iniettiva possiamo dedurre $\sigma(\sigma(n))\neq \sigma(n)$ 
%%       che è proprio il passo induttivo.
%%   \end{proof}    
%% \end{comment}

\begin{theorem}[definizione per induzione]
  \label{th:induzione}%
  Sia $X$ un insieme, sia $\alpha\in X$ e sia $g\colon X\to X$ una funzione.
  Allora esiste una unica funzione $f\colon \NN \to X$ tale che
  \begin{equation}\label{eq:4835628}
    \begin{cases}
      f(0) = \alpha, \\
      f(\sigma(n)) = g(f(n)).
    \end{cases}
  \end{equation}
  Si avrà dunque
  \[
    f(0) = \alpha,\quad
    f(1) = g(\alpha),\quad
    f(2) = g(g(\alpha)),\quad
    f(3) = g(g(g(\alpha)))\dots
  \]
  Più in generale se abbiamo $\alpha\in X$ e una funzione $g\colon \NN \times X \to X$
  esisterà una unica funzione $f\colon \NN \to X$ tale che
  %
  \begin{equation}
    \begin{cases}
      f(0) = \alpha, \\
      f(\sigma(n)) = g(n, f(n)).
    \end{cases}
  \end{equation}
\end{theorem}
%
\begin{proof}
Dobbiamo ricordarci che le funzioni $f\colon \NN \to X$ non sono altro che relazioni 
e cioè sottoinsiemi del prodotto $\NN\times X$.
L'idea è quindi di prendere il più piccolo sottoinsieme di $\NN\times X$ 
che possa rappresentare una funzione con le proprietà richieste.
Consideriamo dunque la famiglia di insiemi:
\[
\mathcal F = \ENCLOSE{F\in \mathcal P(\NN\times X)\colon 
  (0,\alpha)\in F,\quad (n,x)\in F \Rightarrow (\sigma(n),g(x))\in F}.
\]
Chiaramente $\mathcal F$ non è vuota in quanto $\NN\times X \in \mathcal F$.
Possiamo dunque farne l'intersezione e definire un insieme $f$:
\[
  f = \bigcap_{F\in \mathcal F} F.
\]
L'insieme $f$ che abbiamo definito rappresenta una relazione tra $\NN$ e $X$.
Visto che $(0,\alpha)\in F$ per ogni $F\in \mathcal F$ dovrà essere 
$(0,\alpha)\in f$.
Inoltre se $(n,x)\in f$ allora $(n,x)\in F$ per ogni $F\in \mathcal F$ 
e quindi $(\sigma(n),g(x))\in F$ per ogni $F\in \mathcal F$
da cui $(\sigma(n),g(x))\in f$. Significa che $f\in \mathcal F$.

Vogliamo ora dimostrare che $f$ è una funzione, cioè che è univocamente definita 
su tutto $\NN$.
Per prima cosa consideriamo l'insieme su cui $f$ è definita 
e cioè $A=\ENCLOSE{n\in \NN\colon \exists x\in X\colon (n,x)\in f}$
e dimostriamo, per induzione, che $A=\NN$.
In effetti $(0,\alpha)\in f$ quindi $0\in A$. 
E se $n\in A$ sappiamo che esiste $x\in X$ tale che $(n,x)\in f$ 
e dunque, essendo $f\in \mathcal F$, anche $(\sigma(n),g(x))\in f$
da cui $\sigma(n)\in A$. 
Abbiamo dimostrato che $f$ è definita su tutto $\NN$.

Dimostriamo ora che $f$ è univoca. Consideriamo 
l'insieme su cui $f$ è univocamente definita: 
$B=\ENCLOSE{n\in \NN\colon \exists! x\in X\colon (n,x)\in f}$.
Di nuovo vogliamo dimostrare per induzione che $B=\NN$. 
Per dimostrare che $0\in B$, visto che già sappiamo che $(0,\alpha)\in f$, 
dobbiamo dimostrare che se $x\neq \alpha$ si ha $(0,x)\not\in f$.
Sia dunque $x\neq \alpha$ e consideriamo 
l'insieme $F=f\setminus\ENCLOSE{(0,x)}$.
Chiaramente $F\in \mathcal F$ in quanto $(0,\alpha)\in F$
visto che $(0,\alpha)\in f$ e $(0,\alpha)\neq (0,x)$
inoltre se $(n,y)\in F$ allora $(n,y)\in f$ 
e quindi $(\sigma(n),g(y)) \in f$.
Ma certamente $(\sigma(n),g(y))\neq (0,x)$ in quanto $\sigma(n)\neq 0$ 
dunque $(\sigma(n),g(y))\in F$.
Visto che $F\in \mathcal F$ si deve avere $f\subset F$ e dunque 
$(0,x)\not \in f$. Dunque $f$ è univocamente definita in $0$.

Dobbiamo ora mostrare che se $n\in B$ anche $\sigma(n)\in B$.
Se $n\in B$ significa che c'è un unico $x\in X$ tale che $(n,x)\in f$
e certamente anche $(\sigma(n),g(x))\in f$.
Prendiamo allora $y\neq g(x)$, vorremo dimostrare che $(\sigma(n),y)\not \in f$.
Consideriamo, similmente a prima, l'insieme $F=f\setminus\ENCLOSE{(\sigma(n),y)}$
e cerchiamo di dimostrare che $F\in \mathcal F$.
Chiaramente $(0,\alpha)\in F$ perché $(0,\alpha)\in f$ 
e non può essere $(0,\alpha)=(\sigma(n),y)$ in quanto $\sigma(n)\neq 0$.
Se ora supponiamo che sia $(m,z)\in F$ certamente sarà $(m,z)\in f$ 
e dunque $(\sigma(m),g(z))\in f$: 
dobbiamo mostrare che $(\sigma(m),g(z))\in F$. 
D'altra parte se fosse $(\sigma(m), g(z))=(\sigma(n),y)$ 
dovrebbe essere $m=n$ in quanto $\sigma$ è iniettiva. 
Ma visto che $f$ è univocamente definita su $n$ dovrà allora essere 
anche $(n,z) = (n,x)$ e quindi $g(z)=g(x) \neq y$. 
Dunque $(\sigma(m),g(z))\in F$ e $F\in \mathcal F$.
Ma allora $f\subset F$ e quindi $(\sigma(n),y)\not \in f$.
Significa che $f$ è univocamente definita anche in $\sigma(n)$.
Per induzione $B=\NN$ ed $f$ è una funzione $f\colon \NN\to X$.

Ovviamente visto che $f\in \mathcal F$ sappiamo che $f$ 
soddisfa le proprietà richieste dal teorema.

Nella seconda parte del teorema, dove $g\colon \NN\times X \to X$,
possiamo considerare l'insieme $Y=\NN\times X$ e la funzione 
$G\colon Y\to Y$ definita da $G(n,x) = (\sigma(n), g(n,x))$.
Allora applicando la prima parte possiamo trovare $F\colon \NN\to Y$
tale che $F(0) = (0,\alpha)$ e $F(\sigma(n)) = G(F(n))$.
Basterà prendere come $f(n)$ la seconda componente di $G(n)$.
\end{proof}
  
In particolare il teorema precedente ci permette di definire l'iterata $n$-esima $f^n$ 
di una qualunque funzione $f\colon X\to X$.
Se $\id_X\colon X\to X$ rappresenta la funzione identità su $X$:
$\id_X(x)=x$,
possiamo definire:
\index{iterazioni}%
\index{funzione!iterata}%
\index{funzione!composta}%
\begin{equation}\label{def:iterata}
  \begin{cases}
    f^0 = id_X,\\
    f^{\sigma(n)} = f\circ f^n.
  \end{cases}
\end{equation}

Le iterate della funzione $\sigma$ ci permettono di definire l'addizione:
\begin{equation}\label{def:addizione}
  n+k = \sigma^k(n).  
\end{equation}
Definita in questo modo l'addizione $+$ è una funzione 
che ad ogni $k\in \NN$ associa la funzione $\sigma^k$
che a sua volta ad ogni $n\in \NN$ associa $\sigma^k(n)=n+k$.
Si ha dunque $+\colon \NN \to (\NN\to \NN)$.
Si potrebbe anche pensare all'addizione come ad una funzione 
che ad ogni coppia $(n,k)\in \NN\times \NN$ associa 
la somma $n+k\in \NN$. Dunque si avrebbe $+\colon (\NN\times \NN)\to \NN$.
Ovviamente le due definizioni sono equivalenti, 
il grafico della prima funzione è composto dai punti 
$(k,(n,n+k))\in \NN\times (\NN\times \NN)$
mentre nel secondo caso i punti sono $((n,k),n+k)\in (\NN\times \NN)\times \NN$.
Il vantaggio di definire una funzione $\NN\to(\NN\to\NN)$ invece che una 
funzione $(\NN\times\NN)\to\NN$ è che nel primo caso possiamo utilizzare 
una definizione per induzione. 
In effetti la definizione~\eqref{def:addizione} è equivalente 
a definire l'addizione come l'unica operazione su $\NN$ tale che:
\[
\begin{cases}
  n+0=n\\
  n+(k+1)=(n+k)+1.
\end{cases}  
\]

% \begin{theorem}
%   \label{th:iterata_composta}%
%   Sia $f\colon A \to A$ una qualunque funzione.
%   Allora 
%   \begin{equation}\label{eq:iterata_composta}
%     f^{n+m} = f^m \circ f^n = f^n \circ f^m.
%   \end{equation}
% \end{theorem}
% %
% \begin{proof}
% Dimostriamo la prima uguaglianza in~\eqref{eq:iterata_composta} 
% per induzione su $m$.
% Per $m=0$ è ovvio in quanto $n+0 = \sigma^0(n)=\id_{\NN}(n)=n$ e $f^m=\id_A$
% dunque $f^{n+0} = f^n = f^0 \circ f^n$.
% Il passo induttivo richiede \eqref{eq:476554},
% \eqref{def:iterata} e l'ipotesi induttiva~\eqref{eq:iterata_composta}:
% \[
% f^{n+\sigma(m)} 
% = f^{\sigma(n+m)}
% = f\circ f^{n+m}
% = f\circ f^m \circ f^n
% = f^{\sigma(m)}\circ f^n.
% \]
% 
% Per dimostrare la seconda uguaglianza in~\eqref{eq:iterata_composta}
% dobbiamo preliminarmente dimostrare che vale 
% \begin{equation}\label{eq:5ty349}
%   f\circ f^n = f^n \circ f.  
% \end{equation}
% Lo facciamo per induzione su $n$. Se $n=0$ ambo i lati sono uguali a $f$.
% Il passaggio induttivo è il seguente:
% \[
%   f\circ f^{\sigma(n)} = f\circ f\circ f^n = f\circ f^n \circ f 
%   = f^{\sigma(n)} \circ f.
% \]
% Allora possiamo dimostrare, per induzione su $m$, 
% che $f^m\circ f^n = f^n\circ f^m$:
% \[
%   f^{\sigma(m)}\circ f^n 
%   = f\circ f^m \circ f^n 
%   = f\circ f^n \circ f^m
%   = f^n \circ f\circ f^m
%   = f^n \circ f^{\sigma(m)}.
% \]
% \end{proof}

\begin{theorem}\label{th:proprieta_addizione}
  L'addizione su $\NN$ soddisfa le seguenti 
  proprietà:
  \begin{enumerate}
    \item associativa: $(n+m)+k = n + (m + k)$,
    \item elemento neutro: $n+0 = n = 0+n$,
    \item commutativa: $n+m = m+n$,
    \item invariantiva: se $m+k = n+k$ allora $m=n$.
  \end{enumerate}
\end{theorem}
\begin{proof}
Le proprietà andranno dimostrate per induzione.
I casi base sono sempre banali, vediamo i passi induttivi.

Per la proprietà associativa si ha
\[
(n+m)+(k+1) 
= ((n+m)+k)+1 
= (n+(m+k))+1
= n +((m+k)+1)
= n + (m + (k+1)).
\]
Per l'elemento neutro $n+0=n$ per definizione. Mentre:
\[
0+(n+1) = (0+n)+1 = n+1.
\]
Per la proprietà commutativa dobbiamo prima dimostrare che $n+1=1+n$.
Ma infatti per induzione si ha $n+1+1 = 1+n+1$.
Allora il passo induttivo della proprietà commutativa risulta:
\[
n + m + 1 
= m + n + 1
= m + 1 + n.
\]

Per dimostrare la proprietà invariantiva se vale $m+k+1=n+k+1$
allora $\sigma(m+k)=\sigma(n+k)$ ed essendo $\sigma$ iniettiva 
possiamo dedurre $m+k=n+k$. Ma allora, per ipotesi induttiva, $m=n$. 
\end{proof}

% \begin{exercise}
%   Dimostrare, utilizzando il principio di induzione e le proprietà 
%   dell'addizione, la seguente 
%   proposizione:
%   \[
%   \forall n\in \NN\colon \enclose{\enclose{\exists m\in \NN \colon n = m + m} 
%   \lor \enclose{\exists m\in \NN\colon n = m + m + 1}}.  
%   \]
% \end{exercise}

% La moltiplicazione $\cdot$ su $\NN$ è intesa come somma ripetuta:
% $3\cdot n = n + n + n$. 

Possiamo ora definire la moltiplicazione su $\NN$ come addizione ripetuta.
Formalmente è l'unica operazione che soddisfa la seguente definizione 
induttiva:
\[
\begin{cases}
  n\cdot 0 = 0,\\
  n\cdot(k+1) = n\cdot k + k.
\end{cases}  
\]

% Chiaramente $(\sigma^n)^0=\id_\NN$ dunque $n\cdot 0=\id(0) = 0$.
% Mentre $(\sigma^n)^1=\sigma^n$ dunque $n\cdot 1 = \sigma^n(0) = n$.  
% 
% Ci servirà anche osservare che risulta:
% \begin{equation}\label{eq:70138875}
%   n\cdot (k+1) = n\cdot k + n
% \end{equation}
% infatti:
% \[
%  n\cdot(k+1) 
%  = (\sigma^n)^{k+1}(0)  
%  = \sigma^n ((\sigma^n)^k(0))
%  = \sigma^n (n\cdot k) = n\cdot k + n.
% \]
% 
% La moltiplicazione tra naturali è legata 
% alle iterazioni dai seguenti teoremi.
% \begin{theorem}
%   \label{th:commuta_composta}%
%   Se $f,g\colon X\to X$ commutano, cioè se $f\circ g = g\circ f$ 
%   allora per ogni $n\in \NN$ anche $f^n$ e $g$ commutano
%   e inoltre si ha:
%   \begin{equation}\label{eq:commuta_composta}
%     f^n\circ g^n = (f\circ g)^n.
%   \end{equation}
% \end{theorem}
% \begin{proof}
% Supponiamo ora che $f\circ g = g\circ f$ e dimostriamo per 
% induzione che $f^n\circ g = g\circ f^n$:
% \[
% f^{n+1}\circ g 
% = f \circ f^n \circ g 
% = f \circ g \circ f^n 
% = g \circ f \circ f^n
% = g \circ f^{n+1}.
% \]
% Possiamo quindi dimostrare per induzione che $f^n\circ g^n = (f\circ g)^n$:
% \[
% f^{n+1}\circ g^{n+1}
% =f\circ f^n \circ g \circ g^n
% =f\circ g\circ f^n\circ g^n 
% = (f\circ g) \circ (f\circ g)^n
% = (f\circ g)^{n+1}. 
% \]
% \end{proof}
% 
% \begin{theorem}[iterata dell'iterata]
% Sia $f\colon X\to X$ una funzione qualunque e siano $n,m\in\NN$.
% Allora 
% \begin{equation}\label{eq:iterazione_iterazione}
%   f^{n\cdot m} = (f^n)^m = (f^m)^n.    
% \end{equation}
% \end{theorem}
% %
% \begin{proof}
% Dimostriamo entrambe le uguaglianze per induzione su $m$.
% Per $m=0$ tutti e tre i termini sono uguali all'identità.
% 
% Il passo induttivo per dimostrare $f^{n\cdot m}=(f^n)^m$ 
% usa~\eqref{eq:70138875}:
% \[
% f^{n\cdot (m+1)}
% =f^{n\cdot m + n } 
% = f^n \circ f^{n\cdot m}
% = f^n \circ (f^n)^m
% = (f^n)^{m+1}.
% \]
% 
% Per dimostrare $(f^n)^m = (f^m)^n$ 
% il passo induttivo sfrutta~\eqref{eq:iterazione_iterazione}
% e~\eqref{eq:commuta_composta}:
% \[
% (f^n)^{m+1}
% = f^n \circ (f^n)^m
% = f^n \circ (f^m)^n
% = (f\circ f^m)^n
% = (f^{m+1})^n.
% \]
% \end{proof}
% 
\begin{theorem}[proprietà della moltiplicazione]
  \label{th:proprieta_moltiplicazione}%
La moltiplicazione su $\NN$ soddisfa le seguenti proprietà:
\begin{enumerate}
  \item elemento neutro e assorbente: $n\cdot 1=n$, $n\cdot 0=0$,
  \item proprietà commutativa: $n\cdot m = m\cdot n$.
  \item proprietà distributiva: $n\cdot(k+j) = n\cdot k + n\cdot j$,
  \item proprietà associativa: $n\cdot(m\cdot k) = (n\cdot m)\cdot k$,
\end{enumerate}
\end{theorem}
\begin{proof}
  L'elemento assorbente è dato per definizione: $n\cdot 0 = 0$.
  Per l'elemento neutro si ha $n\cdot 1 = n\cdot 0 + n = 0+n=n$.

  Le altre proprietà andranno tutte dimostrate 
  per induzione. 
  I casi base saranno sempre banali, ci limitiamo a verificare 
  i passi induttivi. 

  Per la proprietà commutativa verifichiamo innanzitutto che 
  $(m+1)\cdot n = m\cdot n + n$. 
  Il passo induttivo è
  \[
  (m+1)\cdot(n+1) 
  = (m+1)\cdot n + m + 1  
  = m\cdot n + n + m + 1
  = m\cdot (n+1) + n + 1. 
  \]
  Allora il passo induttivo della proprietà commutativa risulta:
  \[
  n\cdot (m+1) 
  = n\cdot m + n 
  = m \cdot n + n 
  = (m+1)\cdot n.  
  \]
  Per la proprietà distributiva:
  \[
  (n+1)\cdot(k+j) 
  = (k+j)\cdot (n+1)  
  = (k+j)\cdot n + k + j
  = n\cdot(k+j) + k + j
  = n\cdot k +k + n\cdot j + j
  = (n+1)\cdot k + (n+1)\cdot j.
  \]
  Infine la proprietà associativa:
  \[
  n\cdot(m\cdot(k+1))
  = n\cdot (m\cdot k + m) 
  = n\cdot (m\cdot k) + n\cdot m
  = (n\cdot m)\cdot k + n\cdot m
  = (n\cdot m)\cdot (k+1). 
  \]
\end{proof}

% L'elevamento a potenza si definisce come un prodotto ripetuto: 
% $k^3 = k\cdot k\cdot k$.
% Se consideriamo la funzione $\cdot k$ 
% che moltiplica un numero naturale per $k$:
% \[
%  \cdot k(n) = n\cdot k  
% \]
% Possiamo definire le potenze di $k$ in questo modo:
% \[
%    k^n = (\cdot k)^n(1).  
% \]
% Osserviamo che più in generale si ha 
% \begin{equation}\label{eq:3954820}
%   (\cdot k)^n(m) = m\cdot k^n
%   \qquad\text{ovvero}\qquad
%  (\cdot k)^n = \cdot(k^n).
% \end{equation}
% 

Definiamo l'operazione \emph{potenza} $m^n$ su $\NN$ come un prodotto 
ripetuto, mediante la seguente definizione per induzione:
\[
\begin{cases}
  m^0 = 1\\
  m^{n+1} = m\cdot m^n.
\end{cases}  
\]
\begin{theorem}[proprietà delle potenze]
  \label{th:proprieta_potenza}%
  L'elevamento a potenza definito su $\NN$ ha le seguenti proprietà
  (per ogni $k,n,m\in \NN$):
  \begin{enumerate}
    \item $k^0 = 1$,
    \item $k^{n+m} = k^n \cdot k^m$,
    \item $(k^n)^m = k^{n\cdot m}$,
    \item $(k\cdot j)^n = k^n\cdot j^n$. 
  \end{enumerate}
\end{theorem}
%
\begin{proof}
La prima proprietà $k^0=1$ è data per definizione.

Le altre proprietà si dimostrano per induzione.
I casi base sono tutti banali, ci limitiamo quindi 
a mostrare i passi induttivi.

Per dimostrare $k^{n+m}=k^n\cdot k^m$ il passo induttivo è:
\[
k^{n+m+1} 
= k\cdot k^{n+m}  
= k\cdot k^n\cdot k^m
= k^n \cdot k^{m+1}.
\]

Per dimostrare $(k^n)^m = k^{n\cdot m}$ il passo induttivo è 
\[
  (k^n)^{m+1} 
  = k^n\cdot (k^n)^m
  = k^n\cdot k^{n\cdot m}
  = k^{n+n\cdot m}
  = k^{n\cdot(m+1)}.
\]

Infine per dimostrare $(k\cdot j)^n = k^n\cdot j^n$ 
il passo induttivo è
\[
(k\cdot j)^{n+1} 
= k\cdot j\cdot (k\cdot j)^n  
= k\cdot j\cdot k^n\cdot j^n
= k^{n+1}\cdot j^{n+1}.
\]
\end{proof}

Possiamo definire la relazione $\le$ su $\NN$ ponendo 
\[
  m \le n \iff \exists k\in \NN\colon n = m+k.  
\]
Se $m\le n$ tale numero 
$k$ è unico in quanto se fosse $m+j=n=m+k$ potremmo dedurre $k=j$
dalla proprietà invariantiva della addizione. 
Possiamo dunque definire la \emph{differenza}%
\mymargin{differenza}%
\index{differenza}: $k=n-m$.

\begin{theorem}
  \label{th:proprieta_ordine_N}
La relazione $\le$ è una relazione d'ordine totale su $\NN$.
\end{theorem}
%
\begin{proof}
  Il fatto che $x+0=x$ dimostra la proprietà riflessiva.
  
  Per la proprietà transitiva è sufficiente osservare
  che se $y=x+k$ e $z=y+j$ allora $z=x+k+j$.

  Per la proprietà antisimmetrica supponiamo di avere $x=y+k$ 
  e $y=x+j$. Deduciamo che $x+k+j = y +k+j$ da cui $x=y$
  per la proprietà invariantiva dell'addizione.
  
  Per mostrare che l'ordinamento è totale dobbiamo invece 
  procedere con una dimostrazione per induzione. 
  Per induzione su $x\in \NN$ vogliamo mostrare 
  che per ogni $y\in \NN$ si ha $x\le y$ oppure $y\le x$. 
  Se $x=0$ il fatto è ovvio in quanto $y=y+0$ e quindi $y\ge 0$
  per ogni $y\in \NN$.
  Supponiamo ora di sapere che $x\le y$ oppure $y\le x$.
  Se $y\le x$ significa che $x = y + k$ per un qualche 
  $k\in \NN$. Ma allora $x+1 = y + k + 1$ e quindi 
  vale anche $y\le x+1$. 
  Se invece $x\le y$ significa che esiste $k\in \NN$ 
  per cui $y=x+k$. Se $k\neq 0$ allora $k=1+j$ con $j\in \NN$
  da cui $y=x+1+j$ e dunque anche $x+1\le y$.
  Se $k=0$ allora $x=y$ e dunque $x+1=y+1$ che ci porta 
  alla disuguaglianza inversa $x+1\ge y$.
  In ogni caso il passo induttivo è dimostrato.
\end{proof}

\begin{theorem}[proprietà di monotonia sui naturali]
  \label{th:monotonia_naturali}%
  Se $m,n,k\in \NN$ e $m\le n$ allora si ha 
  \[
    m+k\le n+k,
    \qquad 
    m\cdot k \le n\cdot k,
    \qquad
    m^k\le n^k,
    \qquad
    k^m\le k^n.
  \]
\end{theorem}
\begin{proof}
  Se $m\le n$ si ha $n=m+j$. 
  Dunque
  \[
  n+k = m + j + k \ge m+k   
  \] 
  e
  \[
  m\cdot k = (n+j)\cdot k = n\cdot k + j\cdot k \ge n\cdot k.  
  \]

  Per la monotonia delle potenze procediamo per induzione.
  Basterà dimostrare che $m^k\le(m+1)^k$ e $k^m\le k^{m+1}$.
  Lo facciamo, a loro volta, per induzione:
  \[
  (m+1)^{k+1} = (m+1)\cdot (m+1)^k \ge m\cdot m^k
  \]
  e (supponendo $k\ge 1$)
  \[
  k^{m+1} = k\cdot k^m \ge 1\cdot k^m = k^m. 
  \]
\end{proof}

\subsection{buon ordinamento e unicità dei numeri naturali}

\begin{theorem}[principio del buon ordinamento]
  \label{th:buon_ordinamento}
  Sia $A\subset \NN$, $A\neq \emptyset$. 
  Allora $A$ ha minimo.
\end{theorem}
%
\begin{proof}
  Osserviamo che se $n\in \NN$ è un minorante di $A\subset \NN$ 
  allora o $n\in A$ e quindi $n$ è il minimo di $A$
  oppure anche $n+1$ è un minorante di $A$ in quanto 
  non ci sono numeri naturali strettamente compresi tra $n$ e $n+1$.
  \mynote{Se ci fosse un numero naturale tra $n$ e $n+1$ 
  sottraendo $n$ avremmo un numero naturale tra $0$ e $1$. 
  Ma non esiste $x\in \NN$ tale che $0<x<1$.
  Infatti se $x\in \NN$, $x>0$ allora $x\neq 0$ e quindi 
  esiste $m\in \NN$ tale che $x=m+1$. 
  Ma allora $x\ge 1$.}
  Dunque se $A$ non avesse minimo, per il principio di induzione 
  ogni $n\in\NN$ sarebbe un minorante di $A$.
  In tal caso $A$ dovrebbe essere vuoto perché se esistesse $a\in A$ 
  certamente $a+1$ non sarebbe un minorante di $A$.
\end{proof}

Vogliamo ora dimostrare che l'insieme dei numeri naturali è sostanzialmente 
unico nel senso che se ci sono due insiemi che soddisfano gli assiomi di 
Peano allora è possibile mettere in corrispondenza gli elementi dei due insiemi 
in modo che lo zero vada in zero e numeri corrispondenti abbiano successori 
corrispondenti.

\begin{theorem}[unicità dei numeri naturali]
  \label{th:unicitaN}%
  Se $\NN$ e $\NN'$ sono due insiemi che soddisfano gli assiomi di Peano 
  con zero $0\in \NN$ e $0'\in \NN'$ e funzioni 
  successore $\sigma$ su $\NN$ e $\sigma'$ su $\NN'$ allora
  esiste una funzione bigettiva $f\colon \NN\to \NN'$ tale che 
  $f(0) = 0'$ e $f(\sigma(n)) = \sigma'(f(n))$.
\end{theorem}
%
\begin{proof}
Possiamo definire $f$ per induzione:
\[
\begin{cases}
  f(0) = 0' \\ 
  f(\sigma(n)) = \sigma'(f(n))
\end{cases}  
\]
così rimane solo da dimostrare che $f$ è una bigezione.

Per dimostrare che $f$ è iniettiva consideriamo l'insieme 
\[
  A=\ENCLOSE{a\in \NN\colon \exists b\in \NN\colon b\neq a, f(a)=f(b)}.
\]
Se tale insieme è vuoto allora $f$ è effettivamente iniettiva.
Supponiamo allora per assurdo che $A$ non sia vuoto.
In tal caso possiamo considerare il minimo $a=\min A$ 
(grazie al teorema~\ref{th:buon_ordinamento}).
Dovrà quindi esistere $b\in \NN$ tale che $b\neq a$ e $f(b)=f(a)$.
Ovviamente anche $b\in A$ e quindi dovrà essere $b>a$ in quanto
$a$ è il minimo. Dunque $b>0$ e $f(b) = f(\sigma(b-1))
=\sigma'(f(b-1))$. Se $a=0$ abbiamo $f(a)=0'$ e quindi da $f(a)=f(b)$ 
otteniamo che $0'$ è nell'immagine di $\sigma'$ che è contrario 
agli assiomi di Peano. 
Se invece $a>0$ si avrà, come per $b$,
$f(a)=\sigma'(f(a-1))$ e dunque $\sigma'(f(a-1)) = \sigma'(f(b-1))$.
Per l'iniettività di $\sigma'$ si deduce $f(a-1)=f(b-1)$ da cui 
$a-1 \in A$. Ma questo è assurdo in quanto $a$ era il minimo di $A$.

Per dimostrare che $f$ è surgettiva consideriamo l'immagine 
$B'=f(\NN)$ e usiamo il principio di induzione su $\NN'$ 
per dimostrare che $B'=\NN'$.
Per prima cosa $0'\in B'$ in quanto $0'=f(0)$.
Se poi $n'\in B'$ allora esiste $n\in \NN$ tale che $f(n)=n'$.
Ma allora $f(\sigma(n))=\sigma'(f(n))=\sigma'(n')$ 
e dunque anche $\sigma'(n')\in B$. 
\end{proof}

\subsection{cardinalità finite}

Vogliamo dimostrare che i numeri naturali possono essere utilizzati 
per contare gli elementi degli insiemi finiti.

\begin{definition}[numero di elementi]
Se $n\in \NN$ denotiamo con 
\[
  \Enclose{n} 
  = \ENCLOSE{0,1,\dots, n-1}
  = \ENCLOSE{k\in \NN\colon k<n}
\]
l'insieme dei primi $n$ numeri naturali.
\mynote{Siccome per noi il primo numero naturale è $0$, c'è uno scarto di 
una unità tra gli ordinali \emph{primo}, \emph{secondo}, \emph{terzo}\dots e i cardinali \emph{zero},
 \emph{uno}, \emph{due}\dots
La cosa può sembrare strana ma in realtà è perfettamente \emph{naturale}
e infatti in gran parte dei linguaggi di programmazione moderni quando 
si itera una operazione $n$ volte si conta da $0$ a $n-1$.}
Questo è il prototipo di insieme con $n$ elementi.
Allora informalmente andremo a identificare $\#\Enclose{n}$ con $n$ e 
dunque scriveremo 
\[
   \#A = n, \qquad \#A\ge n, \qquad \#A\le n
\]
per significare rispettivamente
\[
  \#A = \#\Enclose{n}, \qquad \#A \ge \#\Enclose{n}, \qquad \#A\le \#\Enclose{n}.  
\]

La cardinalità di $\NN$ viene a volte indicata con il simbolo
\index{$\aleph_0$}%
\index{aleph}%
$\aleph_0$ dunque potremo scrivere  
\mynote{%
Il simbolo $\aleph$ è la prima lettera dell'alfabeto ebraico 
e si legge \emph{aleph}. \\
Abbiamo già osservato che $\NN$ è il \emph{più piccolo} insieme 
infinito dunque $\aleph_0$ è la più piccola 
cardinalità infinita. 
Si potrebbe anche dimostrare che deve esistere una cardinalità $\aleph_1>\aleph_0$
che è la più piccola cardinalità maggiore di $\aleph_0$.
Sappiamo che $\# \mathcal P(\NN)>\# \NN$ ma curiosamente non è possibile dimostrare 
che $\aleph_1 = \# \mathcal P(\NN)$ (questa si chiama \emph{ipotesi del continuo}
\index{ipotesi!del continuo}%
\index{continuo!ipotesi del}%
in quanto vedremo nel teorema~\ref{th:cantor_secondo} che $\# \mathcal P(\NN)=\#\RR$
e l'insieme $\RR$ viene chiamato \emph{continuo} là dove $\NN$ 
viene chiamato \emph{discreto}).
Dunque l'ipotesi del continuo è un ulteriore assioma che potrebbe essere 
aggiunto agli assiomi della teoria degli insiemi: 
non esistono insiemi con cardinalità strettamente compresa tra 
le cardinalità di $\NN$ e di $\mathcal P(\NN)$.
}%
\[
    \# A = \aleph_0, \qquad
    \#A \le \aleph_0, \qquad
    \# A\ge \aleph_0
\]
per significare rispettivamente 
\[
  \# A = \# \NN, \qquad
  \# A \le \# \NN, \qquad
  \# A \ge \# \NN.  
\]
\end{definition}
  
\begin{lemma}\label{lm:nfinito}
Per ogni $n\in \NN$ l'insieme $[n]=\ENCLOSE{0,1,2, \dots, n-1}$ è 
finito (nel senso di Dedekind, definizione~\ref{def:infinito}).
\end{lemma}
%
\begin{proof}
Lo possiamo dimostrare per induzione su $n\in \NN$.
Per $n=0$ si ha $[n]=\emptyset$ ed è ovvio che ogni funzione 
$f\colon \emptyset \to \emptyset$ è bigettiva. 
Dunque $[0]=\emptyset$ è finito.

Supponiamo allora che $[n]$ sia finito e supponiamo per assurdo che $[n+1]$ sia 
infinito.
Allora esiste $f\colon [n+1]\to[n+1]$ iniettiva ma non surgettiva. 
Se componiamo $f$ con la funzione $s$ che scambia $n$ con $f(n)$ otteniamo 
una funzione $g = s\circ f$ tale che $g(n)=n$.
Se $f$ era iniettiva ma non surgettiva anche $g$ mantiene le due proprietà.
Inoltre $g$ può essere ristretta a $[n]$ (in quanto $g(k)=n$ solo per $k=n$ 
visto che $g$ è iniettiva): $g\colon [n]\to [n]$
e anche la funzione ristretta risulterebbe essere iniettiva e suriettiva.
Ma questo è assurdo perché per ipotesi induttiva $[n]$ è finito.
\end{proof}

\begin{theorem}[caratterizzazione insiemi finiti]
Un insieme $A$ è infinito (nel senso di Dedekind, definizione~\ref{def:infinito})
se e solo se $\# A \ge \aleph_0$.
Viceversa $A$ è finito se e solo se esiste $n\in\NN$ tale che $\# A = n$.
\end{theorem}
%
\begin{proof}
Se $\#A \ge \aleph_0$ significa che esiste una funzione 
iniettiva $f\colon \NN\to A$. 
Dunque se definiamo $\sigma \colon A \to A$ come
\[
\sigma(a) = \begin{cases}
  a & \text{se $a\not\in f(\NN)$}\\
  f(n+1) & \text{se $a=f(n)$ con $n\in\NN$}
\end{cases}
\]
otteniamo una funzione iniettiva ma non suriettiva 
(in quanto $\sigma(A) = A\setminus\ENCLOSE{f(0)}$).
Dunque se $\#A\ge \aleph_0$ deduciamo che $A$ è infinito

Viceversa se $A$ è infinito significa che esiste $\sigma\colon A \to A$ 
iniettiva ma non suriettiva. 
Preso $\alpha \in A \setminus\sigma(A)$
possiamo definire induttivamente $f\colon \NN\to A$ come segue:
\[
\begin{cases}
  f(0) = \alpha\\
  f(n+1) = \sigma(f(n))
\end{cases}
\]
e possiamo verificare che $f$ è iniettiva. 
Infatti supponiamo che sia $f(n)=f(m)$.
Se $f(n)=f(m)=\alpha$ allora necessariamente $n=m=0$ 
perché $f(n)=\alpha$ è equivalente a $n=0$ visto che $\sigma$
non assume mai il valore $\alpha$.
Se $f(n)=f(m)\neq \alpha$ significa dunque che $n>0$ e $m>0$
ma allora $f(n)=\sigma(f(n-1))$ e $f(m)=\sigma(f(m-1))$
e dunque, essendo $\sigma$ iniettiva, $n-1=m-1$ da cui $n=m$. 
Dunque se $A$ è infinito si ha $\#\NN\le \#A$
(in realtà l'avevamo già osservato: $\NN$ è il più piccolo insieme infinito).

Dobbiamo ora caratterizzare gli insiemi finiti.
Abbiamo già verificato che se $\#A \ge \aleph_0$ 
allora $\#A$ è infinito, dunque se $\#A$ è finito 
non può essere $\#A \ge \aleph_0$.
Ma da questo non possiamo immediatamente 
dedurre che%
\mynote{%
Per dimostrare che ogni cardinalità può essere confrontata 
con ogni altra (cioè che l'ordinamento dato dalla cardinalità è totale) 
bisogna dimostrare che ogni insieme ammette un buon 
ordinamento. La cosa diventa molto tecnica e richiede,
l'utilizzo dell'assioma della scelta.
}%
$\#A < \aleph_0$%

Se $\#A = n$ significa che $\#A = \#\Enclose{n}$ e 
grazie al lemma~\ref{lm:nfinito} sappiamo che $[n]$ è finito 
e dunque anche $A$ è finito.

Supponiamo ora che $A$ sia un insieme per il quale per ogni $n\in \NN$ 
si ha $\#A > n$. Dobbiamo dimostrare che $A$ è infinito.
Per ogni $n\in \NN$ scegliamo%
\mynote{Qui stiamo usando l'assioma della scelta}
una funzione iniettiva $f_n\colon\Enclose{n}\to A$.
Dalle $f_n$ possiamo definire le funzioni iniettive $g_n\colon \Enclose{n}\to A$
in modo che $g_{n+1}$ sia una estensione di $g_n$.
Basterà infatti porre:
\[
  \begin{cases}
  g_0 = f_0,\\
  g_{n+1}(k) = \begin{cases}
    g_n(k) &\text{se $k<n$}\\
    \min f_{n+1}([n+1])\setminus g_n([n]) & \text{altrimenti}
    \end{cases}
  \end{cases}
\]
l'insieme $f_{n+1}([n+1])\setminus g_n([n])$ essendo non vuoto 
in quanto $\# g_n([n]) = n < n+1 = \# f_{n+1}([n+1])$.
Ma allora possiamo definire la funzione iniettiva $g\colon \NN \to A$
come $g(n) = g_{n+1}(n)$ ottenendo $\#A \ge \#\NN$ e dunque $A$ 
è Dedekind infinito.
\end{proof}

Nella dimostrazione precedente abbiamo implicitamente utilizzato il seguente lemma,
la cui dimostrazione lasciamo per esercizio.
\begin{lemma}
  Se $\#A = n$ e $\#B = m$ con $n,m\in \NN$ allora 
  \[
    \#A = \#B \iff n=m
    \qquad \text{e}\qquad
    \#A \le \#B \iff n \le m.
  \]
\end{lemma}


\begin{exercise}[operazioni con le cardinalità finite]
  \label{th:combinatoria}
  Se $A$ e $B$ sono insiemi disgiunti e finiti allora  
  \[
    \#(A\cup B) = \#A + \#B.
  \]
  Se $A$ e $B$ sono insiemi finiti allora 
  (ricordiamo che $B^A$ è l'insieme delle funzioni $A\to B$,
  mentre i coefficienti binomiali sono definiti nel capitolo
  \ref{ch:binomiale})
  \begin{gather*}
    \#(A \times B) = \#A\cdot \#B, \qquad
    \#(B^A) = \#B^{\#A}, \qquad
    \#\mathcal P(A) = 2^{\# A}, \\
    \#\ENCLOSE{f\in A^A\colon \text{$f$ bigettiva}} = (\#A)!, \\
     \#\ENCLOSE{X\in \mathcal P(A)\colon \#X = k}  
     = {\#A \choose k}. 
  \end{gather*}
\end{exercise}

\subsection{sequenze finite o ennuple}

Abbiamo già visto che se $A$ è un insieme possiamo definire l'insieme 
delle coppie di elementi di $A$ con il prodotto cartesiano tra insiemi: 
$A\times A$. 
Cosa rappresenta un prodotto ripetuto?
Gli insiemi $(A\times A)\times A$ e $A\times(A\times A)$ non sono 
uguali, in quanto il primo è un insieme di coppie il cui primo elemento 
è a sua volta una coppia, il secondo insieme invece è un insieme di coppie 
il cui secondo elemento è una coppia:
\begin{gather*}
  (A\times A)\times A = \ENCLOSE{((a_0,a_1),a_2)\colon a_0,a_1,a_2\in A},
  \\
  A \times (A\times A) = \ENCLOSE{(a_0,(a_1,a_2))\colon a_0,a_1,a_2\in A}.
\end{gather*}
Questi insiemi sono molto simili tra loro e potrebbero essere identificati.
Un'altro insieme molto simile è l'insieme $A^{[3]}$.  
Ricordando che $[3] = \ENCLOSE{0,1,2}$ l'insieme 
$A^{[3]}$ rappresenta l'insieme di tutte le funzioni $\vec a\colon \ENCLOSE{0,1,2}\to A$.
La funzione $\vec a$ è univocamente determinata dal suo valore nei 
tre punti del dominio: $a_0 = \vec a(0)$, $a_1=\vec a(1)$, $a_2=\vec a(2)$
in quanto $\vec a = \ENCLOSE{0\mapsto a_0, 1\mapsto a_1, 2\mapsto a_2}$.
Possiamo quindi identificare $\vec a$ con la tripla (o vettore) di valori:
\[
  \vec a = (a_0, a_1, a_2), \qquad a_0,a_1,a_2 \in A.  
\]
I valori $a_0$, $a_1$ e $a_2$ vengono anche chiamate \emph{coordinate}
o \emph{componenti} del \emph{vettore} $\vec a$.
\mymargin{coordinate, componenti, vettore}%
\index{coordinate, componenti, vettore}%
\index{coordinate!vettore}%
\index{componenti!vettore}%
\index{vettore}%
In effetti l'insieme $A^{[2]}$ può essere identificato con $A\times A$, l'insieme 
$A^{[3]}$ sarà l'insieme delle triple di elementi di $A$ e in generale se $n\in\NN$ 
identifichiamo con $A^{[n]}$ l'insieme delle $n$-uple (leggi: ennuple) di elementi 
\index{ennuple}%
di $A$%
\mynote{
  Usualmente si scrive $A^n$ invece che $A^{[n]}$ e $2^A$ invece che $[2]^A$ 
  anche perché la notazione $[n]$ per indicare i primi $n$ numeri naturali 
  non è molto utilizzata.
  Si potrebbe anche definire l'insieme $\NN$ in modo tale che 
  risulti $n=[n]$: basta definire $0=\emptyset$ e $\sigma(n) = n\cup \ENCLOSE{n}$
  (si faccia la dimostrazione per induzione).
}:
\[
   \vec a \in A^{[n]} \iff 
   \vec a = (a_0, a_1, \dots, a_{n-1}).  
\]

L'insieme $[2]^A$ è invece l'insieme delle funzioni $A\to [2]$ che hanno 
quindi valore $0$ o $1$. 
Ci si può convincere facilmente che $[2]^A$ può essere messo in corrispondenza 
biunivoca con $\mathcal P(A)$:
\[
    \# ([2]^A) = \# \mathcal P(A).
\]
Infatti possiamo considerare la funzione $\phi\colon [2]^A \to \mathcal P(A)$
che associa ad ogni funzione $f\colon A\to \ENCLOSE{0,1}$
l'insieme $\ENCLOSE{x\in A \colon f(x)=1}$.

\section{i numeri interi}

L'addizione su $\ZZ$ non sarà altro che la composizione 
di due traslazioni.
Nel seguito useremo le ultime lettere dell'alfabeto 
$x,y,z$ per denotare gli elementi di $\ZZ$ mentre 
useremo le prime lettere $a,b,c,d$ per denotare 
gli elementi di $\NN$.

Se $a\stackrel x \mapsto b$ e $a'\stackrel y \mapsto b'$
allora $x=y$ se e solo se $a+b'=a'+b$.
\mynote{
  Si noti ad esempio che se $b=a+n$ e $b'=a'+m$ allora 
  $a+b' = a + a' + m$ mentre $a'+b = a+a'+n$ dunque 
  $m=n$ se e solo se $a+b'=a'+b$. 
  Similmente se $a=b+n$ e $a'=b'+m$.
  Se invece $b=a+n$ e $a'=b'+m$ le due traslazioni 
  sono in verso opposto e l'uguaglianza $a+b'=a'+b$ 
  risulta equivalente a $m+n=0$ che significa $m=n=0$.
}
Inoltre se $x$ è l'unica traslazione che manda $a\stackrel x \mapsto b$ 
allora per ogni $c\in \NN$ si ha $a+c\stackrel x \mapsto b+c$
(proprietà invariantiva).

Dati $x,y\in \ZZ$ vogliamo definire $x+y\in \ZZ$.
Se $a\stackrel x\mapsto b$
e $c\stackrel y\mapsto d$ 
(con $a,b,c,d\in \NN$)
allora applicando prima $x$ poi $y$ 
si avrà $a+c\stackrel x\mapsto b+c$ 
e $b+c\stackrel y\mapsto b+d$. 
Definiamo quindi $x+y\in \ZZ$ come l'unica traslazione 
$a+c \stackrel{x+y} \mapsto b+d$.
Questa è una buona definizione perché se $a'\stackrel x \mapsto b'$ 
e $c'\stackrel y \mapsto d'$ allora 
si ha $a+b'=a'+b$, $c+d'=c'+d$ 
da cui 
$b'+d' + a+c = a'+c' + b+d$ 
che significa che $a'+c'\mapsto b'+d'$ 
è la stessa traslazione di $a+c\mapsto b+d$.
Osserviamo che $x+y=y+x$ in quanto $a+c=c+a$ e $b+d=d+b$
(su $\NN$ sappiamo già che l'addizione è commutativa).

La traslazione $+0\in \ZZ$ manda $0\stackrel{+0}\mapsto 0$ 
e quindi $x+(+0)=x$ per ogni $x\in \ZZ$.
Inoltre se $a\stackrel x\mapsto b$ e $b\stackrel y \mapsto a$ 
allora $a \stackrel{x+y} \mapsto a$ e quindi $x+y = +0$.

Possiamo anche definire un ordinamento su $\ZZ$ ponendo $x\le y$ 
se la traslazione $y$ manda i punti \emph{più avanti} di quanto 
non faccia $x$. 
Se $a\stackrel x\mapsto b$ e $c\stackrel y\mapsto d$ 
il punto $a+c$ viene mandato in $b+c$ da $x$ e in $a+d$ da $y$:
poniamo dunque $x\le y$ quando $b+c\le a+d$.
Se $x,y,z\in \ZZ$ vogliamo mostrare 
che $x\le y \implies x+z\le y+z$.
Posto $a\stackrel x\mapsto b$, $c\stackrel y\mapsto d$, 
$e\stackrel z\mapsto f$ 
allora $a+c+e \stackrel{x+z} \mapsto b + c+f$ 
mentre $a+c+e \stackrel{y+z} \mapsto a+d+f$ dunque 
se $x\le y$ si ha $b+c \le a+d$ da cui $b+c+f \le a+d+f$
e quindi $x+z \le y+z$ come volevamo dimostrare.

Se $x>0$ diremo che $x$ è \emph{positivo}%
\mymargin{positivo}%
\index{positivo}
se invece $x<0$ diremo che $x$ è \emph{negativo}%
\mymargin{negativo}%
\index{negativo}.

In base a quanto abbiamo osservato l'insieme $\ZZ$ con l'operazione 
di addizione $x+y$ e la relazione $x\le y$ che abbiamo appena definito 
risulta essere, in base alle seguenti definizioni, 
un gruppo additivo, abeliano, ordinato.

  Osserviamo che la funzione $n\stackrel+\mapsto +n$ mette in corrispondenza 
  ogni numero $n\in \NN$ con la traslazione non negativa $+n\in \ZZ$.
  Ovviamente $+(n+m) = (+n)+(+m)$ e $n\le m \iff +n \le +m$
  dunque identificando $n$ con $+n$ possiamo considerare $\ZZ$ una
  estensione di $\NN$ che mantiene inalterata l'addizione e la relazione d'ordine.
  Sarà quindi consuetudine supporre che sia $\NN\subset \ZZ$.

  L'insieme $\NN$ non è un gruppo additivo in quanto non esiste l'elemento opposto.
  Significa che l'operazione inversa dell'addizione, la sottrazione, non è definita
  per ogni coppia di numeri naturali. 
  Aggiungendo i numeri negativi abbiamo quindi esteso $\NN$ a $\ZZ$ in modo che 
  ogni elemento abbia opposto, ottenendo guindi un gruppo additivo.
  La differenza $x-y$ di due numeri interi $x,y\in \ZZ$ è definita come la somma 
  con l'opposto: $x-y = x + (-y)$ ed estende la differenza che avevamo già 
  definito su $\NN$ quando $x\ge y$.

  L'operazione di moltiplicazione che abbiamo definito su $\NN$ può essere 
  estesa in modo naturale a tutto $\ZZ$. 
  Se vogliamo mantenere la proprietà distributiva dovremo avere:
  \[
    0 = 0 \cdot n = (m+(-m))\cdot n = m\cdot n + (-m)\cdot n
  \]
  per cui poniamo per definizione, per ogni $n,m\in \NN$:
  \[
    (-m) \cdot n = -(m\cdot n), \qquad  m \cdot (-n) = -(m\cdot n).
  \]

  E' noioso ma facile dimostrare che valgono le usuali proprietà della
  moltiplicazione su $\ZZ$.

  \begin{theorem}[proprietà moltiplicazione]
    La moltiplicazione su $\ZZ$ soddisfa le seguenti proprietà.
    \begin{enumerate}
      \item[1.] elemento neutro e assorbente: $n\cdot 1 = n$, $n\cdot 0 = 0$,
      \item[2.] proprietà associativa: $(n\cdot m)\cdot k = n \cdot (m\cdot k)$,
      \item[3.] proprietà commutativa: $n\cdot m = m\cdot n$,
      \item[4.] proprietà distributiva: $k\cdot(m+n) = k\cdot m + k\cdot n$. 
    \end{enumerate}
  \end{theorem}

In effetti scopriamo che $\ZZ$ è un \emph{anello}%
\mymargin{anello}%
\index{anello} abeliano con unità, in base alla seguente.
%
\begin{definition}[anello]
  \label{def:anello}%
  Sia $A$ un insieme su cui sono definite due operazioni: $+$ (addizione) e $\cdot$ (moltiplicazione).  
  Diremo che $A$ è un \emph{anello} se $A$ è un gruppo abeliano rispetto alla 
  addizione, 
  se la moltiplicazione è associativa $(x\cdot y)\cdot z = x\cdot (y\cdot z)$ 
  e se vale la proprietà distributiva $x\cdot(y+z) = x\cdot y + x\cdot z$,
  $(x+y)\cdot z = x\cdot z + y\cdot z$.
  Se inoltre la moltiplicazione è commutativa diremo che $A$ è un \emph{anello abeliano}.
  Se inoltre esiste un elemento $1\in A$ neutro per la moltiplicazione 
  diremo che $A$ è un anello \emph{con unità}.
\end{definition}

In generale se $A$ è un anello allora si può dimostrare che valgono anche le usuali proprietà:
\[
  0\cdot x = x\cdot 0 = 0, \qquad
  (-1)\cdot x = x \cdot (-1) = -x.
\]
Per la prima abbiamo: 
\[
  0\cdot x = 0\cdot x + x + (-x) = (0+1)\cdot x + (-x) = x + (-x) = 0
\]
e scrivendo gli addendi in ordine opposto si ottiene anche $x\cdot 0 = 0$.
Allora possiamo dimostrare anche la seconda proprietà:
\[
   (-1)\cdot x = (-1)\cdot x + x + (-x) = (-1 + 1)\cdot x + (-x) = 0 + (-x) = -x
\]
e anche in questo caso scrivendo gli addendi in ordine opposto si ottiene $x\cdot(-1)=-x$.

Se $n,m\in\ZZ$ con $m\neq 0$ e 
\mymargin{multiplo}%
se esiste $k\in\ZZ$ tale 
che $n=km$ diremo che $n$ è un \emph{multiplo}
di $m$ oppure che $m$ è un \emph{divisore}%
\mymargin{divisore}%
\index{divisore} di $n$
e scriveremo:
\[
  \frac{n}{m} = k.  
\]

I multipli di $2$ si chiamano numeri \emph{pari}%
\mymargin{pari}%
\index{pari},
gli interi non pari si dicono \emph{dispari}%
\mymargin{dispari}%
\index{dispari}. 

\section{polinomi}
\label{ch:polinomi}

Il concetto di ``polinomio'' è una astrazione che non risulta facile esplicitare formalmente.
Nel capitolo~\ref{ch:edo} ci sarà molto utile applicare un polinomio ad un operatore 
differenziale e dunque ci teniamo ora a rendere molto chiaro cosa intendiamo per polinomio.
In particolare distingueremo tre concetti che spesso vengono sovrapposti: 
\emph{espressione polinomiale}, \emph{polinomio} e \emph{funzione polinomiale}.

Se $A$ è un anello abeliano con unità 
(definizione~\ref{def:anello}, per ora noi abbiamo un unico esempio $A=\ZZ$ ma più avanti potremo scegliere 
$A=\QQ$, $A=\RR$ o $A=\CC$)
possiamo considerare tutte le espressioni
(formalmente: alberi di valutazione)
che possono essere costruite utilizzando le operazioni
di addizione e moltiplicazione che coinvolgono coefficienti presi da $A$
e una variabile che possiamo chiamare $x$ 
(e più in generale è possibile considerare polinomi in più variabili).
Queste espressioni verranno chiamate \emph{espressioni polinomiali}%
\mymargin{espressioni polinomiali}%
\index{espressioni!polinomiali} su $A$
(o a coefficienti in $A$) nella variabile $x$. 
Un esempio di espressione polinomiale
è riportato in Figura~\ref{fig:49389}.
\begin{figure}
\begin{center}
\begin{tikzpicture}
\node [circle,draw] {$\cdot$}
  child {
    node [circle,draw,xshift=-15mm] {$+$}
    child {
      node [circle,draw] {$\cdot$}
      child {node [draw]{$-7$}}
      child {node [draw]{$x$}}
    }
    child {node [draw] {$42$}}
  }
  child {
    node [circle,draw,xshift=15mm] (M) {$+$}
    child {
      node [circle,draw,xshift=-10mm]{$\cdot$}
      child {node [draw]{$x$}}
      child {
        node [circle,draw]{$+$}
        child {node [draw] {$2$}}
        child {node [draw] {$x$}}
        }
      }
    child {
      node [circle,draw,xshift=10mm]{$\cdot$}
      child {node [draw] {$3$}}
      child {
        node [circle,draw] {$+$}
        child {node [draw] {$-1$}}
        child {node [draw] {$x$}}
        }
    }
  };
\end{tikzpicture}
\end{center}
\caption{Il polinomio $P = \enclose{(-7)\cdot x + 42}\cdot(x\cdot (2+x) + 3\cdot(-1+x))$
rappresentato come albero di valutazione.}
\label{fig:49389}%
\end{figure}
Le espressioni polinomiali possono essere sommate e moltiplicate tra loro 
per ottenere nuove espressioni. 
Come comoda notazione si utilizzano le potenze intere per 
denotare un prodotto ripetuto $n$ volte: $x^n = x\cdots x$.

Diremo che due espressioni sono \emph{equivalenti} se è possibile trasformare 
una nell'altra utilizzando le proprietà valide negli anelli abeliani: 
proprietà associativa 
e commutativa per somma e prodotto, proprietà distributiva, elementi 
neutri $1\cdot x = x$, $0 + x = x$.
Inoltre se un ramo dell'espressione non contiene la variabile $x$ è possibile valutare 
(nell'anello $A$) le operazioni e sostituire l'intero ramo con il 
risultato di tali operazioni (e viceversa).

In particolare se abbiamo una qualunque espressione possiamo sviluppare tutti i prodotti
mediante la proprietà distributiva e poi riassociare i termini 
con le stesse potenze di $x$, ad esempio:
\begin{align*}
P &= \enclose{(-7)\cdot x+42}\cdot(x\cdot (2 + x)
+ 3\cdot(-1+x))\\
&= \enclose{(-7)\cdot x + 42}\cdot (2x+x^2-3+3x) \\
% (42-7x)  *  (-3 + 5x + x^2)
&= 126 + 231 x + 7 x^2 - 7 x^3
\end{align*}
In generale si otterrà una espressione polinomiale che diremo 
essere in \emph{forma canonica}:
\[
  P = a_0 + a_1 x + a_2 x^2 + \dots + a_n x^n
       = \sum_{k=0}^n a_k x^k.
\]
Dunque ogni espressione polinomiale è equivalente ad una 
espressione polinomiale in forma canonica. 

Se due espressioni polinomiali sono equivalenti 
diremo che rappresentano lo stesso \emph{polinomio}%
\mymargin{polinomio}%
\index{polinomio}.
Se $A$ è un anello denoteremo con 
$A[x]$ \index{$\KK[x]$}\index{$A[x]$}%
l'insieme di tutti i polinomi con coefficienti in $A$ e variabile $x$. 
Formalmente $A[x]$ è il quoziente dell'insieme 
di tutte le espressioni polinomiali rispetto alla relazione di equivalenza
che abbiamo appena definito. 
Possiamo fare la somma e il prodotto di polinomi e queste operazioni 
rendono $A[x]$ anch'esso un anello.

Più esplicitamente se $P$ e $Q$ sono polinomi in forma canonica
\[
  P = \sum_{k=0}^n a_k x^k, \qquad Q = \sum_{k=0}^m b_k x^k
\]
si avrà:
\[
  P + Q = \sum_{k=0}^{N} (a_k+b_k) \cdot x^k
\]
dove $N$ è il più grande tra $n$ e $m$ e si intende che $a_k=0$ per $k>n$ e 
$b_k=0$ per $k>m$.
Se $t\in A$ avremo:
\[
  t P = \sum_{k=0}^n (ta_k)\cdot x^k.
\]
Per quanto riguarda il prodotto si avrà invece:
\begin{align*}
  P\cdot Q
  &= \enclose{\sum_{k=0}^n a_k x^k}\cdot \enclose{\sum_{j=0}^n b_j x^j}
  = \enclose{\sum_{k=0}^n a_k \enclose{\sum_{j=0}^m b_j x^j} x^k} \\
  &= \enclose{\sum_{k=0}^n \sum_{j=0}^m a_k b_j x^{j+k}}
  = \sum_{s=0}^{m+n} \enclose{\sum_{j=0}^s a_j b_{s-j}} x^s.
\end{align*}

Le formule per la somma e il prodotto dei polinomi in forma canonica 
possono essere applicate ad ognuna delle proprietà di anello per dimostrare
che polinomi equivalenti hanno la stessa forma canonica e che quindi 
due espressioni polinomiali sono equivalenti se e solo se i coefficienti 
della loro forma canonica coincidono.
In pratica i polinomi a coefficienti in $A$ sono 
rappresentati dai coefficienti delle loro forme canoniche che 
non sono altro che sequenze finite:
\begin{align*}
  A[x] 
  &= \{\sum_{k=0}^n a_k x^k\colon a_k\in A, n\in \NN\}\\
  &\approx A^\NN_c 
  = \ENCLOSE{\vec a \in A^\NN\colon \ENCLOSE{k\in \NN\colon \vec a(k)\neq 0} \text{ è finito}}.  
\end{align*}

Se il polinomio $P$ si scrive in forma canonica
\[
  P = \sum_{k=0}^n a_k x^k
\]
se $n>0$ possiamo supporre che il coefficiente $a_n$ sia diverso da $0$ 
perché altrimenti potremmo rimuovere il termine corrispondente e ridurci 
ad una somma di $n-1$ termini. 
In tal caso diremo che il polinomio $P$ ha \emph{grado}%
\mymargin{grado}%
\index{grado} $n$.
\index{polinomio!grado}%
\index{grado!polinomio}% 
Se $n=0$ diremo anche che il polinomio è costante.
Se $n=0$ e $a_n=0$ diremo che $P$ è il polinomio \emph{nullo}.
I polinomi costanti non nulli hanno grado pari a $0$, mentre il polinomio nullo, 
per convenzione, diremo avere grado $-\infty$ (questa definizione ha senso se si pensa 
al grado di un polinomio come al più piccolo numero per cui tutti i coefficienti di
indice maggiore a lui sono nulli).

Finora abbiamo definito il polinomio come una classe di equivalenza di 
espressioni polinomiali. 
Se prendiamo una espressione $P$ e al posto della variabile $x$ 
mettiamo un elemento $a$ dell'anello $A$ (o di qualunque anello che estende 
$A$) allora possiamo valutare tutte le operazioni (somme e prodotti) ed 
ottenere un risultato nell'anello scelto: il risultato 
ottenuto si indica con $P(a)$ (stessa notazione utilizzata per le funzioni).
Se $P$ è un polinomio nella variabile $x$ la funzione $x\mapsto P(x)$ 
può essere chiamata \emph{funzione polinomiale}%
\mymargin{funzione polinomiale}%
\index{funzione!polinomiale} e si indica usualmente 
con con lo stesso nome $P$ dato al polinomio. Sarà il contesto a dirci 
se con $P$ si intende il polinomio (l'espressione) o la funzione.

Come già anticipato è utile tenere separato il concetto di polinomio da quello di 
funzione polinomiale in quanto lo stesso polinomio può essere valutato su diversi 
anelli (ad esempio nel corso di geometria si potrà sostituire
la variabile $x$ di un polinomio con una matrice, 
e nell'ultimo capitolo di questi appunti 
sostituiremo la $x$ con un operatore differenziale).

Proseguiremo con la trattazione dei polinomi 
nel capitolo~\ref{ch:ancora_polinomi}.

\section{coefficienti binomiali}
\label{ch:binomiale}

Capiterà spesso di imbattersi in alcuni polinomi particolari, che vengono 
a volte chiamati \emph{prodotti notevoli}%
\mymargin{prodotti notevoli}%
\index{prodotto!notevole}. 
Di fondamentale iportanza quelli di grado 2:
\[
  (x+1)\cdot(x-1) = x^2-1, \qquad (x+1)^2 = x^2+2x + 1.
\]
Ma anche quelli di grado $n$:
\[
  (x-1)\cdot \sum_{k=0}^{n-1} x^k
  = \sum_{k=0}^{n-1} x^{k+1} - \sum_{k=0}^n x^k = x^n - 1
\]
e, mettendo $-x$ al posto di $x$ e cambiando di segno:
\[
  (x+1)\cdot \sum_{k=0}^{n-1} (-1)^k x^k
  = 1 + (-1)^n x^n.
\]

Le potenze del binomio $x+1$ sono decisamente rilevanti. 
Se scriviamo il polinomio $(x+1)^n$ in forma canonica:
\[
  (x+1)^n = \sum_{k=0}^n a_k x^k  
\]
saremmo interessati a calcolare il valore
dei coefficienti $a_k$. 
Innanzitutto gli diamo un nome: questi coefficienti vengono chiamati 
\emph{coefficienti binomiali}%
\mymargin{coefficienti binomiali}%
\index{coefficienti binomiali}
\index{binomio!coefficienti}%
e si denotano nel modo seguente:%
\mynote{%
I coefficienti binomiali nell'ambito del calcolo combinatorio 
vengono anche chiamati \emph{combinazioni}
e denotati con il simbolo $C_{n,k} = {n \choose k}$.
La coincidenza tra le combinazioni e i coefficienti binomiali 
è espressa nel teorema~\ref{th:combinatoria}.
}%
\[
    a_k = {n \choose k}.
\]

Non è difficile convincersi che se sviluppiamo il binomio $(x+y)^n$
si ottengono gli stessi coefficienti dunque
in generale vale la seguente formula per la \emph{potenza del binomio}%
\mymargin{potenza del binomio}%
\index{potenza!del binomio}:
\index{binomio!potenza}% 
\begin{equation*}
  (x+y)^n = \sum_{k=0}^n {n \choose k} x^k y^{n-k}. 
\end{equation*}

Per determinare il valore effettivo dei coefficienti 
binomiali si utilizza usualmente il seguente.
  
\begin{theorem}[triangolo di Tartaglia]
\mymark{*}%
\label{th:tartaglia}%
Per ogni $n\in \NN$ e $k \in \NN$ con $1 \le k \le n$ si ha
\[
  {n+1 \choose k} =
      {n \choose k-1} + {n \choose k}
\]
mentre
\[
  {n+1 \choose 0} = 1 = {n+1 \choose n+1}
\]
\end{theorem}
  %
  \begin{proof}
  Infatti si ha 
  \begin{align*}
    \sum_{k=0}^{n+1} {n+1 \choose k} x^k
    &= (1+x)^{n+1} 
    = (1+x)\cdot \sum_{k=0}^n {n \choose k} x^k \\
    &= \sum_{k=0}^n {n\choose k} x^k 
    + \sum_{k=0}^n {n\choose k} x^{k+1}\\
    &= \sum_{k=0}^n {n\choose k} x^k 
    + \sum_{k=1}^{n+1} {n\choose k-1} x^{k}\\
    &= {n\choose 0} x^0 
      + \sum_{k=1}^n \Enclose{{n\choose k} + {n\choose k-1}} x^k
      + {n \choose n} x^{n+1}.
    \end{align*}
  Per l'unicità della forma canonica dei polinomi 
  i coefficienti corrispondenti devono essere uguali e quindi 
  troviamo
  \[
  {n+1 \choose 0} = {n \choose 0}, \qquad 
  {n+1 \choose k} = {n \choose k} + {n \choose k-1}, \qquad 
  {n+1 \choose n+1} = {n \choose n}.
  \]

  Si può facilmente verificare che  ${0 \choose 0} = 1$ 
  e questo conclude la dimostrazione.
  \end{proof}
  
  In base al teorema precedente i coefficienti binomiali si possono
  elencare come nella tabella~\ref{tab:binomiali}:
  ogni riga inizia e finisce con il numero $1$
  e ogni termine intermedio coincide con la somma dei
  due termini nella riga precedente sopra e
  a sinistra del numero considerato.
  
  \begin{table}
  \begin{tabular}{c|ccccccccc}
  $\displaystyle{n \choose k}$& 0 & 1 & 2 & 3 & 4 & 5 & 6 & $k$ &\\ \hline
    0 & 1 &   &   &   &   &   &   & &\\
    1 & 1 & 1 &   &   &   &   &   & &\\
    2 & 1 & 2 & 1 &   &   &   &   & &\\
    3 & 1 & 3 & 3 & 1 &   &   &   & &\\
    4 & 1 & 4 & 6 & 4 & 1 &   &   & &\\
    5 & 1 & 5 & 10& 10& 5 & 1 &   & &\\
    6 & 1 & 6 & 15& 20& 15& 6 & 1 & &\\
  $n$ &$\vdots$&&   &   &   &   &   & $\ddots$ &
  \end{tabular}
  \caption{Il triangolo di Tartaglia (o di Pascal):
  ogni numero è la somma dei due numeri che si trovano 
  nella riga precedente sopra e immediatamente a sinistra.}
  \label{tab:binomiali}
  \end{table}
  
  \begin{theorem}[formula per i coefficienti binomiali]
  \mymark{***}%
  Se $n\in \NN$ e $k\in \NN$, $k\le n$
  si ha 
  \[
  {n \choose k}  
  = \frac{n!}{k!(n-k)!}.  
  \]
  \end{theorem}
  %
  \begin{proof}
  Lo dimostriamo per induzione su $n$.
  Per $n=0$ sappiamo che ${0 \choose 0} = 1$ che 
  è ugule a $\frac{0!}{0!0!}$.
  Utilizziamo il teorema~\ref{th:tartaglia}.
  Se $k=0$ o $k=n$ sappiamo che ${n \choose k}=1$ che coincide 
  con la formula enunciata. 
  Per gli altri casi, supponendo per induzione che la formula sia 
  vera per un certo $n\in \NN$ si ha: 
  \begin{align*}
   {n+1 \choose k} 
   &= {n \choose k} + {n \choose k-1}
   =  \frac{n!}{k!(n-k)!} + \frac{n!}{(k-1)!(n-k+1)!} \\
   &= \frac{n!(n-k+1) + n! k}{k!(n-k+1)!}
   = \frac{n!(n+1)}{k!(n+1-k)!} \\
   &= \frac{(n+1)!}{k!(n+1-k)!}
  \end{align*}
  che è quanto dovevamo dimostrare.
\end{proof}

Si osservi che risulta, per ogni $n\in \NN$:
\[
  {n \choose 1} = {n \choose n-1} = n.
\]
  
\begin{exercise}
  Provare che
  \[
   \sum_{k=0}^n {n \choose k} = 2^n.
  \]
\end{exercise}  
  
\section{frazioni e numeri razionali}

Così come abbiamo esteso i numeri naturali affinché l'operazione di addizione abbia 
una operazione inversa (la sottrazione) in modo simile possiamo estendere l'insieme 
dei numeri interi in modo che anche la moltiplicazione abbia una operazione inversa 
(la divisione).

Per fare questo consideriamo i ``riscalamenti'' su $\ZZ$. Dati $p,q\in \ZZ$ se 
$q\neq 0$ possiamo considerare la funzione $s$ definita sui multipli di $q$ 
tale che $s(k\cdot q) = k\cdot p$ per ogni $k\in \ZZ$.
Si tratta del riscalamento che manda $q$ in $p$ e lo possiamo identificare 
con la frazione $\frac{p}{q}$ (formalmente una frazione non è altro che una 
coppia di interi $(q,p)$ con $q\neq 0$).
Osserviamo però che frazioni diverse possono rappresentare lo stesso riscalamento 
nell'intersezione dei loro domini. 
Infatti i due riscalamenti rappresentati dalle frazioni $\frac{p}{q}$ e $\frac{p'}{q'}$
sono definiti entrambi sul numero $q\cdot q'$ e si ha
$\frac{p}{q}(q\cdot q') = p\cdot q'$ e  $\frac{p'}{q'}(q\cdot q')=p'\cdot q$ per cui se $p\cdot q'=p'\cdot q$
i due riscalamenti coincidono in un punto. Ma se coincidono in un punto coincidono 
in tutti i punti in cui sono entrambi definiti. 
Diremo in tal caso che le due frazioni sono equivalenti e scriveremo 
\[
  \frac{p}{q} \sim \frac{p'}{q'} \iff p\cdot q'=p'\cdot q.  
\]

I \emph{numeri razionali} non sono altro che le frazioni identificate 
a meno di equivalenza:
\[
  \QQ = ((\ZZ\setminus\ENCLOSE{0})\times \ZZ)/\sim.
\]

La moltiplicazione tra numeri razionali può essere definita tramite la composizione 
dei riscalamenti:
\[
  \frac{p}{q}\enclose{\frac{p'}{q'} (q\cdot q')} = \frac{p}{q}(p'\cdot q) = p\cdot p'
\]
da cui 
\[
 \frac{p}{q} \cdot \frac{p'}{q'} = \frac{p\cdot p'}{q\cdot q'}.  
\]
Si verifica facilmente che questa definizione \emph{passa al quoziente} 
nel senso che il prodotto di frazioni equivalenti è equivalente al prodotto 
delle frazioni. 
Dunque la moltiplicazione è ben definita su $\QQ$.

Le frazioni del tipo $\frac{q}{q}$ sono tutte equivalenti tra loro e rappresentano 
l'elemento neutro della moltiplicazione.
Le frazioni del tipo $\frac{p}{1}$ rappresentano la moltiplicazione per $p\in\ZZ$ 
e quindi identificano i numeri interi $\ZZ$ all'interno di $\QQ$.
Per poter estendere l'addizione da $\ZZ$ a $\QQ$ in modo da preservare 
la proprietà distributiva sarà quindi necessario che 
si abbia 
\[
  \enclose{\frac{p}{q}+\frac{p'}{q'}} (q\cdot q') = \frac{p}{q}(q\cdot q') + \frac{p'}{q'}(q\cdot q')
  = p\cdot q' + p'\cdot q
\]
da cui
\[
  \frac{p}{q} + \frac{p'}{q'} = \frac{p\cdot q'+p'\cdot q}{q\cdot q'}.  
\]
E' facile verificare che anche questa definizione \emph{passa al quoziente}
ovvero che la somma di frazioni equivalenti è equivalente alla somma delle 
frazioni dunque anche l'addizione in questo modo è ben definita su $\QQ$.

Abbiamo quindi esteso l'addizione e la moltiplicazione da $\ZZ$ a $\QQ$
e ora, su $\QQ$.
Rispetto all'addizione $\QQ$ risulta essere un gruppo commutativo (come lo era $\ZZ$)
in quanto ogni frazione $\frac{p}{q}$ ha come opposto $\frac{-p}{q}$ 
e l'addizione rimane associativa e commutativa.

Inoltre in $\QQ$ ogni elemento diverso da $0$
\mynote{$0\in \ZZ$ si identifica in $\QQ$ con la classe di equivalenza 
delle frazioni del tipo $\frac{0}{q}$ e $1$ è rappresentato 
dalle frazioni del tipo $\frac{q}{q}$.}
ha anche un inverso moltiplicativo, infatti se $p\neq 0$ e $q\neq 0$ si ha:
\[
 \frac{p}{q}\cdot \frac{q}{p} = \frac{p\cdot q}{q\cdot p} \sim \frac{1}{1}.  
\]
Dunque $\QQ\setminus\ENCLOSE{0}$ risulta essere a sua volta un gruppo 
moltiplicativo.

In maniera naturale diremo che una frazione $\frac{p}{q}$ è non-negativa 
se, visto come riscalamento, manda gli interi non-negativi in interi non-negativi. 
Questo succede se $p$ e $q$ non hanno segno opposto, ovvero se $p\cdot q\ge 0$.
Possiamo quindi definire un ordinamento su $\QQ$ 
ponendo $x\le y$ se $y-x$ è non-negativo e cioè, visto sulle frazioni,
\[
 \frac{p}{q} \le \frac{p'}{q'} \iff \frac{p'\cdot q-p\cdot q'}{q\cdot q'}\ge 0 
 \iff (p'\cdot q-p\cdot q')\cdot q\cdot q' \ge 0. 
\]
Questo ordinamento estende quello di $\ZZ$ e rende sia il gruppo additivo $\QQ$
che il gruppo moltiplicativo $\QQ^+ = \ENCLOSE{q\in \QQ\colon q>0}$
un gruppo totalmente ordinato. 

In conclusione, visto che vale la proprietà distributiva,  
$\QQ$ risulta essere un campo ordinato. 
*** TO BE COMPLETED ***


\section{costruzione dei numeri reali}
\label{sec:costruzione_reali}

Osserviamo che l'insieme 
\[
 A = \ENCLOSE{x\in \QQ \colon x^2 \le 2}  
\]
non ha massimo. 
Infatti se $x\in A$ non può essere $x^2=2$ (per il teorema~\ref{th:pitagora})
e quindi dovrà essere $x^2<2$. 
Ma possiamo dimostrare che se $x^2<2$ esiste $\eps>0$ tale per cui $(x+\eps)^2<2$. 
Infatti si ha $(x+\eps)^2 = x^2 + 2\eps x + \eps^2$
ma se $x^2<2$ certamente dovrà essere $x<2$ (altrimenti $x^2\ge 4$)
e quindi $x^2+2\eps x+\eps^2 < x^2+ 4\eps^2+\eps^2 = x^2+5\eps^2$.
Se ora scegliamo $\eps<1$ sappiamo che $\eps^2<\eps$ 
e quindi $x^2+5\eps^2 < x^2 + 5\eps$. 
E affinché sia $x^2 + 5\eps < 2$ è sufficiente scegliere 
$\eps < \frac{2-x^2}{5}$. 
In tal modo abbiamo dimostrato che se $x\in A$ allora esiste 
$\eps>0$ tale che $x+\eps \in A$. 
Questo significa che l'insieme $A$ non ha massimo.

L'idea è ora che una ipotetica soluzione positiva dell'equazione $x^2=2$ 
dovrebbe potersi approssimare, per difetto, con gli elementi dell'insieme $A$.
Allora vogliamo utilizzare l'insieme $A$ per rappresentare un tale ipotetico numero.

Procediamo dunque col definire (si ricordi la definizione~\ref{def:limitato}):
\[
\mathcal R = \ENCLOSE{A \in \mathcal P(\QQ)\colon A\neq \emptyset, \text{ $A$ superiormente 
limitato}}.
\]
Dati $A,B\in \mathcal R$ scriveremo $A\sim B$ 
se $A$ e $B$ hanno gli stessi maggioranti, ovvero
se per ogni $q\in \QQ$ si ha $q\ge A \iff q\ge B$.
Chiaramente $\sim$ è una relazione di equivalenza (vedi definizione~\ref{def:equivalenza})
l'insieme degli insiemi equivalenti ad $A$ si chiama 
classe di equivalenza e viene denotata con $\Enclose{A}_\sim$.
Possiamo quindi definire il quoziente
\[
  \RR = \mathcal R / \sim = \ENCLOSE{\Enclose{A}_\sim\colon A\in \mathcal R},
  \qquad 
  \Enclose{A}_\sim = \ENCLOSE{B\in \mathcal R\colon B\sim A}
\]
che chiameremo insieme dei \emph{numeri reali}%
\mymargin{numeri reali}%
\index{numeri!reali}.
L'idea di questa definizione è che i numeri reali 
possono essere identificati dalle loro approssimazioni 
per difetto tramite numeri razionali. 
Diverse approssimazioni, però, possono rappresentare 
lo stesso numero reale.

\begin{exercise}
  Si mostri che si ha $A \sim B \sim C$ se
  \[
  A = \ENCLOSE{1}, \qquad 
  B = \ENCLOSE{x\in \QQ\colon x<1}, \qquad
  C = \ENCLOSE{\frac{n}{n+1}\colon n\in \NN}.
  \]
\end{exercise}

\begin{exercise}
  Si mostri che si ha $A\sim B$ se
  \[
  A = \ENCLOSE{x\in \QQ\colon x^2<2}, \qquad
  B = \ENCLOSE{\frac{n}{10^k}\colon n\in \NN, k\in \NN, n^2 \le 2\cdot 100^k < (n+1)^2}.  
  \]
  Verificare che 
  \[
    B \supset \ENCLOSE{
      1,\frac{14}{10},\frac{141}{100},\frac{1414}{1000},
      \frac{14142}{10000}}.
  \]
\end{exercise}

Dati $A,B\in\mathcal R$ definiamo $A+B=\ENCLOSE{a+b\colon a\in A, b\in B}$.
Chiaramente $A+B\in \mathcal R$ inoltre non è difficile dimostarare che 
se $A\sim A'$ e $B\sim B'$ si ha $A+B\sim A'+B'$.
Questo significa che è ben definita l'addizione su $\RR$ in modo che 
per ogni $A,B \in \mathcal R$ si abbia:
\[
\Enclose{A}_\sim + \Enclose{B}_\sim = \Enclose{A+B}_\sim.
\]
Ovviamente l'addizione su $\RR$ è associativa e commutativa in quanto 
queste proprietà valgono su $\QQ$ e di conseguenza su $\mathcal R$:
$(A+B)+C = \{ a+b+c\colon a\in A, b\in B, c\in C\} = A + (B+C)$ e 
$A+B = \{a+b\colon a\in A, b\in B\} = B+A$.
Visto che $A+\ENCLOSE{0} = A$ notiamo anche che $\Enclose{\ENCLOSE{0}}_\sim$ 
è elemento neutro dell'addizione.

Dato $A\in \mathcal R$ denotiamo con $A'$ l'insieme dei maggioranti di $A$.
Vogliamo dimostrare che $A$ e $A'$ contengono punti arbitrariamente vicini:
\begin{equation}\label{eq:48023775}
\forall \eps\in\QQ\colon (\eps >0 \implies \exists a \in A\colon \exists b\in A'\colon
b-a < \eps).
\end{equation}
Per fare ciò prendiamo qualunque $c\in A$ (visto che $A\neq \emptyset$)
e qualunque $q\in A'$ (visto che $A$ è superiormente limitato, $A'\neq \emptyset$).
L'insieme $I=\ENCLOSE{n\in \NN \colon c+n\eps\in A'}$
non è vuoto in quanto se $n>\frac{q-c}\eps$ si ha $c+n\eps>q$ ed essendo $q$ 
un maggiorante di $A$ anche $c+n\eps$ lo è.
Dunque per il buon ordinamento di $\NN$ (teorema~\ref{th:buon_ordinamento})
l'insieme $I$ ha minimo, che chiamiamo $k$.
Se $k=0$ significa che $c\in A'$ e posto $a=b=c$ si ottiene 
che \eqref{eq:48023775} è banalmente verificata.
Altrimenti prendiamo $b=c+k\eps$.
Per la scelta di $k$ sappiamo che $b\in A'$ e $c+(k-1)\eps$ non è un maggiorante di 
$A$. 
Ma allora esiste $a\in A$ tale che $a>c+(k-1)\eps$ da cui 
$b-a < \eps$ e \eqref{eq:48023775} è ancora verificata.

Possiamo ora osservare che posto $B=-A' = \{-x\in \QQ \colon x$ maggiorante di $A\}$
si ha $(A+B)\sim\ENCLOSE{0}$. 
I maggioranti di $\ENCLOSE{0}$ sono i $q\in \QQ$ con $q\ge 0$, 
vogliamo dimostrare che anche $A+B$ ha gli stessi maggioranti.
Ma se $a\in A$ e $b\in B$ risulta che $-b\ge a$ ovvero $a+b\le 0$ 
dunque $0$ è un maggiorante di $A+B$ e qualunque $q\ge 0$ lo è a maggior ragione.
Se invece $q<0$ per la proprietà \eqref{eq:48023775} esistono $a\in A$ 
e $-b\in A'$ (cioè $b\in B$) tali che $(-b)-a<-q$ cioè $a+b>q$. Dunque 
$q<0$ non è un maggiorante di $A+B$ e concludiamo che $A+B\sim\ENCLOSE{0}$.

Questo dimostra che dato qualunque $x\in \RR$ se $x=\Enclose{A}_\sim$
posto $y=\Enclose{-A'}$ si ha $x+y=\Enclose{0}_\sim$.
Dunque ogni $x\in \RR$ ha opposto $y$ che denoteremo 
con $y=-x$.

Abbiamo quindi dimostrato che $\RR$ è un gruppo rispetto all'operazione 
di addizione. L'ordinamento di $\QQ$ può essere facilmente esteso a $\RR$
definendo $\Enclose{A}_\sim \le \Enclose{B}_\sim$ se ogni maggiorante di $B$ 
è anche maggiorante di $A$. Questa definizione non dipende dalla scelta 
di $A$ e $B$ nella loro classe di equivalenza perché all'interno della classe di 
equivalenza l'insieme dei maggioranti 
non cambia.
Chiaramente $x\le x$, se $x\le y$ e $y\le x$ si ha $x=y$ 
e se $x\le y$, $y\le z$ si ha $x\le z$: dunque $\le$ è un ordinamento su $\RR$.
E' inoltre un ordinamento totale perché dati $x=\Enclose{A}_\sim$ e 
$y=\Enclose{B}$ se $x\neq y$ significa che l'insieme dei maggioranti 
di $A$ è diverso dall'insieme dei maggioranti di $B$. 
Supponiamo allora che esista $q$ che è maggiorante di $A$ 
ma non maggiorante di $B$ (se succede il viceversa basta scambiare $A$ e $B$).
In tal caso $B\ge A$ in quanto ogni maggiorante di $B$ deve essere maggiore 
di $q$ e quindi deve essere anche un maggiorante di $A$ da cui $x\le y$. 

L'ordinamento è compatibile con l'addizione perché se $\Enclose{A}_\sim 
\le \Enclose{B}_\sim$ e $A,B,C\in \mathcal R$ 
allora se $q$ è maggiorante di $B+C$ per ogni $c\in C$ si ha che $q-c$ 
è maggiorante di $B$ e dunque $q-c$ è anche maggiorante di $A$.
Ma allora $q$ è maggiorante di $A+C$. Dunque se $x\le y$ e $x,y,z\in \RR$ 
si ha $x+z\le y+z$ e $\RR$ è un gruppo ordinato.

L'ordinamento di $\RR$ è denso (definizione~\ref{def:ordinamento_denso}). 
Infatti se $x<y$ posto $y-x=\Enclose{A}_\sim$ 
sappiamo che ogni maggiorante di $A$ è maggiore o uguale a $0$.
Ma se ogni numero positivo fosse maggiorante di $A$ avremmo $A\sim\ENCLOSE 0$
che non è possibile in quanto $x\neq y$. Dunque esiste $q>0$ 
che non è maggiorante di $A$. 
Ma allora $y-x \ge \Enclose{\ENCLOSE q}_\sim > \Enclose{\ENCLOSE{\frac q 2}}_\sim = \eps > 0$
da cui $x<x+\eps<y$.

\begin{theorem}  
L'ordinamento di $\RR$ è continuo
(definizione~\ref{def:ordinamento_continuo}).
\end{theorem}
\begin{proof}
Siano dati $\mathcal A$ e $\mathcal B$ sottoinsiemi non vuoti di $\RR$ 
con $\mathcal A\le \mathcal B$. 
Possiamo definire
\[
  C = \bigcup\ENCLOSE{A\in \mathcal R\colon \Enclose{A}_\sim \in \mathcal A} 
    = \ENCLOSE{q\in \QQ\colon \exists A\in \mathcal R\colon 
    (\Enclose{A}_\sim \in \mathcal A) \land (q\in A)}.
\]
Preso qualunque $a=\Enclose{A}_\sim \in \mathcal A$ risulta 
ovviamente $A\subset C$. In particolare $C$ non è vuoto.
Mentre se $b=\Enclose{B}_\sim \in \mathcal B$ 
preso qualunque $q\in C$ si ha $q\in A$ per un qualche $A$ tale che 
$\Enclose{A}_\sim = a\in \mathcal A$.
Ma visto che $a\le b$, ogni maggiorante di $B$ è anche maggiorante di $A$.
In particolare preso un qualunque $r$ maggiorante di $B$
risulta che $r$ è un maggiorante di $C$.
Allora $C\in \mathcal R$ e posto $c=\Enclose{C}_\sim$ 
risulta che ogni maggiorante di $C$ è maggiorante di 
ogni $A$ con $\Enclose{A}_\sim \in \mathcal A$ mentre 
ogni maggiorante di un qualunque $B$ con $\Enclose{B}_\sim\in \mathcal B$ 
è anche un maggiorante di $C$.
Questo significa che $c\ge a$ per ogni $a\in A$ 
e $b\ge c$ per ogni $b\in B$: la dimostrazione è conclusa.
\end{proof}

Ad ogni $q\in \QQ$ si può associare l'elemento $\Enclose{\ENCLOSE q}_\sim$ 
di $\RR$. Ovviamente se $q\neq s$ in $\QQ$ allora $\ENCLOSE{q}$ 
ed $\ENCLOSE{s}$ non hanno gli stessi maggioranti e quindi si ottengono 
elementi distinti di $\RR$. Identificando $q$ con $\Enclose{\ENCLOSE q}_\sim$ 
possiamo quindi pensare che $\QQ \subset \RR$.

Prendiamo però l'insieme $A=\{x\in \QQ\colon x^2<2\}$.
Se fosse $A\sim \ENCLOSE q$ per qualche $q\in \QQ$ dovremmo 
concludere che $q$ è un maggiorante di $A$ e che ogni $s<q$ 
non è maggiorante di $A$. 
Ma allora dovrebbe essere $q^2\le 2$ perché se fosse $q^2>2$
potremmo trovare $s<q$ tale che anche $s^2>2$ 
(basta notare che $q<2$ e prendere $s=q-\eps$ con $\eps>0$, $\eps<1$, 
$\eps<\frac{q^2-2}{4}$ così $s^2=(q-\eps)^2 > q^2-2\eps q > q^2-4\eps
> q^2-(q^2-2) = 2$).
Dunque $\RR\neq \QQ$, l'elemento $x=[A]_\sim$ di $\RR$ sarà 
proprio quello che chiameremo $\sqrt 2$ e vedremo che 
risolverà l'equazione $x^2=2$. 

\subsection{isomorfismi di gruppi ordinati}

\begin{definition}[omomorfismo]%  
\label{def:omomorfismo}%
Siano $R$ ed $S$ due gruppi con operazione $*$ su $R$ e operazione $\circ$ su $S$.
Diremo che una funzione $\phi\colon R\to S$ è un \emph{omomorfismo}%
\mymargin{omomorfismo}%
\index{omomorfismo} se per ogni $x,y \in R$
\[
  \phi(x*y) = \phi(x)\circ \phi(y).
\]
\end{definition}

Sull'insieme $\RR$ è possibile definire la moltiplicazione per 
un numero $n$ naturale come somma ripetuta ed è possibile definire anche la divisione in $n$
parti uguali (teorema~\ref{th:divisibile}) e di conseguenza la moltiplicazione per ogni numero razionale.
Allora se fissiamo $u\in \RR$ che rappresenti l'unità, è possibile immergere $\QQ$ dentro $\RR$
associando $q\in \QQ$ al numero $q\cdot u\in \RR$.
E' quello che facciamo nel seguente.

\begin{lemma}
Sia $R$ un gruppo totalmente ordinato, denso, continuo e 
sia $\QQ$ il gruppo dei numeri razionali con l'operazione 
di addizione.
Allora fissato $u\in R$ esiste un unico omomorfismo $\phi\colon \QQ \to R$ 
tale che $\phi(1)=u$.

Se l'operazione su $R$ viene denotata con il simbolo di addizione $+$ 
l'omomorfismo $\phi$ può essere naturalmente denotato con il simbolo di moltiplicazione:
\[
   \phi(q) = q\cdot u
\]
cosicché la proprietà di omomorfismo è semplicemente la proprietà distributiva 
$(q+s)\cdot u = q\cdot u + s\cdot u$.

Denotiamo con $\QQ\cdot u$ l'immagine di $f$ ovvero
\[
  \QQ \cdot u = \ENCLOSE{q\cdot y \colon q\in \QQ}.
\] 
\end{lemma}
%
\begin{proof}
Usiamo il simbolo $+$ per denotare l'operazione del gruppo $R$ e chiamiamo $0$ 
l'elemento neutro di $R$.

\end{proof}

\begin{definition}[funzioni monotòne]
  \label{def:monotonia}%
  \mymark{***}%
  \mymargin{funzioni monotòne}%
  \index{funzione!monotòna}%
  \index{monotonia}%
  Una funzione $f\colon A \to B$ con $A,B$ insiemi ordinati, si
  dice essere
  \begin{enumerate}
  \item \emph{crescente}%
\mymargin{crescente}%
\index{crescente} se $x \le y \implies f(x) \le f(y)$;
  \item \emph{decrescente}%
\mymargin{decrescente}%
\index{decrescente} se $x \le y \implies f(x) \ge f(y)$;
  \item \emph{monotòna}%
\mymargin{monotòna}%
\index{monotòna} se crescente o decrescente;
  \item \emph{costante}%
\mymargin{costante}%
\index{costante} se crescente e decrescente;
  \item \emph{strettamente crescente}%
\mymargin{strettamente crescente}%
\index{strettamente!crescente} se $x<y \implies f(x) < f(y)$;
  \item \emph{strettamente decrescente}%
\mymargin{strettamente decrescente}%
\index{strettamente!decrescente} se $x<y \implies f(x) > f(y)$;
  \item \emph{strettamente monotòna}%
\mymargin{strettamente monotòna}%
\index{strettamente!monotòna} se strettamente crescente o strettamente decrescente.
  \end{enumerate}
\end{definition}

Si osservi che se $f\colon A \to \RR$ è costante allora esiste $c$ tale che
$f(x)=c$ per ogni $x\in A$. 
Si osservi anche che ogni funzione strettamente monotona è anche iniettiva. Si osservi infine (fare un esempio!) che esistono funzioni che non rientrano in nessuna delle categorie sopra elencate (cioè che non sono né crescenti né decrescenti).

Si faccia attenzione alla terminologia.
In alcuni testi (in particolare nei testi anglosassoni) si utilizza il termine
\emph{crescente} con il significato di \emph{strettamente crescente} 
e si usa la dizione \emph{non decrescente} o \emph{debolmente crescente} 
per indicare il concetto che noi abbiamo definito con \emph{crescente}. 
In effetti con le nostre definizioni una funzione crescente può essere costante
e quindi non crescere affatto!


Come conseguenza immediata del teorema precedente otteniamo il seguente corollario il quale
garantisce che gli assiomi con cui abbiamo definito $\RR$
lo identificano univocamente a meno di isomorfismi.

\begin{corollary}[unicità di $\RR$]%
  \label{th:unicitaR}%
  Sia $S$ un gruppo totalmente ordinato, denso e continuo (come $\RR$)
  e scegliamo $u\in S$, $u>0$ come unità.
  Allora $S$ è isomorfo ad $\RR$ nel senso che esiste una funzione (unica)
  $f\colon \RR\to S$ bigettiva, additiva, crescente con $f(1)=u$.
\end{corollary}

Nella definizione assiomatica di $\RR$ non abbiamo mai assunto che l'addizione fosse 
commutativa perché in effetti possiamo dimostrare che deve esserlo, come si vede nel seguente.
\begin{theorem}
Ogni gruppo totalmente ordinato, denso e continuo 
è commutativo.
\end{theorem}
%
\begin{proof}
Se consideriamo il gruppo additivo $R$ ma definiamo 
l'addizione con gli addendi scambiati, 
fissato $u>0$ in $R$
per il teorema~\ref{th:isomorfismo}
deve esistere una funzione $f\colon R\to R$ crescente, con $f(1)=1$, tale che 
\begin{equation}\label{eq:49613244}
  f(x+y) = f(y)+f(x).
\end{equation}
Inoltre $f$ è l'unica funzione crescente tale che 
per ogni $x\in \QQ\cdot u$ soddisfa $f(x)=x$. 
Visto che la funzione identica ha queste proprietà 
$f$ dovrà in effetti essere la funzione identica $f(x)=x$ 
per ogni $x\in \RR$ 
e l'equazione~\eqref{eq:49613244} si riduce a $x+y=y+x$.
\end{proof}

OMISSIS

L'operazione di moltiplicazione che abbiamo appena definito su $\RR$ 
rende $\RR$ un campo ordinato continuo in base alla definizione~\ref{def:campo}.

Si può osservare che in un campo ordinato l'ordinamento 
è necessariamente denso in quanto dati $a,b$ elementi 
del campo possiamo definire $\frac{a+b}{2}$ che è certamente 
strettamente compreso tra $a$ e $b$.

\section{punti all'infinito}
\label{sec:reali_estesi}
%%%%%%%%%%%%%%%%%%%
%%%%%%%%%%%%%%%%%%%
%%%%%%%%%%%%%%%%%%%

\begin{definition}[reali estesi]
\mymargin{$\bar\RR$}%
\index{$\bar{\RR}$}
Denotiamo con $\bar \RR=\RR \cup \ENCLOSE{+\infty, -\infty}$ l'insieme dei numeri reali
\mymargin{$+\infty$, $-\infty$}%
\index{$+\infty$, $-\infty$}
a cui vengono aggiunti due ulteriori \emph{quantità} che chiameremo
\emph{infinite} e che denotiamo con $+\infty$ e $-\infty$.
Diremo che $x\in \bar \RR$ è \emph{finito} se $x\in \RR$.
\end{definition}


Estendiamo la relazione d'ordine imponendo che valga
\[
  -\infty \le x \le +\infty, \qquad \forall x \in \bar\RR.
\]

Estendiamo anche la addizione e moltiplicazione
tra reali estesi imponendo che valga per ogni $x\in \bar \RR$
\begin{gather*}
  x + (+\infty) = +\infty, \qquad \text{se $x\neq -\infty$}\\
  x + (-\infty) = -\infty, \qquad \text{se $x\neq +\infty$}\\
  x \cdot (+\infty) = +\infty, \qquad
  x \cdot (-\infty) = -\infty, \qquad \text{se $x>0$} \\
  x \cdot (+\infty) = -\infty, \qquad
  x \cdot (-\infty) = +\infty, \qquad \text{se $x<0$}.
\end{gather*}

Si definiscono anche:
\[
 -(+\infty) = -\infty, \qquad
 -(-\infty) = +\infty, \qquad
 \frac{1}{+\infty} = \frac{1}{-\infty}=0
\]
facendo però attenzione che
questi formalmente non sono \emph{opposto}
e \emph{reciproco} in quanto
su $\bar \RR$ non sono più garantite
le regole: $x + (-x) = 0$ e $x \cdot (1/x) = 1$.
Infatti
le operazioni $(+\infty) + (-\infty)$ e $+\infty \cdot 0$ vengono
lasciate indefinite.

Definiamo anche il valore assoluto: $\abs{+\infty} = \abs{-\infty} = +\infty$.

Possiamo infine definire la sottrazione e la divisione tramite
addizione e moltiplicazione:
\[
  x - y = x + (-y), \qquad \frac{x}{y} = x \cdot \frac{1}{y}.
\]

Possiamo definire gli operatori $\sup$ e $\inf$
anche sugli insiemi illimitati ponendo:
\begin{align*}
  \sup A = +\infty \qquad \text{se $A$ non è superiormente limitato}\\
  \inf A = -\infty \qquad \text{se $A$ non è inferiormente limitato}.
\end{align*}
Osserviamo infatti che su $\bar \RR$ la quantità $+\infty$
è maggiorante di qualunque insieme e $-\infty$ è minorante, dunque
queste definizioni mantengono su $\bar \RR$ le proprietà caratterizzanti:
l'estremo superiore è il minimo dei maggioranti e
l'estremo inferiore è il massimo dei minoranti.
Definiamo infine
\begin{align*}
  \sup \emptyset = -\infty\\
  \inf \emptyset = +\infty.
\end{align*}
Queste ultime definizioni possono essere comprese da un punto di vista
strettamente logico: ogni numero reale è sia maggiorante che minorante
dell'insieme vuoto, dunque il minimo dei maggioranti non esiste in $\RR$
ma in $\bar \RR$ è $-\infty$
e il massimo dei minoranti è $+\infty$.

\section{intervalli}

\begin{definition}[intervallo]
\label{def:intervallo}%
\mymargin{intervallo}%
\index{intervallo}%
Un insieme $I\subset \bar\RR$ si dice essere un \emph{intervallo}
se contiene tutti i punti intermedi:
\[
  \text{se $x, y \in I$ e $x<z<y$ allora $z \in I$.}
\]
\end{definition}
%
\begin{theorem}[caratterizzazione intervalli di $\RR$]
Sia $I\subset \RR$ un intervallo e siano $a=\inf I$, $b=\sup I$
i suoi estremi. Allora
$z\in I$ se $a < z < b$.
\end{theorem}
%
\begin{proof}
Se $I=\emptyset$ si ha $a>b$ e quindi nessuno $z$ verifica $a<z<b$.
Supponiamo $I\neq \emptyset$ e
sia $a < z < b$.
Visto che $a$ è il massimo dei minoranti di $I$
il numero $z$ non è un minorante dunque
deve esistere $x \in I$ tale
che $x < z$. Analogamente dovrebbe esistere $y\in I$
con $z<y$.
Ma allora, per definizione di intervallo, anche $z\in I$.
\end{proof}

Il teorema precedente ci dice che una volta identificati i due estremi
di un intervallo, tutti i punti intermedi devono stare nell'intervallo.
Gli estremi, invece, possono essere o non essere inclusi nell'intervallo.
Punti esterni agli estremi non possono invece essere elementi dell'intervallo.
Possiamo quindi caratterizzare tutti gli intervalli di $\RR$
introducendo le seguenti notazioni. Dati $a,b\in \bar \RR$ con $a\le b$
tutti i possibili intervalli con estremi $a$ e $b$ sono i seguenti:
\begin{equation}\label{eq:499494}
\begin{aligned}
\closeinterval{a}{b} &= \ENCLOSE{x\in \bar \RR\colon a \le x \le b} \\
\closeopeninterval{a}{b} &= \ENCLOSE{x\in \bar \RR\colon a \le x < b} \\
\opencloseinterval{a}{b} &= \ENCLOSE{x\in \bar \RR\colon a < x \le b}\\
\openinterval{a}{b} &= \ENCLOSE{x\in \bar \RR\colon a < x < b}.
\end{aligned}
\end{equation}
Abbiamo utilizzato le parentesi quadre per indicare che gli estremi
sono inclusi e le parentesi tonde per indicare che gli estremi sono esclusi.
Osserviamo che in alcuni testi si usano le parentesi quadre rovesciate al posto
delle parentesi tonde.

Se invece $a>b$ potremmo definire per convenzione:
\begin{equation}\label{eq:488364}
  [a,b] = [b,a], \quad
  [a,b) = (b,a], \quad
  (a,b] = [b,a), \quad
  (a,b) = (b,a).
\end{equation}
Si faccia però attenzione che in altri testi gli intervalli con gli estremi
scambiati non vengono definiti oppure vengono considerati vuoti.

La convenzione può essere utile perché in generale se $\vec a, \vec b$ sono
elementi di uno spazio vettoriale reale $V$ allora ha senso
definire:
\begin{align*}
    [\vec a,\vec b] &= \ENCLOSE{(1-t)\vec a + t \vec b\colon t\in [0,1]},\\
    [\vec a,\vec b) &= \ENCLOSE{(1-t)\vec a + t \vec b\colon t\in [0,1)},\\
    (\vec a,\vec b] &= \ENCLOSE{(1-t)\vec a + t \vec b\colon t\in (0,1]},\\
    (\vec a,\vec b) &= \ENCLOSE{(1-t)\vec a + t \vec b\colon t\in (0,1)}.
\end{align*}
L'intervallo $[\vec a,\vec b]$ è quindi il segmento di estremi
$\vec a$ e $\vec b$ e può essere definito anche se sullo spazio
vettoriale non è dato un ordinamento.
Questo rimane coerente con la definizione~\eqref{eq:499494}
data sopra solamente se adottiamo la convenzione~\eqref{eq:488364}.

Noi considereremo per lo più intervalli di $\RR$ (non di $\bar \RR$): in tal
caso gli estremi infiniti non potranno mai essere inclusi nell'intervallo.

\section{andamento del grafico di una funzione}
%
Se $f\colon A \subset \RR\to \RR$ è una funzione, un modo molto
utile di rappresentarla graficamente è quello di disegnarne il
grafico, ovvero la curva del piano cartesiano:
\[
   G_f = \ENCLOSE{(x,y)\in A\times \RR\colon y = f(x)}.
\]
Molte proprietà della funzione potranno essere riconosciute
geometricamente guardandone il grafico.

\begin{definition}[simmetrie]
Sia $f\colon A \subset \RR \to \RR$ una funzione.
Diremo che $f$ è:
\begin{enumerate}
\item \emph{pari}%
\mymargin{pari}%
\index{pari}
\index{funzione!pari}%
se $A=-A$ (significa che se $x\in A$ allora anche $-x\in A$) e
\[
  f(-x) = f(x);
\]
\item \emph{dispari}%
\mymargin{dispari}%
\index{dispari}
\index{funzione!dispari}%
se $A=-A$ e
\[
  f(-x) = -f(x);
\]
\item \emph{periodica}%
\mymargin{periodica}%
\index{periodico}
\index{funzione!periodica}%
di periodo $T$ se $A+T=A$
(significa che $x\in A \iff x+T \in A$)
se per ogni $x\in A$ si ha
\[
  f(x+T)=f(x)
\]
\end{enumerate}
\end{definition}

Ad esempio se $n\in \ZZ$ la funzione $f(x)=x^n$
è pari se $n$ è pari ed è dispari se $n$ è dispari.
Il grafico di una funzione dispari ha una simmetria
centrale, in quanto se $(x,f(x))\in G_f$ allora
anche $(-x,-f(x)) = (-x,f(-x))\in G_f$.
Il grafico di una funzione pari ha invece una
simmetria rispetto all'asse delle ordinate $x=0$
infatti se $(x,f(x))\in G_f$ allora $(-x,f(x)) = (-x,f(-x)) \in G_f$.

La funzione $f(x) = x - \lfloor x\rfloor$ (la parte frazionaria di $x$)
è un esempio di funzione periodica di periodo $T=1$. Infatti
è chiaro che $\lfloor x+1\rfloor = \lfloor x \rfloor +1$ e quindi
$f(x+1)=f(x)$.

Si osservi che \emph{dispari} per le funzioni non è la negazione
di \emph{pari}.
La funzione $f(x) = x+1$ non è né pari, né dispari, né periodica
(verificare).

\begin{definition}[zeri]
  Se $f\colon A\subset \RR \to \RR$ è una funzione diremo che
  $x\in A$ è uno \emph{zero} di $f$ se $f(x)=0$.
  L'\emph{insieme degli zeri}%
\mymargin{insieme degli zeri}%
\index{insieme!degli zeri}
  \index{zero!di una funzione}%
  è quindi dato da
  \[
    f^{-1}(\ENCLOSE{0}) = \ENCLOSE{x\in \RR\colon f(x) = 0}.
  \]
\end{definition}

Abbiamo già accennato al fatto che uno dei problemi più comuni in
matematica è quello di invertire una funzione. In particolare 
dato $y\in \RR$ ci si chiede quali siano gli $x\in \RR$ 
tali che $f(x)=y$. 
Questo problema si riconduce
a trovare gli zeri della funzione $f(x)-y$ e per questo motivo 
siamo interessati allo studio degli zeri.

Riprendiamo ora la definizione~\ref{def:monotonia} (monotonia) che da ora in avanti 
potrà essere applicata alle funzioni $f\colon A \to \RR$ definite 
su un insieme $A\subset \RR$.

Dal punto di vista grafico una funzione $f$ è crescente
se preso qualunque punto $(x,f(x))$ sul grafico della funzione
e tracciati gli assi paralleli agli assi cartesiani, passanti
per il punto fissato, si osserva che il grafico della funzione
è tutto contenuto nel primo e terzo quadrante determinati
dagli assi traslati.

E' facile verificare che la funzione $f\colon [0,+\infty)\to \RR$
definita da $f(x)=x^n$
è strettamente crescente se $n$ è un intero positivo.
Se però consideriamo la funzione definita su tutto
$\RR$: $f\colon \RR \to \RR$,
$f(x)=x^n$ allora solo se $n$ è dispari la funzione rimane
strettamente crescente
(le funzioni pari non possono mai essere strettamente crescenti se
il loro dominio contiene almeno tre punti distinti).

Se una funzione non è monotona è piuttosto comune studiare 
la monotonia della funzione ristretta a particolari intervalli: 
su alcuni intervalli la funzione (ristretta) potrà essere crescente e su altri 
intervalli potrà essere decrescente.

\begin{exercise}
Verificare che la composizione di funzioni monotone è una
funzione monotona e la composizione di funzioni strettamente
monotone è strettamente monotona.
Quando è che la funzione composta risulta crescente?
Quando decrescente?
\end{exercise}

\begin{exercise}
Si dimostri che applicando una funzione strettamente crescente ai due
membri di una equazione o disequazione (stretta o larga che sia)
si ottiene una equazione o disequazione equivalente.
Ovviamente è necessario che la funzione sia definita dove viene applicata.

Lo stesso vale per le funzioni strettamente decrescenti 
se però si cambia il verso della disequazione.
\end{exercise}

\begin{definition}[funzioni limitate, massimo/minimo]
\label{def:funzione_limitata}%
Se $f\colon A \to \RR$ è una funzione allora definiamo
l'estremo superiore di $f$ come l'estremo superiore
dell'immagine di $f$:
\[
  \sup f = \sup_{x\in A} f(x) = \sup f(A).
\]
In maniera analoga si definiscono l'estremo inferiore $\inf f$,
il massimo $\max f$ e il minimo $\min f$.

Dunque il massimo di una funzione è (se esiste) il valore massimo
che la funzione può assumere. I punti $x$ in cui
la funzione assume il valore massimo $f(x)$ vengono chiamati
\emph{punti di massimo}.
\mymargin{punto di massimo/minimo}%
\index{punto di massimo/minimo}%
\index{punto!di massimo}%
\index{punto!di minimo}%
Analogamente i punti in cui la funzione
assume il valore minimo (sempre che esistano) vengono
chiamati \emph{punti di minimo}.

Diremo che la funzione $f$ è
\emph{superiormente limitata}%
\mymargin{funzione superiormente limitata}%
\index{superiormente limitata}
se $\sup f<+\infty$
ovvero se esiste $M\in \RR$ tale che
\[
\forall x\in A \colon f(x) \le M.
\]
Diremo che la funzione $f$ è
\emph{inferiormente limitato}%
\mymargin{funzione inferiormente limitata}%
\index{inferiormente!limitato}
se $\inf f > -\infty$ ovvero se esiste $M\in \RR$ tale che
\[
 \forall x \in A \colon f(x) \ge M.
\]
Diremo che la funzione $f$ è \emph{limitata}%
\mymargin{funzione limitata}%
\index{limitata}
se è sia superiormente che inferiormente limitata ovvero
se $\sup\abs{f}<+\infty$ cioè se esiste $M\in \RR$ tale che
\[
\forall x \in A \colon \abs{f(x)}\le M.
\]
\end{definition}

Nel seguente esercizio abbiamo un esempio di funzione limitata.
\begin{exercise}
Si consideri la funzione $f\colon \RR\to\RR$
\[
 f(x) = \frac{1}{1+x^2}.
\]
Verificare che $\max f = \sup f = 1$, che $0$ è l'unico punto di massimo,
che $\inf f = 0$ e che $\min f$ non esiste.
\end{exercise}

\section{funzioni lineari}

\begin{figure}
  \begin{center}
    \begin{tikzpicture}[x=1.0cm,y=1.0cm]
    	\draw[->] (-1,0) -- (4,0);
      \draw[->] (0,-1) -- (0,3);
      % y = (3/5)*x + 1/5
      \fill[fill=lightgray,draw] (1,0.8) -- (2,0.8) node [midway,below]{$\Delta x$} -- (2,1.4) node [midway,right]{$\Delta f(x)$};
      \draw[thick] (-1,-0.4) -- (4,2.6);
      \draw[->] (1,0) node[below]{$x_1$} -- (1,0.8)
      -- (0,0.8) node[left]{$f(x_1)$};
      \draw[->] (2,0) node[below]{$x_2$} -- (2,1.4)
      -- (0,1.4) node[left]{$f(x_2)$};
      \fill (1,0.8) circle[radius=1.5pt];
      \fill (2,1.4) circle[radius=1.5pt];
      \node[left] at (3.0,2.5) {$y=mx+q$};
    \end{tikzpicture}
  \end{center}
  \caption{Il grafico di una funzione lineare.}
  \label{fig:funzione_lineare}
\end{figure}

Le \emph{funzioni lineari}%
\mymargin{funzione lineare}%
\index{funzioni lineari}
$f\colon \RR \to \RR$ sono le funzioni per le quali
esistono $m,q\in\RR$ tali che%
\mynote{%
Attenzione: nell'ambito dell'algebra lineare queste
funzioni verrebbero chiamate \emph{lineari affini}, mentre
le funzioni lineari dovrebbero sempre avere $q=0$.
Noi invece (come spesso accade nell'ambito dell'analisi)
chiameremo lineari queste funzioni e chiameremo
\emph{lineari omogenee} quelle con $q=0$.
Il termine \emph{lineare} pervade tutta la matematica 
e si applica in particolare alle equazioni che si ottengono 
tramite le funzioni lineari.
Purtroppo il nome scelto è fuorviante: la parola \emph{linea} viene 
usata a volte come abbreviazione di \emph{linea retta}, quando 
invece sarebbe più giusto utilizzare l'abbreviazione \emph{retta}
in quanto una linea può benissimo essere curva.
In altri contesti (come ad esempio nell'ambito degli ordinamenti)
il termine \emph{lineare} rappresenta un oggetto unidimensionale
senza ramificazioni ed è quindi maggiormente aderente 
al significato originale della parola.
} % marginnote
\[
  f(x) = mx + q.
\]

Se prendiamo due punti $(x_1,f(x_1))$
e $(x_2,f(x_2))$ sul grafico di una funzione lineare
possiamo osservare che si ha
\[
  \frac{f(x_2) - f(x_1)}{x_2 - x_1} = m.
\]
Il coefficiente $m$, dunque, rappresenta la pendenza del
grafico di $f$, ovvero il rapporto tra la variazione
dei valori della funzione $\Delta f = f(x_2) - f(x_1)$
e la variazione della variabile in ingresso
$\Delta x = x_2 - x_1$.
Geometricamente questo è il rapporto tra i due cateti
(base e altezza) che formano un triangolo rettangolo la
cui ipotenusa è il segmento che congiunge i due punti sul grafico.
Il fatto che questo rapporto sia costante significa,
in base al teorema di Talete, che i punti del grafico sono
allineati ovvero che il grafico di una funzione lineare è,
dal punto di vista geometrico, una retta.

\begin{definition}[retta]
  \index{retta}%
  \index{linea!retta}%
  Una \emph{linea retta} (più semplicemente: \emph{retta}) in è un sottospazio affine di dimensione 1 
  ovvero la traslazione di un sottospazio vettoriale di dimensione 1
  (si rimanda al corso di geometria).
\end{definition}

Tutte le rette del piano, 
tranne quelle parallele all'asse delle ordinate,
sono grafico di una funzione lineare.

Si osservi che per $m>0$ la funzione è strettamente crescente,
per $m=0$ la funzione è costante e per $m<0$ la funzione è
strettamente decrescente.

\section{funzioni quadratiche}
\label{sec:funzioni_quadratiche}

\begin{figure}
  \begin{center}
    \begin{tikzpicture}[x=1.0cm,y=1.0cm]
      \clip(-1,-1) rectangle (4,3);
    	\draw[->] (-1,0) -- (4,0);
      \draw[->] (0,-1) -- (0,3);
      % y = (3/5)*x + 1/5
      \draw[dashed] (1.75,-1) -- (1.75,4);
      \node[right] at (1.75,2.5){$x=-\frac b {2a}$};
      \draw[domain=-1.0:4.0,smooth,variable=\x,thick] plot
      ({\x}, {0.5*(\x-0.5)*(\x-3.0)});
      \fill (0.5,0.0) node[below]{$x_1$} circle[radius=1.5pt];
      \fill (3,0.0) node[below]{$x_2$} circle[radius=1.5pt];
      \node at (1.8,1) {$y=ax^2+bx+c$};
    \end{tikzpicture}
  \end{center}
  \caption{Il grafico di una funzione quadratica.}
  \label{fig:funzione_quadratica}
\end{figure}

Le funzioni espresse mediante un \emph{polinomio di secondo grado}%
\mymargin{polinomio di secondo grado}%
\index{polinomio!di secondo grado}
\begin{equation}\label{eq:funzione_quadratica}
  f(x) = ax^2 + bx +c
\end{equation}
con $a,b,c\in \RR$, $a\neq 0$, si possono chiamare
\emph{funzioni quadratiche}%
\mymargin{funzioni quadratiche}%
\index{funzione!quadratica}.

Il modello di funzione quadratica è la funzione
$f(x) = x^2$ che (come tutte le potenze di esponente positivo e pari)
risulta essere una funzione pari, strettamente crescente
sull'intervallo $[0,+\infty)$ e strettamente decrescente
su $(-\infty,0]$. La funzione assume solamente valori non negativi
e si annulla solo per $x=0$.
Dunque l'equazione
\[
  x^2 = b
\]
non ha soluzione se $b<0$ ed ha come unica soluzione $x=0$ se $b=0$.
Se $b>0$ sappiamo che
questa equazione ha una unica soluzione positiva $x_1 = \sqrt{b}$
e, per simmetria, ha anche una soluzione negativa $x_2 = -\sqrt{b}$.
Sintetizzando si usa scrivere $x_{1,2} = \pm \sqrt{b}$
per unire in una unica riga le due definizioni.

La generica funzione quadratica~\eqref{eq:funzione_quadratica}
può essere ricondotta al caso modello tramite un cambio
di variabile lineare. In pratica si cerca di comporre il quadrato
di un binomio con un procedimento chiamato
\emph{completamento del quadrato}%
\mymargin{completamento del quadrato}%
\index{completamento del quadrato}:
\begin{equation}\label{eq:24589}
\begin{aligned}
f(x) = ax^2+bx+c
  &= a \Enclose{x^2+\frac b a x + \frac c a}\\
  &= a \Enclose{x^2+2 \frac{b}{2a} x + \frac{b^2}{4a^2} - \frac{b^2}{4a^2} + \frac c a}\\
  &= a \Enclose{\enclose{x+\frac{b}{2a}}^2 - \frac{b^2-4ac}{4a^2}} \\
  &= a\enclose{x+\frac b{2a}}^2  - \frac{b^2-4ac}{4a}.
\end{aligned}
\end{equation}

Ponendo $X=x+\frac b{2a}$ e $Y=y+\frac{b^2-4ac}{4a}$
l'equazione $y=ax^2+bx+c$ diventa quindi $Y=aX^2$. 
Significa
che il grafico della funzione quadratica~\eqref{eq:funzione_quadratica}
si ottiene traslando la curva $y = a x^2$ che, 
dal punto di vista geometrico, si può facilmente
dimostrare essere una parabola con fuoco
nel punto di coordinate $\enclose{0,\frac 1 {4a}}$
e asse la retta di equazione $x=0$.
Dunque il grafico di ogni funzione quadratica è una parabola, 
e più precisamente: ogni parabola con direttrice parallela all'asse delle
ascisse è il grafico di una funzione quadratica.

\begin{definition}[parabola]
  Una parabola con fuoco nel punto $\vec F=(F_1,F_2)\in \RR\times \RR$ 
  e retta direttrice 
  la retta $r \subset \RR\times\RR$ 
  è l'insieme dei punti  $\vec x = (x_1,x_2)\in \RR\times\RR$ 
  equidistanti da $\vec F$ e da $r$:
  \[
  \ENCLOSE{\vec x\in \RR\times\RR\colon 
  \sqrt{(x_1-F_1)^2 + (x_2-F_2)^2} 
  = \inf_{\vec P\in r}\sqrt{(x_1-P_1)^2+(y_1-P_1)^2}
  }.
  \]
\end{definition}

\begin{exercise}
  Si dimostri che per ogni $a\neq 0$ esiste $s\neq 0$ 
  per cui il riscalamento $X=sx$, $Y=sy$ porta il grafico della 
  parabola $y=ax^2$ nel grafico della parabola $Y=X^2$.
  Significa che a meno di isometrie e riscalamenti c'è una 
  unica parabola.
\end{exercise}

Ricordando le proprietà di monotonia della funzione $X\mapsto X^2$
possiamo dedurre che se $a>0$ la funzione $f(x)$ è strettamente
decrescente se ristretta all'intervallo 
$\left(-\infty,-\frac b {2a}\right]$ ed è invece strettamente crescente 
sull'intervallo $\left[-\frac b{2a},+\infty\right)$. 
Ha dunque un punto di minimo in $x=-\frac{b}{2a}$.
Inoltre (sempre se $a>0$) la funzione è superiormente illimitata.
Viceversa se $a<0$ la funzione è inferiormente illimitata ed ha 
un massimo nel punto $x=-\frac{b}{2a}$.

E' molto importante saper risolvere equazioni e disequazioni
quadratiche. Grazie a~\eqref{eq:24589} l'equazione
\[
 a x^2 + bx + c = 0
\]
risulta equivalente a
\[
  \enclose{x+\frac{b}{2a}}^2 = \frac{b^2-4ac}{4a^2}.
\]
Dunque se $b^2-4ac<0$ l'equazione $ax^2+bx+c=0$ non ha soluzioni.
Se $b^2-4ac=0$ l'equazione ha una unica soluzione $x=-\frac{b}{2a}$.
Infine se $b^2-4ac>0$ si ottiene
\[
  x+\frac b{2a} = \pm \frac{\sqrt{b^2-4ac}}{2a}
\]
da cui la famosa formula risolutiva
\mymark{***}
\begin{equation}\label{eq:secondo_grado}
  x_{1,2} = \frac{-b \pm \sqrt{b^2-4ac}}{2a}.
\end{equation}

Risolvendo le disequazioni allo stesso modo, si trova
che la funzione $ax^2+bx+c$ quando $a>0$ è positiva
nei punti esterni alle soluzioni dell'equazione
(in tutti i punti se le soluzioni non esistono) ed
è negativa nei punti interni alle due soluzioni.
Viceversa se $a<0$ la funzione è positiva all'interno
delle due soluzioni e negativa all'esterno.

\section{funzione esponenziale e potenza}
\label{sec:esponenziale}

Osserviamo che l'insieme $\RR_+ = \ENCLOSE{x\in \RR\colon x>0}$ 
dei reali positivi rispetto alla operazione di moltiplicazione 
risulta essere un gruppo moltiplicativo totalmente ordinato, 
denso e continuo. 
Abbiamo già dimostrato tutte queste proprietà della moltiplicazione.
La densità e continuità non vanno verificate in quanto sono proprietà 
dell'ordinamento e non dell'operazione, e l'ordinamento considerato 
su $\RR_+$ è quello ereditato da $\RR$.

Dunque fissato $a>0$ per il teorema~\ref{th:isomorfismo}
esiste una unica funzione $\exp_a\colon \RR \to \RR_+$ 
(chiamata \emph{funzione esponenziale}%
\mymargin{funzione esponenziale}%
\index{funzione!esponenziale}%
\index{esponenziale}%
di base $a$)
che sia un omomorfismo monotono del gruppo additivo $\RR$ con il gruppo
moltiplicativo $\RR_+$ e tale che $\exp_a(1)=a$.
La proprietà di omomorfismo in questo caso 
si esprime nella forma seguente: 
\begin{equation}\label{eq:3589673}
  \exp_a(x+y) = \exp_a(x)\cdot \exp_a(y).
\end{equation}
Il teorema~\ref{th:isomorfismo}
ci dice anche che se $n\in \NN$ il valore $\exp_a(n)$ 
non è altro che il prodotto di $a$ 
con sé stesso $n$ volte: $\exp_a(n) = a^n$. 
Ha dunque senso definire 
$a^x = \exp_a(x)$ per ogni $x\in \RR$.
Avremo dunque la proprietà fondamentale della funzione 
esponenziale:
\[
  a^{x+y} = a^x\cdot a^y.
\]

Per ogni $a>0$ e $b\in \RR$ è dunque definita l'espressione $a^b$
(si legga: \emph{$a$ elevato alla potenza $b$}) che si chiama 
\emph{potenza}%
\mymargin{potenza}%
\index{potenza}%
di base $a$ ed esponente $b$.

Quando l'esponente è negativo si ha 
\[
  a^{-x} = \frac{1}{a^x}  
\]
in quanto $a^x\cdot a^{-x} = a^{x-x} = a^0 = 1$.
In particolare $a^{-1} = \frac{1}{a}$ è un modo per denotare  
l'inverso moltiplicativo.

Se $a>1$ 
la funzione $a^x$ risulta essere strettamente crescente, 
se $a<1$ la funzione è strettamente decrescente 
e se $a=1$ la funzione $1^x=1$ è costante.

Se $a>0$ e $b>0$ vogliamo dimostrare che vale la proprietà:
\[
  (a\cdot b)^x = a^x\cdot b^x.
\]
Sarà sufficiente dimostrarlo per $a>1$ e $b>1$ in quanto gli 
altri casi si ottengono di conseguenza passando al reciproco.
Osservare che la funzione $f(x)=a^x\cdot b^x$ 
è un omomorfismo
\[
f(x+y) = a^{x+y}b^{x+y} = a^x a^y b^x b^y =  f(x)\cdot f(y)
\]
è crescente (in quanto prodotto di funzioni positive e crescenti)
e vale $f(1) = a\cdot b$. 
Dunque per l'unicità data dal teorema~\ref{th:isomorfismo}
possiamo affermare che $f(x)=(a\cdot b)^x$.

Infine possiamo dimostrare la proprietà delle potenze ripetute:
\[
  \enclose{a^b}^c = a^{b\cdot c}.
\]
Fissati $a>0$ e $b\in \RR$ consideriamo la funzione $f(x) = a^{b\cdot x}$.
osserviamo che $f(x+y) = a^{bx+by}=a^{bx}\cdot a^{by}$ dunque 
la funzione $f$ soddisfa la proprietà di additività. 
Inoltre è una funzione monotona perché composizione di funzioni monotone.
Infine $f(1) = a^b$ e dunque, per l'unicità data dal teorema~\ref{th:isomorfismo}
dovrà essere $f(x) = \enclose{a^b}^x$, come volevamo dimostrare. 

\begin{figure}
  \begin{center}
    \begin{tikzpicture}[x=1.0cm,y=1.0cm]
      \clip(-4,-2) rectangle (4,3);
    	\draw[->] (-4,0) -- (4,0);
      \draw[->] (0,-2) -- (0,3);
      % y = (3/5)*x + 1/5
      \draw[dotted] (1,0) -- (1,2) -- (0,2);
      \draw[dotted] (2,0) -- (2,1) -- (0,1);
%      \node[right] at (1.75,2.5){$x=-\frac b {2a}$};
      \draw[domain=-2.0:4.0,smooth,variable=\x,dashed] plot
        ({\x}, {pow(2,-\x)});
      \draw[domain=0.1:4.0,smooth,variable=\x,dashed] plot
        ({\x}, {-ln(\x) / ln(2)});
      \draw[domain=-4.0:2.0,smooth,variable=\x,thick] plot
        ({\x}, {pow(2,\x)});
      \node at (2,2.5) {$y=a^x$};
      \node at (-2,2.1) {$y=\enclose{\frac 1 a}^x$};
      \fill (0,1.0) node[left]{\!\!$1$} circle[radius=1.5pt];
      \draw[domain=0.1:4.0,smooth,variable=\x,thick] plot
        ({\x}, {ln(\x) / ln(2)});
      \node at (3.2,1.1) {$y=\log_a x$};
      \node at (3.2,-1.1) {$y=\log_{\frac 1 a} x$};
      \fill (1,0) node[below]{$1$} circle[radius=1.5pt];
      \fill (0,2) node[left]{$a$} circle[radius=1.5pt];
      \fill (2,0) node[below]{$a$} circle[radius=1.5pt];
      \fill (2,1) circle[radius=1.5pt];
      \fill (1,2) circle[radius=1.5pt];
    \end{tikzpicture}
  \end{center}
  \caption{Il grafico della funzione esponenziale e logaritmo 
  in base $a>1$ e $\frac 1 a < 1$ ($a=2$ in figura).}
  \label{fig:esponenziale_logaritmo}
\end{figure}
%
\begin{figure}
  \begin{center}
    \begin{tikzpicture}[x=1.0cm,y=1.0cm]
      \clip(-3,-2) rectangle (4,3);
    	\draw[->] (-3,0) -- (4,0);
      \draw[->] (0,-2) -- (0,3);
      \draw[dotted] (1,0) -- (1,1) -- (0,1);
      % \draw[dotted] (2,0) -- (2,1) -- (0,1);
      \draw[domain=0.5:4,samples=50,variable=\x,thick,dashed] plot
        ({\x}, {1/(\x*\x))});
      \draw[domain=-3:-0.5,samples=50,variable=\x,thick,dashed] plot
        ({\x}, {1/(\x*\x))});
      \draw[domain=0.2:4,samples=50,variable=\x,dashed,color=gray] plot
        ({\x}, {1/(\x))});
      \draw[domain=-3:-0.5,samples=50,variable=\x,dashed,color=gray] plot
        ({\x}, {1/(\x))});
      \draw[domain=-2.0:2.0,smooth,variable=\x,thick,color=gray] plot
        ({\x}, {\x*\x*\x)});
      \draw[domain=-3.1:4.0,samples=200,variable=\x,color=gray] plot
        ({\x}, {pow(abs(\x),1/3)*\x / abs(\x)});
      \draw[domain=-3.0:3.0,smooth,variable=\x,thick] plot
        ({\x}, {\x*\x});
      \draw[domain=0.0:4.0,samples=200,variable=\x] plot
      ({\x}, {pow(\x,1/2)});
      \node at (-2.0,1.8) {$y=x^2$};
      \node at (3,2) {$y=\sqrt{x}$};
      \node[color=gray] at (-1.7,-1.8) {$y=x^{3}$};
      \node[color=gray] at (-2.0,-1) {$y=\sqrt[3]{x}$};
      \node[color=gray] at (-1.7,-0.3) {$y=\frac 1 x$};
      \node at (-2.5,0.6) {$y=\frac 1 {x^2}$};
      \fill (1,0) node[below]{$1$} circle[radius=1.5pt];
      \fill (0,1) node[left]{$1$} circle[radius=1.5pt];
    \end{tikzpicture}
  \end{center}
  \caption{Grafici tipici di potenze e radici.}
  \label{fig:potenza_radice}
\end{figure}

Osserviamo che la funzione potenza $x^\alpha$ (dove si tiene fisso 
l'esponente e si fa variare la base) è definita tramite la funzione esponenziale 
per ogni $x>0$ e per ogni $\alpha\in \RR$.
Se $\alpha>0$ è naturale definire $0^\alpha = 0$ e quindi 
se $\alpha>0$ 
possiamo considerare definito $x^\alpha$ per ogni $x\ge 0$.
Verifichiamo che la funzione $f(x)=x^\alpha$ è strettamente crescente
 se $\alpha>0$. 
Infatti se $x_2>x_1>0$ e se $\alpha>0$ si ha 
\[
  x_2^\alpha 
  = \enclose{\frac {x_2} {x_1}}^\alpha\cdot {x_1}^\alpha 
  > \enclose{\frac {x_2} {x_1}}^0 \cdot {x_1}^\alpha 
  = x_1^\alpha.
\]


\section{radice $n$-esima}

Dalle proprietà delle potenze ripetute deduciamo che se $x>0$
e $n\in\NN$, $n\ge 1$, si ha $(x^n)^\frac 1 n = x$.
Significa che $x^{\frac 1 n}$ è la funzione inversa di $x^n$ per $x>0$.
Ma la funzione $x^n$ è definita anche per $x\le 0$ e si distinguono due 
casi qualitativamente diversi in base alla parità di $n$.

Se $n$ è pari la funzione $x\mapsto x^n$ non è invertibile in quanto 
$(-x)^n = x^n$. 
Se però restringiamo la funzione $x^n$ ai numeri reali non negativi
la funzione risulta invertibile e la funzione inversa 
che si denota con $x\mapsto\sqrt[n]{x}$ 
può essere definita mediante le potenze come segue:
\[
   \sqrt[n]{x} = \begin{cases}
      x^{\frac 1 n} & \text{se $x>0$}\\
      0 & \text{se $x=0$}
   \end{cases} \qquad \text{se $n$ è pari}.
\]
Se $n$ è dispari la funzione $x\mapsto x^n$ è invertibile su tutto 
$\RR$. 
Infatti $(-x)^n = -x^n$ dunque tale funzione mantiene il segno 
del suo argomento. In tal caso la funzione inversa 
denotata con $x\mapsto \sqrt[n]{x}$
è anch'essa definita su tutto $\RR$ e può essere definita 
nel modo seguente
\[
   \sqrt[n]{x} = \begin{cases}
      x^{\frac 1 n} & \text{se $x>0$} \\ 
      0 & \text{se $x=0$}\\
      -(-x)^{\frac 1 n} & \text{se $x<0$}
   \end{cases}
   \qquad\text{se $n$ è dispari}
\]

Quando l'argomento è positivo la radice $n$-esima 
potrebbe anche essere definita utilizzando il teorema~\ref{th:divisibile}
di divisibilità nell'ambito del gruppo moltiplicativo $\RR_+$.

La radice seconda $\sqrt[2]{x}$ viene usualmente chiamata 
\emph{radice quadrata}%
\mymargin{radice quadrata}%
\index{radice!quadrata} e
viene denotata con il simbolo $\sqrt{x}$.
Analogamente la radice terza $\sqrt[3]{x}$ viene usualmente chiamata 
\emph{radice cubica}%
\mymargin{radice cubica}%
\index{radice!cubica}.

Per come è definita la radice $n$-esima si ha 
\[
  \sqrt[n]{x^n} = 
  \begin{cases}
     \abs{x} & \text{se $n$ è pari,} \\
     x & \text{se $n$ è dispari.}
  \end{cases}
\]
E' facile verificare che le proprietà 
$\sqrt[n]{x\cdot y} = \sqrt[n]{x} \cdot \sqrt[n]{y}$
e $\sqrt[n]{\sqrt[m]{x}} = \sqrt[nm]{x}$
sono valide in tutti i casi in cui ambo i membri sono ben definiti. 

Ricordiamo che in base al teorema~\ref{th:pitagora} sappiamo 
che $\sqrt 2$ è irrazionale 
(non è razionale, ovvero non è elemento di $\QQ$)
in quanto risolve l'equazione $x^2=2$.

Possiamo quindi osservare che $\QQ$, come $\RR$, è un campo totalmente 
ordinato.
Evidentemente $\QQ$ non può essere continuo, perché se lo fosse 
dovrebbe essere uguale ad $\RR$ per il teorema di unicità degli 
isomorfismi dei gruppi continui ordinati e dovrebbe quindi 
contenere il numero $\sqrt 2$. 

\section{logaritmo}
%
Se $a>0$, $a\neq 1$ la funzione esponenziale 
$\exp_a\colon \RR \to \RR_+$ è bigettiva. 
La funzione inversa $\log_a\colon \RR_+\to \RR$ si chiama \emph{logaritmo}%
\mymargin{logaritmo}%
\index{logaritmo} in base $a$ 
e si caratterizza con la seguente proprietà:
\[
  \log_a x = y \iff a^y = x.
\]
La proprietà di omomorfismo diventa:
\[
  \log_a(x\cdot y) =  \log_a x + \log_a y.
\]
Ovviamente vale $\log_a a = 1$ e $\log_a 1 = 0$.
Se $a>1$ la funzione logaritmo è strettamente crescente, se $a<1$ è strettamente decrescente.
La proprietà dell'esponenziale ripetuto si traduce nella seguente:
\[
  \log_a (x^y) = y \log_a x.
\]
E' anche piuttosto utile ricordare la formula per il cambiamento di base:
\[
\log_a x = \frac{\log_b x}{\log_a b}
\]
(valida, come tutte le formule enunciate, se l'argomento del logaritmo 
è positivo e se le basi sono positive diverse da $1$ e diverse tra loro).
Quest'ultima formula si dimostra moltiplicando ambo i membri per $\log_a b$ 
e prendendo quindi l'esponenziale con base $b$ di ambo i membri.


\section{equazioni e disequazioni}

Un problema matematico molto comune è quello di dover risolvere 
equazioni e disequazioni del tipo:
\begin{equation}\label{eq:573197}
  f(x) = b, \quad f(x) \ge b, \quad f(x) > b, 
  \quad f(x) \le b, \quad f(x) < b
\end{equation}
dove $f\colon A \subset \RR \to\RR$ è una funzione data e 
$b\in \RR$ è fissato.

Quando $f$ è strettamente crescente e $b\in f(A)$ 
la soluzione può essere 
scritta banalmente: 
ovviamente deve essere $x\in A$
e ogni equazione o disequazione in~\eqref{eq:573197}
avrà la corrispondente soluzione:
\[
  x= f^{-1}(b), \quad x \ge f^{-1}(b), \quad x>f^{-1}(b),
  \quad x \le f^{-1}(b), \quad x < f^{-1}(b).
\]
Se la funzione fosse strettamente decrescente 
si può procedere allo stesso modo, ma le disuguaglianze si invertono.
Se la funzione fosse strettamente crescente su alcuni intervalli 
e strettamente decrescente su altri si potranno separare i diversi 
casi e si otterranno più soluzioni espresse da uguaglianze
o disuguaglianze.

\begin{example}
  Si risolva la disequazione 
  \[
   \log_2\Enclose{\sqrt[3]{(x+1)^4-3}-2} \le 3. 
  \]
\end{example}%
\begin{proof}[Svolgimento.]
La funzione logaritmo è strettamente crescente ed è definita 
quando l'argomento è positivo. 
Dunque la disequazione data 
è equivalente al sistema di disequazioni:
\[
0 < \sqrt[3]{(x+1)^4 - 3} - 2 \le 8.  
\]
Possiamo sommare $2$ per ottenere 
\[
  2 < \sqrt[3]{(x+1)^4 - 3} \le 10.  
\]
La funzione radice cubica è strettamente crescente 
su tutto $\RR$ quindi possiamo invertirla elevando 
tutto al cubo:
\[
 8 < (x+1)^4 - 3 \le 1000.
\]
Sommiamo $3$:
\[
11 < (x+1)^4 \le 1003.  
\]
L'elevamento alla quarta potenza è strettamente crescente 
solo quando l'argomento è positivo, ed è una funzione pari.
Possiamo quindi affermare che le nostre disequazioni sono 
equivalenti all'unione delle soluzioni di due sistemi:
\[
  \sqrt[4]{11} < x+1 \le \sqrt[4]{1003}
  \qquad\text{o}\qquad 
  -\sqrt[4]{1003} \le x+1 < \sqrt[4]{11}.
\]
Sottraendo $1$ otteniamo infine 
\[
  \sqrt[4]{11} -1 < x \le \sqrt[4]{1003} - 1
  \qquad\text{o}\qquad 
  -\sqrt[4]{1003} -1 \le x < \sqrt[4]{11} -1.
\]
In definitiva l'insieme delle soluzioni è 
\[
\left[-\sqrt[4]{1003} - 1, \sqrt[4]{11}-1\right)
\cup \left(\sqrt[4]{11}-1 , \sqrt[4]{1003} -1\right].  
\]
\end{proof}

Nell'esempio precedente la funzione 
$f(x) = \log_2\Enclose{\sqrt[3]{(x+1)^4-3}-2}$
è ottenuta mediante composizione di funzioni elementari:
\begin{align*}
f &= (x\mapsto \log_2 x)\circ(x\mapsto x-2)\circ (x \mapsto \sqrt[3]{x})\\
  &\quad \circ (x \mapsto x-3) \circ (x\mapsto x^4) \circ (x\mapsto x+1).
\end{align*}
Negli intervalli in cui tutte queste funzioni sono invertibili 
la funzione inversa si ottiene componendo, in ordine opposto,
tutte le inverse:
\begin{align*}
  f^{-1} &= (x\mapsto x-1) \circ (x\mapsto \sqrt[4]{x}) \circ (x \mapsto x+3) \\
    &\quad \circ (x \mapsto x^3) \circ (x \mapsto x+2) \circ (x\mapsto 2^x).
\end{align*}
In effetti il caposaldo $\sqrt[4]{1003}-1$ è proprio tale 
funzione valutata in $b=3$.

Il metodo precedente è puramente algebrico e 
si applica alle equazioni 
come la~\eqref{eq:573197} dove la variabile $x$ 
compare una sola volta e dove la funzione $f$ si esprime 
come composizione di funzioni elementari di cui sappiamo 
scrivere la funzione inversa. 

Ben diverso è il caso in cui nell'equazione la variabile $x$ 
compare più di una volta.
In alcuni casi, come ad esempio,
\[
  x^2 > 2x - 1  
\]
queste equazioni 
possono essere ricondotte al caso precedente tramite 
opportune manipolazioni algebriche.
Il caso delle equazioni quadratiche lo abbiamo 
fatto nel paragrafo precedente utilizzato il completamento 
del quadrato: $x^2-2x = (x-1)^2-1$. 
In altri casi, come ad esempio l'equazione
\[
  2^x = x^2
\]  
le manipolazioni algebriche non sono utili.
Nel capitolo sul calcolo differenziale svilupperemo degli strumenti 
che ci permetteranno di determinare l'andamento di molte di queste 
di funzioni. 
Nel capitolo sulle successioni svilupperemo invece gli strumenti 
che ci permetteranno di determinare le soluzioni mediante 
algoritmi di approssimazione.
Questi strumenti presuppongono il concetto 
di limite e continuità: è sostanzialmente questo che identifica 
la materia chiamata analisi matematica.

\section{cardinalità infinite}

Abbiamo già osservato che $\NN$ è un insieme infinito e dunque (per il teorema~\ref{th:cantor_bernstein})
anche $\ZZ$, $\QQ$ e $\RR$ sono infiniti.
Per gli insiemi finiti se $A\subset B$ ma $A\neq B$ risulta $\#A < \# B$
in quanto $\#A - \#B = \#(B\setminus A) > 0$. 
Dobbiamo però osservare che lo stesso non vale per gli insiemi infiniti.
In effetti la funzione $s\colon \NN \to \NN$ definita da $s(x)=x+1$
risulta essere una bigezione tra $\NN$ e $\NN\setminus\ENCLOSE{0}$. 
Dunque $\#\NN = \#(\NN\setminus\ENCLOSE{0})$ 
nonostante la differenza tra i due insiemi $\ENCLOSE{0}$ abbia 
cardinalità $1$. 
Anche l'insieme dei numeri pari o l'insieme 
dei quadrati perfetti (come aveva notato già Galileo)
hanno la stessa cardinalità di $\NN$ in quanto le funzioni $n\mapsto 2n$
e la funzione $n\mapsto n^2$ sono iniettive.

Non è difficile trovare una funzione iniettiva da $\ZZ$ in $\NN$
(lo lasciamo per esercizio): questo dimostra che anche $\#\ZZ = \# \NN$

Cantor dimostra addirittura che $\#\QQ = \NN$: per farlo utilizza il seguente.

\begin{theorem}[primo metodo diagonale di Cantor]
\label{th:Cantor_primo}%
\index{Cantor!primo metodo diagonale}%
L'insieme $\NN \times \NN$ ha la stessa cardinalità di $\NN$. Di conseguenza
\[
  \# \NN = \# \ZZ = \# \QQ.
  \]
\end{theorem}
%
\begin{proof}[Idea di dimostrazione]
  Numerare le caselle di una scacchiera infinita
  come in Figura~\ref{fig:cantor1}.
  
  Visto che è piuttosto facile trovare una funzione iniettiva 
  da $\QQ$ in $\NN \times \ZZ$ si deduce facilmente che $\#\QQ = \#\NN$.
\end{proof}

\begin{figure}
  \begin{tabular}{c|ccccccc}
   $m$ & $\ddots$\\
   $\vdots$ & $\ddots$ & $\ddots$\\
   3 & 9 & $\nwarrow$ & $\ddots$ \\
   2 & 5 & 8 & 12 & $\ddots$ \\
   1 & 2 & 4 & 7 & 11 & $\ddots$ \\
   0 & 0 & 1 & 3 & 6 & 10 & $\ddots$ \\ \hline
     & 0 & 1 & 2 & 3 & 4 & $\dots$ & $n$
  \end{tabular}
  \caption{
    La numerazione diagonale delle caselle
    di una scacchiera infinita. Si potrebbe verificare
    che il numero presente nella casella di coordinate $(n,m)$
    si scrive come $f(n,m) = \frac{(n+m+1)(n+m)}{2}+m$
    ed è una funzione bigettiva $f\colon \NN\times\NN \to\NN$.}
  \label{fig:cantor1}
\end{figure}

A questo punto si potrebbe pensare che
tutti gli insiemi infiniti siano numerabili.
Invece preso un qualunque insieme (anche infinito)
esiste un insieme con cardinalità strettamente maggiore
come dimostrato nel seguente teorema che è un piccolo gioiello della logica.
In effetti il metodo utilizzato è assimilabile al paradosso del mentitore 
ed è la stessa idea usata nel paradosso di Russell (teorema~\ref{th:Russell}).
L'insieme delle parti $\mathcal P(A)$ è definito a pag.~\pageref{def:insieme_parti}.
%
\begin{theorem}[Cantor]%
\label{th:Cantor}%
  Se $A$ è un qualunque insieme allora $\# \mathcal P(A) > \# A$.
\end{theorem}
%
\begin{proof}
  E' chiaro che $\# A \le \#\mathcal P(A)$ in quanto 
  la funzione $f\colon A \to \mathcal P(A)$ definita da $f(x)=\ENCLOSE{x}$
  è ovviamente iniettiva.

  Supponiamo allora per assurdo che esista $f\colon A\to \mathcal P(A)$
  bigettiva e consideriamo l'insieme 
  \[
    C = \ENCLOSE{x\in A \colon x\not\in f(x)}.  
  \]
  Se $f$ fosse surgettiva dovrebbe esistere $c\in A$ tale che $f(c) = C$.
  Possiamo allora chiederci se $c\in C$ e scoprire che, 
  per definizione di $C$ la proposizione $c\in C$ è equivalente 
  a $c\not\in f(c) = C$. 
  Dunque $c\in C \iff c\not\in C$, che è assurdo.
\end{proof}
%
\begin{corollary}[non esiste l'insieme di tutti gli insiemi]
  Se esistesse un insieme $\mathcal U$ (universo) 
  tale che $\forall x\colon x \in U$, allora 
  si avrebbe $\mathcal \P(\mathcal U) \subset \mathcal U$
  contraddicendo il teorema precedente.
\end{corollary}
%
Per quanto riguarda l'insieme $\RR$ si
scopre in effetti che $\#\RR > \#\NN$
(sempre grazie a Cantor, teorema~\ref{th:cantor_secondo})
e si potrebbe dimostrare che effettivamente $\#\RR = \#\mathcal P(\NN)$.

\section{i numeri complessi}

Dal punto di vista geometrico l'insieme $\CC$ dei \emph{numeri complessi}%
\mymargin{numeri complessi}%
\index{numeri!complessi}
\index{$\CC$}
può essere visto come un modello del piano euclideo.
Sul piano euclideo fissiamo arbitrarimente un punto $0$ in modo da ottenere uno spazio
vettoriale e fissiamo, arbitrariamente, una base $e_1$, $e_2$.
Identifichiamo ogni punto del piano con i corrispondenti vettori
applicati in $0$. La retta generata dal vettore $e_1$ la identifichiamo
con la retta $\RR$ dei numeri reali e quindi poniamo $1=e_1$.
La retta ortogonale generata dal vettore $e_2$ verrà chiamata
retta dei \emph{numeri immaginari} e definiamo $i=e_2$.

Un generico punto $z$ del piano $\CC$ potrà essere scritto in
maniera univoca nella base scelta: $z = x e_1 + y e_2$ ovvero,
per come abbiamo chiamato $e_1$ ed $e_2$:
\[
z = x + i y.
\]
Tale $z$ viene chiamato
\emph{numero complesso} con parte reale $x$ e parte immaginaria $y$.
Questa rappresentazione del numero complesso $z$ viene
chiamata \emph{rappresentazione cartesiana}%
\mymargin{rappresentazione cartesiana}%
\index{rappresentazione!cartesiana} in quanto definisce
il punto $z$ del piano complesso tramite le sue coordinate cartesiane
$x$ e $y$.
I numeri reali sono \emph{immersi} nei complessi, nel senso che se
$x\in \RR$ allora $z= x + i\cdot 0 = x$ è anche un numero complesso.
Il numero complesso $i = 0 + i\cdot 1$ viene chiamata \emph{unità immaginaria}%
\mymargin{unità immaginaria}%
\index{unità!immaginaria}
e i numeri complessi della forma $iy$ sono chiamati \emph{immaginari}.
\index{numeri!immaginari}
\index{immaginario}
Un numero
complesso $z = x+iy$ è quindi una somma tra un numero reale ed un numero
immaginario. Il numero reale $x$ viene chiamato \emph{parte reale}
\index{parte!reale}
di $z$ e
si denota con $x=\Re z$.
\mymargin{$\Re z$}%
\index{$\Re z$}
Il numero reale $y$ viene chiamato
\emph{parte immaginaria}
\index{parte!immaginaria}
di $z$ e si denota con $y=\Im z$
\mymargin{$\Im z$}%
\index{$\Im z$}
(osserviamo che la parte immaginaria di un numero complesso è un numero
reale, non immaginario). Dunque $z= \Re z + i \Im z$.

L'insieme $\CC$, per come
è stato costruito, è uno spazio vettoriale reale di dimensione $2$.
Abbiamo quindi già definite la \emph{addizione}%
\mymargin{addizione}%
\index{addizione}
\index{complessi!addizione}
tra elementi di $\CC$ e la moltiplicazione
tra elementi di $\CC$ ed elementi di $\RR$,
se $a,b,c,d,t\in \RR$ si ha:
\begin{gather*}
 (a+ib) + (c+id) = (a+c) + i (b+d), \\
 t(a+ib) = ta + itb.
\end{gather*}

Vogliamo estendere la \emph{moltiplicazione}%
\mymargin{moltiplicazione}%
\index{moltiplicazione} a tutte le coppie di numeri complessi.
\index{complessi!moltiplicazione}
Imponendo (arbitrariamente) che valga $i\cdot i = -1$ e che rimanga
valida la proprietà distributiva, si ottiene
questa definizione:
\[
   (a+ib) \cdot (c+id) = (ac-bd) + i(ad+bc).
\]

Si può verificare che questa moltiplicazione estende quella ``scalare'' definita
in precedenza.
E' anche facile verificare che addizione e moltiplicazione soddisfano
le proprietà commutativa associativa e distributiva,
che $0$ è elemento neutro per la addizione, che $1$ è elemento neutro
della moltiplicazione.
Si osservi che se $z=x+iy$ non è nullo, allora
\[
  (x+iy) \cdot \frac{x-iy}{x^2+y^2} = 1.
\]
Significa che ogni $z\neq 0$ ammette inverso moltiplicativo e quindi $\CC$
risulta essere un campo.

Osserviamo che su $\CC$ non si definisce una relazione d'ordine perché
in effetti non è possibile definire un ordine ``compatibile'' con le operazioni
appena definite.%
\mynote{%
Se $\CC$ fosse un campo ordinato per assurdo
si dovrebbe avere,
che $z^2\ge 0$ per ogni $z\in \CC$ (questo è vero in tutti i campi ordinati). 
Ma
allora $-1 =i^2 \ge 0$ cioè $1\le 0$ che è in contraddizione
con la proprietà $0<1$ valida in ogni campo ordinato.
} % marginnote

Su $\CC$ definiamo delle ulteriori operazioni.
Il \emph{coniugato}%
\mymargin{coniugato}%
\index{coniugato}%
\index{complessi!coniugio}
di un numero complesso $z=x+iy$ è il numero
$\bar z = x - iy$. Geometricamente l'operazione di coniugio è una simmetria
rispetto alla retta reale. I numeri reali sono in effetti punti fissi del
coniugio (il coniugato di un numero reale è il numero stesso).
E' un semplice esercizio verificare che il coniugio ``attraversa''
somma e prodotto:
\[
\overline{z+w} = \bar z + \bar w, \qquad
\overline{z\cdot w} = \bar z \cdot \bar w.
\]
Ovviamente risulta $\overline {\bar z} = z$.
E' anche utile osservare che si ha:
\begin{equation}\label{eq:re_im}
  \Re z = \frac{z+\bar z}{2}, \qquad
  \Im z = \frac{z-\bar z}{2i}
\end{equation}
e
\[
z \cdot \bar z = (x+iy)(x-iy) = x^2-i^2y^2 = x^2+y^2.
\]

Possiamo allora definire il
\emph{modulo}%
\mymargin{modulo}%
\index{modulo}%
\index{complessi!modulo}
 di un numero complesso $z=x+iy$
come il numero reale
\[
\abs{z} = \sqrt{z\cdot\bar z} = \sqrt{x^2+y^2}.
\]
Geometricamente tale quantità rappresenta la distanza del punto $z$
dal punto $0$ e quindi la distanza tra due numeri complessi $z$ e
$w$ si potrà rappresentare con $\abs{z-w}$.

Osserviamo che se $z = x \in \RR \subset \CC$ il modulo di $z$ coincide
con il valore assoluto: $\abs{z} = \sqrt{x^2} = \abs{x}$ e per questo
motivo non distinguiamo, nelle notazioni, il modulo dal valore assoluto.
Più in generale risulta per ogni $z\in \CC$ (la verifica è immediata):
\[
  \abs{\Re z} \le \abs{z}, \qquad
  \abs{\Im z} \le \abs{z}.
\]

Possiamo a questo punto trovare una utile formula per calcolare
il reciproco di un numero complesso. Essendo infatti
$z\cdot \bar z = \abs{z}^2$ si osserva che
\[
  \frac{1}{z}
  = \frac{\bar z}{ \bar z \cdot z}
  = \frac{\bar z}{\abs{z}^2}.
\]

\begin{theorem}
Il modulo di un numero complesso soddisfa (come il valore assoluto)
le seguenti proprietà
\begin{enumerate}
\item $\big\lvert\abs{z}\big\rvert = \abs{z}$,
\item $\abs{-z} = \abs{z}$ = $\abs{\bar z}$,
\item $\abs{z\cdot w} = \abs{z}\cdot\abs{w}$.
\item $\abs{z+w} \le \abs{z}+\abs{w}$ (convessità),
\item $\abs{z-w} \le \abs{z-v} + \abs{v-w}$ (disuguaglianza triangolare),
\end{enumerate}
\end{theorem}
%
\begin{proof}
La prima proprietà è ovvia in quanto il valore assoluto di un numero reale
non negativo è il numero stesso.

La seconda proprietà viene immediatamente dalla definizione.

Per la terza proprietà sia $z=x+iy$, $w=a+ib$.
Allora:
\begin{align*}
\abs{z\cdot w}
&= \abs{(x+iy)\cdot(a+ib)}
=\abs{xa - y b+ i(xb + ay)} \\
&= \sqrt{(xa-yb)^2 + (xb+ay)^2}\\
&=\sqrt{x^2 a^2 + y^2b^2 - 2xayb + x^2b^2+a^2y^2+2xbay} \\
&=\sqrt{x^2 a^2 + y^2 b^2 + x^2 b^2 + a^2 y^2}\\
&=\sqrt{x^2(a^2+b^2) + y^2(a^2+b^2)}\\
&=\sqrt{(x^2+y^2)(a^2+b^2)}
=\abs{x+iy} \cdot \abs{a+ib}\\
&=\abs{z}\cdot\abs{w}.
\end{align*}

Per la quarta disuguaglianza osserviamo che si ha
\[
  \abs{z+w}^2 = (z+w)\cdot(\bar z + \bar w)
  = \abs{z}^2 + \abs{w}^2 + z\cdot \bar w + \bar z \cdot w
\]
e visto che
\[
  z\cdot \bar w + \bar z \cdot w
  = z \cdot \bar w + \overline{z \cdot \bar w}
  = 2 \Re(z\bar w)
  \le 2 \abs {z\bar w}
  = 2 \abs{z}\cdot\abs{\bar w}
  = 2 \abs{z}\cdot\abs{w}
\]
otteniamo
\[
 \abs{z+w}^2 \le \abs{z}^2+\abs{w}^2 + 2 \abs{z}\cdot\abs{w}
 =\enclose{\abs z + \abs w}^2
\]
che è equivalente alla disuguaglianza di convessità.

La disuguaglianza triangolare è conseguenza immediata della convessità, infatti
\[
  \abs{z-w} = \abs{(z-v) + (v-w)}
  \le \abs{z-v} + \abs{v-w}.
\]
\end{proof}

\begin{figure}
  \begin{center}
    \begin{tikzpicture}[x=0.5cm,y=0.5cm]
    	\draw[->] (-1,0) -- (9,0);
      \draw[->] (0,-1) -- (0,8);
      %
      \draw[dotted] (0,0) -- ({sqrt(45)},{sqrt(20)}) -- ({sqrt(45)},0);
      \draw[fill] ({sqrt(45)},{sqrt(20)}) node[right] {$w$} circle [radius=0.1];
      %
      \draw[thick] (0,0) -- (6,0);
      \draw[thick] (0,0) -- (6,3);
      \draw[thick] (0,0) -- (4,7);
      \draw[thick] (6,0) -- (6,3);
      \draw[thick] (6,3) -- (4,7);
      %
      \draw[dashed] (4,0) -- (4,7);
      \draw[dashed] (4,3) -- (6,3);
      %
      \draw (1,0) arc (0:{atan(1/2)}:1);
      \draw (6,3)+(-1,0) arc (180:{180+atan(1/2)}:1);
      \draw (4,7)+(0,-1) arc (-90:{-90+atan(1/2)}:1);
      %
      \draw[fill] (6,3) node[right] {$z$} circle [radius=0.1];
      \draw[fill] (4,7) node[above] {$u$} circle [radius=0.1];
      %
      \node[below] at (3,0) {$a$};
      \node[left] at (6,1.5) {$b$};
      \node[above] at (3,1.5) {$\alpha$};
      \node[right] at (5,5) {$\beta$};
      \node[above] at (5,3) {$s$};
      \node[left] at (4,4.5) {$t$};
      \node[above] at (8.5,0) {$x$};
      \node[left] at (0,7.5) {$y$};
      \node[below left] at (0,0) {$0$};
    \end{tikzpicture}
  \end{center}
  \caption{Consideriamo i numeri complessi $z=a+ib$ e $w=\alpha+i\beta$
  e supponiamo che sia $\alpha^2 = a^2+b^2$.
  Si consideri il punto $u$ che si ottiene ruotando il punto $w$ dell'angolo
  individuato dal punto $z$. Si avrà allora $u=a-s + i(b+t)$ dove $s$ e $t$
  sono i cateti del triangolo rettangolo con ipotenusa $\beta$.
  Grazie alle proprietà di similitudine dei triangoli si ha
  $\frac{s}{\beta} = \frac{b}{\alpha}$ e $\frac{t}{\beta} = \frac{a}{\alpha}$
  da cui si ottiene quindi $u = a-\frac{b\beta}{\alpha}+i(b+\frac{a\beta}{\alpha})$
  ovvero $\alpha u = (a\alpha - b \beta) + i (b\alpha + a \beta) = z\cdot w$.
  Significa che il numero complesso $z\cdot w$ si trova sulla semiretta
  che individua un angolo che è la somma degli angoli individuati
  dai numeri complessi $z$ e $w$.
  }
  \label{fig:prodotto_complesso}
\end{figure}

Possiamo ora dare una interpretazione geometrica del prodotto $z\cdot w$
tra due numeri complessi. In primo luogo sappiamo che $\abs{z\cdot w} = \abs{z} \cdot \abs{w}$ e dunque il punto del piano che rappresenta il prodotto $z\cdot w$ si trova ad una distanza dall'origine che è pari al prodotto delle distanze
dei punti $z$ e $w$. Inoltre l'angolo individuato da $z\cdot w$ rispetto
all'asse delle $x$ positive risulta uguale alla somma
degli angoli individuati dai punti $z$ e $w$ come mostrato in figura~\ref{fig:prodotto_complesso}.

Anche il piano dei numeri complessi può essere esteso aggiungendoci
un punto all'\emph{infinito}%
\mymargin{infinito}%
\index{infinito}.
A differenza dei reali, su cui era presente un ordinamento che era utile conservare,
nel caso dei numeri complessi è più usuale utilizzare un unico punto infinito
che si denota con \emph{$\infty$}%
\mymargin{$\infty$}%
\index{$\infty$}.
Definiamo il piano dei complessi estesi $\bar \CC$ come
\[
\bar \CC = \CC \cup \ENCLOSE{\infty}.
\]
Definiamo
\begin{align*}
  z + \infty &= \infty \qquad \forall z \in \CC\\
  z - \infty &= \infty \qquad \forall z \in \CC\\
   z\cdot \infty &= \infty \qquad \forall z \in \bar\CC\setminus\ENCLOSE{0} \\
   z / \infty &= 0 \qquad \forall z \in \CC \\
   z / 0 &= \infty \qquad \forall z \in \bar \CC \setminus\ENCLOSE{0}\\
   \bar \infty &= \infty \\
   \abs{\infty} &= +\infty \in \bar \RR.
\end{align*}
Si noti che abbiamo definito la divisione per zero di numeri complessi
(e quindi anche reali) diversi da zero. Il risultato è $\infty$ e quindi
rimane confermato che la divisione per zero non è una operazione valida
se vogliamo un risultato finito.
Una quantità $z\in \bar \CC$ sarà detta \emph{finita} se $z\in \CC$.

\begin{example}
  La funzione $f\colon \bar \CC \to \bar \CC$ 
  definita da 
  \[
  f(z) = \frac{1}{z}
  \]
  è una funzione bigettiva di $\bar \CC$ in sé.
  Dal punto di vista geometrico il coniugato di tale funzione
  ovvero la funzione $z\mapsto \frac 1 {\bar z}$ 
  è l'inversione circolare rispetto al cerchio unitario di $\CC$:
  i punti sulla circonferenza unitaria vengono lasciati fissi,
  i punti all'interno vengono mandati all'esterno rimanendo sullo 
  stesso raggio uscente dall'origine e invertendo il proprio modulo.
  I punti $0$ e $\infty$ si scambiano.
\end{example}

%% % ancora non abbiamo definito il concetto di funzione continua
%% Per le funzioni di variabile complessa e/o a valori complessi 
%% si applica la stessa definizione~\ref{def:continua} di continuità 
%% che abbiamo dato per le funzioni reali utilizzando il modulo 
%% complesso al posto del valore assoluto.
%% \index{continuità!campo complesso}%
%% \index{funzione!continua!complessa}%

\begin{exercise}[vertici di un triangolo equilatero]
Si risolva l'equazione 
\[ 
 z^3 - 1 = 0
\]
nel campo complesso.
\end{exercise}
%
\begin{proof}[Svolgimento.]
Ricordiamo il prodotto notevole:
\[
  z^3 - 1 = (z-1)(z^2+z+1).
\]
Dunque $z=1$ è una soluzione e le altre soluzioni devono risolvere 
l'equazione $z^2+z+1=0$. 
Certamente $z=0$ non è soluzione e dunque possiamo dividere per $z$ 
e ottenere:
\[
  z + 1 + \frac 1 z = 0.
\]
Osserviamo ora che se $z$ è soluzione si ha $1=z^3$
e quindi: $1 = \abs{z^3}= \abs{z}^3$ da cui $\abs z = 1$.
Ma allora $z\cdot \bar z = \abs{z}^2 = 1$ ovvero $\frac 1 z = \bar z$.
Dunque si ha 
\[
    z + 1 + \bar z = 0.
\]
Osserviamo ora che se $z=x+iy$ con $x,y\in \RR$ allora $z+\bar z = 2x$
e quindi 
\[
    2x + 1 = 0 
\]
da cui $x=-\frac 1 2$. Essendo inoltre $x^2+y^2=\abs{z}^2=1$ si ottiene
$y^2 = 1-x^2 = \frac 3 4$ da cui $y=\pm \frac{\sqrt 3}{2}$.

L'equazione data ha quindi $3$ soluzioni:
\[
z_0 = 1, \qquad 
z_{1,2} = -\frac 1 2 \pm i \frac{\sqrt 3} 2.
\]

Questi tre punti, se disegnati sul piano di Gauss, si trovano 
ai vertici di un triangolo equilatero iscritto nella circonferenza unitaria.
Infatti l'interpretazione geometrica del prodotto di numeri complessi ci dice 
che il punto $z_1$ individua sul piano di Gauss un angolo pari 
ad un terzo dell'angolo giro. Inoltre si ha $z_2 = z_1^2$ e dunque 
$z_2$ corrisponde a $\frac 2 3$ di angolo giro e $z_0=z_1^3 = 1$ rappresenta 
l'angolo giro (o l'angolo nullo).
\end{proof}

\section{funzioni trigonometriche 1}
\label{sec:avvolgimento}

Consideriamo la circonferenza unitaria 
\[
  U = \ENCLOSE{z\in \CC\colon \abs{z} = 1}.
\]
Ogni punto $z\in U$ individua un angolo geometrico con l'asse reale.
Il nostro obiettivo è di assegnare ad ogni angolo una misura.
Il punto $z=1$ individuerà un angolo di misura nulla e 
ruotando $z$ in senso antiorario vogliamo ottenere angoli crescenti
in modo additivo, ovvero vogliamo fare in modo che l'angolo che si ottiene 
giustapponendo uno di seguito all'altro gli angoli individuati 
dal punto $z$ e dal punto $w$ abbia misura pari alla somma delle misure 
degli angoli individuati da $z$ e da $w$.
Per fare questo gli angoli devono essere definiti con segno e \emph{molteplicità}.

L'idea intuitiva è quella di pensare alla retta reale $\RR$ come un filo 
elastico che viene arrotolato su $U$ in modo che il punto $0\in \RR$ 
si sovrapponga al punto $1\in \CC$ e poi, avvolgendo il filo in senso antiorario 
si vada a completare un giro sovrapponendo il punto $1\in \RR$ di nuovo sul punto 
$1\in \CC$. 
Usando il teorema degli isomorfismi di gruppi ordinati si riuscira a definire 
questa corrispondenza in modo che l'addizione su $\RR$ corrisponda alla moltiplicazione 
su $U\subset \CC$. 
Visto che la moltiplicazione per un numero unitario su $\CC$ rappresenta 
una rotazione pari all'angolo identificato dal numero complesso, 
quello che si ottiene è una corrispondenza tra $\RR$ e $U$ in cui 
segmenti congruenti in $\RR$ corrispondono ad angoli congruenti in $U$.
La proprietà di isomorfismo si può interpretare pensando che l'elastico 
viene avvolto con una tensione costante. 

L'insieme $U$ eredita la struttura di gruppo moltiplicativo di $\CC$: se $z,w\in U$ 
allora $z\cdot w \in \U$ in quanto $\abs{z\cdot w} = \abs{z}\cdot \abs{w} = 1$
essendo $\abs{z}=\abs{w}=1$. Purtroppo però non è possibile mettere su $U$ 
un ordinamento compatibile con l'operazione di gruppo: intuitivamente 
quando mi muovo su $U$ in senso antiorario mi ritrovo a ripercorrere più volte 
la stessa circonferenza.

Per distinguere i diversi avvolgimenti costruiamo infinite copie identiche di $U$,
le tagliamo tutte nel punto $1\in \CC$ e li attacchiamo 
uno di seguito all'altro come per ottenere una molla.
Formalmente definiamo 
$$
  G = \ZZ \times U
$$
e ordiniamo i punti $(k,z)$ di $G$ prima secondo $k$ 
(che rappresenta il numero di avvolgimento) e poi ordinando i punti di $U$ 
partendo da $1\in \CC$ passando da $i$, da $-1$, da $-i$ fino a tornare a $1$
senza però toccarlo:
\[
  (k_1, x_1+iy_1) \stackrel G< (k_2, x_2+iy_2) 
  \iff 
  \begin{cases}
    \text{$k_1 < k_2$, oppure} \\
    \text{$k_1=k_2$, $y_1<0$, $y_2<0$ e $x_1<x_2$, oppure} \\
    \text{$k_1 = k_2$ e $y_1\ge 0$ e }
    \begin{cases}
      y_2<0 \text{ oppure}\\
      x_1 > x_2
    \end{cases} \\
  \end{cases}
\]
(se i due punti sono su due \emph{spire} diverse viene prima quello 
sulla prima spira, se sono sulla stesso spira ed entrambi nella parte 
inferiore basta guarda la coordinata $x$, 
se invece sono nella stessa spira ma uno sopra e uno sotto l'asse delle $x$, 
quello sopra deve essre il primo dei due e l'altro deve stare sotto 
oppure avere una coordinata $x$ inferiore).
Definiamo quindi una operazione di addizione su $G$ preservando 
l'idea di sommare gli angoli geometrici con orientazione e 
molteplicità. 
Posto $z_1=x_1+iy_1$ e $z_2=x_2+iy_2$
definiamo
\begin{equation}\label{eq:somma_su_G}
(k_1,z_1) \stackrel G+ (k_2,z_2) 
= 
\begin{cases} 
  \scriptstyle(k_1+k_2+1, z_1\cdot z_2) & 
  \begin{cases}
    \text{se $y_1<0$ e $y_2<0$ oppure }\\
    \text{se $y_1<0$ e $y_2\ge 0$ e $\Im (z_1\cdot z_2)\ge 0$ oppure }\\
    \text{se $y_2<0$ e $y_1\ge 0$ e $\Im (z_1\cdot z_2)\ge 0$}
  \end{cases} \\
  \scriptstyle(k_1+k_2, z_1\cdot z_2) & \text{altrimenti}
\end{cases}
\end{equation}
(la somma dei due \emph{angoli} va nell'avvolgimento 
successivo se entrambi i punti stanno sotto l'asse delle $x$ oppure 
uno sta sopra e uno sotto e il loro prodotto non rimane sotto).

\begin{theorem}[ordinamento su $\ZZ\times U$]%
  \label{th:gruppo_su_U}%
L'insieme $G=\ZZ \times U$ con l'ordinamento 
e l'addizione sopra definiti, risulta essere un gruppo 
totalmente ordinato, divisibile e continuo.
\end{theorem}

\begin{proof}
  Da fare...
\end{proof}

\begin{theorem}[omomorfismo su $U$]%
  \label{th:omomorfismo_U}%
Esiste una unica funzione 
\[
 \phi\colon \RR \to \CC  
\]
con le seguenti proprietà:
\begin{enumerate}
  \item $\phi(x+y) = \phi(x)\cdot \phi(y)$ per ogni $x,y\in\RR$;
  \item $\phi(1) = 1$;
  \item la funzione $f\colon\closeinterval{0}{\frac 1 4}\to \CC$
  è crescente.
\end{enumerate}

Inoltre 
$\phi$ è $1$-periodica ovvero per ogni $x\in \RR$:
  \[
      \phi(x+1) = \phi(x)
  \]
ed è strettamente crescente sull'intervallo $\closeinterval{-\frac 1 4}{\frac 1 4}$.
\end{theorem}
%
\begin{proof}
  Consideriamo il gruppo $G$ definito nel teorema~\ref{th:gruppo_su_U}.
  In base al teorema~\ref{th:isomorfismo} esiste una unica funzione strettamente crescente 
  $f\colon \RR \to G$ tale che $f(x+y) = f(x)+_G f(y)$ e 
  $f(1/4)=(0,i)\in G$
  (quest'ultima condizione significa che l'angolo retto, rappresentato dal 
  numero complesso $i$, vogliamo ottenerlo ad un quarto di giro). 
  L'elemento neutro va necessariamente nell'elemento neutro quindi
  si avrà $f(0)=(0,1)$
  Tale funzione avrà due componenti: $f(x) = (\psi(x),\phi(x))$
  con $\psi\colon \RR\to\ZZ$ e $\phi\colon \RR\to U\subset \CC$ 
  (quest'ultima è la funzione che ci interessa).
  Guardando la seconda componente dell'uguaglianza $f(x+y) = f(x)+_G f(y)$ 
  ricordando la definizione~\eqref{eq:somma_su_G}
  si deduce che $\phi(x+y) = \phi(x)\cdot \phi(y)$. 
  Mentre se guardiamo la seconda componente di $f(0)=(0,1)$ 
  e di $f(1/4) = (0,i)$ deduciamo che $\phi(0) = 1$ e $\phi(1/4) = i$.
  Per additività otteniamo 
  $\phi(1/2)=\phi(1/4+1/4)= \phi(1/4)^2 = i^2=-1$ 
  e analogamente $\phi(1) = \phi(1/2)^2 = (-1)^2 = 1$.
  Sempre per additività sappiamo che $\phi(-x) = 1/\phi(x)$ 
  e dunque $\phi(-1/4)=-i$, $\phi(-1/2)=-1$, $\phi(-1)=1$.
  
  Guardiamo alla monotonia. 
  Se $x,y\in \closeinterval{0}{1/4}$ 
  si ha che $\phi(x)$ e $\phi(y)$ sono non inferiori (in $G$)
  a $\phi(0)=(0,1)$ e non superiori a $\phi(1/4)=(0,i)$.
  Se $x<y$ si ha $f(x) \stackrel G< f(y)$ e 
  guardando l'ordinamento di $G$ in questo intervallo, deduciamo
  che $\Re f(y) < \Re f(x)$. 
  Ma visto che $\Im f(x) = \sqrt{1-\Re^2 f(x)}$ (e lo stesso per $y$)
  si deduce che $\Im f(x) < \Im f(y)$ e dunque $\Im f$ è
  strettamente crescente su $\closeinterval{0}{1/4}$.
  Per simmetria la monotonia rimane valida su 
  tutto $\closeinterval{-1/4}{1/4}$, come volevamo dimostrare.
  
  Per l'unicità osserviamo che se $\tilde\phi\colon\RR\to U$ è una funzione 
  con le proprietà richieste allora si può definire 
  $\tilde f\colon \RR \to G$ ponendo $\tilde f(x) = (\lfloor x\rfloor, \tilde \phi(x))$.
  Si può verificare che $\tilde f$ è additiva, $\tilde f(1)=(1, 1)$ 
  e $\tilde f$ è crescente su $[-1/4,1/4]$. 
  Dunque, per il teorema~\ref{th:isomorfismo}, $\tilde f = f$ 
  per unicità dell'isomorfismo.

  Per dimostrare che $\phi$ è $1$-periodica 
  basta osservare che essendo $\phi(1)=1$, 
  per additività si ha $\phi(x+1) = \phi(x)\cdot \phi(1) = \phi(x)$.
\end{proof}

Avendo definito la funzione $\phi\colon \RR\to U$ grazie 
al teorema~\ref{th:omomorfismo_U} possiamo definire le 
funzioni trigonometriche. 
Scelta una unità di misura $\tau > 0$ per l'angolo giro\mynote{%
si potrebbe ora porre $\tau=360$ ma quando avremo 
definito $\pi$ vedremo che sarà più naturale scegliere 
$\tau = 2\pi$}
possiamo definire le funzioni $\cos, \sin \colon \RR \to \RR$ come 
\[
 \cos x = \Re \phi(x/\tau), \qquad
 \sin x = \Im \phi(x/\tau).
\]
Visto che $\abs{phi(x/\tau)}^2=1$ si ha $1 = \abs{\cos x + i \sin x}$ da cui 
si ottiene la \emph{formula fondamentale della trigonometria:}
\[
\cos^2 x + \sin^2 y = 1.  
\]  
Dall'additività di $\phi$ si ha
\begin{align*}
  \cos(x+y) + i\sin(x+y)
  &= \phi\frac{x+y}{\tau} 
  = \phi(x/\tau) \cdot \phi(y/\tau)\\
  &= (\cos x + i\sin x) \cdot (\cos y + i \sin y)\\
  &= \cos x\cos y - \sin x \sin y + i (\sin x\cos y + \cos x \sin y)
\end{align*}
e quindi si ottengono le \emph{formule di addizione} 
per le funzioni trigonometriche:
\[
\cos(x+y) = \cos x\cos y - \sin x \sin y, \qquad 
\sin(x+y) = \sin x \cos y + \cos x\sin y.  
\]
Visto che (ricordiamo che $\abs{\phi(x)}=1$)
\[
  \phi(-x) = 1/\phi(x) = \overline{\phi(x)}
\]
si ha $\cos(-x)+i\sin(-x) = \cos x-i\sin x$ 
da cui risulta che $\sin$ è una funzione \emph{dispari}
mentre $\cos$ è una funzione pari:
\[
\sin(-x) = -\sin x,\qquad 
\cos(-x) = \cos x.  
\]
Infine essendo $\phi$ una funzione di periodo $1$ le funzioni $\sin x$ 
e $\cos x$ risultano avere periodo $\tau$.
Per quanto riguarda la monotonia, sappiamo che 
la funzione $\sin x = \Im f(x/\tau)$ è 
strettamente crescente se $x\in\closeinterval{-\frac \tau 4}{\frac \tau 4}$.


\section{note storiche}

\label{nota:Peano}%
\index{Peano!Giuseppe}%
\emph{Giuseppe Peano} (1858--1932), matematico torinese, contribuì a porre 
i fondamenti della logica matematica. 
La notazione $\exists$ per il quantificatore universale si deve a lui.
La definizione originale di Peano prendeva $1$ come primo numero
naturale ma nella matematica moderna risulta più comodo includere anche $0$ 
tra i numeri naturali, così come si considera il vuoto tra gli insiemi.

\label{nota:Galileo}%
\index{Galileo!Galilei}%
\emph{Galileo Galilei} (1564--1642) osservò che i quadrati 
perfetti: $1,4,9,16,\dots$ sono da un lato una piccola parte 
di tutti i numeri naturali (questi numeri si distanziano 
sempre di più tra loro) ma d'altro canto sono tanti quanti i numeri naturali 
perché la corrispondenza $n\mapsto n^2$ è biunivoca.

\label{nota:Cantor}%
\index{Cantor!Georg}%
\label{nota:Russell}%
\index{Russell!Bertrand}%
\label{nota:Frege}%
\index{Frege!Gottlob}%
La teoria degli insiemi
è stata introdotta da \emph{Georg Cantor} (1845--1918) senza una vera formalizzazione logica
(oggi la chiameremmo \emph{teoria ingenua degli insiemi}).
\emph{Gottlob Frege} (1848--1925) fu il primo matematico che tentò di formalizzare 
la teoria degli insiemi di Cantor. 
Nel 1902 \emph{Bertrand Russell}, avendo letto il lavoro di Frege, 
gli invio una lettera che enunciava il paradosso da lui scovato:
``Mi trovo in completo accordo con lei in tutte le parti essenziali, in particolare
quando lei rifiuta ogni elemento psicologico dando un grande valore
all'ideografia %[Begriffsschrift]
per il fondamento della matematica e della logica formale [\dots] c'è solo
un punto dove ho incontrato una difficoltà [...]''.
La risposta di Frege (22 giugno 1902) è deprimente:
``La sua scoperta della contraddizione mi ha causato una grandissima sorpresa e,
direi, costernazione, perché ha scosso le basi su cui intendevo costruire l'aritmetica.''

\chapter{continuità e limiti}
\label{ch:successioni}

%\subsection{assiomatica dei reali}

Nel capitolo precedente abbiamo costruito l'insieme numerico $\RR$ dei numeri reali, e abbiamo 
messo in evidenza il fatto che si tratta di un campo ordinato e continuo
(definizione~\ref{def:campo_ordinato}). 
Abbiamo anche visto che la proprietà di essere un campo ordinato e continuo caratterizza univocamente $\RR$.
Dunque, in effetti, non sarà più necessario fare riferimento alla costruzione di $\RR$ ma potremo 
d'ora in poi semplicemente pensare che $\RR$ è un dato campo ordinato e continuo.
Tale proprietà caratterizza $\RR$ ed è quindi una informazione sufficiente per poter dedurre 
qualunque altra sua proprietà. 

A questo punto possiamo identificare anche gli insiemi $\NN$, $\ZZ$ e $\QQ$ come particolari sottoinsiemi 
dell'insieme $\RR$, senza doverli costruire esplicitamente.

Per identificare $\NN$ l'idea è quella di considerare il più piccolo sottoinsieme di $\RR$ che 
contiene $0$ e che se contiene $n$ contiene anche $n+1$. 
Per formalizzare questa idea diremo che
$X\subset \RR$ è \emph{induttivo} se $0\in X$ e se $n\in X$ implica $n+1\in X$. 
Definiamo quindi $\NN$ come l'intersezione di tutti gli insiemi induttivi:
\[
 \NN = \cap\ENCLOSE{X\subset \RR\colon X \text{ induttivo}}.
\]
Chiaramente $\NN$ è esso stesso induttivo. Ed è facile dimostrare che soddisfa 
gli assiomi di Peano (def~\ref{def:assiomi_peano}).

Si osserva poi che $\NN$ è superiormente illimitato perché se esistesse $x\in \RR$ maggiorante di 
$\NN$ allora per assurdo anche $x-1$ sarebbe un maggiorante, 
contro l'ipotesi di continuità di $\RR$ che garantisce l'esistenza di un minimo dei maggioranti.
Da questo si ricava la proprietà archimedea: per ogni $x>0$ e $y>0$ esiste $n \in \NN$ tale che $nx>y$.
%E tramite la proprietà archimedea si deduce la densità di $\QQ$ in $\RR$: se $x<y$ posto $\eps=y-x>0$
%si trova $q\in \NN$ tale che $\frac{1}{q}<\eps$ e dunque si trova $p\in \ZZ$ tale che $x<\frac{p}{q}<y$.

Dopodiché si definisce $\ZZ$ come l'insieme di tutte le differenze tra coppie di numeri naturali: 
$\ZZ = \NN - \NN = \ENCLOSE{n-m\colon n,m\in \NN}$.

Infine si definisce $\QQ$ come l'insieme di tutti i quozienti $\frac{p}{q}$ con $p\in \ZZ$ e $q\in \NN$, 
$q\neq 0$.

\subsection{valore assoluto}

\begin{definition}[valore assoluto]
\mymark{***}
Definiamo il \emph{valore assoluto}%
\mymargin{valore assoluto}%
\index{valore!assoluto} $\abs{x}$ di un numero $x\in \RR$ nel seguente modo:
\[
\abs{x} =
\begin{cases}
  x & \text{se $x\ge 0$}, \\
  -x & \text{se $x<0 $}.
\end{cases}
\]
\end{definition}
  
\begin{proposition}[proprietà del valore assoluto]
\mymark{**}
Si ha
\begin{enumerate}
\item $\abs{x}\ge 0$ (positività)
\item $\big\lvert\abs{x}\big\rvert = \abs{x}$ (idempotenza)
\item $\abs{-x} = \abs{x}$ (simmetria)
\item $\abs{x\cdot y} = \abs{x}\cdot \abs{y}$ (omogenità)
\item $\abs{x+y} \le \abs{x} + \abs{y}$ (convessità)
\item $\abs{x-y} \le \abs{x-z} + \abs{z-y}$ (disuguaglianza triangolare)
\item $\big\lvert\abs{x}-\abs{y}\big\rvert \le \abs{x-y}$ (disuguaglianza triangolare inversa)
\end{enumerate}
Useremo inoltre spesso la seguente equivalenza (valida
anche con $<$ al posto di $\le$). Se $r\ge 0$ allora
\[
  \abs{x-y} \le r
  \iff
  y - r \le x \le y + r.
\]
\end{proposition}
%
\begin{proof}
\mymark{*}
Positività, idempotenza, simmetria e omogenità sono immediate conseguenze della definizione.

Dimostriamo ora l'ultima osservazione.
Se $x\ge y$ allora $x-y\ge 0$ e quindi $\abs{x-y} \le r$ è
equivalente a $x-y\le r$ cioè $x\le y+r$.
Se $x<y$ allora $x-y<0$ e quindi $\abs{x-y} \le r$ è
equivalente a $y-x \le r$ cioè $x\ge y-r$.
Viceversa se $y-r \le x \le y+r$ allora vale sia $x-y \le r$ che $y-x \le r$ e 
dunque $\abs{x-y}\le r$.

Osserviamo allora che per la precedente osservazione applicata
a $\abs{x-0} \le \abs{x}$ si ottiene
\[
  -\abs{x} \le x \le \abs{x}
\]
e sommando la stessa disuguaglianza con $y$ al posto di $x$ si
ottiene
\[
  -(\abs{x} + \abs{y}) \le x + y \le \abs{x} + \abs{y}
\]
che è equivalente alla proprietà di convessità:
\[
  \abs{x+y} \le \big\lvert\abs{x} + \abs{y}\big\rvert = \abs{x} + \abs{y}.
\]

Ponendo $y=z-x$ nella disuguaglianza precedente, si ottiene
\[
  \abs{z} \le \abs{x} + \abs{z-x}
\]
da cui
\[
  \abs{z} - \abs{x} \le \abs{z-x}.
\]
Scambiando $z$ con $x$ si ottiene la disuguaglianza opposta
e mettendole assieme si ottiene
la disuguaglianza triangolare inversa:
\[
\big\lvert \abs{z}-\abs{x} \big\rvert  \le \abs{z-x}.
\]

La disuguaglianza triangolare segue dalla convessità:
\[
  \abs{x-y} = \abs{x-z + z-y} \le \abs{x-z} + \abs{z-y}.
\]
\end{proof}

Osserviamo che dal punto di vista geometrico
$\abs{x-y}$ rappresenta la \emph{distanza} tra i punti
$x$ e $y$.

\subsubsection{parte intera}

\begin{theorem}[parte intera]
\mymark{*}%
  Dato $x\in \RR$ esiste un unico $m\in \ZZ$ tale che $m-1 < x \le m$.
\end{theorem}
%
\begin{proof}
  Supponiamo per un attimo che sia $x > 0$
  e consideriamo l'insieme $A=\ENCLOSE{n \in \NN \colon n\ge x}$.
  Tale insieme non è vuoto per la proprietà archimedea 
  e dunque ammette minimo per il principio del buon ordinamento.
  Se $m=\min A$ risulta quindi $m\in \NN$ e $m-1< x \le m$.

  Se $x\le 0$ per la proprietà archimedea esiste $k\in \NN$ tale che 
  $k>-x$. Allora applichiamo il risultato precedente a $x+k$ e consideriamo 
  $m-k$ al posto di $m$.
\end{proof}

\begin{definition}[parte intera]
  \mymark{**}%
  \mymargin{parte intera}%
\index{parte!intera}%
  Dato $x\in \RR$ denotiamo con $\lfloor x\rfloor$ l'unico intero
  che soddisfa
  \mymargin{$\lfloor\cdot\rfloor$} %% *** non viene bene nell'indice!
  \[
    x - 1 < \lfloor x \rfloor \le x
  \]
  e denotiamo con $\lceil x \rceil = - \lfloor -x \rfloor$ l'unico intero che soddisfa (verificare!)
  \mymargin{$\lceil\cdot\rceil$} %% *** non viene bene nell'indice!
  \[
    x \le \lceil x \rceil < x + 1.
  \]
  Si ha dunque
  \[
    \lfloor x \rfloor \le x \le \lceil x \rceil
  \]
  con entrambe le uguaglianze che si realizzano quando $x\in \ZZ$.
  I due interi $\lfloor x \rfloor$ e $\lceil x \rceil$
  sono la migliore approssimazione intera di $x$ rispettivamente
  per difetto e per eccesso.
  L'intero più vicino ad $x$ (approssimazione per \emph{arrotondamento}%
\mymargin{arrotondamento}%
\index{arrotondamento})
  è
  \[
    \left\lfloor x + \frac 1 2 \right\rfloor
  \quad \text{ossia} \quad
    \left\lceil x-\frac 1 2 \right\rceil
  \]
  (le due espressioni differiscono solamente quando $x$ si trova nel punto medio tra 
  due interi consecutivi, nel qual caso la prima approssima per eccesso e la seconda 
  per difetto).
\end{definition}

In alcuni testi si usa la notazione $[x]$ per denotare la parte intera $\lfloor x \rfloor$ e si definisce
anche la \emph{parte frazionaria}
\[
  \ENCLOSE{x} = x - [x].
\]
Per evitare ambiguità con il normale utilizzo delle parentesi
non useremo queste notazioni.

\subsection{potenza e radice $n$-esima}
%
%
\label{sec:potenza}%
\label{sec:radice}%

Se $n \in \NN$ è un numero naturale possiamo definire 
l'elevamento a potenza $x^n$ induttivamente, come moltiplicazione 
ripetuta, ad esempio: $x^3 = x\cdot x \cdot x$
(si veda il teorema~\ref{th:operazione_ripetuta}).

Si osservi anche che abbiamo definito $0^0=1$ dove lo $0$ 
alla base è un numero reale e lo $0$ all'esponente 
è un numero naturale.

% Il teorema~\ref{th:divisibile}
% ci garantiva l'esistenza della frazione $\frac x n$ come operazione inversa 
% della somma ripetuta: $n\cdot x$. 
% Lo stesso teorema, applicato alla struttura moltiplicativa, può essere utilizzato 
% per definire la radice $n$-esima: $\sqrt[n]{x}$.

Definiamo l'insieme dei \emph{reali positivi}:
\[
\RR_+ = \ENCLOSE{x\in \RR\colon x>0}
\]
e su di esso consideriamo l'operazione di moltiplicazione.

\begin{theorem}[$\RR_+$ è un gruppo moltiplicativo]
  \label{th:gruppo_moltiplicativo}%
  \index{gruppo!moltiplicativo}%
  \index{reali!positivi}%
L'insieme $\RR_+$ dei reali positivi con l'operazione di moltiplicazione 
e l'ordinamento usuale ereditato da $\RR$ risulta essere 
un gruppo totalmente ordinato, denso e continuo.
\end{theorem}
%
\begin{proof}
Sappiamo che il prodotto di due numeri positivi 
è un numero positivo. Anche il reciproco di un numero positivo
esiste sempre (perché escludiamo $0$) ed è positivo. 
L'elemento neutro $1$ è anch'esso positivo. 
Dunque risulta che $\RR_+$ è un gruppo moltiplicativo.
Non solo, l'ordinamento di $\RR$ è ovviamente un ordinamento anche su $\RR_+$
e mantiene le proprietà di essere un ordinamento totale 
denso e continuo visto che queste proprietà sono indipendenti dalla struttura di gruppo.
La compatibilità dell'ordinamento con la moltiplicazione 
è data dalla proprietà di monotonia della moltiplicazione.
\end{proof}

\begin{theorem}[radice $n$-esima]
  \label{radice!$n$-esima}%
Sia $n\in \NN$, $n\ge 1$ fissato.
Dato $y\in \RR$, $y \ge 0$, esiste un unico $x\in \RR$, $x\ge 0$ tale che $x^n = y$.
Tale numero $x$ si chiama \emph{radice} $n$-esima di $y$ e si 
denota con questo simbolo:
\[
  x = \sqrt[n]{y}.
\]

Se $n$ è pari e $y>0$ l'equazione $x^n=y$ ha due soluzioni, oltre alla soluzione positiva 
$\sqrt[n]{y}$ c'è anche una soluzione negativa $-\sqrt[n]{y}$.
Se $y=0$ l'equazione $x^n=0$ ha come unica soluzione $x=\sqrt[n]{0}=0$.
E se $y<0$ (sempre nel caso $n$ pari) l'equazione $x^n = y$ non ha soluzioni.

Se $n$ è dispari l'equazione $x^n=y$ ha una unica soluzione per ogni $y\in \RR$.
Se $y<0$ la soluzione è negativa ed è $x=-\sqrt[n]{\abs y}$.
Per comodità, se $n$ è dispari, definiamo la radice $n$-esima anche sui numeri negativi 
ponendo appunto 
\[
  \sqrt[n]{-y} \defeq  -\sqrt[n]{y}\qquad\text{(se $n$ è dispari).}
\]
\end{theorem}
\begin{proof}
Il teorema~\ref{th:divisibile} (divisibilità) può essere applicato al gruppo moltiplicativo 
$\RR_+$ e ci dice che se $n\neq 0$ dato $y\in \RR_+$ esiste un unico $x\in \RR_+$ tale 
che $x^n = y$. Se $y=0$ chiaramente solo $x=0$ è soluzione di $x^n= y$ per la regola 
di annullamento del prodotto. 

Le altre proprietà si deducono per simmetria osservando che $(-x)^n = (-1)^n\cdot x^n$
e $(-1)^n$ vale $1$ se $n$ è pari e $-1$ se $n$ è dispari per la regola del segno 
del prodotto.
\end{proof}

\index{radice!quadrata}%
\index{radice!cubica}%
Per $n=2$ si omette l'indicazione dell'esponente scrivendo $\sqrt y$ al posto di $\sqrt[2] y$.
La $\sqrt y$ si chiama \emph{radice quadrata} di $y$.
Per $n=3$ la radice $\sqrt[3]{y}$ si chiama \emph{radice cubica} di $y$.

\begin{figure}
  \begin{center}
    \begin{tikzpicture}[x=1.0cm,y=1.0cm]
      \clip(-3,-2) rectangle (4,3);
    	\draw[->] (-3,0) -- (4,0);
      \draw[->] (0,-2) -- (0,3);
      \draw[dotted] (1,0) -- (1,1) -- (0,1);
%      \draw[domain=0.2:4,samples=50,variable=\x,dashed,color=gray] plot
%        ({\x}, {1/(\x))});
%      \draw[domain=-3:-0.5,samples=50,variable=\x,dashed,color=gray] plot
%        ({\x}, {1/(\x))});
      \draw[domain=-2.0:2.0,smooth,variable=\x,thick,color=gray] plot
        ({\x}, {\x*\x*\x)});
      \draw[domain=-3.1:4.0,samples=200,variable=\x,color=gray] plot
        ({\x}, {pow(abs(\x),1/3)*\x / abs(\x)});
      \draw[domain=-3.0:3.0,smooth,variable=\x,thick] plot
        ({\x}, {\x*\x});
      \draw[domain=0.0:4.0,samples=200,variable=\x] plot
      ({\x}, {pow(\x,1/2)});
      \node at (-2.0,1.8) {$y=x^2$};
      \node at (3,2) {$y=\sqrt{x}$};
      \node[color=gray] at (-1.7,-1.8) {$y=x^{3}$};
      \node[color=gray] at (-2.0,-1) {$y=\sqrt[3]{x}$};
      \fill (1,0) node[below]{$1$} circle[radius=1.5pt];
      \fill (0,1) node[left]{$1$} circle[radius=1.5pt];
    \end{tikzpicture}
  \end{center}
  \caption{Grafici delle potenze $x^2$, $x^3$ e radici
  $\sqrt x$, $\sqrt[3] x$.}
  \label{fig:potenza_intera_radice}
\end{figure}

%\subsubsection{numeri irrazionali}
%
%
%Abbiamo quindi dimostrato che $\QQ$ non può essere continuo (altrimenti 
%sarebbe isomorfo a $\RR$ dove l'equazione $x^2=2$ ha soluzione). 
%Utilizzando l'irrazionalità di $\sqrt 2$ 
%possiamo ora esibire un esempio di sottoinsiemi di $\QQ$ che sono 
%\emph{separati} (nel senso della definizione~\ref{def:ordinamento_continuo})
%ma non hanno elemento di separazione in $\QQ$:
%\[
%A= \ENCLOSE{x\in\QQ\colon x^2< 2},
%\qquad
%B = \ENCLOSE{x\in \QQ\colon x<0, x^2>2}.  
%\]
%Chiaramente $A\le B$ ma in $\RR$ questi due insiemi 
%hanno come unico elemento di separazione $\sqrt 2$ che non è elemento di
%$\QQ$.
%
%% Abbiamo appena verificato che esistono numeri irrazionali. 
%% Può però essere sorprendente scoprire che non solo i numeri irrazionali 
%% sono infiniti (chiaramente se $x$ è irrazionale anche $n\cdot x$ è irrazionale 
%% se $n$ è intero o razionale) ma addirittura la cardinalità 
%% dei numeri irrazionali è maggiore di quella dei numeri razionali 
%% $\# (\RR\setminus \QQ) > \# \QQ$ (teorema~\ref{th:cantor_secondo}).
%
%Ogni numero razionale ha una rappresentazione finita. 
%Se $x=\frac p q$ possiamo scrivere $p$ e $q$ nella loro rappresentazione 
%decimale utilizzando una sequenza finita di cifre.
%Possiamo dare \emph{un nome} anche a molti
%numeri irrazionali. 
%Ad esempio $\sqrt 2$, $e$, $\pi$, $\sin 1$, $\ln 2$ identificheranno 
%alcuni numeri irrazionali. 
%Ma non è possibile dare una rappresentazione \emph{finita} di ogni numero reale.
%Per questo motivo è rilevante il fatto che i numeri razionali siano \emph{densi}
%nei numeri reali perché questo significa che ogni numero reale può essere 
%approssimato, con un errore piccolo a piacere, tramite un numero razionale.
%
\begin{theorem}[densità di $\QQ$ in $\RR$]
\label{th:densita_frazioni}%
\index{densità!frazioni}%
\index{approssimazioni!decimali}%
Dati $a,b\in \RR$ se $a<b$ esiste $q\in \QQ$ tale che $a< q < b$.

Equivalentemente dato $x\in \RR$ e dato $\eps>0$ esiste $q\in \QQ$ 
tale che $\abs{x-q} < \eps$.

Inoltre la frazione $q= \frac{p}{n}$ può essere scelta 
in modo tale che il denominatore sia una potenza di $10$: $n=10^k$.
\end{theorem}
\begin{proof}
La prima parte è conseguenza della seconda perché 
dati $a<b$
se scegliamo 
$x= \frac{a+b}{2}$ ed $\eps = \frac{b-a} 2$ se 
esiste $q$ tale che $\abs{x-q}< \eps$ 
allora 
\[
   a = \frac{a+b}{2} - \eps < q < \frac{a+b}{2} + \eps = b.
\]

Per dimostrare la seconda parte dati $x$ e $\eps$ basterà scegliere 
$n\in \NN$ tale che $\frac 1 n < \eps$ (l'esistenza di $n$ è garantita 
dal teorema~\ref{th:archimede}, proprietà archimedea) e predere $q = \frac{\lfloor nx\rfloor}{n}$
cosicché, per le proprietà della parte intera, si ha:
\[
  nx-1\le \lfloor nx\rfloor \le nx 
  \qquad\text{e quindi}\qquad
  x - \frac 1 n \le q \le x
\]
da cui $0 \le x-q < \frac 1 n < \eps$ come volevamo dimostrare.

Per avere una frazione con denominatore potenza di $10$ basta osservare 
che se aumentiamo $n$ la precisione della approssimazione aumenta. 
Dunque basta osservare che per ogni $n\in \NN$ esiste $d\in \NN$
tale che $10^d \ge n$ (basta scegliere $d=n$ e dimostrare per induzione 
che $10^n\ge n$).
\end{proof}

\begin{exercise}\label{ex:densita_irrazionali}
  Dimostrare che anche gli irrazionali 
  sono densi in $\RR$ cioè che se $a<b$ allora esiste 
  $x \in \openinterval{a}{b}$, $x\not \in \QQ$.
\end{exercise}

\subsubsection{frazioni decimali}
%
Le frazioni il cui denominatore è una potenza
di $10$ si chiamano frazioni decimali:
\[
  x = \frac{p}{10^d}, \qquad p\in \ZZ, d\in \NN.
\]
Tali frazioni si possono rappresentare
scrivendo il numero
intero $p$ e segnando un punto
\mynote{%
in Italia si preferisce utilizzare la virgola, ma
ci rassegnamo alla notazione anglosassone che ormai è
ubiqua in tutta la strumentazione elettronica.
}%
di separazione
prima della $d$-esima cifra a partire da destra.
Ad esempio scriveremo:
\[
  1.4142 = \frac{14142}{10^4}.
\]
In generale una frazione $\frac{p}{q}\in \QQ$
può essere scritta in forma decimale solamente
se, quando ridotta ai minimi termini,
risulta che $q$ non ha fattori primi diversi
da $2$ e $5$ (in quanto le potenze di dieci
hanno solo questi fattori).

Anche se non tutte le frazioni hanno una rappresentazione 
decimale finita, in ogni caso le frazioni decimali 
sono dense in $\RR$ (come dimostrato nel teorema~\ref{th:densita_frazioni})
e quindi ogni numero reale può essere approssimato, con errore piccolo a piacere,
mediante una frazione decimale.

Scriveremo
\mymargin{$\approx$}%
\index{$\approx$}%
\[
  x \approx \frac{p}{10^d}
\]
(si può leggere: ``$x$ è approssimativamente uguale a\dots'')
se
\[
    \frac{p-1}{10^d} < x < \frac{p+1}{10^d}.
\]

Ad esempio possiamo scrivere
\[
  \sqrt 2 \approx 1.41 = \frac{141}{10^2}
\]
per intendere%
\mynote{%
Si osservi che in base alla definizione data sarebbe anche corretto 
scrivere $\sqrt 2 \approx 1.42$ che però è una approssimazione 
peggiore. 
Questa ambiguità è necessaria se vogliamo evitare i casi 
limite in cui bisogna conoscere molte più cifre decimali di quelle richieste 
per capire qual è la migliore approssimazione.
}%
\begin{equation}\label{eq:approx_sqrt2}
\frac{140}{100} < \sqrt 2 < \frac{142}{100}.
\end{equation}
Nel calcolo numerico scientifico ogni uguaglianza numerica è intesa nel 
senso precedente, se non specificato diversamente. 

\begin{exercise}
Dimostrare \ref{eq:approx_sqrt2} senza 
utilizzare la calcolatrice.
\end{exercise}

Osserviamo infine che anche i calcolatori utilizzano una rappresentazione frazionaria, 
nel caso specifico \emph{binaria}, dei numeri.
I numeri frazionari su cui opera un calcolatore sono tutti della forma $\frac{p}{2^d}$.
Visto che $2$ divide $10$, le frazioni binarie sono anche sempre frazioni decimali, 
in particolare anche questi numeri sono densi in $\RR$.
\mynote{In un moderno calcolatore a 64 bit la rappresentazione binaria (float64) 
utilizza frazioni binarie con esponente $d\le 1024$. 
Questo permette di rappresentare numeri con una precisione non superiore alle 
$308$ cifre decimali. 
La rappresentazione è a virgola mobile e quindi questa precisione si può raggiungere 
solamente con numeri vicini allo zero, per numeri vicini a $1$ 
si ha invece $d\le 53$ che corrisponde ad una precisione di quasi $16$ cifre decimali.}

\subsubsection{frazioni decimali periodiche}

Le frazioni non decimali si possono scrivere con uno sviluppo
decimale \emph{periodico}. 
Non useremo mai questa notazione
che ricordiamo solamente con un esempio.
Il numero
\[
  x = 12.34\overline{567}
    = 12.34567\overline{567}
\]
è la frazione $x$ che risolve l'equazione
\[
  \frac{100x - 1234}{1000}
  = 100x-1234.567
  \qquad
\enclose
{\frac{0.\overline{567}}{1000}
= 0.000\overline{567} }
\]
ovvero
\[
  1234567 - 1234 = 99900 \cdot x,
  \qquad x = \frac{1234567-1234}{99900}.
\]

\begin{exercise}
Si verifichi che $0.\overline 9 = 1$.
\end{exercise}
    
Per passare dalla rappresentazione frazionaria alla rappresentazione 
decimale periodica basta svolgere la divisione in colonna.
Se l'algoritmo termina la frazione era decimale, altrimenti l'algoritmo 
diventa periodico e le cifre decimali si ripetono indefinitamente.


\subsection{punti all'infinito}
\label{sec:reali_estesi}
%%%%%%%%%%%%%%%%%%%
%%%%%%%%%%%%%%%%%%%
%%%%%%%%%%%%%%%%%%%

\begin{definition}[reali estesi]
\mymargin{$\bar\RR$}%
\index{$\bar{\RR}$}
Denotiamo con $\bar \RR=\RR \cup \ENCLOSE{+\infty, -\infty}$ l'insieme dei numeri reali
\mymargin{$+\infty$, $-\infty$}%
\index{$+\infty$, $-\infty$}
a cui vengono aggiunti due ulteriori \emph{quantità} che chiameremo
\emph{infinite} e che denotiamo con $+\infty$ e $-\infty$.
Diremo che $x\in \bar \RR$ è \emph{finito} se $x\in \RR$.
\end{definition}


Estendiamo la relazione d'ordine imponendo che valga
\[
  -\infty \le x \le +\infty, \qquad \forall x \in \bar\RR.
\]

Estendiamo anche la addizione e moltiplicazione
tra reali estesi imponendo che valga per ogni $x\in \bar \RR$
\begin{gather*}
  x + (+\infty) = +\infty, \qquad \text{se $x\neq -\infty$}\\
  x + (-\infty) = -\infty, \qquad \text{se $x\neq +\infty$}\\
  x \cdot (+\infty) = +\infty, \qquad
  x \cdot (-\infty) = -\infty, \qquad \text{se $x>0$} \\
  x \cdot (+\infty) = -\infty, \qquad
  x \cdot (-\infty) = +\infty, \qquad \text{se $x<0$}.
\end{gather*}

Si definiscono anche:
\[
 -(+\infty) = -\infty, \qquad
 -(-\infty) = +\infty, \qquad
 \frac{1}{+\infty} = \frac{1}{-\infty}=0
\]
facendo però attenzione che
questi formalmente non sono \emph{opposto}
e \emph{reciproco} in quanto
su $\bar \RR$ non sono più garantite
le regole: $x + (-x) = 0$ e $x \cdot (1/x) = 1$.
Infatti
le operazioni $(+\infty) + (-\infty)$ e $+\infty \cdot 0$ vengono
lasciate indefinite.

Definiamo anche il valore assoluto: $\abs{+\infty} = \abs{-\infty} = +\infty$.

Possiamo infine definire la sottrazione e la divisione tramite
addizione e moltiplicazione:
\[
  x - y = x + (-y), \qquad \frac{x}{y} = x \cdot \frac{1}{y}.
\]

Osserviamo che rispetto all'ordinamento di $\bar \RR$ tutti 
i sottoinsiemi sono limitati in quanto $+\infty = \max \bar \RR$ e  
$-\infty = \min \bar \RR$.
Dunque se diciamo che $A\subset \RR \subset \bar \RR$ è limitato 
(o illimitato) stiamo sempre facendo riferimento all'ordinamento di $\RR$ 
per il semplice fatto che la limitatezza in $\bar \RR$ è banale.
Diremo quindi sempre che $\NN$ è superiormente illimitato anche se
in $\bar \RR$ esiste il maggiorante $+\infty$.

D'altro canto è utile osservare che $\bar \RR$ ha un ordinamento 
continuo, come quello di $\RR$ e dunque, per il Teorema~\ref{th:sup}
sappiamo che ogni sottoinsieme di $\bar \RR$ ha sempre 
estremo superiore ed estremo inferiore anche se è vuoto o illimitato.

Dunque è consuetudine estendere gli operatori $\sup$ e $\inf$
a tutti i sottoinsiemi di $\RR$ ponendo 
\begin{align*}
  \sup A &= +\infty \quad \text{se $A$ non è superiormente limitato},\\
  \inf A &= -\infty \quad \text{se $A$ non è inferiormente limitato},\\
  \sup \emptyset &= -\infty, \\ 
  \inf \emptyset &= +\infty.
\end{align*}

\subsection{intervalli}

\begin{definition}[intervallo]
\label{def:intervallo}%
\mymargin{intervallo}%
\index{intervallo}%
Un insieme $I\subset \bar\RR$ si dice essere un \emph{intervallo}
se contiene tutti i punti intermedi:
\[
  \text{se $x, y \in I$ e $x<z<y$ allora $z \in I$.}
\]
\end{definition}
%
\begin{theorem}[caratterizzazione intervalli di $\RR$]
Sia $I\subset \RR$ e siano $a=\inf I$, $b=\sup I$,
$a,b\in \bar \RR$,
i suoi estremi. 
Allora $I$ è un intervallo se e solo se per ogni $x$ tale 
che $a<x<b$ si ha $x\in I$
\end{theorem}
%
\begin{proof}
Supponiamo che $I$ sia un intervallo.
Se $I=\emptyset$ si ha $a>b$ e quindi nessun $c$ verifica $a<x<b$.
Supponiamo $I\neq \emptyset$ e
sia $a < x < b$.
Visto che $a$ è il massimo dei minoranti di $I$
il numero $x$ non è un minorante dunque
deve esistere $y \in I$ tale
che $y < x$. 
Analogamente dovrebbe esistere $z\in I$
con $x < z$.
Ma allora, per definizione di intervallo, anche $x\in I$.

Viceversa supponiamo di avere $y,z\in I$ e $x\in \RR$ con $y<x<z$.
Visto che $y\ge a$ e $z\le b$ si ha $a<x<b$ e dunque, per ipotesi, 
$x\in I$, come richiesto dalla definizione di intervallo.
\end{proof}

Il teorema precedente ci dice che un intervallo contiene tutti i punti intermedi 
ai propri estremi. 
Gli estremi, tuttavia, possono essere o non essere inclusi nell'intervallo.
Punti esterni agli estremi non possono invece essere elementi dell'intervallo.
Possiamo quindi caratterizzare tutti gli intervalli di $\bar \RR$
introducendo le seguenti notazioni. Dati $a,b\in \bar \RR$ con $a\le b$
tutti i possibili intervalli con estremi $a$ e $b$ sono i seguenti:
\begin{equation}\label{eq:499494}
\begin{aligned}
\closeinterval{a}{b} &= \ENCLOSE{x\in \bar \RR\colon a \le x \le b} \\
\closeopeninterval{a}{b} &= \ENCLOSE{x\in \bar \RR\colon a \le x < b} \\
\opencloseinterval{a}{b} &= \ENCLOSE{x\in \bar \RR\colon a < x \le b}\\
\openinterval{a}{b} &= \ENCLOSE{x\in \bar \RR\colon a < x < b}.
\end{aligned}
\end{equation}
Abbiamo utilizzato le parentesi quadre per indicare che gli estremi
sono inclusi e le parentesi tonde per indicare che gli estremi sono esclusi.
Osserviamo che in alcuni testi si usano le parentesi quadre rovesciate al posto
delle parentesi tonde.

Se $a>b$ potremmo definire per convenzione:
\begin{equation}\label{eq:488364}
  [a,b] = [b,a], \quad
  [a,b) = (b,a], \quad
  (a,b] = [b,a), \quad
  (a,b) = (b,a).
\end{equation}
Si faccia però attenzione che in altri testi gli intervalli con gli estremi
scambiati non vengono definiti oppure vengono considerati vuoti.

La convenzione può essere utile perché in generale se $\vec a, \vec b$ sono
elementi di uno spazio vettoriale reale $V$ allora ha senso
definire:
\begin{align*}
    [\vec a,\vec b] &= \ENCLOSE{(1-t)\vec a + t \vec b\colon t\in [0,1]},\\
    [\vec a,\vec b) &= \ENCLOSE{(1-t)\vec a + t \vec b\colon t\in [0,1)},\\
    (\vec a,\vec b] &= \ENCLOSE{(1-t)\vec a + t \vec b\colon t\in (0,1]},\\
    (\vec a,\vec b) &= \ENCLOSE{(1-t)\vec a + t \vec b\colon t\in (0,1)}.
\end{align*}
L'intervallo $[\vec a,\vec b]$ è quindi il segmento di estremi
$\vec a$ e $\vec b$ e può essere definito anche se sullo spazio
vettoriale non è dato un ordinamento.
Questo rimane coerente con la definizione~\eqref{eq:499494}
data sopra solamente se adottiamo la convenzione~\eqref{eq:488364}.

Noi considereremo per lo più intervalli di $\RR$ (non di $\bar \RR$): in tal
caso gli estremi infiniti saranno quindi sempre esclusi dall'intervallo.

\subsection{andamento del grafico di una funzione}
%
Se $f\colon A \subset \RR\to \RR$ è una funzione, un modo molto
utile di rappresentarla graficamente è quello di disegnarne il
grafico, ovvero la curva del piano cartesiano:
\[
   G_f = \ENCLOSE{(x,y)\in A\times \RR\colon y = f(x)}.
\]
Molte proprietà della funzione potranno essere riconosciute
geometricamente guardandone il grafico.

\begin{definition}[simmetrie]
Sia $f\colon A \subset \RR \to \RR$ una funzione.
Diremo che $f$ è:
\begin{enumerate}
\item \emph{pari}%
\mymargin{pari}%
\index{pari}
\index{funzione!pari}%
se $A=-A$ (significa che se $x\in A$ allora anche $-x\in A$) e
\[
  f(-x) = f(x);
\]
\item \emph{dispari}%
\mymargin{dispari}%
\index{dispari}
\index{funzione!dispari}%
se $A=-A$ e
\[
  f(-x) = -f(x);
\]
\item \emph{periodica}%
\mymargin{periodica}%
\index{periodico}
\index{funzione!periodica}%
di periodo $T$ se $A+T=A$
(significa che $x\in A \iff x+T \in A$)
e se per ogni $x\in A$ si ha
\[
  f(x+T)=f(x)
\]
\end{enumerate}
\end{definition}

Ad esempio se $n\in \ZZ$ la funzione $f(x)=x^n$
è pari se $n$ è pari ed è dispari se $n$ è dispari.
Il grafico di una funzione dispari ha una simmetria
centrale, in quanto se $(x,f(x))\in G_f$ allora
anche $(-x,-f(x)) = (-x,f(-x))\in G_f$.
Il grafico di una funzione pari ha invece una
simmetria rispetto all'asse delle ordinate $x=0$
infatti se $(x,f(x))\in G_f$ allora $(-x,f(x)) = (-x,f(-x)) \in G_f$.

La funzione $f(x) = x - \lfloor x\rfloor$ (la parte frazionaria di $x$)
è un esempio di funzione periodica di periodo $T=1$. Infatti
è chiaro che $\lfloor x+1\rfloor = \lfloor x \rfloor +1$ e quindi
$f(x+1)=f(x)$.

Si osservi che \emph{dispari}, per le funzioni, non è la negazione
di \emph{pari}.
La funzione $f(x) = x+1$ non è né pari, né dispari, né periodica
(verificare).


\begin{definition}[zeri]
  Se $f\colon A\subset \RR \to \RR$ è una funzione diremo che
  $x\in A$ è uno \emph{zero} di $f$ se $f(x)=0$.
  L'\emph{insieme degli zeri}%
\mymargin{insieme degli zeri}%
\index{insieme!degli zeri}
  \index{zero!di una funzione}%
  è quindi dato da
  \[
    f^{-1}(\ENCLOSE{0}) = \ENCLOSE{x\in \RR\colon f(x) = 0}.
  \]
\end{definition}

Abbiamo già accennato al fatto che uno dei problemi più comuni in
matematica è quello di invertire una funzione. In particolare 
dato $y\in \RR$ ci si chiede quali siano gli $x\in \RR$ 
tali che $f(x)=y$. 
Questo problema si riconduce
a trovare gli zeri della funzione $f(x)-y$ e per questo motivo 
siamo interessati allo studio degli zeri.

Riprendiamo ora la definizione~\ref{def:monotonia} (monotonia) che da ora in avanti 
potrà essere applicata alle funzioni $f\colon A \to \RR$ definite 
su un insieme $A\subset \RR$.

Dal punto di vista grafico una funzione $f$ è crescente
se preso qualunque punto $(x,f(x))$ sul grafico della funzione
e tracciati gli assi paralleli agli assi cartesiani, passanti
per il punto fissato, si osserva che il grafico della funzione
è tutto contenuto nel primo e terzo quadrante determinati
dagli assi traslati.

E' facile verificare che la funzione $f\colon [0,+\infty)\to \RR$
definita da $f(x)=x^n$
è strettamente crescente se $n$ è un intero positivo.
Se però consideriamo la funzione definita su tutto
$\RR$: $f\colon \RR \to \RR$,
$f(x)=x^n$ allora solo se $n$ è dispari la funzione rimane
strettamente crescente
(le funzioni pari non possono mai essere strettamente crescenti se
il loro dominio contiene almeno tre punti).

Se una funzione non è monotona è piuttosto comune studiare 
la monotonia della funzione ristretta a particolari intervalli: 
su alcuni intervalli la funzione (ristretta) potrà essere crescente e su altri 
intervalli potrà essere decrescente.

\begin{exercise}
Verificare che la composizione di funzioni monotone è una
funzione monotona e la composizione di funzioni strettamente
monotone è strettamente monotona.
Quando è che la funzione composta risulta crescente?
Quando decrescente?
\end{exercise}

\begin{exercise}
Si dimostri che applicando una funzione strettamente crescente ai due
membri di una equazione o disequazione (stretta o larga che sia)
si ottiene una equazione o disequazione equivalente.
Ovviamente è necessario che la funzione sia definita dove viene applicata.

Lo stesso vale per le funzioni strettamente decrescenti 
se però si cambia il verso della disequazione.
\end{exercise}

\begin{definition}[funzioni limitate, massimo/minimo]
\label{def:funzione_limitata}%
Se $f\colon A \to \RR$ è una funzione allora definiamo
l'estremo superiore di $f$ come l'estremo superiore
dell'immagine di $f$:
\[
  \sup f = \sup_{x\in A} f(x) = \sup f(A).
\]
In maniera analoga si definiscono l'estremo inferiore $\inf f$,
il massimo $\max f$ e il minimo $\min f$.

Dunque il massimo di una funzione è (se esiste) il valore massimo
che la funzione può assumere. I punti $x$ in cui
la funzione assume il valore massimo $f(x)$ vengono chiamati
\emph{punti di massimo}.
\mymargin{punto di massimo/minimo}%
\index{punto di massimo/minimo}%
\index{punto!di massimo}%
\index{punto!di minimo}%
Analogamente i punti in cui la funzione
assume il valore minimo (sempre che esistano) vengono
chiamati \emph{punti di minimo}.

Diremo che la funzione $f$ è
\emph{superiormente limitata}%
\mymargin{funzione superiormente limitata}%
\index{superiormente limitata}
se $\sup f<+\infty$
ovvero se esiste $M\in \RR$ tale che
\[
\forall x\in A \colon f(x) \le M.
\]
Diremo che la funzione $f$ è
\emph{inferiormente limitata}%
\mymargin{funzione inferiormente limitata}%
\index{inferiormente!limitato}
se $\inf f > -\infty$ ovvero se esiste $M\in \RR$ tale che
\[
 \forall x \in A \colon f(x) \ge M.
\]
Diremo che la funzione $f$ è \emph{limitata}%
\mymargin{funzione limitata}%
\index{limitata}
se è sia superiormente che inferiormente limitata ovvero
se $\sup\abs{f}<+\infty$ cioè se esiste $M\in \RR$ tale che
\[
\forall x \in A \colon \abs{f(x)}\le M.
\]
\end{definition}

Nel seguente esercizio abbiamo un esempio di funzione limitata.
\begin{exercise}
Si consideri la funzione $f\colon \RR\to\RR$
\[
 f(x) = \frac{1}{1+x^2}.
\]
Verificare che $f$ è pari, che $\max f = \sup f = 1$, che $0$ è l'unico punto di massimo,
che $\inf f = 0$ e che $\min f$ non esiste.
\end{exercise}

\subsection{funzioni lineari}

\begin{figure}
  \begin{center}
    \begin{tikzpicture}[x=1.0cm,y=1.0cm]
    	\draw[->] (-1,0) -- (4,0);
      \draw[->] (0,-1) -- (0,3);
      % y = (3/5)*x + 1/5
      \fill[fill=lightgray,draw] (1,0.8) -- (2,0.8) node [midway,below]{$\Delta x$} -- (2,1.4) node [midway,right]{$\Delta f(x)$};
      \draw[thick] (-1,-0.4) -- (4,2.6);
      \draw[->] (1,0) node[below]{$x_1$} -- (1,0.8)
      -- (0,0.8) node[left]{$f(x_1)$};
      \draw[->] (2,0) node[below]{$x_2$} -- (2,1.4)
      -- (0,1.4) node[left]{$f(x_2)$};
      \fill (1,0.8) circle[radius=1.5pt];
      \fill (2,1.4) circle[radius=1.5pt];
      \node[left] at (3.0,2.5) {$y=mx+q$};
    \end{tikzpicture}
  \end{center}
  \caption{Il grafico di una funzione lineare.}
  \label{fig:funzione_lineare}
\end{figure}

Le \emph{funzioni lineari}%
\mymargin{funzione lineare}%
\index{funzioni lineari}
$f\colon \RR \to \RR$ sono le funzioni per le quali
esistono $m,q\in\RR$ tali che%
\mynote{%
Attenzione: nell'ambito dell'algebra lineare queste
funzioni verrebbero chiamate \emph{lineari affini}, mentre
le funzioni lineari dovrebbero sempre avere $q=0$.
Noi invece (come spesso accade nell'ambito dell'analisi)
chiameremo lineari queste funzioni e chiameremo
\emph{lineari omogenee} quelle con $q=0$.
Il termine \emph{lineare} pervade tutta la matematica 
e si applica in particolare alle equazioni che si ottengono 
tramite le funzioni lineari.
Purtroppo il nome scelto è fuorviante: la parola \emph{linea} viene 
usata a volte come abbreviazione di \emph{linea retta}, quando 
invece sarebbe più giusto utilizzare l'abbreviazione \emph{retta}
in quanto una linea può benissimo essere curva.
In altri contesti (come ad esempio nell'ambito degli ordinamenti)
il termine \emph{lineare} rappresenta un oggetto unidimensionale
senza ramificazioni ed è quindi maggiormente aderente 
al significato originale della parola.
} % marginnote
\[
  f(x) = mx + q.
\]

Se prendiamo due punti $(x_1,f(x_1))$
e $(x_2,f(x_2))$ sul grafico di una funzione lineare
possiamo osservare che si ha
\[
  \frac{f(x_2) - f(x_1)}{x_2 - x_1} = m.
\]
Il coefficiente $m$, dunque, rappresenta la pendenza del
grafico di $f$, ovvero il rapporto tra la variazione
dei valori della funzione $\Delta f = f(x_2) - f(x_1)$
e la variazione della variabile in ingresso
$\Delta x = x_2 - x_1$.
Geometricamente questo è il rapporto tra i due cateti
(base e altezza) che formano un triangolo rettangolo la
cui ipotenusa è il segmento che congiunge i due punti sul grafico.
Il fatto che questo rapporto sia costante significa,
in base al teorema di Talete, che i punti del grafico sono
allineati ovvero che il grafico di una funzione lineare è,
dal punto di vista geometrico, una retta.

\begin{definition}[retta]
  \index{retta}%
  \index{linea!retta}%
  Una \emph{linea retta} (più semplicemente: \emph{retta}) in uno spazio affine
  è un sottospazio affine di dimensione 1 
  ovvero la traslazione di un sottospazio vettoriale di dimensione 1
  (si rimanda al corso di geometria).
\end{definition}

Tutte le rette del piano, 
tranne quelle parallele all'asse delle ordinate,
sono grafico di una funzione lineare.

Si osservi che per $m>0$ la funzione è strettamente crescente,
per $m=0$ la funzione è costante e per $m<0$ la funzione è
strettamente decrescente.

\subsection{funzioni quadratiche}
\label{sec:funzioni_quadratiche}

\begin{figure}
  \begin{center}
    \begin{tikzpicture}[x=1.0cm,y=1.0cm]
      \clip(-1,-1) rectangle (4,3);
    	\draw[->] (-1,0) -- (4,0);
      \draw[->] (0,-1) -- (0,3);
      % y = (3/5)*x + 1/5
      \draw[dashed] (1.75,-1) -- (1.75,4);
      \node[right] at (1.75,2.5){$x=-\frac b {2a}$};
      \draw[domain=-1.0:4.0,smooth,variable=\x,thick] plot
      ({\x}, {0.5*(\x-0.5)*(\x-3.0)});
      \fill (0.5,0.0) node[below]{$x_1$} circle[radius=1.5pt];
      \fill (3,0.0) node[below]{$x_2$} circle[radius=1.5pt];
      \node at (1.8,1) {$y=ax^2+bx+c$};
    \end{tikzpicture}
  \end{center}
  \caption{Il grafico di una funzione quadratica.}
  \label{fig:funzione_quadratica}
\end{figure}

Le funzioni espresse mediante un \emph{polinomio di secondo grado}%
\mymargin{polinomio di secondo grado}%
\index{polinomio!di secondo grado}
\begin{equation}\label{eq:funzione_quadratica}
  f(x) = ax^2 + bx +c
\end{equation}
con $a,b,c\in \RR$, $a\neq 0$, si possono chiamare
\emph{funzioni quadratiche}%
\mymargin{funzioni quadratiche}%
\index{funzione!quadratica}.

Il modello di funzione quadratica è la funzione
$f(x) = x^2$ che (come tutte le potenze di esponente positivo e pari)
risulta essere una funzione pari, strettamente crescente
sull'intervallo $[0,+\infty)$ e strettamente decrescente
su $(-\infty,0]$. La funzione assume solamente valori non negativi
e si annulla solo per $x=0$.
Dunque l'equazione
\[
  x^2 = b
\]
non ha soluzione se $b<0$ ed ha come unica soluzione $x=0$ se $b=0$.
Se $b>0$ sappiamo che
questa equazione ha una unica soluzione positiva $x_1 = \sqrt{b}$
e, per simmetria, ha anche una soluzione negativa $x_2 = -\sqrt{b}$.
Sintetizzando si usa scrivere $x_{1,2} = \pm \sqrt{b}$
per condensare in una unica riga le due definizioni.

La generica funzione quadratica~\eqref{eq:funzione_quadratica}
può essere ricondotta al caso modello tramite un cambio
di variabile lineare. In pratica si cerca di comporre il quadrato
di un binomio con un procedimento chiamato
\emph{completamento del quadrato}%
\mymargin{completamento del quadrato}%
\index{completamento del quadrato}:
\begin{equation}\label{eq:24589}
\begin{aligned}
f(x) = ax^2+bx+c
  &= a \Enclose{x^2+\frac b a x + \frac c a}\\
  &= a \Enclose{x^2+2 \frac{b}{2a} x + \frac{b^2}{4a^2} - \frac{b^2}{4a^2} + \frac c a}\\
  &= a \Enclose{\enclose{x+\frac{b}{2a}}^2 - \frac{b^2-4ac}{4a^2}} \\
  &= a\enclose{x+\frac b{2a}}^2  - \frac{b^2-4ac}{4a}.
\end{aligned}
\end{equation}

Ponendo $X=x+\frac b{2a}$ e $Y=y+\frac{b^2-4ac}{4a}$
l'equazione $y=ax^2+bx+c$ diventa quindi $Y=aX^2$. 
Significa
che il grafico della funzione quadratica~\eqref{eq:funzione_quadratica}
si ottiene traslando la curva $y = a x^2$ che, 
dal punto di vista geometrico, si può facilmente
dimostrare essere una parabola con fuoco
nel punto di coordinate $\enclose{0,\frac 1 {4a}}$
e asse la retta di equazione $x=0$.
Dunque il grafico di ogni funzione quadratica è una parabola, 
e più precisamente: ogni parabola con direttrice parallela all'asse delle
ascisse è il grafico di una funzione quadratica.

\begin{definition}[parabola]
  \index{parabola}%
  \index{fuoco!parabola}%
  \index{direttrice!parabola}%
  La parabola $P_{\vec F,r}$ con \emph{fuoco} nel punto $\vec F=(F_1,F_2)\in \RR\times \RR$ 
  e retta \emph{direttrice}
  la retta $r \subset \RR\times\RR$ 
  è l'insieme dei punti  $\vec x = (x_1,x_2)\in \RR\times\RR$ 
  equidistanti da $\vec F$ e da $r$:
  \[
  P_{\vec F,r} = \ENCLOSE{\vec x\in \RR\times\RR\colon 
  \sqrt{(x_1-F_1)^2 + (x_2-F_2)^2} 
  = \inf_{\vec P\in r}\sqrt{(x_1-P_1)^2+(y_1-P_1)^2}
  }.
  \]
\end{definition}

\begin{exercise}
  Si dimostri che per ogni $a\neq 0$ esiste $s\neq 0$ 
  per cui il riscalamento $X=sx$, $Y=sy$ porta il grafico della 
  parabola $y=ax^2$ nel grafico della parabola $Y=X^2$.
  Significa che 
  c'è una unica parabola 
  a meno di isometrie e riscalamenti.
\end{exercise}

Ricordando le proprietà di monotonia della funzione $X\mapsto X^2$
possiamo dedurre che se $a>0$ la funzione $f(x)$ è strettamente
decrescente se ristretta all'intervallo 
$\left(-\infty,-\frac b {2a}\right]$ ed è invece strettamente crescente 
sull'intervallo $\left[-\frac b{2a},+\infty\right)$. 
Ha dunque un punto di minimo in $x=-\frac{b}{2a}$.
Inoltre (sempre se $a>0$) la funzione è superiormente illimitata.
Viceversa se $a<0$ la funzione è inferiormente illimitata ed ha 
un massimo nel punto $x=-\frac{b}{2a}$.

E' molto importante saper risolvere equazioni e disequazioni
quadratiche. Grazie a~\eqref{eq:24589} l'equazione
\[
 a x^2 + bx + c = 0
\]
risulta equivalente a
\[
  \enclose{x+\frac{b}{2a}}^2 = \frac{b^2-4ac}{4a^2}.
\]
Dunque se $b^2-4ac<0$ l'equazione $ax^2+bx+c=0$ non ha soluzioni.
Se $b^2-4ac=0$ l'equazione ha una unica soluzione $x=-\frac{b}{2a}$.
Infine se $b^2-4ac>0$ si ottiene
\[
  x+\frac b{2a} = \pm \frac{\sqrt{b^2-4ac}}{2a}
\]
da cui la famosa formula risolutiva
\mymark{***}
\begin{equation}\label{eq:secondo_grado}
  x_{1,2} = \frac{-b \pm \sqrt{b^2-4ac}}{2a}.
\end{equation}

Risolvendo le disequazioni allo stesso modo, si trova
che la funzione $ax^2+bx+c$, quando $a>0$ è positiva
nei punti esterni alle soluzioni dell'equazione
(in tutti i punti se le soluzioni non esistono) ed
è negativa nei punti interni alle due soluzioni.
Viceversa se $a<0$ la funzione è positiva all'interno
delle due soluzioni e negativa all'esterno.


\subsection{funzioni esponenziali}
%
%
\label{sec:esponenziale}%

Il teorema di isomorfismo ci ha permesso di definire la moltiplicazione sui 
numeri reali. 
Lo stesso identico metodo ci permetterà di definire la funzione esponenziale
ovvero la potenza $a^x$ con base $a>0$ fissata ed esponente variabile $x\in \RR$.
Abbiamo infatti già osservato che l'insieme $\RR_+$ dei reali positivi
risulta essere un gruppo moltiplicativo totalmente ordinato, denso e continuo 
(teorema~\ref{th:gruppo_moltiplicativo}).

\begin{theorem}[funzione esponenziale]
  \label{th:esponenziale}%
Dato $a\in \RR$, $a\ge 1$ per ogni $x\in \RR$ si può definire in modo unico 
la funzione esponenziale $x\mapsto a^x$ con le seguenti proprietà:
\begin{enumerate}
  \item $a^1=a$;
  \item $a^{x+y} = a^x \cdot a^y$;
  \item $a^x\ge 1$ se $x\ge 0$.
\end{enumerate}
Se $0<a \le 1$ si può definire in modo unico l'esponenziale $x\mapsto a^x$ 
con le stesse proprietà, salvo che per ogni $x\ge 0$ si ha $a^x\le 1$.

Inoltre l'esponenziale ha le seguenti proprietà, 
valide per $a,b>0$, $x,y\in \RR$:
\begin{enumerate}
  \item $a^0=1$;
  \item $a^{-1} = \frac{1}{a}$;
  \item $(a\cdot b)^x = a^x\cdot b^x$;
  \item $(a^x)^y = a^{x\cdot y}$;
  \item $a^x \le a^y$ se $a\ge 1$ e $x\le y$;
  \item $1^x=1$.
\end{enumerate}
\end{theorem}
%
\begin{proof}
Fissato $a>0$ possiamo applicare il teorema~\ref{th:isomorfismo} (isomorfismo)
con $R=\RR$ gruppo additivo e $S=\RR_+$ gruppo moltiplicativo.
Se $a\ge 1$ si ottiene l'esistenza di una, unica, funzione $\phi_a\colon \RR \to \RR_+$ 
che soddisfa le seguenti proprietà:
$\phi_a(1)=a$, $\phi_a(x+y) = \phi_a(x)\cdot \phi_a(y)$, 
$\phi_a(x)\ge 1$ se $x\ge 0$. 
Se $0<a\le 1$ otteniamo una unica funzione con le stesse proprietà 
ma $\phi_a(x) \le 1$ se $x\ge 0$.

Definiamo $a^x = \phi_a(x)$ e la chiamiamo \emph{funzione esponenziale}
con \emph{base} $a>0$ ed esponente $x\in \RR$.
  
L'omomorfismo manda sempre l'elemento neutro nell'elemento 
neutro dunque $a^0 = 1$.
\mynote{Ricordiamo che in partenza abbiamo il gruppo additivo $\RR$
con elemento neutro $0$ mentre in arrivo 
abbiamo il gruppo moltiplicativo $\RR_+$ con elemento neutro $1$.}

\mymargin{$a^{-1}=\frac 1 a$}
Per la proprietà di omomorfismo si ha $a^{x-x} = a^x \cdot a^{-x}$
da cui $a^{-x}$ risulta essere il reciproco di $a^x$ cioè 
$a^{-x}= 1/a^x$.

Per la potenza del prodotto fissati $a,b\ge 1$ basta considerare la 
funzione $f(x) = a^x\cdot b^x$. 
\mymargin{$(a\cdot b)^x = a^x\cdot b^x$}
Chiaramente $f$ è un omomorfismo in quanto $a^x$ e $a^y$ lo sono 
(e il prodotto è commutativo): 
$a^{x+y}\cdot b^{x+y} = a^x \cdot b^x\cdot a^yb^y$.
Inoltre se $x\ge 0$ si ha $a^x\ge1$ e $b^x\ge 1$ da cui $f_(x)\ge 1$.
Essendo $f(1) = a^1\cdot b^1 = a\cdot b$ 
per l'unicità dell'omomorfismo positivo concludiamo che $f = \phi_{a\cdot b}$
cioè $a^x\cdot b^x = (a\cdot b)^x$.
Usando la regola del reciproco possiamo estendere questa proprietà 
quando $a<1$ e/o $b<1$ (ma sempre $a,b\ge 0$).

\mymargin{$(a^x)^y = a^{x\cdot y}$}
Infine per la potenza di potenza fissato $a\ge 1$ e $x\ge 0$ 
consideriamo la funzione $f(y) = a^{x\cdot y}$.
Per le proprietà precedenti si verifica facilmente che  
$f(y+z) = f(x)\cdot f(z)$. 
Inoltre se $a\ge 1$, $x\ge 0$ e $y\ge 0$ si ha 
$f(y)\ge 1$ che è la positività. 
Dunque per il teorema di isomorfismo,
essendo $f(1)=a^x$, si ottiene $f(y) = (a^x)^y$
che è quanto volevamo dimostrare.
La regola del reciproco estende questa proprietà 
ai casi $x\le 0$ e $a\le 1$ (sempre con $a\ge 0$).

Se $a\ge 1$ e $x\le y$ allora $a^{y-x}\ge 1$.
\mymargin{monotonia}
Ma $a^{y-x} = \frac{a^y}{a^x}$ e dunque $a^y\ge a^x$.

Se $a=1$ per $x\ge 0$ si ha contemporaneamente $1^x\ge 1$ e $1^x\le 1$
dunque $1^x=1$. 
Lo stesso vale se $x\le 0$ in quanto $1^{-x}=\frac 1{1^x}$.
\mymargin{$1^x=1$}
\end{proof}

\begin{figure}
  \begin{center}
    \begin{tikzpicture}[x=1.0cm,y=1.0cm]
      \clip(-4,-2) rectangle (4,3);
    	\draw[->] (-4,0) -- (4,0);
      \draw[->] (0,-2) -- (0,3);
      % y = (3/5)*x + 1/5
      \draw[dotted] (1,0) -- (1,2) -- (0,2);
      \draw[dotted] (2,0) -- (2,1) -- (0,1);
%      \node[right] at (1.75,2.5){$x=-\frac b {2a}$};
      \draw[domain=-2.0:4.0,smooth,variable=\x,dashed] plot
        ({\x}, {pow(2,-\x)});
      \draw[domain=0.1:4.0,smooth,variable=\x,dashed] plot
        ({\x}, {-ln(\x) / ln(2)});
      \draw[domain=-4.0:2.0,smooth,variable=\x,thick] plot
        ({\x}, {pow(2,\x)});
      \node at (2,2.5) {$y=a^x$};
      \node at (-2,2.1) {$y=\enclose{\frac 1 a}^x$};
      \fill (0,1.0) node[left]{\!\!$1$} circle[radius=1.5pt];
      \draw[domain=0.1:4.0,smooth,variable=\x,thick] plot
        ({\x}, {ln(\x) / ln(2)});
      \node at (3.2,1.1) {$y=\log_a x$};
      \node at (3.2,-1.1) {$y=\log_{\frac 1 a} x$};
      \fill (1,0) node[below]{$1$} circle[radius=1.5pt];
      \fill (0,2) node[left]{$a$} circle[radius=1.5pt];
      \fill (2,0) node[below]{$a$} circle[radius=1.5pt];
      \fill (2,1) circle[radius=1.5pt];
      \fill (1,2) circle[radius=1.5pt];
    \end{tikzpicture}
  \end{center}
  \caption{Il grafico della funzione esponenziale e logaritmo 
  in base $a>1$ e $\frac 1 a < 1$ ($a=2$ in figura).}
  \label{fig:esponenziale_logaritmo}
\end{figure}




%
\subsection{logaritmo}
%
%
%

\label{sec:logaritmo}
\index{logaritmo}%

Il teorema di isomorfismo ci dice che fissato $a\in \RR_+$, $a\neq 1$ 
la funzione $\phi_a\colon \RR\to\RR_+$ definita nel paragrafo 
precedente ($\phi_a(x)=a^x$) è bigettiva.
La funzione inversa si chiama \emph{logaritmo in base $a$}
e si denota con $\log_a\colon \RR_+ \to \RR$.

Grazie alle proprietà della funzione esponenziale, già dimostrate
nel teorema~\ref{th:esponenziale} possiamo ottenere 
le proprietà del logaritmo.

\begin{theorem}[proprietà del logaritmo]
Per ogni $a>1$
esiste una unica funzione $\log_a\colon \RR_+\to \RR$ 
tale che 
\begin{enumerate}
  \item $\log_a(a) = 1$;
  \item $\log_a(x\cdot y) =\log_a x + \log_a y$;
  \item $\log_a x \ge 0$ se $x\ge 1$. 
\end{enumerate}
Se $0<a<1$ 
esiste una unica funzione $\log_a$ con le stesse proprietà 
salvo che $\log_a x \le 0$ se $x\ge 1$.

La funzione $\log_a$ viene chiamata \emph{logaritmo in base $a$}
ed ha inoltre le seguenti proprietà valide 
per ogni $a>0$, $a\neq 1$, $x,y\in \RR$, $b>0$, $c>0$, $c\neq 1$:
\begin{enumerate}
  \item $\log_a x = y$ se e solo se $a^y = x$;
  \item $\log_a 1 = 0$;
  \item $\log_a (b^x) = x\log_a b$;
  \item $\log_a x = \frac{\log_c x}{\log_c a}$.
\end{enumerate}
\end{theorem}

\subsection{potenze con esponente reale}

Nel capitolo~\ref{sec:potenza} abbiamo definito 
la funzione potenza $x^n$ con base $x\in \RR$ 
ed esponente $n\in \NN$. 
Invertendo la fuzione $x^n$ 
(solo per $x\ge 0$ quando $n$ è pari e positivo, per ogni $x\in \RR$ se 
$n$ è dispari) abbiamo definito la radice $n$-esima 
$\sqrt[n]{x}$.
Nel capitolo~\ref{sec:esponenziale} abbiamo 
definito la funzione esponenziale $a^x$ con 
base $a\in \RR_+$ ed esponente $x\in \RR$.
La funzione inversa dell'esponenziale 
è il logaritmo $\log_a x$ definito per $x>0$.

Quando $a>0$ 
e $b\in \NN$,
abbiamo formalmente due diverse definizioni 
dell'operazione \emph{potenza} $a^b$ ma, 
ovviamente, queste due definizioni 
coincidono in quanto il teorema di isomorfismo 
ci garantisce che la funzione esponenziale 
$a^x$ coincide con la moltiplicazione ripetuta 
quando $x$ è un numero naturale.
Grazie a questa osservazione possiamo osservare 
che se $x>0$, $p\in \NN$ e $q\in \NN$, $q\neq 0$ si ha 
\begin{equation}
\label{eq:9783023}
  x^{\frac p q} = \sqrt[q]{x^p}, 
  \qquad 
  x^{-\frac p q} = \frac{1}{\sqrt[q]{x^p}}
\end{equation}
in quanto 
\[
  \enclose{x^{\frac p q}}^q = x^p,
  \qquad
  x^{-y} = \frac{1}{x^y}.
\]
Se ripercorriamo la dimostrazione del teorema di isomorfismo 
usando in arrivo il gruppo additivo (è così che abbiamo 
definito la funzione esponenziale) ci accorgiamo 
che la funzione $a^x$ viene dapprima definita sugli $x$
naturali come moltiplicazione ripetuta, 
poi sugli $x$ razionali 
positivi tramite la radice $n$-esima e poi sugli 
$x$ negativi passando al reciproco.

Senz'altro potrà essere utile definire 
$a^{-n} = \frac 1 {a^n}$ avendo quindi una definizione 
di potenza $a^n$ valida per ogni $a$ se $n\in \NN$ 
e per ogni $a\neq 0$ se $n\in \ZZ$.
In alcuni testi si va oltre e si considera $a^b$ 
definito anche quando $a<0$ e $b$ è razionale
utilizzando il lato destro delle equazioni 
in~\eqref{eq:9783023}.
Ognuno può scegliere le definizioni che preferisce 
ed è bene ricordare che le definizioni si scelgono, 
non si dimostrano. 
E' però anche bene sapere che certe definizioni 
possono essere fuorvianti.

Ci sono buoni motivi per pensare che la funzione 
esponenziale $a^x$ ($a>0$) e la funzione potenza $x^n$ 
con $n\in \NN$ siano funzioni sostanzialmente diverse.
Si noti ad esempio che le regole delle potenze sono 
soddisfatte da entrambe le definizioni, separatamente, 
ma se mescoliamo le due definizioni le proprietà 
possono cadere. Ad esempio:
\[
  \enclose{(-2)^6}^{\frac 1 2} 
  \neq \enclose{-2}^{6\cdot \frac 1 2}.
\]

Anche il fatto che abbiamo definito $0^0=1$ andrebbe 
interpretato nel senso che lo $0$ all'esponente 
è un numero naturale, non un numero reale:
La funzione esponenziale $a^x$ è definita 
solo per $a>0$ mentre la funzione $x^n$ è definita anche 
per $x=0$ ma solo per $n\in \NN$.


% \begin{figure}
%   \begin{center}
%     \begin{tikzpicture}[x=1.0cm,y=1.0cm]
%       \clip(-3,-2) rectangle (4,3);
%     	\draw[->] (-3,0) -- (4,0);
%       \draw[->] (0,-2) -- (0,3);
%       \draw[dotted] (1,0) -- (1,1) -- (0,1);
%       % \draw[dotted] (2,0) -- (2,1) -- (0,1);
%       \draw[domain=0.5:4,samples=50,variable=\x,thick,dashed] plot
%         ({\x}, {1/(\x*\x))});
%       \draw[domain=-3:-0.5,samples=50,variable=\x,thick,dashed] plot
%         ({\x}, {1/(\x*\x))});
%       \draw[domain=0.2:4,samples=50,variable=\x,dashed,color=gray] plot
%         ({\x}, {1/(\x))});
%       \draw[domain=-3:-0.5,samples=50,variable=\x,dashed,color=gray] plot
%         ({\x}, {1/(\x))});
%       \draw[domain=-2.0:2.0,smooth,variable=\x,thick,color=gray] plot
%         ({\x}, {\x*\x*\x)});
%       \draw[domain=-3.1:4.0,samples=200,variable=\x,color=gray] plot
%         ({\x}, {pow(abs(\x),1/3)*\x / abs(\x)});
%       \draw[domain=-3.0:3.0,smooth,variable=\x,thick] plot
%         ({\x}, {\x*\x});
%       \draw[domain=0.0:4.0,samples=200,variable=\x] plot
%       ({\x}, {pow(\x,1/2)});
%       \node at (-2.0,1.8) {$y=x^2$};
%       \node at (3,2) {$y=\sqrt{x}$};
%       \node[color=gray] at (-1.7,-1.8) {$y=x^{3}$};
%       \node[color=gray] at (-2.0,-1) {$y=\sqrt[3]{x}$};
%       \node[color=gray] at (-1.7,-0.3) {$y=\frac 1 x$};
%       \node at (-2.5,0.6) {$y=\frac 1 {x^2}$};
%       \fill (1,0) node[below]{$1$} circle[radius=1.5pt];
%       \fill (0,1) node[left]{$1$} circle[radius=1.5pt];
%     \end{tikzpicture}
%   \end{center}
%   \caption{Grafici tipici di potenze e radici.}
%   \label{fig:potenza_radice}
% \end{figure}


\subsection{equazioni e disequazioni}

Un problema matematico molto comune è quello di dover risolvere 
equazioni e disequazioni del tipo:
\begin{equation}\label{eq:573197}
  f(x) = b, \quad f(x) \ge b, \quad f(x) > b, 
  \quad f(x) \le b, \quad f(x) < b
\end{equation}
dove $f\colon A \subset \RR \to\RR$ è una funzione data e 
$b\in \RR$ è fissato.

Quando $f$ è strettamente crescente e $b\in f(A)$ 
la soluzione può essere 
scritta banalmente: 
ovviamente deve essere $x\in A$
e ogni equazione o disequazione in~\eqref{eq:573197}
avrà la corrispondente soluzione:
\[
  x= f^{-1}(b), \quad x \ge f^{-1}(b), \quad x>f^{-1}(b),
  \quad x \le f^{-1}(b), \quad x < f^{-1}(b).
\]
Se la funzione fosse strettamente decrescente 
si può procedere allo stesso modo, ma le disuguaglianze si invertono.
Se la funzione fosse strettamente crescente su alcuni intervalli 
e strettamente decrescente su altri si potranno separare i diversi 
casi e si otterranno più soluzioni espresse da uguaglianze
o disuguaglianze.

\begin{example}
  Si risolva la disequazione 
  \[
   \log_2\Enclose{\sqrt[3]{(x+1)^4-3}-2} \le 3. 
  \]
\end{example}%
\begin{proof}[Svolgimento.]
La funzione logaritmo è strettamente crescente ed è definita 
quando l'argomento è positivo. 
Dunque la disequazione data 
è equivalente al sistema di disequazioni:
\[
0 < \sqrt[3]{(x+1)^4 - 3} - 2 \le 8.  
\]
Possiamo sommare $2$ per ottenere 
\[
  2 < \sqrt[3]{(x+1)^4 - 3} \le 10.  
\]
La funzione radice cubica è strettamente crescente 
su tutto $\RR$ quindi possiamo invertirla elevando 
tutto al cubo:
\[
 8 < (x+1)^4 - 3 \le 1000.
\]
Sommiamo $3$:
\[
11 < (x+1)^4 \le 1003.  
\]
L'elevamento alla quarta potenza è strettamente crescente 
solo quando l'argomento è positivo, ed è una funzione pari.
Possiamo quindi affermare che le nostre disequazioni sono 
equivalenti all'unione delle soluzioni di due sistemi:
\[
  \sqrt[4]{11} < x+1 \le \sqrt[4]{1003}
  \qquad\text{o}\qquad 
  -\sqrt[4]{1003} \le x+1 < -\sqrt[4]{11}.
\]
Sottraendo $1$ otteniamo infine 
\[
  \sqrt[4]{11} -1 < x \le \sqrt[4]{1003} - 1
  \qquad\text{o}\qquad 
  -\sqrt[4]{1003} -1 \le x < -\sqrt[4]{11} -1.
\]
In definitiva l'insieme delle soluzioni è 
\[
\left[-\sqrt[4]{1003} - 1, -\sqrt[4]{11}-1\right)
\cup \left(\sqrt[4]{11}-1 , \sqrt[4]{1003} -1\right].  
\]
\end{proof}

Nell'esempio precedente la funzione 
$f(x) = \log_2\Enclose{\sqrt[3]{(x+1)^4-3}-2}$
è ottenuta mediante composizione di funzioni elementari:
\begin{align*}
f &= (x\mapsto \log_2 x)\circ(x\mapsto x-2)\circ (x \mapsto \sqrt[3]{x})\\
  &\quad \circ (x \mapsto x-3) \circ (x\mapsto x^4) \circ (x\mapsto x+1).
\end{align*}
Negli intervalli in cui tutte queste funzioni sono invertibili 
la funzione inversa si ottiene componendo, in ordine opposto,
tutte le inverse:
\begin{align*}
  f^{-1} &= (x\mapsto x-1) \circ (x\mapsto \sqrt[4]{x}) \circ (x \mapsto x+3) \\
    &\quad \circ (x \mapsto x^3) \circ (x \mapsto x+2) \circ (x\mapsto 2^x).
\end{align*}
In effetti il caposaldo $\sqrt[4]{1003}-1$ è proprio tale 
funzione valutata in $b=3$.

Il metodo precedente è puramente algebrico e 
si applica alle equazioni 
come la~\eqref{eq:573197} dove la variabile $x$ 
compare una sola volta e dove la funzione $f$ si esprime 
come composizione di funzioni elementari di cui sappiamo 
scrivere la funzione inversa. 

Ben diverso è il caso in cui nell'equazione la variabile $x$ 
compare più di una volta.
In alcuni casi, come ad esempio,
\[
  x^2 > 2x - 1  
\]
queste equazioni 
possono essere ricondotte al caso precedente tramite 
opportune manipolazioni algebriche.
Il caso delle equazioni quadratiche lo abbiamo 
fatto nel paragrafo precedente utilizzato il completamento 
del quadrato: $x^2-2x = (x-1)^2-1$. 
In altri casi, come ad esempio l'equazione
\[
  2^x = x^2
\]  
le manipolazioni algebriche non sono utili.
Nel capitolo sul calcolo differenziale svilupperemo degli strumenti 
che ci permetteranno di determinare l'andamento di molte di queste 
di funzioni. 
Nel capitolo sulle successioni svilupperemo invece gli strumenti 
che ci permetteranno di determinare le soluzioni mediante 
algoritmi di approssimazione.
Questi strumenti presuppongono il concetto 
di limite e continuità: è sostanzialmente questo che identifica 
la materia chiamata analisi matematica.

\section{i numeri complessi}
%
%
%
\label{sec:complessi}

Dal punto di vista geometrico l'insieme $\CC$ dei \emph{numeri complessi}%
\mymargin{numeri complessi}%
\index{numeri!complessi}
\index{$\CC$}
può essere visto come una rappresentazione cartesiana 
del piano euclideo.
Sul piano fissiamo arbitrarimente un punto $0$ (l'origine) 
e fissiamo, arbitrariamente, una base $e_1$, $e_2$ di vettori ortonormali.
Identifichiamo ogni punto del piano con i corrispondenti vettori
applicati in $0$. La retta generata dal vettore $e_1$ la identifichiamo
con la retta $\RR$ dei numeri reali e quindi poniamo $1=e_1$.
La retta ortogonale generata dal vettore $e_2$ verrà chiamata
retta dei \emph{numeri immaginari} e definiamo $i=e_2$.

Un generico punto $z$ del piano $\CC$ potrà essere scritto in
maniera univoca nella base scelta: $z = x e_1 + y e_2$ ovvero,
per come abbiamo chiamato $e_1$ ed $e_2$:
\[
z = x + i y.
\]
Tale $z$ viene chiamato
\emph{numero complesso} con parte reale $x$ e parte immaginaria $y$.
Questa rappresentazione del numero complesso $z$ viene
chiamata \emph{rappresentazione cartesiana}%
\mymargin{rappresentazione cartesiana}%
\index{rappresentazione!cartesiana} in quanto definisce
il punto $z$ del piano complesso tramite le sue coordinate cartesiane
$x$ e $y$.
I numeri reali sono \emph{immersi} nei complessi, nel senso che se
$x\in \RR$ allora $z= x + i\cdot 0 = x$ è anche un numero complesso.
Il numero complesso $i = 0 + i\cdot 1$ viene chiamata \emph{unità immaginaria}%
\mymargin{unità immaginaria}%
\index{unità!immaginaria}
e i numeri complessi della forma $iy$ sono chiamati \emph{immaginari}.
\index{numeri!immaginari}
\index{immaginario}
Un numero
complesso $z = x+iy$ è quindi una somma tra un numero reale ed un numero
immaginario. Il numero reale $x$ viene chiamato \emph{parte reale}
\index{parte!reale}
di $z$ e
si denota con $x=\Re z$.
\mymargin{$\Re z$}%
\index{$\Re z$}
Il numero reale $y$ viene chiamato
\emph{parte immaginaria}
\index{parte!immaginaria}
di $z$ e si denota con $y=\Im z$
\mymargin{$\Im z$}%
\index{$\Im z$}
(osserviamo che la parte immaginaria di un numero complesso è un numero
reale, non immaginario). Dunque $z= \Re z + i \Im z$.

L'insieme $\CC$, per come
è stato costruito, è uno spazio vettoriale reale di dimensione $2$.
Abbiamo quindi già definite la \emph{addizione}%
\mymargin{addizione}%
\index{addizione}
\index{complessi!addizione}
tra elementi di $\CC$ e la moltiplicazione
tra elementi di $\CC$ ed elementi di $\RR$.
Se $a,b,c,d,t\in \RR$ si ha:
\begin{gather*}
 (a+ib) + (c+id) = (a+c) + i (b+d), \\
 t(a+ib) = ta + itb.
\end{gather*}

Vogliamo estendere la \emph{moltiplicazione}%
\mymargin{moltiplicazione}%
\index{moltiplicazione} a tutte le coppie di numeri complessi.
\index{complessi!moltiplicazione}
Imponendo (arbitrariamente) che valga $i\cdot i = -1$ e che rimanga
valida la proprietà distributiva, si ottiene
questa definizione:
\[
   (a+ib) \cdot (c+id) = (ac-bd) + i(ad+bc).
\]

Si può verificare che questa moltiplicazione estende quella ``scalare'' definita
in precedenza.
E' anche facile verificare che addizione e moltiplicazione soddisfano
le proprietà commutativa associativa e distributiva,
che $0$ è elemento neutro per la addizione, che $1$ è elemento neutro
della moltiplicazione.
Si osservi che se $z=x+iy$ non è nullo, allora
\[
  (x+iy) \cdot \frac{x-iy}{x^2+y^2} = 1.
\]
Significa che ogni $z\neq 0$ ammette inverso moltiplicativo e quindi $\CC$,
in definitiva,
risulta essere un campo.

Osserviamo che su $\CC$ non si definisce una relazione d'ordine perché
in effetti non è possibile definire un ordine ``compatibile'' con le operazioni
appena definite.%
\mynote{%
Se $\CC$ fosse un campo ordinato per assurdo
si dovrebbe avere,
che $z^2\ge 0$ per ogni $z\in \CC$ (questo è vero in tutti i campi ordinati). 
Ma
allora $-1 =i^2 \ge 0$ cioè $1\le 0$ che è in contraddizione
con la proprietà $0<1$ valida in ogni campo ordinato.
} % marginnote

Su $\CC$ definiamo delle ulteriori operazioni.
Il \emph{coniugato}%
\mymargin{coniugato}%
\index{coniugato}%
\index{complessi!coniugio}
di un numero complesso $z=x+iy$ è il numero
$\bar z = x - iy$. Geometricamente l'operazione di coniugio è una simmetria
rispetto alla retta reale. I numeri reali sono in effetti punti fissi del
coniugio (il coniugato di un numero reale è il numero stesso).
E' un semplice esercizio verificare che il coniugio ``attraversa''
somma e prodotto:
\[
\overline{z+w} = \bar z + \bar w, \qquad
\overline{z\cdot w} = \bar z \cdot \bar w.
\]
Ovviamente risulta $\overline {\bar z} = z$.
E' anche utile osservare che si ha:
\begin{equation}\label{eq:re_im}
  \Re z = \frac{z+\bar z}{2}, \qquad
  \Im z = \frac{z-\bar z}{2i}
\end{equation}
e
\[
z \cdot \bar z = (x+iy)(x-iy) = x^2-i^2y^2 = x^2+y^2.
\]

Possiamo allora definire il
\emph{modulo}%
\mymargin{modulo}%
\index{modulo}%
\index{complessi!modulo}
 di un numero complesso $z=x+iy$
come il numero reale
\[
\abs{z} = \sqrt{z\cdot\bar z} = \sqrt{x^2+y^2}.
\]
Geometricamente tale quantità rappresenta la distanza del punto $z$
dal punto $0$ e quindi la distanza tra due numeri complessi $z$ e
$w$ si potrà rappresentare con $\abs{z-w}$.

Osserviamo che se $z = x \in \RR \subset \CC$ il modulo di $z$ coincide
con il valore assoluto: $\abs{z} = \sqrt{x^2} = \abs{x}$ e per questo
motivo non distinguiamo, nelle notazioni, il modulo dal valore assoluto.
Più in generale risulta per ogni $z\in \CC$ (la verifica è immediata):
\[
  \abs{\Re z} \le \abs{z}, \qquad
  \abs{\Im z} \le \abs{z}.
\]

Possiamo a questo punto trovare una utile formula per calcolare
il reciproco di un numero complesso. Essendo infatti
$z\cdot \bar z = \abs{z}^2$ si osserva che
\[
  \frac{1}{z}
  = \frac{\bar z}{ \bar z \cdot z}
  = \frac{\bar z}{\abs{z}^2}.
\]

\begin{theorem}
Il modulo di un numero complesso soddisfa (come il valore assoluto)
le seguenti proprietà
\begin{enumerate}
\item $\big\lvert\abs{z}\big\rvert = \abs{z}$,
\item $\abs{-z} = \abs{z}$ = $\abs{\bar z}$,
\item $\abs{z\cdot w} = \abs{z}\cdot\abs{w}$.
\item $\abs{z+w} \le \abs{z}+\abs{w}$ (convessità),
\item $\abs{z-w} \le \abs{z-v} + \abs{v-w}$ (disuguaglianza triangolare),
\end{enumerate}
\end{theorem}
%
\begin{proof}
La prima proprietà è ovvia in quanto il valore assoluto di un numero reale
non negativo è il numero stesso.

La seconda proprietà viene immediatamente dalla definizione.

Per la terza proprietà sia $z=x+iy$, $w=a+ib$.
Allora:
\begin{align*}
\abs{z\cdot w}
&= \abs{(x+iy)\cdot(a+ib)}
=\abs{xa - y b+ i(xb + ay)} \\
&= \sqrt{(xa-yb)^2 + (xb+ay)^2}\\
&=\sqrt{x^2 a^2 + y^2b^2 - 2xayb + x^2b^2+a^2y^2+2xbay} \\
&=\sqrt{x^2 a^2 + y^2 b^2 + x^2 b^2 + a^2 y^2}\\
&=\sqrt{x^2(a^2+b^2) + y^2(a^2+b^2)}\\
&=\sqrt{(x^2+y^2)(a^2+b^2)}
=\abs{x+iy} \cdot \abs{a+ib}\\
&=\abs{z}\cdot\abs{w}.
\end{align*}

Per la quarta disuguaglianza osserviamo che si ha
\[
  \abs{z+w}^2 = (z+w)\cdot(\bar z + \bar w)
  = \abs{z}^2 + \abs{w}^2 + z\cdot \bar w + \bar z \cdot w
\]
e visto che
\[
  z\cdot \bar w + \bar z \cdot w
  = z \cdot \bar w + \overline{z \cdot \bar w}
  = 2 \Re(z\bar w)
  \le 2 \abs {z\bar w}
  = 2 \abs{z}\cdot\abs{\bar w}
  = 2 \abs{z}\cdot\abs{w}
\]
otteniamo
\[
 \abs{z+w}^2 \le \abs{z}^2+\abs{w}^2 + 2 \abs{z}\cdot\abs{w}
 =\enclose{\abs z + \abs w}^2
\]
che è equivalente alla disuguaglianza di convessità.

La disuguaglianza triangolare è conseguenza immediata della convessità, infatti
\[
  \abs{z-w} = \abs{(z-v) + (v-w)}
  \le \abs{z-v} + \abs{v-w}.
\]
\end{proof}

\begin{figure}
  \begin{center}
    \begin{tikzpicture}[x=0.5cm,y=0.5cm]
    	\draw[->] (-1,0) -- (9,0);
      \draw[->] (0,-1) -- (0,8);
      %
      \draw[dotted] (0,0) -- ({sqrt(45)},{sqrt(20)}) -- ({sqrt(45)},0);
      \draw[fill] ({sqrt(45)},{sqrt(20)}) node[right] {$w$} circle [radius=0.1];
      %
      \draw[thick] (0,0) -- (6,0);
      \draw[thick] (0,0) -- (6,3);
      \draw[thick] (0,0) -- (4,7);
      \draw[thick] (6,0) -- (6,3);
      \draw[thick] (6,3) -- (4,7);
      %
      \draw[dashed] (4,0) -- (4,7);
      \draw[dashed] (4,3) -- (6,3);
      %
      \draw (1,0) arc (0:{atan(1/2)}:1);
      \draw (6,3)+(-1,0) arc (180:{180+atan(1/2)}:1);
      \draw (4,7)+(0,-1) arc (-90:{-90+atan(1/2)}:1);
      %
      \draw[fill] (6,3) node[right] {$z$} circle [radius=0.1];
      \draw[fill] (4,7) node[above] {$u$} circle [radius=0.1];
      %
      \node[below] at (3,0) {$a$};
      \node[left] at (6,1.5) {$b$};
      \node[above] at (3,1.5) {$\alpha$};
      \node[right] at (5,5) {$\beta$};
      \node[above] at (5,3) {$s$};
      \node[left] at (4,4.5) {$t$};
      \node[above] at (8.5,0) {$x$};
      \node[left] at (0,7.5) {$y$};
      \node[below left] at (0,0) {$0$};
    \end{tikzpicture}
  \end{center}
  \caption{Consideriamo i numeri complessi $z=a+ib$ e $w=\alpha+i\beta$
  e supponiamo che sia $\alpha^2 = a^2+b^2$.
  Si consideri il punto $u$ che si ottiene ruotando il punto $w$ dell'angolo
  individuato dal punto $z$. Si avrà allora $u=a-s + i(b+t)$ dove $s$ e $t$
  sono i cateti del triangolo rettangolo con ipotenusa $\beta$.
  Grazie alle proprietà di similitudine dei triangoli si ha
  $\frac{s}{\beta} = \frac{b}{\alpha}$ e $\frac{t}{\beta} = \frac{a}{\alpha}$
  da cui si ottiene quindi $u = a-\frac{b\beta}{\alpha}+i(b+\frac{a\beta}{\alpha})$
  ovvero $\alpha u = (a\alpha - b \beta) + i (b\alpha + a \beta) = z\cdot w$.
  Significa che il numero complesso $z\cdot w$ si trova sulla semiretta
  che individua un angolo che è la somma degli angoli individuati
  dai numeri complessi $z$ e $w$.
  }
  \label{fig:prodotto_complesso}
\end{figure}

Possiamo ora dare una interpretazione geometrica del prodotto $z\cdot w$
tra due numeri complessi. In primo luogo sappiamo che $\abs{z\cdot w} = \abs{z} \cdot \abs{w}$ e dunque il punto del piano che rappresenta il prodotto $z\cdot w$ si trova ad una distanza dall'origine che è pari al prodotto delle distanze
dei punti $z$ e $w$. Inoltre l'angolo individuato da $z\cdot w$ rispetto
all'asse delle $x$ positive risulta uguale alla somma
degli angoli individuati dai punti $z$ e $w$ come mostrato in figura~\ref{fig:prodotto_complesso}.

Anche il piano dei numeri complessi può essere esteso aggiungendoci
un punto all'\emph{infinito}%
\mymargin{infinito}%
\index{infinito}.
A differenza dei reali, su cui era presente un ordinamento che era utile conservare,
nel caso dei numeri complessi è più usuale utilizzare un unico punto infinito
che si denota con \emph{$\infty$}%
\mymargin{$\infty$}%
\index{$\infty$}.
Definiamo l'insieme dei complessi estesi $\bar \CC$ come
\[
\bar \CC = \CC \cup \ENCLOSE{\infty}.
\]
Definiamo
\begin{align*}
  z + \infty &= \infty \qquad \forall z \in \CC\\
  z - \infty &= \infty \qquad \forall z \in \CC\\
   z\cdot \infty &= \infty \qquad \forall z \in \bar\CC\setminus\ENCLOSE{0} \\
   z / \infty &= 0 \qquad \forall z \in \CC \\
   z / 0 &= \infty \qquad \forall z \in \bar \CC \setminus\ENCLOSE{0}\\
   \bar \infty &= \infty \\
   \abs{\infty} &= +\infty \in \bar \RR.
\end{align*}
Si noti che abbiamo definito la divisione per zero di numeri complessi
(e quindi anche reali) diversi da zero. Il risultato è $\infty$ e quindi
rimane confermato che la divisione per zero non è una operazione valida
se vogliamo un risultato finito.
Una quantità $z\in \bar \CC$ sarà detta \emph{finita} se $z\in \CC$.

\begin{example}
  La funzione $f\colon \bar \CC \to \bar \CC$ 
  definita da 
  \[
  f(z) = \frac{1}{z}
  \]
  è una funzione bigettiva di $\bar \CC$ in sé.
  Dal punto di vista geometrico il coniugato di tale funzione
  ovvero la funzione $z\mapsto \frac 1 {\bar z}$ 
  è l'inversione circolare rispetto al cerchio unitario di $\CC$:
  i punti sulla circonferenza unitaria vengono lasciati fissi,
  i punti all'interno vengono mandati all'esterno rimanendo sullo 
  stesso raggio uscente dall'origine e invertendo il proprio modulo.
  I punti $0$ e $\infty$ si scambiano.
\end{example}

%% % ancora non abbiamo definito il concetto di funzione continua
%% Per le funzioni di variabile complessa e/o a valori complessi 
%% si applica la stessa definizione~\ref{def:continua} di continuità 
%% che abbiamo dato per le funzioni reali utilizzando il modulo 
%% complesso al posto del valore assoluto.
%% \index{continuità!campo complesso}%
%% \index{funzione!continua!complessa}%

\begin{exercise}[vertici di un triangolo equilatero]
Si risolva l'equazione 
\[ 
 z^3 - 1 = 0
\]
nel campo complesso.
\end{exercise}
%
\begin{proof}[Svolgimento.]
Ricordiamo il prodotto notevole:
\[
  z^3 - 1 = (z-1)(z^2+z+1).
\]
Dunque $z=1$ è una soluzione e le altre soluzioni devono risolvere 
l'equazione $z^2+z+1=0$. 
Certamente $z=0$ non è soluzione e dunque possiamo dividere per $z$ 
e ottenere:
\[
  z + 1 + \frac 1 z = 0.
\]
Osserviamo ora che se $z$ è soluzione si ha $1=z^3$
e quindi: $1 = \abs{z^3}= \abs{z}^3$ da cui $\abs z = 1$.
Ma allora $z\cdot \bar z = \abs{z}^2 = 1$ ovvero $\frac 1 z = \bar z$.
Dunque si ha 
\[
    z + 1 + \bar z = 0.
\]
Ma se $z=x+iy$ con $x,y\in \RR$ allora $z+\bar z = 2x$
e quindi 
\[
    2x + 1 = 0 
\]
da cui $x=-\frac 1 2$. Essendo inoltre $x^2+y^2=\abs{z}^2=1$ si ottiene
$y^2 = 1-x^2 = \frac 3 4$ da cui $y=\pm \frac{\sqrt 3}{2}$.

L'equazione data ha quindi $3$ soluzioni:
\[
z_0 = 1, \qquad 
z_{1,2} = -\frac 1 2 \pm i \frac{\sqrt 3} 2.
\]

Questi tre punti, se disegnati sul piano di Gauss, si trovano 
ai vertici di un triangolo equilatero iscritto nella circonferenza unitaria.
Infatti l'interpretazione geometrica del prodotto di numeri complessi ci dice 
che il punto $z_1$ individua sul piano di Gauss un angolo pari 
ad un terzo dell'angolo giro. Inoltre si ha $z_2 = z_1^2$ e dunque 
$z_2$ corrisponde a $\frac 2 3$ di angolo giro e $z_0=z_1^3 = 1$ rappresenta 
l'angolo giro (o l'angolo nullo).
\end{proof}

\begin{exercise}[pentagono regolare]
Si determini la massima distanza tra due soluzioni 
dell'equazione $z^5=1$. 
\end{exercise}

%%%%%%%%%%%%%%%%%%%%%%%%%%%%%%%%%%%%%%%%%%
%%%%%%%%%%%%%%%%%%%%%%%%%%%%%%%%%%%%%%%%%%
%%%%%%%%%%%%%%%%%%%%%%%%%%%%%%%%%%%%%%%%%%

\section{funzioni trigonometriche}
\label{sec:funzioni_trigonometriche}%
\label{sec:avvolgimento}%

Consideriamo la circonferenza unitaria 
\[
  U = \ENCLOSE{z\in \CC\colon \abs{z} = 1}.
\]
Ogni punto $z\in U$ individua un angolo geometrico con l'asse reale.
Il nostro obiettivo è ora quello di definire la misura di un angolo.
Il punto $z=1$ individuerà un angolo di misura nulla e 
ruotando $z$ in senso antiorario vogliamo ottenere angoli 
di misura crescente in modo che l'angolo che si ottiene 
giustapponendo uno di seguito all'altro gli angoli individuati 
dal punto $z$ e dal punto $w$ abbia misura pari alla somma delle misure 
degli angoli individuati da $z$ e da $w$.

Scegliamo arbitrariamente di dare una misura $\tau>0$ all'angolo giro.
L'idea intuitiva è quella di \emph{arrotolare} la retta $\RR$ come un filo 
attorno alla circonferenza $U\subset \CC$ 
mandando il punto $0\in \RR$ sul punto $1\in \CC$,
e poi il punto $\frac \tau 4$ su $i$, il punto $\frac \tau 2$ su $-1$,
il punto $3\frac \tau 4$ su $-i$ e il punto $\tau$ di nuovo su $1$.
Punti intermedi andranno su punti intermedi di $U$ in modo da avere 
l'additività degli angoli: la misura della somma di due angoli dovrà 
essere la somma delle misure.

L'insieme $U$ eredita la struttura di gruppo moltiplicativo di $\CC$: 
se $z,w\in U$ 
allora $z\cdot w \in U$ in quanto $\abs{z\cdot w} = \abs{z}\cdot \abs{w} = 1$
essendo $\abs{z}=\abs{w}=1$.
Ricordiamo inoltre che la moltiplicazione di numeri complessi unitari 
corrisponde alla somma dei loro angoli geometrici.
Dunque la funzione $\phi\colon \RR \to U$ che misura gli angoli, 
dovrà avere la proprietà di \emph{omomorfismo}:
\begin{equation}\label{eq:4012387}
  \phi(x+y) = \phi(x)\cdot \phi(y).
\end{equation}
Purtroppo sul gruppo $U$ non possiamo mettere un ordinamento e 
dunque non possiamo utilizzare il teorema~\ref{th:isomorfismo} 
di isomorfismo 
come abbiamo fatto per definire la funzione esponenziale.
L'idea sarà invece quella di definire la funzione $\phi$ 
bisezionando l'angolo giro, quindi estendendola ai multipli 
degli angoli bisecati con \eqref{eq:4012387}
e, infine, estendendola per monotonia a tutti gli angoli.

Per fissare le idee scegliamo $\tau=4$ cioè decidiamo di voler 
dare misura $1$ all'angolo retto e quindi misura $4$ all'angolo 
giro.
Vogliamo innanzitutto definire $\phi\colon[0,1]\to U$ la funzione
che associa alla misura $t\in[0,1]$ un angolo rappresentato da 
un numero complesso unitario $z\in U$ con $\Re z\ge 0$ e $\Im z\ge 0$
(nel primo quadrante del piano complesso).
Vorremo definire $\phi$ in modo che sia $\phi(0)=1$, $\phi(1)=i$,
$\phi(1/2)=\frac{1+i}{\sqrt 2}$ (angolo di $45^\circ$) e così via.

Per fare questo dobbiamo innanzitutto bisecare un generico angolo 
$u\in U$.
Se $u\in U$ l'angolo metà è rappresentato da un numero 
complesso $z\in U$ tale che $z^2=u$. 
L'equazione $z^2=u$ ha in generale due soluzioni, ma se $u$ 
è nel primo quadrante $Q=\ENCLOSE{z\in U\colon \Re z\ge 0, \Im z\ge 0}$
ci sarà una sola soluzione nel primo quadrante.
Posto $u=a+ib$ e $z=x+iy$ l'equazione $z^2=u$ si scrive:
\[
  (x^2-y^2) + i(2xy) = a+ib
\]
da cui $a=x^2-y^2$. Visto che $x^2+y^2=1$ (essendo $u\in U$)
si ha $a=2x^2-1$ da cui $x=\pm\sqrt{\frac{1+a}{2}}$.
La soluzione in $Q$ dovrà avere $x\ge 0$ e dunque segliamo 
il segno positivo:
\[
  x = \sqrt{\frac{1+a}{2}}, \qquad 
  y = \sqrt{1-x^2} = \sqrt{\frac{1-a}{2}}.
\]
Denotiamo con $z=\sqrt u$ la soluzione con parte reale positiva 
dell'equazione $z^2=u$ che abbiamo appena calcolato.
Possiamo allora definire la successione degli angoli dimezzati:
\[
\begin{cases}
u_0 = i\\
u_{n+1} = \sqrt{u_n}.
\end{cases}
\]
Questi angoli sono uno la metà dell'altro $u_{n+1}^2 = u_n$.
Induttivamente si dimostra facilmente che vale $u_{n+k}^{2^k} = u_n$
per ogni $n,k\in \NN$.
Avendo posto $\tau=4$ dovrà essere $\phi(1)=i$ e quindi 
$\phi(1/2^n)=u_n$ affinchè sia rispettata l'equazione \eqref{eq:4012387}.
Inoltre dovrà essere
\begin{equation}
\phi\enclose{\frac{p}{2^n}} = (u_n)^p,
\qquad n\in \NN, p\in \NN, p\le 2^n.
\end{equation}
Questa definizione è ben posta in quanto se $\frac{p}{2^n}=\frac{q}{2^{n+k}}$
si ha $p=2^k q$ e quindi $(u_n)^p = (u_n)^{2^k q} = u_{n+k}^q$.

Posto 
\[
B = \ENCLOSE{\frac{p}{2^n}\colon n\in \NN, p\in \NN, p\le 2^n} \subset [0,1]
\]
abbiamo definito $\phi\colon B\to U$ in modo che \eqref{eq:4012387}
sia soddisfatta. 
Inoltre la definizione di $\phi$ è unica se imponiamo  
$\phi(1)=i$.
Si tratta ora di estendere la definizione di $\phi$ a tutto 
l'intervallo $[0,1]$. 
Possiamo riutilizzare il teorema~\ref{th:estensione_monotona} di 
estensione monotona.

Innanzitutto vogliamo dimostrare che $\Re \phi\colon B\to \RR$ 
e $\Im \phi\colon B\to \RR$ sono funzioni monotone.
In particolare vogliamo dimostrare che se $t\ge s$ allora 
$\Re \phi(t) \le \Re \phi(s)$.
Osserviamo che si ha $\phi(t) = \phi(s)\cdot \phi(t-s)$, 
quindi basta capire come si comporta la moltiplicazione per un 
numero complesso del primo quadrante.
Se $\phi(s)=z=a+ib$ e $\phi(t-s)=w=x+iy$ si ha
$\Re (zw) = ax-by$. Banalmente se $a,b,x,y\in[0,1]$ si ha 
$ax-by \le a$. 
Dunque $\Re \phi(t)\le \Re \phi(s)$ come volevamo dimostrare.
Visto che $\Im \phi(t) = \sqrt{1-\Re \phi(t)^2}$ si ha
anche $\Im \phi(t) \le \Im \phi(s)$.

Questa proprietà ci dice in particolare che $\phi$ assume valori 
nel primo quadrante, in quanto sapendo che agli estremi 
$\phi(0) = 1$ 
e $\phi(1)=i$ ogni altro valore deve avere parte reale 
e parte immaginaria compresa tra $0$ e $1$.

Possiamo allora considerare la funzione 
$f=\Im \phi$, $f\colon B\subset [0,1]\to[0,1]$
a cui applicare il teorema~\ref{th:estensione_monotona}.
Dobbiamo verificare che $f(B)$ è denso in $J=[0,1]$. 
Dati $y_1,y_2\in [0,1]$ con $y_1<y_2$ dobbiamo mosrare che esiste $t\in B$ tale 
che $y_1 < \Im phi(t) < y_2$.
Sarà $y_1=\Im z_1$ e $y_2=\Im z_2$ con $z_1,z_2\in U$.
L'idea è quella di trovare $n$ abbastanza grande in modo che uno dei punti 
$u_n^p$ stia tra $z_1$ e $z_2$. 
L'angolo differenza tra $z_1$ e $z_2$ è rappresentato da 
$w=\frac{z_2}{z_1}$. 
Siccome $u_n\to 1$ per $n\to +\infty$ 
esisterà $n$ tale che $\abs{u_n-1}<y_2-y_1$.
La successione $\Im u_n^p$, al crescere di $p\in \NN$ è crescente 
(come abbiamo già osservato) e varia tra $0$ (quando $p=0$) e $1$ 
(quando $p=2^n$). 
Si ha
\[
 \Im u_n^{p+1} - \Im u_n^p = \Im (u_n^{p+1}-u_n^p) 
 \le \abs{u_n^{p+1}-u_n^p} < y_2-y_1.
\]
Dunque dovrà esistere $p\in \NN$, $p\le 2^n$ tale che 
$y_1 < \Im u_n^p <y_2$. 
Visto che $\Im u_n^p = f(p/2^n)$ abbiamo dimostrato 
che $f(B)$ è denso in $[0,1]$.
Per il teorema~\ref{th:estensione_monotona} possiamo allora estendere 
$f$ a tutto $[0,1]$ preservando la monotonia.
Ma allora possiamo estendere $\phi$ a tutto $[0,1]$ 
ponendo $\phi(t) = \sqrt{1-f^2(t)} + i f(t)$. 
In questo modo $\Im \phi = f$ è monotona crescente mentre 
$\Re \phi=\sqrt{1-f^2}$ risulta decrescente.

Vogliamo ora verificare che la proprietà di omomorfismo 
\eqref{eq:4012387} è verificata
ora per ogni $t\in [0,1]$.
Prendiamo dunque $s,t\in[0,1]$ tali che anche $s+t\in [0,1]$
e puntiamo a dimostrare che per ogni $\eps>0$ 
si ha:
\begin{equation}\label{eq:4438844}
  \Im(\phi(s+t)- \phi(s)\phi(t)) < 4\eps,
  \qquad
  \Re(\phi(s+t)-\phi(s)\phi(t)) < 4\eps.
\end{equation}
Questo direbbe che $\phi(s+t)=\phi(s)\phi(t)$.

Per dimostrare~\eqref{eq:4438844} consideriamo 
$s_1,s_2,t_1,t_2\in B$ tali che $s_1<s<s_2$, 
$t_1<t<t_2$
e $0<\Im \phi(s_2)-\Im \phi(s_1)<\eps$, $0<\Im \phi(t_2)-\Im \phi(t_1)<\eps$,
  $0<\Re \phi(s_1)-\Re \phi(s_2)<\eps$, $0<\Re \phi(t_1)-\Re \phi(t_2)<\eps$.
Questo si può fare perché abbiamo visto che $\Im  \phi$ è crescente 
e $\Im \phi(B)$ è denso in $[0,1]$.
Analogamente si verifica che $\Re \phi$ è decrescente e
e anche $\Re \phi(B)$ è denso in $[0,1]$.
Si ha allora:
\[
\Im \phi(s+t) - \Im (\phi(s)\phi(t))
= \Im \phi(s+t) - \Re \phi(s)\Im \phi(t) - \Re \phi(t)\Im \phi(s)
\]
e usando la monotonia di $\Im \phi$ e $\Re \phi$ si ottiene:
\begin{align*}
\Im \phi(s+t) &- \Im (\phi(s)\phi(t))
\le \Im \phi(s_2+t_2) - \Re \phi(s_2)\Im \phi(t_1) - \Re \phi(t_2)\Im \phi(s_1)\\ 
&= \Im (\phi(s_2)\phi(t_2)) - \Re \phi(s_2)\Im \phi(t_1) - \Re \phi(t_2)\Im \phi(s_1)\\ 
&= \Re\phi(s_2)(\Im\phi(t_2) - \Im \phi(t_1))+\Re\phi(t_2)(\Im\phi(s_2) - \Im \phi(s_1))\\
&< 2\eps.
\end{align*}
Analogamente si ottiene:
\begin{align*}
\Im \phi(s+t) - \Im (\phi(s)\phi(t))
&\ge -2\eps,\\
\Re \phi(s+t) - \Re (\phi(s)\phi(t))
&\le 4\eps,\\
\Re \phi(s+t) - \Re (\phi(s)\phi(t))
&\ge -4\eps.
\end{align*}
Nelle disuguaglianze con la parte reale si ottengono 
termini del tipo $a_1 b_1 - a_2 b_2$ 
ai quali bisogna aggiungere e togliere $a_1 b_2$ 
per ottenere la forma $a_1(b_1-b_2) + (a_1-a_2)b_2$: 
per questo motivo si ottiene $4\eps$ invece che $2\eps$.

**** WIP ****

Si tratta ora di estendere $\phi$ a tutto $\RR$.


\begin{theorem}[definizione funzioni trigonometriche]%
\label{def:sin_cos}%
\label{th:proprieta_trigonometriche}%
Per ogni $\tau>0$ esistono due uniche funzioni $\sin,\cos\colon \RR\to \RR$ 
tali che per ogni $x,y\in \RR$ valgono le seguenti proprietà:
\mynote{Quando avremo definito $\pi$ otterremo 
le usuali funzioni trigonometriche scegliendo $\tau=2\pi$. 
Ma altre scelte di $\tau$ non sono inusuali. 
Ad esempio con $\tau=360$ otteniamo le funzioni trigonometriche 
con gli angoli misurati in gradi. 
Sulle calcolatrici scientifiche 
c'è usualmente l'opzione \texttt{RAD} (radiante) che pone $\tau=2\pi$ e l'opzione 
\texttt{DEG} (\emph{degree} ovvero grado) che pone $\tau=360$.
Si può anche trovare l'opzione \texttt{GRAD} (\emph{gradian} o grado centesimale)
che corrisponde alla scelta $\tau=400$ utilizzata a volte in topografia.}%
\begin{enumerate}
  \item $\sin$ e $\cos$ sono $\tau$-periodiche;
  \item $\sin^2 x + \cos^2 x =1$;
  \item $\sin(x+y) = \sin x \cdot \cos y + \cos x\cdot \sin y$;
  \item $\cos(x+y) = \cos x \cdot \cos y - \sin x\cdot \sin y$;
  \item $\sin(-x) = -\sin(x)$ ($\sin$ è dispari), $\cos(-x) = \cos(x)$ ($\cos$ è pari);
  \item $\sin \colon \closeinterval{-\frac \tau 4}{\frac \tau 4} \to [-1,1]$
  è strettamente crescente e bigettiva;
  \item $\cos \colon \closeinterval{0}{\frac \tau 2} \to [-1,1]$.
  è strettamente decrescente e bigettiva.
\end{enumerate}
\end{theorem}
%
\begin{proof}
Consideriamo la funzione $\phi\colon \RR \to U\subset \CC$ 
definita nel teorema precedente (\ref{th:omomorfismo_U}) 
e definiamo $\cos x = \Re \phi(x)$, $\sin x = \Im \phi(x)$
cosicché $\phi(x) = \cos(x) + i \sin (x)$.
Chiaramente $\cos$ e $\sin$ hanno periodo $\tau$ in quanto 
$\phi$ ha periodo $\tau$.

Visto che $\phi(x)\in U$ si ha $\abs{\phi(x)}^2=1$.
Ma allora vale il punto 2:
\[
 1 = \abs{\phi(x)}^2 
 = \abs{\cos x + i\sin x}^2 
 = \cos^2 x + \sin^2 x.
\]

La proprietà di omomorfismo 
$\phi(x+y)=\phi(x)\cdot \phi(y)$ diventa 
\begin{align*}
  \cos(x+y) + i \sin(x+y)
  &=(\cos x + i \sin x)\cdot(\cos y + i \sin y) \\
  &= 
  \Enclose{\cos x\cdot \cos y - \sin x \sin y}
  + i\Enclose{\sin x\cdot \cos y + \cos x\cdot \sin y}
\end{align*}
da cui, eguagliando parte reale e parte immaginaria, 
si ottengono le formule di addizione dei punti 3 e 4.

Visto che $\abs{\phi(x)}=1$ sappiamo che 
$1/\phi(x) = \overline{\phi(x)}$. 
Ma, per le proprietà di omomorfismo, $1/\phi(x)=\phi(-x)$ e quindi 
\[
\cos (-x) + i \sin(-x) 
= \phi(-x) 
= \overline{\phi(x)}
= \cos(x) - i \sin(x)
\]
da cui, uguagliando parte reale e parte immaginaria, 
si ottiene il punto 5.

Visto che $\Im \phi\colon[-\tau/4,\tau/4]\to[0,1]$ è strettamente crescente e bigettiva
deduciamo che anche $\sin\colon[-\tau/4,\tau/4]\to [0,1]$ lo è.
Ricordando che $\phi(\tau/4)=i$, 
si ha $\cos(\tau/4)=0$, $\sin(\tau/4)=1$ e grazie alle formule 
di addizione troviamo $\sin(\tau/4-x)=\cos(x)$.
Dunque se $\sin$ è strettamente crescente sull'intervallo 
$[-\tau/4,\tau/4]$ scopriamo che $\cos$ è strettamente 
decrescente sull'intervallo $[0,\tau/2]$
e su tale intervallo assume 
tutti i valori compresi tra $[-1,1]$.
\end{proof}

\begin{figure}
  \centering%
  \begin{tikzpicture}[scale=0.75]
  \draw[->] (-2,0) -- (10,0) node[right] {$x$};
  \draw[->] (0,-3) -- (0,3) node[above] {$y$};
  \foreach \x/\xtext in
    {{pi/2}/{\frac \tau 4}, {pi}/{\frac \tau 2},
     {2*pi}/{\tau}, {3*pi/2}/{\frac {3\tau} 4}, {-pi/2}/{-\frac {\tau}{4}}} {
    \draw[shift={(\x,0)},lightgray] (0,-3) -- (0,3);
    \draw[shift={(\x,0)}] (0pt,2pt) -- (0pt,-2pt) node[below] {$\xtext$};
  }
  \foreach \y in {1, -1} {
    \draw[shift={(0,\y)},lightgray] (-1,0) -- (11,0);
  }
  \draw (0,1) node [above right] {$1$};
  \draw (0,-1) node [right] {$-1$};
  \draw[domain=-2:10,smooth,variable=\x,blue,thick] plot ({\x},{cos(deg(\x))});
  \draw[domain=-2:10,smooth,variable=\x,red,thick] plot ({\x},{sin(deg(\x))});
  \draw (3.7,-1) node[blue,below] {$y=\cos x$};
  \draw (2,0.9)  node[red,above] {$y=\sin x$};
  \end{tikzpicture}
  \caption{%
  I grafici delle funzioni $\sin$, $\cos$ di 
  un generico periodo $\tau$.}
\end{figure}


\section{funzioni continue}
\label{sec:continuita}

Intuitivamente una funzione continua ha la proprietà
che se in un punto $x_0$ assume un valore $y_0=f(x_0)$ allora
in punti abbastanza vicini ad $x_0$ i valori assunti
non saranno molto diversi dal valore $y_0$.
Nella definizione seguente questo viene formalizzato:
fissato il punto $x_0$ e scelto un errore $\eps>0$
che siamo disposti a commettere sui valori della funzione
possiamo trovare un errore $\delta>0$ per cui nei punti
che differiscono da $x_0$ per meno di $\delta$ il valore
della funzione differisce da $y_0$ per meno di $\eps$.

Se $x$ e $y$ sono due punti di $\RR$ la loro distanza 
è data da $\abs{x-y}$. 
Una disuguaglianza del tipo $\abs{x-y}<r$ significa 
che il punto $x$ si trova nell'intervallo $(x-r,x+r)$.
Nel seguito ci concentreremo sulle funzioni con dominio 
e codominio nei numeri reali. 
Ma potremo osservare che molte delle definizioni e molti 
dei teoremi possono essere enunciati e dimostrati in maniera 
identica anche per le funzioni con dominio e codominio 
nei numeri complessi%
\footnote{Ci sono in realtà quattro casi rilevanti:
dominio e codominio reale, dominio e codominio complesso, 
dominio reale e codominio complesso, 
dominio complesso e codominio reale, cominio reale 
e codominio complesso. 
Per dire che il dominio è sottoinsieme dei numeri complessi 
diremo che la funzione $f$ è \emph{di variabile complessa},
mentre per dire che la funzione $f$ ha codominio nei numeri 
complessi diremo che $f$ è \emph{a valori complessi}
o più semplicemente che $f$ è complessa.
Il caso di funzioni complesse di variabile complessa 
è ovviamente il caso più generale. 
Ma a volte può essere utile restringersi alla variabile 
reale dove abbiamo la struttura lineare data dall'ordinamento.
Sempre per mantenere l'ordinamento il caso reale e complesso 
si distinguono anche per i punti all'infinito: nel caso reale 
ne introduciamo due, $+\infty$ e $-\infty$ mentre nel caso 
complesso si introduce un unico punto $\infty$.
} 
Ad una prima lettura si consiglia di trascurare 
il caso complesso.
Anche quando $x$ e $y$ sono punti di $\CC$ il modulo della 
differenza $\abs{x-y}$ rappresenta la distanza tra i due 
punti $x$ e $y$. 
Fissato $y\in \CC$ e $r>0$
\mynote{La condizione $r>0$ non ha senso se $r\in \CC$ 
quindi se si impone questa condizione è sottointeso 
che $r\in \RR$}
la condizione $\abs{x-y}<r$ identifica i punti $x$ 
che si trovano all'interno di un cerchio di raggio $r$
intorno a $y$.

\begin{definition}[funzione continua]
  \label{def:continua}%
  \index{funzione!continua}%
  \index{continuità}%
  Sia $f\colon A \subset \RR \to \RR$ 
  (oppure $f\colon A\subset \CC \to \CC$) una funzione. 
  Diremo che $f$ è \emph{continua nel punto}%
\mymargin{continua nel punto}%
\index{continuo!nel punto} $x_0\in A$ se
  \begin{equation}\label{eq:continuita}
  \forall \eps>0 \colon \exists \delta>0 \colon
  \forall x\in A\colon
  \abs{x-x_0} < \delta \implies \abs{f(x)-f(x_0)} < \eps.
  \end{equation}
  
  Diremo che $f$ è \emph{continua}%
\mymargin{continua}%
\index{continuo} se è continua in ogni punto $x_0 \in A$.
  \end{definition}
  
  %%%%%%%%%%%%%%%%%%%
  %%%%%%%%%%%%%%%%%%%
  %%%%%%%%%%%%%%%%%%%
  
  Attenzione: la funzione $f(x)=\frac{1}{x}$
  assume vicino a $x=0$ valori molto diversi tra loro (ad esempio $f(0.01)-f(-0.01)=200$)
  ma ciò non toglie che la funzione possa essere continua in quanto il punto
  $x=0$ non appartiene al dominio e quindi,
  in base alla definizione precedente, non ha senso e non ha importanza
  verificare se la funzione è continua in tale punto.
  
  \begin{theorem}[continuità del reciproco]
  \label{th:cont_reciproco}%
  La funzione $f\colon \RR\setminus\ENCLOSE{0}\to \RR$ definita
  da $f(x)=\frac 1 x$ è una funzione continua.
  \end{theorem}
  %
  \begin{proof}
  Siano $x,x_0\neq 0$.
  Se prendiamo $\delta < \frac{\abs{x_0}}2$
  e se $\abs{x-x_0} < \delta$ si avrà,
  per disuguaglianza triangolare inversa,
  $\abs{x} > \frac{\abs{x_0}}2$. Dunque
  \[
  \abs{\frac 1 x - \frac 1 {x_0}}
  = \frac{\abs{x-x_0}}{\abs{x\cdot x_0}}
  \le \frac{2\delta}{\abs{x_0}^2}.
  \]
  Si ottiene quindi la condizione $\abs{f(x)-f(x_0)}<\eps$
  se si sceglie $\delta$ in modo che
  risulti anche $\delta < \frac{\abs{x_0}^2\cdot \eps}{2}$.
  \end{proof}
  
  \begin{example}
  La \emph{funzione segno}%
\mymargin{funzione segno}%
\index{funzione!segno} $\sgn\colon \RR \to \RR$ definita da
  \[
    \sgn(x) = \begin{cases}
    1 & \text{se $x> 0$}\\
    0 & \text{se $x=0$}\\
    -1 & \text{se $x<0$}
    \end{cases}
  \]
  è un esempio di funzione non continua.
  \end{example}
  %
  \begin{proof}
  Verifichiamo che la funzione non è continua
  nel punto $x_0=0$.
  Infatti se $x\neq 0$ si ha
  \[
  \abs{\sgn(x)-\sgn(x_0)} = \abs{\sgn(x)} = 1
  \]
  e quindi se scegliamo $\eps<1$ non è possibile
  trovare $\delta>0$ per cui valga la condizione
  di continuità~\eqref{eq:continuita}
  nel punto $x_0=0$.
  \end{proof}
  
  \begin{exercise}
  Dimostrare che le funzioni $f(x) = x$ e $g(x)=\abs{x}$ sono continue.
  Dimostrare che le funzioni $h(x) = \lfloor x\rfloor$ e $k(x)=\lceil x \rceil$
  non sono continue (ma sono continue in ogni punto
  $x_0\in \RR \setminus \ZZ$).
  \end{exercise}
  
  \begin{definition}[operazioni sulle funzioni]
  Sia $A \subset \RR$ (oppure $A\subset \CC$) 
  e siano $f,g$ funzioni $A \to \RR$ (oppure $A\to \CC$).
  Possiamo allora definire
  $f+g$, $-f$, $f-g$, $f\cdot g$, $\abs f$, $f^n$ (con $n\in \NN$)
  e (se $g(x)\neq 0$ per ogni $x\in A$) anche $f/g$
  come funzioni $A \to \RR$ (o $A\to \CC$) mediante le seguenti ovvie
  definizioni
  \begin{gather*}
  (f+g)(x) = f(x) + g(x), \qquad
  (f-g)(x) = f(x) - g(x), \\
  (f\cdot g)(x) = f(x) \cdot g(x), \qquad
  (f/g)(x) = f(x) / g(x), \\
  (-f)(x) = -(f(x)), \qquad
  \abs f(x) = \abs{f(x)}, \qquad
  f^n(x) = (f(x))^n.
  \end{gather*}
  
  Se $f\colon A \to B$ e $g\colon B\to C$ ricordiamo
  che abbiamo definito la funzione composta
  $g\circ f\colon A \to C$:
  \[
    (g\circ f)(x) = g(f(x)).
  \]
  
  Se $c\in \RR$ (o $c\in \CC$) è un numero considereremo a volte 
  $c\colon A \to \RR$ come una funzione \emph{costante}.
  Risulta quindi inteso che $c\cdot f$ è la funzione definita da
  $(c\cdot f)(x) = c\cdot (f(x))$.
  La somma e il prodotto per costante rendono l'insieme $\RR^A$
  delle funzioni $A \to \RR$ uno spazio vettoriale sul campo $\RR$
  (lo stesso vale per $\CC^A$ con $A\subset \CC$)
  \end{definition}
  
  \begin{theorem}[composizione di funzioni continue]
  \label{th:continuita_composizione}%
  Se $f$ e $g$ sono funzioni (reali o complesse) definite e continue
  in uno stesso punto $x_0$ (reale o complesso)
  allora anche
  \[
    f+g, \qquad
    f\cdot g, \qquad
    f-g, \qquad
    \abs{f}, \qquad
    f^n\ \text{(con $n\in \NN$)}
  \]
  sono funzioni definite e continue nel punto $x_0$.
  Se inoltre $g(x_0)\neq 0$ anche la funzione
  $f/g$
  è definita e continua nel punto $x_0$.
  
  Se $f\colon A\subset \RR \to \RR$ è una funzione continua
  nel punto $x_0\in A$ e
  $g\colon B\subset \RR \to \RR$ è una funzione
  continua nel punto $y_0=f(x_0)\in B$ allora la funzione $g\circ f$
  definita in $f^{-1}(B)$ è continua nel punto $x_0$
  (lo stesso vale sostituendo $\CC$ ad $\RR$)
  \end{theorem}
  %
  \begin{proof}
  Mostriamo prima di tutto la continuità
  della composizione $g\circ f$.
  Per la continuità di $f$ in $x_0$ e di $g$ in $y_0=f(x_0)$
  si ha che per ogni $\eps>0$ esiste un $\gamma>0$
  e per ogni $\gamma>0$ esiste un $\delta>0$ per cui
  \begin{align*}
   \abs{x-x_0}< \delta &\implies \abs{f(x)-f(x_0)}<\gamma,\\
   \abs{y-y_0}< \gamma & \implies \abs{g(y)-g(y_0)}<\eps
  \end{align*}
  da cui
  \[
  \abs{x-x_0}< \delta
  \implies \abs{f(x)-f(x_0)}< \gamma
  \implies \abs{g(f(x))-g(f(x_0))} < \eps
  \]
  che non è altro che la condizione di
  continuità~\eqref{eq:continuita} per $g\circ f$.
  
  Se $f$ e $g$ sono continue nel punto $x_0$
  allora per ogni $\eps'>0$ esistono $\delta_1$
  e $\delta_2$ tali che
  \begin{align*}
   \abs{x-x_0} < \delta_1 &\implies \abs{f(x)-f(x_0)} < \eps',
   \\
   \abs{x-x_0} < \delta_2 &\implies \abs{g(x)-g(x_0)} < \eps'.
  \end{align*}
  In particolare scegliendo $\delta = \min\ENCLOSE{\delta_1,\delta_2}$
  se $\abs{x-x_0} < \delta$ valgono contemporaneamente
  entrambe le stime:
  \[
   \abs{f(x)-f(x_0)}< \eps', \qquad
   \abs{g(x)-g(x_0)}< \eps'.
  \]
  
  Dunque per la somma osserviamo che se $\abs{x-x_0}<\delta$
  si ha
  \begin{align*}
   \abs{(f+g)(x) - (f+g)(x_0)}
    &= \abs{f(x)+g(x)-f(x_0)-g(x_0)}\\
    &\le \abs{f(x)-f(x_0)} + \abs{g(x)-g(x_0)}
    \le 2 \eps'.
  \end{align*}
  Dato $\eps>0$ basterà quindi scegliere $\eps' = \eps / 2$
  e $\delta$ come sopra per ottenere la condizione di continuità.
  
  Per il prodotto si ha
  \begin{align*}
    \MoveEqLeft \abs{f(x)g(x)-f(x_0)g(x_0)}\\
    &= \abs{f(x)g(x) - f(x_0)g(x) + f(x_0)g(x) - f(x_0)g(x_0)}\\
    &\le \abs{f(x)-f(x_0)}\cdot\abs{g(x)} + \abs{f(x_0)}\cdot \abs{g(x)-g(x_0)} \\
    &\le \eps' \abs{g(x)} + \abs{f(x_0)} \eps'.
  \end{align*}
  Osserviamo ora che $\abs{g(x)}\le \abs{g(x)-g(x_0)} + \abs{g(x_0)}$
  e quindi $\abs{g(x)}\le \abs{g(x_0)}+\eps'$ da cui:
  \begin{align*}
    \abs{f(x)g(x)-f(x_0)g(x_0)}
    &\le \eps' \enclose{\abs{g(x_0)}+\eps'} + \abs{f(x_0)} \eps' \\
    &= \eps' \cdot(\abs{f(x_0)} + \abs{g(x_0)} + \eps').
  \end{align*}
  Possiamo facilmente rendere questa quantità inferiore a
  qualunque $\eps>0$: basterà prendere $\eps'<1$ e
  \[
    \eps' < \frac{\eps}{\abs{f(x_0)} + \abs{g(x_0)} + 1}.
  \]
  
  La funzione $f^n$ è continua per induzione su $n$
  in quanto prodotto di funzioni
  continue: $f^{n+1} = f^{n} \cdot f$.
  
  Abbiamo già visto che la funzione $h(x) = \frac{1}{x}$ è continua,
  dunque se $g$ è continua anche la funzione $\frac{1}{g(x)} = h\circ g$
  è continua essendo composizione di funzioni continue. 
  Di conseguenza anche la funzione $\frac{f}{g} = f \cdot \frac{1}{g}$
  è continua, essendo il prodotto di funzioni continue.
  
  Analogamente la funzione $f-g$ è la somma di $f$ con $-g$ e
  la funzione $-g$ è la composizione di $g$ con $h(x)=-x$.
  E' immediato verificare che la funzione $h(x)=-x$ è continua
  e dunque anche la differenza $f-g$ è continua.
  Lo stesso vale per la funzione $\abs f$ che è la composizione
  di $f$ con la funzione $h(x) = \abs{x}$.
  \end{proof}
  
  Il precedente teorema è molto importante ed utile in quanto
  garantisce che ogni funzione definita tramite una espressione
  che coinvolge solamente le funzioni e le operazioni
  elencate nell'enunciato del teorema, risulta certamente
  essere una funzione continua. Come nel seguente.
  
  \begin{example}
  La funzione
  \[
  f(x) = \frac{(x-3)\cdot x -\frac{1}{x+x^2}}{\abs{x-\frac{1-x^3}{x}}}
  \]
  è continua.
  
  Si intende che tale funzione è definita sull'insieme degli $x\in \RR$
  per cui tutte le operazioni coinvolte sono definite ovvero
  $f\colon D \subset \RR \to \RR$
  con
  \[
    D = \ENCLOSE{x\in \RR \colon x+x^2 \neq 0,\ x\neq 0,\ \abs{x-\frac{1-x^3}{x}}\neq 0}.
  \]
  \end{example}
  \begin{proof}
  Per convincersi che questa funzione $f$ è continua
  si nota che le funzioni $x$ e le costanti $3$ e $1$ sono
  funzioni continue.
  Ma allora, per il teorema~\ref{th:continuita_composizione},
  anche le funzioni $x-3$ e $x^2=x\cdot x$ sono continue.
  Dunque anche $(x-3)\cdot x$, $x+x^2$ e $x^3$ e $1-x^3$ sono continue.
  Di conseguenza sono continue pure $\frac 1{1+x^2}$ e $\frac{1-x^3}{x}$.
  E poi saranno continue anche $(x-3)\cdot x - \frac 1{1+x^2}$ e $x-\frac{1-x^3}{x}$
  e quindi $x-\frac{1-x^3}{x}$ e pure $\abs{x-\frac{1-x^3}{x}}$. Infine sarà
  dunque continua $f(x)$.
  \end{proof}
  
  In particolare è chiaro che le funzioni lineari e quadratiche 
  che abbiamo introdotto nelle sezioni precedenti sono funzioni continue
  in quanto sono ottenute sommando e moltiplicando tra loro funzioni continue.
  
  %%%%%%%%%%%%%%%%%%%
  
  \begin{theorem}[continuità delle funzioni monotone]
    \label{th:monotona_continua}%
    \mymark{*}%
    \index{funzione!continua}%
    \index{funzione!monotòna}%
    \index{funzione!suriettiva}%
    Sia $A\subset \RR$ e sia
    $f\colon I \to \RR$ una funzione monotòna.
    Se $f(I)$ è un intervallo allora $f$ è continua.
%    \label{th:monotona_continua_reverse}%
%    Allora $f$ è continua se e solo se $f(I)$ è 
%    un intervallo.
\end{theorem}
  %
  \begin{proof}
  Senza perdita di generalità possiamo supporre che $f$ 
  sia crescente. 
  % Dimostriamo innanzitutto che se $f(I)$ è intervallo allora $f$ è continua.
  
  Prendiamo un punto $x_0\in I$ e sia $\eps>0$.
  Vogliamo trovare $x_1<x_0$ tale che  
  per ogni $x\in [x_1,x_0]\cap I$ si abbia $f(x)>f(x_0)-\eps$.
  Siccome $f(I)$ è un intervallo che contiene il punto $f(x_0)$ 
  ci sono due possibilità: o $f(x)>f(x_0)-\eps$ per ogni $x\in I$
  e quindi possiamo scegliere $x_1<x_0$ a piacere
  oppure esiste $x_1\in I$ tale che $f(x_1)=f(x_0) - \frac \eps 2$.
  In questo secondo caso dovrà essere $x_1<x_0$ (in quanto $f$ è crescente)
  e per monotonia si avrà, come voluto $f(x)\ge f(x_1) > f(x_0)-\eps$ 
  per ogni $x\in [x_1,x_0]$.
  
  In modo analogo possiamo trovare $x_2>x_0$ tale 
  che per ogni $x\in [x_0,x_2]\cap I$ si abbia $f(x) < f(x_0)+\eps$.
  Essendo $f$ crescente se $x\ge x_0$ si avrà anche $f(x)\ge f(x_0)$ 
  e se $x\le x_0$ si avrà $f(x) \le f(x_0)$. 
  Dunque per ogni $x\in [x_1,x_2]$ si avrà $\abs{f(x)-f(x_0)}<\eps$.
  Basterà scegliere $\delta = \min\ENCLOSE{x_0-x_1,x_2-x_0}$ 
  per ottenere la continuità di $f$ in $x_0$.
  
%  Supponiamo ora che $f$ sia una funzione continua e crescente.
%  Vogliamo dimostrare che $f(I)$ è un intervallo.
%  
%  Siano $y_1, y_2 \in f(I)$ e sia $y_0\in \RR$
%  con $y_1 < y_0 < y_2$.
%  Vogliamo dimostrare che $y_0\in f(I)$ cioè che esiste $x_0\in I$
%  tale che $f(x_0)=y_0$.
%  Sappiamo che esistono $x_1,x_2$ in $I$ tali che
%  $f(x_1) = y_1$ e $f(x_2) = y_2$.
%  Dovrà essere $x_1<x_2$ perché $f(x_1)<f(x_2)$ e
%  $f$ è crescente.
%  Definiamo:
%  \[
%   x_0 = \sup A
%  \qquad
%  \text{con } A=\ENCLOSE{x\in I\colon f(x)<y_0},
%  \]
%  e dimostriamo che $f(x_0)=y_0$. 
%  Chiaramente $x_0\in [x_1,x_2]$ in quanto 
%  $x_1\in A$ e $x_2$ è un maggiorante di $A$
%  e dunque $f$ è definita e continua in $x_0$.
%  
%  Se fosse $f(x_0)<y_0$ scelto $\eps = y_0-f(x_0)$
%  per la definizione di continuità dovrebbe esistere
%  $\delta>0$ tale che per ogni $x\in I$ con $\abs{x-x_0}<\delta$
%  si abbia $\abs{f(x)-f(x_0)}<\eps$.
%  In particolare scelto $x=x_0+\frac \delta 2$
%  si avrebbe $f(x) < f(x_0) + \eps = y_0$.
%  Ma allora avremmo $x\in A$ che è assurdo in quanto
%  $\sup A = x_0 < x$.
%  
%  Se fosse invece $f(x_0)>y_0$ posto $\eps = f(x_0)-y_0$,
%  grazie alla continuità di $f$ in $x_0$,
%  possiamo trovare $\delta>0$ tale che per ogni $x\in I$
%  con $\abs{x-x_0}<\delta$ si abbia $\abs{f(x)-f(x_0)}<\eps$.
%  In particolare scelto $t = x_0 - \frac \delta 2$
%  si ha $f(t) > f(x_0)-\eps = y_0$. 
%  Ma essendo $t<x_0 = \sup A$ sappiamo che $t$ non può essere 
%  un maggiorante di $A$ dunque deve esistere $x\in A$
%  tale che $t<x$. 
%  Ma se $x\in A$ allora $f(x) < y_0 < f(t)$
%  che è assurdo in quanto $f$ è crescente.
%  
%  Abbiamo quindi mostrato che $f(x_0)=y_0$. 
%  Questo è vero per ogni scelta di $y_1<y_2$ in $f(I)$
%  dunque $f(I)$ è un intervallo.
\end{proof}
  %

Il teorema precedente (teorema~\ref{th:monotona_continua}) 
ci permette di affermare che per $a>0$, $a\neq 1$ 
le funzioni $\exp_a$ e $\log_a$ sono continue. 
Infatti tali funzioni sono bigezioni monotone tra 
gli intervalli $\RR$ e $\RR_+$.

Anche le funzioni trigonometriche $\sin$ e $\cos$ sono funzioni continue.
Sappiamo infatti che $\sin \colon [-\tau/4,\tau /4]\to [-1,1]$
e $\cos\colon [0,\tau/2]\to [-1,1]$ sono strettamente monotone 
e bigettive dunque sono continue su tali intervalli. 
Grazie alle proprietà di simmetria e periodicità
(oppure utilizzando le formule di addizione) è facile verificare 
che tali funzioni sono continue su tutto $\RR$.

Grazie al teorema~\ref{th:continuita_composizione} sappiamo 
che se $n\in \NN$ la funzione $x^n$ è continua su tutto il suo 
dominio $\RR$. 
La funzione inversa $\sqrt[n]{x}$ è anch'essa crescente e 
bigettiva come funzione $[0,+\infty)\to[0,+\infty)$ e dunque 
anch'essa risulta essere continua.
Per simmetria (si veda l'esercizio~\ref{ex:simmetrica_continua}) 
possiamo concludere 
%che sia la potenza $x^n$ 
che le radici $\sqrt[n]{x}$ sono 
funzioni continue su tutto il loro dominio.
Anche la funzione $f(x) = x^\alpha$, $f\colon [0,+\infty)\to [0,+\infty]$ 
con $\alpha>0$ è crescente ed è invertibile (l'inversa è $x^{\frac 1 \alpha}$)
e dunque è surgettiva e continua.
Se $\alpha<0$ la funzione $f(x)=x^\alpha$ è definita sull'intervallo 
$(0,+\infty)$ ed è anch'essa continua in quanto 
composizione di funzioni continue: $x^{\alpha} = \frac{1}{x^{-\alpha}}$.

%
\begin{exercise}\label{ex:simmetrica_continua}
  Sia $f\colon \RR\to \RR$ una funzione pari oppure dispari.
  Se la restrizione di $f$ all'intervallo $[0,+\infty)$ 
  è continua allora $f$ è continua su tutto $\RR$.
\end{exercise}

\subsection{limite}

Se una funzione $f$ non è definita in un punto $x_0$ non ha senso chiedersi
se in tale punto è continua. 
Possiamo però chiederci se è possibile definire la funzione anche nel punto 
$x_0$ dando un valore opportuno $\ell$ in modo da rendere $f$ continua 
in quel punto. 
Se ciò è possibile diremo che la funzione $f(x)$ ha limite $\ell$ 
per $x$ che tende a $x_0$ e scriveremo:
\[
  f(x) \to \ell \qquad \text{per $x\to x_0$}
\]
(daremo tra poco una definizione con maggiore precisione e generalità).

Ad esempio la funzione $f(x) = \frac{x^2-1}{x-1}$ è una funzione definita per $x\neq 1$
e coincide, se $x\neq 1$ con la funzione lineare $\tilde f(x) = x+1$ definita 
su tutto $\RR$. 
Visto che $\tilde f$ è continua e $\tilde f(1)=2$
scriveremo:
\[
  \frac{x^2-1}{x-1} \to 2
  \qquad \text{per $x\to 1$.}
\]

Se estendiamo $f$ con un valore $\ell$ nel punto $x_0$ otteniamo 
in generale la funzione 
\[
  \tilde f(x) = \begin{cases}
    f(x) & \text{se $x\neq x_0$}\\
    \ell & \text{se $x=x_0$}
  \end{cases}  
\]
e la continuità di $\tilde f$ nel punto $x_0$ si scrive così:
\[
\forall\eps>0\colon \exists \delta>0\colon
\abs{x-x_0}<\delta \implies \abs{\tilde f(x)-\tilde f(x_0)}<\eps.  
\]
Osserviamo che se $x=x_0$ allora ovviamente $\abs{\tilde f(x)-\tilde f(x_0)}
= 0 < \eps$ dunque possiamo supporre, nella condizione precedente, 
che sia $x\neq x_0$ e dunque $\tilde f(x)=f(x)$. 
Inoltre visto che $\tilde f(x_0)=\ell$ 
e $\tilde f(x)=f(x)$ se $x\neq x_0$,
si ottiene 
la seguente condizione:
\begin{equation}\label{eq:55338}
\forall \eps>0\colon \exists \delta >0 \colon 
  \forall x\neq x_0\colon
  \abs{x-x_0}<\delta \implies \abs{f(x) - \ell} < \eps.
\end{equation}
La~\eqref{eq:55338} potrebbe dunque essere utilizzata per definire 
il concetto di limite $f(x)\to \ell$ per $x\to x_0$.
Sarà però molto utile estendere il concetto di limite $f(x)\to \ell$ 
per $x\to x_0$ anche nei casi in cui $\ell$ e/o $x_0$ possano 
essere infiniti (cioè elementi dei reali estesi $\bar \RR$).

Per fare ciò osserviamo che
se definiamo%
\mynote{%
la lettera $B$ sta per \emph{ball} in quanto 
più in generale (se fossimo in $\RR^3$ invece che in $\RR$)
l'insieme dei punti che distano meno di $\rho$ da un punto fissato 
è l'interno di una sfera piena. 
In geometria una sfera piena si chiama \emph{palla} 
se contiene solo i punti interni (e non la superficie sferica)
e si chiama \emph{disco} o \emph{palla chiusa} se contiene 
anche i punti della superficie.
}
$B_\rho(x_0) = \ENCLOSE{x\in \RR \colon \abs{x-x_0}<\rho}$
la condizione \eqref{eq:55338}
può essere scritta anche nella forma 
\[
  \forall \eps>0\colon \exists \delta >0 \colon 
  x \in B_\delta(x_0)\setminus\ENCLOSE{x_0} \implies f(x) \in B_\eps(\ell)  
\]
che a sua volta si può scrivere nella forma:
\[
  \forall \eps>0\colon \exists \delta >0 \colon 
  f(B_\delta(x_0)\setminus\ENCLOSE{x_0}) \subset  B_\eps(\ell).    
\]

L'idea è che gli insiemi del tipo
$B_\rho(x_0)$ 
rappresentano i punti \emph{vicini} al punto $x_0$. 
In analogia potremmo pensare che i punti \emph{vicini} 
a $+\infty$ siano i punti di una qualunque semiretta 
del tipo $(M,+\infty]$.
Si dà quindi la seguente.


\begin{definition}[intorno]
Per $x\in \RR$ definiamo la famiglia degli \emph{intorni}%
\mymargin{intorni}%
\index{intorno} (basilari) di $x$
come l'insieme di tutti gli intervallini aperti, simmetrici, centrati in $x$:
\[
  \B_x = \ENCLOSE{ (x-\eps, x+\eps) \colon \eps>0}.
\]
Definiamo poi le famiglie 
degli \emph{intorni destri} e \emph{intorni sinistri}
\mymargin{intorni destri/sinistri}%
\index{intorni!destri/sinistri}
\index{intorni}
di $x$ come
\[
  \B_{x^+} = \ENCLOSE{ [x, x+\eps) \colon \eps>0},
  \qquad
  \B_{x^-} = \ENCLOSE{ (x-\eps , x] \colon \eps>0}
\]
e le famiglie degli intorni di $+\infty$ e $-\infty$ come segue
\[
  \B_{+\infty} = \ENCLOSE{ (\alpha,+\infty], \colon \alpha \in \RR },\qquad
  \B_{-\infty} = \ENCLOSE{ [-\infty, -\beta), \colon \beta\in \RR}.
\]

Per ogni $x\in \bar \RR = [-\infty, +\infty]$
risultano quindi definiti gli intorni $\B_x$ e per
ogni $x\in \RR$ sono definiti gli intorni $\B_{x^+}$ e $\B_{x^-}$.

Più in generale se $x\in \CC$ o $x\in \RR^n$
si definiscono gli intorni di $x$ 
utilizzando la norma $\abs{\cdot}$.
Posto $B_\rho(x) = \ENCLOSE{y\colon \abs{y-x}<\rho}$
(palla di raggio $\rho$ centrata in $x$)
la famiglia di intorni di $x$ non è altro che
\[
  \B_x = \ENCLOSE{B_\rho(x)\colon \rho>0}.
\]
Nel caso $x\in \RR$ si riottiene la stessa definizione che 
abbiamo dato sopra.

In $\CC$ o in $\RR^n$ non c'è un ordinamento quindi 
tipicamente si aggiunge un solo punto all'infinito: $\infty$.
Gli intorni di $\infty$
sono dati dall'esterno di una palla:
\[
  \B_\infty = \ENCLOSE{\ENCLOSE{y\colon \abs{y}\ge R}\colon R>0}.
\]

Anche su $\RR$ si può considerare un unico punto 
all'infinito (che potremmo denotare con $\infty$)
invece che i due punti $+\infty$ e $-\infty$.
\end{definition}

% \begin{remark}
% Sarebbe possibile definire in maniera analoga gli intorni dei punti in $\RR^n$
% (per l'analisi di funzioni di più variabili)
% o in $\CC$ (per l'analisi complessa).
% Ma in questo corso e in questo capitolo in particolare siamo interessati 
% allo studio delle funzioni di una singola variabile.
% 
% Su $\RR$ c'è una struttura di ordine totale che è utile preservare aggiungendo
% due punti all'infinito: $+\infty$ e $-\infty$.
% Su $\RR^n$ (e su $\CC$, che in questo contesto possiamo identificare con $\RR^2$)
% non c'è una struttura d'ordine naturale e quindi
% usualmente si considera un unico punto all'infinito $\infty$ i cui intorni
% saranno
% \[
%   \B_\infty = \ENCLOSE{\ENCLOSE{x \in \RR^n\colon \abs{x}>R}\colon R>0}.
% \]
% 
% Su $\RR^n$ (e su $\CC$) non esiste il concetto di intorno \emph{destro}
% e \emph{sinistro} proprio perché questi concetti presuppongono un ordinamento.
% 
% In certi casi può tornare utile considerare un unico punto all'infinito,
% denotato con $\infty$, anche in $\RR$ (in molti testi tale punto verrebbe
% denotato con il simbolo $\pm\infty$)
% e si potrebbero usare le notazioni $+\infty = \infty^-$ e $-\infty = \infty^
% +$ visto che gli intorni di $+\infty$ e $-\infty$ sono in effetti intorni
% unilaterali del punto all'infinito.
% \end{remark}

\begin{definition}[limite di funzione]
\mymark{***}
Sia $A\subset \RR$ e $f\colon A \to \RR$. 
Sia $x_0\in [-\infty,+\infty]$
% un punto di accumulazione di $A$ 
e sia $\ell \in [-\infty,+\infty]$.
Allora diremo che la funzione $f$ ha limite $\ell$ per $x$ che tende a $x_0$ 
e scriveremo\mymargin{limite di funzione}%
\index{limite!di funzione}
\[
  f(x) \to \ell \qquad \text{per $x\to x_0$}
\]
se per ogni intorno di $\ell$ esiste un intorno di $x_0$ tale che
la funzione valutata nell'intorno di $x_0$, tolto eventualmente $x_0$,
assume valori
nell'intorno di $\ell$:
\begin{equation}\label{eq:def_limite}
  \forall V \in \B_\ell \colon \exists U \in \B_{x_0} \colon f(U\setminus\ENCLOSE{x_0}) \subset V.
\end{equation}

La stessa definizione può essere data restringendosi agli intorni destri/sinistri del punto $x_0$ (nel caso $x_0 \in \RR$). Si otterranno quindi le definizioni
di \emph{limite destro} e \emph{limite sinistro}
\mymargin{limite destro/sinistro}%
\index{limite!destro/sinistro} %
semplicemente sostituendo $\B_{x_0^+}$ o $\B_{x_0^-}$ al posto di 
$\B_{x_0}$ nella definizione
precedente. 
In tal caso scriveremo $f(x)\to \ell$ per $x\to x_0^+$ per il limite 
destro e $f(x)\to \ell$ per $x\to x_0^-$ per il limite sinistro.

Infine anche il risultato del limite può essere $\ell^+$ o $\ell^-$, 
in tal caso useremo $\B_{\ell^+}$ o $\B_{\ell^-}$ 
al posto di $\B_{\ell}$.

Anche per le funzioni complesse $f\colon A\subset \CC \to \CC$ 
si applica la stessa identica definizione.
\end{definition}
  
\begin{example}
Si consideri la funzione segno:
\[
\sgn(x) =
\begin{cases}
  1 & \text{se $x>0$},\\
  0 & \text{se $x=0$},\\
  -1 & \text{se $x<0$}.
\end{cases}
\]
Si può verificare che
\[
\sgn(x) \to 1 \qquad \text{per $x\to 0^+$}
\]
e
\[
\sgn(x) \to -1 \qquad \text{per $x\to 0^-$}
\]
\end{example}

Abbiamo già visto che nel caso in cui $x_0\in \RR$ e $\ell\in \RR$ 
siano entrambi finiti 
la definizione di limite $f(x)\to \ell$ per $x\to x_0$
si traduce nella condizione~\eqref{eq:55338}.

Anche negli altri casi esplicitando le definizioni di intorno
si possono ottenere delle condizioni più esplicite.
Ad esempio la condizione $f(x)\to -\infty$ per $x\to x_0^+$
si traduce nel modo seguente:
\[
\forall \beta\in \RR\colon \exists \delta>0 \colon x_0 < x < x_0+\delta 
\implies f(x) < -\beta.  
\]
Visto che ci sono cinque diversi casi per il punto $x_0$:
$x_0\in \RR$, $x_0^+$, $x_0^-$, $+\infty$, $-\infty$ e altrettanti 
casi per $\ell$ (in effetti anche il risultato del limite 
può essere \emph{destro} o \emph{sinistro}), si ottengono 
in tutto 25 diverse definizioni di limite.

In tutte queste definizioni di limite è sottointeso che i punti 
$x$ che vengono presi in considerazione sono punti 
del dominio di $f\colon A \subset \RR \to \RR$.
Accade allora che se esiste un intorno $V\in \B_{x_0}$
per cui $A \cap V \setminus \ENCLOSE{x_0}$ è vuoto allora 
la condizione di limite è vuota ed è quindi sempre verificata 
qualunque sia il valore di $\ell$. 
E' quindi inutile fare il limite per $x\to x_0$ 
se $x_0$ non soddisfa la seguente.

\begin{definition}[punto di accumulazione]
  \mymark{*}
  Siano $A\subset  \RR$ un insieme e $x_0\in [-\infty, +\infty]$.
  Diremo che $x_0$ è un \emph{punto di accumulazione}%
\mymargin{punto di accumulazione}%
\index{punto!di accumulazione} di $A$
  se ogni intorno di $x_0$ contiene punti di $A$ diversi da $x_0$, ovvero:
  \[
   \forall U \in \B_{x_0}\colon (A\setminus \ENCLOSE{x_0}) \cap U \neq \emptyset.
  \]

  Stessa definizione si può dare per $x_0^+$ e $x_0^-$ 
  utilizzando gli intorni destri/sinistri di $x_0$. 

  Sempre la stessa definizione si può applicare 
  quando $A\subset \CC$ e $x_0\in \bar \CC$.
\end{definition}

\begin{theorem}[unicità del limite]
\mymark{*}
Sia $A\subset \RR$, $f\colon A \to \RR$, $x_0$
punto di accumulazione per $A$ e $\ell_1, \ell_2\in [-\infty,+\infty]$.
Se per $x\to x_0$ si ha
\[
  f(x) \to \ell_1 \qquad\text{e}\qquad f(x) \to \ell_2
\]
allora $\ell_1=\ell_2$.

Risultato analogo si ha per i limiti destro e sinistro: $x\to x_0^+$, 
$x\to x_0^-$.
Anche su $\CC$ vale lo stesso risultato
prendendo $\ell_1,\ell_2\in \bar \CC$.
\end{theorem}
%
\begin{proof}
\mymark{*}
Supponiamo per assurdo che $\ell_1\neq \ell_2$.
Allora esiste un intorno $V_1$ di $\ell_1$ ed un intorno $V_2$ di $\ell_2$
tali che $V_1\cap V_2 = \emptyset$ (basta prendere degli intorni abbastanza piccoli). 
Ma per le definizioni di limite $f(x)\to \ell_1$ e $f(x)\to \ell_2$ 
dovranno esistere $U_1$ e $U_2$ intorni di $x_0$ su cui si ha 
$f(U_1)\subset V_1$ e $f(U_2)\subset V_2$. 
Ma allora $f((A\setminus\ENCLOSE{x_0})\cap U_1)\cap f((A\setminus\ENCLOSE{x_0})\cap U_2)\subset V_1\cap V_2 = \emptyset$... 
e questo è assurdo perché certamente $U_1\cap U_2$ 
contiene punti di $A$ diversi da $x_0$ in quanto 
$U_1$ e $U_2$ sono uno contenuto nell'altro e $x_0$ 
è un punto di accumulazione per $A$.
\end{proof}

Il teorema precedente garantisce che se $x_0$ è un punto di accumulazione 
del dominio di $f$ allora il limite per $x\to x_0$ se esiste è unico. 
In tal caso possiamo dunque dare la seguente.
%
\begin{definition}[operatore di limite]
Sia $f\colon A\subset \RR \to \RR$ una funzione e $x_0$ 
un punto di accumulazione per $A$. 
Se esiste $\ell\in\closeinterval{-\infty}{+\infty}$ tale che
$f(x)\to \ell$ per $x\to x_0$ allora poniamo
\[
  \lim_{x\to x_0} f(x) = \ell.
\]
Lo stesso si può fare per il limite destro $x\to x_0^+$ 
e sinistro $x\to x_0^-$.
Anche per funzioni di variabile e/o di valore 
complesso si può dare la stessa definizione prendendo $x_0\in \bar \CC$ 
e/o $\ell\in \bar \CC$.
\end{definition}

Abbiamo visto che in un punto di accumulazione se il limite 
esiste allora è unico.
Sarà però utile osservare che in effetti il limite può non 
esistere, come si può verificare con un esempio.

\begin{example}[in generale il limite non esiste]
Il limite 
\[
  \lim_{x\to 0} \frac{x}{\abs{x}}
\]
non esiste. 
\end{example}
\begin{proof}
Osserviamo infatti che la funzione $f(x) = \frac{x}{\abs x}$ 
vale $1$ se $x>0$ e $-1$ se $x<0$.
Dunque se il limite esistesse e fosse $\ell\in \bar \RR$, 
scelto $\eps=1$, dovrebbe esistere un intervallo intorno di $0$ 
della forma $\openinterval{-\delta}{\delta}$ tale che 
$\abs{f(x)-\ell}<1$ per ogni $x$ in tale intervallo.
In particolare siccome in tale intervallo ci sono sicuramente 
sia numeri positivi che negativi si dovrà avere 
contemporaneamente
\[
  \abs{1-\ell}<1, \qquad 
  \abs{-1-\ell}<1  
\]
che è impossibile in quanto per disuguaglianza triangolare 
\[
 2 = \abs{1-(-1)} 
 \le \abs{1-\ell} + \abs{-1-\ell}
 < 1 + 1 = 2
\]
che è assurdo.
\end{proof}

Si noti che nell'esempio precedente il limite per $x\to 0$ 
non esiste ma in realtà esistono sia il limite per $x\to 0^+$ 
(vale $1$) che i limite per $x\to 0^-$ (vale $-1$).
Possiamo esibire un esempio ``patologico'' di funzione 
che non ammette limite (né da destra né da sinistra)
in nessun punto:
\[
  f(x) = 
  \begin{cases}
     1 & \text{se $x\in \QQ$}\\ 
     0 & \text{se $x\in \RR\setminus \QQ$}.
  \end{cases}
\]
In ogni intervallo $\openinterval a b$ con $a<b$ questa funzione 
assume sia il valore $0$ che il valore $1$ 
in quanto sia $\QQ$ che $\RR\setminus \QQ$ intersecano 
l'intervallo (sono insiemi densi).
Ma se $f$ avesse limite $\ell$ dovrebbe esistere un intervallo 
in cui i valori distano meno di $\eps$ da $\ell$ e quindi 
dovrebbe essere $\abs{1-\ell}<\eps$ e $\abs{0-\ell}<\eps$
che è impossibile se scegliamo $\eps < \frac 1 2$.

\begin{theorem}[collegamento tra limiti e continuità]%
\mymark{***}%
  Sia $A\subset \RR$, $f\colon A \to \RR$. 
  Se $x_0\in A$ è un punto di accumulazione di $A$
  allora $f$ è continua in $x_0$ se e solo se
  \[
    \lim_{x\to x_0}f(x) = f(x_0).
  \]
  Se $x_0\in A$ non è punto di accumulazione diremo 
  che $x_0$ è un \emph{punto isolato}%
\mymargin{punto isolato}%
\index{punto!isolato} di $A$.
  In tal caso la funzione $f$ è sempre continua nel punto $x_0$.

Risultato analogo vale per funzioni complesse e/o di variabile complessa.
\end{theorem}
  
  \begin{proof}
  In base alla definizione~\ref{def:continua} la funzione $f$ è continua nel
  punto $x_0\in \bar \RR$ se
  \[
   \forall \eps>0 \colon \exists \delta >0 \colon
   \forall x \in A\colon
   \abs{x-x_0}<\delta \implies \abs{f(x)-f(x_0)} < \eps
  \]
  mentre la definizione di limite $f(x)\to f(x_0)$ per $x\to x_0$
  si espande in
  \[
  \forall \eps>0 \colon \exists \delta>0\colon
  \forall x \in A, x\neq x_0\colon
  \abs{x-x_0}<\delta \implies \abs{f(x)-f(x_0)} < \eps.
  \]
  L'unica differenza è che nella definizione di limite
  c'è la condizione $x\neq x_0$. Ma visto che per $x=x_0$
  si ha $\abs{f(x)-f(x_0)}=0$ tale condizione è in questo caso 
  inutile e quindi le due definizioni sono equivalenti.

  Se il punto $x_0$ è isolato la definizione di continuità
  è sempre verificata in quanto esiste un $\delta>0$ 
  tale che il punto $x=x_0$ è l'unico punto di $A$ 
  nell'intorno $(x_0-\delta,x_0+\delta)$.
  \end{proof}

\begin{example}
  Non è difficile convincersi che gli unici punti di accumulazione 
  per l'insieme $\ZZ$ sono $+\infty$ e $-\infty$.
  Dunque qualunque funzione $f\colon \ZZ \to \RR$ è continua in quanto 
  tutti i punti del suo dominio sono punti isolati.
\end{example}
  
\begin{theorem}[località del limite]%
\label{th:localita_limite}%
Il limite di una funzione per $x\to x_0$ dipende solamente dai valori di $f$
in un intorno di $x_0$ e non dipende dal valore di $f$ in $x_0$.

Più precisamente: se $A,B\subset \RR$, $f\colon A\to \RR$, $g\colon B\to \RR$, 
$x_0\in \RR$ sono tali che 
esiste un intorno $V$ di $x_0$ per cui 
$(A\setminus\ENCLOSE{x_0}) \cap  V = (B\setminus \ENCLOSE{x_0}) \cap V$ 
e $f(x)=g(x)$ per ogni $x\in(A\setminus\ENCLOSE{x_0}) \cap  V$ 
allora se $f(x)\to \ell$ per $x\to x_0$ anche $g(x)\to \ell$ 
per $x\to x_0$.

Lo stesso risultato vale per funzioni complesse e/o di variabile complessa.
\end{theorem}
%
\begin{proof}
  Basta osservare che nella definizione di limite 
  non è restrittivo supporre che l'intorno del punto $x_0$ 
  sia sempre preso all'interno dell'intorno $V$ su cui 
  le due funzioni coincidono.
\end{proof}

\begin{theorem}[restrizione del limite]
Se una funzione ha limite $\ell$ per $x\to x_0$ 
e se restringiamo l'insieme di definizione della funzione 
allora il limite della funzione non cambia. 
Più precisamente
se $f\colon A \to \RR$ è una funzione tale che $f(x)\to \ell$ per $x\to x_0$
e se $B \subset A$ e $g\colon B\to \RR$ è la restrizione di $f$ 
a $B$ allora anche $g(x)\to \ell$ per $x\to x_0$. 
\end{theorem}
%
\begin{proof}
Il teorema segue immediatamente dalla definizione di limite se si osserva
che restringendo il dominio la condizione di validità del limite si indebolisce
in quanto gli intorni di $x_0$ vengono intersecati con il dominio della funzione.
\end{proof}

Si osservi che a differenza del teorema sulla località del limite è
possibile che la funzione ristretta $g$ abbia limite quando la funzione
$f$ non aveva limite.
Si osservi anche che in entrambi questi teoremi sarà opportuno 
che $x_0$ sia un punto di accumulazione, altrimenti la condizione $f(x)\to \ell$ 
risulta essere vuota.

\begin{theorem}[legame tra limite, limite destro e limite sinistro]%
\mymark{*}%
Sia $A\subset \RR$, $f\colon A \to \RR$ una funzione e $x_0$ un punto di accumulazione
di $A$. Sia $A^+ = A \cap [x_0,+\infty)$ e $A^- = A \cap (-\infty, x_0]$.

Se $x_0$ è punto di accumulazione sia di $A^+$ che di $A^-$
allora si ha
\[
  \lim_{x\to x_0} f(x) = \ell
\]
se e solo se
\[
  \lim_{x\to x_0^+} f(x) = \lim_{x\to x_0^-} f(x) = \ell.
\]

Se $x_0$ è punto di accumulazione di $A^+$ ma non di $A^-$ allora
i limiti
\[
  \lim_{x\to x_0} f(x) \qquad \text{e}\qquad \lim_{x\to x_0^+} f(x)
\]
sono equivalenti. Analogamente se $x_0$ è punto di accumulazione
di $A^-$ ma non di $A^+$ risultano equivalenti
\[
  \lim_{x\to x_0} f(x) \qquad \text{e}\qquad \lim_{x\to x_0^-} f(x).
\]
\end{theorem}
%
\begin{proof}
Si tratta semplicemente di verificare le definizioni di limite sfruttando il fatto che intorni di un punto $x_0$ sono formati dall'unione di intorno destro e intorno sinistro.
\end{proof}

\begin{theorem}[limite della funzione composta/cambio di variabile]
\label{th:limite_composta}
Siano $A\subset \RR$, $B\subset \RR$,
$x_0$ un punto di accumulazione di $A$,
$y_0$ un punto di accumulazione di $B$,
$\ell\in [-\infty,+\infty]$.
Siano $f\colon A \to B$, $g\colon B\to \RR$
funzioni tali che $f(x)\neq y_0$ se $x\neq x_0$ e 
\[
  \lim_{x\to x_0} f(x) = y_0,
\qquad
  \lim_{y\to y_0} g(y) = \ell.
\]
Allora nel secondo limite si può porre $y=f(x)$ e al posto di $y\to y_0$ 
si può mettere $x\to x_0$ (in quanto il primo limite ci dice che se $x\to x_0$ 
allora $y\to y_0$) e dunque vale:
\[
 \lim_{x\to x_0} g(f(x)) = \ell.
\]
\end{theorem}
%
\begin{proof}
Visto che $g(y)\to \ell$
per ogni $U$ intorno di $\ell$ deve esistere un $V$ intorno di $y_0$
tale che $g((B\setminus\ENCLOSE{y_0})\cap V) \subset U$
e visto  che $f(x)\to y_0$ deve esistere un intorno $W$ di $x_0$
tale che $f((A\setminus\ENCLOSE{x_0})\cap W) \subset V$.
Ma visto che per ipotesi $f$ assume valori in $B\setminus\ENCLOSE{y_0}$
si ha anche $f((A\setminus\ENCLOSE{x_0})\cap W)\subset (B\setminus\ENCLOSE{y_0}) \cap V$
e quindi
\[
  g(f((A\setminus\ENCLOSE{x_0})\cap W)) \subset g((B\setminus \ENCLOSE{y_0}) \cap V)
  \subset U
\]
che significa che $g(f(x)) \to \ell$.
\end{proof}

\begin{exercise}
  Si faccia un esempio di una funzione $f\colon\RR\to \RR$ 
  e una funzione $g\colon\RR\to \RR$ tali che 
  \[
  \lim_{x\to 0} f(x) = 0, \qquad 
  \lim_{x\to 0} g(x) = 0
  \]
  ma
  \[
  \lim_{x\to 0} g(f(x)) = 1.
  \]
  Quale ipotesi nel teorema precedente viene a mancare?
\end{exercise}

\begin{theorem}[limite di funzioni monotòne]%
  \mymark{**}%
  \label{th:limite_monotona}%
Sia $f\colon A \subset \RR \to \RR$ una funzione crescente. 
Se $x_0$ è un punto di accumulazione sinistro per $A$ 
il seguente limite esiste e vale
\[
   \lim_{x\to x_0^-}f(x) = \sup f(\ENCLOSE{x\in A\colon x<x_0})
\]
e se $x_0$ è un punto di accumulazione destro per $A$ 
il seguente limite esiste e vale
\[
   \lim_{x\to x_0^+} f(x) = \inf f(\ENCLOSE{x\in A\colon x>x_0}).
\]
In particolare se $+\infty$ è punto di accumulazione per $A$ 
si ha 
\[
  \lim_{x\to +\infty} f(x) = \sup f(A)
\]
e se $-\infty$ è punto di accumulazione per $A$ si ha 
\[
  \lim_{x\to -\infty} f(x) = \inf f(A).
\]

Gli stessi risultati valgono per le funzioni decrescenti, 
scambiando $\sup$ e $\inf$.
\end{theorem}
%
\begin{proof}\mymark{**}
Supponiamo che $f$ sia crescente e consideriamo il limite 
sinistro $x\to x_0^-$. Può anche essere $x_0=+\infty$, in tal 
caso il limite sinistro $x\to x_0^-$ equivale a $x\to +\infty$.
Poniamo $B=\ENCLOSE{x\in A\colon x<x_0}$
e $\ell=\sup f(B)$ e ricordiamo le proprietà che caratterizzano 
l'estremo superiore
(sappiamo che $B$ non è vuoto in quanto $x_0^-$ per ipotesi 
è un punto di accumulazione per $A$):
\begin{gather*}
  \forall x \in B \colon f(x) \le \ell \\
  \forall y < \ell \colon \exists \alpha \in B \colon f(\alpha) > y.
\end{gather*}
Siccome $f$ è crescente, dalla seconda condizione 
si ottiene che per ogni $x>\alpha$ si ha $f(x)\ge f(\alpha)> y$
e mettendo insieme le due condizioni si ottiene che per ogni $y<\ell$
esiste $\alpha\in\RR$ tale che per ogni $x>\alpha$ si ha $y<f(x)\le \ell$.
Certamente si ha $\alpha < x_0$ perché $\alpha\in B$.   
Dunque la condizione $x>\alpha$ identifica un intorno sinistro 
del punto $x_0$ e si ha dunque $f(x)\to \ell$ per $x\to x_0^-$.

Ragionamento analogo si può fare per il limite destro  
e per le funzioni decrescenti.
\end{proof}

\begin{exercise}
  Utilizzando il teorema precedente si studino le \emph{discontinuità}
  delle funzioni monotone. 
  Si dimostri quindi che l'insieme dei punti in cui una funzione monotona 
  \emph{non} è continua può essere messo in corrispondenza biunivoca 
  con un sottoinsieme di $\QQ$ e dunque tale insieme è numerabile.
\end{exercise}

Il teorema precedente si applica in particolare alle funzioni 
esponenziali, potenze, radici e logaritmi negli estremi dei loro 
domini. Ad esempio è chiaro che se $x_0\in (0,+\infty)$ 
si ha 
\[
  \lim_{x\to x_0} \log_a x = \log_a x_0
\]
in quanto il logaritmo è una funzione continua. 
Se $a>1$ il logaritmo è crescente e la sua immagine è tutto $\RR$ 
dunque si ha 
\[
  \lim_{x\to 0^+} \log_a x = \inf \RR = -\infty
  \qquad
  \lim_{x\to +\infty} \log_a x = \sup \RR = +\infty.
\]
Per l'esponenziale avremo 
\[
 \lim_{x\to x_0} a^x = a^{x_0}
\]
se $x_0\in \RR$ per continuità. 
Se $a>1$ l'esponenziale è crescente 
ed ha immagine $(0,+\infty)$ dunque 
\[
  \lim_{x\to -\infty} a^x = 0, \qquad 
  \lim_{x\to +\infty} a^x = +\infty.
\]
Se $0<a<1$ i limiti all'infinito 
si scambiano in quanto $a^{x}= \frac{1}{a^{-x}}$.
Per le potenze avremo, se $x_0\in[0,+\infty)$ e $\alpha>0$
\[
  \lim_{x\to x_0} x^\alpha = x_0^\alpha
\]
per continuità. Invece essendo $x^\alpha$ crescente e bigettiva 
su $[0,+\infty)$ si avrà 
\[
  \lim_{x\to +\infty} x^\alpha = +\infty.
\]
Se $\alpha<0$ la funzione $x^\alpha$ è continua, decrescente e bigettiva 
su $(0,+\infty)$ e dunque in tal caso:
\[
  \lim_{x\to +\infty} x^\alpha = 0.
\]
Tutti questi limiti \emph{notevoli} possono essere facilmente 
ricordati se teniamo in mente i grafici delle funzioni elementari
Figura~\ref{fig:esponenziale_logaritmo} (a pagina \pageref{fig:esponenziale_logaritmo})
e~\ref{fig:potenza_intera_radice} (a pagina \pageref{fig:potenza_intera_radice}).

Anche la funzione valore assoluto $f(x)=\abs{x}$ è separatamente 
monotona sugli intervalli $[0,+\infty)$ e $(-\infty,0]$. 
Dunque sappiamo che 
\[
  \lim_{x\to +\infty} \abs{x} = \abs{+\infty} = +\infty, 
  \lim_{x\to -\infty} \abs{x} = \abs{-\infty} = +\infty.
\]
Se $x_0\in \RR$ ovviamente si ha pure
\[
  \lim_{x\to x_0} \abs{x_0} = \abs{x_0}
\]
in quanto è banale verificare che la funzione $\abs{x}$ è continua.
Lo stesso vale per la funzione \emph{modulo} $\abs{z}$ quando $z\in \CC$.

Ci sarà anche utile osservare che vale anche questa proprietà 
\[
\lim_{x\to x_0} \abs{f(x)} = 0 \iff 
\lim_{x\to x_0} f(x) = 0
\]
valida perché le definizioni di limite $f(x)\to 0$ e $\abs{f(x)}\to 0$ 
coincidono in quanto $\abs{f(x)-0} = \big\lvert{\abs{f(x)} - 0}\big\rvert$.

\begin{theorem}[permanenza del segno]%
\mymark{***}%
\index{permanenza del segno}%
\index{teorema!della permanenza del segno}%
\mymargin{permanenza del segno}%
Se
\[
  \lim_{x\to x_0} f(x) > 0
\]
allora esiste un intorno $U$ di $x_0$ tale che 
per ogni $x\in U$ si ha $f(x) > 0$.

Viceversa se esiste un intorno $U$ di $x_0$ tale che per ogni $x \in U$  
si ha $f(x)\ge 0$ e se esiste $\ell$ tale che 
\[
    \lim_{x\to x_0} = \ell
\]
allora $\ell\ge 0$.
\end{theorem}
%
Intuitivamente il teorema precedente ci dice che le disuguaglianze strette 
si preservano (permangono) togliendo l'operatore di limite mentre 
le disuguaglianze larghe si preservano passando al limite.
%
\begin{proof}
Sia $\ell\in [-\infty,+\infty]$ il valore del limite.
Se $\ell>0$ esiste certamente un intorno $V$ di $\ell$ 
tale che $V\subset (0,+\infty]$: se $\ell\in \RR$ basta prendere 
$V=(\ell/2,3 \ell/2)$, se $\ell=+\infty$ basta prendere $V=(1,+\infty]$.
Per la definizione di limite esiste $U$ intorno di $x_0$ 
tale che $f(U)\subset V$ e il risultato segue.

Per il viceversa si ragiona per assurdo. 
Se il limite $\ell$ di $f$ esistesse e fosse $\ell < 0$
allora per il punto precedente (applicato a $-f$) dovremmo 
concludere che c'è un intorno di $x_0$ in cui $f(x)<0$.
Ma allora non è possibile che ci sia un intorno in cui $f(x)\ge 0$.
\end{proof}

\begin{theorem}[operazioni con i limiti di funzione]
  \label{th:operazioni_limiti}%
  \index{limite!della somma}%
  \index{limite!del prodotto}%
  \index{limite!del rapporto}%
\mymark{***}%
Se
\[
  \lim_{x\to x_0}f(x) = \ell_1,\qquad
  \lim_{x\to x_0}g(x) = \ell_2
\]
allora si ha
\begin{gather*}
  \lim_{x\to x_0} \enclose{f(x) + g(x)} = \ell_1 + \ell_2, \qquad
  \lim_{x\to x_0} \enclose{f(x) - g(x)} = \ell_1 - \ell_2, \\
  \lim_{x\to x_0} f(x)\cdot g(x) = \ell_1 \cdot \ell_2, \qquad
  \lim_{x\to x_0} \frac{f(x)}{g(x)} = \frac{\ell_1}{\ell_2}
\end{gather*}
sempre che le operazioni utilizzate sul lato destro delle uguaglianze
siano state definite\mynote{%
Si veda la sezione~\ref{sec:reali_estesi}.
I casi in cui le operazioni non sono definite si chiamano 
usualmente \emph{forme indeterminate}
\index{forme indeterminate}
e sono: $+\infty - (+\infty)$, 
$-\infty - (-\infty)$, $+\infty + (-\infty)$, 
$-\infty + (+\infty)$, $0\cdot (+\infty)$, $+\infty \cdot 0$
$0\cdot (-\infty)$, $-\infty \cdot 0$, $\frac 0 0$,
$\frac{+\infty}{+\infty}$, $\frac{-\infty}{-\infty}$,
$\frac{+\infty}{-\infty}$, $\frac{-\infty}{+\infty}$. 
}.
Inoltre se $g(x)\to 0^+$ 
si ha
\[
    \lim_{x\to x_0} \frac{1}{g(x)} = +\infty
\]
(informalmente potremmo scrivere: $\frac{1}{0^+} = +\infty$).

Gli stessi risultati valgono per funzioni di variabile e/o valore complesso.
Anzi i limiti a $\infty$ si comportano meglio, visto che 
$g(z)\to 0$ risulta equivalente a $1/g(z)\to \infty$.

\end{theorem}
%
\begin{proof}
Consideriamo la somma dei limiti
e innanzitutto il caso in cui $\ell_1$ ed $\ell_2$ 
siano entrambi finiti. 
In tal caso
dato un qualunque intorno $U=(\ell-\eps,\ell+\eps)$ 
di $\ell=\ell_1+\ell_2$
se prendiamo gli intorni $U_1=(\ell_1-\eps/2,\ell_1+\eps/2)$ 
e $U_2 = (\ell_2-\eps/2,\ell_2+\eps/2)$ si possono trovare 
degli intorni $V_1$ e $V_2$ di $x_0$ per cui si ha 
$f(V_1\setminus\ENCLOSE{x_0}) \subset U_1$ e 
$f(V_2\setminus \ENCLOSE{x_0}) \subset U_2$
e a maggior ragione questo risulta se prendiamo $V=V_1\cap V_2$ 
al posto di $V_1$ e $V_2$. 
Visto che $U_1+U_2 = U$ si avrà allora 
$(f+g)(V\setminus\ENCLOSE{x_0})\subset U$.

Sempre nel caso della somma 
se $\ell_1=+\infty$ e $\ell_2\neq -\infty$
allora $\ell_1+\ell_2=+\infty$ e 
dato un intorno di $+\infty$ 
della forma $U=(\alpha,+\infty]$ con $\alpha>0$
possiamo prendere 
$U_1 = (2\alpha,+\infty)$ e $U_2 = (\ell_2-\alpha,\ell_2+\alpha)$ 
se $\ell_2\in \RR$ oppure $U_2 = (\alpha,+\infty)$ se $\ell_2=+\infty$.
In ogni caso si ha $U_1+U_2 = U$ e si procede quindi come nel caso precedente.
Il caso $\ell_1=-\infty$ si dimostra in maniera del tutto analoga.

Per la differenza basta osservare che $f(x)-g(x) = f(x) + (-g(x))$.
Se $\ell_2$ è finito la continuità della funzione $h(y)=-y$
ci garantisce che $-g(x)\to -\ell_2$.
Lo stesso si verifica facilmente quando $\ell_2$ è infinito 
(basta osservare che se $U=(\alpha,+\infty]$ è un intorno di $+\infty$
allora $-U = [-\infty,-\alpha)$ è un intorno di $-\infty$). 
Dunque il limite della differenza si riconduce al limite della somma.

Per quanto riguarda il prodotto ci ricordiamo che grazie 
al teorema~\ref{th:isomorfismo} il gruppo additivo totalmente 
ordinato $\RR = \openinterval{-\infty}{+\infty}$ 
è isomorfo al gruppo moltiplicativo $\RR_+ = \openinterval 0 {+\infty}$
e di conseguenza $\bar \RR = \closeinterval{-\infty}{+\infty}$
corrisponde a $\closeinterval {0^+}{+\infty}$
(l'isomorfismo $\RR_+\to \RR$ è dato dalla funzione $\log_a x$ con $a>1$ 
fissato,
tale funzione preserva gli intorni dei punti corrispondenti salvo il fatto 
che gli intorni di $-\infty$ si trasformano in intorni destri di $0$).

Dunque se $f > 0$ e $g > 0$ i risultati per il limite del prodotto 
sono conseguenza dei risultati analoghi per il limite della somma.
Cambiando segno ad una delle due funzioni o ad entrambe ci si può 
ricondurre al caso $f > 0$ e $g > 0$ quando le due funzioni assumono 
un segno costante in almeno un intorno del punto $x_0$.
Per il teorema della permanenza del segno questo è vero se 
$\ell_1\neq 0$ e $\ell_2\neq 0$: il limite del prodotto in questo caso 
è dunque $\ell_1\cdot \ell_2$.
Se invece $\ell_1=0$ e $\ell_2\in\RR$ sappiamo che $f(x)\to 0$ 
è equivalente a $\abs{f(x)}\to 0$ dunque in tal caso si può rimpiazzare $f(x)$ con $\abs{f(x)}\ge 0$
(gli eventuali punti in cui $f(x)=0$ sono irrilevanti e possono essere tolti 
dal dominio di $f$ per garantire $\abs{f}>0$) e ottenere che il limite 
del prodotto è $0$.
Lo stesso accade se $\ell_2=0$ con la funzione $g(x)$.

Per quanto riguarda il rapporto il teorema~\ref{th:isomorfismo}
ci dice che le proprietà additive dell'opposto diventano (tramite logaritmo)
proprietà moltiplicative del reciproco, almeno se siamo nell'ambito di numeri positivi.
Dunque se $f(x)\to \ell$ con $\ell>0$ 
allora $\frac{1}{f(x)} \to \frac{1}{\ell}$.
Se $\ell=0$ non possiamo concludere niente in quanto le proprietà 
additive a $-\infty$ si rispecchiano nelle proprietà moltiplicative 
in $0^+$, non in $0$. 
Dunque i risultati sul limite del rapporto discendono dai risultati 
sul limite del prodotto con il reciproco.
\end{proof}


Se dobbiamo calcolare un limite in cui compare un elevamento 
a potenza $f(x)^{g(x)}$ con base ed esponente variabile converrà 
scrivere
\[
 f(x)^{g(x)} = a^{g(x)\cdot \log_a f(x)}
\]
per poter applicare il teorema precedente al prodotto $g(x)\cdot \log_a f(x)$.

\begin{example}
  Si calcoli
  \[
  \lim_{x\to +\infty} \frac{2+x^2-x^3}{(x^2-x+2)^2}.
  \]
  \end{example}
  \begin{proof}[Svolgimento.]
  Per $x\to +\infty$ per il teorema sul limite del
  prodotto sappiamo che $x^2\to +\infty$ e
  $x^3\to +\infty$.
  Per il teorema sulla somma dei limiti
  sappiamo che $2+x^2\to +\infty$ ma
  a priori non possiamo applicare tale teorema
  al limite di $(2+x^2)-x^3$ in quanto
  $(+\infty)-(+\infty)$ è una forma indeterminata
  (non rientra nelle ipotesi di quel teorema).
  Bisogna allora intuire che le potenze di $x$
  con esponente maggiore sono preponderanti e vanno
  quindi messe in evidenza tramite
  manipolazioni algebriche. L'espressione
  di cui vogliamo calcolare il limite
  si può quindi riscrivere in questo modo:
  \begin{align*}
  \frac{2+x^2-x^3}{(x^2-x+2)^2}
  &= \frac{x^3 \enclose{\frac 2 {x^3}+\frac{1}{x}-1}}{x^4\enclose{1-\frac 1 x + \frac{2}{x^2}}^2}
  = \frac{1}{x}\cdot \frac{\frac 2 {x^3}+\frac{1}{x}-1}{\enclose{1-\frac 1 x + \frac{2}{x^2}}^2}.
  \end{align*}
  A questo punto l'espressione non presenta più forme indeterminate.
  Applicando i teoremi precedenti possiamo allora affermare che
  si ha
  \begin{align*}
  \lim_{x\to+\infty}\frac{1}{x}\cdot \frac{\frac 2 {x^3}+\frac{1}{x}-1}{\enclose{1-\frac 1 x + \frac{2}{x^2}}^2}
  &= \frac{1}{+\infty} \cdot \frac{\frac 2 {(+\infty)^3}+\frac{1}{+\infty}-1}{\enclose{1-\frac 1 {+\infty} + \frac{2}{(+\infty)^2}}^2}\\
  &= 0 \cdot \frac{0+0-1}{(1-0+0)^2} = 0\cdot (-1) = 0.
  \end{align*}
\end{proof}

L'aver definito le operazioni sulle quantità infinite
risulta in effetti comodo nello svolgimento dei limiti.
Bisogna però essere consapevoli che $\bar \RR$ non è un campo
e quindi le operazioni con i simboli $+\infty$ e $-\infty$
non rispettano molte delle regole che siamo abituati
ad avere sui numeri finiti.
Sarà quindi opportuno ricondursi immediatamente ad una espressione
che abbia senso in $\RR$, sulla quale potremo
applicare le manipolazioni algebriche con più tranquillità.
Per questo motivo in molti testi l'espressione intermedia in cui compaiono 
le operazioni eseguite sulle quantità $+\infty$ e $-\infty$ non è ritenuta valida
e si preferisce scrivere direttamente il risultato.
  
\begin{example}
Si calcoli il seguente limite nel campo complesso:
\[
\lim_{z\to 0} \enclose{\frac 1 {z} - \frac 1 {z^2-z}}. 
\]
\end{example}
\begin{proof}
  Anche in questo caso non possiamo applicare direttamente le regole 
  di calcolo del limite in quanto otterremmo la forma indeterminata 
  $\infty - \infty$.
  Possiamo però operare una semplice manipolazione algebrica
  per eliminare l'indeterminazione:
  \[
    \frac 1 {z} - \frac 1 {z^2-z}
    = \frac{z- 1 - 1}{z^2-z}
    = \frac{z-2}{z^2-z}
    \to \frac{0-2}{0^2-0} = \frac{-2}{0} = \infty 
    \qquad \text{per $z\to 0.$}
  \]
\end{proof}

\begin{comment} % non abbiamo ancora gli strumenti per fare questi esercizi
\begin{exercise}
  Calcolare 
  \[
  \lim_{x\to 0^+} x^x.
  \]
\end{exercise}

\begin{exercise}
  Trovare un esempio di funzioni $f(x)$ e $g(x)$ tali che 
  \[
     \lim_{x\to 0} f(x) = 0, \qquad 
     \lim_{x\to 0} g(x) = 0
  \]
  ma 
  \[
    \lim_{x\to 0} f(x)^{g(x)} \neq 1.
  \]
\end{exercise}
\end{comment}

\subsection{proprietà frequenti e definitive}

\begin{definition}[proprietà frequenti e definitive]
Diremo che un predicato $P(x)$ definito su un insieme $A\subset \RR$ 
di cui $x_0$ è punto di accumulazione vale 
\emph{definitivamente}%
\mymargin{definitivamente}%
\index{definitivamente} per $x\to x_0$ se
vale in un intorno di $x_0$ ovvero:
\[
  \exists U\in \B_{x_0}\colon \forall x \in A\cap U\setminus\ENCLOSE{x_0}\colon P(x).
\]
Diremo che $P(x)$ vale \emph{frequentemente}%
\mymargin{frequentemente}%
\index{frequentemente} per $x\to x_0$
se in ogni intorno di $x_0$ c'è almeno un punto $x\neq x_0$ 
in cui vale:
\[
  \forall U\in \B_{x_0}\colon \exists x\in A\cap U\setminus\ENCLOSE{x_0}
  \colon P(x).
\]
\end{definition}

Chiaramente se una proprietà vale definitivamente vale anche frequentemente
infatti se c'è un intorno $U$ su cui vale la proprietà per ogni altro intorno 
$V$ la proprietà risulta valida su $U\cap V$ che non è mai vuoto.
Se una proprietà vale frequentemente significa in particolare che vale per 
infiniti valori diversi in quanto se vale in punto $x\neq x_0$ 
posso sempre trovare un intorno $V$ di $x_0$ che non contiene $x$
e in tale intorno trovo un ulteriore punto in cui vale la proprietà. 
Iterando il procedimento posso trovare infiniti punti diversi su cui 
la proprietà è valida.

Le due proprietà sono complementari nel senso che
in base alle proprietà dei quantificatori logici vale 
la seguente relazione:
\[
  \text{non frequentemente $P(x)$} \iff
  \text{definitivamente non $P(x)$}
\]

Se due proprietà $P(x)$ e $Q(x)$ valgono definitivamente allora anche
$P(x)\land Q(x)$ vale definitivamente. Se invece valgono entrambe
frequentemente allora anche $P(x) \lor Q(x)$ vale frequentemente.

\begin{example}
La proprietà $x^3 - x > 1000 x^2 + 1$
vale definitivamente per $x\to +\infty$.
La proprietà $x>0$ vale frequentemente per $x\to 0$.
\end{example}


La definizione di limite $f(x) \to \ell$ per $x\to x_0$ 
potrebbe quindi enunciarsi così:
per ogni intorno $U$ di $\ell$ si ha $f(x)\in U$ definitivamente.
E la sua negazione è: esiste un intorno $U$ di $\ell$ per cui
frequentemente $f(x)\not\in U$.


\subsection{criteri di confronto}

\begin{theorem}[criteri di confronto]
\label{th:confronto}%
\index{teorema!del confronto}%
\index{limite!confronto}%
\index{teorema!dei due carabinieri}%
\index{carabineri!teorema dei}%
\mymark{***}%
Siano 
$f$, $g$, $h$ tre funzioni reali%
\mynote{Non è possibile fare confronti tra valori complessi, visto che
sui numeri complessi non abbiamo un ordinamento}
definite su uno stesso dominio $A\subset \RR$ 
con punto di accumulazione $x_0$
\mymargin{confronto tra limiti}%
\index{confronto tra limiti}
\begin{enumerate}
\item
Se per ogni $x\in A$ si ha
\[
f(x) \le g(x)
\]
e se entrambe le funzioni ammettono limite: $f(x) \to \ell_1$ 
e $g(x) \to \ell_2$ per $x\to x_0$
allora
\[
\ell_1 \le \ell_2.
\]

\item
Se per ogni $x\in A$ si ha:
\[
f(x) \le g(x)
\]
e se $f(x)\to +\infty$ allora anche $g(x) \to +\infty$ per $x\to x_0$.
Viceversa se $f(x) \le g(x)$ e $g(x) \to -\infty$ allora anche $f(x) \to -\infty$.

\item
(teorema dei carabinieri)
Se per ogni $x\in A$ vale
\mymargin{teorema dei carabinieri}%
\index{teorema!dei carabinieri}%
\[
f(x) \le h(x) \le g(x)
\]
 e se le due
funzioni $f$ e $g$ hanno lo stesso limite: $f(x) \to \ell$ e $g(x)\to \ell$
per $x\to x_0$
allora anche $h(x) \to \ell$.
\end{enumerate}
\end{theorem}
%
\begin{proof}
\mymark{**}
\begin{enumerate}
\item
Se per assurdo fosse $\ell_1 > \ell_2$
la funzione $f(x)-g(x)$ avrebbe limite positivo e per il teorema
della permanenza del segno dovrebbe essere definitivamente positiva. 
Ma questo contraddice l'ipotesi $f(x)-g(x)\le 0$. 

\item 
Se non fosse $g(x)\to +\infty$ significa che 
esiste $\alpha \in \RR$ tale che $g(x)<\alpha$ frequentemente. 
Ma visto che definitivamente $f(x)>\alpha$ troviamo 
che frequentemente si ha $f(x)>g(x)$ in contrasto con l'ipotesi 
$f(x)\le g(x)$.

\item
Se $f$ e $g$ hanno lo stesso limite $\ell$ significa che per ogni
$U$ intorno di $\ell$ si ha definitivamente $f(x)\in U$ e $g(x)\in U$.
Visto che $U$ è un intervallo anche $h(x)$ è definitivamente in $U$ 
e quindi $h(x)\to \ell$.
\end{enumerate}
\end{proof}

\begin{example}
  Dimostrare che 
  \[
    \lim_{x\to +\infty} \enclose{2x - \lfloor x\rfloor} = +\infty. 
  \]
\end{example}
\begin{proof}[Svolgimento.]
  Sappiamo che $\lfloor x\rfloor \le x$ dunque 
  \[
    2x - \lfloor x \rfloor \ge 2x -x = x \to +\infty.  
  \]
  Dunque per confronto si ottiene il risultato richiesto.
\end{proof}

\begin{corollary}[limitata per infinitesima]%
\label{cor:limitata_per_infinitesima}%
\mymark{**}%
  Se $f(x)$ è una funzione limitata e $g(x)\to 0$ per 
  $x\to x_0$ allora anche
  \[
    f(x)\cdot g(x) \to 0 \qquad \text{per $x\to x_0.$} 
  \]
\end{corollary}
%
\begin{proof}
  Se $f$ è limitata significa che esiste $R>0$ tale che $\abs{f(x)}\le R$
  per ogni $x$ nel dominio di $f$.
  Ma allora 
  \[
     0\le \abs{f(x)\cdot g(x)} = \abs{f(x)}\cdot \abs{g(x)} 
     \le R \abs{g(x)} \to R\cdot 0 = 0.
  \]
  Per confronto deduciamo che $\abs{f(x)\cdot g(x)}\to 0$ 
  che è equivalente alla tesi.
\end{proof}

\begin{exercise}
  Mostrare che 
  \[
   \lim_{x\to +\infty} \frac{x-\lfloor x \rfloor}{x} = 0.  
  \]
\end{exercise}
%\chapter{funzioni continue}

\begin{comment}
Lo scopo principale di questo corso è quello di studiare le
funzioni con dominio e codominio nei numeri reali:
\[
  f \colon A \subset \RR \to \RR.
\]
Ad esempio per ogni $n\in \ZZ$ abbiamo già definito la funzione
potenza:
\[
  f(x) = x^n.
\]
Se $n\ge 0$ si ha $f\colon \RR \to \RR$ (la funzione è definita
per ogni $x\in \RR$), se invece $n<0$ si ha
$f\colon \RR\setminus\ENCLOSE{0}\to \RR$ in quanto le potenze
con esponente negativo non sono definite se la base è $0$.
\end{comment}

\section{limiti}

Se una funzione non è definita in un punto potremmo comunque 
essere interessati a capire se, come per le funzioni continue,
intorno a quel punto i valori assunti dalla funzione si avvicinano 
ad un determinato valore.

\begin{definition}[limite finito]
  \label{def:limite_finito}%
  \index{limite!definizione}%
  Sia $f\colon A\subset \RR \to \RR$ una funzione, 
  $x_0\in \RR$ e $\ell \in \RR$.
  Diremo che $f(x)$ tende a $\ell$ (oppure: ha limite $\ell$) 
  quando $x$ tende a $x_0$ 
  e scriveremo
  \[
    f(x) \to \ell \qquad \text{per $x\to x_0$}
  \]
  se 
  \begin{equation}\label{eq:limite_finito}
    \forall \eps>0\colon \exists \delta>0 \colon 
    \forall x \in A\setminus\ENCLOSE{x_0}\colon
    \abs{x-x_0}<\delta \implies \abs{f(x)-\ell}< \eps.
  \end{equation}

  (Stessa identica definizione può essere data per funzioni 
  di variabile complessa e/o valori complessi.)
\end{definition}

Possiamo osservare che la definizione è molto simile alla 
definizione~\ref{def:continua}. 
In effetti confrontando le due definizioni si deduce che 
la condizione $f(x)\to \ell$ per $x\to x_0$ 
è equivalente a richiedere che l'estensione $\tilde f \colon A \cup \ENCLOSE{x_0}\to \RR$ 
definita da 
\[
 \tilde f(x) = \begin{cases}
    f(x) & \text{se $x\neq x_0$}\\ 
    \ell & \text{se $x=x_0$}
 \end{cases}
\]
sia continua nel punto $x_0$. 
Si osservi che non importa che la funzione $f$ sia definita nel punto 
$x_0$ e se è definita il valore $f(x_0)$ è irrilevante nella 
definizione di limite.

\begin{example}
  Si ha 
  \[
  \frac{\sqrt{x}-1}{x-1} \to \frac 1 2 \qquad \text{per $x\to 1$}. 
  \]
  Infatti, moltiplicando numeratore e denominatore 
  per $\sqrt x + 1$ si nota che se $x\ge 0$, $x\neq 1$ si ha:
  \[
      \frac{\sqrt x-1}{x-1}
      = \frac{x-1}{(x-1)(\sqrt x +1)}
      = \frac{1}{\sqrt x + 1}.
  \]
  Osserviamo che la funzione a lato sinistro è definita sull'insieme 
  $A = [0,1)\cup(1,+\infty)$ mentre la funzione a lato
  destro è definita su tutto l'intervallo $[0,+\infty)$.
  Inoltre quest'ultima funzione è continua 
  (in quanto la radice è una funzione continua e somma e rapporto di funzioni 
  continue è una funzione continua) che assume il valore $\frac 1 2$
  quando $x=1$. 
  Dunque la funzione sinistra pur non essendo definita per $x=1$ 
  ha limite $\frac 1 2$ per $x\to 1$.
\end{example}

Sarà molto utile estendere il concetto di limite 
ai punti all'infinito della retta reale estesa (o del piano complesso).
Per fare ciò la definizione di limite che abbiamo dato deve 
essere opportunamente modificata a seconda che la variabile o il valore 
del limite o entrambi siano finiti o infiniti (distinguendo poi $+\infty$ 
da $-\infty$). 
Molti altri casi devono essere poi presi in considerazione 
per introdurre i limiti destro e sinistro.
Per rendere la trattazione più compatta e omogenea useremo 
quindi l'approccio utilizzato nella \myemph{topologia} che è 
il campo della matematica che si occupa di limiti e continuità nella 
massima generalità possibile. 
Faremo questo senza definire cos'è la \emph{topologia} 
(cosa che ci porterebbe troppo lontano) ma semplicemente 
utilizzando la stessa terminologia in maniera trasparente.

\begin{definition}[intorno]
Sia $U\subset\bar \RR$. Andremo a definire cosa 
significa che $U$ è un \myemph{intorno} di $x_0$
per ogni $x_0\in \bar \RR$.

Se $x_0\in \RR$ diremo che $U$ è un intorno di $x_0$ 
quando vale questa condizione:
\[
\exists \eps>0 \colon \forall x\in \RR\colon 
\abs{x-x_0} < \eps \implies x\in U.
\]
Diremo invece che $U$ è un intorno di $+\infty$ 
se 
\[
 \exists M>0 \colon \forall x\in \bar \RR \colon 
 x > M \implies x\in U;
\]
e analogamente diremo che $U$ è un intorno di $-\infty$
se 
\[
 \exists M>0 \colon \forall x\in \bar \RR\colon 
 x < -M \implies x \in U.  
\]

Dato $x_0\in \bar \RR$ denotiamo con
\[
  \mathcal U_{x_0} = \ENCLOSE{U\in \mathcal P(\bar \RR)
  \colon \text{$U$ è un intorno di $x_0$}}
\]
la famiglia di tutti i suoi intorni.

(Definizioni analoghe possono essere date sul piano complesso: al finito 
la definizione è formalmente la stessa; gli intorni di $\infty\in \bar \CC$ 
sono gli insiemi $U\subset \bar \CC$ per cui esiste $M>0$ tale che per ogni $z\in \bar \CC$ 
se $\abs{z}>M$ si ha $x\in U$.)
\end{definition}

\begin{definition}[limite generale]
  Sia $f\colon A\subset \bar \RR \to \bar \RR$ una funzione, 
  sia $x_0\in \bar \RR$ e $\ell\in \bar \RR$. 
  Scriveremo 
  \[
    f(x)\to \ell\qquad\text{per $x\to x_0$}
  \]
  se vale la seguente condizione:
  \begin{equation}\label{def:limite}
  \forall U\in \mathcal U_\ell\colon 
  \exists V\in \mathcal U_{x_0}\colon 
  \forall x\in A\setminus\ENCLOSE{x_0}\colon 
  x\in V \implies f(x) \in U.
  \end{equation}
\end{definition}

Osserviamo innanzitutto che questa definizione è equivalente 
alla definizione~\ref{def:limite_finito}.
Infatti la proprietà~\eqref{eq:limite_finito} può essere riscritta 
così:
\[
\forall \eps>0\colon \exists \delta>0\colon 
\forall x\in A\setminus\ENCLOSE{x_0}\colon 
 x \in (x_0-\delta,x_0+\delta) \implies f(x) \in (\ell-\eps,\ell+\eps).
\]
Supponiamo che quest'ultima proprietà sia valida e dimostriamo che allora 
anche 


\section{successioni}
\label{sec:successioni}

Una \emph{successione}%
\mymargin{successione}%
\index{successione} in un insieme $X$ è una
funzione $\vec a\colon \NN \to X$.
L'insieme delle funzioni $\NN \to X$ viene usualmente indicato
con $X^\NN$ e potremmo dunque scrivere $\vec a \in X^\NN$. 
In effetti una successione $\vec a$ può essere interpretata
come una sequenza infinita di elementi di $X$:
\mynote{%
Utilizziamo il grassetto per evidenziare il fatto
che $\vec a$ non è un numero ma una \emph{lista} (o vettore) di numeri.
Nella scrittura a mano non è possibile usare il grassetto, ma potremmo
sottolineare il nome della variabile scrivendo $\underline a$ invece che $\vec a$.
In alternativa si potrebbe scrivere $\stackrel{\rightarrow}a$ oppure evitare
qualunque distinzione e scrivere più semplicemente $a$.}
\[
  \vec a = (a_0, a_1, a_2, \dots, a_n, \dots )
\]
dove si intende
\[
   a_n = \vec a(n).
\]
Le componenti $a_n$ si chiamano \emph{termini} della successione.
I numeri $n$ si chiamano, invece, \emph{indici}.
L'intera
successione $\vec a$ può essere indicata con $(a_n)_{n=0}^\infty$
oppure $(a_n)_n$ oppure,
più semplicemente, con $a_n$ quando sia chiaro che si intende l'intera
successione $\vec a$ e non un singolo termine della stessa.%

Per noi il caso più interessante sarà quello delle \emph{successioni reali}
$a_n\in \RR$ cioè il caso $X=\RR$. Considereremo però anche il caso di
\emph{successioni complesse} $a_n\in \CC$ (dunque $X=\CC$) perché questo
potrà essere utile in alcune situazioni.%
\mynote{In alcuni testi le successioni vengono indicate
con la notazione $\ENCLOSE{a_n}$ che però è fuorviante in quanto
una successione in $X$ non è semplicemente un sottoinsieme di $X$:
l'ordine in cui vengono presi gli elementi è rilevante.}

\subsection{limite di successione}

Visto che $\vec a \colon \NN\to\RR$ (o in $\CC$) è una funzione 
con dominio $\NN\subset \RR$ 
e visto che $+\infty$ è l'unico punto di accumulazione di $\NN$ 
sarà possibile considerare il limite
\[
  \ell = \lim_{n\to +\infty} \vec a(n)
\]
che potremmo anche scrivere con una delle seguenti notazioni:
\[
\ell = \lim \vec a = \lim_n a_n = \lim_{n\to +\infty} a_n
\]
o più concisamente:
\[
  a_n \to \ell. 
\]

Esplicitando la definizione astratta di limite osserviamo che se $(\alpha,+\infty]$
è un intorno di $+\infty$ con $\alpha>0$ la sua intersezione con $\NN$ (il dominio della funzione)
è l'insieme $\ENCLOSE{n\in \NN\colon n> N}$ dove $N=\lfloor \alpha \rfloor$ è 
un numero naturale. Dunque se $\ell \in \RR$ 
la condizione $a_n \to \ell$ si esplicita nel modo seguente:
\begin{equation}\label{eq:limite_successione_convergente}
\forall \eps>0\colon \exists N\in \NN\colon n>N\implies \abs{a_n-\ell} < \eps.  
\end{equation}
La condizione $a_n\to +\infty$ si scrive
\[
  \forall \alpha\in \RR \colon \exists N\in \NN \colon n>N \implies a_n > \alpha  
\]
e infine $a_n\to-\infty$ diventa
\[
  \forall \alpha\in \RR \colon \exists N\in \NN \colon n>N \implies a_n < -\alpha.    
\]

\begin{definition}[carattere di una successione]
Sia $a_n$ una successione a valori reali o complessi
e si consideri il limite 
\[
  \lim_n a_n.  
\]
Se tale limite esiste diremo che la successione $a_n$ è 
\emph{regolare}%
\mymargin{regolare}%
\index{regolare},
\index{successione!regolare}%
se il limite esiste ed è finito diremo che la successione è 
\emph{convergente}%
\mymargin{convergente}%
\index{convergente}%
\index{successione!convergente}
se il limite è infinito diremo che la successione è 
\emph{divergente}%
\mymargin{divergente}%
\index{divergente}%
\index{successione!divergente}
se il limite non esiste diremo infine che la successione è
\emph{indeterminata}%
\mymargin{indeterminata}%
\index{indeterminato}%
\index{successione!indeterminata}.

Abbiamo dunque le seguenti alternative
\begin{enumerate}
 \item la successione è convergente (ha limite finito);
 \item la successione è divergente (ha limite infinito);
 \item la successione è indeterminata (non ha limite).
\end{enumerate}
Determinare il \emph{carattere}
\mymargin{carattere di una successione}%
\index{carattere!di una successione}%
\index{successione!carattere}%
di una successione
significa specificare a quale delle tre categorie appartiene.
\end{definition}

\begin{exercise}
Una successione complessa $z_n\in \CC$ potrà essere scritta
nella forma $z_n = x_n + i y_n$ dove $x_n$ e $y_n$ sono successioni
reali. Si può allora verificare che $z_n\to z$ per $z\in \CC$ se e solo se
$x_n\to x$ e $y_n\to y$ dove $z=x+ iy$ con $x,y\in \RR$.
\end{exercise}

Si faccia attenzione al fatto che una successione $a_n\in \RR \subset \CC$
può essere interpretata sia come successione reale che come successione complessa.
Le condizioni di convergenza in $\RR$ o $\CC$ sono allora equivalenti. Ma
se $a_n \to \infty$ (in $\bar \CC$) significa che $\abs{a_n}\to +\infty$
e può capitare che $a_n$ non abbia limite né $+\infty$
né $-\infty$ in quanto potrebbe frequentemente cambiare di segno
(un esempio è $a_n = (-n)^n$).

In generale potremmo avere delle successioni definite solamente 
su un sottoinsieme $\NN$ dei numeri naturali. 
Finché la successione è definita in un intorno di $+\infty$ (cioè 
da un certo $n_0$ in poi) questo non ha nessuna conseguenza sul 
limite per $n\to +\infty$ (grazie alla località del limite, 
teorema~\ref{th:localita_limite}).

\begin{example}
  La successione $a_n = 1/n$ è definita per $n\in \NN$ ma $n\neq 0$.
  Ciò non toglie che possiamo studiarne il limite come qualunque altra
  successione. Per evidenziare il fatto che il primo indice è $n=1$
  si potrà usare la notazione $(a_n)_{n=1}^\infty$.
\end{example}
 
\subsection{successioni limitate}
\label{sec:successione_limitata}%

Se abbiamo una successione $a_n$ reale possiamo anche considerare gli operatori:
\[
  \sup_n a_n, \qquad 
  \inf_n a_n, \qquad 
  \max_n a_n, \qquad 
  \min_n a_n 
\]
che sono definiti come per qualunque altra funzione. 
Sappiamo che $\sup$ e $\inf$ esistono sempre (in $\bar\RR$) mentre $\max$ e $\min$
potrebbero non esistere.

Anche il concetto di limitatezza per una successione è ereditato dallo stesso 
concetto valido per le funzioni.
Una successione $a_n$ (a valori reali o complessi) si dice essere 
\emph{limitata}%
\mymargin{limitata}%
\index{limitato} se 
l'insieme $\{\abs{a_n}\colon n \in \NN\}$ è limitato. Cioè se 
\[
\sup_n \abs{a_n} < +\infty.  
\]
\mynote{si noti che anche quando $a_n$ è complessa la successione 
dei moduli $\abs{a_n}$ è reale (e non negativa).
Vedremo nel seguito che le successioni a valori complessi non creano 
molte più difficoltà di quanto ne creino le successioni che assumono 
valori negativi. 
Il vero vantaggio si avrà con le successioni a valori positivi, 
per questo spesso si considera il modulo di una successione.
}
In caso contrario la successione si dice essere \emph{illimitata}%
\mymargin{illimitata}%
\index{illimiatato}.
Queste definizioni si applicano tali e quali alle successioni di numeri 
complessi (rimpiazzando il valore assoluto con il modulo).

Se la successione ha valori reali potremo anche parlare di 
\emph{limitatezza superiore} quando $\sup a_n <+\infty$ 
e \emph{limitatezza inferiore} quando $\inf a_n > -\infty$.


\begin{example}
Si consideri $a_n = \frac{1}{n+1}$ definita per $n\in \NN$.
Allora
\[
  \sup a_n = \max a_n = 1, \qquad
  \inf a_n = 0, \qquad \text{non esiste }\min a_n.
\]
\end{example}
\begin{proof}
Per ogni $n \in \NN$ si ha $n+1\ge 1$ e quindi $a_n = 1/(n+1) \le 1$.
Visto poi che $a_0 = 1$ si ottiene immediatamente che $\max a_n = 1$
e di conseguenza $\sup a_n = 1$.

Per verificare che $\inf a_n = 0$ dobbiamo verificare innanzitutto
che $0$ è minorante, e questo è vero in quanto $a_n = 1/(n+1)> 0$ essendo $n+1\ge 1 \ge 0$.
Inoltre dobbiamo verificare che per ogni $\eps >0$ esiste $n\in \NN$ tale
che $a_n < 0 + \eps = \eps$. Questo succede se $1/(n+1) < \eps$ ovvero
se $n > 1/\eps -1$ ad esempio per $n=\lceil 1/\eps\rceil$.
Abbiamo dunque verificato che $\inf a_n = 0$.
Il minimo di $a_n$ non esiste perché se esistesse dovrebbe essere uguale
all'estremo inferiore cioè dovrebbe essere $0$. Ma questo è impossibile
perché per ogni $n\in \NN$ si ha $a_n = 1/(n+1)\neq 0$.
\end{proof}

Visto che l'insieme $\NN$ su cui sono definite le successioni 
ha un unico punto di accumulazione $+\infty$ se la successione 
è convergente non potrà essere illimitata, come viene espresso nel seguente.

\begin{theorem}[limitatezza delle successioni convergenti]
\mymark{**}%
Se una successione (reale o complessa) è convergente allora è anche limitata.
Se una successione reale diverge a $+\infty$ allora è inferiormente limitata,
se diverge a $-\infty$ è superiormente limitata.
\end{theorem}
%
\begin{proof}
  Per la definizione di limite~\eqref{eq:limite_successione_convergente} 
  se $a_n\to \ell$ con $\ell$ finito sappiamo 
  che scelto $\eps=1$ dovrà esistere $N\in \NN$ tale che 
  \[
      \forall n>N\colon \abs{a_n - \ell} < 1.
  \]
  Osserviamo ora che gli indici $n$ per cui $n\le N$ sono in numero finito 
  (in effetti sono $N+1$) e quindi esiste
  \[
   M = \max_{n\le N} \abs{a_n} 
   = \max \ENCLOSE{\abs{a_0}, \abs{a_1}, \dots , \abs{a_N}}.  
  \]
  Dunque scelto $R=\max\ENCLOSE{\abs{\ell}+1,M}$ 
  se $n\le N$ si ha certamente $\abs{a_n}\le M \le R$
  e se $n>N$ si ha comunque, per disuguaglianza triangolare,
  \[
  \abs{a_n} \le \abs{a_n-\ell} + \abs{\ell} < 1+ \abs{\ell} \le R.  
  \]
  Dunque nel complesso risulta $\sup_n \abs{a_n}\le R$ e la successione 
  è limitata.

  Se $a_n\to +\infty$ per la definizione di limite 
  scelto $(0,+\infty]$ come intorno di $+\infty$ 
  deve esistere un indice $N\in \NN$ tale che se $n>N$ si ha $a_n>0$. 
  Ma allora posto $R=\min\ENCLOSE{0, a_0,\dots,a_N}$
  si avrà $a_n\ge R$ per ogni $n\in \NN$ e dunque la successione 
  risulta essere inferiormente limitata.

  Dimostrazione analoga si fa nel caso $a_n\to -\infty$.
\end{proof}


% \begin{theorem}[prodotto di limitata per infinitesima]
% \mymark{**}%
% \mymargin{prodotto limitata per infinitesima}%
\index{prodotto limitata per infinitesima}%
% Se $a_n$ è una successione limitata e $b_n\to 0$ allora
% $a_n\cdot b_n \to 0$ (il risultato vale sia per successioni reali 
% che per successioni complesse).
% \end{theorem}
% %
% \begin{proof}
%   Se $a_n$ è limitata significa che esiste $R>0$ tale che $\abs{a_n}\le R$ per
%   ogni $n\in \NN$.
%   Se $b_n\to 0$ sappiamo che anche $\abs{b_n}\to 0$, dunque
%   \[  
%     0
%     \le \abs{a_n\cdot b_n} 
%     = \abs{a_n}\cdot \abs{b_n} 
%     \le R\cdot \abs{b_n}\to R\cdot 0 = 0.
%   \]
%   Per il teorema dei due carabinieri deduciamo che $\abs{a_n\cdot b_n}\to 0$ 
%   e quindi $a_n\cdot b_n \to 0$.
% \end{proof}

\subsection{successioni monotòne}

Le definizioni~\ref{def:monotonia} si applicano alle successioni. 
Abbiamo quindi già definito cosa significa che una successione è 
crescente, decrescente, strettamente crescente, strettamente decrescente,
monotona, costante. Ad esempio, una successione $a_n$ si dice essere 
crescente se $n\ge m$ implica $a_n \ge a_m$.

La verifica della monotonia di una successione può essere però dimostrata 
più facilmente rispetto al caso generale delle funzioni in quanto 
possiamo utilizzare il principio di induzione, come mostrato 
nel seguente teorema.


\begin{theorem}[successioni monotòne]
\mymark{***}
Una successione $a_n$ è
\begin{enumerate}
\item \emph{crescente}: se per ogni $n\in \NN$ si ha $a_{n+1} \ge a_n$;
\item \emph{decrescente}: se per ogni $n\in \NN$ si ha $a_{n+1} \le a_n$;
\item \emph{strettamente crescente}: se per ogni $n\in \NN$ si ha $a_{n+1}>a_n$;
\item \emph{strettamente decrescente}: se per ogni $n\in \NN$ si ha
$a_{n+1}<a_n$;
\end{enumerate}
\end{theorem}
%
\begin{proof}
  Consideriamo ad esempio la prima condizione (funzione crescente).
Ovviamente se $a_n$ è crescente si ha $a_{n+1}\ge a_n$ in quanto $n+1 > n$.
Viceversa supponiamo che per ogni $n\in \NN$ si abbia $a_{n+1}\ge a_n$.  Allora
chiaramente $a_1\ge a_0$, $a_2\ge a_1 \ge a_0$, $a_3 \ge a_2 \ge a_1 \ge a_0$...
Per induzione si può quindi dimostrare che  $a_n \ge a_{n-1} \ge \dots \ge a_2
\ge a_1$.

Gli altri casi si svolgono in maniera analoga.
\end{proof}

\begin{example}
  La successione $a_n = n!$ è crescente 
  in quanto per ogni $n$ si ha $(n+1)! = (n+1)\cdot n! \ge n!$.
\end{example}

La monotònia di una successione è di estrema rilevanza in quanto,
come enunciato nel teorema seguente, le successioni monotòne sono regolari, 
cioè hanno sempre limite (finito o infinito). 
Questo teorema verrà utilizzato molte volte, in particolare quando vogliamo 
dimostrare che una successione ha limite ma non conosciamo esplicitamente il valore 
del limite. 
Il primo esempio notevole si ha nel Teorema~\ref{th:5767684} dove definiremo 
la costante di Nepero $e$ come il limite di una particolare successione crescente.

\begin{theorem}[regolarità delle successioni monotòne]
Le successioni monotòne sono regolari.
Più precisamente se $a_n$ è crescente allora $\lim a_n = \sup a_n$ 
mentre se $a_n$ è decrescente $\lim a_n = \inf a_n$.
\end{theorem}
\begin{proof}
In realtà questo teorema è un caso particolare del 
Teorema~\ref{th:limite_monotona} se ricordiamo che una successione 
$a_n$ non è altro che una funzione $f(n)=a_n$ definita sui numeri naturali.
E il limite a $+\infty$ è un limite sinistro.
Vista però l'importanza di questo risultato possiamo ripetere la dimostrazione 
in questo caso specifico.

Per fissare le idee supponiamo che la successione $a_n$ sia crescente e dimostriamo 
che $\lim a_n = \ell$ con $\ell=\sup a_n$.
Per la definizione di limite dobbiamo verificare che scelto un qualunque intorno $U$ 
di $\ell$ si ha definitivamente $a_n\in U$.
Dimostriamo anzi una condizione più forte e cioè che $a_n$ sta definitivamente 
in un intorno sinistro di $\ell$ cioè che per ogni $q<\ell$ esiste $N$ tale che 
per ogni $n\ge N$ si ha $a_n\in (q,\ell]$ (si noti che $\ell$ può essere finito o $+\infty$ 
e quanto abbiamo scritto vale in entrambi i casi).
Che valga $a_n\le \ell$ è ovvio in quanto $\ell=\sup a_n$ è, per definizione, un maggiorante 
di $a_n$. Inoltre $\ell$ è il minimo dei maggioranti e quindi preso 
$q<\ell$ sappiamo che $q$ non è un maggiorante ovvero deve esistere $N$ tale che 
$a_N > q$. Ma, essendo $a_n$ crescente, sappiamo che per ogni $n\ge N$ si ha 
$a_n \ge a_N >q$ e questo conclude la dimostrazione.
\end{proof}

%%%%%%%%%%%%%%%%%
%%%%%%%%%%%%%%%%%
%%%%%%%%%%%%%%%%%
%%%%%%%%%%%%%%%%%

\subsection{successioni ricorsive}

Le successioni ricorsive sono le successioni
definite per induzione (tramite il teorema~\ref{th:induzione}):
\[
 \begin{cases}
   a_0 = \alpha \\
   a_{n+1} = f(a_n).
 \end{cases}
\]
Si potrebbe affrontare ora il capitolo~\ref{ch:successioni_ricorsive}
per uno studio di questo tipo di successioni.
La prima parte di quel capitolo è infatti accessibile con le nozioni
che sono state sviluppate finora.
La seconda parte richiederà invece degli strumenti più avanzati.

\subsection{la costante di Nepero}

\begin{figure}
  \begin{center}
    \begin{tikzpicture}[x=4.0cm,y=1.8cm]
    	\draw[thick,->] (-0.5,0) -- (1.2,0);
      \draw[thick,->] (0,-0.2) -- (0,3.0);
      %
      \draw[domain=-0.5:1.05,smooth,variable=\x,thick,blue] plot
      ({\x}, {exp(\x)});
      \draw[blue] (1.0,-0.05) -- (1.0,{exp(1.0)})
      -- (-0.2, {exp(1.0)}) node[left]{$e^q$};
      \draw (-0.2,1) node[left]{$\scriptstyle 1$} -- (0,1);
      \draw (0.2,-0.05) node[below]{$\frac 1 n$} -- (0.2,1.2)
      -- (-0.05,1.2) node[left]{$\scriptstyle 1+\frac q n$};
      \draw (0.4,-0.05) node[below]{$\frac 2 n$} -- (0.4,1.44)
      -- (-0.2,1.44) node[left]{$\scriptstyle\enclose{ 1+\frac q n}^2$};
      \draw (0.6,-0.05) node[below]{$\frac 3 n$} -- (0.6,1.738)
      -- (-0.05,1.738) node[left]{$\scriptstyle\enclose{1+\frac q n}^3$};
      \draw (0.8,-0.05) node[below]{$\rule{0mm}{2mm}\dots$} -- (0.8,2.0736)
      -- (-0.2,2.0736) node[left]{$\vdots$};
      \draw (1.0,-0.05) node [below] {$1$} -- (1.0,2.48832)
      -- (-0.05,2.48832) node[left]{$\scriptstyle\enclose{1+\frac q n}^n$};
      \draw[thick] (0,1) -- (0.2,1.2)
      -- (0.4,1.44) -- (0.6,1.738)
      -- (0.8,2.0736) -- (1.0,2.48832);
    \end{tikzpicture}
  \end{center}
  \caption{L'interesse composto.
  Se dividiamo l'intervallo $[0,1]$
  in $n$ parti (in figura $n=5$), al passo zero
  partiamo con un capitale pari ad $1$ e ad ogni
  passo di ampiezza $\frac 1 n$ moltiplichiamo
  il nostro capitale per $1+\frac q n$ (supponendo
  di avere un interesse pari a $q$ che
  nel tempo $\frac 1 n$ ci paga quindi $\frac q n$
  volte il nostro capitale).
  Dopo
  $n$ passi, cioè al tempo $1$, avremo
  un capitale pari a $\enclose{1+\frac q n}^n$.
  Se $n\to+\infty$ questa quantità tende ad $e^q$.}
  \label{fig:nepero}
\end{figure}


La funzione esponenziale è legata ad un modello di crescita che si trova spesso
in natura: la \emph{crescita esponenziale}%
\mymargin{crescita esponenziale}%
\index{crescita!esponenziale}.
Prendiamo come esempio una popolazione di batteri che cresce senza
limitazioni di spazio e di nutrimento oppure
pensiamo alla crescita di un capitale dovuto ad una rendita finanziaria.

Supponiamo che una popolazione che al tempo $t_0=0$
ammonta ad un certo numero $c$ di batteri, al tempo
$t>0$ raggiunga una numerosità $q(t,c)$.
Se lascio crescere la popolazione per un ulteriore
tempo $s>0$ troverò al tempo $t+s$ la stessa
popolazione che avrei al tempo $s$ se al tempo
zero fossi partito con la popolazione $q(t)$:
\[
  q(t+s,c) = q(s,q(t,c)).
\]
L'equazione precedente si chiama proprietà
di \emph{semigruppo}
\index{semigruppo}%
continuo.
Ma fissato $t$ la popolazione $q(t,c)$ deve
essere proporzionale a $c$ perché ogni batterio
ha la sua discendenza indipendentemente dalla numerosità
totale della popolazione. In pratica
si deve avere $q(t,c) = k(t) \cdot c$ per una opportuna
funzione $k(t)$ che non dipende da $c$.
Dunque
\[
  q(t+s,c)
  = q(s,q(t,c))
  = k(s) \cdot q(t,c)
  = k(s) \cdot k(t) \cdot c
\]
da cui
\[
  k(t+s) = k(s) \cdot k(t).
\]
In base al teorema~\ref{th:isomorfismo}
possiamo affermare
che $k(t)$ è una funzione esponenziale $k(t)=a^t$
per una qualche costante $a$.
La costante $a$ può essere determinata mediante la formula:
\[
  a = k(1) = \frac{q(1,c)}{c}
\]
ma questa espressione non ha un preciso significato fisico in quanto
dipende dall'unità di tempo scelta.

La costante a cui possiamo dare significato è invece l'aumento relativo
istantaneo della popolazione. Possiamo infatti supporre che
se lasciamo la popolazione crescere per un tempo $\Delta t$ molto piccolo,
si otterrà un aumento di popolazione proporzionale al tempo $\Delta t$
e alla popolazione:%
\mynote{stiamo qui anticipando il concetto di derivata}
\begin{equation}\label{eq:488464}
  q(\Delta t,c) = c + r c \Delta t = (1+r \Delta t) c.
\end{equation}
La costante $r$ rappresenta quindi l'aumento
relativo istantaneo della popolazione (nel caso dell'investimento
$r$ sarebbe il tasso di interesse istantaneo).
Questa definizione ha senso
quando $\Delta t$ è piccolo in quanto non tiene conto del fatto che
nell'intervallo di tempo $[t,t+\Delta t]$ la popolazione che si
è aggiunta genera anch'essa nuova popolazione (ovvero l'interesse
accumulato genera anch'esso interesse\mynote{%
In effetti la scoperta della costante $e$
è dovuta a Jacob Bernoulli (1655--1705) vedi note storiche 
a pag~\pageref{note:Bernoulli}}).

Per calcolare l'aumento della popolazione su tempi ``grandi'' possiamo
suddividere gli intervalli temporali in $n$ intervallini di ampiezza
$\Delta t$ e applicare in ognuno di essi la relazione \eqref{eq:488464}.
Si trova:
\begin{align*}
 q(\Delta t,c) &= c(1+r\Delta t) \\
 q(2\Delta t,c) &= q(\Delta t,c) (1+r\Delta t)  = c(1+r\Delta t)^2\\
 &\vdots \\
 q(n\Delta t) &= c(1+r\Delta t)^n.
\end{align*}
Dunque, ponendo $\Delta t=t/n$ si ha
\[
  q(t,c) = c\enclose{1+r\frac{t}{n}}^n
\]
in particolare per $t=1/r$ si ottiene:
\[
  q(1/r,c) = c\enclose{1+\frac{1}{n}}^n
\]
e ricordando che $q(t,c)=c a^t$ otteniamo:
\[
  a^{\frac 1 r} = \enclose{1+\frac{1}{n}}^n.
\]
Se per $n\to +\infty$ (che corrisponde a $\Delta t \to 0$)
la quantità sul lato destro tende ad un numero $e$ (che chiameremo costante
di Nepero) avremo allora
\[
  a = e^r, \qquad q(t,c) = c e^{rt}
\]
che è la relazione che lega le due costanti $a$ e $r$ che definiscono
la crescita esponenziale.

Risulta in effetti valido il seguente.

\begin{theorem}[costante di Nepero]
\mymark{**}%
\label{th:5767684}
La successione
\[
  a_n = \enclose{1+\frac 1 n}^n
\]
è crescente e limitata, dunque è convergente.
\end{theorem}
%
Per dimostrare il teorema precedente ci serve 
preliminarmente il seguente risultato.
%
\begin{theorem}[disuguaglianza di Bernoulli]
  \label{th:disuguaglianza_bernoulli}%
  \mymark{**}%
  \mymargin{disuguaglianza di Bernoulli}%
  \index{Bernoulli!disuguaglianza di}%
  \index{disuguaglianza!di Bernoulli}%
  Se $x > -1$ e $n\in \NN$ si ha
  \begin{equation}
  \label{eq:bernoulli}
  (1+x)^n \ge 1 + nx.
  \end{equation}
  \end{theorem}
  %
  \begin{proof}
  \mymark{**}
  Dimostriamo che vale~\eqref{eq:bernoulli}
  per induzione su $n$.
  Per $n=0$ la disequazione \eqref{eq:bernoulli} diventa $1\ge 1$
  ed è quindi verificata.
  Se~\eqref{eq:bernoulli}
  è verificata per un certo $n$
  moltiplicando ambo i membri per $1+x > 0$ si ottiene
  \[
  (1+x)^{n+1} \ge (1+x) (1+nx) = 1 + (n+1)x + n x^2
  \ge 1 + (n+1)x
  \]
  che è proprio la disuguaglianza~\eqref{eq:bernoulli}
  con $n+1$ al posto di $n$.
  \end{proof}
%
\begin{proof}[Dimostrazione del teorema~\ref{th:5767684}]
Dimostriamo innanzitutto che $a_n$ è crescente, cioè che
per ogni $n\ge 2$ si ha $a_n \ge a_{n-1}$.
E' chiaro che $a_n>0$ per ogni $n$,
quindi ci riconduciamo a
verificare che $\frac{a_n}{a_{n-1}} \ge 1$.

Si ha
\begin{align*}
\frac{a_n}{a_{n-1}}
&= \frac{\enclose{1+\frac 1 n}^n}{\enclose{1+\frac 1 {n-1}}^{n-1}}
= \frac{\enclose{\frac{n+1}{n}}^n}{\enclose{\frac{n}{n-1}}^{n-1}}\\
&= \enclose{\frac{n+1}{n}\cdot\frac{n-1}{n}}^n \cdot \frac{n}{n-1}
= \enclose{\frac{n^2- 1}{n^2}}^n \cdot \frac{n}{n-1}
\end{align*}
Osserviamo ora che la disuguaglianza di Bernoulli, 
teorema~\ref{th:disuguaglianza_bernoulli},
garantisce
\[
  \enclose{\frac{n^2 -1}{n^2}}^n
  = \enclose{1-\frac{1}{n^2}}^n
  \ge 1 - \frac{n}{n^2} = 1 - \frac{1}{n} = \frac{n-1}{n}
\]
da cui si ottiene, come volevamo, $a_n / a_{n-1} \ge 1$ cioè
$a_n$ è crescente.

Se ora consideriamo la successione
\[
  b_n = \enclose{1+\frac 1 n}^{n+1}
\]
osserviamo che si ha
\[
  b_n = \enclose{1+\frac 1 n}^n \cdot \enclose{1+\frac 1 n}
   = a_n\cdot \enclose{1+\frac 1 n} > a_n.
\]
Per dimostrare che $a_n$ è limitata sarà quindi sufficiente dimostrare
che $b_n$ è superiormente limitata. Vedremo ora che $b_n$ è decrescente (e quindi $a_n \le b_n \le b_1$ è superiormente limitata).

Procediamo in maniera analoga a quanto fatto per $a_n$:
\begin{align*}
\frac{b_{n-1}}{b_n}
& = \frac{\enclose{1+\frac{1}{n-1}}^n}{\enclose{1+\frac{1}{n}}^{n+1}}
  = \frac{\enclose{\frac{n}{n-1}}^n}{\enclose{\frac{n+1}{n}}^{n+1}}
  = \enclose{\frac{n}{n-1}\cdot\frac{n}{n+1}}^{n+1}\cdot\frac{n-1}{n} \\
& = \enclose{\frac{n^2}{n^2-1}}^{n+1} \cdot \frac{n-1}{n}
  = \enclose{1 + \frac{1}{n^2-1}}^{n+1} \cdot \frac{n-1}{n}.
\end{align*}
In base alla disuguaglianza di Bernoulli otteniamo
\[
  \enclose{1 + \frac{1}{n^2-1}}^{n+1}
  \ge 1 + (n+1) \cdot \frac{1}{n^2-1}
  = 1 + \frac{1}{n-1} = \frac{n}{n-1}.
\]
Mettendo insieme le due stime si ottiene dunque $b_{n-1}/b_n \ge 1$
che è quanto ci rimaneva da dimostrare.
\end{proof}

E' quindi giustificata la seguente.

\begin{definition}[costante di Nepero]
\mymark{***}
Definiamo la \emph{costante di Nepero}%
\mymargin{costante di Nepero}%
\index{costante!di Nepero}%
\index{$e$}%
\mynote{John Napier (1550-1617) vedi note storiche a pag~\pageref{nota:Nepero}}%
\[
  e = \lim_{n\to +\infty} \enclose{1+\frac 1 n}^n,
  \qquad n\in\NN.
\]
\end{definition}

Sapendo che
\[
  \enclose{1+\frac 1 n}^n \le e \le \enclose{1+\frac 1 n}^{n+1}
\]
e ponendo $n=1$ otteniamo $2\le e \le 4$.

\begin{exercise}\label{ex:4876765}
Posto $a_n = \frac{n^n}{n!}$ mostrare che $\frac{a_{n+1}}{a_n} \to e$.
\end{exercise}


\begin{theorem}[limiti che si riconducono al numero $e$]
\mymargin{limiti che si riconducono al numero $e$}%
\index{limiti che si riconducono al numero $e$}
Si ha 
\[
\lim_{x\to 0}  \enclose{1 + x}^{\frac 1 {x}} = e.
\]
\end{theorem}
%
\begin{proof}
  Iniziamo col dimostrare che 
  \begin{equation}\label{eq:2373793}
    \lim_{x\to +\infty} \enclose{1+\frac 1 x}^x = e.
  \end{equation}
  Ogni numero reale $x$ è compreso tra due interi consecutivi:
  \[
      \lfloor x \rfloor \le x \le \lfloor x\rfloor + 1
  \]
  e dunque
  \[
    1+\frac{1}{\lfloor x\rfloor +1 } 
    \le 1 + \frac 1 x 
    \le 1 + \frac 1 {\lfloor x\rfloor} 
  \]
  da cui 
  \begin{equation}\label{eq:48675248}
    \enclose{1+\frac{1}{\lfloor x\rfloor +1 }}^{\lfloor x\rfloor} 
    \le \enclose{1 + \frac 1 x }^x
    \le \enclose{1 + \frac 1 {\lfloor x\rfloor}}^{\lfloor x\rfloor +1}. 
  \end{equation}
  Per calcolare il limite della espressione a sinistra 
  operiamo il cambio di variabile $n=\lfloor x\rfloor + 1$ per 
  ottenere:
  \begin{align*}
  \lim_{x\to+\infty} \enclose{1+\frac{1}{\lfloor x\rfloor +1 }}^{\lfloor x\rfloor}
  &= \lim_{n\to +\infty} \enclose{1+\frac 1 n}^{n-1}\\
  &= \lim_{n\to +\infty} \frac{\enclose{1+\frac 1 n}^{n}}{1+\frac 1 n}
   = \frac{e}{1} = e.
  \end{align*}
  Similmente per quanto riguarda il limite della espressione a destra 
  si può operare il cambio di variabile $n=\lfloor x\rfloor$:
  \begin{align*}
    \enclose{1 + \frac 1 {\lfloor x\rfloor}}^{\lfloor x\rfloor +1}
    &= \lim_{n\to+\infty} \enclose{1+\frac 1 n}^{n+1} \\
    &= \lim_{n\to+\infty} \enclose{1+\frac 1 n}^n 
    \cdot \enclose{1 + \frac 1 n} 
    = e \cdot 1 = e.
  \end{align*}
  Dunque per il teorema dei due carabinieri otteniamo
  che anche l'espressione 
  al centro in~\eqref{eq:48675248} tende ad $e$.
  Come conseguenza, tramite il cambio di variabile $x\mapsto \frac 1 x$
  si ottiene 
  \[
  \lim_{x\to 0^+}\enclose{1+x}^{\frac 1 x} = e.  
  \]

  Consideriamo ora il limite per $x\to -\infty$ 
  della solita espressione. 
  Tramite il cambio di variabile $y=-x$ si ha 
  \begin{align*}
  \lim_{x\to -\infty} \enclose{1+\frac{1}{x}}^x 
  &= \lim_{y\to +\infty} \enclose{1 - \frac 1 y}^{-y}
   = \lim_{y\to +\infty} \enclose{\frac {y-1} y}^{-y}\\
  &= \lim_{y\to +\infty} \enclose{\frac{y}{y-1}}^y
   = \lim_{y\to +\infty} \enclose{1 + \frac{1}{y-1}}^y\\
  &= \lim_{y\to +\infty} \enclose{1 + \frac{1}{y-1}}^{y-1}
    \cdot\enclose{1+\frac 1 {y-1}} 
    = e\cdot 1 = e.\\
  \end{align*}
  Di conseguenza, facendo il cambio di variabile $x\mapsto \frac 1 x$ 
  si ottiene
  \[
  \lim_{x\to 0^-} \enclose{1+x}^{\frac 1 x} = e.
  \]
  Mettendo insieme limite destro e limite sinistro si ottiene 
  infine il limite completo per $x\to 0$. 
\end{proof}
%
\begin{corollary}%
\label{cor:limite_notevole_ex}%
\mymark{**}%
Per ogni $x\in \RR$ si ha
\[
  \lim_{n\to +\infty} \enclose{1+ \frac x n}^n = e^x.
\]
\end{corollary}
%
\begin{proof}
Infatti, per il teorema precedente, posto $a_n = x/n$ si ha
\[
\lim_{n\to +\infty}\enclose{1+\frac x n}^{\frac n x} = e.
\]
Ma allora
\[
\enclose{1+ \frac x n}^n = \enclose{\enclose{1+\frac x n}^{\frac n x}}^x
\to e^x
\]
\end{proof}

\begin{definition}[logaritmi naturali]
Vedremo che il numero $e$ risulta essere una base naturale per la funzione
esponenziale e di conseguenza per il logaritmo. Il logaritmo in base
$e$ viene chiamato \emph{logaritmo naturale}%
\mymargin{logaritmo naturale}%
\index{logaritmo!naturale} e viene indicato con $\ln = \log_e$.
\end{definition}

\mynote{%
In alcuni testi si utilizza l'operatore $\log$, indicato senza una base esplicita,
ma la definizione non è completamente condivisa.
In certi testi (per lo più in ambito matematico)
si definisce $\log  = \ln = \log_e$,
in altri testi si considera $\log = \log_{10}$.
}

\begin{corollary}[limiti notevoli]\label{cor:limite_notevole_e}
\mymark{*}%
Si ha
\begin{gather}
 \lim_{x\to 0} \frac{\ln \enclose{1+ x}}{x} = 1; \\
 \lim_{x\to 0} \frac{e^x-1}{x} = 1.
\end{gather}
\end{corollary}
%
\begin{proof}
Per quanto riguarda il logaritmo ci si riconduce al teorema precedente
osservando che:
\[
  \frac{\ln(1+x)}{x}
  = \ln \enclose{(1+x)^{\frac 1 {x}}}
  \to \ln e = 1.
\]
Per l'esponenziale ci si riconduce al logaritmo
osservando che posto
\[
  y = e^x-1
\]
se $x\to 0$ anche $y\to 0$ e quindi si ha:
\[
\frac{e^{x}-1}{x} = \frac{y}{\ln(1+y)} \to 1.
\]

\end{proof}

\begin{exercise}
Mostrare che
\begin{gather*}
  \lim_{n\to+\infty} n^n\cdot \enclose{\frac{n+1}{n^2+1}}^n = e; \\
  \lim_{n\to+\infty} n\cdot \ln\enclose{1 + \frac 1 n} = 1; \\
  \lim_{n\to+\infty} \enclose{1-\frac{n+1}{n!}}^{(n-1)!} = \frac 1 e; \\
  \lim_{n\to+\infty} n \cdot \enclose{\sqrt[n]{2}-1} = \ln 2.
\end{gather*}
\end{exercise}

%%%%%%%%%%%%%%%%%%%
%%%%%%%%%%%%%%%%%%%
%%%%%%%%%%%%%%%%%%%
%%%%%%%%%%%%%%%%%%%
\subsection{criteri del rapporto e della radice}

Succede spesso di dover determinare il limite
del rapporto di due successioni che tendono entrambe a infinito
oppure entrambe a zero.
In queste situazioni il teorema del limite del rapporto non
si applica in quanto siamo di fronte ad una forma indeterminata.
Dovremo quindi capire quale delle due successioni
tende a infinito o a zero ``più velocemente''.

I teoremi che seguono si basano (in maniera più o meno implicita)
sull'andamento della successione geometrica:
\[
  a_n = q^n
\]
dove $q>0$ è una costante fissata.
Dalle proprietà della funzione esponenziale 
sappiamo che se $q>1$ si ha $q^n\to \infty$.
Se $q=1$ chiaramente $q^n=1\to 1$
e se $q<1$ si ha $q^{-1}>1$ e quindi $q^{-n} \to +\infty$
da cui $q^n \to 0$.

\begin{theorem}[criterio del rapporto]
\label{th:criterio_rapporto}
\index{criterio!del rapporto per le successioni}
\index{rapporto!criterio del}
  Sia $a_n$ una successione reale a termini positivi
  $a_n > 0$ tale che esista il limite del rapporto di due termini consecutivi:
  \[
     \frac{a_{n+1}}{a_n} \to \ell \in [0,+\infty].
  \]
  Se $\ell < 1$ allora $a_n \to 0$, se $\ell >1$ allora $a_n \to +\infty$.
\end{theorem}
%
%
% \begin{proof}\mynote{%
%   la dimostrazione di questo teorema si potrebbe fare in maniera
%   molto simile alla dimostrazione del teorema~\ref{th:criterio_radice}
%   senza tirare in ballo il teorema~\ref{th:criterio_cesaro} che è decisamente più complesso.
%   }
% Grazie al teorema~\ref{th:criterio_cesaro} sappiamo
% che $\sqrt[n]{a_n}\to \ell$ e quindi il risultato
% segue direttametne dal teorema~\ref{th:criterio_radice}.
% \end{proof}
%

\begin{proof}
Supponiamo sia $\ell<1$. Posto $q=(1+\ell)/2$ si ha $\ell < q < 1$ 
e posto $\eps=q-\ell>0$ per la definizione di limite $\frac{a_{n+1}}{a_n}\to \ell$ 
dovrà esistere un $N\in \NN$ tale
che per ogni $n\ge N$ si abbia:
\[
  \frac{a_{n+1}}{a_n} < \ell + \eps = q
\]
ovvero $a_{n+1} < q \cdot a_n$. In particolare si avrà:
\begin{align*}
  a_{N+1} &< q \cdot a_N \\
  a_{N+2} &< q \cdot a_{N+1} < q^2\cdot a_N \\
  a_{N+3} &< q \cdot a_{N+2} < q^3\cdot a_N \\
  \vdots
\end{align*}
ed è chiaro che per induzione potremo dimostrare che per
ogni $k\in \NN$ si ha
\[
  a_{N+k} < q^k\cdot a_N.
\]
Osserviamo però che $q^k \cdot a_N \to 0$ per $k\to +\infty$
in quanto $q<1$ e quindi $q^k \to 0$. 
Dunque, tolti i primi $N$ termini, la successione $a_n$ tende a zero. 
Ma i primi $N$ termini non influenzano né il carattere né il limite 
della successione e quindi l'intera successione $a_n$ tende a zero.

Il caso $\ell>1$ si fa in maniera analoga. Si sceglie $q$ tale
che $1<q<\ell$ e si trova, in maniera analoga al caso precedente,
che per un certo $N\in \NN$ e per ogni $k\in \NN$ si ha
\[
  a_{N+k} > q^k \cdot a_N \to +\infty.
\]
\end{proof}

Osserviamo che, nel teorema precedente (ma anche nel criterio della radice teorema~\ref{th:criterio_radice}),
non si può concludere alcunché nel
caso in cui sia $\ell = 1$.
Ad esempio le due successioni $a_n = 1/n$ e $b_n = n$
hanno limiti diversi ($a_n \to 0$, $b_n\to +\infty$) ma per entrambe
il limite del rapporto di termini consecutivi tende ad $\ell=1$.

\begin{exercise}
Mostrare che
\begin{gather*}
  \lim \frac{n!}{n^n} = 0 \\
  \lim \frac{(2n)!}{(2n)^n} = +\infty
\end{gather*}
\end{exercise}

\begin{theorem}[criterio della radice]
\label{th:criterio_radice}%
\mymark{***}%
\mymargin{criterio della radice}%
\index{criterio!della radice per successioni}%
Sia $a_n$ una successione a termini non negativi, $a_n\ge 0$, tale che
\[
  \sqrt[n]{a_n} \to \ell
\]
con $\ell \in \bar \RR$.
Allora se $\ell<1$ si ha $a_n \to 0$ se invece $\ell > 1$ si ha $a_n \to +\infty$.
\end{theorem}
%
\begin{comment}
\begin{proof}
\mymark{**}
Consideriamo prima il caso $\ell < 1$.
Se $\lim \sqrt[n]{a_n} = \ell$ significa che per ogni $\eps>0$ la successione
$\sqrt[n]{a_n}$ risulta definitivamente minore di $\ell +\eps$.
Scegliendo opportunamente $\eps$ (ad esempio $\eps = (1-\ell)/2$) si potrà
avere $q = \ell+\eps < 1$. Dunque avremo definitivamente $\sqrt[n]{a_n}< q$
ovvero $a_n < q^n$. Per ipotesi $a_n\ge 0$
e quindi, tolto un numero finito di termini, si ottiene $0 \le a_n < q^n \to 0$
da cui $a_n \to 0$ (in quanto l'aver tolto un numero finito di termini non
cambia né il carattere né il limite della successione).

Se $\ell>1$ si potrà procedere in maniera analoga. Esisterà $q$ con $1 < q < \ell$ tale che definitivamente $\sqrt[n]{a_n} > q$ da cui $a_n > q^n \to +\infty$.
\end{proof}
\end{comment}

\begin{proof}
  Si può osservare che
  \[
    a_n = \enclose{\sqrt[n]{a_n}}^n
     = e^{n \cdot \ln \sqrt[n]{a_n}}.
  \]
  Se $\ell <1$ allora il logaritmo tende ad un numero negativo,
  l'argomento dell'esponenziale tende a $-\infty$ e quindi l'esponenziale tende a zero.

  Se invece $\ell>1$ il logaritmo tende ad un numero positivo e quindi l'esponenziale tende a $+\infty$.
\end{proof}

%%%%%
%%%%%
\section{ordini di infinito, equivalenza asintotica}
%%%%%
%%%%%

\begin{definition}[ordine di infinito/infinitesimo]%
  \label{def:ordine_infinito}%
  \mymark{***}%
  \mymargin{ordine di infinito/infinitesimo}%
\index{ordine di infinito/infinitesimo}%
  \index{ordine!di infinito}%
  \index{infinito}%
  \index{infinitesimo}%
  Sia $A\subset \RR$, $f,g\colon A \to (0,+\infty)$.
  Sia $x_0\in \bar \RR$ un punto di accumulazione di $A$.
  \begin{enumerate}
  \item
  Diremo che
  per $x\to x_0$ la funzione $f$ è \emph{molto più piccola}
  della funzione $g$ e scriveremo $f(x) \ll g(x)$ se vale
  \mymargin{$\ll$}%
\index{$\ll$}
  \[
  \frac{f(x)}{g(x)} \to 0, \qquad \text{per $x\to x_0$}
  \]
  diremo invece che $f$ è \emph{molto più grande}
  di $g$ e scriveremo $f(x) \gg g(x)$ se
  \mymargin{$\gg$}%
\index{$\gg$}
  \[
  \frac{f(x)}{g(x)} \to +\infty, \qquad \text{per $x\to x_0$.}
  \]
  \item
  Diremo infine che $f$ e $g$
  sono \emph{asintoticamente equivalenti}%
\mymargin{equivalenza asintotica}%
\index{asintoticamente equivalenti}
  \mymargin{equivalenza asintotica}%
\index{equivalenza!asintotica}%
  per $x\to x_0$
  e scriveremo $f(x) \sim g(x)$ se
  \mymargin{$\sim$}%
\index{$\sim$}
  \[
  \frac{f(x)}{g(x)} \to 1, \qquad \text{per $x\to x_0.$}
  \]
  \end{enumerate}
\end{definition}
  
Ad esempio è facile verificare che se $\alpha > \beta > 0$
allora $x^\alpha \gg x^\beta$ per $x\to +\infty$
mentre $x^\alpha \ll x^\beta$ per $x\to 0^+$.
Analogamente se $a>b>1$ allora $a^x\gg b^x$ per $x\to +\infty$
mentre $a^x \ll b^x$ per $x\to -\infty$.

E' molto facile verificare che le relazioni
$\ll$ e $\gg$ sono una l'inversa dell'altra
e soddisfano la proprietà transitiva
mentre la relazione $\sim$ soddisfa la proprietà simmetrica
e la proprietà transitiva.

\begin{theorem}[ordini di infinito]
\label{th:ordine_infinito}%
\mymargin{ordini di infinito}%
\index{ordini di infinito}%
\mymark{***}%
Siano $a>1$ e $\alpha>0$. Per $n\to +\infty$, $n\in\NN$
si ha
\[
  n^\alpha \ll a^n \ll n! \ll n^n
\]
e per $x\to +\infty$, $x\in \RR$ si ha 
\[
\log_a x \ll x^\alpha \ll a^x.
\]
\end{theorem}
%
\begin{proof}
\mymark{**}
Cominciamo col mostrare che $a^n \ll n!$
applicando il criterio del rapporto alla successione $\frac{a^n}{n!}$:
\[
\frac{\displaystyle \frac{a^{n+1}}{(n+1)!}}{\displaystyle \frac{a^n}{n!}}
= \frac{a^{n+1}}{a^n}\cdot \frac{n!}{(n+1)!}
= a \cdot \frac {1}{n + 1} \to 0 < 1.
\]
Dunque si ha, come richiesto, $a^n / n! \to 0$.
Si procede in modo analogo per mostrare che $n! \ll n^n$:
\begin{align*}
\frac{(n+1)!}{n!}\cdot \frac{n^n}{(n+1)^{n+1}}
&= (n+1) \cdot \enclose{\frac{n}{n+1}}^n \frac {1}{n+1}\\
&= \frac{1}{\enclose{1+\frac 1 n}^n} \to \frac 1 e < 1.
\end{align*}
  
Per dimostrare che
$n^\alpha \ll a^n$
si può procedere con il criterio del rapporto, come nei casi precedenti:
\[
\frac{(n+1)^\alpha}{n^\alpha}\cdot \frac{a^n}{a^{n+1}}
= \frac 1 a \cdot \enclose{\frac{n+1}{n}}^\alpha \to \frac 1 a \cdot 1^\alpha = \frac 1 a < 1
\]
da cui $n^\alpha / a^n \to 0$.

Per $x\in \RR$,
cerchiamo di ricondurci ad una successione a valori interi.
Osserviamo che si ha
\[
\lfloor x \rfloor
\le x
\le \lfloor x \rfloor + 1
\]
da cui, per monotonia,
\[
\lfloor x \rfloor^\alpha
\le x^\alpha
\le (\lfloor x \rfloor + 1)^\alpha
= \lfloor x \rfloor^\alpha \enclose{1+ \frac{1}{\lfloor x \rfloor}}^\alpha
\]
e
\[
a^{\lfloor x \rfloor}
\le a^{x}
\le a^{\lfloor x \rfloor + 1}
= a \cdot a^{\lfloor x \rfloor}.
\]
Dunque
\[
\frac{\lfloor x \rfloor^\alpha}{a \cdot a^{\lfloor x \rfloor}}
\le \frac{x^\alpha}{a^{x}}
\le \frac{\lfloor x \rfloor^\alpha \enclose{1+ \frac{1}{\lfloor x \rfloor}}^\alpha}
    {a^{\lfloor x \rfloor}}.
\]
Ma ora, se $x\to +\infty$ sapendo che $n = \lfloor x\rfloor \to +\infty$ 
possiamo effettuare un cambio di variabile nel limite
\[
\lim_{x\to +\infty} \frac{\lfloor x \rfloor^\alpha}{a^{\lfloor x \rfloor}} 
= \lim_{n\to+\infty} \frac{n^\alpha}{a^n} = 0
\]
da cui segue che $\frac{x^\alpha}{a^{x}}\to 0$.

Per dimostrare l'ultima relazione, $\log_a x\ll x^\alpha$,
operiamo il cambio di variabile $y = \alpha \cdot \log_a x$
cosicché $a^y = x^\alpha$.
Notiamo che se $x\to +\infty$
anche $y \to +\infty$.
Dunque, per le proprietà precedenti,
sappiamo che $y \ll a^y$ e dunque
\[
\frac{\log_a x}{x^\alpha}
= \frac{1}{\alpha}\cdot\frac{y}{a^{y}} \to 0.
\]
\end{proof}

Le notazioni e gli
ordini di infinito individuati nel teorema precedente
sono strumenti molto utili nel calcolo dei limiti.

L'equivalenza asintotica
si mantiene per prodotto e rapporto:
se $f\sim F$ e $g\sim G$ allora
\[
 f \cdot g \sim F \cdot G,
 \qquad
 \frac{f}{g} \sim \frac{F}{G}.
\]
Osserviamo inoltre che se
$f \sim g$ e se $f\to \ell$ allora
anche $g\to \ell$.
Se poi $\ell\in(0,+\infty)$
la relazione $f\sim \ell$ è equivalente ad $f\to \ell$.

Per quanto riguarda la somma
è facile verificare che se $f\ll g$ allora
$(f+g) \sim g$ in quanto
\[
  \frac{f + g}{g} = \frac{f}{g} + 1 \to 1.
\]

In un limite in cui compaiono somme di termini
di ordini diversi potremo allora raccogliere i termini di ordine
massimo per individuare il limite, come facciamo
nel seguente.

\begin{example}
Calcolare il limite
\[
\lim_{n\to+\infty}
\frac{2n^4 + 3^n - 3 \ln n}{n! - 3\sqrt n}.
\]
\end{example}
\begin{proof}[Svolgimento.]
Si ha
\[
\frac{2n^4 + 3^n - 3 \ln n}{n! - 3\sqrt{n}}
= \frac
{3^n \cdot \enclose{2\frac{n^4}{3^n}+ 1 - 3\frac{\ln n}{3^n}}}
{n!\cdot \enclose{1-3\frac{\sqrt n}{n!}}}
\]
e ricordando che risulta (teorema~\ref{th:ordine_infinito})
\[
n^4 \ll 3^n, \qquad
\ln n \ll 3^n, \qquad
\sqrt n \ll n!, \qquad
3^n \ll n!
\]
avremo
\[
\frac{2n^4 + 3^n - 3 \ln n}{n! - 3\sqrt{n}}
\sim \frac{3^n}{n!} \to 0.
\]
\end{proof}


\begin{exercise}
Calcolare i seguenti limiti
\begin{gather*}
  \lim_{n\to +\infty} \frac{\displaystyle \ln\sqrt{n^2+n^n}}
  {\displaystyle e^{1 + \ln n}\cdot \ln(n^2-n\sqrt n)}, \qquad
  \lim_{n\to +\infty} \frac{\sqrt{n! + 2^n}}{3^n}, \\
  \lim_{n\to +\infty} \frac{\sqrt{(2n)!}}{n^n}, \qquad
  \lim_{n\to +\infty} \sqrt[n]{e^n + \sqrt{10^n}}.
\end{gather*}
\end{exercise}

\begin{exercise}
  Dimostrare che $\sqrt[n]{n}\to 1$ per $n\to +\infty$.
\end{exercise}



%%%%%%%%%%%
%%%%%%%%%%%
\section{successioni estratte}
%%%%%%%%%%%
%%%%%%%%%%%

\begin{definition}[sottosuccessione]
\mymark{*}
Se $a_n$ è una successione e $n_k$ è una successione strettamente crescente i cui valori sono numeri naturali, allora la successione
$b_k = a_{n_k}$ si dice essere una \emph{sottosuccessione}%
\mymargin{sottosuccessione}%
\index{sottosuccessione} di $a_n$
(o anche \emph{successione estratta} da $a_n$).
\end{definition}

Ricordando che una successione $a_n$ non è altro che una funzione
$\vec a\colon \NN \to \RR$, la successione $n_k$ corrisponde ad una funzione
$\vec n\colon \NN \to \NN$ e la sottosuccessione $a_{n_k}$ corrisponde alla
funzione composta $\vec a \circ \vec n$.

Si osservi che nella definizione precedente la variabile $n$ rappresenta
una variabile muta quando scriviamo la successione $a_n$, ma
rappresenta anche il nome della successione fissata $n_k$.
Questo sovraccarico
di significato è voluto e se usato correttamente rende più semplice
le notazioni, in quanto la successione $n_k$ viene sostituita alla
variabile $n$, con lo stesso nome, nella successione $a_n$.
La sottosuccessione $a_{n_k}$ risulta essere una successione nella variabile $k$, non nella variabile $n$.

\begin{example}
Sia $a_n = n^2$ la successione dei quadrati perfetti:
\begin{center}
\begin{tabular}{l|rrrrrrrrr}
$n$   & $0$ & $1$ & $2$ & $3$ & $4$  & $5$  & $6$  & \dots \\ \hline
$a_n$ & $0$ & $1$ & $4$ & $9$ & $16$ & $25$ & $36$ & \dots
\end{tabular}
\end{center}
Consideriamo la successione dei numeri pari $n_k = 2k$.
la corrispondente sottosuccessione dei quadrati perfetti
$b_k = a_{n_k}$
rappresenta la successione dei quadrati dei numeri pari:
\begin{center}
\begin{tabular}{l|rrrrrrrrr}
$k$       & $0$ & $1$ & $2$ & $3$ & $4$  & $5$  & $6$  & \dots \\ \hline
$n_k$ & $0$ & $2$ & $4$ & $6$ & $8$ & $10$ & $12$ & \dots \\
$a_{n_k}$ & $0$ & $4$ & $16$ & $36$ & $64$ & $100$ & $144$ & \dots
\end{tabular}
\end{center}
Si ha in pratica
  $b_k = a_{n_k} = a_{2k} = (2k)^2$.

Abbiamo in effetti \emph{estratto} alcuni dei termini della successione
originaria.
\end{example}

\begin{example}
Se $a_n = (-1)^n$ e $n_k=2k$ allora $a_{n_k} = 1$.
Vediamo quindi che una successione irregolare
può contenere una sottosuccessione regolare.
\end{example}

Osserviamo che se $\vec n\colon \NN\to \NN$ è una
funzione strettamente crescente
(cioè $n_k=\vec n(k)$ è una successione strettamente crescente di indici)
allora posto $A=\vec n(\NN)=\ENCLOSE{n_k\colon k\in \NN}$ si ha che
$\vec n \colon \NN \to A$ è una bigezione. Quindi $A$ è un insieme infinito.
Viceversa dato un qualunque insieme infinito $A\subset \NN$ esiste una
unica successione $\vec n\colon \NN \to A$ bigettiva e strettamente crescente:
basterà porre, per induzione,
$n_0 = \min A$, $n_1 = \min \ENCLOSE{n\in A \colon n > n_0}$
e, in generale,
 $n_{k+1} = \min\ENCLOSE{ n\in A \colon n > n_k}$.

Dunque possiamo identificare le sottosuccessioni di una successione
$\vec a \colon \NN \to \RR$ con le restrizioni ai sottoinsiemi infiniti di $\NN$.
Nell'esempio precedente, si è considerata la sottosuccessione
di tutti i termini con indice pari $n_k=2k$ per ottenere la sottosuccessione
$a_{n_k} = a_{2k}$. Si può equivalentemente pensare di prendere l'insieme di
tutti i numeri pari $A=2\NN$ e considerare la successione ristretta ai soli
indici pari:
\[
  a_0, a_2, a_4, \dots
\]
Se rinumeriamo gli indici pari usando tutti i numeri naturali otteniamo
la sottosuccessione $b_k=a_{n_k}$:
\[
  b_0 = a_0,\ b_1 = a_2,\ b_2 = a_4,\ \dots,\ b_k = a_{n_k},\ \dots
\]

\begin{lemma}[estratte monotone]%
\label{lem:estratte_monotone}%
Ogni successione $a_n\in \RR$ ha una estratta $a_{n_k}$ monotona.
Inoltre se $\sup a_n = +\infty$ c'è una estratta $a_{n_k}$
strettamente crescente che tende a $+\infty$.
\end{lemma}
%
\begin{proof}
Consideriamo l'insieme $P$ dei punti di ``picco'', ovvero degli indici
di quei termini della successione che sono maggiori o uguali a tutti i termini
seguenti:
\[
  P = \ENCLOSE{n\in \NN\colon m\ge n \implies a_n\ge a_m}.
\]
Se $P$ è finito
significa che esiste un indice $n_1\in \NN$ tale
che non ci sono picchi da $n_1$ in poi. In particolare $n_1$ non è un punto di
picco
quindi deve esistere $n_2>n_1$ tale che $a_{n_2}>a_{n_1}$.
Ma neanche $n_2$ è un punto di picco quindi deve esistere $n_3>n_2$ tale
che $a_{n_3}>a_{n_2}$... procedendo induttivamente si riesce quindi a definire
una successione $n_k$ di indici tali che $a_{n_k}$ risulta essere strettamente
crescente.

In particolare se $\sup a_n = +\infty$ siamo nella situazione precedente 
perché chiaramente in tal caso $P$ è vuoto visto che per ogni $n\in \NN$ 
deve esistere $m\in \NN$ tale che $a_m > \max\ENCLOSE{a_0, a_1, \dots a_n}$
e certamente $m>n$.

Se, viceversa, $P$ è infinito allora elencando in ordine i suoi elementi otterremo
una successione $n_1 < n_2 < n_3, \dots$ di indici ognuno dei quali corrisponde ad un valore
di picco. 
Se $j>k$ si ha dunque $n_j>n_k$ ed essendo $n_k\in P$ significa che
$a_{n_j} \le a_{n_k}$. 
Dunque la successione $a_{n_k}$ risulta essere decrescente.
\end{proof}


\begin{theorem}[Bolzano-Weierstrass]\label{th:Bolzano}
\label{th:bolzano_weierstrass}%
\mymark{***}%
\mymargin{Bolzano-Weierstrass}%
\index{Bolzano-Weierstrass}%
\index{teorema!di Bolzano-Weierstrass}%
Ogni successione $a_n$ ha una estratta regolare.
Più precisamente se $a_n$ è una successione limitata
allora esiste una sottosuccessione
$a_{n_k}$ convergente.
Se $a_n$ è una successione non limitata allora 
esiste una estratta $a_{n_k}$ divergente.
\end{theorem}
%
\begin{proof}
\mymark{***}
Cominciamo con il caso reale $a_n\in \RR$.
Il lemma~\ref{lem:estratte_monotone} garantisce 
che esiste una estratta $a_{n_k}$ monotona. 
Ma per il teorema~\ref{th:limite_monotona} concludiamo 
immediatamente che $a_{n_k}$ ha limite.
Se $a_n$ è limitata anche $a_{n_k}$ è limitata e quindi 
in tal caso il limite è finito e dunque la successione 
estratta è convergente.

Se $a_n$ non è superiormente limitata il lemma 
ci garantisce che l'estratta tende a $+\infty$ 
e dunque è divergente.
Per simmetria se $a_n$ non è inferiormente limitata 
esiste una estratta che tende a $-\infty$ e quindi anche in
questo caso l'estratta è divergente.

Se $a_n$ è una successione di numeri complessi, si potrà scrivere
$a_n = x_n + i y_n$ con
$x_n$ e $y_n$ successioni reali. Se $a_n$ è limitata significa che $\abs{a_n}$
è superiormente limitata. Ma risulta $\abs{x_n} \le \abs{a_n}$ e
$\abs{y_n}\le \abs{a_n}$ quindi se $a_n$ è limitata anche la parte reale
$x_n$ e la parte immaginaria $y_n$ sono successioni limitate.
Allora $x_n$ ammette una sotto-successione convergente $x_{n_k}$.
Ma $y_{n_k}$ è anch'essa limitata e quindi anch'essa ammette una
sotto-sotto-successione $y_{n_{k_j}}$ convergente.
Dunque la sotto-sotto-successione $a_{n_{k_j}}$ è convergente.

Se $a_n\in \CC$ non è limitata, applicando il teorema al suo modulo $\abs{a_n}$
troviamo che c'è una estratta $a_{n_k}$ tale che $\abs{a_{n_k}}\to +\infty$.
Ma questo significa che $a_{n_k}\to \infty \in \bar \CC$.
\end{proof}

%% \begin{comment}%% SECONDO METODO DIAGONALE DI CANTOR
%% 
%% Il seguente teorema è molto importante dal punto di vista
%% culturale, ma non verrà utilizzato nella teoria seguente.
%% 
%% \begin{theorem}[Cantor: secondo metodo diagonale]
%% \label{th:cantor_secondo}%
%% \mymargin{non numerabilità dei reali}%
%% \index{non numerabilità dei reali}%
%% \index{secondo metodo diagonale di Cantor}%
%% \index{Cantor}%
%% \index{teorema!di Cantor}%
%% L'insieme dei numeri reali non è numerabile: $\#\RR > \#\NN$.
%% \end{theorem}
%% %
%% \begin{proof}
%% E' sufficiente dimostrare che $\#[0,1] > \#\NN$ in quanto
%% chiaramente $\#\RR \ge \#[0,1]$.
%% Supponiamo per assurdo che esista una funzione biettiva $a\colon \NN \to [0,1]$.
%% Questa funzione corrisponde dunque ad una successione $a_n$.
%% Consideriamo l'intervallo $[0,1]$ e dividiamolo in tre intervalli  di lunghezza $1/3$: $[0,1/3]$, $[1/3,2/3]$, $[2/3,1]$. Il punto $a_0$ non può stare in tutti e tre questi intervalli. Sia $[A_0,B_0]$ un intervallo (dei tre) che non contiene $a_0$: $a_0 \not \in [A_0,B_0]$.
%% Dividiamo anche $[A_0,B_0]$ in tre intervalli di lunghezza $1/9$.
%% Almeno uno di questi tre intervalli, che chiamiamo $[A_1,B_1]$,
%% non contiene $a_1$: $a_1 \not \in [A_1,B_1]$.
%% Procediamo così all'infinito in maniera simile al teorema precedente.
%% Otterremo due successioni $A_n$, $B_n$ che soddisfano queste proprietà:
%% \begin{enumerate}
%% \item $A_n$ crescente, $B_n$ decrescente;
%% \item $0\le A_n \le B_n \le 1$;
%% \item $a_n \not \in [A_n, B_n]$;
%% \item $B_n - A_n = 1/3^{n+1}$.
%% \end{enumerate}
%% 
%% Essendo $A_n$ monotona e limitata, essa ha limite finito $\lim A_n = \ell$.
%% Fissato $n$ osserviamo che per ogni $k\ge n$ si ha
%% \[
%%   A_n \le A_k \le b_k \le B_n
%% \]
%% passando al limite in $k$ (con $n$ fissato) si ottiene
%% \[
%%   A_n \le \ell \le B_n
%% \]
%% che significa che $\ell \in [A_n, B_n]$ per ogni $n\in \NN$
%% (in particolare $\ell \in [0,1]$).
%% Visto che invece $a_n \not \in [A_n, B_n]$ risulta che per
%% ogni $n\in \NN$ si ha $\ell \neq a_n$.
%% Dunque il numero $\ell$ non è un termine della successione $a_n$
%% ovvero la funzione $a\colon \NN \to [0,1]$ non è suriettiva.
%% \end{proof}
%% \end{comment}


%%%%%%%%%%%%
%%%%%%%%%%%%
\subsection{punti limite}
%%%%%%%%%%%%
%%%%%%%%%%%%

\begin{theorem}[ponte di collegamento tra limiti di funzione e limiti di successione]%
\label{th:ponte}%
\mymark{***}%
Sia $A \subset \RR$, $f\colon A \to \RR$, sia $x_0$ un punto di accumulazione di $A$ e sia
$\ell \in [-\infty, +\infty]$.
Le due seguenti condizioni sono equivalenti:
\begin{enumerate}
\item $\displaystyle \lim_{x\to x_0} f(x) = \ell$;
\item per ogni successione $a_n\to x_0$ con $a_n\in A\setminus\ENCLOSE{x_0}$ risulta
\[
\lim_{n\to+\infty} f(a_n) = \ell. 
\]
\end{enumerate}
\end{theorem}
%
\begin{proof}
\mymark{***}
Se per $x\to x_0$ si ha $f(x)\to \ell$ e se $a_n \to x_0$ con $a_n\in A\setminus\ENCLOSE{x_0}$ la successione $f(a_n)$ non è altro che la composizione
della funzione $f$ con la funzione $n\mapsto a_n$. Si può quindi applicare
il teorema sul limite della funzione composta per ottenere che $f(a_n)\to \ell$.

Supponiamo viceversa di sapere che per ogni successione $a_n\to x_0$ si ha $f(a_n)\to \ell$. Vogliamo mostrare allora che $f(x)\to \ell$. Lo facciamo per assurdo: supponiamo che esista un intorno $U$ di $\ell$ tale che preso un qualunque intorno $V$ di $x_0$ non si abbia $f((A\setminus\ENCLOSE{x_0})\cap V)\subset U$.
Possiamo considerare per ogni $n\in \NN$ degli intorni $V_n$ sempre più piccoli. Ad esempio nel caso $x_0 \in \RR$ potremo scegliere $V_n = (x_0-1/n, x_0+1/n)$, nel caso $x_0 = +\infty$ si potrà scegliere $V_n = (n,+\infty]$ e nel caso $x_0=-\infty$ si sceglierà $V_n = [-\infty, -n)$.
Se per assurdo $f((A\setminus\ENCLOSE{x_0}\cap V_n))$ non fosse contenuto in $U$
significherebbe che per ogni $n\in\NN$ esisterebbe $a_n \in (A\setminus\ENCLOSE{x_0})\cap V_n$ tale che $f(a_n)\not \in U$. Ma allora $a_n$ risulterebbe essere una successione in
$A\setminus \ENCLOSE{x_0}$ con limite $x_0$
(in quanto per ogni intorno di $x_0$ esiste un $N$ tale che $V_N$ sia contenuto in tale intorno e per ogni $n>N$ si ha $V_n\subset V_N$)
ma $f(a_n)$ non potrebbe avere limite $\ell$
(essendo fuori dall'intorno $U$).
Ma questo nega l'ipotesi e conclude quindi la dimostrazione del teorema.
\end{proof}
\begin{proposition}[proprietà caratteristica della convergenza]
  \label{prop:convergenza}
Sia $f\colon A\subset \RR\to \RR$ e $x_0$ punto di accumulazione di $A$.
Sia $\ell\in \bar \RR$.
Se per ogni $a_n\to x_0$, $a_n\in A\setminus\ENCLOSE{x_0}$
esiste una estratta $a_{n_k}$ tale che $f(a_{n_k})\to \ell$ 
allora $f(x)\to \ell$ per $x\to x_0$.
\end{proposition}
%
\begin{proof}
  Se per assurdo non fosse $f(x) \to \ell$
  esisterebbe un intorno $U\in \B_\ell$
  per cui frequentemente $f(x) \not \in U$.
  Questo significa che esiste una successione
  $a_n\to x_0$ con $a_n\neq x_0$ tale che $f(a_n)\not \in U$.
  Ma allora da $f(a_n)$
  non è possibile estrarre una sottosuccessione
  che abbia limite $\ell$, e questo è contrario alle
  ipotesi.
\end{proof}

\begin{definition}[punti limite, limite superiore, limite inferiore]
  Sia $f\colon A\subset \RR \to \RR$ una funzione e $x_0\in \bar \RR$ 
  un punto di accumulazione per $A$.
  Una quantità $\ell\in \bar\RR$ si dice essere
  un \emph{punto limite}%
\mymargin{punto limite}%
\index{punto!limite} di $f(x)$ per $x\to x_0$ se 
  per ogni $U$ intorno di $\ell$ si ha $f(x)\in U$ 
  frequentemente per $x\to x_0$.
  
  Se denotiamo con $L\subset \bar\RR$ l'insieme
  dei punti limite possiamo definire il \emph{limite superiore} e il
  \emph{limite inferiore}
  \mymargin{limite superiore/inferiore}%
\index{limite!superiore/inferiore}
  rispettivamente come
  \[
  \limsup_{x\to x_0} f(x) = \sup L, \qquad
  \liminf_{x\to x_0} f(x) = \inf L.
  \]
\end{definition}
  
\begin{proposition}[base numerabile di intorni]%
  \label{prop:base_numerabile}%
  Per ogni $\ell\in \bar \RR$ esiste una successione $U_n$ di intorni di $\ell$ 
  con queste proprietà:
  \begin{enumerate}
    \item $U_{n+1}\subset U_n$;
    \item per ogni $U$ intorno di $\ell$ esiste $n\in \NN$ tale che $U_n\subset U$;
    \item se $a_n\in U_n$ allora $a_n\to \ell$.
  \end{enumerate}
\end{proposition}
\begin{proof}
Possiamo esplicitamente definire gli intorni $U_n$ come segue:
\[
  U_n = 
  \begin{cases}
    \openinterval{\ell-\frac 1 n}{\ell+\frac 1 n} & \text{se $\ell\in \RR$}\\ 
    \opencloseinterval{n}{+\infty} & \text{se $\ell=+\infty$}\\
    \closeopeninterval{-\infty}{-n} & \text{se $\ell=-\infty$.}
  \end{cases}
\]
La prima proprietà è ovviamente verificata.
La proprietà archimedea dei numeri reali garantisce la validità della seconda 
proprietà, da cui segue immediatamente la terza.
\end{proof}

\begin{proposition}[caratterizzazione dei punti limite]
  Sia $f\colon A\subset \RR\to \RR$, $x_0\in \bar \RR$ 
  punto di accumulazione per $A$. 
  Sono equivalenti:
  \begin{enumerate}
    \item $\ell$ è un punto limite di $f(x)$ per $x\to x_0$.
    \item esiste una successione $a_n\in A$, $a_n\neq x_0$, $a_n\to x_0$ 
    tale che $f(a_n)\to \ell$.
  \end{enumerate}
\end{proposition}
%
\begin{proof}
  Siano $U_n$ intorni di $\ell$ e $V_n$ intorni di $x_0$ definiti 
  come nella proposizione~\ref{prop:base_numerabile}.
  Se $\ell$ è un punto limite di $f(x)$ per $x\to x_0$ 
  significa che per ogni $U$ intorno di $\ell$ e per ogni $V$ 
  intorno di $x_0$ esiste $x\in A\cap V\setminus\ENCLOSE{x_0}$ 
  tale che $f(x)\in U$.
  Dunque per ogni $n$ esiste $a_n\in A \cap V\setminus\ENCLOSE{x_0}$ 
  tale che $a_n\in V_n$ e $f(a_n)\in U_n$. 
  Significa che $a_n\to x_0$, $a_n\neq x_0$, $f(a_n)\to \ell$ 
  come volevasi dimostrare.
  
  Viceversa se esiste una tale successione $a_n\to x_0$ allora 
  per ogni $V$ intorno di $x_0$ si ha definitivamente $a_n\in V$.
  E se $f(a_n)\to \ell$ per ogni $U$ intorno di $\ell$
  si ha anche $f(a_n)\in U$ definitivamente.
  Dunque esiste $n$ tale che $a_n\in V$ e $f(a_n)\in U$ confermando 
  quindi che $f(x)\in U$ frequentemente. 
\end{proof}

Molto spesso considereremo i punti limite di una successione $a_n$
per $n\to +\infty$.
In tal caso la caratterizzazione precedente ci dice che $\ell$ 
è un punto limite di $a_n$ per $n\to+\infty$ se esiste 
una successione di indici $n_k\to+\infty$ tale che $a_{n_k}\to \ell$.
Potremmo anche supporre $n_k$ strettamente crescente in quanto se 
$n_k\to \infty$ possiamo estrarre da $n_k$ una sottosuccessione 
strettamente crescente. 
Dunque l'insieme dei punti limite di una successione $a_n$ 
corrisponde all'insieme dei limiti di tutte le possibili 
sottosuccessioni di $a_n$.

\begin{example}
  La successione $a_n=(-1)^n$ ha due punti limite:
  \[
    \limsup_{n\to +\infty}(-1)^n = 1, \qquad \liminf_{n\to+\infty} (-1)^n = -1.
  \]
  Infatti la sottosuccessione dei termini con indice pari è costante $1$ mentre
  quella dei termini di indice dispari è costante $-1$.
  Non ci possono essere altri punti limite perché se ci fosse $a_{n_k}\to \ell$
  allora certamente $a_{n_k}=1$ frequentemente oppure $a_{n_k}=-1$
  frequentemente da cui o $\ell=1$ oppure $\ell=-1$.
\end{example}
%
\begin{example}
  Visto che $\#\QQ=\#\NN$ esiste una funzione $\vec a \colon \NN \to \QQ$
  surgettiva. Utilizzando la proprietà di densità dei numeri razionali
  si può dimostrare che l'insieme dei punti limite della corrispondente successione
  $a_n = \vec a(n)$ è tutto $\bar \RR$.
\end{example}

\begin{exercise}
  Si consideri la successione $a_n = \sqrt n -\lfloor \sqrt n\rfloor$.
  Calcolare 
  \[
    \liminf_{n\to+\infty} a_n, \qquad 
    \limsup_{n\to+\infty} a_n.
  \]
  Qual è l'insieme dei punti limite di $a_n$ per $n\to +\infty$?
\end{exercise}

%
\begin{theorem}[proprietà del limite superiore/inferiore]
Sia $f\colon A\subset \RR \to \RR$ e sia $x_0$ un punto di accumulazione per $A$.
Sia $L$ l'insieme dei punti limite di $f(x)$ per $x\to x_0$. 
Allora:
\begin{enumerate}
  \item %1
  $L\neq \emptyset$;

  \item %2
  $\displaystyle\limsup_{x\to x_0} f(x) \ge \liminf_{x\to x_0} f(x)$;

  \item %3
  se $\displaystyle\limsup_{x\to x_0} f(x) = \liminf_{x\to x_0} f(x) = \ell$ 
  allora $\displaystyle\lim_{x\to x_0} f(x) = \ell$;

  \item %4
  l'insieme dei punti limite è chiuso per passaggio al limite:
  se $\ell_k\in L$ e $\ell_k \to \ell$ per qualche $\ell \in \bar \RR$ 
  allora $\ell \in L$;

  \item %5
  $\displaystyle\limsup_{x\to x_0} f(x)$ e $\displaystyle\liminf_{x\to x_0} f(x)$ sono punti limite;

  \item %6
  la condizione
  $\displaystyle\limsup_{x\to x_0} f(x) = \ell$ è equivalente a
  \[
  \begin{cases}
   \forall q > \ell \colon f(x) < q \text{ definitivamente,} \\
   \forall q < \ell \colon f(x) > q \text{ frequentemente,}
  \end{cases}
  \]
  e la condizione $\displaystyle\liminf_{x\to x_0} f(x) = \ell$ è equivalente a
  \[
  \begin{cases}
  \forall q > \ell \colon f(x) < q \text{ frequentemente,} \\
  \forall q < \ell \colon f(x) > q \text{ definitivamente;}
  \end{cases}
  \]

  \item %7
  se $a_n$ è una successione 
  \[
    \limsup_{n\to +\infty} a_n = \lim_{n\to +\infty} \sup_{k\ge n} a_k,
    \quad
    \liminf_{n\to +\infty} a_n = \lim_{n\to +\infty} \inf_{k\ge n} a_k.
  \]
 \end{enumerate}
\end{theorem}
%
\begin{proof}
  Per quanto riguarda il punto 1.\
  il teorema~\ref{th:bolzano_weierstrass} di Bolzano-Weierstrass
  garantisce che sia $L\neq \emptyset$ e
  quindi $\sup L \ge \inf L$ da cui discende il punto 2.

  Per il punto 3.\ se $\limsup = \liminf = \ell$ significa che $L=\ENCLOSE{\ell}$
  ha un solo elemento. Ma per il corollario al teorema di Bolzano Weierstrass
  sappiamo che da ogni successione $f(a_n)$ con $a_n\to x_0$ 
  è possibile estrarre una sottosuccessione $f(a_{n_k})$ regolare. 
  Il limite di tale sottosuccessione deve essere un elemento di $L$
  e quindi non può che essere $\ell$. 
  Allora per la proposizione~\ref{prop:convergenza}
  deduciamo che l'intera successione ha limite $\ell$.

  Per il punto 4.\ sia $\ell_k\in L$ una successione di punti limite
  e sia $\ell\in \bar \RR$ tale che $\ell_k \to \ell$ per $k\to +\infty$.
  Ora consideriamo la successione $U_n$ di intorni di $\ell$ 
  data dalla proposizione~\ref{prop:base_numerabile}.
  Visto che $\ell_k\to \ell$ per ogni $n$ deve esistere $k_n$ tale 
  che $\ell_{k_n}\in U_n$. Senza perdita di generalità possiamo 
  supporre direttamente che sia $\ell_n\in U_n$.
  Consideriamo anche gli intorni $V_n$ di $x_0$ dati sempre 
  dalla proposizione~\ref{prop:base_numerabile}.
  Per ogni $\ell_n \in L$ deve esistere una successione $a_k\to x_0$ 
  con $a_k\neq x_0$ tale che $f(a_k)\to \ell_n$.
  Ma allora certamente esiste $k=k_n$ tale che $a_{k_n}\in V_n$ 
  e $f(a_{k_n})\in U_n$. 
  Posto $b_n = a_{k_n}$ abbiamo trovato una successione $b_n$ 
  tale che $b_n\to x_0$ (in quanto $b_n\in V_n$) e $f(b_n)\to \ell$
  in quanto $f(b_n)\in U_n$. Dunque $\ell$ è anch'esso un punto limite. 

  Per il punto 5
  visto che $L\neq \emptyset$ la caratterizzazione del $\sup$
  garantisce che esista una successione $\ell_n\in L$ tale che $\ell_n \to \sup L$.
  Ma allora per il punto precedente si deve avere $\sup L \in L$.
  Lo stesso per l'$\inf$.

  Per il punto 6.\ facciamo la dimostrazione per il $\limsup$ (per il $\liminf$
  sarà analogo). Se $\limsup f(x) = \ell$ e se fosse frequentemente
  $f(x) \ge q > \ell$ allora esisterebbe una successione $a_n\to x_0$, 
  $a_n\neq x_0$ tale che $f(a_n)\ge q$.
  Tale successione avrebbe una estratta regolare
  che ha limite $\ge q> \ell$, il che è assurdo.
  Dunque definitivamente deve essere $f(x) < q$ per ogni $q>\ell$.
  Se invece fosse definitivamente $f(x) \le q < \ell$ per ogni
  successione $a_n\to x_0$ con $a_n\neq x_0$ si dovrebbe avere 
  $f(a_n) \le q$ e questo contraddice il fatto che $\ell>q$ 
  è un punto limite.

  Viceversa se per ogni $q>\ell$ si ha $f(x)<q$ definitivamente,
  significa che per ogni successione $a_n\to x_0$ si ha
  $f(a_n) < q$ definitivamente e quindi se $f(a_n)\to \ell$
  dovrà essere $\ell\le q$. 
  Dunque ogni $q>\ell$ è un maggiorante di $L$ e $\sup L\le \ell$.
  Se poi per ogni $q<\ell$ si ha $f(x)\ge q$ frequentemente
  significa che esiste una successione $a_n\to x_0$, $a_n\neq x_0$ 
  tale che $f(a_n)\ge q$. 
  Per il teorema di Bolzano-Weierstrass si potrà trovare 
  una estratta convergente $f(a_{n_k})\to \ell'\ge q$.
  Dunque per ogni $q<\ell$ risulta
  $\sup L \ge q$ da cui in definitiva $\sup L = \ell$. 

  Per il punto 7.\ facciamo la dimostrazione per il $\limsup$ (per il $\liminf$
  sarà analogo).
  Posto $A_n = \sup_{k\ge n} a_k$ osserviamo che $A_n$ è decrescente
  in quanto all'aumentare di $n$ l'insieme
  $\ENCLOSE{k\colon k \ge n}$ decresce (nel senso dell'inclusione insiemistica)
  e quindi
  il $\sup$ non aumenta.
  Dunque $\ell=\lim A_n$ esiste certamente in $\bar \RR$.
  Inoltre per definizione di limite per ogni $\ell'>\ell$
  si dovrà avere $A_n < \ell'$ definitivamente e quindi
  (essendo ovviamente $a_n \le A_n$) si avrà
  anche $a_n < \ell'$ definitivamente.
  Viceversa preso $\ell'<\ell$ si avrà definitivamente
  $A_n > \ell'$. Ma questo significa che esiste $k>n$
  per cui si ha $a_k > \ell'$ e quindi
  risulta $a_n > \ell'$ frequentemente.
  Per il punto precedente si può quindi concludere che
  $\ell = \limsup a_n$.
\end{proof}

\begin{theorem}[operazioni con $\limsup$ e $\liminf$]
  Siano $f,g\colon A\subset \RR\to\RR$ e sia $x_0$ un punto di accumulazione di $A$. 
  \begin{enumerate}
    \item 
      Per $x\to x_0$ si ha 
      \begin{align*}
      \limsup_{x\to x_0} (-f(x)) &= -\liminf_{x\to x_0} f(x)\\ 
      \liminf_{x\to x_0} (-f(x)) &= -\limsup_{x\to x_0} f(x);
      \end{align*}
    \item
      se $\lambda \ge 0$ allora
        \begin{align*}
        \limsup_{x\to x_0} (\lambda \cdot f(x)) &= \lambda \cdot \limsup_{x\to x_0} f(x), \\
        \liminf_{x\to x_0} (\lambda \cdot f(x)) &= \lambda \cdot \liminf_{x\to x_0} f(x);
        \end{align*}
    \item
    inoltre
    \begin{gather*}
    \limsup_{x\to x_0} (f(x) + g(x)) \le \limsup_{x\to x_0} f(x) + \limsup_{x\to x_0} g(x),\\
    \liminf_{x\to x_0} (f(x) + g(x)) \ge \liminf_{x\to x_0} f(x) + \liminf_{x\to x_0} g(x);
    \end{gather*}
    \item
    e infine se $f(x) \ge g(x)$ allora
    \[
     \limsup_{x\to x_0} f(x) \ge \limsup_{x\to x_0} g(x), \qquad \liminf_{x\to x_0} f(x) \ge \liminf_{x\to x_0} g(x).
    \]
  \end{enumerate}
\end{theorem}
%
\begin{proof}
  Per il punto 1.\ basti osservare che se la funzione $f(x)$ ha $L$ come
  insieme dei punti limite allora la funzione $-f(x)$ ha $-L$ come punti
  limite. Quindi $\sup (-L) = -\inf L$ e $\inf(-L) = -\sup L$.

  Per il punto 2.\ si osservi che se $L$ è l'insieme dei punti limite
  di $f(x)$ allora l'insieme dei punti limite di $\lambda f(x)$ è $\lambda L$.
  Se $\lambda \ge 0$ si ha dunque $\sup \lambda L = \lambda \sup L$
  e $\inf \lambda L = \lambda \inf L$ (se $\lambda<0$ invece $\inf$ e $\sup$
  si scambiano, come nel punto 1).

  Per il punto 3.\ consideriamo il caso del $\limsup$ (per il $\liminf$ sarà analogo).
  Se $\ell = \limsup(f(x)+g(x))$ significa che esiste una successione
  di $a_n\to x_0$ tale che $f(a_n)+g(a_n)\to \ell$. 
  Ma posso estrarre una sottosuccessione
  tale che anche il primo addendo $f(a_{n_k})$ abbia limite. 
  E poi posso estrarre
  una sotto-sottosuccesione in modo che anche il secondo addendo $g(a_{n_{k_j}})$
  abbia limite. 
  Dunque avremo $f(a_{n_{k_j}}) \to \ell_1$, $g(a_{n_{k_j}} \to \ell_2$
  con $\ell_1+\ell_2=\ell$.
  Ma allora per definizione di $\limsup$ si avrà $\limsup f(x) \ge \ell_1$
  e $\limsup g(x) \ge \ell_2$ da cui $\limsup f(x) + \limsup g(x) \ge \ell_1+\ell_2 = \ell$.

  Per il punto 4.\ si osserva che se $\ell = \limsup f(x)$ allora
  per ogni $q>\ell$ si ha definitivamente
  $f(x)<q$ e di conseguenza anche $g(x) < q$ definitivamente.
  Dunque $\limsup g(x) \le \ell$.
  Se invece poniamo $\ell = \liminf h(x)$ allora
  per ogni $q<l$ si ha definitivamente $g(x) \ge q$ ma
  allora anche $f(x)\ge q$ defitiviamente e quindi
  $\liminf f(x) \ge \ell$.
\end{proof}
    
\begin{theorem}[convergenza alla Cesàro]
  \label{th:criterio_cesaro}%
  \mymark{*}%
  \mymargin{Cesàro}%
\index{Cesàro}%
  \index{criterio!del rapporto alla Cesàro}%
  \index{somma!di Cesàro}%
  Sia $a_n$ una successione a termini positivi.
  \begin{enumerate}
  \item
    Se
    $   a_n \to \ell \in \bar \RR$
    allora
    \[
    \frac 1 n \cdot \displaystyle\sum_{k=1}^n a_k \to \ell.
    \]
  
  \item
    Se $a_n>0$,
    $\displaystyle\frac{a_{n+1}}{a_n} \to \ell \in [0,+\infty]$
    allora
    $\displaystyle \sqrt[n]{a_n}\to \ell$.
  \end{enumerate}
  \end{theorem}
  %
  \begin{proof}
  Dimostriamo il primo punto.
  Per ogni $q<\ell$ visto che $a_n\to \ell$ deve esistere $N$ tale che 
  $a_n\ge q$ se $n> N$. 
  Dunque 
  \[
    b_n = \frac 1 n \sum_{k=1}^{n} a_k 
    \ge \frac 1 n \sum_{k=1}^N a_k + \frac 1 n \sum_{k=N+1}^n q 
    = \frac{\sum_{k=1}^N a_k}{n} + \frac{n-N}{n}\cdot q \to q.
  \]
  Significa che per ogni $q<\ell$ si ha $\liminf b_n \ge q$ e 
  dunque $\liminf b_n \ge \ell$.
  Viceversa per ogni $q>\ell$ ragionando in maniera analoga si ottiene 
  che $b_n$ non supera una successione che tende a $q$.
  Dunque $\limsup b_n\le \ell$. 
  Mettendo insieme le cose si deduce che $\lim b_n = \ell$.
    
  Il punto 2 si può ricondurre all'1.
  Infatti
  \[
    \ln \sqrt[n]{a_n}
    = \ln \sqrt[n]{a_0\cdot \frac{a_1}{a_0} \cdots \frac{a_n}{a_{n-1}}}
    = \frac{\ln a_0 + \ln \frac{a_1}{a_0} + \dots + \ln \frac{a_n}{a_{n-1}}}{n}.
  \]
  Posto $x_n = \ln \frac{a_n}{a_{n-1}}$
  se $\frac{a_n}{a_{n-1}}\to \ell$
  si ha $x_n\to \ln \ell$ (intendendo $\ln 0 = -\infty$ e $\ln (+\infty)=+\infty$) e dunque
  \[
    \ln \sqrt[n]{a_n} = \frac{\ln a_0}{n} + \frac{x_1 + \dots + x_n}{n}
    \to \ln \ell.
  \]
  Facendo l'esponenziale di ambo i membri si trova il risultato desiderato.
  \end{proof}
  
  \begin{exercise}\label{ex:7340098}
  Si applichi il risultato precedente per
  verificare che
  \[
     \lim \sqrt[n]{n} = 1
  \]
 e
 \index{fattoriale!stima asintotica}%
 \[
   \lim \frac{n}{\sqrt[n]{n!}} = e.
 \]
 \end{exercise}
 \mynote{Nell'esercizio\ref{ex:7340098} si dimostra che 
 \[
  \sqrt[n]{n!} \sim \frac{n}{e}  
  \qquad\text{per $n\to +\infty$.}
 \]
Si faccia attenzione che non è possibile elevare alla $n$ 
entrambi i lati di questa stima asintotica.
Infatti nel Teorema~\ref{th:stirling} troveremo questa stima 
asintotica per il fattoriale:
 \[
  n! \sim \sqrt{2\pi n} \cdot \frac{n^n}{e^n}.
 \]
 Si veda l'esempio~\ref{ex:498124} per un risultato simile.%
 }

 \begin{exercise}
 Si consideri la successione
 \[
 a_n =
 \begin{cases}
    2^n &\text{se $n$ pari},\\
    n\cdot 2^n &\text{se $n$ dispari}.
 \end{cases}
 \]
 Verificare che alla successione $a_n$
  si può applicare il criterio della radice ma
  non il criterio del rapporto.
  Usare questo esempio per mostrare che le implicazioni
  enunciate nel teorema~\ref{th:criterio_cesaro} non
  possono essere invertite.
  \end{exercise}  

\subsection{il teorema degli zeri}

\begin{theorem}[degli zeri]
\mymark{***}%
\mymargin{teorema degli zeri}%
\index{teorema!degli zeri}%
\label{th:zeri}%
Sia $f\colon[a,b] \to \RR$, una funzione
continua tale che $f(a)\le 0$ e $f(b)\ge 0$.
\mynote{Se $b<a$ si intende $[a,b]=[b,a]$.}
Allora esiste
 $c\in [a,b]$ tale che $f(c)=0$.
\end{theorem}

\begin{proof}
\mymark{***}
La dimostrazione che adottiamo è di particolare rilevanza in quanto
non solo permette di dimostrare l'esistenza del punto $c$ che risolve
$f(x)=0$
ma ci presenta
un algoritmo, il \emph{metodo di bisezione}%
\mymargin{metodo di bisezione}%
\index{metodo!di bisezione},
\index{bisezione!metodo di}
che può essere effettivamente utilizzato per approssimare
tale soluzione.

Possiamo supporre senza perdere di  generalità che sia $a<b$.
Poniamo $A_0 = a$, $B_0= b$ e consideriamo il punto medio $C_0 = (A_0+B_0)/2$.
Scegliamo tra i due intervalli $[A_0, C_0]$ e $[C_0,B_0]$ quello per cui
il segno ai due estremi è discorde (o, caso fortunato, nullo).
Più precisamente se $f(C_0)\ge 0$ poniamo $[A_1,B_1] = [A_0,C_0]$ altrimenti
scegliamo $[A_1,B_1] = [C_0,B_0]$ così si ha, in ogni caso,
$f(A_1)\le 0$, $f(B_1)\ge 0$.

Consideriamo il punto medio $C_1$ del nuovo intervallo $[A_1,B_1]$ e ripetiamo
il procedimento indefinitamente. Quello che otteniamo sono due successioni
$A_n$, $B_n$ con queste proprietà (che potrebbero essere dimostrate per induzione):
\begin{enumerate}
\item $A_n < B_n$, $B_n - A_n = (b-a)/2^n$;
\item $A_n$ è crescente, $B_n$ è decrescente;
\item $f(A_n)\le 0$, $f(B_n)\ge 0$.
\end{enumerate}

Essendo $A_n$ monotòna sappiamo che $A_n$ converge $A_n\to c$.
Inoltre visto che $A_n \in [a,b]$ anche $c\in [a,b]$ (per la permanenza del
segno delle successione $A_n-a$ e $b-A_n$).
Passando al limite nell'uguaglianza $B_n = A_n + (b-a)/2^n$
si ottiene che anche $B_n \to c$. Essendo $f$ continua
avremo
\[
f(A_n) \to f(c), \qquad
f(B_n) \to f(c).
\]
Ma $f(A_n)\le 0$ e quindi per la permanenza del segno anche $f(c)\le 0$.
D'altra parte $f(B_n) \ge 0$ e quindi $f(c)\ge 0$.
Si ottiene dunque $f(c) = 0$, come volevamo dimostrare.
\end{proof}

\begin{example}\label{ex:75445}
Si voglia risolvere l'equazione
\[
  x^5-x-1=0.
\]
\end{example}
%
\begin{proof}[Svolgimento]
Posto $f(x) = x^5-x-1$ è chiaro che la funzione $f\colon \RR\to \RR$
è continua (in quanto composizione di funzioni continue).
Osserviamo che $f(0) = -1$ e $f(2)=29$, dunque la funzione
soddisfa le ipotesi del teorema degli zeri sull'intervallo $[0,2]$.
Sappiamo quindi che l'equazione in questione ha almeno una soluzione
in tale intervallo.

Utilizzando il metodo di bisezione possiamo determinare una soluzione
con precisione arbitraria. Posto $A_0=0$, $B_0=2$ abbiamo verificato che
$f(A_0)<0$ e $f(B_0)>0$.
Prendiamo il punto
medio $C_0=1$ e calcoliamo la funzione: $f(C_0)=-1 < 0$. Sappiamo
allora che una soluzione deve essere compresa nell'intevallo
$[A_1,B_1] = [C_0,B_0] = [1,2]$ perché anche in tale intervallo valgono le ipotesi
del teorema degli zeri.
Il punto medio di tale intervallo è $C_1=3/2 = 1.5$
e risulta $f(3/2) = 163/32>0$ dunque l'intervallo successivo
che andremo a considerare è $[A_2,B_2]=[1,3/2]$.
Per non dover lavorare con troppe cifre decimali invece di suddividere
esattamente a metà quest'ultimo intervallo consideriamo un punto
intermedio $C_2 = 6/5 = 1.2$ dove si ha $f(C_2)=901/3125>0$.
Sappiamo allora che una soluzione è compresa nell'intervallo
$[A_3,B_3] = [1,1.2]$. Prendiamo il punto medio $C_3=11/10=1.1$
e troviamo $f(C_3) = -48949/10^5 <0$. Abbiamo quindi ottenuto
che esiste $x\in (1.1,1.2)$ tale che $f(x)=0$. Sappiamo quindi
che $\abs{x-1.15} < 0.05$ cioè abbiamo trovato $x$ con un errore
inferiore a $0.05$.

Con molta pazienza si può procedere
con il metodo di bisezione fino ad arrivare a verificare
che $f(116/10^2)$ $=$ $-596583424/10^{10}<0$ e $f(117/10^2)=224480357/10^{10}>0$ da cui
si ottiene che una soluzione è compresa tra $1.16$ e $1.17$ con un errore
inferiore a $0.005$.
Con il calcolatore (si veda ad esempio il codice a pagina \pageref{code:bisection})
si possono ottenere più cifre significative: $x=1.1673039782614187\ldots$
\end{proof}

\begin{table}
\begin{center}
\begin{tabular}{r}
$\sqrt 2 \approx $ \ttfamily\footnotesize 
1.4142135623 7309504880 1688724209 6980785696 7187537694 \\ \ttfamily\footnotesize
8073176679 7379907324 7846210703 8850387534 3276415727 \\
% 3501384623 0912297024 9248360558 5073721264 4121497099 \\ \ttfamily\footnotesize
% 9358314132 2266592750 5592755799 9505011527 8206057147 \\ \ttfamily\footnotesize
% 0109559971 6059702745 3459686201 4728517418 6408891986 \\ \ttfamily\footnotesize
% 0955232923 0484308714 3214508397 6260362799 5251407989 \\ \ttfamily\footnotesize
% 6872533965 4633180882 9640620615 2583523950 5474575028 \\ \ttfamily\footnotesize
% 7759961729 8355752203 3753185701 1354374603 4084988471 \\ \ttfamily\footnotesize
% 6038689997 0699004815 0305440277 9031645424 7823068492 \\ \ttfamily\footnotesize
% 9369186215 8057846311 1596668713 0130156185 6898723723 \\ \ttfamily\footnotesize
% 5288509264 8612494977 1542183342 0428568606 0146824720 \\ \ttfamily\footnotesize
% 7714358548 7415565706 9677653720 2264854470 1585880162 \\ \ttfamily\footnotesize
% 0758474922 6572260020 8558446652 1458398893 9443709265 \\ \ttfamily\footnotesize
% 9180031138 8246468157 0826301005 9485870400 3186480342 \\ \ttfamily\footnotesize
% 1948972782 9064104507 2636881313 7398552561 1732204024 \\ \ttfamily\footnotesize
% 5091227700 2269411275 7362728049 5738108967 5040183698 \\ \ttfamily\footnotesize
% 6836845072 5799364729 0607629969 4138047565 4823728997 \\ \ttfamily\footnotesize
% 1803268024 7442062926 9124859052 1810044598 4215059112 \\ \ttfamily\footnotesize
% 0249441341 7285314781 0580360337 1077309182 8693147101 \\ \ttfamily\footnotesize
% 7111168391 6581726889 4197587165 8215212822 9518488472 \\
\\
$\sqrt 3 \approx$ \ttfamily\footnotesize 
1.7320508075 6887729352 7446341505 8723669428 0525381038 \\ \ttfamily\footnotesize
0628055806 9794519330 1690880003 7081146186 7572485757 \\
\\
$\phi = \frac{\sqrt 5+1}{2} \approx $ \ttfamily\footnotesize 
1.6180339887 4989484820 4586834365 6381177203 0917980576 \\ \ttfamily\footnotesize
2862135448 6227052604 6281890244 9707207204 1893911375
  \end{tabular}
\end{center}
\caption{Le prime 100 cifre decimali di alcune
costanti calcolate con il metodo di bisezione usato nella dimostrazione
del teorema~\ref{th:zeri}.
Si veda il codice a pagina~\pageref{code:bisection}.}
\label{fig:cifre_sqrt2}
\index{$\sqrt 2$!cifre decimali}
\index{cifre!$\sqrt 2$}
\end{table}

\begin{theorem}[teorema dei valori intermedi]
\label{th:valori_intermedi}%
\mymark{**}%
\mymargin{proprietà dei valori intermedi}%
\index{proprietà!dei valori intermedi}%
\index{teorema!dei valori intermedi}%
Sia $I\subset \RR$ un intervallo e $f\colon I \to \RR$ una
funzione continua.
Allora se $f$ assume due valori $y_1$ e $y_2$ allora $f$
assume anche tutti i valori intermedi tra $y_1$ e $y_2$.
Detto altrimenti: una funzione continua
manda intervalli in intervalli.
\end{theorem}
%
\begin{proof}
Se $y_1$ e $y_2$ sono valori assunti da $f$ significa
che esistono $x_1,x_2 \in I$ tali che $f(x_1)= y_1$ e $f(x_2)=y_2$.
Allora scelto $y$ si consideri la funzione $g(x) = f(x)-y$.
Se $y$ è intermedio tra $y_1$ e $y_2$ la funzione $g$ assumerà
segni opposti in $x_1$ e $x_2$ e dunque, per il teorema degli zeri,
dovrà esserci un punto $x$ in cui $g$ si annulla. In tale punto
si avrà dunque $f(x)=y$, come volevamo dimostrare.
\end{proof}

\begin{lemma}
\label{ex:inversa_monotona}%
Se $I$ è un intervallo di $\RR$ ogni funzione $f\colon I \to \RR$
iniettiva e continua è strettamente monotona.
\end{lemma}
%
\begin{proof}
Si può osservare che una funzione è strettamente monotona se mantiene i valori
intermedi cioè se dati tre punti
$x<y<z$ risulta sempre che $f(y)$ è un valore intermedio tra $f(x)$ e $f(z)$:
\[
  f(x)< f(y) <f(z) \qquad\text{oppure} \qquad f(x)> f(y) > f(z).
\]
Se ciò non accadesse, ad esempio se fosse $f(y)>f(z)>f(x)$ con $x<y<z$
allora per la continuità di $f$ dovrebbe esistere un valore intermedio
tra $x$ e $y$ in cui la funzione assume il valore $f(z)$. Ma allora la funzione
non sarebbe iniettiva.
\end{proof}

%%%%%%%%%%%%%%%%%%%%%%%%
%%%%%%%%%%%%%%%%%%%%%%%%
%%%%%%%%%%%%%%%%%%%%%%%%
%%%%%%%%%%%%%%%%%%%%%%%%
\section{il teorema di Weierstrass}
    
Ricordiamo che nella definizione~\ref{def:funzione_limitata}
abbiamo definito i concetti di massimo, minimo, estremo superiore
e inferiore di una funzione a valori reali.

\begin{lemma}[successioni minimizzanti/massimizzanti]%
\mymargin{successioni mi\-ni\-miz\-zan\-ti}%
\index{successione!minimizzanti}%
\index{successione!massimizzanti}%
Sia $A$ un insieme non vuoto e
sia $f\colon A \to \RR$ una funzione. Allora esistono
due successioni $a_n$ e $b_n$ di punti di $A$ tali che
\[
  \lim_{n\to +\infty} f(a_n) = \inf f(A), \qquad
  \lim_{n\to +\infty} f(b_n) = \sup f(A).
\]
\end{lemma}
\mymargin{successioni mi\-ni\-miz\-zan\-ti}
%
\begin{proof}
Ricordiamo che $f(A) = \ENCLOSE{f(x)\colon x \in A}$ è l'immagine
della funzione $f$. Facciamo la dimostrazione per l'estremo inferiore,
risultato analogo si potrà ottenere per l'estremo superiore.

Sia $m=\inf f(A)$.
Se $m=-\infty$ significa che $f(A)$ non è inferiormente limitato,
in particolare per ogni $n\in \RR$ esiste $a_n$ tale che
$f(a_n) < - n$.
Dunque (per confronto) $f(a_n) \to -\infty$
come volevamo dimostrare.

Se $m\in \RR$ per le proprietà caratterizzanti l'estremo inferiore
sappiamo che per ogni $\eps>0$ esiste $a\in A$ tale che
$f(a) < m + \eps$.
Per ogni $n\in\NN$ possiamo scegliere $\eps=1/n$ e ottenere quindi
una successione $a_n$ tale che $f(a_n) < m + 1/n$.
D'altra parte essendo $m$ un minorante di $f(A)$ sappiamo che
$m \le f(a_n)$.
Abbiamo dunque $m \le f(a_n) < m+ 1/n$ e per il teorema dei
carabinieri possiamo quindi concludere che $f(a_n) \to m$
per $n\to +\infty$.
\end{proof}

\begin{theorem}[Weierstrass]%
\label{th:weierstrass}%
\mymark{***}%
\mymargin{teorema di Weierstrass}%
\index{teorema!di Weierstrass}%
\mynote{vedi note storiche a pag~\pageref{note:isoperimetrico}}%
Siano $a,b\in \RR$ e $f\colon [a,b]\to \RR$ una funzione continua.
\mynote{Ricordiamo che se $b<a$ si intende $[a,b]=[b,a]$ dunque 
l'intervallo $[a,b]$ non è vuoto.}%
Allora esistono punti di massimo e di minimo per $f$ su $[a,b]$.
\end{theorem}
%
\begin{proof}
\mymark{***}%
Dimostriamo solamente che $f$ ha minimo, per il massimo la dimostrazione procede
infatti in maniera del tutto analoga.

Sia $m= \inf f([a,b])$.
Per il lemma precedente sappiamo che esiste una successione $a_n$ minimizzante ovvero tale che
$a_n \in A$ e $f(a_n)\to m$ per $n\to +\infty$.

Per il teorema di Bolzano-Weierstrass dalla successione $a_n$ possiamo estrarre una sottosuccessione 
$a_{n_k}$ convergente: $a_{n_k} \to x_0$.
Visto che $a_{n_k} \in [a,b]$ si avrà, per il teorema della permanenza del segno, anche 
$x_0 \in [a,b]$ (si applichi la permanenza del segno alle successioni $a_{n_k}-a$ e $b-a_{n_k}$).

Dunque abbiamo una successione $a_{n_k}\to x_0$ con $a_{n_k}\in [a,b]$ e
$x_0 \in [a,b]$. Essendo $f$ continua si avrà dunque $f(a_{n_k}) \to f(x_0)$.
Ma noi sapevamo che $f(a_n)\to m$ e dunque anche $f(a_{n_k}) \to m$.
Concludiamo quindi che $f(x_0) = m$ cioè $m$, l'estremo inferiore,
è un valore assunto dalla funzione ed è quindi un minimo.
Dal canto suo $x_0$ è un punto di minimo assoluto.
\end{proof}

\begin{corollary}[limitatezza delle funzioni continue]
Sia $f\colon [a,b]\to \RR$ una funzione continua. Allora $f$ è limitata.
\end{corollary}
\begin{proof}
Visto che $f$ ha massimo $M$ e minimo $m$ si ha $f(x)\in [m,M]$ per ogni $x\in[a,b]$.
Ovviamente $m>-\infty$ e $M<+\infty$ in quanto $m$ e $M$ sono valori della funzione $f$.
\end{proof}


%%%%%%%%%%%
%%%%%%%%%%%
\section{funzioni trigonometriche, radianti}
%%%%%%%%%%%
%%%%%%%%%%%

Nel capitolo~\ref{sec:avvolgimento} abbiamo definito le 
funzioni $\sin$ e $\cos\colon \RR\to \RR$ con periodo $\tau>0$ (arbitrario)
in modo tale che la funzione $\phi\colon \RR\to\RR^2$, $\phi(t)=(\cos t, \sin t)$ 
percorra, con moto circolare uniforme, in senso antiorario, la circonferenza 
unitaria $\ENCLOSE{(x,y)\in \RR^2\colon x^2+y^2=1}$.
Tale funzione descrive in effetti un moto circolare uniforme 
in quanto percorre archi congruenti in tempi uguali.
\mynote{Due figure geometriche sono \emph{congruenti} se c'è una isometria 
che manda una nell'altra.
Due archi sono congruenti se hanno la stessa lunghezza e lo stesso raggio, 
ma non è necessario introdurre il concetto di \emph{lunghezza d'arco} per parlare 
di congruenza.
}

Vogliamo ora mostrare che è possibile scegliere la costante $\tau$
in modo tale che la linea descritta da $\phi$ percorra la circonferenza 
con \emph{velocità} unitaria.
\mynote{Ricordiamo che $\tau$ è il periodo delle funzioni $\phi$,
$\cos$ e $\sin$}%
Se seguo il punto $f(t)$ a partire dall'istante $t=0$ per un breve tempo 
$\Delta t\neq 0$
osservo che il punto si sposta tra $f(0) = (1,0)$ e 
$f(\Delta t) = (\cos \Delta t, \sin \Delta t)$.
\mynote{La notazione $\Delta t$ indica un qualunque numero positivo.
Usiamo tale notazione perché tale numero rappresenta 
una differenza ($\Delta$ è una lettera $D$ greca)
di tempi tra loro vicini.}%
Intuitivamente quando $t=0$ il punto $f(t)$ si muove verticalmente, 
verso l'alto. 
Lo spazio percorso $\Delta s$ sarà quindi approssimativamente 
uguale alla variazione della coordinata $y$: 
$\Delta y = \sin \Delta t$.
\mynote{Ricordiamo che $\sin 0 = 0$}
Dunque la velocità sarà pari alla componente verticale della velocità 
e sarà, intuitivamente, il limite di $\frac{\Delta y}{\Delta t}$ ovvero 
\mynote{Vedremo tra poco che questo limite è la  
\emph{derivata} della funzione $f$ nel punto $t=0$.}
\[
\lim_{\Delta t\to 0} \frac{\sin(\Delta t)}{\Delta t}.  
\]

Dal punto di vista formale non è del tutto ovvio che questo limite esista.
Nel seguito lo dimostriamo, con un metodo molto simile a quello utilizzato 
per definire la costante $e$ di Nepero.
Così come l'esponenziale reale $x\mapsto a^x$ 
è un isomorfismo tra il gruppo additivo di $\RR$ e il gruppo moltiplicativo 
di $\RR^+$, così la funzione $\phi(t) = (\cos(t), \sin(t))$ 
risulta essere un isomorfismo 
tra il gruppo additivo $\RR$ e il gruppo delle rotazioni del piano che può essere 
rappresentato dall'insieme $U$ dei numeri complessi unitari.
Abbiamo già visto che affinché il punto che si muove con legge 
oraria $x\mapsto a^x$ abbia velocità $1$ per $x=0$ 
si deve porre $a=e$, la base dei logaritmi naturali.
Allo stesso modo vedremo ora che affinché la velocità 
del punto $t\mapsto \phi(t)$ 
sia anch'essa unitaria bisogna avere $\tau = 2\pi$.
Mettendo assieme i due isomorfismi $e^t$ e $\phi(t)$ si potrà definire,
come vedremo tra poco, un isomorfismo tra il gruppo additivo $\CC$ 
e il gruppo moltiplicativo $\CC\setminus\ENCLOSE 0$.
Quello che si ottiene è l'esponenziale complesso $z\mapsto e^z$. 
La velocità di $\phi(t)$ rappresenta in effetti la derivata di tale funzione 
complessa nel punto $1$ nella direzione dell'asse immaginario 
così come la derivata della funzione esponenziale reale è la derivata 
nella direzione reale. 
Si intuisce così come le costanti fondamentali $e$ e $\pi$ siano legate 
tra loro dalla condizione di fare sì che la funzione esponenziale complessa 
sia derivabile in senso complesso ed abbia derivata complessa pari a $1$.

\subsection{il numero $\pi$}

In questo capitolo dobbiamo lavorare con le funzioni 
$\sin$, $\cos$ e $\tg=\frac{\sin }{\cos}$ introdotte 
nel capitolo~\ref{sec:avvolgimento}. 
Queste funzioni sono state definite con un periodo $\tau>0$
fissato in modo arbitrario. 
Scriveremo 
$\sin_\tau$ 
e $\cos_\tau$ invece di $\sin$ e $\cos$ se è utile mettere in evidenza 
la dipendenza da $\tau$.

\begin{theorem}[limite notevole $\sin$]
Per ogni $\tau>0$, $x\in \RR$, $x\neq 0$, la successione
\[
    a_n = \frac{\sin_\tau \frac xn} {\frac xn}
\]
converge ad un limite positivo (e finito).
\end{theorem}
%
\begin{proof}
Faremo uso della funzione $\tg x= \frac{\sin x}{\cos x}$.
Useremo le formule di addizione:
\begin{align*}
 \sin(x+y) &= \sin x\cos y + \cos x \sin y,\\
 \cos(x+y) &= \cos x \cos y - \sin x \sin y  
\end{align*}
da cui, facendo il rapporto e dividendo numeratore 
e denominatore per $\cos x \cos y$ si ottiene:
\[
  \tg(x+y) = \frac{\tg x + \tg y}{1-\tg x \tg y}.
\]
Useremo anche il fatto che sull'intervallo 
$\closeinterval{0}{\frac \tau 4}$
\mynote{ricordiamo che $\tau$ è il periodo (scelto arbitrariamente)
delle funzioni $\sin$ e $\cos$} 
la funzione $\sin$ è crescente, la funzione $\cos $
è decrescente ed entrambe hanno valori compresi 
tra $0$ e $1$.
Tutte queste proprietà sono garantite 
dal teorema~\ref{th:proprieta_trigonometriche}.

\emph{Passo 1:}
vogliamo dimostrare che se $n$ è un intero positivo 
e $x>0$ è tale che $0\le nx <\frac \tau 4$ allora si ha
\begin{equation}\label{eq:4813509}
  \sin n x \le n\sin x\le \tg nx.
\end{equation}
Possiamo dimostrare per induzione l'implicazione 
$nx\le \frac \tau 4 \implies$~\eqref{eq:4813509}.
Se $n=1$ è ovvio.
Supponiamo allora che \eqref{eq:4813509} 
sia valida per un certo $n$ e che  
sia $(n+1)x<\frac \tau 4$. 
Allora da un lato si ha
\begin{align*}
  \sin (n+1)x 
  &= \sin nx \cdot \cos x + \cos nx\sin x
  \le \sin nx + \sin x
\end{align*}
e usando l'ipotesi induttiva $\sin nx\le n\sin x$ 
si ottiene 
\[
 \sin(n+1)x \le n\sin x+\sin x = (n+1)\sin x  
\]
che è la prima disuguaglianza che dovevamo dimostrare.
Per l'altra disuguaglianza osserviamo che si ha 
\begin{align*}
  \tg(n+1)x 
  &= \frac{\tg nx + \tg x}{1-\tg nx \tg x}
  \ge \tg nx + \tg x
  \ge \tg nx + \sin x
\end{align*}
e applicando l'ipotesi induttiva 
$\tg nx\ge n\sin x$ si ottiene 
\[
 \tg(n+1)x 
 \ge n\sin x +\sin x 
 =   (n+1)\sin x
\]
come volevamo dimostrare.

\emph{Passo 2:}
dimostriamo che se $x>0$,
posto
\[
  a_n = \frac{\sin \frac x n}{\frac x n} = \frac{n\sin \frac x n}{x}
\]
si ha $a_{n+1}\ge a_n$ per ogni $n \ge \frac{4x}{\tau}$
e dunque, essendo definitivamente monotona, 
la successione ammette limite 
(grazie al teorema~\ref{th:limite_monotona}):
$a_n \to \ell$ per $n\to +\infty$.
Posto $y=\frac{x}{n(n+1)}$ basta mostrare che 
se $\frac{x}{n} \le \frac{\tau}{4}$, si ha
\[
  (n+1) \sin ny \ge n \sin (n+1)y.  
\]
Ma usando $n\sin y\le \tg ny$ da~\eqref{eq:4813509} si ha 
\begin{align*}
  n \sin(n+1)y
  &=n(\sin ny\cos y + \cos ny \sin y) \\
  &\le n\sin ny + n\sin y\cos ny\\
  &\le n\sin ny + \tg ny\cos ny\\
  &= (n+1)\sin ny
\end{align*}
che è quanto volevamo dimostrare.

\emph{Passo 3.}
vogliamo mostrare che il limite $\ell=\lim a_n$ è positivo e 
finito.
Basta osservare che se $x>0$ e $\frac x m <\frac \tau 4$
per $m$ intero sufficientemente grande, 
usando~\eqref{eq:4813509} con 
con $\frac x {mn}$ al posto di $x$,
si ha $\sin \frac x m \le n\sin\frac x {m n} \le \tg \frac{x}{m}$ 
e quindi
\[
  \frac{\sin \frac x m}{\frac x m} 
  \le \frac{\sin \frac x {mn}}{\frac x {mn}} 
  \le \frac{\tg \frac x m}{\frac x m}.
  \]
Sappiamo che la sottosuccessione $a_{mn}\to \ell$ e, dunque,
dovrà essere 
\[
  0 <
  \frac{\sin \frac x m}{\frac x m} 
  \le \ell
  \le \frac{\tg \frac x m}{\frac x m} < +\infty.
\]

\emph{Conclusione.}
Abbiamo dimostrato il teorema per ogni $x>0$.
Se $x<0$ il risultato si ottiene per simmetria usando: 
$\sin(-x)=-\sin x$.
\end{proof}

Se applichiamo il teorema precedente 
con $\tau=1$ e $x=1$ otteniamo che la successione 
$a_n = n\cdot \sin_1\enclose{\frac 1 n}$ converge 
ad una costante universale.
In particolare avendo posto la misura dell'angolo giro pari a 
$\tau=1$, l'angolo di misura $\frac 1 n$ 
è l'angolo formato dal lato di un $n$-agono 
regolare inscritto nella circonferenza unitaria.
Dunque $a_n$ non è altro che il perimetro di tale 
$n$-agono regolare e il limite a cui tende deve essere,
geometricamente, la lunghezza della circonferenza unitaria.

Questo dà significato geometrico alla seguente definizione.

\begin{definition}[$\pi$ costante fondamentale della trigonometria]%
  \label{def:pi}%
Definiamo
\[
 \pi 
  \defeq \frac 1 2 \lim_{n\to +\infty} n\cdot \sin_1 \enclose{\frac 1 n}.
\]
dove $\sin_1$ è la funzione $\sin$ costruita nel 
teorema~\ref{def:sin_cos} con periodo $\tau=1$.
\end{definition}

\begin{theorem}[limite notevole $\sin$]
  \label{th:limite_notevole_sin}%
Sia $\sin_\tau$ la funzione seno definita 
dal teorema~\ref{def:sin_cos} con periodo $\tau>0$.
Per ogni $x\in\RR$, $x\neq 0$ si ha 
\[
\lim_{x\to 0}\frac{\sin_{\tau} x}{x}
= \frac{2\pi}{\tau}.
\]
\end{theorem}
%
\begin{proof}
Risulta ovviamente $\sin_\tau(\tau x) = \sin_1 x$ in quanto 
la funzione $\sin_\tau(\tau x)$ ha periodo $1$ e soddisfa tutte le proprietà 
enunciate nel teorema~\ref{def:sin_cos}.
Ora per ogni $x>0$ esiste $n\in \NN$ tale che 
$\frac \tau {n+1} \le x \le \frac \tau n$ 
e se $nx<\frac \tau 4$ usando la monotonia della funzione $\sin_\tau$ 
si ottiene:
\begin{align*}
  n \sin_1 \frac 1 {n+1}
  = n \sin_\tau \frac \tau {(n+1)}
  \le n \sin_\tau x 
  = n \sin_1 \frac x \tau 
  \le n \sin_1 \frac 1 n.
\end{align*}
Ovviamente se $x\to 0^+$ si ha $n=n(x)\to +\infty$.
Per come abbiamo definito $\pi$ 
sappiamo che $n \sin_1 \frac 1 n\to 2\pi$ per $n\to +\infty$
e visto che $n\sim n+1$ entrambi i lati della precedente equazione 
tendono allo stesso limite $2\pi$ e dunque, per confronto, 
\[
  \lim_{x\to 0^+} \tau \cdot \frac{\sin_\tau x}{x} = 2\pi.  
\]
Per simmetria lo stesso limite vale quanto $x\to 0^-$
e si ottiene dunque il risultato voluto.
\end{proof}

D'ora in avanti sceglieremo $\tau=2\pi$ come periodo 
delle funzioni trigonometriche.
E d'ora in avanti si intenderà $\sin = \sin_{2\pi}$, 
$\cos = \cos_{2\pi}$. Si avrà quindi il limite notevole 
\[
   \lim_{x\to 0} \frac{\sin x}{x} = 1.  
\]

Quando introdurremo le derivate potremo anche osservare che 
questa scelta del periodo $\tau=2\pi$ 
fa sì che il punto di coordinate
$(\cos t,\sin t)$ si muova con velocità unitaria lungo 
la circonferenza unitaria. 
Dunque tale punto percorre un arco di lunghezza $t$ nel tempo $t$.
In particolare in un periodo $\tau=2\pi$ il punto percorre l'intera 
circonferenza e dunque risulta che $2\pi$ è la lunghezza 
della circonferenza unitaria (ovvero $2\pi r$ è la lunghezza della circonferenza 
di raggio $r$).
Al tempo $t=1$ il punto $(\cos t,\sin t)$ individua un angolo che 
viene chiamato \emph{radiante} in quanto corrisponde ad un arco di lunghezza 
pari al raggio (unitario). 
Dunque il \emph{radiante} è l'unità di misura naturale per misurare gli angoli 
e le funzioni $\sin$ e $\cos$ d'ora in poi saranno fissate con tale unità di misura.

\newsavebox{\qrfigtrigo}\sbox{\qrfigtrigo}{\myurlhere{figtrigo}{funzioni trigonometriche}}%
\begin{figure}
  \centering%
  \begin{tikzpicture}[scale=0.75]
  \draw[->] (-6,0) -- (6,0) node[right] {$x$};
  \draw[->] (0,-3) -- (0,3) node[above] {$y$};
  \foreach \x/\xtext in
    {{pi/2}/{\frac \pi 2}, {pi}/{\pi}, {3*pi/2}/{\frac{3}{2}\pi},
    {-pi/2}/{-\frac \pi 2}, {-pi}/{-\pi}, {-3*pi/2}/{-\frac{3}{2}\pi}} {
    \draw[shift={(\x,0)},lightgray] (0,-3) -- (0,3);
    \draw[shift={(\x,0)}] (0pt,2pt) -- (0pt,-2pt) node[below] {$\xtext$};
  }
  \foreach \y in {1, -1} {
    \draw[shift={(0,\y)},lightgray] (-6,0) -- (6,0);
  }
  \draw (0,1) node [above right] {1};
  \draw (0,-1) node [right] {-1};
  \draw[domain=-rad(atan(3)):rad(atan(3)),smooth,variable=\x,brown,thick] plot ({\x},{tan(deg(\x))});
  \draw[domain=pi-rad(atan(3)):pi+rad(atan(3)),smooth,variable=\x,brown,thick] plot ({\x},{tan(deg(\x))});
  \draw[domain=-pi-rad(atan(3)):-pi+rad(atan(3)),smooth,variable=\x,brown,thick] plot ({\x},{tan(deg(\x))});
  \draw[domain=-6:-2*pi+rad(atan(3)),smooth,variable=\x,brown,thick] plot ({\x},{tan(deg(\x))});
  \draw[domain=2*pi-rad(atan(3)):6,smooth,variable=\x,brown,thick] plot ({\x},{tan(deg(\x))});
  \draw[domain=-6:6,smooth,variable=\x,blue,thick] plot ({\x},{cos(deg(\x))});
  \draw[domain=-6:6,smooth,variable=\x,red,thick] plot ({\x},{sin(deg(\x))});
  \draw (3.7,-1) node[blue,below] {$y=\cos x$};
  \draw (2,0.9)  node[red,above] {$y=\sin x$};
  \draw (1.3,2.5) node[brown,right] {$y=\tg x$};
  \end{tikzpicture}
  \caption{%
  I grafici delle funzioni $\sin$, $\cos$ e $\tg$.
  \ifwidemargin\\\\\fi%
  \usebox{\qrfigtrigo}}
\end{figure}

\subsection{funzioni trigonometriche inverse}

La funzione $\sin\colon[-\pi/2,\pi/2]\to [-1,1]$ 
risulta essere strettamente crescente e bigettiva. 
Dunque restringendo il dominio all'intervallo $[-\pi/2, \pi/2]$
e il codominio all'intervallo $[-1,1]$ la funzione risulta invertibile.
La funzione inversa
\[
  \arcsin\colon[-1,1]\to [-\pi/2, \pi/2]
\]
si chiama \emph{arco seno}. 
Per definizione di funzione inversa si ha
\[
  \arcsin(\sin x) = x, \qquad \forall x \in [-\pi/2, \pi/2]
\]
e
\[
  \sin(\arcsin x) = x, \qquad \forall x \in [-1, 1].
\]

La funzione $\cos \colon[0,\pi] \to [-1,1]$ risulta essere strettamente
decrescente e bigettiva.
La funzione inversa
\[
  \arccos\colon[-1,1] \to [0,\pi]
\]
si chiama \emph{arco coseno}. Per definizione si ha
\[
  \arccos(\cos x) = x, \qquad \forall x \in [0,\pi]
\]
e
\[
    \cos(\arccos x) = x, \qquad \forall x \in [-1,1].
\]

La funzione
\index{tangente!funzione trigonometrica}
\mymargin{$\tg x$}%
\index{$\tg x$}
\[
\tg x = \frac{\sin x}{\cos x}
\]
è definita quando $\cos x\neq 0$ ovvero:
\[
  \tg \colon \RR \setminus\ENCLOSE{\frac \pi 2+ k\pi\colon k\in \ZZ} \to \RR.
\]
Se restringiamo la funzione all'intervallo $\enclose{-\pi/2, \pi/2}$ possiamo
facilmente osservare che la funzione $\tg\colon(-\pi/2,\pi/2)\to \RR$ è strettamente crescente. Inoltre se $a_n \to \pi/2$, $a_n<\pi/2$ si ha $\cos(a_n)\to 0$ (per continuità del coseno) e $\sin(a_n)\to 1$ dunque $\tg a_n\to +\infty$. Analogamente per $a_n \to -\pi/2$ si trova $\tg a_n \to -\infty$. Dunque per il teorema dei valori intermedi possiamo affermare che la funzione $\tg\colon(-\pi/2,\pi/2)\to \RR$ è suriettiva. E' quindi invertibile
e la funzione inversa
\[
  \arctg \colon \RR \to (-\pi/2,\pi/2)
\]
si chiama \emph{arco tangente}. Per definizione si ha
\[
  \arctg \tg x = x, \qquad \forall x \in (-\pi/2, \pi/2)
\]
e
\[
  \tg\arctg x = x, \qquad \forall x \in \RR.
\]

Grazie al teorema~\ref{th:monotona_continua}
visto che queste funzioni sono monotone e bigettive
possiamo affermare che
le funzioni $\arcsin$, $\arccos$ e $\arctg$ 
sono tutte funzioni continue.

\begin{exercise}[funzioni misteriose]
  Si disegni il grafico delle seguenti funzioni
  \[
    f(x) = \arcsin \sin x, \qquad 
    g(x) = \arccos \cos x, \qquad 
    h(x) = \arctg \tg x.
  \]
  Si osservi che:
  \begin{enumerate} 
    \item $f$ e $g$ sono definite su tutto $\RR$,
  mentre $h$ è definita su $\RR\setminus\ENCLOSE{\frac{\pi}2+k\pi\colon k\in \ZZ}$;
    \item $f$ e $h$ hanno valori 
  in $\closeinterval{-\frac \pi 2}{\frac \pi 2}$
  mentre $g$ ha valori in $\closeinterval{0}{\pi}$;
    \item $f$ e $g$ hanno periodo $2\pi$,
  $h$ ha periodo $\pi$;
    \item tutte e tre sono funzioni continue (sul loro dominio).
  \end{enumerate}
\end{exercise}

\begin{exercise}
Dimostrare che per ogni $x>0$ si ha
\[
  \arctg \frac 1 x = \frac \pi 2 - \arctg x.
\]
\end{exercise}

\begin{exercise}[lunghezza della circonferenza]
Si osservi che un $n$-agono regolare iscritto a una circonferenza 
di raggio $r$ ha lati di lunghezza $l_n = 2r\sin \frac{\pi}{n}$
e quindi perimetro $p_n = n\cdot l_n$.
Si verifichi che 
\[
  \lim_{n\to +\infty} p_n = 2\pi r.
\]
\end{exercise}

\section{funzioni iperboliche}

\begin{definition}[funzioni iperboliche]
Le funzioni
\emph{coseno iperbolico} e \emph{seno iperbolico}
sono definite, per ogni $x\in \RR$,
come segue:
\mymargin{$\sinh$, $\cosh$}%
\index{$\sinh$, $\cosh$}%
\index{funzioni!iperboliche}%
\index{seno!iperbolico}%
\index{coseno!iperbolico}%
\index{$\sinh$}%
\index{$\cosh$}%
\begin{equation}
\label{eq:sinh_cosh}
  \cosh x = \frac{e^x + e^{-x}}{2},
  \qquad
  \sinh x = \frac{e^x - e^{-x}}{2}.
\end{equation}
\end{definition}

\begin{theorem}[proprietà delle funzioni iperboliche]
Valgono le seguenti proprietà.
\begin{enumerate}
\item
la funzione $\sinh$ è dispari, $\cosh$ è pari:
\[
\sinh(-x) = -\sinh(x),
\qquad
\cosh(-x) = \cosh(x);
\]

\item
i punti del piano con coordinate $(\cosh x, \sinh x)$
per $x\in \RR$
sono disposti su un ramo di iperbole in quanto vale:
\[
  \cosh^2 x - \sinh^2 x = 1;
\]

\item formule di addizione:
\begin{align*}
  \cosh(\alpha+\beta) &= \cosh \alpha \cosh \beta + \sinh \alpha \sinh \beta,\\
  \sinh(\alpha+\beta) &= \sinh \alpha \cosh \beta + \cosh \alpha \sinh \beta;
\end{align*}

% \item si ha
% \begin{align*}
%   \cosh x
%   &= \sum_{k=0}^{+\infty} \frac{x^{2k}}{(2k)!}
%   = 1 + \frac{x^2}{2} + \frac{x^4}{4!} + \frac{x^6}{6!} + \dots \\
%   \sinh x
%   &= \sum_{k=0}^{+\infty} \frac{x^{2k+1}}{(2k+1)!}
%   = x + \frac{x^3}{6} + \frac{x^5}{5!} + \frac{x^7}{7!} + \dots
% \end{align*}

\item
la funzione $\sinh$ è strettamente crescente su tutto $\RR$,
la funzione $\cosh$
è strettamente crescente sull'intervallo
$[0,+\infty)$ e strettamente decrescente
nell'intervallo $(-\infty,0]$;

\item per $x\to +\infty$ si ha $\sinh x \to +\infty$, $\cosh x \to +\infty$, 
per $x\to -\infty$ si ha $\sinh x \to -\infty$, $\cosh x \to +\infty$.

\end{enumerate}
\end{theorem}

\newsavebox{\qrfigiperb}\sbox{\qrfigiperb}{\myurlhere{figiperb}{funzioni iperboliche}}%
\begin{figure}
  \centering%
  \begin{tikzpicture}[scale=0.75]
  \draw[->] (-6,0) -- (6,0) node[right] {$x$};
  \draw[->] (0,-3) -- (0,3) node[above] {$y$};
  \foreach \y in {1, -1} {
    \draw[shift={(0,\y)},lightgray] (-6,0) -- (6,0);
  }
  \draw (0,1) node [above right] {$1$};
  \draw (0,-1) node [below right] {$-1$};
  \draw[domain=-6:6,smooth,variable=\x,brown,thick] plot ({\x},{tanh(\x)});
  \draw[domain=-1.75:1.75,smooth,variable=\x,blue,thick] plot ({\x},{cosh(\x)});
  \draw[domain=-1.75:1.75,smooth,variable=\x,red,thick] plot ({\x},{sinh(\x)});
  \draw (-2.5,2) node[blue,below] {$y=\cosh x$};
  \draw (2.7,1.5)  node[red,above] {$y=\sinh x$};
  \draw (5,1) node[brown,above] {$y=\tgh x$};
  \end{tikzpicture}
  \caption{%
    I grafici delle funzioni $\sinh$, $\cosh$ e $\tgh$.
    \ifwidemargin\\\\\fi%
    \usebox{\qrfigiperb}%
  }
\end{figure}

\begin{proof}
I primi tre punti si dimostrano facilmente per verifica diretta,
utilizzando la definizione~\eqref{eq:sinh_cosh}.

% Gli sviluppi in serie si ottengono anch'essi sostituendo
% gli sviluppi dell'esponenziale nella definizione.
% Nel $\cosh$ i termini di grado dispari si cancellano, nel $\sinh$ si cancellano
% i termini di grado pari.

Per quanto riguarda la monotonia si osserva che se $x\ge 0$ ogni
addendo delle due serie esposte nel punto 4 è strettamente crescente
(in quanto i coefficienti sono tutti positivi) e dunque le somme delle serie,
cioè la funzione $\cosh$ e la funzione $\sinh$ sono strettamente crescenti
sull'intervallo $[0,+\infty)$. La funzione $\sinh$, essendo dispari,
risulta inoltre crescente anche sull'intervallo $(-\infty,0]$ e quindi
è crescente su tutto $\RR$.

Per l'ultima proprietà basterà usare la definizione~\eqref{eq:sinh_cosh}
e ricordare che (teorema~\ref{th:ordine_infinito})
se $x\to +\infty$ allora
$e^x\to +\infty$ ed $e^{-x}=\frac{1}{e^{x}} \to 0$.
\end{proof}

Osserviamo che $\cosh 0 = 1$ e, per le proprietà di monotonia viste nel teorema
precedente si ha $\cosh x \ge \cosh 0 = 1 > 0$. Dunque $\cosh x$ non si annulla
mai e si può definire per ogni $x\in \RR$ la \emph{tangente iperbolica}
\index{tangente!iperbolica}%
\mymargin{$\tgh$}%
\index{$\tgh$}%
\[
    \tgh x = \frac{\sinh x}{\cosh x}.
\]

La funzione $\sinh\colon \RR\to\RR$ è iniettiva in quanto strettamente crescente ed
è surgettiva in quanto è continua e quindi assume tutti i valori compresi tra
$\lim_{x\to+\infty} \sinh(x) = +\infty$ e $\lim_{x\to -\infty} \sinh(x) = -\infty$. 
Dunque $\sinh\colon \RR \to \RR$
è invertibile e la funzione inversa si chiama \emph{settore di seno iperbolico}
e si denota con
\mymargin{$\settsinh$}%
\index{$\settsinh$}
\index{settore!di seno iperbolico}
\[
    \settsinh \colon \RR \to \RR.
\]
Analogamente la funzione ristretta $\cosh\colon [0,+\infty)\to [1,+\infty)$ è
iniettiva in quanto strettamente crescente ed è surgettiva in quanto
è continua e assume su $[0,+\infty)$ tutti i valori compresi tra $\cosh(0)=1$ e
$\lim_{x\to +\infty} \cosh x = +\infty$.
Dunque la funzione $\cosh x$ ristretta a $[0,+\infty)\to [1,+\infty)$
è invertibile e la funzione inversa si chiama \emph{settore di coseno iperbolico}
\mymargin{$\settcosh$}%
\index{$\settcosh$}
\index{settore!di coseno iperbolico}
\[
    \settcosh \colon [1,+\infty)\to [0,+\infty).
\]

La funzione $\tgh x$ è strettamente crescente su tutto $\RR$ e assume tutti i valori strettamente compresi tra $-1$ e $1$.
La funzione inversa si chiama $\setttgh$.

\begin{exercise}
Fissato $y\in \RR$ si risolva l'equazione
\[
    \frac{e^x - e^{-x}}{2} = y
\]
riconducendola ad una equazione di secondo grado nella variabile $t=e^x$.
Si dimostri quindi che vale
\[
    \settsinh x = \ln\enclose{x + \sqrt{x^2 + 1}}.
\]
In modo analogo si dimostri che vale
\[
    \settcosh x = \ln\enclose{x + \sqrt{x^2 - 1}}
\]
e
\[
    \setttgh x = \ln \sqrt{\frac{1+x}{1-x}}.
\]
\end{exercise}

\section{l'esponenziale complesso}
\label{sec:esponenziale_complesso}%

L'esponenziale reale e le funzioni trigonometriche possono essere pensate 
come strumenti intermedi per definire un isomorfismo naturale 
da $\CC$ come gruppo additivo in $\CC$ come gruppo moltiplicativo.

Possiamo infatti definire la funzione $\exp \colon \CC \to \CC$ mediante 
la \emph{formula di Eulero}
\[
 \exp(x+iy) = e^x \cdot (\cos y + i \sin y).
\]
Questa funzione ha le seguenti proprietà:
\begin{enumerate}
  \item $\exp(z+w) = \exp z \cdot \exp w$;
  \item $\exp\bar z = \overline{\exp z}$;
  \item $\abs{\exp z} = e^{\Re z}$;
  \item $\exp x = e^x$ se $x\in \RR$.
\end{enumerate}
Visto che $\exp$ estende la funzione esponenziale reale $e^x$
sarà anche naturale usare la stessa notazione ponendo:
\mynote{%
La notazione $e^z$ non ci deve far pensare che abbiamo definito 
una operazione di elevamento a potenza tra numeri complessi. 
In effetti non è possibile definire in maniera sensata e univoca 
la potenza $z^w$ se la base $z$ non è un numero reale positivo.
Questo è legato al fatto che la funzione esponenziale $e^z$
non è univocamente invertibile e quindi non si può definire il 
logaritmo $\ln z$ se non come funzione \emph{multivoca}.
}%
\[
  e^z = \exp z = e^x\cdot (\cos y + i\sin y), \qquad \text{se $z=x+iy$}.  
\]

Si noti che la funzione esponenziale complessa è $2\pi i$ periodica, infatti 
se $z=x+iy$ si ha 
\[
\exp(z+2\pi i) 
= e^x\cdot (\cos(y+2\pi) + i\sin(y+2\pi)) 
= e^x\cdot (\cos y + i \sin y) 
= \exp(z).
\]

% Nel capitolo precedente abbiamo introdotto l'esponenziale complesso ed
% abbiamo osservato che la funzione $f\colon \RR \to \CC$ definita da
% $f(t) = e^{it}$ ha valori sulla circonferenza unitaria in quanto
% $\abs{e^{it}}=1$. Tramite la definizione~\ref{def:sincos}
% abbiamo introdotto le funzioni seno e coseno in
% modo che risulti $f(t) = \cos t + i \sin t$.
% Sappiamo che $f(0) = e^0 = 1$ e, per come abbiamo definito $\pi$,
% sappiamo che $f(\pi/2) = i$.

\subsection{rappresentazione polare dei numeri complessi}

Per come li abbiamo definiti i numeri complessi $z\in \CC$ si possono 
rappresentare nella forma:
\[
  z = x + i y.
\]
Questa rappresentazione dei numeri complessi viene chiamata \emph{cartesiana}
perché fa corrispondere ogni numero $z$ alle sue coordinate $(x,y)$ nel
piano complesso (piano di Gauss). 

Grazie alla definizione di esponenziale complesso possiamo anche 
dare una rappresentazione \emph{polare} dei numeri complessi.
Se $z=x+iy$ è un qualunque numero complesso possiamo definire 
$\rho = \abs{z} = \sqrt{x^2+y^2}$ il suo \emph{modulo} 
ovvero la distanza geometrica tra il punto $z$ del piano complesso 
e l'origine $0\in \CC$.
Se $z\neq 0$ possiamo definire la misura dell'angolo individuato 
da $z$ con l'asse delle $x$ come quell'unico 
$\theta \in \closeopeninterval{0}{2\pi}$ tale che
\[
  \begin{cases}
    x = \rho \cos \theta,\\
    y = \rho \sin \theta.
  \end{cases}
\]
Si ha quindi 
\[
  z = \rho \cdot (\cos \theta + i \sin \theta). 
\]
La coppia di numeri $(\theta,\rho)$ con $\rho>0$ e $\theta\in\closeopeninterval{0}{2\pi}$ 
si chiamano \emph{coordinate polari} del numero complesso $z$ 
e identificano univocamente $z$. 
Per la periodicità delle funzioni $\sin$ e $\cos$ risulta chiaro 
che l'angolo $\theta$ può essere sostitutito con $\theta +2k\pi$ 
per qualunque $k\in \ZZ$ lasciando invariato il punto $z$.
Se $z=0$ allora $\rho=\abs{z}=0$ e la coordinata $\theta$ è irrilevante.

La rappresentazione \emph{esponenziale} di un numero complesso 
è sostanzialmente identica alla rappresentazione polare 
ma utilizza l'esponenziale complesso invece che 
le funzioni trigonometriche. 
Essendo $e^{i\theta} = \cos \theta + i\sin \theta$
se $z=\rho\cdot (\cos \theta + i \sin \theta)$ potremo scrivere:
\[
  z = \rho \cdot e^{i\theta}.  
\]

Se $z=\rho e^{i\theta}$
il numero $\theta$ viene usualmente chiamato \emph{argomento}
del numero complesso $z$ e si denota a volte 
in questo modo:
\[
  \theta = \arg z.  
\]
La definizione di argomento è intrinsecamente ambigua 
in quanto $\theta$ non è univocamente determinato
(al posto di $\theta$ possiamo scegliere $\theta+2k\pi)$ 
con qualunque $k\in \ZZ$).
Per avere una definizione univoca si può imporre 
la condizione $\theta\in\closeopeninterval{0}{2\pi}$.
\mynote{Ma la condizione 
$\theta\in\closeinterval{-\pi}{\pi}$ 
andrebbe ugualmente bene.}
Se $z=0$ possiamo definire, arbitrariamente, 
$\arg z = 0$. 
In formule si ha:
\[
  \arg z =
  \begin{cases}
%   \arctg \frac y x & \text{se $x>0$,} \\
   \frac \pi 2 - \arctg \frac x y & \text{se $y>0$,} \\
   \frac 3 2 \pi- \arctg \frac x y & \text{se $y<0$,} \\
   \pi & \text{se $y=0$ e $x<0$,} \\
   0 & \text{se $y=0$ e $x\ge 0$.}
   \end{cases}
\]

\subsection{radici complesse $n$-esime}

Sia $c\in \CC$ un numero
complesso $c\neq 0$.
Ci poniamo il problema di determinare le soluzioni complesse
dell'equazione
\[
  z^n = c.
\]
Tali soluzioni saranno chiamate \emph{radici $n$-esime}%
\mymargin{radici $n$-esime}%
\index{radice!$n$-esima} di $c$.

Scriviamo $c$ e $z$ in forma esponenziale:
\[
  c = r e^{i\alpha}, \qquad
  z = \rho e^{i\theta}.
\]
Si avrà allora
\[
  z^n = \rho^n (e^{i\theta})^n = \rho^n e^{i n \theta}.
\]
Affinche sia $z^n = c$ si dovrà avere l'uguaglianza dei moduli, cioè $\rho^n = r$ e l'uguaglianza a meno di multipli interi di $2\pi$ degli argomenti:
$n \theta = \alpha + 2 k \pi$ con $k\in \ZZ$.
Dunque si trova
\[
  \theta = \frac{\alpha}{n} + k\frac{2\pi}{n}
\qquad k \in \ZZ.
\]
Osserviamo ora che per $k=0,\dots, n-1$ il secondo addendo
$k 2\pi /n$ assume $n$ valori distinti compresi in $[0,2\pi)$.
Per gli altri valori di $k$ si ottengono degli angoli che differiscono
da questi di un multiplo di $2\pi$ e quindi non si trovano
altre soluzioni.

Dunque l'equazione $z^n = c$ per $c\neq 0$ ha $n$ soluzioni distinte date
da
\[
z_k = \sqrt[n]{r} \cdot e^{i\alpha/n + 2k\pi i /n},
\qquad k=0,1, \dots, n-1
\]
dove $\alpha = \arg(c)$ e $r = \abs{c}$.
Dal punto di vista geometrico si osserva che
$z_0$ è il numero complesso con modulo la radice $n$-esima del numero
dato $c$ e argomento pari ad un $n$-esimo dell'argomento di $c$.
Tutte le altre soluzioni si trovano sulla circonferenza centrata in $0$
e passante per $z_0$ e risultano essere, insieme ad $z_0$, i vertici
di un $n$-agono regolare.

In particolare nel caso $c=1$ si osserva che le radici $n$-esime dell'unità
si rappresentano geometricamente come i vertici dell'$n$-agono regolare iscritto
nella circonferenza unitaria e con un vertice in $z_0=1$.

\begin{exercise}
Si trovino le soluzioni $z \in \CC$ delle seguenti equazioni.
Scrivere le soluzioni in forma polare e cartesiana.
\begin{gather*}
   z^4 = -4 \\
   z^6 = i\\
   z^3 = -8i \\
   z^4 = z\\
   z^2 + 1 = i\sqrt{3} \\
   (z-i)^4 = 1\\
   1 + z + z^2 + z^3 = 0\\
   z^{14} - z^6 - z^8 + 1 = 0
\end{gather*}
\end{exercise}

\section{polinomi complessi}
\label{ch:ancora_polinomi}

\subsection{il teorema fondamentale dell'algebra}

Il teorema fondamentale dell'algebra afferma che ogni polinomio non costante 
si annulla in almeno un punto del piano complesso.
\mynote{si vedano le note storiche a fine capitolo}
Per dimostrare il teorema dobbiamo estendere il teorema 
di Weierstrass alle funzioni di una variabile complessa.
Nel teorema di Weierstrass reale la funzione per ipotesi è definita su un intervallo
chiuso e limitato. 
Nel piano complesso non esiste il concetto di \emph{intervallo} in quanto non abbiamo 
un ordinamento ma vedremo che comunque il teorema di Weierstrass rimane valido per 
le funzioni continue definite su insiemi chiusi e limitati secondo le seguenti definizioni.

\begin{definition}[chiusura sequenziale]
Un insieme $A\subset \CC$ si dice
essere \emph{sequenzialmente chiuso}
\mymargin{sequenzialmente chiuso}%
\index{sequenzialmente!chiuso}%
se presa una qualunque successione
di punti $a_n\in A$ se $a_n \to a$ per qualche $a\in \CC$
allora $a\in A$.
\end{definition}

\begin{definition}[limitatezza]
Un insieme $A\subset \CC$ si dice essere \emph{limitato}%
\mymargin{limitato}%
\index{limitato}
se
\[
  \sup \ENCLOSE{ \abs{z}\colon z \in A} < +\infty.
\]
\end{definition}

Si noti che una successione $a_n\in \CC$ 
è limitata (si veda capitolo~\ref{sec:successione_limitata})
se e solo se la sua immagine $\ENCLOSE{a_n\colon n\in \NN}\subset \CC$
è un insieme limitato.

% \begin{theorem}[Bolzano-Weierstrass complesso]
% Se $z_n\in \CC$ è una successione limitata allora
% è possibile estrarre una sottosuccessione $z_{n_k}$ convergente:
% $z_{n_k} \to z$ con $z\in \CC$.
% \end{theorem}
% %
% \begin{proof}
% Siano $x_n$ e $y_n$ la parte reale ed immaginaria di $z_n$: $z_n = x_n + i y_n$. Visto che $\abs{x_n} =\sqrt{x_n^2}\le \sqrt{x_n^2+y_n^2} = \abs{z_n}$ e, allo stesso modo $\abs{y_n} \le \abs{z_n}$,
% possiamo affermare che entrambe le successioni $x_n$ e $y_n$ sono limitate (ma stavolta in $\RR$).
% Dunque possiamo applicare il teorema di Bolzano-Weierstrass (reale) alla successione $x_n$ per trovare una sottosuccessione $x_{n_j}\to x$ convergente. E possiamo applicare di nuovo il teorema di Bolzano-Weierstrass alla sottosuccessione $y_{n_j}$ per trovare una sotto-sottosuccessione $y_{n_{j_k}}\to y$ anch'essa convergente.
% Posto $n_k = n_{j_k}$ avremo dunque trovato una sottosuccessione $z_{n_k} = x_{n_k} + i y_{n_k} \to x+iy$ convergente.
% \end{proof}

\begin{theorem}[Weierstrass complesso]
Sia $A\subset \CC$ un insieme non vuoto, sequenzialmente chiuso e limitato e sia $f\colon A \to \RR$ una funzione continua.
Allora $f$ ha massimo e minimo su $A$.
\end{theorem}
%
\begin{proof}
Dimostriamo l'esistenza del minimo: per il massimo la dimostrazione è perfettamente analoga.
Sia $m=\inf f(A)$.
Essendo $A$ non vuoto, per il lemma sull'esistenza delle successioni minimizzanti sappiamo esistere una successione $z_n \in A$ tale che $f(z_n) \to m$.
Essendo $A$ limitato possiamo applicare il teorema di Bolzano-Weierstrass per trovare $z\in \CC$ e una sottosuccessione $z_{n_k} \to z$. Essendo $A$ sequenzialmente chiuso possiamo quindi affermare che $z\in A$. Essendo $f$ continua concludiamo che
\[
f(z) = \lim_{k\to+\infty} f(z_{n_k}) = m
\]
e dunque $z$ è un punto di minimo per $f$.
\end{proof}

\begin{theorem}[esistenza del minimo per funzioni coercive]
Sia $f\colon \CC \to \RR$ una funzione continua tale che per ogni
successione $z_n \to \infty$ (ovvero $\abs{z_n}\to +\infty$)
si abbia $f(z_n) \to +\infty$.
Allora $f$ ha minimo.
\end{theorem}
%
\begin{proof}
Consideriamo l'insieme
\[
  A = \ENCLOSE{z \in \CC \colon f(z) \le f(0)}.
\]
Chiaramente $0\in A$ e quindi $A$ non è vuoto.
L'insieme $A$ è anche sequenzialmente chiuso in quanto se $z_k\in A$ allora $f(0) - f(z_k)\ge 0$,
per continuità $f(0)-f(z_k)\to f(0)-f(z)$
e per il teorema della permanenza del segno si ottiene $f(0)-f(z) \ge 0$ cioè $z \in A$.
Dimostriamo ora che $A$ è anche limitato. 
Se non lo fosse esisterebbe, per assurdo, una successione $z_n \in A$ tale che $\abs{z_n}\to +\infty$ cioè $z_n \to \infty$. 
Ma allora, per ipotesi su $f$, si avrebbe $f(z_n)\to +\infty$ che contraddice la condizione $f(z_n) \le f(0)$. 
Essendo $A$ non vuoto, sequenzialmente chiuso e limitato ed essendo $f\colon A \to \RR$ continua, 
possiamo applicare il teorema di Weierstrass complesso per dedurre che $f$ ha minimo su $A$ in un punto $w \in A$. 
Ma essendo $0\in A$ si avrà sicuramente $f(w)\le f(0)$ e per ogni $z\in \CC \setminus A$ si ha invece $f(z) > f(0)$ per come è stato definito $A$. 
Dunque $w$ è minimo di $f$ su tutto $\CC$, non solo su $A$.
\end{proof}

\begin{theorem}[teorema fondamentale dell'algebra]
\label{th:fondamentale_algebra}
\mymargin{teorema fondamentale dell'algebra}%
\index{teorema!fondamentale dell'algebra}%
Sia $f(z)$ un polinomio di grado $N\ge 1$ a coefficienti complessi:
\[
  f(z) = \sum_{j=0}^N a_j \cdot z^j
\]
con $a_j\in \CC$ per $j=0,\dots,N$ e $a_N \neq 0$.
Allora esiste $w\in \CC$ tale che $f(w) = 0$.
\end{theorem}
%
\begin{proof}
  Osserviamo innanzitutto che $\abs{f(z)}$ è coerciva cioè che
  se $z_n \to \infty$ allora $\abs{f(z_n)}\to +\infty$.
  Infatti si ha
  \begin{align*}
    \abs{f(z_n)}
    &= \abs{\sum_{j=0}^N a_j z_n^j}
    = \abs{a_N z_n^N  + \sum_{j=0}^{N-1} a_j z_n^j}\\
    &= \abs{z_n}^N \cdot \abs{a_N + \sum_{j=0}^{N-1} \frac{a_j}{z_n^{N-j}}}
    \to +\infty
  \end{align*}
  se $z_n \to \infty$.
  
  Sappiamo che tutti i polinomi sono funzioni continue in quanto somme di 
  prodotti di funzioni continue e il modulo è anch'esso una funzione continua 
  dunque $\abs{f(z)}$ è certamente una funzione continua.
  
  Dunque possiamo applicare il teorema di esistenza del minimo per le funzioni 
  coercive: esiste $w\in \CC$ tale che $\abs{f(w)}$ è minimo.
  
  Per concludere il teorema basterà dimostrare che $f(w)=0$.
  L'idea che vogliamo sviluppare è che i polinomi complessi se assumono un valore
  $f(w)$ in un punto $w\in \CC$ allora assumono anche tutti i valori vicini
  ad esso in quanto \emph{localmente} il polinomio assomiglia ad una potenza $z^n$
  e l'equazione $z^n=c$ ha sempre soluzione, come abbiamo già visto.
  Dunque vicino a $w$ ci saranno dei punti in cui $f$ assume valori che in modulo 
  sono minori a $f(w)$: a meno che non sia proprio $f(w)=0$, nel qual caso 
  ovviamente non è possibile avere numeri con modulo inferiore a $0$.
  
  Possiamo scrivere il polinomio $f(z)$ nella forma
  \[
    f(z) = \sum_{j=0}^N b_j (z-w)^j
  \]
  in quanto la traslazione di un polinomio è ancora un polinomio dello stesso 
  grado.
  Dimostrare che $f(w)=0$ è quindi equivalente a dimostrare che $b_0=0$.
  Supponiamo allora per assurdo che sia $b_0\neq 0$. 
  L'andamento del polinomio vicino al punto $w$ è dominato dai termini di grado 
  più basso: alcuni di questi potrebbero essere nulli, ma possiamo considerare 
  il primo indice $k>0$ per cui si ha $b_k\neq 0$. Allora potremo scrivere:
  \[
      f(z) = b_0 + b_k(z-w)^k + \sum_{j=k+1}^N b_j (z-w)^j.
  \]
Ora scriviamo $z-w$, $b_0$ e $b_k$ in forma esponenziale:
$z - w = \rho e^{i\theta}$, $b_0 = r e^{i \alpha_0}$,
$b_k = r_k e^{i\alpha_k}$ 
e applichiamo la disuguaglianza triangolare
  \begin{align*}
    \abs{f(z)} 
    &= \abs{r e^{i\alpha} + r_k \rho^k e^{i(\theta k+\alpha_k)} 
    + \sum_{j=k+1}^N b_j \rho^j e^{i\theta j}} \\
    &\le \abs{r e^{i\alpha} + r_k \rho^k e^{i(\theta k + \alpha_k )}} 
    + \sum_{j=k+1}^N \abs{b_j} \rho^j.
  \end{align*}
Per raggiungere un assurdo vogliamo dimostrare che esiste $z$ (cioè esistono $\rho$ e $\theta$)
per cui il valore della funzione risulti in modulo minore di $r = \abs{b_0} = \abs{f(w)}$.
Innanzitutto se $\rho$ è sufficientemente piccolo possiamo rendere arbitrariamente piccole 
le potenze $\rho^j$ con $j>k$: basta scegliere $\rho\le 1$ 
e $\rho \le r_k/(2\sum \abs{b_j})$
cosicché si avrà 
\[
  \sum_{j=k+1}^N \abs{b_j} \rho^j
  \le \rho^{k+1} \sum_{j=k+1}^N \abs{b_j} \le r_k\frac{\rho^{k}}{2}.
\]
Per quanto riguarda il termine
$r e^{i\alpha} + r_k \rho^k e^{i(\theta k + \alpha_k)}$ 
sarà sufficiente scegliere $\theta = (\pi + \alpha - \alpha_k) /k$ 
in modo che i due addendi 
abbiano fase opposta e la somma sia distruttiva:
\[
  \abs{r e^{i\alpha} + r_k \rho^k e^{i(\theta k + \alpha_k)}}
  = \abs{e^{i\alpha}(r + r_k \rho^k e^{i\pi})}
  = r - r_k\rho^k.
\]
In definitiva abbiamo
\[
\abs{f(z)} = \abs{f(w+\rho e^{i\theta})} 
\le r - r_k\rho^k + r_k\frac{\rho^k}{2} 
< r = \abs{f(w)}
\]
che è assurdo in quanto $\abs{f(w)}$ doveva essere il valore minimo.
\end{proof}
  
\subsection{fattorizzazione dei polinomi}

\begin{theorem}[fattorizzazione dei polinomi complessi]
\index{decomposizione!dei polinomi complessi}%
\index{fattorizzazione!dei polinomi complessi}%
\index{polinomio!fattorizzazione complessa}%
Sia $p(z)$ un polinomio non nullo. Allora posto $n=\deg p$ esistono dei numeri complessi $z_1, z_2, \dots, z_n$ ed un numero complesso $c\neq 0$ tali che
\begin{equation}\label{eq:34985}
  p(z) = c \prod_{k=1}^n (z-z_k).
\end{equation}
Gli $z_k$ sono unici a meno dell'ordine e $c$ pure è univocamente determinato.
\end{theorem}
%
\begin{proof}
Dimostriamo il teorema per induzione su $n=\deg p$. Se $n=0$ il polinomio $p$ è costante: $p(z) = c$. Ricordando che un prodotto di $n=0$ fattori è uguale a $1$ si ottiene quindi il risultato voluto.

Sia ora $p(z)$ un qualunque polinomio di grado $n>0$. Per il teorema fondamentale dell'algebra sappiamo che esiste un numero complesso $z_n$ tale che $p(z_n)=0$. Per il teorema di Ruffini si ha allora
\[
  p(z) = (z-z_n) q(z)
\]
con $q$ un qualche polinomio di grado $n-1$. Per ipotesi induttiva possiamo dunque supporre che esistano $z_1, \dots, z_{n-1}$ e $c$ numeri complessi tali che
\[
   q(z) = c \prod_{k=1}^{n-1} (z-z_k)
\]
e la tesi segue.
\end{proof}

Nella fattorizzazione~\eqref{eq:34985} le \emph{radici} $z_k$ possono
anche ripetersi. Se mettiamo insieme i fattori corrispondenti alla stessa radice
si ottiene una decomposizione della forma
\begin{equation}\label{eq:358925}
P(z) = c \prod_{k=1}^m (z-z_k)^{p_k}
\end{equation}
con $z_1, \dots, z_m$ numeri complessi distinti (le radici del polinomio $P$)
e $p_k$ interi positivi.
L'esponente $p_k$ si chiama
\emph{molteplicità}%
\mymargin{molteplicità}%
\index{molteplicità} della radice $z_k$ e risulta
\[
  p_1 + \dots + p_m = n.
\]
Questa uguaglianza si esprime dicendo che un polinomio $P\in \CC[z]$
di grado $n$ ha sempre $n$ radici contate con la loro molteplicità.
Le radici distinte sono invece $m\le n$.

Se invece $P\in \RR[x]$ è un polinomio a coefficienti reali
non è detto che $P$ abbia radici: ad esempio il
polinomio $x^2+1$ non ha radici reali in quanto $x^2+1>0$
come succede in ogni campo ordinato.
Potrà allora essere utile pensare a $P$ come ad un polinomio
in $\CC[x]$ con coefficienti reali.

Se $P\in \RR[x]$ è un polinomio a coefficienti reali
\[
  P = \sum_{k=0}^n a_k x^k, \qquad a_k\in \RR
\]
possiamo pensare a $P$ anche come un polinomio in
$\CC[x]$, visto che $a_k\in \RR\subset \CC$.
Il seguente teorema ci dà un criterio per distinguere
i polinomi a coefficienti reali dentro a $\CC[x]$.

\begin{theorem}
\label{th:caratterizzazione_polinomi_reali}%
Se $P\in \CC[x]$ è un polinomio a coefficienti complessi
\[
  P = \sum_{k=0}^n a_k x^k,\qquad a_k\in \CC
\]
e se esiste un insieme infinito $A\subset \RR$
tale che per ogni $x\in A$ si abbia $P(x)\in \RR$
allora $a_k\in\RR$ per ogni $k=0,\dots,n$ e dunque $P\in \RR[x]$.

In particolare se la funzione polinomiale associata a
$P\in\CC[x]$ manda $\RR$ in $\RR$ allora $P$ è un polinomio
a coefficienti reali.
\end{theorem}
%
\begin{proof}
Consideriamo il polinomio:
\[
  Q = \sum_{k=0}^n (a_k-\bar a_k) x^k.
\]
Allora per ogni $x\in A$ si ha
\[
  Q(x) = \sum_{k=0}^n a_k x^k - \sum_{k=0}^n \bar a_k x^k
     = P(x) - \overline{P(x)} = P(x) - P(x) = 0
\]
in quanto se $x\in \RR$ risulta
$\overline{x^k} = {\bar x}^k = x^k$.
Visto che $A$ è infinito,
per il principio di annullamento dei polinomi
(teorema~\ref{th:annullamento_polinomi})
deduciamo che $Q=0$.
Ma allora tutti i coefficienti di $Q$ devono
essere nulli e cioè $\bar a_k = a_k$.
Ne consegue che $a_k\in \RR$ per ogni $k=0,\dots,n$.
\end{proof}

Rifacendoci alla fattorizzazione dei polinomi a coefficienti
complessi possiamo fattorizzare anche i polinomi a coefficienti
reali se ci accontentiamo di avere fattori quadratici
invece che lineari.

\begin{theorem}[fattorizzazione dei polinomi reali]
  \label{th:fattorizzazione_polinomio_reale}%
Se $P\in \RR[x]$ è un polinomio a coefficienti reali
potremo scrivere
\begin{equation}\label{eq:35549}
  P = a \cdot \prod_{k=1}^n (x-x_k)^{p_k} \cdot \prod_{j=1}^m (x^2 + \alpha_j x + \beta_j)^{q_j}
\end{equation}
dove $a\in \RR$, $x_k\in \RR$, $p_k\in \NN\setminus\ENCLOSE{0}$,
$\alpha_j,\beta_j\in \RR$, $q_j\in \NN\setminus\ENCLOSE{0}$
con $\alpha_j^2 - 4 \beta_j < 0$
e
\[
  \sum_{k=1}^n p_k + 2 \sum_{j=1}^m q_j = \deg P.
\]

La fattorizzazione~\eqref{eq:35549} è unica a meno
dell'ordine dei fattori.

I numeri $x_1,\dots,x_n$ sono tutte le radici reali distinte
del polinomio $P$ con rispettive molteplicità
$p_1,\dots,p_n$ mentre tutte le radici complesse (non reali)
distinte saranno $\mu_1,\dots, \mu_m$
e $\bar \mu_1, \dots, \bar \mu_m$ con
\[
  x^2 + \alpha_j x + \beta_j = (x-\mu)\cdot(x-\bar \mu)
\]
da cui
\[
  \alpha_j = -2\Re \mu_j, \qquad \beta_j=\abs{\mu_j}^2
\]
e $q_j$ saranno le molteplicità delle radici $\mu_j$
 e $\bar \mu_j$.
\end{theorem}
%
\begin{proof}
Osserviamo innanzitutto che se $P$ è a coefficienti reali
si ha
\[
  \overline{P(z)} = P(\bar z), \qquad \forall z\in \CC
\]
dunque se $\mu$ è una radice di $P$ anche $\bar \mu$
è una radice di $P$.
Ora se $\mu$ è una radice non reale di $P$ sappiamo
che in campo complesso $P$ risulta divisibile
per $x-\mu$ (teorema~\ref{th:Ruffini} di Ruffini):
\[
  P = (x-\mu) \cdot P_1.
\]
Ma anche $\bar \mu$ è radice di $P$ ed essendo
$\bar \mu \neq \mu$ si dovrà avere
\[
Q(\bar \mu) = \frac{P(\bar \mu)}{\bar \mu - \mu} = 0
\]
e dunque applicando nuovamente il teorema di Ruffini
\[
 P = (x-\mu)\cdot (x-\bar \mu)\cdot P_2.
\]
Ora osserviamo che
\[
(x-\mu)\cdot(x-\bar \mu) = x^2 - (\mu + \bar \mu) x + \mu \bar \mu
 = x^2 - 2 (\Re \mu)\cdot x + \abs{\mu}^2
\]
è un polinomio a coefficienti reali e
visto che $P_2$ si ottiene dividendo $P$ per tale polinomio,
anche $P_2$ è un polinomio a coefficienti reali.

Ripetendo lo stesso procedimento sul polinomio $P_2$
potremo fattorizzare $P$ diminuendo il grado a passi
di $2$ finché non si esauriscono tutte le radici complesse
non reali accoppiandole a due a due.
Dunque se $\mu$ è una radice complessa
del polinomio reale $P$ la molteplicità di $\mu$ è
uguale alla molteplicità di $\bar \mu$.

Dopodiché
si potrà completare la fattorizzazione dividendo
per i fattori lineari $x-x_k$ corrispondenti
alle radici reali del polinomio $P$.
Si otterrà quindi la fattorizzazione desiderata.
\end{proof}


\section{note storiche}

\subsection{il numero $e$}

\begin{figure}
  \begin{center}
  \includegraphics[width=10cm]{napier_tables.jpg}
  \end{center}
  \caption{Una pagina delle tavole calcolate da John Napier
  con i valori della funzione $\sin$ (Sinus) e del suo logaritmo
  (Logarithmi).
  In questa pagina ci sono i valori 
  per gli angoli dai 9 gradi (Gr.) ai 9 gradi e mezzo (i gradi 
  sessagesimali sono suddivisi in 60 minuti) e 
  dei loro complementari dagli 80 gradi e mezzo agli 81 gradi.
  Il codice scritto a pag.~\pageref{code:napier} permette
  di calcolare in pochi centesimi di secondo gli stessi valori
  mettendo anche in evidenza alcuni errori sulle ultime cifre decimali.
  Napier ci mise 20 anni a completare la stesura 
  delle tavole da 0 a 90 gradi.
  Se ad esempio volessimo calcolare il valore della tangente di $\alpha=9$ gradi 
  potremmo osservare che il logaritmo di $\tg \alpha$ 
  è la differenza tra il logaritmo di $\sin \alpha$ e il 
  logaritmo di $\cos \alpha$.
  Essendo il $\cos$ uguale al seno dell'angolo complementare 
  troviamo questo valore nella tabella in figura, 
  riga 0, colonna 
  \emph{Differenti\ae}: $1.8427293$ 
  (tutti i valori sono moltiplicati per $10^7$).
  Il valore cercato è l'esponenziale di questo 
  logaritmo e possiamo quindi calcolarlo cercando 
  lo stesso valore nella colonna \emph{Logarithmi}
  e prendendo il corrispondente valore nella colonna 
  \emph{Sinus}. Il valore è compreso tra i valori 
  della riga 6 e della riga 7. Più precisamente 
  differisce dal valore nella riga 7 
  $\frac{3842}{18143}$ volte quanto differisce il valore 
  della riga 6 dal valore della riga 7. 
  Applicando lo stesso rapporto ai valori trovati 
  nella colonna \emph{Sinus} si ottiene
  per interpolazione che al valore 
  della riga 7 (pari a 1.584453) và 
  sottratto 608 ottenendo quindi $\tg \alpha \approx 0.1583845$
  che differisce dal valore 
  esatto per meno di $10^{-7}$.
  }
  \label{fig:napier}
\end{figure}

\label{nota:Nepero}%
\index{Napier!John}%
\index{Nepero}%
\label{Euler!Leonhard}%
\label{Eulero}%
Nepero è l'italianizzazione del nome del
matematico scozzese \emph{John Napier} (1550-1617)
che per compilare le tavole della funzione seno
con una precisione di 7 cifre decimali ha introdotto 
per primo la funzione logaritmo (si veda la figura~\ref{fig:napier}).
E' interessante notare che Napier ha definito direttamente 
il logaritmo naturale senza introdurre il numero $e$ 
e senza fare alcun collegamento con la funzione esponenziale.
% https://archive.org/details/johnnapierinvent00hobsiala/page/18/mode/2up

\label{note:Bernoulli}%
\index{Bernoulli!Jacob}%
L'individuazione della costante $e$
è dovuta a \emph{Jacob Bernoulli} (1655--1705) nel 1683.
Bernoulli si chiedeva qual è l'interesse annuo effettivo
che si ottiene da un capitale che dia una rendita
giornaliera.
Se investo un capitale $c$ ad un tasso di interesse annuo 
pari a $r$, alla fine dell'anno mi viene restituito 
il capitale $c$ più un interesse $rc$.
Se divido il guadagno $rc$ per il numero di giorni che ci sono in 
un anno, posso considerare di aver avuto un guadagno giornaliero
pari a $\frac{r}{365} c$.
Se il guadagno giornaliero $\frac{r}{365} c$ mi venisse restituito 
immediatamente ogni giorno dell'anno, potrei reinvestire immediatamente 
il guadagno facendo aumentare il capitale durante l'anno.
In tal modo se parto con un capitale $c$ ogni giorno il mio capitale 
aumenterebbe di un fattore $1+\frac{r}{365}$ e
alla fine dell'anno avrei quindi un guadagno
pari a $\enclose{1+\frac{r}{365}}^{365} \approx e^r$ 
volte il capitale iniziale (interesse composto).

Il nome $e$ è stato introdotto da Eulero (Leonhard Euler 1707-1783).
%http://eulerarchive.maa.org//docs/originals/E853.pdf

\subsection{il teorema fondamentale dell'algebra}

La formula risolutiva per determinare le radici di un polinomio di secondo grado 
era già nota ai Babilonesi e sebbene i numeri complessi non erano noti 
(e neanche i numeri negativi) la stessa formula si applica 
nel caso generale e fornisce le radici complesse.

Lo studio delle equazioni di grado superiore al secondo si sviluppa invece 
nel 1500 ad opera dei matematici italiani Scipione del Ferro, 
Gerolamo Cardano, Niccolò Tartaglia e Lodovico Ferrari.
Essi determinano le formule risolutive per determinare le radici 
dei polinomi di terzo e quarto grado. 
Nell'applicare la formula risolutiva per l'equazione di terzo grado 
può capitare di dover calcolare la radice quadrata di un numero negativo
anche in situazioni in cui tutte le soluzioni alla fine risultano reali.
Ed è proprio in questo contesto che nascono i numeri complessi, come 
mero artificio matematico utilizzato per portare a termine il conto.

Successivamente Paolo Ruffini (1765-1822) e Niels Abel (1802-1829) dimostrarono che le equazioni 
di grado maggiore del quarto non ammettono una formula risolutiva 
esprimibile mediante radicali. 
Il criterio per determinare se un polinomio
ammette o meno formule risolutive è dovuto al matematico francese Évariste Galois
(1811-1832) che è considerato il fondatore della teoria dei gruppi.

Il teorema fondamentale dell'algebra, e cioè l'esistenza di radici 
complesse per polinomi di grado qualunque, viene dimostrato dal 
matematico tedesco Carl Friedrich Gauss (1777--1855). 
Questo teorema opera una svolta nel pensiero matematico in quanto 
per la prima volta si dà rilevanza ad un risultato astratto di esistenza 
slegato da una formula risolutiva. 
Il teorema non era affatto scontato, basti pensare che sia Leibniz 
che Nikolas Bernoulli erano convinti di aver trovato dei polinomi 
di quarto grado che non possono essere fattorizzati in contrasto 
con il teorema~\ref{th:fattorizzazione_polinomio_reale}.

\subsection{il problema isoperimetrico}

\label{note:isoperimetrico}%
\index{problema!isoperimetrico}%
Probabilmente già dal IX secolo a.C.\ era noto ai greci che 
la circonferenza, tra tutte le linee piane di lunghezza prefissata, 
è quella che racchiude la maggiore area (proprietà isoperimetrica).
Nell'Eneide Virgilio racconta che Didone si trovò ad affrontare 
un problema simile nella fondazione di Cartagine.
\index{Didone}%
\index{isoperimetrico!problema}%
\index{problema!isoperimetrico}%
\index{problema!di Didone}%
Salendo alla dimensione $n=3$ è naturale immaginare che 
la superficie che racchiude il volume maggiore con area fissata 
sia la sfera. 
Questo spiega il motivo per cui una goccia d'acqua, in assenza di gravità,
dovrebbe assumere una forma sferica: a causa della tensione superficiale, 
infatti, la superficie della goccia tenderà ad avere l'area minima.
Eulero (\emph{Leonhard Euler} 1707--1783)
\index{Euler!Leonhard}%
\index{Eulero}%
pensò di aver
trovato una dimostrazione
che si fondava su questo risultato:
presa una qualunque superficie chiusa nello spazio,
se questa non è una sfera è possibile
farne una piccola modifica in modo da ottenere una nuova superficie
che racchiude lo stesso volume ma che ha area strettamente inferiore.
Fu proprio Karl Weiestrass (1815--1897)
\index{Weiestrass!Karl}%
ad accorgersi che quanto dimostrato da Eulero 
(e poi successivamente da Steiner) 
non è sufficiente a garantire la minimalità della sfera. 
Infatti Eulero dimostra che non ci può essere un minimo che non sia la sfera, 
ma non dimostra che il minimo esiste e quindi non è detto che la sfera sia il minimo: 
potrebbero esserci infinite superfici $S_1, S_2, \dots, S_n, \dots$ 
ognuna con area minore di quella della sfera e ognuna con area maggiore della successiva: 
$A(S_{n+1})< A(S_n)$. 
Per concludere il ragionamento di Eulero era necessario dimostrare 
che in una situazione del genere le superifici $S_n$ debbano, in qualche senso, 
tendere alla superficie sferica e che le loro aree debbano
tendere all'area della sfera. 
In astratto questo è quanto viene enunciato nel teorema di Weierstrass 
che così completò la dimostrazione della proprietà isoperimetrica
in dimensione $n=2$ (in realtà il risultato è comunque attribuito a Steiner che,
pur non avendo colto il problema obiettato da Weierstrass, aveva in effetti 
dimostrato tutte le proprietà necessarie per giungere alla conclusione).
La prima dimostrazione formale per $n\ge 3$ 
è dovuta al matematico italiano Ennio De Giorgi (1928--1996)
\index{De Giorgi!Ennio}%
che formalizzò il concetto di perimetro introdotto
da Renato Caccioppoli (1904--1959).%
\index{Caccioppoli!Renato}%
\index{Caccioppoli!perimetro}%

%%%%%%%%%%%%%%%%%%%
%%%%%%%%%%%%%%%%%%%
%%%%%%%%%%%%%%%%%%%
%%%%%%%%%%%%%%%%%%%
\chapter{serie}

Data una successione $a_n$ di numeri reali o complessi
possiamo considerare la successione
delle cosiddette \emph{somme parziali}%
\mymargin{somme parziali}%
\index{somme!parziali}
\[
  S_n = \sum_{k=0}^{n} a_k.
\]
Potremo scrivere più concisamente $S_n = \sum a_n$.
Intuitivamente si intende sommare i termini della successione $a_k$
per $k$ che parte da $0$ fino a $k=n$:
\[
  \sum_{k=0}^n a_k = a_0 + a_1 + a_2 + \dots + a_n.
\]
Formalmente la somma $S_n=\displaystyle \sum_{k=0}^n a_k$
è definita ricorsivamente
dalle seguenti relazioni:
\[
  \begin{cases}
    S_0 = a_0, \\
    S_{n+1} = S_n + a_{n+1}.
  \end{cases}
\]

I numeri $a_n$ si chiamano \emph{termini}%
\mymargin{termini}%
\index{termine} della serie.
\index{termini!di una serie}\index{serie!termini}%
Se la successione delle somme parziali ammette limite il limite viene chiamato
\emph{somma}%
\mymargin{somma}%
\index{somma}
\index{somma!di una serie}\index{serie!somma}%
della serie e si indica con
\[
  \sum_{k=0}^{+\infty} a_k = \lim_{n\to +\infty} S_n = \lim_{n\to+\infty} \sum_{k=0}^n a_k.
\]

La terminologia già introdotta per le successioni si applica anche alle
serie che sono in effetti anch'esse delle successioni.
In particolare una serie può essere convergente, divergente o indeterminata.
Questo è il \emph{carattere della serie}.
\mymargin{carattere}%
\index{carattere!di una serie}%
\index{serie!carattere}%

Più in generale si potrà considerare la somma che parte da un certo
indice $m\in \ZZ$.
Fissato $m$ si potrà ad esempio considerare la serie:
\[
  S_n = \sum_{k=m}^n a_k
\]
che risulta definita per ogni $n\in\NN$, $n\ge m$
come
\[
  S_n = a_m + a_{m+1} + \dots + a_n.
\]

\begin{example}
Consideriamo la serie $S_n$ definita
come la somma dei numeri naturali da $1$ a $n$:
\[
  S_n = \sum_{k=1}^n k = 1 + 2 + \dots + n.
\]
Si può dimostrare facilmente per induzione che si ha
\[
  S_n = \frac{n(n+1)}{2}.
\]
E mediante la definizione di limite si può verificare che
risulta $S_n \to +\infty$.

Questo si esprime dicendo che la serie $\sum n$ è divergente:
\[
     \sum_{k=1}^{+\infty} k = +\infty.
\]
\end{example}

\begin{example}[la serie geometrica]
Fissato $q \in \RR$ o $q\in \CC$ alla successione
$  a_n = q^n$
di termini
\[
  a_0 = 1,\qquad
  a_1 = q,\qquad
  a_2 = q^2,\qquad
  a_3 = q^3, \dots
\]
è associata la
\emph{serie geometrica}%
\mymargin{serie geometrica}%
\index{serie!geometrica}
$\sum q^n$
le cui somme parziali sono
\[
  S_0 = 1, \qquad
  S_1 = 1 + q, \qquad
  S_2 = 1 + q + q^2, \qquad
  \dots
\]
\end{example}

Il seguente teorema ci mostra come per diversi valori di $q$
la serie geometrica assume
tutti i possibili caratteri:
convergente, divergente, indeterminato.

\begin{theorem}[somma della serie geometrica]
\mymark{***}
\mymargin{somma della serie geometrica}%
\index{somma!della serie geometrica}
Sia $q\in \CC$. Se $q\neq 1$ si ha
\[
 \sum_{k=0}^n q^k  = \frac{1-q^{n+1}\!\!\!\!\!\!}{1-q}.
\]
Se $\abs{q} < 1$ la serie geometrica converge:
\[
 \sum_{k=0}^{+\infty} q^k = \frac{1}{1-q}.
\]

Se $\abs{q}\ge 1$ la serie non converge
ma dobbiamo distinguere se la serie
si considera a valori reali o complessi
per determinarne il carattere.

Se consideriamo la serie a valori reali
(quindi $q\in \RR$) se $q\ge 1$ la serie
diverge a $+\infty$,
se $q\le -1$ la serie è indeterminata.

Se consideriamo la serie a valori complessi
$q\in \CC$, se $\abs{q}>1$ la serie diverge
a $\infty$, lo stesso succede se $q=1$.
Se $\abs{q}=1$ ma $q\neq 1$ la serie
è indeterminata.
\end{theorem}
%
\begin{proof}
Il primo risultato riguarda una somma finita.
Si ha
\[
  (1-q)\cdot \sum_{k=0}^n q^k
  = \sum_{k=0}^n q^k - q \cdot \sum_{k=0}^n q^k
  = \sum_{k=0}^n q^k - \sum_{k=1}^{n+1} q^k
  = 1 - q^{n+1}
\]
da cui si ottiene, se $q\neq 1$, il risultato voluto.

Passando al limite per $n\to +\infty$, se $\abs{q} < 1 $
si nota che $\abs{q^{n+1}} = \abs{q}^{n+1} \to 0$
e la serie converge a $\frac 1{1-q}$.

Consideriamo ora la serie a valori reali.
Se $q>1$ osserviamo che
$q^n\to +\infty$ e quindi la serie diverge a $+\infty$
(infatti in questo caso $1-q$ è negativo).
Se $q=1$ si ha $q^k=1$ e quindi
$\displaystyle\sum_{k=0}^n q^k = n+1 \to +\infty$.

Per gli altri casi osserviamo che si ha
\[
  \sum_{k=0}^n q^k = \frac{1-q^{k+1}}{1-q}
  = \frac{1}{1-q} - \frac{q}{1-q}\cdot q^k
\]
e dunque il carattere della serie coincide
con il carattere della successione $q^k$.
Se consideriamo la successione a valori
complessi e se $\abs{q}>1$ la serie è divergente
a $\infty$ in quanto
$\abs{q^k}=\abs{q}^k\to +\infty$.
Se invece consideriamo la serie a valori reali
e $q<1$ (il caso $q\ge 1$ l'abbiamo già considerato)
la serie risulta indeterminata in quanto
in valore assoluto tende a $+\infty$ ma
il segno dei termini pari è opposto al segno dei termini
dispari.

Ci rimane da considerare il caso $\abs{q}=1$, $q\neq 1$.
Se $q$ è reale c'è solo il caso $q=-1$ e sappiamo
che $(-1)^k$ è indeterminata.
Anche se $q$ è complesso vogliamo dimostrare
che la successione $q^k$ è indeterminata:
se non lo fosse si avrebbe $q^k\to \ell$
con $\abs{\ell}=1$ in quanto $\abs{q^k}=1$.
e passando al limite nell'uguaglianza
\[
  q^{k+2} = q\cdot q^{k+1}
\]
si avrebbe $\ell = q\cdot \ell$ da cui
(dividendo per $\ell$) otteniamo $q=1$.
Dunque se $q\neq 1$ la successione $q^k$
è indeterminata e così è la serie
geometrica.
\end{proof}

\begin{theorem}[linearità della somma]
\mymargin{linearità della somma infinita}%
\index{linearità della somma infinita}
Se $\sum a_n$ e $\sum b_n$ sono convergenti
allora per ogni $\lambda,\mu\in \CC$
anche $\sum (\lambda a_n + \mu b_n)$ è convergente
e si h
\[
 \sum_{k=0}^{+\infty} (\lambda a_n + \mu b_n)
 = \lambda \sum_{k=0}^{+\infty} a_n  + \mu \sum_{k=0}^{+\infty} b_n.
\]
\end{theorem}
%
\begin{proof}
Se $S_n$ e $R_n$ sono le somme parziali delle serie $\sum a_n$ e $\sum b_n$
allora le somme parziali della serie $\sum (\lambda a_n + \mu b_n)$ sono
$\lambda S_n + \mu R_n$ (in quanto sulle somme finite vale la proprietà distributiva e commutativa). Ma se $S_n \to S$ e $R_n \to R$ allora
$\lambda S_n + \mu R_n \to \lambda S + \mu R$.
\end{proof}

Osserviamo che le serie (così come le successioni) formano uno spazio
vettoriale in cui le operazioni di somma e prodotto per scalare vengono
eseguite termine a termine: $\sum a_n + \sum b_n = \sum (a_n + b_n)$,
$\lambda \sum a_n = \sum (\lambda a_n)$.
Il teorema precedente ci dice allora che le serie (così come le successioni)
convergenti sono un sottospazio vettoriale e che la somma della serie (così come il limite della successione) è un'operatore lineare definito su tale sottospazio.

\begin{theorem}[condizione necessaria per la convergenza]
\mymark{***}
\mymargin{condizione necessaria}%
\index{condizione necessaria}
Se la serie $\sum a_n$ converge allora $a_n \to 0$.
\end{theorem}
%
\begin{proof}
\mymark{***}
Se la serie $\sum a_n$ converge significa che le somme parziali
$S_n = \sum_{k=0}^n a_k$ convergono: $S_n \to S$. Ma allora
\[
  a_n = S_n - S_{n-1} \to S - S = 0.
\]
\end{proof}

\begin{theorem}[stabilità del carattere]%
\index{carattere!di una serie}%
\index{stabilità!del carattere}%
\index{serie!stabilità del carattere}%
\index{serie!che differiscono su un numero finito di termini}%
Se le due successioni $a_n$ e $b_n$ differiscono solo su un numero finito
\mymargin{stabilità del carattere}%
di termini, allora le serie corrispondenti $\sum a_n$ e $\sum b_n$ hanno lo stesso carattere.
\end{theorem}
%
\begin{proof}
Se le successioni differiscono su un numero finito di termini significa
che esiste un $N\in \NN$ tale che per ogni $k>N$ si ha $a_k=b_k$.
Dunque se indichiamo con $S_n = \sum_{k=0}^n a_k$ e $R_n = \sum_{k=0}^n b_k$
le corrispondenti successioni delle somme parziali, si avrà per ogni $n>N$
\[
  S_n - R_n
    = \sum_{k=0}^n a_k - \sum_{k=0}^n b_k
    = \sum_{k=0}^N (a_k - b_k) = C
\]
dove $C$ è una costante indipendente da $n$. Dunque
\[
  S_n = R_n + C.
\]
Se il limite di $R_n$ non esiste allora non esiste neanche il limite
di $S_n$ (altrimenti essendo $R_n = S_n -C$ anche il limite di $R_n$ dovrebbe esistere). Se il limite di $R_n$ è infinito allora il limite di $S_n$ è uguale
al limite di $R_n$. E se il limite di $R_n$ è finito anche il limite di $S_n$ è finito.

Dunque il carattere della successione $S_n$ è lo stesso della successione $R_n$
cioè le due serie hanno lo stesso carattere.
\end{proof}

Se una serie ha
primo termine con un indice diverso da $0$
ci si potrà sempre ricondurre (con un cambio di variabile)
ad una serie il cui indice parte da zero. Ad esempio
(facendo il cambio di variabile $j=k-1$ da cui $j=0$ quando $k=1$
e ricordando che l'indice utilizzato nelle somme delle
serie è una variabile muta):
\[
 \sum_{k=1}^{+\infty} \frac{1}{2^k}
 = \sum_{j=0}^{+\infty} \frac{1}{2^{j+1}}
 = \sum_{k=0}^{+\infty} \frac{1}{2^{k+1}}.
\]

Si osservi inoltre che in base al teorema precedente quale sia il primo indice
da cui si comincia a sommare non è rilevante per quanto riguarda il carattere della serie.
Se però la serie è convergente la sua somma può variare, ad esempio:
\[
 \sum_{k=1}^{+\infty} \frac{1}{2^k}
 = \enclose{\sum_{k=0}^{+\infty} \frac{1}{2^k}} - 2^0.
\]

Nota bene: in molti libri si scrive $\infty$ al posto di $+\infty$.
Risulta quindi molto comune omettere il segno $+$ davanti a $\infty$
nella terminologia delle serie (e anche delle successioni) visto
che gli indici si intendono numeri naturali e quindi $-\infty$ non avrebbe
senso.

Ci sono però casi in cui può essere utile usare anche gli indici negativi,
ad esempio
se $a_k$ è definita per ogni $k\in \ZZ$
si potrebbe definire (ma non lo faremo):
\[
  \sum_{k=-\infty}^{+\infty} a_k
  = \sum_{k=0}^{+\infty} a_k +
  \sum_{k=1}^{+\infty} a_{(-k)}
\]
richiedendo che entrambe le serie al lato destro
dell'uguaglianza esistano e non abbiano somme infinite di segno opposto.

\begin{theorem}[coda di una serie convergente]
\label{th:coda}%
\mymark{*}%
Sia $\sum a_n$ una serie convergente. Allora
\[
  \lim_{n\to +\infty} \sum_{k=n+1}^{+\infty} a_k = 0.
\]
\end{theorem}
%
\begin{proof}
\mymark{*}
Posto
\[
  S_n = \sum_{k=0}^n a_k,
\]
per definizione di serie convergente sappiamo che esiste $S$ finito
tale che $S_n \to S$. Osserviamo allora che
\[
  \sum_{k=n+1}^{+\infty} a_k = \lim_{N\to+\infty} \sum_{k=n+1}^N a_k
   = \lim_{N\to +\infty} S_N - S_n = S - S_n
\]
e, per $n\to +\infty$ si ha ovviamente $S - S_n \to S - S = 0$.
\end{proof}

\section{serie telescopiche}

Una serie scritta nella forma
% $\Sigma \Delta \vec a$ cioè
% del tipo:
\[
  \sum (a_{k} - a_{k+1})
\]
viene detta \emph{telescopica}
\mymargin{serie telescopica}%
\index{serie!telescopica}
in quanto i singoli termini della somma (come i tubi di un cannocchiale),
si semplificano uno con l'altro (permettendo al cannocchiale di chiudersi):
\[
  S_n = \sum_{k=0}^n (a_{k} - a_{k+1})
  = \sum_{k=0}^{n} a_k - \sum_{k=1}^{n+1} a_k
  = a_0 - a_{n+1}.
\]

In linea teorica ogni serie può essere scritta in forma telescopica, basta infatti scegliere $a_0=0$, $a_n = -S_{n-1}$, affinché valga la relazione precedente. Scrivere una serie in forma telescopica è quindi equivalente a determinare la successione delle somme parziali.

\begin{example}[serie di Mengoli]
\mymark{**}
Si ha
\[
  \sum_{n=1}^{+\infty} \frac{1}{n(n+1)} = 1.
\]
\end{example}
%
\begin{proof}
\mymark{**}
Infatti
\[
  \sum_{k=1}^n \frac{1}{k(k+1)}
  = \sum_{k=1}^n \enclose{\frac{1}{k} - \frac{1}{k+1}}
  = \sum_{k=1}^n \frac{1}{k} - \sum_{k=2}^{n+1} \frac{1}{k}
  = 1 - \frac{1}{n+1} \to 1.
\]
\end{proof}

\section{serie a termini positivi}

\index{serie!a termini positivi}
Nel seguito considereremo serie i cui termini sono numeri reali
positivi (o almeno non negativi).
Quando scriveremo $a_n >0$ (o $a_n \ge 0$) sarà sempre
sottointeso che $a_n\in \RR$ visto che per i numeri complessi non
reali non abbiamo definito la relazione d'ordine.

\begin{theorem}[carattere delle serie a termini positivi]\label{th:serie_positiva}
\mymark{***}%
\mymargin{carattere delle serie a termini positivi}%
\index{carattere!di una serie a termini positivi}%
Se $a_n\ge 0$
la serie $\sum a_n$ è regolare:
o converge oppure diverge a $+\infty$.
\end{theorem}
%
\begin{proof}
\mymark{***}
Se $a_n \ge 0$ essendo $a_n = S_n - S_{n-1}$ significa che
la successione $S_n$ delle somme parziali è crescente.
Dunque il limite delle $S_n$ esiste e non può essere negativo.
\end{proof}

\begin{theorem}[criterio del confronto]
\mymark{**}
\mymargin{criterio del confronto}
\index{criterio!del confronto per serie}
Siano $\sum a_n$ e $\sum b_n$ serie a
termini positivi che si confrontano: $0\le a_n\le b_n$.
Allora
\[
  \sum_{k=0}^{+\infty} a_n \le \sum_{k=0}^{+\infty} b_n.
\]
In particolare se $\sum b_n$ converge anche $\sum a_n$ converge
e se $\sum a_n$ diverge anche $\sum b_n$ diverge.

Quest'ultimo risultato vale anche se $0 \le a_n \ll b_n$.
\end{theorem}
%
\begin{proof}
\mymark{*}
Se $S_n$ sono le somme parziali di $\sum a_n$ e $R_n$ sono le somme
parziali di $\sum b_n$ si ha $S_n \le R_n$ e il risultato
si riconduce al confronto tra successioni.

Nel caso in cui $a_n \ll b_n$ per definizione sappiamo che $\frac{a_n}{b_n}\to 0$
e quindi dalla definizione di limite sappiamo che
esiste $N$ tale che per ogni $n>N$ si ha (avendo scelto $\eps=1$)
\[
  \frac{a_n}{b_n} < 1.
\]
Dunque si ottiene $a_n \le b_n$ per tutti gli $n$ tranne al più un numero
finito. Sapendo che il carattere della serie non cambia se si modifica
la serie su un numero finito di termini ci si riconduce al caso precedente.
\end{proof}

\begin{example}\label{ex:52573}
\mymark{***}
La serie
\begin{equation}\label{eq:296453}
 \sum_{k=1}^{+\infty} \frac{1}{k^2}
\end{equation}
è convergente.
Infatti osservando che si ha per ogni $n>0$
\[
  \frac{1}{(n+1)^2} \le \frac{1}{n(n+1)}
\]
possiamo affermare che
\[
  \sum_{k=1}^{+\infty} \frac{1}{k^2}
  = 1 + \sum_{k=1}^{+\infty} \frac{1}{(k+1)^2}
  \le 1+ \sum_{k=1}^{+\infty} \frac{1}{k(k+1)}
  = 2
\]
in quanto ci siamo ricondotti alla
serie telescopica di Mengoli che ha somma pari a $1$.

Sappiamo quindi che la serie~\eqref{eq:296453} è convergente
senza sapere esattamente quale sia la sua somma.
Possiamo però trovare numericamente delle approssimazioni
della somma, facendo la somma dei primi termini
e stimando l'errore tramite la serie di Mengoli,
di cui sappiamo calcolare la somma.
Infatti se $S_N$ è la somma parziale dei primi
$N$ termini e $S = \lim S_N$ è la somma della serie,
essendo $1/(k+1)^2 \le 1/(k^2+k)$ si ha
\[
S_N
\le S
\le S_N + \sum_{k=N}^{+\infty} \frac{1}{k(k+1)}
\le S_N + \frac{1}{N}.
\]
Per calcolare le prime 6 cifre decimali esatte basterà
quindi sommare il primo milione di termini della serie.
Lo si può fare, ad esempio, con il codice riportato
a pagina~\pageref{code:series}, ottenendo $S=1.644934\ldots$

Utilizzando strumenti molto più avanzati
Eulero \index{Eulero}
(Leonard Euler, 1707--1783) è
riuscito ad esprimere la somma di questa serie mediante costanti matematiche fondamentali
(noi lo faremo con un metodo diverso nell'esercizio~\ref{ex:Basilea}).
\end{example}

\begin{corollary}[criterio del confronto asintotico]
\mymark{*}
\mymargin{criterio del confronto asintotico}
\index{criterio!del confronto asintotico}
Se $a_n$ e $b_n$ sono successioni a termini positivi,
asintoticamente equivalenti (definizione~\ref{def:ordine_infinito}),
allora le serie corrispondenti $\sum a_n$ e $\sum b_n$
hanno lo stesso carattere.
\end{corollary}
%
\begin{proof}
\mymark{*}
Le serie a termini positivi non possono essere indeterminate
quindi è sufficiente verificare che se una serie converge, converge anche l'altra.
Essendo $a_n / b_n$ convergente tale rapporto deve anche essere
limitato, quindi esiste $C\in \RR$ tale che
\[
   a_n \le C \cdot b_n.
\]
Se la serie $\sum b_n$ converge anche $\sum C \cdot b_n$ converge e, per confronto,
converge anche $\sum a_n$.

Viceversa, scambiando il ruolo di $a_n$ e $b_n$ si verifica che se $a_n$
converge, converge anche $b_n$.
\end{proof}

\begin{example}
La serie
\[
\sum_n \frac{n^2+2n+3}{2n^4-n^3+n+1}
\]
è convergente. Infatti si può facilmente verificare che
\[
   \frac{n^2+2n+3}{2n^4-n^3+n+1} \sim \frac{1}{2n^2}.
\]
Ma sappiamo che la serie $\sum 1/n^2$ è convergente, di conseguenza
anche la serie $\sum 1/(2n^2)$ lo è (per linearità della somma)
e quindi, per confronto
asintotico, anche la serie data è convergente.
\end{example}

\begin{theorem}[criterio della radice]
\mymargin{criterio della radice}
\index{criterio!della radice}
Sia $\sum a_n$ una serie a termini non negativi
(cioè $a_n\ge 0$) tale che
\mymark{***}
$\sqrt[n]{a_n} \to \ell \in [0,+\infty]$.
Se $\ell<1$ allora la serie converge.
Se $\ell>1$ allora la serie diverge.

Più in generale il risultato è valido con
\[
  \ell = \limsup \sqrt[n]{a_n}
\]
anche nel caso in cui il limite di $\sqrt[n]{a_n}$ non dovesse esistere.
\end{theorem}
%
\begin{proof}
\mymark{***}
Nel caso $\ell < 1$
prendiamo $q$ con $\ell < q < 1$ e poniamo $\eps = q-\ell$.
Per la definizione di limite $\sqrt[n]{a_n}\to \ell$
(ma basta che sia $\limsup \sqrt[n]{a_n}=\ell$)
sappiamo
esistere $N$ tale che per ogni $n > N$ si abbia
\[
  \sqrt[n]{a_n} < \ell + \eps = q
\]
cioè
\[
   a_n < q^n.
\]
Sapendo che $\sum q^n$ converge, sapendo anche che il carattere
della serie non cambia modificando un numero finito di termini,
per confronto possiamo concludere che anche la serie $\sum a_n$ converge.

Se $\ell>1$ si ha che $\sqrt[n]{a_n}>1$ e quindi $a_n>1$ per infiniti valori di $n$. La successione $a_n$ non è infinitesima e quindi la serie non può convergere.
\end{proof}

\begin{example}
La serie
\[
  \sum_k 2^{(\ln k) - k}
\]
è convergente. Infatti si ha
\[
 \sqrt[k]{2^{\ln k - k}}
 = 2^{\frac{\ln k - k}{k}}
 = 2^{\frac{\ln k }k - 1}
 \to 2^{-1}
 = \frac{1}{2}
 < 1.
\]
\end{example}

\begin{theorem}[criterio del rapporto]
\mymark{***}
\mymargin{criterio del rapporto}
\index{criterio!del rapporto per serie}
Sia $\sum a_n$ una serie a termini non negativi
($a_n\ge 0$)
tale che $a_{n+1} / a_n \to \ell \in [0,+\infty]$.
Se $\ell <1$ allora la serie converge.
Se $\ell > 1$ la serie diverge.
\end{theorem}
%
\begin{proof}
\mymark{*}
Non sarebbe difficile fare una dimostrazione diretta, simile alla dimostrazione fatta per il criterio della radice.
Possiamo però osservare che
per il criterio di convergenza alla Cesàro (teorema~\ref{th:criterio_cesaro}) si ha $\sqrt[n]{a_n} \to \ell$
quindi ci riconduciamo al criterio della radice senza dover fare ulteriori dimostrazioni.
\end{proof}

\begin{example}
\mymark{***}
Per ogni $x\ge 0$ la serie
\[
  \sum \frac{x^n}{n!}
\]
converge.
\end{example}
%
\begin{proof}
Applichiamo il criterio del rapporto.
Posto $a_n = x^n / n!$ si ha
\[
\frac{a_{n+1}}{a_n}
= \frac{x^{n+1}}{(n+1)!}\cdot \frac{n!}{x^n}
= \frac{x}{n+1} \to 0 < 1.
\]
Dunque la serie converge.
\end{proof}

\subsection{associatività della somma di una serie}

\begin{example}
Posto $a_k = (-1)^k$ sappiamo che la serie corrispondente:
\[
  1 - 1 + 1 - 1 + 1 \dots
\]
è indeterminata perché le somme parziali sono
\[
  S_n = \sum_{k=0}^n (-1)^k
  = \begin{cases}
   1 & \text{se $n$ pari}\\
   0 & \text{se $n$ è dispari.}
  \end{cases}
\]
Se però associamo i termini a due a due otteniamo la serie
\[
  (1-1) + (1-1) + (1-1) \dots = 0 + 0 + 0 + \dots = 0
\]
che è convergente.
Il motivo è che la successione delle somme
parziali di questa nuova serie è una particolare estratta (quella
con gli indici pari) della successione delle somme parziali della
serie originale. Quindi non ci dovrebbe sorprendere il fatto che
nonostante la serie originale fosse indeterminata è possibile
associare i termini della serie in modo da ottenere una serie
regolare (in questo caso convergente).
In effetti per il teorema~\ref{th:bolzano_weierstrass} 
di Bolzano-Weierstrass sappiamo che
è sempre possibile estrarre una sottosuccessione regolare (convergente o divergente)
da qualunque successione.
Anche quando da una serie indeterminata si estrae una serie
convergente la somma della serie può dipendere da come i termini
vengono associati.
Se ad esempio nella serie precedente prendessimo
solamente le somme di indice dispari
otterremmo:
\[
  1 + (-1+1) + (-1+1) + \dots = 1.
\]
\end{example}

L'esempio precedente ci mostra che associando i termini di una
serie indeterminata è possibile ottenere serie con carattere
e somma diversi.
Il seguente teorema ci dice che questo fenomeno \emph{cattivo}
può solo avvenire quando si parte da una serie indeterminata.
Se la serie è regolare allora possiamo associarne i termini
senza modificarne né il carattere né la somma.
In particolare questo è vero per le serie a termini positivi.

\begin{theorem}[associatività delle serie regolari]%
\label{th:serie_associativa}%
Se $\sum a_k$ è una serie, scelta comunque
una successione crescente $k_n$ con $k_0=0$
possiamo considerare la serie $\sum b_n$
i cui termini
\[
  b_n = \sum_{j=k_n}^{k_{n+1}-1} a_j
\]
si ottengono associando i termini di $a_k$ a gruppi
consecutivi delimitati dalla successione di indici
$k_n$.

Se la serie $\sum a_k$ è regolare (convergente o divergente)
allora anche la serie $\sum b_n$ è regolare e si ha
\[
\sum_{n=0}^{+\infty} b_n
= \sum_{k=0}^{+\infty} a_k.
\]

In particolare questo vale se la serie $\sum a_k$ è a termini
positivi.
\end{theorem}
%
\begin{proof}
Siano $S_k = \sum_{j=0}^k a_j$ le somme parziali della
serie $\sum a_j$. Allora le somme parziali della serie $\sum b_n$
non sono altro che la sottosuccessione $S_{k_n}$.
Dunque se $S_k$ converge anche ogni sua sottosuccessione
converge allo stesso limite.

Il teorema~\ref{th:serie_positiva} ci dice
che le serie a termini positivi sono regolari e quindi
soddisfano le ipotesi del teorema.
\end{proof}

\subsection{la serie armonica}

Osserviamo che il criterio del rapporto non si applica alla
\emph{serie armonica}%
\mymargin{serie armonica}%
\index{serie!armonica}%
\[
  \sum_k \frac{1}{k}
\]
in quanto
\[
 \frac{\frac{1}{k+1}}{\frac{1}{k}}
 = \frac{k}{k+1} \to 1.
\]

Per capire se la serie armonica converge o diverge presentiamo il metodo
di \emph{condensazione} che verrà enunciato in generale nel prossimo teorema
ma che può essere meglio compreso se applicato al caso particolare
della serie armonica.

Mostreremo che la serie armonica diverge.
L'idea è semplicemente quella di associare gli addendi della serie armonica
in gruppi di lunghezza potenze di due e stimare la somma di ogni gruppo dal basso
con il termine più piccolo (cioè l'ultimo) di ogni gruppo:
\begin{align*}
 \sum_{k=1}^{+\infty} \frac{1}{k}
 & = 1 + \frac 1 2
     + \enclose{\frac 1 3 + \frac 1 4}
     + \enclose{\frac 1 5 + \frac 1 6 + \frac 1 7 + \frac 1 8}
     + \dots\\
 & > 1 + \frac 1 2 + 2 \cdot \frac 1 4 + 4 \cdot \frac 1 8 + \dots \\
   & = 1 + \frac 1 2 + \frac 1 2 + \frac 1 2 + \dots
    = +\infty.
\end{align*}

\begin{theorem}[criterio di condensazione di Cauchy]%
\mymark{**}%
\mymargin{criterio di condensazione di Cauchy}%
\index{criterio!di condensazione di Cauchy}%
Sia $a_n$ una successione decrescente di numeri reali non negativi:
$a_n \ge 0$.
Allora la serie $\sum a_k$ converge se e solo se converge
la serie
\[
  \sum 2^k a_{2^k}.
\]
\end{theorem}
%
\begin{proof}
\mymark{**}
Supponiamo per comodità che le somme partano da $k=1$.
Si tratta di raggruppare i termini $a_k$ in gruppi di potenze di due:
\begin{align*}
  b_0 & = a_1, \\
  b_1 &= a_2 +  a_3, \\
  b_2 &= a_4 +  a_5 +  a_6 +  a_7, \\
  b_3 &= a_8 +  a_9 +  a_{10} + \dots + a_{15}, \\
  &\vdots\\
  b_n &= a_{2^n} +  a_{2^{n}+1} + \dots + a_{2^{n+1}-1},\\
  &\vdots
\end{align*}
Grazie al teorema~\ref{th:serie_associativa} sulla associatività
delle serie a termini positivi sappiamo che
  \[
  \sum_{k=1}^{+\infty} a_k = \sum_{n=0}^{+\infty} b_n.
  \]
Ogni termine $b_n$ è la somma di $2^n$ termini della
successione $a_k$ che, essendo la successione decrescente,
possono essere stimati dall'alto e dal basso con il primo
e l'ultimo termine di ogni somma, dunque:
\[
  2^n a_{2^{n+1}-1} \le b_n \le 2^n a_{2^{n}}.
\]
Per ipotesi la serie $\sum 2^n a_{2^n}$ è convergente
dunque per confronto anche la serie $\sum b_n$ è convergente.

Viceversa se la serie $\sum 2^n a_{2^n}$ è divergente
anche la serie $\sum 2^{n+1} a_{2^{n+1}}$ è divergente
e sapendo che $b_n\ge 2^n a_{2^{n+1}-1}\ge \frac 1 2 2^{n+1} a_{2^{n+1}}$
otteniamo, per confronto, che anche la serie $\sum b_n$ è divergente.
\end{proof}

\begin{corollary}[serie armonica generalizzata]
\mymark{***}
\mymargin{serie armonica generalizzata}%
\index{serie!armonica!generalizzata}
La serie
\[
 \sum_n \frac{1}{n^\alpha}
\]
converge se $\alpha>1$,
diverge se $0\le \alpha\le 1$.
\end{corollary}
%
\begin{proof}
\mymark{***}
Applichiamo il criterio di condensazione. Posto $a_n = \frac 1{n^\alpha}$ Si ha
\[
  \sum_n 2^n a_{2^n} = \sum_n 2^n \frac{1}{(2^n)^\alpha}
  = \sum_n 2^{n(1-\alpha)}
  = \sum_n \enclose{2^{(1-\alpha)}}^n
\]
che è una serie geometrica di ragione $q=2^{1-\alpha}$.
Se $\alpha>1$ allora $q<1$ e la serie armonica è convergente
se invece $\alpha \le 1$ allora $q\ge 1$ e la serie
armonica è divergente.
\end{proof}

\begin{exercise}
Utilizzare il criterio di condensazione per dimostrare che la serie
\[
  \sum \frac{1}{n \cdot \ln n}
\]
diverge.
\end{exercise}

\begin{exercise}
  Per quali valori dei parametri $\alpha\in \RR$ e $\beta\in \RR$
  la serie
  \[
    \sum n^\alpha (\ln n)^\beta
  \]
  converge?
\end{exercise}

%%%
%%%
\section{rappresentazione posizionale dei numeri reali}
%%%
%%%
Quando scriviamo $\frac{3}{8} = 0.375$ intendiamo che vale 
\[
\frac 3 8 = \frac{3}{10} + \frac{7}{10^2} + \frac{5}{10^3}.  
\]
Più in generale data una base $d\in \NN$, $d\ge 2$, ($d=10$ nel 
caso della rappresentazione decimale)
consideriamo l'insieme $\Enclose{d} = \ENCLOSE{0,1,2,\dots, d-1}$ 
delle cifre in base $d$.
Una sequenza infinita di cifre sarà quindi un elemento 
$\vec a \in \Enclose{d}^\NN$, $\vec a = (a_0,a_1,\dots, a_n, \dots)$
con $a_k\in \Enclose{d}$.
Potremo quindi considerare il numero ``$0.a_0 a_1 a_2 \ldots$'' rappresentato 
dalla sequenza di cifre $\vec a$:
\[
  r(\vec a) = \sum_{k=0}^{+\infty} \frac{a_k}{d^{k+1}}.
\]
Chiaramente $r(\vec a)\in [0,1]$ in quanto essendo $0\le a_k\le d-1$ 
risulta
\begin{equation}\label{eq:10445934}
 0 \le r(\vec a) \le \sum_{k=0}^{+\infty} \frac{d-1}{d^{k+1}}
  = \frac{d-1}{d}\sum_{k=0}^{+\infty}\frac 1 {d^k}
  = \frac{d-1}{d}\cdot \frac{1}{1-\frac 1 d} = 1.
\end{equation}

Ogni numero $x\in [0,1)$ ammette una rappresentazione in cifre $x=r(\vec a)$ 
con $\vec a \in \Enclose{d}^\NN$. 
Infatti per ogni $N\in \NN$ possiamo scrivere 
\[
  \lfloor x\cdot 10^N \rfloor= \sum_{k=0}^{N-1} a_k 10^{N-1-k}
\]
e al crescere di $N$ otteniamo una sequenza di cifre $a_k\in \Enclose{d}$ 
tali che 
\[
  \abs{x \cdot d^N - \sum_{k=0}^{N-1} a_k \cdot d^{N-1-k}} \le 1
\]
da cui 
\[
  \abs{x - \sum_{k=0}^{N-1} \frac{a_k}{d^{k+1}}} \le \frac{1}{d^{N}}
\]
che, facendo tendere $N\to +\infty$, significa $r(\vec a) = x$.
Il numero $x=1$ può essere anch'esso rappresentato, basta prendere 
$a_k=d-1$ per ogni $k\in \NN$ cosicché si ottiene l'uguaglianza 
nel lato destro di \eqref{eq:10445934}. 
In base $d=10$ questo si esprime dicendo che 
\[
 0.999\ldots = 1.  
\]

Ci possiamo chiedere se è possibile che lo stesso numero 
abbia due rappresentazioni in cifre distinte.
Supponiamo quindi che esistano $\vec a,\vec b\in \Enclose{d}^\NN$
con $\vec a \neq \vec b$ 
tali che $r(\vec a) = r(\vec b)$. 
Sia $m = \min\ENCLOSE{n\in\NN\colon a_n\neq b_n}$ la posizione 
della prima cifra diversa tra $\vec a$ e $\vec b$.
Si ha allora 
\[
 r(b) - r(a) = \sum_{k=0}^{+\infty}\frac{b_k - a_k}{d^{k+1}}
 = \frac{b_m - a_m}{d^{m+1}} + \sum_{k=m+1}^{+\infty} \frac{b_k - a_k}{d^{k+1}}
 = A+B
\]
con 
\[
\abs{A} = \abs{\frac{b_m - a_m}{d^{m+1}}} \ge \frac{1}{d^{m+1}}  
\]
e 
\begin{align*}
\abs{B} &= \abs{\sum_{k=m+1}^{+\infty} \frac{b_k - a_k}{d^{k+1}}}
        \le \sum_{k=m+1}^{+\infty} \frac{d-1}{d^{k+1}}\\
        &= \frac{d-1}{d^{m+2}}\sum_{k=0}^{+\infty}\frac{1}{d^k}
        = \frac{d-1}{d^{m+2}}\cdot \frac{1}{1-\frac 1 d} = \frac{1}{d^{m+1}}.
\end{align*}
Dunque, per disuguaglianza triangolare inversa,
\[
0 = \abs{r(b)-r(a)}\ge \abs{A} - \abs{B} \ge 0.  
\]
Significa che tutte le disuguaglianze sono in realtà uguaglianze
e quindi deve essere $\abs{b_m-a_m}=1$ 
e per ogni $k>m$ deve essere $\abs{b_k-a_k}=d-1$. 
Supponendo che sia $b_m=a_m+1$ (l'altro caso è analogo)
per $k>m$ dovrà necessariamente essere $b_k=0$ e $a_k=d-1$.

Ad esempio se $d=10$, $\vec a = (1,2,3,9,9,9,9,\dots )$ 
e $\vec b = (1,2,4,0,0,0,0,\dots)$ si avrà 
$r(\vec a) = r(\vec b) = 0.124$.

\begin{theorem}[Cantor]
  \label{th:cantor_secondo}
Fissata una base $d>2$ l'insieme di Cantor
\[
  C = \{r(\vec a)\colon \vec a \in \ENCLOSE{0,2}^\NN\}
\]
ha cardinalità
\[
\# C = \# \mathcal P(\NN).
\]
In particolare $\#\RR > \#\NN$.
\end{theorem}
%
\begin{proof}
Vogliamo verificare che $r\colon \ENCLOSE{0,2}^\NN \to \closeinterval{0}{1}$
è iniettiva.
Abbiamo già visto in generale che $r(\vec a) = r(\vec b)$ solamente 
se la prima cifra diversa in $\vec a$ e $\vec b$ differisce di una unità.
Ma siccome tutte le cifre di $\vec a$ e $\vec b$ per ipotesi sono $0$ oppure 
$2$, non differiscono di una unità e quindi $r$ è iniettiva.
Dunque $\# C = \#\ENCLOSE{0,2}^\NN$.
D'altra parte $\#\ENCLOSE{0,2}^\NN = \#\mathcal P(\NN)$ in quanto 
ogni $\vec a \in \ENCLOSE{0,2}^\NN$ può essere messo in corrispondenza 
biunivoca con l'insieme $\vec a^{-1}(\ENCLOSE{0})$ degli indici $k\in \NN$
per cui $a_k=0$.

Dal teorema~\ref{th:Cantor} deduciamo che $\# C = \#\mathcal P(\NN) > \#\NN$ 
e visto che $C\subset \RR$ a maggior ragione $\# \RR \ge \# C >\#\NN$.

D'altra parte possiamo mostrare che $\#\RR \le \# \mathcal P(\NN)$
in quanto possiamo costruire una funzione $f\colon \RR \to \mathcal P(\QQ)$
in questo modo:
\[
f(x) = \ENCLOSE{q\in \QQ\colon q<x}.
\]
Per la densità di $\QQ$ in $\RR$ sappiamo che se $x<y$ 
esiste $q\in \QQ$ con $x<q<y$ e dunque $q\in f(y)\setminus f(x)$ 
da cui $f(x)\neq f(y)$. 
Dunque $f$ è iniettiva, che significa $\#\RR\le \# \mathcal P(\QQ)$
Ma $\#\NN = \#\QQ$ dunque $\#\mathcal P(\QQ) = \#\mathcal P(\NN)$ 
e la dimostrazione è conclusa.
\end{proof}

L'insieme $C$ definito nella precedente dimostrazione con $d=3$ 
è lo stesso \emph{insieme di Cantor}%
\mymargin{insieme di Cantor}%
\index{insieme!di Cantor}
\index{Cantor!insieme di}% 
considerato nell'esempio~\ref{ex:insieme_Cantor}.
Infatti si potrebbe mostrare facilmente che 
$C=\frac 1 3 C \cup (\frac 2 3 + \frac 1 3 C)$.

%%%%%%%%%%%%%
%%%%%%%%%%%%%
\section{convergenza assoluta}

Per le serie a termini positivi abbiamo molti criteri di convergenza
che invece, in generale, non si applicano alle serie di segno qualunque
o alle serie di numeri complessi.
La convergenza di queste ultime, però, può a volte ricondursi
facilmente
alla
convergenza delle serie a termini positivi, passando al modulo
ogni termine.

\begin{definition}[convergenza assoluta]
\mymark{***}
Diremo che una serie (a termini reali o complessi) $\sum a_n$
è \emph{assolutamente convergente}%
\mymargin{assolutamente convergente}%
\index{assolutamente!convergente} se la serie $\sum \abs{a_n}$
è convergente.
\end{definition}

\begin{theorem}[convergenza assoluta]\label{th:convergenza_assoluta}
\mymark{***}%
Se una serie $\sum a_n$ (reale o complessa)
è assolutamente convergente allora è convergente e vale
\[
  \abs{\sum_{k=0}^{+\infty} a_k} \le \sum_{k=0}^{+\infty} \abs{a_k}.
\]
\end{theorem}
%
\begin{proof}
\mymark{*}
Supponiamo inizialmente che gli $a_n$ siano numeri reali.
Definiamo $a_n^+ = \max\ENCLOSE{0, a_n}$ e $a_n^- = -\min \ENCLOSE{0, a_n}$.
Cioè se $a_n\ge 0$ si ha $a_n^+ = a_n$ e $a_n^-=0$ se invece $a_n\le 0$
si ha $a_n^+ =0$ e $a_n^- = -a_n$.
Dunque $a_n^+\ge 0$, $a_n^-\ge 0$,
\[
   a_n = a_n^+  - a_n^-
   \qquad\text{e}\qquad
   \abs{a_n} = a_n^+ + a_n^-.
\]
Allora se $\sum \abs{a_n}$ converge,
per confronto anche $\sum a_n^+$ e $\sum a_n^-$ convergono.
Dunque, per il teorema sulla somma dei limiti,
$\sum a_n = \sum a_n^+ - \sum a_n^-$
e quindi anche $\sum a_n$ converge.

Se abbiamo una successione di complessi $a_n = x_n + i y_n$
e se
$\sum \abs{a_n}$ converge allora, per confronto,
anche $\sum \abs{x_n}$ e $\sum\abs{y_n}$ convergono
(si osservi infatti che $\abs{x} \le \abs{x+iy}$ e $\abs{y}\le \abs{x+iy}$).
Dunque $\sum x_n$ e $\sum y_n$ convergono per quanto
già dimostrato sulle serie a termini reali.
Ma allora anche $\sum i y_n$ e $\sum a_n = \sum (x + iy_n)$ convergono.

Poniamo ora
\[
  S_n  = \sum_{k=0}^n a_k.
\]
Per la subadditività
del modulo sappiamo che per le somme finite si ha
\[
 \abs{S_n} =\abs{\sum_{k=0}^n a_k}
 \le \sum_{k=0}^n \abs{a_k} \le \sum_{k=0}^{+\infty} \abs{a_k}.
\]
E per continuità del modulo, posto $S= \lim S_n$ si ha
\[
  \abs{\sum_{k=0}^{+\infty} a_k}
  = \abs{S}
  = \lim_{n\to +\infty} \abs{S_n}
  \le \sum_{k=0}^{+\infty} \abs{a_k}.
\]
\end{proof}

\begin{theorem}[scambio delle serie]
\label{th:scambio_somma}
Sia $a_{k,j}\in \RR$ o $a_{k,j}\in \CC$ una successione a due indici $k\in \NN$, $j\in \NN$.
Se 
\[
  \sum_{k=0}^{+\infty}\sum_{j=0}^{+\infty}\abs{a_{k,j}}<+\infty  
\]
allora
\begin{equation}\label{eq:scambio_somma}
  \sum_{k=0}^{+\infty} \sum_{j=0}^{+\infty} a_{k,j}
  = \sum_{j=0}^{+\infty} \sum_{k=0}^{+\infty} a_{k,j}.
\end{equation}
\end{theorem}
%
\begin{proof}
Si intende, ovviamente, che per ogni $k$ 
la serie $\sum_j \abs{a_{k,j}}$ è convergente
dunque $\sum_j a_{k,j}$ è assolutamente convergente.
% e, per il teorema~\ref{th:convergenza_assoluta}, 
% anche $\sum_k \abs{\sum_j a_{k,j}} 
% \le \sum_k \sum_j \abs{a_{k,j}}$
% è assolutamente convergente.
Ovviamente 
$\sum_k \abs{a_{k,j}} \le \sum_k \sum_j \abs{a_{k,j}}$
e dunque per ogni $j$ anche la serie 
$\sum_k a_{k,j}$ è assolutamente convergente.
Posto
\[
S = \sum_{k=0}^{+\infty}\sum_{j=0}^{+\infty} a_{k,j},
\qquad
S_n = \sum_{j=0}^{n-1}\sum_{k=0}^{+\infty} a_{k,j}  
\]
dobbiamo dimostrare che $S_n \to S$
per $n\to +\infty$. 
Applichiamo la definizione di limite:
sia $\eps>0$ fissato.
Per il teorema~\ref{th:coda} della coda 
esiste $K$ tale che 
\[
  \sum_{k=K}^{+\infty} \sum_{j=0}^{+\infty} \abs{a_{k,j}}<\frac \eps 2
\]
e per ogni $k<K$ esiste $J_k$ tale che 
\[
  \sum_{j=J_k}^{+\infty} \abs{a_{k,j}} < \frac{\eps}{2K}.  
\]
Dunque posto $J=\max\ENCLOSE{J_0,J_1, \dots, J_{K-1}}$
se $n>J$ si ha:
\begin{align*}
  \abs{S-S_n}
  &= \abs{\sum_{k=0}^{+\infty}\sum_{j=0}^{+\infty}a_{k,j} 
  - \sum_{j=0}^{n-1}\sum_{k=0}^{+\infty}a_{k,j}}
  = 
  \abs{\sum_{k=0}^{+\infty}\sum_{j=0}^{+\infty}a_{k,j} 
  - \sum_{k=0}^{+\infty}\sum_{j=0}^{n-1}a_{k,j}}\\
  &=
  \abs{\sum_{k=0}^{+\infty}\sum_{j=n}^{+\infty}a_{k,j} }
  = \abs{\sum_{k=0}^{K-1}\sum_{j=n}^{+\infty} a_{k,j}
   + \sum_{k=K}^{+\infty}\sum_{j=n}^{+\infty} a_{k,j}}\\
   &\le \sum_{k=0}^{K-1}\sum_{j=n}^{+\infty} \abs{a_{k,j}}
    + \sum_{k=K}^{+\infty}\sum_{j=0}^{+\infty} \abs{a_{k,j}}
   \le \sum_{k=0}^{K-1}\sum_{j=J_k}^{+\infty} \abs{a_{k,j}}
   + \frac \eps 2
  \le \frac \eps 2 + \frac \eps 2 = \eps
\end{align*}
come volevamo dimostrare.
\end{proof}

\begin{theorem}[convergenza incondizionata]%
\label{th:convergenza_incondizionata}%
\mymark{*}%
\index{convergenza!incondizionata}%
Se $\sum a_n$ è una serie assolutamente convergente e $\sigma\colon \NN \to \NN$
è una qualunque funzione biettiva (permutazione dei numeri naturali)
si ha
\[
  \sum_{n=0}^{+\infty} a_n = \sum_{n=0}^{+\infty} a_{\sigma(n)}.
\]
\end{theorem}
\begin{proof}
Definiamo la successione $b_{k,j}$ a due indici:
\[
  b_{k,j} = \begin{cases}
    a_k & \text{se $k=\sigma(j)$},\\
    0 & \text{altrimenti}.
  \end{cases}  
\] 
Chiaramente $\sum_j \abs{b_{k,j}} = a_k$
e dunque $\sum_k \sum_j \abs{b_{k,j}} = \sum_k \abs{a_k} < +\infty$.
Dunque, per il teorema~\ref{th:scambio_somma}
\[
 \sum_{k=0}^{+\infty} a_k 
 = \sum_{k=0}^{+\infty} \sum_{j=0}^{+\infty}b_{k,j}
 = \sum_{j=0}^{+\infty} \sum_{k=0}^{+\infty}b_{k,j}
 = \sum_{j=0}^{+\infty} a_{\sigma(j)}.  
\]
\end{proof}

\begin{theorem}[somme alla Cauchy]
  \label{th:somma_Cauchy}%
Sia $a_{k,j}\in \RR$ o $a_{k,j}\in \CC$ una successione a due indici $k\in \NN$, $j\in\NN$.
Se 
\[
  \sum_{k=0}^{+\infty} \sum_{j=0}^{+\infty} \abs{a_{k,j}} < +\infty
\]  
allora 
\begin{equation}\label{eq:somma_Cauchy}
  \sum_{k=0}^{+\infty} \sum_{j=0}^{+\infty} a_{k,j}
   = \sum_{n=0}^{+\infty} \sum_{k=0}^{n} a_{k,n-k}.
\end{equation}
\end{theorem}
%
\begin{proof}
Poniamo 
\[
  b_{k,n} = \begin{cases}
    a_{k,n-k} & \text{se $k\le n$}\\
    0 & \text{se $k>n$}.
  \end{cases}  
\]
Osserviamo allora che 
\[
  \sum_{k=0}^{+\infty} \sum_{n=0}^{+\infty} b_{k,n}
  = \sum_{k=0}^{+\infty} \sum_{n=k}^{+\infty} a_{k,n-k} 
  = \sum_{k=0}^{+\infty} \sum_{j=0}^{+\infty} a_{k,j}
\]
e analogamente
\[
  \sum_{k=0}^{+\infty} \sum_{n=0}^{+\infty} \abs{b_{k,n}}
  = \sum_{k=0}^{+\infty} \sum_{j=0}^{+\infty} \abs{a_{k,j}}
  < +\infty.
\]
Mentre 
\[
  \sum_{n=0}^{+\infty}\sum_{k=0}^{+\infty} b_{k,n} 
  = \sum_{n=0}^{+\infty}\sum_{k=0}^{n} a_{k,n-k}
\]
dunque si può applicare il teorema~\ref{th:scambio_somma} alla 
successione $b_{k,n}$ per ottenere
\[
  \sum_{k=0}^{+\infty} \sum_{j=0}^{+\infty} a_{k,j}
  = \sum_{k=0}^{+\infty}\sum_{n=0}^{+\infty} b_{k,n} 
  = \sum_{n=0}^{+\infty} \sum_{k=0}^{+\infty} b_{k,n}
  = \sum_{n=0}^{+\infty} \sum_{k=0}^{n} a_{k,n-k}.
\]
%da cui~\eqref{eq:somma_Cauchy}.
\end{proof}

%%%%%%%%%%%
%%%%%%%%%%%
\section{serie a segno variabile}
\index{serie!a segni alterni}

La serie $\sum \frac{(-1)^k}{k+1}$ la cui somma si può scrivere come
\mymargin{serie armonica a segni alterni}%
\index{serie!armonica!a segni alterni}%
\[
1 - \frac{1}{2} + \frac{1}{3} - \frac{1}{4} +  \frac{1}{5} \dots
\]
non è assolutamente convergente
(in quanto la serie $\sum \frac 1 {k+1}$ è divergente) ma ha il termine generico
infinitesimo. 
Non abbiamo quindi nessun criterio che ci permetta di
determinarne il carattere.
Possiamo però sfruttare il fatto che i segni sono \emph{alterni} cioè
che i termini di indice pari hanno segno opposto ai termini di indice dispari. 
Si nota infatti che posto
\[
  S_n = \sum_{k=0}^n \frac{(-1)^k}{k+1}
\]
si ha
\begin{align*}
S_{2n+2}
  &= S_{2n+1} + \frac{1}{2n+3}
  = S_{2n} - \frac{1}{2n+2} + \frac{1}{2n+3}
  < S_{2n}\\
S_{2n+3}
  &= S_{2n+2} - \frac{1}{2n+4}
  = S_{2n+1} + \frac{1}{2n+3} - \frac{1}{2n+4}
  > S_{2n+1} \\
\end{align*}
Dunque la successione delle somme parziali di indice pari è decrescente mentre
sui termini di indice dispari è crescente. Avremo quindi che entrambe
le sottosuccessioni hanno limite: $S_{2n} \to S$, $S_{2n+1} \to R$.

Ma
\[
  S - R = \lim_{n\to +\infty} (S_{2n} - S_{2n+1}) = \lim_{n\to+\infty}\frac{1}{2n+2} = 0.
\]
Dunque $S=R$ e l'intera successione ha limite $S$. 
D'altra parte $S \le S_0$ in quanto $S_{2n}$ è decrescente e $S\ge S_1$ in quanto $S_{2n+1}$ è crescente. 
Concludiamo che $S$ è finito e dunque la serie è convergente.%
\mynote{%
Per determinare il valore della somma di questa serie ci serviranno degli strumenti più avanzati.
Si veda l'equazione~\ref{eq:serie_ln2}.
}

Questa dimostrazione può essere resa più in generale nel seguente.

\begin{theorem}[criterio di Leibniz per le serie a segno alterno]
\label{th:Leibniz}%
\mymark{***}%
\mymargin{criterio di Leibniz}%
\index{criterio!di Leibniz}%
\index{teorema!di Leibniz}%
\index{Leibniz!criterio di}%
Sia $b_n$ una successione monotòna e infinitesima. Allora
la serie
\[
  \sum (-1)^{n} b_n
\]
è convergente.

Più precisamente se $\displaystyle S_n = \sum_{k=0}^n (-1)^k b_k$
sono le somme parziali
si osserva che la somma della serie $S= \lim S_n$ è sempre compresa
tra due termini consecutivi della successione $S_n$. 
Cioè, se $b_k\ge 0$, si ha
\[
  S_{2n+1} \le S \le S_{2n}.
\]
\end{theorem}
%
\begin{proof}
\mymark{***}
Senza perdere di generalità possiamo supporre che la successione $b_n$ sia decrescente e quindi $b_n \ge 0$ (visto che il limite è zero).
Posto
\[
 S_n = \sum_{k=0}^n (-1)^k b_k
\]
si ha
\begin{align*}
  S_{2n+2} &= S_{2n} - b_{2n+1} + b_{2n+2} \\
  S_{2n+3} &= S_{2n+1} + b_{2n+2} - b_{2n+3}.
\end{align*}
Essendo $b_n$ decrescente si ha $b_{2n+2} < b_{2n+1}$ e $b_{2n+3} < b_{2n+2}$ da cui
\[
  S_{2n+2} < S_{2n}, \qquad S_{2n+3} > S_{2n+1}.
\]
Dunque le successioni $S_{2n}$ e $S_{2n+1}$ sono monotone e di conseguenza
hanno limite:
\[
  S_{2n} \to S, \qquad S_{2n+1} \to R
\]
con $S, R  \in [-\infty, +\infty]$.
D'altronde, essendo $b_n$ infinitesima
\[
  S - R
  = \lim_{n\to +\infty} S_{2n} - S_{2n+1}
  = \lim_{n\to +\infty} b_{2n+1} = 0.
\]
Dunque $S=R$. Inoltre essendo $S_{2n}$ decrescente si ha
$S \le S_0$ ed essendo $S_{2n+1}$ crescente si ha $S\ge S_1$.
Dunque $S$ è finito e la serie converge.

Abbiamo anche ottenuto che
$S_{2n-1} \le S \le S_{2n}$ e $S_{2n+1} \le S \le S_{2n}$
dunque è verificata anche la seconda parte dell'enunciato.
\end{proof}

Abbiamo dunque un esempio, la serie $\sum (-1)^k / k$, di una serie convergente ma non assolutamente convergente.
Il seguente teorema ci dice che per le serie di questo tipo 
la somma della serie dipende dall'ordine in cui sono stati presi i termini:
anzi, si può ottenere come somma qualunque valore si voglia.

\begin{theorem}[convergenza condizionata]%
\label{th:convergenza_condizionata}%
\index{convergenza!condizionata di una serie}%
\mymargin{convergenza condizionata}%
Sia $\sum a_k$ una serie convergente ma non assolutamente convergente a termini reali.
Allora fissato qualunque $x \in [-\infty , +\infty]$ esiste un riordinamento
$\sigma \colon \NN \to \NN$ biettivo tale che
\[
  \sum_{k=0}^{+\infty}  a_{\sigma(k)} = x.
\]
\end{theorem}
%
\begin{proof}
Dividiamo i termini della successione $a_k$ in termini maggiori o uguali a zero
e in termini negativi. Sia $a^+_k$ la sottosuccessione dei termini non negativi
e $-a^-_k$ la sottosuccessione
dei termini negativi (quindi $a^+_k\ge 0$ e $a^-_k > 0$). Si avrà
\begin{align*}
  \sum_{k=0}^n a_k &= \sum_{k=0}^{n^+} a^+_k - \sum_{k=0}^{n^-} a^-_k \\
  \sum_{k=0}^n \abs{a_k} &= \sum_{k=0}^{n^+} a^+_k + \sum_{k=0}^{n^-} a^-_k
\end{align*}
dove $n^+ +1$ e $n^-+1$ sono rispettivamente
il numero di termini non-negativi e negativi
tra i primi $n+1$ termini della successione $a_k$.

Osserviamo ora che dovrà essere
\[
\sum_{k=0}^{+\infty} a_k^+ = +\infty \qquad \text{e} \qquad
\sum_{k=0}^{+\infty} a_k^- = +\infty.
\]
Innanzitutto le somme esistono perché le serie sono a termini non negativi.
Se entrambe queste somme fossero finite allora la serie $\sum\abs{a_k}$ sarebbe convergente, ma per ipotesi abbiamo assunto che $\sum a_k$ non fosse
assolutamente convergente.
Quindi almeno una delle due somme è infinita. Se la somma dei termini positivi
fosse infinita e quella dei termini negativi fosse finita potremmo però
concludere che anche la somma della serie $\sum a_k$ sarebbe infinita.
Viceversa se la somma dei termini positivi fosse finita e quella dei termini
negativi fosse infinita la somma $\sum a_k$ sarebbe $-\infty$. Ma per ipotesi
abbiamo richiesto che la serie $\sum a_k$ fosse convergente.

Fissato $x\in \RR$ possiamo quindi cominciare a sommare i termini positivi
$a^+_k$ finché non si raggiunge o si supera il valore $x$. A quel punto cominciamo a sommare i termini negativi finché non torniamo sotto al valore $x$.
Poi continuiamo a sommare i termini positivi finché non si torna a superare $x$
e di nuovo poi continuiamo con i termini negativi finché non si torna a scendere
sotto $x$. Intermezzando opportunamente termini positivi e termini negativi
riusciamo quindi ad ottenere delle somme parziali che oscillano intorno al valore 
di $x$ e si avvicinano sempre di più a $x$ in quanto ad ogni cambio di ``rotta'' 
la distanza da $x$ è inferiore al valore assoluto dell'ultimo termine sommato 
e la successione dei termini $a_k$ è infinitesima in quanto la serie $\sum a_k$ è convergente.

Stessa cosa si può fare per ottenere una somma $x=+\infty$. Fissata una qualunque
successione $x_n \to +\infty$ comincio a sommare i termini positivi finché non supero il valore $x_1+a_1^-$. 
Poi sommo un solo termine negativo, $-a_1^-$ e ottengo una somma maggiore di $x_1$. 
Poi sommo tanti positivi finché non supero $x_2+a_2^-$. 
Poi sommo un altro unico termine negativo e così via. 
Chiaramente le somme tenderanno a $+\infty$.

Il caso $x=-\infty$ si tratta in maniera analoga.
\end{proof}

\section{somma per parti}
\begin{theorem}[somma per parti]
\label{th:somma_per_parti}%
Siano $a_k$ e $B_k$ successioni (reali o complesse).
Posto
\[
  A_n = \sum_{k=0}^{n-1} a_k, \qquad
  b_n = B_{n+1} - B_n
\]
si ha
\mymargin{somma per parti}%
\index{somma!per parti}%
\begin{equation}\label{eq:somma_per_parti}
 \sum_{k=m}^{n-1} a_k B_k = A_n B_n - A_m B_m - \sum_{k=m}^{n-1} A_{k+1}b_k.
\end{equation}
\end{theorem}
%
\begin{proof}
Osserviamo che $a_n = A_{n+1}-A_n$ dunque
\begin{align*}
  A_n B_n - A_m B_m
  &= \sum_{k=m}^{n-1} (A_{k+1}B_{k+1}-A_k B_k)\\
  &= \sum_{k=m}^{n-1} (A_{k+1}B_{k+1}-A_{k+1}B_k + A_{k+1}B_k - A_k B_k)\\
  &= \sum_{k=m}^{n-1} A_{k+1}b_k + \sum_{k=m}^{n-1} a_k B_k.
\end{align*}
\end{proof}

Se prendiamo $a_k=(-1)^k$ si può osservare che $A_n = \sum_{k=0}^{n-1} (-1)^k$
è una successione limitata in quanto $A_n = 1$ se $n$ è dispari mentre
$A_n=0$ se $n$ è pari.
Se invece scegliamo una successione $B_n$ è positiva,
decrescente e infinitesima
e poniamo $b_n = B_{n+1}-B_n$
si ha
\[
  \sum_{k=0}^{+\infty} \abs{b_k}
  = \lim_{n\to+\infty}\sum_{k=0}^{n-1} (B_k-B_{k+1})
  = \lim_{n\to +\infty} (B_0 - B_n) = B_0 < +\infty.
\]
Dunque il seguente teorema è una estensione del criterio
di Leibniz per le serie a segni alterni.

\begin{theorem}[criterio di Dirichlet]%
\label{th:dirichlet}%
Siano $a_n$ e $B_n$ successioni (reali o complesse)
e poniamo $\displaystyle A_n = \sum_{k=0}^{n-1} a_k$,
$b_n = B_{n+1} - B_n$.
Se $A_n$ è limitata,
$B_n\to 0$
e $\sum \abs{b_n} < +\infty$
allora la serie $\sum a_k B_k$ è convergente.
\end{theorem}
%
\begin{proof}
Per la formula di somma per parti~\eqref{eq:somma_per_parti}
 si ha
\begin{equation}\label{eq:3498954}
 \sum_{k=0}^{n-1} a_k B_k
 = A_n B_n - A_0 B_0 - \sum_{k=0}^{n-1} A_{k+1}b_k.
\end{equation}
Il primo addendo $A_n B_n$ tende a zero per $n\to +\infty$
in quanto prodotto di una successione limitata per una infinitesima.
Il secondo addendo è costante.
La serie $\sum A_{k+1} b_k$ è assolutamente convergente
in quanto essendo $\abs{A_n}\le L$ limitata
e $\sum b_n$ assolutamente convergente si ha
\[
  \sum \abs{A_{k+1}} \cdot \abs{b_k}
  \le L \cdot \sum \abs{b_k} < +\infty.
\]

Dunque il limite della somma sul lato destro converge e quindi
la somma sul lato sinistro
dell'equazione~\eqref{eq:3498954}
è convergente per $n\to +\infty$.
\end{proof}

\begin{exercise}
Fissato $z\in \CC$, $\abs{z} \le 1$, $z\neq 1$ la serie
\[
  \sum_{k=1}^{+\infty} \frac{z^k}{k}
\]
è convergente.
\end{exercise}
\begin{proof}
Si noti che per $\abs{z}<1$ si può facilmente applicare
il criterio del rapporto o della radice.
Ma per $\abs{z}=1$ quei criteri non si applicano e
bisogna invece utilizzare il teorema~\ref{th:dirichlet}.

Posto $a_k=z^k$ si osserva che $\sum a_k$ è una
serie geometrica e (essendo $z\neq 1$) si ha
\[
  A_n = \sum_{k=0}^{n-1} z^k = \frac{1-z^n}{1-z}
\]
che è limitata, infatti:
\[
  \abs{A_n} = \frac{\abs{1-z^n}}{\abs{1-z}} \le \frac{1+\abs{z^n}}{\abs{1-z}}
  = \frac{2}{\abs{1-z}}.
\]
Mentre posto $B_n = 1/n$ è chiaro che, essendo $B_n$ reale,
decrescente si ha
\begin{align*}
\sum_{k=1}^{+\infty} \abs{B_{k+1}-B_{k}}
&= \lim_{n\to +\infty}\sum_{k=1}^{n-1} (B_k-B_{k+1})
= \lim_{n\to +\infty}(B_1 - B_n)\\
&= \lim_{n\to +\infty}\enclose{1 - \frac{1}{n}} = 1
< +\infty.
\end{align*}
Si applica quindi il teorema~\ref{th:dirichlet}
per ottenere la convergenza
della serie data.

Si osservi che se $\abs{z}>1$ la serie in questione non
converge perché il termine generico $z^k/k$ non è infinitesimo.
Per $z=1$ si ottiene la serie armonica, che pure non converge.
\end{proof}

\begin{exercise}
  La serie
  \[
    \sum_{k=1}^{+\infty} \frac{\sin k}{k}
  \]
  è convergente.
  \end{exercise}
  \begin{proof}
  Applichiamo il teorema~\ref{th:dirichlet}.
  Posto $a_k = \sin k$ e $B_k=1/k$
  si ha
  \[
    A_n = \sum_{k=0}^{n-1} \sin k = \Im \sum_{k=0}^{n-1} e^{ik}.
  \]
  Osserviamo allora che $e^{ik}=(e^i)^k$ e dunque $A_n$ è la parte immaginaria
  della somma di una serie geometrica. 
  Si può quindi calcolare esplicitamente
  \[
    A_n = \Im \frac{1-(e^i)^n}{1-e^i}
  \]
  da cui
  \[
    \abs{A_n} \le \abs{\frac{1-e^{in}}{1-e^i}} \le \frac{1+\abs{e^{in}}}{\abs{1-e^i}}
    = \frac{2}{\abs{1-e^i}}
  \]
  e dunque $A_n$ è limitata.
  
  D'altro canto posto $B_k = 1/k$ è chiaro che $B_k$ è
  decrescente e infinitesima.
  \end{proof}
  
%% \begin{comment}
%% \section{somme su insiemi qualunque}
%% 
%% La teoria che abbiamo visto finora ci permette di definire
%% la somma di una sequenza di numeri $\vec a \colon \NN\to \RR$
%% (o $\vec a \colon \NN \to \CC$).
%% In questo capitolo daremo una definizione di somma che può
%% essere applicata a una funzione definita su un insieme di indici qualunque, non necessariamente l'insieme $\NN$ dei
%% numeri naturali.
%% 
%% Scopriremo che la definizione che stiamo introducendo se
%% applicata al caso $A=\NN$ corrisponde esattamente alla convergenza assoluta.
%% 
%% \begin{definition}[funzioni sommabili]
%% Sia $A$ un insieme qualunque.
%% Sia $A_n$ una successione di sottoinsiemi finiti di $A$.
%% Diremo che $A_n$ tende ad $A$ e scriveremo
%% \[
%%   A_n \to A
%% \]
%% se per ogni insieme finito $B\subset A$ esiste $N\in \NN$
%% tale che per ogni $n\ge N$ si ha $A_n\supset B$.
%% 
%% Sia ora $f$ una funzione reale (o complessa) definita
%% su $A$: $f\colon A \to \RR$ (oppure $f\colon A \to \CC$).
%% 
%% Se $A = \ENCLOSE{a_1, a_2, \dots, a_m}$ è finito si può definire
%% \[
%%   \sum_{a \in A} f(a) = f(a_1) + f(a_2) + \dots + f(a_m).
%% \]
%% (la definizione può essere formalizzata mediante una
%% induzione sulla cardinalità dell'insieme finito $A$).
%% 
%% Se invece $A$ è infinito scriveremo
%% \[
%%   \sum_{a\in A} f(a) = S
%% \]
%% con $S\in \bar \RR$ (oppure $S\in \bar \CC$)
%% se per ogni successione di sottoinsiemi finiti $A_n \to A$
%% si ha
%% \begin{equation}\label{eq:589421}
%%   \lim_{n\to +\infty}\sum_{a\in A_n}f(a) = S.
%% \end{equation}
%% 
%% In tal caso diremo che esiste la somma di $f$ su $A$.
%% Se tale somma $S$ è finita diremo che $f$ è sommabile su $A$.
%% \end{definition}
%% 
%% \begin{theorem}[invarianza per permutazioni]
%%   Sia $f$ una funzione definita su un insieme $A$
%%   a valori reali o complessi. Sia $\sigma\colon A \to A$
%%   una qualunque funzione bigettiva.
%%   Allora la funzione $f$ ha somma su $A$ se e solo se
%%   la funzione $f\circ \sigma$ ha somma su $A$ e in tal
%%   caso le somme coincidono:
%%   \[
%%     \sum_{x\in A} f(x) = \sum_{x\in A} f(\sigma(x)).
%%   \]
%% \end{theorem}
%% %
%% \begin{proof}
%%   Basta osservare che se $A_n$ è una qualunque successione
%%   di sottoinsiemi finiti di $A$ tale che $A_n\to A$
%%   allora anche $B_n=\sigma(A_n)$ è una successione
%%   di sottoinsiemi finiti di $A$ tale che $B_n\to A$.
%%   E viceversa.
%%   Dunque le definizione di somma per $f$ e per $\sigma\circ f$ sono equivalenti.
%% \end{proof}
%% 
%% \begin{theorem}[collegamento tra serie e somme arbitrarie]
%% Sia $a_k$ con $k\in \NN$, una successione numerica (reale o complessa).
%% 
%% Se $S$ è un numero finito (reale o complesso)
%% sono equivalenti:
%% \begin{enumerate}
%%   \item la funzione $f(k)=a_k$ è sommabile su $\NN$ e
%% \begin{equation}\label{eq:488484}
%%   \sum_{k\in \NN} a_k = S;
%% \end{equation}
%% \item la serie $\sum_k a_k$ è assolutamente
%% convergente e
%% \begin{equation}\label{eq:497494}
%%   \sum_{k=0}^{+\infty} a_k = S.
%% \end{equation}
%% \end{enumerate}
%% 
%% Se $a_k\ge 0$ (dunque reale) allora anche
%% \[
%%   \sum_{k\in \NN} a_k = +\infty
%% \qquad
%% \text{e}
%% \qquad
%% \sum_{k=0}^{+\infty} a_k = +\infty
%% \]
%% sono equivalenti.
%% \end{theorem}
%% %
%% \begin{proof}
%%   Se vale \eqref{eq:488484} allora per ogni successione
%%   di insiemi finiti $A_n \to \NN$ si deve avere
%%   \[
%%     S_n = \sum_{k\in A_n} a_k  \to S.
%%   \]
%%   In particolare se scegliamo $A_k=\ENCLOSE{0,1,2, \dots, n}$
%%   risulta che $S_n$ non è altro che la successione delle somme parziali della serie $\sum a_k$ 
%%   e dunque deve valere \eqref{eq:497494}.
%%   Ma più in generale se prendiamo qualunque permutazione $\sigma$
%%   si dovrà avere $\sum_k a_{\sigma(k)} = S$
%%   e quindi la serie deve convergere assolutamente altrimenti 
%%   si violerebbe il teorema~\ref{th:convergenza_condizionata} 
%%   di convergenza condizionata.
%% 
%%   Viceversa supponiamo che $\sum a_n$ sia una serie assolutamente convergente e poniamo
%%   \[
%%     S = \sum_{k=0}^{\infty} a_k.
%%   \]
%%   Dovremo dimostrare che per ogni successione
%%   $A_n\to A$ di sottoinsiemi finiti vale~\eqref{eq:589421}.
%%   Cioè dovremo dimostrare che per ogni $\eps>0$
%%   esiste $N\in \NN$ tale che per ogni $n>N$ si ha
%%   \begin{equation}\label{eq:4389567}
%%     \abs{\sum_{k\in A_n} a_k - S} < \eps.
%%   \end{equation}
%%   Dato $\eps>0$ per il teorema~\ref{th:coda} esiste $M\in \NN$
%%   tale che
%%   \[
%%     \sum_{k=M+1}^{+\infty} \abs{a_k} < \eps.
%%   \]
%%   Ma visto che $A_n \to A$ deve esistere $N\in \NN$
%%   tale che per ogni $n\ge N$ si ha $A_n \supset \ENCLOSE{0,1,2,\dots ,M}$
%%   e allora per ogni $n\ge N$
%%   \[
%%     \sum_{k \in A_n} a_k - S = \sum_{k=0}^M a_k + \sum_{k\in A_n, k>M} a_k - \sum_{k=0}^{+\infty} a_k
%%     = \sum_{k\in A_n, k>M} a_k - \sum_{k=M+1}^{+\infty} a_k
%%   \]
%%   ma è chiaro che
%%   \[
%%     \abs{\sum_{k\in A_n, k>M} a_k - \sum_{k=M+1}^{+\infty} a_k}
%%     \le \sum_{k=M+1}^{+\infty} \abs{a_k} < \eps
%%   \]
%%   da cui si ottiene, come voluto, \eqref{eq:4389567}.
%% \end{proof}
%% 
%% \end{comment}
%% \begin{comment}
%% \begin{theorem}[somme più che numerabili]
%%   Sia $f$ una funzione definita su un insieme $A$
%%   a valori reali o complessi. Se $f$ è sommabile
%%   su $A$ allora l'insieme
%%   \[
%%     \ENCLOSE{x\in A\colon f(x)\neq 0}
%%   \]
%%   è al più numerabile.
%% \end{theorem}
%% %
%% \begin{proof}
%% Supponiamo dapprima che $f$ abbia valori reali.
%% Se l'insieme $\ENCLOSE{f\neq 0}$ fosse più che numerabile almeno uno dei due insiemi $\ENCLOSE{f>0}$ o $\ENCLOSE{f<0}$
%% sarebbe più che numerabile.
%% Senza perdita di generalità possiamo supporre che sia
%% il primo cioè che $A_+ = \ENCLOSE{x \in A\colon f(x)>0}$
%% sia più che numerabile. Considero allora per ogni $n\in \NN$
%% gli insiemi $A_n = \ENCLOSE{x\in A\colon \frac 1 {f(x)} \in (n,n+1]}$. Almeno uno di questi insiemi
%% deve essere infinito perché se gli $A_n$ fossero
%% tutti finiti allora l'insieme $A=\bigcup_n A_n$ sarebbe
%% numerabile. Dunque esiste $N\in \NN$ tale che $A_N$ è infinito.
%% Posto $\eps = \frac{1}{N+1}>0$
%% esistono dunque $x_0,x_1, \dots, x_k, \dots$ infiniti punti di $A$ tali che $f(x_k) \ge \eps$.
%% 
%% ***DUBBIO*** Se $A$ è più che numerabile
%% esiste $A_n$ finito tale che $A_n\to A$?
%% \end{proof}
%% \end{comment}
%% \begin{comment}
%% 
%% \begin{theorem}[somme iterate]
%% Sia $f$ una funzione definita su un insieme $A$ a
%% valori reali o complessi. Supponiamo che sia
%% \[
%%   A = \bigcup_{k\in K} A_k
%% \]
%% dove $K$ è un arbitrario insieme di indici e
%% $A_k$ sono sottoinsiemi di $A$ a due a due disgiunti.
%% Allora
%% \[
%%   \sum_{x\in A} f(x) = \sum_{k\in K} \sum_{x\in A_k} f(x)
%% \]
%% se almeno uno dei due lati dell'uguaglianza esiste.
%% \end{theorem}
%% \end{comment}

%%%%%%%%%%
\section{somme su insiemi arbitrari}
\index{serie!su insiemi arbitrari}%
\index{somme!su insiemi arbitrari}%
%%%%%%%%%%

\begin{lemma}\label{lemma:12734}
  Sia $A$ un insieme numerabile e siano 
  $\alpha\colon \NN\to A$ e $\beta\colon \NN\to A$ 
  due funzioni bigettive.
  Sia $f\colon A\to \RR$ o 
  $f\colon A \to \CC$ una funzione tale che 
  \begin{equation}\label{eq:12734}
    \sum_{k=0}^{+\infty} \abs{f(\alpha(k))} < +\infty.  
  \end{equation}
  Oppure sia $f\colon A\to [0,+\infty)$ una funzione 
  qualunque. 
  Allora si ha
  \begin{equation}\label{eq:12735}
    \sum_{k=0}^{+\infty} f(\alpha(k)) =
    \sum_{j=0}^{+\infty} f(\beta(j)). 
  \end{equation}
\end{lemma}
%
\begin{proof}
Posto $a_k=f(\alpha (k))$ e $\sigma = \alpha^{-1}\circ \beta$ si ha 
$a_{\sigma(j)} = f(\beta (j))$.
Dunque se vale~\eqref{eq:12734} si può 
applicare il teorema~\ref{th:convergenza_incondizionata}
per ottenere~\eqref{eq:12735}.

Se invece $f(x)\ge 0$ si ha ovviamente $\abs{f(x)}=f(x)$
dunque se almeno una delle due somme in~\eqref{eq:12735}
è finita allora si applica il teorema~\ref{th:convergenza_incondizionata}
alla serie $\sum a_k$ e si ottiene dunque l'uguaglianza 
in~\eqref{eq:12735}.
Se invece entrambe le serie sono divergenti
(essendo serie a termini positivi non ci sono altre possibilità!)
l'uguaglianza~\eqref{eq:12735} è comunque valida
essendo $+\infty = +\infty$.
\end{proof}

\begin{definition}[somme arbitrarie]
Sia $f\colon A\to [0,+\infty]$ una funzione definita su un insieme 
$A$ numerabile (cioè $\#A \le \#\NN$).
Allora definiamo
  \[
    \sum_{x\in A}f(x)
    = \sum_{k=0}^{+\infty} f(\sigma(k)) 
  \]
dove $\sigma\colon \NN\to A$ è una qualunque funzione bigettiva
(il risultato non dipende da $\sigma$ grazie al lemma~\ref{cor:12734}).

Se $f\colon A\to \CC$ o $f\colon A\to \RR$ (sempre con $A$ numerabile)
diremo che $f$ è sommabile se $\sum_{x\in A}\abs{f(x)}<+\infty$. 
In tal caso possiamo definire $\sum_{x\in A} f(x)$ come prima
visto che anche in questo caso la somma non dipende
dall'ordine degli addendi.

Se $A=\ENCLOSE{a_0, \dots, a_{n-1}}$ è un insieme finito si può definire 
\[
  \sum_{x\in A} f(x) = \sum_{k=0}^{n-1} f(a_k)
\]
ed è chiaro che la somma non dipende dall'ordine degli addendi.
\end{definition}

\begin{theorem}[famiglie sommabili]
  Sia $A = \displaystyle\bigcup_{n\in \NN} A_n$ una unione di insiemi 
  disgiunti: $A_n\cap A_m=\emptyset$ se $n\neq m$.
  Supponiamo che $A$ sia numerabile (o finito).
  Sia $f\colon A \to \RR$ non negativa oppure $f\colon A \to \CC$ 
  una funzione sommabile (cioè $\displaystyle\sum_{x\in A}\abs{f(x)}< +\infty$).
  Allora 
  \[
    \sum_{x\in A}f(x) = \sum_{n\in \NN} \sum_{x\in A_n} f(x).  
  \]
\end{theorem}
%
\begin{proof}
Sia $x_k$ una numerazione degli elementi di $A$ e definiamo 
la successione a due indici:
\[
  a_{n,k} = \begin{cases}
    f(x_k) & \text{se $x_k\in A_n$}\\
    0 & \text{altrimenti.}
  \end{cases}  
\]
Per ogni $k\in \NN$ la seguente somma 
ha un solo addendo non nullo:
\[
 \sum_{n=0}^{+\infty}  a_{n,k} = f(x_k)
\]
mentre per ogni $n\in \NN$ si ha 
\[
 \sum_{k=0}^{+\infty} a_{n,k} = \sum_{x\in A_n} f(x)  
\]
in quanto i termini non nulli della successione 
$a_{n,k}$ sono uguali al valore di $f$ su una enumerazione 
degli elementi di $A_n$.

Allora si ha
\[
 \sum_{k=0}^{+\infty}\sum_{n=0}^{+\infty} \abs{a_{n,k}}
 = \sum_{k=0}^{+\infty} {f(x_k)} = \sum_{x\in A} \abs{f(x_k)}.
\]
Se quest'ultima somma è finita oppure se i termini sono tutti non negativi
possiamo allora scambiare le due somme grazie al teorema~\ref{th:scambio_somma}:
\[
  \sum_{x\in A} f(x) 
  = \sum_{k=0}^{+\infty} \sum_{n=0}^{+\infty} a_{n,k}
  = \sum_{n=0}^{+\infty} \sum_{k=0}^{+\infty} a_{n,k}
  = \sum_{n\in \NN} \sum_{x\in A_n}f(x).  
\]
\end{proof}

%%%
\section{prodotti infiniti}
%%%
\index{prodotti infiniti}

Così come abbiamo fatto la teoria per le somme infinite si potrebbe fare
la teoria dei prodotti infiniti ponendo
\[
  \prod_{k=0}^{+\infty} a_k = \lim_{n\to +\infty} \prod_{k=0}^n a_k
  = \lim_{n\to +\infty} a_0 \cdot a_1 \cdots a_n.
\]

Supporremo sempre $a_k>0$ altrimenti il segno del prodotto difficilmente
sarebbe definito.
Allora, utilizzando il logaritmo (che trasforma prodotti in somme) possiamo
ricondurre i prodotti infiniti
alle serie:
\[
  \prod_{k=0}^{+\infty} a_k = \lim_{n\to +\infty} e^{\sum_{k=0}^n \ln a_k}.
\]

Osserviamo che se la serie dei logaritmi diverge a $-\infty$ il prodotto infinito
ha limite $0$. Avendo richiesto che i termini $a_k$ siano tutti positivi il prodotto
non potrà mai essere minore di zero. Per mantenere l'analogia con le serie diremo
che il prodotto infinito converge se il limite dei prodotti parziali è finito e positivo.
Diremo che diverge se il limite è $+\infty$ oppure $0$.

Dunque potremo dire che il prodotto infinito converge se e solo se la serie dei logaritmi converge.

Osserviamo quindi che condizione necessaria affinché un prodotto infinito
$\prod a_k$ sia convergente
dovrà essere $\ln a_k\to 0$ ovvero $a_k \to 1$. In tal caso visto che
\[
  \ln a_k = \ln (1+(a_k-1)) \sim a_k - 1
\]
si osserva che se $a_k\to 1$ il prodotto infinito $\prod a_k$ converge
se e solo se converge la serie $\sum (a_k-1)$.

\begin{example}[somma dei reciproci dei primi]
\index{primi!somma dei reciproci}%
\index{somma!dei reciproci dei primi}%
Possiamo utilizzare i prodotti infiniti per dimostrare che la somma
dei reciproci dei numeri primi è divergente.
Sia $p_k$ la successione dei numeri
primi ($p_1=2$, $p_2=3$, $p_3=5$, $\dots$ stiamo dando per scontato che i numeri
primi sono infiniti). Allora vogliamo dimostrare che
\begin{equation}\label{eq:489467523}
  \sum_{k=1}^{+\infty} \frac{1}{p_k} = \sum_{n=1}^{+\infty} \frac{1}{n} = +\infty.
\end{equation}

Questo risultato ha una certa rilevanza nell'ambito della teoria dei numeri
in quanto ci dice che $p_k$ non può andare all'infinito come una
potenza $k^\alpha$ con $\alpha>1$ in quanto la serie $\sum 1/k^\alpha$
è convergente.

Mostriamo quindi che vale~\eqref{eq:489467523}.
Si noti che per ogni $n\in \NN$ il termine $\frac 1 n$
può essere decomposto come il prodotto di
potenze dei reciproci dei numeri primi. Dunque:
\begin{equation}\label{eq:8834884}
\begin{aligned}
\sum_{n=1}^{+\infty} \frac{1}{n}
&= (1 + \frac 1 2 + \frac 1 {2^2} + \dots) \cdot
  (1 + \frac 1 3 + \frac 1 {3^2} + \dots) \\
  &\quad \cdot(1 + \frac 1 5 + \frac 1 {5^2} + \dots) \cdots \\
  &= \prod_{k=1}^{+\infty} \sum_{j=0}^{+\infty} \enclose{\frac{1}{p_k}}^j
  = \prod_{k=1}^{+\infty} \frac{1}{1-\frac 1 {p_k}}.
\end{aligned}
\end{equation}

Nei passaggi precedenti abbiamo sfruttato il fatto che nelle serie a termini
positivi possiamo riordinare e associare i termini in qualunque modo.
Una serie a termini positivi è convergente oppure divergente e la convergenza
assoluta coincide con la convergenza semplice.
Visto che riordinando i termini di una serie assolutamente convergente non
si può ottenere una serie divergente, significa che riordinando i termini di una
serie divergente (a termini positivi) si ottiene sempre una serie divergente.

Ora possiamo utilizzare il fatto che il prodotto infinito ottenuto
in \eqref{eq:8834884} ha lo stesso carattere
della seguente serie
\[
  \sum_{k=1}^{+\infty} \enclose{\frac{1}{1-\frac 1 {p_k}}-1}
  = \sum_{k=1}^{+\infty} \frac{\frac 1 {p_k}}{1-\frac{1}{p_k}}
\]
ma visto che $1/p_k\to 0$ si ha
\[
  \frac{\frac 1 {p_k}}{1-\frac{1}{p_k}}
  \sim \frac 1 {p_k}
\]
e dunque, per il criterio del confronto asintotico, la serie precedente ha
lo stesso carattere della serie
\[
\sum_{k=1}^{+\infty} \frac{1}{p_k}
\]
che quindi è divergente.
\end{example}

%%%
\section{le serie di potenze}
%%%

Se $a_k$ è una successione di numeri complessi, la serie
\[
 \sum a_k z^k
\]
dipendente dal parametro $z\in \CC$ si chiama
\emph{serie di potenze}%
\mymargin{serie di potenze}%
\index{serie!di potenze}
di coefficienti $a_k$.
Se chiamiamo $A\subset \CC$ l'insieme dei numeri complessi $z$
per i quali la serie di potenze converge
\[
A= \ENCLOSE{z\in \CC \colon \sum a_k z^k \text{ è convergente}}
\]
la somma della serie risulta essere una funzione
$f \colon A \to \CC$ definita da
\[
  f(z) = \sum_{k=0}^{+\infty} a_k z^k.
\]
L'insieme $A$ si chiama \emph{insieme di convergenza}%
\mymargin{insieme di convergenza}%
\index{insieme!di convergenza}
(o \emph{dominio di convergenza}) della serie
di potenze $\sum a_k z^k$.

Come al solito ci potrà capitare di considerare serie
di potenze con l'indice $k$ che parte da $1$ invece che da $0$
(o da qualunque altro numero naturale).
Ciò non è rilevante, potremo sempre considerare $a_k=0$ per i termini
che non partecipano alla sommatoria.

Osserviamo anche che per $k=0$ il termine corrispondente della serie è
$a_0 \cdot 0^0 = a_0$ in
quanto abbiamo definito $0^0=1$. Dunque potremo anche scrivere:
\[
  \sum_{k=0}^{+\infty} a_k z^k = a_0 + a_1 z + a_2 z^2 + \dots + a_n z^n + \dots.
\]
Le serie di potenze assomigliano quindi a dei polinomi, ma con infiniti termini.

\begin{example}[la serie geometrica]
La serie di potenze di coefficienti $a_k=1$ è la
serie geometrica $\sum z^k$.
L'insieme di convergenza è il cerchio
$A=\ENCLOSE{z\in \CC \colon \abs{z}<1}$.
Per $z\in A$ si ha
\[
 f(z) = \sum_{k=0}^{+\infty} z^k  = \frac{1}{1-z}.
\]
Se $\abs{z}\ge 1$ la serie non può convergere perché il termine $z^k$ non è
infinitesimo: $\abs{z^k} = \abs{z}^k \ge 1$.
\end{example}

\begin{example}
\label{ex:477474}
La serie di potenze $\sum \frac{z^n}{n^n}$ (ottenuta ponendo $a_n=1/n^n$)
ha come insieme di convergenza $A=\CC$
in quanto
\[
\sqrt[n]{\abs{\frac{z^n}{n^n}}} = \frac{\abs{z}}{n} \to 0 < 1
\]
e quindi per il criterio della radice la serie in questione converge assolutamente
qualunque sia $z\in \CC$.
\end{example}

\begin{theorem}[convergenza delle serie di potenze]
\mymark{***}
Se la serie di potenze $\sum a_k z^k$ converge in un punto $z\in \CC$
(anzi, basta che la successione $a_k z^k$ sia limitata)
allora la serie
converge assolutamente per ogni $w\in \CC$ tale che $\abs{w}< \abs{z}$.
Viceversa, se la serie non converge in un punto $z\in \CC$
allora
non  converge in nessun $w$ tale che $\abs{w} > \abs{z}$ (anzi la successione
$a_n w^n$ non è nemmeno limitata e tantomento infinitesima).
\end{theorem}
%
\begin{proof}
\mymark{*}
Se la serie $\sum a_k z^k$ converge significa che la successione
$a_k z^k$ è infinitesima e in particolare è limitata.
Esiste dunque $M$ tale che per ogni $k\in \NN$
\[
 \abs{a_k z^k} \le M.
\]
Se $z=0$ non c'è niente da dimostrare.
Se $z\neq 0$ si ha
\[
 \abs{a_k} \le \frac{M}{\abs{z}^k}.
\]
Scelto ora qualunque $w\in \CC$ con $\abs{w} < \abs{z}$ si ha
\[
  \abs{a_k w^k} = \abs{a_k}\cdot \abs{w}^k \le M \frac{\abs{w}^k}{\abs{z}^k}
  \le M q^k
\]
avendo posto $q = \frac{\abs{w}}{\abs{z}}$.
Essendo $q<1$ la serie geometrica $\sum q^k$ converge e, per confronto,
anche la serie $\sum \abs{a_k w^k}$ converge.
Dunque la serie $\sum a_k w^k$ converge assolutamente.

Viceversa supponiamo che $\sum a_k z^k$ non converga
e prendiamo $w$ con $\abs{w} > \abs{z}$.
Allora $\sum a_k w^k$ non può convergere,
anzi $a_k w^k$ non può neanche essere limitata perché
se lo fosse allora, scambiando i ruoli di $z$ e $w$,
per il punto precedente la serie $\sum a_k z^k$ dovrebbe convergere.
\end{proof}

\begin{corollary}[l'insieme di convergenza è circolare]%
\mymark{**}
\label{cor:insieme_convergenza}%
Sia $\sum a_k z^k$ una serie di potenze e sia $A$ il suo insieme di convergenza.
Allora $A$ non è vuoto e posto
\[
  R= \sup \ENCLOSE{\abs{z}\colon z\in A}.
\]
risulta che $R\in[0,+\infty]$ e $A$ coincide con il cerchio centrato in $0$
e di raggio $R$ a meno dei punti di bordo, nel senso che:
\begin{equation}\label{eq:48463}
   \ENCLOSE{z\in \CC \colon \abs{z} < R}
   \subset A
   \subset \ENCLOSE{z\in \CC \colon \abs{z}\le R}.
\end{equation}
Inoltre la serie converge assolutamente in ogni $z\in \CC$ con $\abs{z}<R$
mentre
il termine generico $a_k z^k$ non è nemmeno limitato (e quindi la serie non converge)
quando $\abs{z}>R$.
\end{corollary}
%
\begin{proof}
Se $\abs{z}<R$ significa che esiste $z_0\in A$ tale che $\abs{z_0} > \abs{z}$.
Ma visto che la serie converge in $z_0$ (per definizione di $A$) grazie al teorema
precedente possiamo affermare che la serie converge, anzi, converge assolutamente
in $z$. Dunque si ottiene la prima inclusione in~\eqref{eq:48463}.

Se invece prendiamo $z\in \CC$ con $\abs{z}>R$ allora
certamente la serie non converge in $z$ altrimenti (per come è definito $R$)
avremmo $R\ge \abs{z}$.
Inoltre possiamo certamente considerare un punto $z_1\in\CC$
tale che $R < \abs{z_1} \le \abs{z}$ e la serie
di potenze non può convergere neanche in $z_1$.
Ma allora, per il teorema precedente,
deduciamo che $a_k z^k$ non è nemmeno limitata.

Abbiamo quindi mostrato l'esistenza di $R\in \bar \RR$ che soddisfa \eqref{eq:48463}.
Chiaramente risulta $R\ge 0$ perché $A$ è un insieme non vuoto
(per ogni serie di potenze si ha $0\in A$).
\end{proof}

\begin{definition}[raggio di convergenza]
\mymark{**}
Il \emph{raggio di convergenza}%
\mymargin{raggio di convergenza}%
\index{raggio!di convergenza} di una serie di potenze $\sum a_k z^k$
è il valore $R\in[0,+\infty]$
dato dal corollario~\ref{cor:insieme_convergenza}:
\begin{align*}
  R &= \sup \ENCLOSE{\abs{z}\colon z\in \CC,\ \sum a_k z^k \text{ è convergente}}.
\end{align*}
\end{definition}

\begin{theorem}[calcolo del raggio di convergenza]
\mymark{***}
\label{th:calcolo_raggio_convergenza}
Sia $\sum a_n z^n$ una serie di potenze. Se esiste il limite
\begin{equation}\label{eq:9267345623}
  \lim_{n\to+\infty}\sqrt[n]{\abs{a_n}} =\ell
\end{equation}
allora $R=1/\ell$ è il raggio di convergenza della serie
(dove si intende $R=+\infty$ se $\ell = 0$ e $R=0$ se $\ell=+\infty$).

Lo stesso accade se esiste il limite
\[
  \lim_{n\to +\infty} \frac{\abs{a_{n+1}}}{\abs{a_n}} = \ell.
\]

Più in generale risulta $R=1/\ell$ se poniamo
\[
   \ell = \limsup_{n\to +\infty} \sqrt[n]{a_n}.
\]
anche nel caso in cui il limite in~\eqref{eq:9267345623}
non dovesse esistere.
\end{theorem}
%
\begin{proof}
\mymark{***}
Prendiamo $r\ge 0$.
Applicando il criterio della radice alla serie $\sum \abs{a_n} r^n$ si ha
\[
  \sqrt[n]{\abs{a_n} r^n}
  = r \sqrt[n]{\abs{a_n}} \to r\ell.
\]
Dunque se scegliamo un $r < 1/\ell$ si ha $r \ell<1$ e la serie
$\sum a_n z^n$ converge
assolutamente per $z=r$.
Dunque per ogni $r<1/\ell$
troviamo che $r\in A$: ne consegue che $R\ge 1/\ell$.
Se invece scegliamo $r > 1/\ell$ si ha $r \ell > 1$ e dunque
$\abs{a_n} r^n \to +\infty$ e la serie non può essere convergente
in $z=r$. Significa che $R\le 1/\ell$.

Il criterio della radice si applica anche nel caso in cui $\ell$ è definito
tramite $\limsup$.

Nel caso esista il limite del rapporto $\abs{a_{n+1}} / \abs{a_n}$
sappiamo (grazie al criterio di convergenza alla Cesàro teorema~\ref{th:criterio_cesaro}) che
il limite della radice coincide con il limite del rapporto e quindi
ci si riconduce al caso precedente (oppure si può ripetere la dimostrazione
utilizzando il criterio del rapporto invece del criterio della radice).
\end{proof}

\begin{example}
Nell'esempio~\ref{ex:477474} abbiamo visto
che l'insieme di convergenza della serie di potenze
\[
  \sum_{k=1}^{+\infty} \frac{z^k}{k}
\]
è
\[
  A = \ENCLOSE{ z\in \CC \colon \abs{z}\le 1, z\neq 1}.
\]
Il raggio di convergenza dovrà quindi essere $R=1$ e questo può essere
facilmente verificato con uno dei criteri precedenti. Ad esempio:
\[
  \lim_{n\to+\infty} \frac{\frac{1}{n+1}}{\frac{1}{n}}
  = \lim_{n\to+\infty} \frac{n}{n+1} = 1
\]
da cui $R=1/1=1$.
Si osservi dunque che nessuna delle due inclusioni in \eqref{eq:48463}
è, in questo caso, una uguaglianza.
\end{example}

\begin{theorem}[stabilità del raggio di convergenza]
\label{th:raggio_serie_derivate}
Le serie di potenze $\sum a_k z^k$ e $\sum k a_k z^k$ hanno
lo stesso raggio di convergenza.
\end{theorem}
%
\begin{proof}
  Sia $R$ il raggio di convergenza della serie $\sum a_k z^k$ 
  e $r$ il raggio di convergenza della serie $\sum k a_k z^k$.
  
  Visto che per $k>0$ si ha $\abs{a_k z^k} \le \abs{k a_k z^k}$
  la serie $\sum a_k z^k$ converge assolutamente nei punti 
  in cui converge assolutamente la serie $\sum k a_k z^k$.
  Questo significa che $R\ge r$.

  Preso ora un punto $w$ con $\abs{w}<R$ consideriamo un
  qualunque punto $z$ con $\abs{w}<\abs{z}<R$.
  La serie $\sum a_k z^k$ converge assolutamente in $z$ 
  quindi $\abs{a_k z^k}$ è infinitesima e limitata.
  Esiste quindi $M$ per cui $\abs{a_k z^k}\le M$ 
  che significa $\abs{a_k} \le \frac{M}{\abs{z}^k}$.
  Ma allora 
  \[
    \abs{k a_k w^k} 
    \le k \abs{a_k} \abs{w}^k
    \le k M \frac{\abs{w}^k}{\abs{z}^k}
  \]
  da cui, posto $q=\frac{\abs w}{\abs z}$, si ottiene 
  $\abs{k a_k w^k} \le k M q^k$. 
  Visto che $q<1$ sappiamo che la serie $\sum k M q^k$ 
  è convergente e dunque, per confronto, anche la 
  serie $\sum k a_k w^k$ è assolutamente convergente.
  Siccome questo è vero per ogni $w$ con $\abs{w}<R$
  significa che $r\ge R$ e questo conclude la dimostrazione.
\end{proof}
%
\begin{proof}[Dimostrazione alternativa]
Visto che $\sqrt[k]{k}\to 1$ si ha
\[
  \limsup \sqrt[k]{k a_k} = \limsup \sqrt[k]{a_k}.
\]
Per il teorema~\ref{th:calcolo_raggio_convergenza} si ottiene
che le due corrispondenti serie di potenze hanno lo stesso raggio di convergenza.
\end{proof}

\begin{theorem}(continuità delle serie di potenze)
\label{th:continuita_somma_serie}%
\mymargin{continuità delle serie di potenze}%
\index{continuità!delle serie di potenze}%
Sia $\sum a_k z^k$ una serie di potenze con raggio di convergenza $R$.
Allora posto $B=\ENCLOSE{z\in \CC\colon \abs{z}<R}$ la funzione $f\colon B \to \CC$
definita da
\[
 f(z) = \sum_{k=0}^{+\infty} a_k z^k
\]
è continua (su $B$).
\end{theorem}
%
\begin{proof}
Supponiamo $R>0$ altrimenti non c'è niente da dimostrare.
Fissiamo $r<R$ e consideriamo due punti $z,w\in \CC$ 
con $\abs{z}<r$ e $\abs{w}<r$.
Ricordiamo che si ha 
\[
  z^n - w^n = (z-w)\cdot(z^{n-1} + wz^{n-2} + \dots + w^{n-2}z + w^{n-1}).
\]
Osservando che per ogni addendo sul lato destro si ha 
$\abs{w^k z^{n-k-1}}\le r^{n-1}$, essendoci $n$ addendi si ottiene 
\[
  \abs{z^n - w^n} \le \abs{z-w}\cdot n\cdot  r^{n-1}.
\]
Dunque 
\begin{align*}
  \abs{f(z)-f(w)} 
  &= \abs{\sum_{n=0}^{+\infty} a_n z^n - \sum_{n=0}^{+\infty} a_n w^n}
   = \abs{\sum_{n=0}^{+\infty} a_n (z^n - w^n)} \\
& \le \sum_{n=0}^{+\infty} \abs{a_n} \cdot \abs{z^n - w^n}
\le \abs{z-w}\sum_{n=0}^{+\infty} \abs{a_n} n r^{n-1}.
\end{align*}
Ma noi sappiamo che la serie $\sum n a_n z^n$ 
ha lo stesso raggio di convergenza $R$ della serie originaria 
(per il teorema~\ref{th:raggio_serie_derivate})
dunque la serie $\sum n a_n r^n$ è assolutamente convergente e, 
dividendo per $r$, 
scopriamo che anche la serie $\sum n a_n r^{n-1}$
è convergente. 
Posto $L = \sum_{n=0}^{+\infty} n \abs{a_n} r^{n-1}$
per quanto scritto sopra possiamo concludere che 
\[
\abs{f(z)-f(w)} \le L\cdot \abs{z-w}.
\]
Dunque per ogni $\eps>0$ scelto $\delta = \frac{\eps}{L}$
se $\abs{z-w}<\delta$ risulta $\abs{f(z)-f(w)}\le \eps$.
Significa che la funzione $f$ è continua nel punto $z$ e 
questo è vero per ogni $z$ con $\abs{z}<R$.
\end{proof}

Il teorema precedente ci garantisce che la somma di una serie di potenze
è una funzione continua all'interno del raggio di convergenza.
Nei punti che si trovano esattamente sulla frontiera del raggio di convergenza
la funzione $f$ potrebbe non essere continua.
Ma se la serie converge in un
punto di frontiera, la somma della serie è continua se mi avvicino al punto di
convergenza lungo il raggio del disco di convergenza, come enunciato nel seguente teorema.

\begin{theorem}[lemma di Abel]
\mymark{*}%
\label{th:lemma_abel}%
\index{lemma!di Abel}%
\index{Abel!lemma di}%
Sia
\[
  f(z) = \sum_{k=0}^{+\infty} a_k z^k
\]
la somma di una serie di potenze. Se la serie converge in un punto $z_0\in \CC$, $z_0\neq 0$, allora la serie converge per ogni $z=t z_0$ con $t\in [0,1]$ inoltre la funzione
\[
  t \mapsto f(tz)
\]
è continua nel punto $t=1$.
\end{theorem}
%
\begin{proof}
Senza perdita di generalità possiamo supporre che sia $f(z_0)=0$ infatti basterà sostituire il primo termine della serie, $a_0$, con $a_0' = a_0 - f(z_0)$ e dimostrare il teorema per la serie modificata.
Dunque posto
\[
  A_n = \sum_{k=0}^{n-1} a_k z_0^k
\]
si ha che $A_n \to f(z_0) = 0$ e, per definizione, $A_0 = 0$.
Utilizzando la formula \eqref{eq:somma_per_parti} di somma per parti, preso $t\in [0,1)$
si avrà
\begin{align*}
\sum_{k=0}^n a_k \cdot (tz_0)^k
= \sum_{k=0}^n a_k z_0^k \cdot t^k
= A_{n+1} \cdot t^{n+1} + \sum_{k=0}^n A_{k+1}\cdot (t^k - t^{k+1})
\end{align*}
e per $n\to +\infty$ si ottiene
\[
  f(t\cdot z_0) = f(z_0)\cdot 0 + \sum_{k=0}^{+\infty}A_{k+1}(t^k-t^{k+1})
  = (1-t)\sum_{k=0}^{+\infty} A_{k+1} t^k.
\]
Visto che $A_n \to 0$ per ogni $\eps>0$ esiste $m$ tale che per ogni $k > m$ si ha $\abs{A_k} \le  \eps$. Dunque, per ogni $t\in [0,1)$, si ha
\begin{align*}
\abs{f(t\cdot z_0)}
 &\le (1-t)\sum_{k=0}^{m-1} \abs{A_{k+1}} t^k
  + (1-t)\sum_{k=m}^{+\infty} \abs{A_{k+1}} t^k \\
 &\le (1-t)\cdot \sum_{k=0}^{m-1}\abs{A_{k+1}} + (1-t)\cdot \eps \cdot \frac{t^{m}}{1-t} \\
 &\le (1-t)\sum_{k=0}^{m-1}\abs{A_{k+1}} + \eps.
\end{align*}
Scelto $\delta \le \frac{\eps} {\sum_{k=0}^{m-1} \abs{A_{k+1}}}$ se $\abs{1-t}<\delta$ si avrà
dunque
\[
  \abs{f(t\cdot z_0)-f(z_0)} = \abs{f(t\cdot z_0)} < 2\eps
\]
che significa che $t\mapsto f(t\cdot z_0)$ è continua nel punto $t=1$.
\end{proof}


\section{la serie esponenziale}

Consideriamo la funzione $f \colon \CC \to \CC$
definita da 
\begin{equation}\label{eq:def_exp}
f(z) = \sum_{k=0}^{+\infty} \frac{z^k}{k!}.
\end{equation}
La funzione è definita su tutto $\CC$ in quanto (grazie al criterio del rapporto)
è facile verificare che il raggio di convergenza di questa serie è $R=+\infty$.

Il nostro obiettivo sarà quello di dimostrare che $f(z)=e^z$ per ogni $z\in \CC$.
In effetti si potrebbe prendere~\ref{eq:def_exp} come definizione 
dell'esponenziale complesso e definire la funzione esponenziale 
reale e le funzioni trigonometriche di conseguenza.

\begin{theorem}[proprietà della serie esponenziale]
\label{th:exp_complesso}%
Sia $f$ la funzione definita da ~\eqref{eq:def_exp}.
Si ha:
\begin{enumerate}
\item
$\displaystyle f(0) = 1$;

\item
$\displaystyle f(\bar z) = \overline{f(z)}$;

\item
per ogni $z,w \in \CC$
\[
  f(z+w) = f(z) \cdot f(w);
\]

\item
per ogni $z\in \CC$ si ha $f(z) \neq 0$ e
\[
 f(-z) = \frac{1}{f(z)};
\]

\item la funzione $f\colon \CC \to \CC$ è continua;

\item se $z_n\to 0$ allora
\begin{equation}\label{eq:limite_exp_complesso}
   \lim_{n\to +\infty}\frac{f(z_n)-1}{z_n} = 1.
\end{equation}
\end{enumerate}
\end{theorem}
%
\begin{proof}
\mymark{*}
\begin{enumerate}
\item
La proprietà $f(0)=1$ si ottiene per verifica diretta (ricordiamo che $0^0=1$
e $0!=1$).

\item La proprietà $f(\bar z) = \overline{f(z)}$ si ottiene
passando al limite la seguente uguaglianza tra le somme parziali
che sfrutta le proprietà del coniugio di somma e prodotto:
\[
\sum_{k=0}^{n} \frac{\bar z^k}{k!} = \overline{\sum_{k=0}^n \frac{z^k}{k!}}.
\]


\item
Consideriamo la matrice infinita
\[
m_{k,j}  = \frac{z^k}{k!} \cdot \frac{w^j}{j!}.
\]
Allora da un lato
\begin{align*}
 f(z+w)
 &= \sum_{n=0}^{+\infty} \frac{(z+w)^n}{n!}\\
 &= \sum_{n=0}^{+\infty} \frac{1}{n!}\sum_{k=0}^n \frac{n!}{k!(n-k)!} z^k\cdot w^{n-k}\\
 &= \sum_{n=0}^{+\infty} \sum_{k=0}^n m_{k, n-k}
\end{align*}
e dall'altro
\begin{align*}
 f(z) \cdot f(w)
 &= \sum_{k=0}^{+\infty}\frac{z^k}{k!} \sum_{j=0}^{+\infty}\frac{w^j}{j!}
 = \sum_{k=0}^{+\infty}\sum_{j=0}^{+\infty}\frac{z^k}{k!} \frac{w^j}{j!} \\
 &= \sum_{k=0}^{+\infty}\sum_{j=0}^{+\infty} m_{k,j}.
\end{align*}

Il risultato segue dunque dal teorema~\ref{th:somma_Cauchy}.

\item
Visto che
\[
  1 = f(0) = f(z-z) = f(z) \cdot f(-z)
\]
ricaviamo che $f(z)\neq 0$ e $f(-z) =  1 / f(z)$.

\item
La continuità discende dal risultato generale sulla continuità della
somma di una serie di potenze: teorema~\ref{th:continuita_somma_serie}.

\item
Direttamente dalla definizione~\eqref{eq:def_exp}
si ottiene
\[
 f(z) - 1 = \sum_{k=1}^{+\infty} \frac{z^k}{k!}
 = z \cdot \sum_{k=0}^{+\infty} \frac{z^k}{(k+1)!}.
\]
Osserviamo ora che la serie $\sum \frac{z^k}{(k+1)!}$ ha raggio di
convergenza infinito e quindi è assolutamente convergente per ogni $z\in \CC$.
In particolare la somma di tale serie è continua e
quindi se $z_n \to 0$ si ha
\[
\lim_{n\to+\infty }\frac{\exp(z_n) - 1}{z_n}
   =  \lim_{n\to+\infty} \sum_{k=0}^{+\infty} \frac{z_n^k}{(k+1)!}
   = \sum_{k=0}^{+\infty}\frac{0^k}{(k+1)!} = 1.
\]
\end{enumerate}
\end{proof}

\begin{theorem}[coincidenza tra funzione esponenziale e serie esponenziale]%
\label{th:exp=ex}%
\index{funzione!esponenziale}%
\mymark{***}%
Per ogni $x\in \RR$ si ha
\[
  \sum_{k=0}^{+\infty} \frac{x^k}{k!} = e^x.
\]
\end{theorem}
%
\begin{proof}\mymark{**}
Vogliamo applicare il teorema~\ref{th:isomorfismo}.
Posto $f(x)=\sum_{k=0}^{+\infty} \frac{x^k}{k!}$, 
$f\colon \RR\to\RR$ sappiamo,
per il teorema precedente, che $f(x+y)=f(x)\cdot f(y)$.
Dobbiamo verificare che $f$ è crescente.
Innanzitutto notiamo che se $x> 0$ la serie che definisce
$f(x)$ è a termini positivi e il primo termine è $1$.
Dunque per ogni $x> 0$ si ha $f(x)>1$.
In particolare posto $a=f(1)$ si ha $a>1$.
Inoltre se $y>x$ si ha $f(y-x)>1$ e dunque
\[
 f(y) = f(x+y-x) = f(x)\cdot f(y-x) > f(x).
\]
La funzione è quindi strettamente crescente. 
Dunque è un omomorfismo monotono dal gruppo additivo $\RR$ al 
gruppo moltiplicativo $(0,+\infty)$.
Fissato $a=f(1)$ sappiamo,
per quanto visto nella sezione~\ref{sec:esponenziale},
che c'è una unica funzione con queste proprietà e dunque 
deve essere: 
\[
  f(x)= a^x.
\]
Non ci rimane che dimostrare che $a=e$.
Ma grazie al teorema~\ref{th:exp_complesso}
appena dimostrato sappiamo che per $x\to 0$ si ha
\[
  \frac{f(x)-1}{x}\to 1.
\]
D'altra parte sappiamo che $f(x)=a^x$ e quindi
per il limite notevole (corollario~\ref{cor:limite_notevole_e})
\[
\frac{f(x)-1}{x} = \frac{a^{x}-1}{x}
= \frac{e^{x \ln a}-1}{x \ln a} \cdot \ln a
\to \ln a.
\]
Deduciamo quindi che $\ln a= 1$ ovvero che $a=e$.
\end{proof}

Con una certa fatica sarebbe anche possibile dimostrare che 
effettivamente $f(z)=e^z$ (nel teorema precedente l'abbiamo dimostrato 
solamente per $x\in \RR$). 
Si tratterebbe di verificare che la funzione 
$\phi(x) = f\enclose{i\frac{x}{2\pi}}$
ha le proprietà enunciate nel teorema~\ref{th:omomorfismo_U}.
Sarà però più semplice utilizzare le derivate, quindi rimandiamo questa 
verifica al prossimo capitolo.

% Aver distinto le due definizioni di $e^x$ (tramite funzione potenza) e di $\exp(x)$ (tramite somma della serie esponenziale) è puramente strumentale.
% C'è una unica funzione esponenziale che può essere definita in
% un modo o nell'altro. Non ci si fissi quindi con l'identificare
% le due diverse notazioni $e^x$ ed $\exp(x)$ con le due diverse definizioni.
% Ogni testo avrà una sua definizione di funzione esponenziale che può
% essere per certi versi arbitraria salvo poi ritrovare le proprietà
% caratterizzanti di tale funzione.
% 
% D'ora in poi scriveremo, per ogni $z\in \CC$
% \[
%   e^z = \exp z
% \]
% considerando quindi $\exp z$ l'estensione a tutto il piano complesso
% della funzione esponenziale già definita
% sulla retta reale.

Per $x\in \RR$ abbiamo dimostrato che 
\[
  e^x = \lim_{n\to +\infty} \enclose{1+\frac x n}^n.  
\]
Il seguente teorema ci dice che la serie esponenziale $f(z)$ 
soddisfa la stessa proprietà per ogni $z \in \CC$.
Dunque in effetti ci fornisce una dimostrazione 
alternativa dell'identità $f(x) = e^x$ e un indizio ulteriore 
della coincidenza tra $f(z)$ ed $e^z$.

\begin{theorem}[limite notevole esponenziale complesso]%
\label{th:limite_notevole_esponenziale_complesso}%
\mymark{*}%
Per ogni $z\in \CC$ risulta
\[
  f(z) = \lim_{n \to +\infty} \enclose{1+\frac z n}^n  
\]
\end{theorem}
%
\begin{proof}
\mymark{*}
Utilizzando lo sviluppo del binomio osserviamo che si ha
\[
 \enclose{1+\frac z n}^n
 = \sum_{k=0}^n \binom{n}{k} \frac{z^k}{n^k}
 = \sum_{k=0}^n \frac{z^k}{k!} \cdot \frac{n!}{n^k\cdot (n-k)!}.
\]
Posto per ogni $k\le n$
\begin{align*}
 c(n,k)
  &= \frac{n!}{n^k\cdot (n-k)!}
  = \frac{n \cdot (n-1) \cdot \ldots \cdot(n-k+1)}{n^k} \\
  &= \frac{n}{n}\cdot {\frac {n-1} n} \cdot \frac {n-2} {n} \cdot \ldots \cdot \frac{n-k+1}{n}
\end{align*}
osserviamo che $0\le c(n,k)\le 1$ in quanto
prodotto di numeri non negativi minori o uguali ad $1$.
Inoltre, fissato $k$, si ha $c(n,k)  \to 1$ per $n\to +\infty$
in quanto ogni fattore $\frac{n-j}{n}$ tende
a $1$ per $n\to +\infty$ (si noti che a $k$ fissato il numero di fattori $k$ è
fissato).


Sia $z\in \CC$ fissato e sia
\[
  S = \sum_{k=0}^{+\infty} \frac{\abs{z}^k}{k!}.
\]
Sappiamo che la serie esponenziale è assolutamente convergente
per ogni $z\in \CC$ (in quanto il raggio di convergenza è $+\infty$)
quindi $S$ è un numero reale (finito).
Dunque per il teorema \ref{th:coda} (della coda) sappiamo che per ogni $\eps>0$
esiste $M$ tale che
\[
   \sum_{k=M+1}^{+\infty} \frac{\abs{z}^k}{k!} < \eps.
\]

Fissato $k\le M$ visto che $c(n,k)\to 1$
esiste $N_k > M$ tale che per ogni $n>N_k$
si abbia $1-c(n,k) < \eps$
(ricordiamo che $c(n,k)\le 1$).
Prendiamo allora
\[
  N=\max\ENCLOSE{N_k\colon k\le M}
\]
cosicchè per ogni $n>N$ e per ogni $k\le M$ si avrà $0 \le 1-c(n,k) < \eps$.
Allora, per ogni $n>N$, possiamo spezzare la somma da $0$ a $n$ nelle due
somme da $0$ a $M$ e da $M+1$ a $n$:
\begin{align*}
\abs{\sum_{k=0}^n \frac{z^k}{k!} - \enclose{1+\frac z n}^n}
&= \abs{\sum_{k=0}^n \enclose{\frac{z^k}{k!} - c(n,k)\frac{z^k}{k!}}}
= \abs{\sum_{k=0}^n  (1-c(n,k))\frac{z^k}{k!}} \\
&\le \sum_{k=0}^n  (1-c(n,k))\frac{\abs{z}^k}{k!} \\
  &= \sum_{k=0}^{M} (1-c(n,k)) \frac{\abs{z}^k}{k!}
   + \sum_{k=M+1}^n (1-c(n,k)) \frac{\abs{z}^k}{k!} \\
&\le \sum_{k=0}^{M} \eps \frac{\abs{z}^k}{k!}
   + \sum_{k=M+1}^n \frac{\abs{z}^k}{k!} \\
&\le  \eps \sum_{k=0}^{+\infty} \frac{\abs{z}^k}{k!}
    + \eps \\
&\le \eps S + \eps
= \eps (S+1).
\end{align*}

Visto che $\eps>0$ era arbitrario abbiamo verificato
tramite la definizione che
\[
\sum_{k=0}^n \frac{z^k}{k!} - \enclose{1+\frac z n}^n \to 0
\]
cioè
\[
\lim_{n\to +\infty} \enclose{1+\frac z n}^n = \sum_{k=0}^{+\infty} \frac{z^k}{k!}.
\]
\end{proof}

\begin{table}
\begin{center}
\begin{tabular}{r}
\ttfamily\footnotesize 2.7182818284 5904523536 0287471352 6624977572 4709369995 \\
\ttfamily\footnotesize   9574966967 6277240766 3035354759 4571382178 5251664274 \\
\ttfamily\footnotesize   2746639193 2003059921 8174135966 2904357290 0334295260 \\
\ttfamily\footnotesize   5956307381 3232862794 3490763233 8298807531 9525101901 \\
\ttfamily\footnotesize   1573834187 9307021540 8914993488 4167509244 7614606680 \\
\ttfamily\footnotesize   8226480016 8477411853 7423454424 3710753907 7744992069 \\
\ttfamily\footnotesize   5517027618 3860626133 1384583000 7520449338 2656029760 \\
\ttfamily\footnotesize   6737113200 7093287091 2744374704 7230696977 2093101416 \\
\ttfamily\footnotesize   9283681902 5515108657 4637721112 5238978442 5056953696 \\
\ttfamily\footnotesize   7707854499 6996794686 4454905987 9316368892 3009879312 \\
\ttfamily\footnotesize   7736178215 4249992295 7635148220 8269895193 6680331825 \\
\ttfamily\footnotesize   2886939849 6465105820 9392398294 8879332036 2509443117 \\
\ttfamily\footnotesize   3012381970 6841614039 7019837679 3206832823 7646480429 \\
\ttfamily\footnotesize   5311802328 7825098194 5581530175 6717361332 0698112509 \\
\ttfamily\footnotesize   9618188159 3041690351 5988885193 4580727386 6738589422 \\
\ttfamily\footnotesize   8792284998 9208680582 5749279610 4841984443 6346324496 \\
\ttfamily\footnotesize   8487560233 6248270419 7862320900 2160990235 3043699418 \\
\ttfamily\footnotesize   4914631409 3431738143 6405462531 5209618369 0888707016 \\
\ttfamily\footnotesize   7683964243 7814059271 4563549061 3031072085 1038375051 \\
\ttfamily\footnotesize   0115747704 1718986106 8739696552 1267154688 9570350354
\end{tabular}
\end{center}
\caption{Le prime 1000 cifre decimali del numero $e$
calcolate con il metodo utilizzato nella dimostrazione
del teorema~\ref{th:approx_e}.
Si veda il codice a pagina~\pageref{code:compute_e}.}
\label{fig:cifre_e}
\index{$e$!cifre decimali}
\index{cifre!$e$}
\end{table}

\begin{theorem}[approssimazione dell'esponenziale]
  \label{th:approx_exp}%
  \label{th:approx_e}%
  \index{$\exp$!approssimazione}%
  \index{approssimazione!di $\exp$}%
  Se $\abs{x}\le 1$ si ha 
  \begin{equation}\label{eq:stima_exp}
    \abs{e^x - \sum_{k=0}^n \frac{x^k}{k!}}
    \le \frac{\abs{x}^{n+1}}{n\cdot n!}.
  \end{equation}
  In particolare ponendo $x=1$ si trova
  \begin{equation}\label{eq:stima_e}
     0 < e - \sum_{k=0}^n \frac{1}{k!} \le \frac{1}{n \cdot n!}
  \end{equation}
  e per $n=5$ si ottiene
  \begin{equation}
    2.716 < e < 2.719
  \end{equation}
  \end{theorem}
  %
  \begin{proof}
  La coda della serie esponenziale può essere stimata 
  tramite la somma di una serie geometrica:
  \begin{align*}
    \abs{f(x) - \sum_{k=0}^n \frac {x^k} {k!}}
    & \le \sum_{k=n+1}^{+\infty} \frac{\abs{x}^k}{k!} 
     = \sum_{k=0}^{+\infty}\frac{\abs{x}^{n+k+1}}{(n+k+1)!} \\
    & \le \sum_{k=0}^{+\infty}\frac{\abs{x}^{n+1+k}}{(n+1)^{k+1}\cdot n!}
     = \frac{\abs{x}^{n+1}}{(n+1)\cdot n!}\sum_{k=0}^{+\infty} \enclose{\frac{\abs{x}}{(n+1)}}^k \\
    & = \frac{\abs{x}^{n+1}}{(n+1)\cdot n!} \cdot \frac{1}{1-\frac{\abs{x}}{(n+1)}} 
     = \frac{\abs{x}^{n+1}}{(n+1-\abs{x})\cdot n!}.
  \end{align*}
  Se $\abs{x}\le 1$ si ottiene quindi la stima~\eqref{eq:stima_exp}.
  Se $x=1$ la serie esponenziale è a termini positivi e dunque 
  approssima $e$ per difetto: si ottiene dunque~\eqref{eq:stima_e}.
  Per $n=5$ si ha
  \[
   \sum_{k=0}^5 \frac{1}{k!} = 1 + 1 + \frac 1 2 + \frac{1}{6} + \frac {1}{24} + \frac{1}{120}
   = \frac{326}{120}
  \]
  Dunque da un lato
  \[
    e \ge \frac{326}{120} \ge 2.716
  \]
  e dall'altro
  \[
   e \le \frac{326}{120} + \frac{1}{5\cdot 5!}
     \le 2.717 + 0.002 = 2.719
  \]
\end{proof}
  
\begin{theorem}[irrazionalità di $e$]
\mymargin{$e\not\in \QQ$}%
\index{$e\not\in \QQ$}%
\mymark{**}%
\index{irrazionalità!di $e$}%
\index{$e$!è irrazionale}%
Il numero $e$ è irrazionale.
\end{theorem}
%
\begin{proof}
Supponiamo per assurdo che sia $e=p/q$ con $p\in \ZZ$ e $q \in \NN$.
Possiamo supporre $q>1$
(non importa che la frazione sia ridotta ai minimi termini).

Per ogni $n\in \NN$ si ha 
\[
  n! e = \sum_{k=0}^{+\infty} \frac{n!}{k!}
   = \sum_{k=0}^n \frac{n!}{k!} + n!\sum_{k=n+1}^{+\infty} \frac{1}{k!}.
\]
Il primo addendo nella somma precedente è intero
in quanto se $k\le n$ il rapporto $n!/k!$ è intero.
Grazie al teorema~\ref{th:approx_e}
per il secondo addendo se $n>1$ si ha:
\[
0 < n! \sum_{k=n+1}^{+\infty} \frac{1}{k!}
\le \frac{1}{n} < 1.
\]
Risulta dunque che per ogni $n\ge 2$ il numero $n! e$
è strettamente compreso tra due interi consecutivi e quindi non 
è mai intero. Dunque $e$ non può essere razionale perché 
se fosse $e=\frac p q$ si avrebbe $n! e \in \ZZ$ per ogni 
$n>q$.
\end{proof}

\begin{comment}
\subsection{definizione alternativa delle funzioni trigonometriche}

In questa sezione proponiamo una definizione alternativa 
delle funzioni trigonometriche a partire dall'esponenziale complesso.

\begin{definition}[funzioni trigonometriche]%
\label{def:sincos}%
\index{$\cos$}%
\index{$\sin$}%
\index{funzioni!trigonometriche}%
\index{funzioni!circolari}%
\index{trigonometria!funzioni elementari}%
Per ogni $x\in \RR$ si potrà definire
\[
  \cos x = \Re \enclose{e^{ix}}, \qquad
  \sin x = \Im \enclose{e^{ix}}
\]
cosicché valga per definizione la
\emph{formula di Eulero}%
\mymargin{formula di Eulero}%
\index{formula!di Eulero}%
\index{Eulero!formula di}%
\mymark{***}%
\[
  e^{ix} = \cos x + i \sin x.
\]
\end{definition}

\begin{theorem}[proprietà delle funzioni seno e coseno]
\index{proprietà!delle funzioni seno e coseno}%
\mymark{***}%
Se definiamo 
\[
\cos x = \Re e^{ix},
\]
Le funzioni $\sin$ e $\cos$ appena definite
soddisfano le stesse proprietà 
delle corrispondenti funzioni definite 
nel capitolo~\ref{sec:funzioni_trigonometriche}:
\mynote{Queste proprietà caratterizzano univocamente le funzioni 
$\sin$ e $\cos$ e quindi effettivamente coincidono con quelle 
che avevamo già definito. 
La dimostrazione più semplice richiede però l'utilizzo delle 
derivate.
}
\begin{enumerate}
\item
$\displaystyle
\cos(x) = \frac{e^{ix}+e^{-ix}}{2}$,
$\displaystyle
\sin(x) = \frac{e^{ix}-e^{-ix}}{2i}$;
\item
$\sin(-x) = -\sin x$ (la funzione $\sin$ è dispari),
$\cos(-x) = \cos x$ (la funzione $\cos$ è pari);
\item identità fondamentale della trigonometria:
\index{trigonometria!identità fondamentale}%
\index{identità!fondamentale della trigonometria}%
\index{formula!fondamentale della trigonometria}%
\[
\cos^2 x + \sin^2 x = 1;
\]
\item
formule di addizione:
\index{formula!di addizione}%
\index{trigonometria!formule di addizione}%
\index{addizione!formule della trigonometria}%
\begin{gather*}
\cos(\alpha+\beta) = \cos \alpha \cos \beta - \sin \alpha \sin \beta,\\
\sin(\alpha+\beta) = \sin \alpha \cos \beta + \cos \alpha \sin \beta;
\end{gather*}
\item
le funzioni $\cos\colon \RR\to\RR$ e $\sin \colon \RR \to \RR$
sono continue;
\item si ha
\index{sviluppo in serie!coseno}%
\index{sviluppo in serie!seno}%
\index{$\cos$!sviluppo in serie}%
\index{$\sin$!sviluppo in serie}%
\begin{align}
\label{eq:serie_cos}
\cos x &= \sum_{k=0}^{+\infty} (-1)^k\frac{x^{2k}}{(2k)!}
  = 1 - \frac{x^2}{2} + \frac{x^4}{4!} - \frac{x^6}{6!} + \dots
\\
\label{eq:serie_sin}
\sin x &= \sum_{k=0}^{+\infty} (-1)^k\frac{x^{2k+1}}{(2k+1)!}
  = x - \frac{x^3}{6} + \frac{x^5}{5!} - \frac{x^7}{7!} + \dots
\end{align}
\item
vale il seguente limite notevole
\[
 \lim_{x\to 0} \frac{\sin x}{x} = 1
 \qquad\text{e}\qquad
 \lim_{x\to 0} \frac{1-\cos x}{x^2} = \frac 1 2.
\]
\end{enumerate}
\end{theorem}
%
%
\begin{proof}
\mymark{*}
\begin{enumerate}
\item
Essendo $\overline{e^{ix}} = e^{\overline{ix}} = e^{-ix}$
discende dalla formula~\eqref{eq:re_im} per il calcolo
di parte reale ed immaginaria.

\item
Si verifica direttamente con le formule precedenti.

\item
Per $x\in \RR$ si ha da un lato
\[
  \abs{\exp (ix)}^2 = \exp(ix)\cdot \overline{\exp(ix)}
  = \exp(ix)\cdot \exp(-ix)
  = \exp(0) = 1
\]
e dall'altro
\[
  \abs{\exp(ix)}^2 = \abs{\cos x + i \sin x}^2
    = \cos^2 x + \sin^2 x.
\]

\item
Grazie alla formula che esprime l'esponenziale
della somma:
\begin{align*}
\cos(\alpha+\beta) + i \sin(\alpha + \beta)
&= \exp(i(\alpha + \beta))
= \exp(i\alpha) \cdot \exp(i\beta) \\
&= (\cos \alpha + i \sin \alpha) \cdot (\cos \beta + i \sin \beta)\\
&= \cos \alpha \cos \beta - \sin \alpha \sin \beta \\
&\quad + i (\sin \alpha \cos \beta + \cos \alpha \sin \beta)
\end{align*}
e uguagliando parte reale e parte immaginaria si ottengono le formule
di addizione.

\item
Visto che la funzione $\exp\colon \CC \to \CC$ è continua
anche la sua restrizione all'asse immaginario lo è.
E dunque anche parte reale (coseno) e parte immaginaria (seno) lo sono.

\item
Sia $x\in \RR$.
Osservando che $i^{2k} = (i^2)^k = (-1)^k$ e $i^{2k+1}= i \cdot i^{2k}
= i\cdot(-1)^k$ suddividendo i termini pari e dispari
della serie che definisce l'esponenziale si ha:
\begin{align*}
  \exp(ix)
  &= \sum_{k=0}^{+\infty} \frac{i^k x^k}{k!}
  = \sum_{k=0}^{+\infty}\frac{i^{2k} x^{2k}}{(2k)!}
    + \sum_{k=0}^{+\infty}\frac{i^{2k+1} x^{2k+1}}{(2k+1)!}\\
  &= \sum_{k=0}^{+\infty}\frac{(-1)^k x^{2k}}{(2k)!}
    +  i \sum_{k=0}^{+\infty}\frac{(-1)^k x^{2k+1}}{(2k+1)!}.
\end{align*}
Osserviamo ora che le due serie che compaiono a destra dell'uguaglianza
sono a termini reali e quindi la loro somma è reale.
Dunque queste due serie coincidono con la parte reale e la parte immaginaria di $\exp(ix) = \cos x + i \sin x$.

\item
Per la corrispondente proprietà dell'esponenziale
sappiamo che per $x \to 0$ si ha
\[
  \frac{e^{i x}-1}{i x} \to 1.
\]
Ma
\[
  \frac{e^{ix}-1}{i x}
  = \frac{\cos x - 1 + i \sin x}{i x}
  = \frac{\sin x}{x} + i\frac{1- \cos x}{x}.
\]
Scopriamo dunque che
\[
  \frac{1-\cos x}{x} \to 0
  \qquad\text{e}\qquad
  \frac{\sin x}{x} \to 1.
\]
D'altra parte si ha
\begin{align*}
  \frac{1-\cos x}{x^2}
  &=\frac{(1-\cos x)\cdot(1+\cos x)}{x^2(1+\cos x)}
  = \frac{1-\cos^2 x}{x^2} \cdot \frac{1}{1+\cos x}\\
  &= \enclose{\frac{\sin x}{x}}^2\cdot \frac 1{1+\cos x}
  \to 1 \cdot \frac 1 {1+\cos 0} = \frac 1 2.
\end{align*}

\end{enumerate}
\end{proof}

Definiamo $\pi$ come la misura in radianti dell'angolo piatto, 
cioè dell'angolo identificato dal numero $e^{i\pi} = -1$ 
sul piano complesso.

\begin{theorem}[definizione di $\pi$]
\index{$\pi$!definizione}%
\index{$\tau$!definizione}%
\label{th:pi}%
\mymargin{$\pi$}%
Definiamo $\pi$ come il più piccolo numero reale
positivo in cui si annulla la funzione $\sin x$.
Si ha $\pi \in [2.8,4]$.
Le funzioni $e^{ix}$, $\sin x$ e $\cos x$ sono
periodiche di periodo $\tau = 2\pi$:
\mymargin{$\tau$}%
\[
  e^{i(x+2\pi)} = e^{ix}, \qquad
  \cos(x + 2\pi) = \cos x, \qquad
  \sin(x + 2\pi) = \sin x.
\]
La funzione $\sin x$ è strettamente crescente nell'intervallo
$\closeinterval{-\frac \pi 2}{\frac \pi 2}$
mentre la funzione $\cos x$ è strettamente decrescente 
nell'intervallo $\closeinterval{0}{\pi}$.
Si ha infine:
\begin{align*}
\sin(\pi + x) = -\sin x, \qquad 
\cos(\pi + x) = -\cos x.
\end{align*}

Vale inoltre la celeberrima formula di Eulero:
\index{Eulero!formula di}%
\index{formula!di Eulero}%
\[
  e^{i\pi} + 1 = 0.
\]
\end{theorem}
%%
%%
\begin{proof}
Il teorema~\ref{th:approx_exp} ci permette di approssimare la 
funzione esponenziale $e^{ix}$ per $x\in [0,1]$. 
Ci proponiamo quindi di determinare l'andamento delle funzioni 
$\sin x$ e $\cos x$ in tale intervallo e definire in tale intervallo 
il valore di $\alpha=\frac \pi 4$.
Tramite tale valore $\alpha$ e tramite le formule di addizione riusciremo 
a determinare l'andamento delle funzioni $\sin$ e $\cos$ su tutta la 
retta reale.

Se $x\in[0,1]$ grazie al teorema~\ref{th:approx_exp} applicato 
con $n=3$ sappiamo che 
\begin{align*}
  \abs{e^{ix} - \enclose{1 + ix  + \frac{(ix)^2}{2} + \frac{(ix)^3}{6}}}
  \le \frac{\abs{ix}^4}{18}
\end{align*}
sviluppando
\begin{align*}
  \abs{\cos x + i \sin x - 1 - ix + \frac{x^2}{2} + i\frac{x^3}{6}}
  \le \frac{x^4}{18}.
\end{align*}
Il valore assoluto delle parti reale e immaginaria di un numero 
complesso sono minori del modulo del numero complesso.
Dunque si ottiene:
\begin{align}\label{eq:86035}
    \abs{\cos x - 1 + \frac{x^2}{2}} \le \frac{x^4}{18}\smallskip
    \qquad\text{e}\qquad
    \abs{\sin x - x + \frac{x^3}{6}} \le \frac{x^4}{18}.
\end{align}
Per la funzione $\sin x$ si ha dunque
\begin{align*}
    x - \frac{x^3}{6} - \frac{x^4}{18}
    \le \sin x 
    \le x - \frac{x^3}{6} + \frac{x^4}{18}
\end{align*}
e sapendo che $0<x<1$ possiamo affermare che 
\[
  x - \frac{x^3}{6} - \frac{x^4}{18}
  \ge x - \frac{x}{6} - \frac{x}{18}
  = \frac{7}{9} x
\]
e 
\[
  x - \frac{x^3}{6} + \frac{x^4}{18}
  \le x - \frac{x^3}{6} + \frac{x^3}{18}
  \le x.
\]
Dunque se $x\in[0,1]$ si ha 
\begin{equation}\label{eq:4757614}
  \frac{7}{9} x \le \sin x \le x.
\end{equation}
Per il coseno da un lato osserviamo che da~\eqref{eq:86035}
si ottiene, se $x\in [0,1]$:
\[
\cos x 
  \ge 1 - \frac{x^2}{2} -\frac{x^4}{18}   
  \ge 1 - \frac{1}{2} - \frac{1}{18}
  = \frac 4 9 > 0.
\]
Dall'altro lato si ha
\[
\cos x 
\le 1 - \frac{x^2}{2} + \frac{x^4}{18}
= 1 - \frac{x^2}{2} + \frac{x^2}{18}
\le 1 - \frac{4}{9} x^2.
\]

Vogliamo ora dimostrare che la funzione $\cos x$ 
è strettamente decrescente sull'intervallo $[0,1]$.
Infatti se $x\in[0,1]$ e $t\in(0,1]$ si ha 
\begin{align*}
  \cos(x+t) 
  = \cos x \cos t - \sin x \sin t
  \le \cos x \cos t < \cos x
\end{align*}
in quanto $\sin x\ge 0$, $\sin t \ge 0$ 
e $\cos t < 1 - \frac{5}{9}t^2 < 1$ se $t>0$.
Dunque la funzione $\cos x$ è strettamente 
decrescente su $[0,1]$. 
Visto che $\cos^2 x + \sin^2 x = 1$ 
deduciamo che la funzione $\sin^2 x$ 
è strettamente crescente in $[0,1]$.
Inoltre $\sin x$ è positiva su tale 
intervallo e dunque anche $\sin x$
è strettamente crescente in $[0,1]$.

Vogliamo ora dimostrare che esiste 
$\alpha \in [0,1]$ tale per cui 
$\sin \alpha = \frac{\sqrt 2}{2}$.
Visto che $\sin x$ è una funzione continua, 
e $\sin 0 = 0$, per utilizzare 
il teorema~\ref{th:zeri} dei valori intermedi 
basterà verificare che $\sin 1 > \frac{\sqrt 2}{2}$.
E infatti si ha:
\[
\sin 1 \ge \frac 7 9 > \frac{\sqrt 2}{2}.
\]
Dunque esiste un unico $\alpha\in [0,1]$ 
tale che $\sin \alpha = \frac{\sqrt 2}{2}$.
Allora anche $\cos \alpha = \frac{\sqrt 2}{2}$ 
(in quanto $\sin^2 + \cos^2 = 1$)
e quindi
\[
  e^{2i\alpha} 
  = \enclose{\frac{\sqrt 2}{2} \enclose{1+i}}^2
  = \frac 1 2 \enclose{1+2i-1} = i.
\]
Dunque $\alpha$ è la misura in radianti di metà 
angolo retto.

Se poniamo $\pi = 4\alpha$ scopriamo dunque che 
\[
  e^{i\pi} 
  = \enclose{e^{2i\alpha}}^2
  = i^2 = -1
\]
da cui $\sin \pi = 0$ e $\cos \pi = -1$.
Analogamente da $e^{2i\alpha}=i$ 
troviamo che $\sin \frac \pi 2 = 1$
$\cos \frac \pi 2 = 0$.

Dalla stima~\eqref{eq:4757614} si ottiene 
\[
  \frac 7 9 \alpha 
  \le \frac{\sqrt 2}{2}
  \le \alpha
\]
da cui $\pi=4\alpha \in [2.8, 4]$.

Abbiamo verificato che sull'intervallo 
$\closeinterval{0}{\frac \pi 4}$ la funzione $\sin x$ è strettamente 
crescente mentre $\cos x$ è strettamente decrescente.
Dalle formule di addizione possiamo quindi dedurre che 
$\sin\enclose{\frac \pi 2 - x} = \cos x$ 
e $\cos\enclose{\frac \pi 2 -x} = \sin x$.
Dunque anche sull'intervallo $\closeinterval{\frac \pi 4}{\frac \pi 2}$ 
la funzione $\sin x$ è crescente e $\cos x$ è decrescente.
Ancora, tramite le formula di addizione troviamo che 
$\sin(\pi - x)=\sin x$ e $\cos(\pi - x) = -\cos x$ 
da cui possiamo dedurre che sia la funzione $\sin x$ 
che la funzione $\cos x$ sono strettamente decrescenti 
sull'intervallo $\closeopeninterval{\frac \pi 2}{\pi}$.
Possiamo dunque affermare che $\sin x>0$ se $0<x<\pi$ 
e dunque $\pi$ è il più piccolo reale positivo in cui 
la funzione $\sin$ si annulla.

Visto che $\sin(\pi+x) = -\sin x$ e $\cos(\pi+x)=-\cos x$
possiamo stabilire l'andamento delle funzioni $\sin$ e $\cos$
anche sull'intervallo $\closeinterval{\pi}{2\pi}$.
Infine essendo $e^{2i\pi} = 1$ si osserva che la funzione 
$e^{2ix}$ e di conseguenza le funzioni $\sin x$ e $\cos x$
sono periodiche di periodo $2\pi$.
\end{proof}
\end{comment}

%%%%
%%%%
%%%%


\section{esercizi}

\begin{exercise}
  Utilizzando il teorema~\ref{th:approx_e} dimostrare che
  \[
  \lim_{n\to +\infty} n \sin(2\pi e\, n!) = 2\pi.
  \]
\end{exercise}

\begin{exercise}
Determinare il carattere delle seguenti serie
\[
    \sum_n \frac{n^2-n^3}{3^n}, \qquad
    \sum_n \frac{(n!)^2}{(2n)!}
\]
\[
\sum_n \frac{(-1)^n}{\ln\abs{n^7 - 10n^5 + 3}},  \qquad
\sum_n \frac{n-10}{n^2+10}
\]
\end{exercise}
  
\begin{exercise}
  Determinare il carattere delle seguenti serie
  \[
    \sum_n \enclose{\frac 1 n - \sin \frac 1 n},\qquad
    \sum_n \sin\enclose{\pi n + \frac 1 n}
  \]
\end{exercise}
  
  

\chapter{i numeri complessi}
%%%%%%%%%%%%%%%%%%%%%%%%
%%%%%%%%%%%%%%%%%%%%%%%%
%%%%%%%%%%%%%%%%%%%%%%%%
%%%%%%%%%%%%%%%%%%%%%%%%

Nel capitolo precedente abbiamo introdotto l'esponenziale complesso ed
abbiamo osservato che la funzione $f\colon \RR \to \CC$ definita da
$f(t) = e^{it}$ ha valori sulla circonferenza unitaria in quanto
$\abs{e^{it}}=1$. Tramite la definizione~\ref{def:sincos}
abbiamo introdotto le funzioni seno e coseno in
modo che risulti $f(t) = \cos t + i \sin t$.
Sappiamo che $f(0) = e^0 = 1$ e, per come abbiamo definito $\pi$,
sappiamo che $f(\pi/2) = i$.

\section{rappresentazione polare dei numeri complessi}

I numeri complessi di modulo uno vengono chiamati \emph{unitari}.
\mynote{complessi unitari}%
\index{complessi!unitari}%
\index{unitario}%
Geometricamente i numeri complessi unitari sono i punti della circonferenza
unitaria centrata nell'origine del piano complesso.
Se $z=x+iy$ è unitario si ha $x^2+y^2=1$.
I prodotti e i
reciproci dei numeri complessi unitari sono anch'essi unitari,
risulta quindi che tali numeri formano un \emph{sottogruppo moltiplicativo}%
\footnote{%
Un \emph{gruppo} è un insieme su cui è definita una operazione
(spesso denotata con il simbolo della moltiplicazione) che sia associativa,
che abbia elemento neutro e tale che ogni elemento abbia un inverso.
}
del gruppo dei numeri complessi.

Ogni numero complesso $z$ potrà essere scritto nella forma
\[
  z = \rho \cdot u
\]
con $\rho\in \RR$, $\rho>0$ e $u\in\CC$ unitario.
Basta infatti
definire $\rho = \abs{z}$ e $u = z / \abs{z}$ (se $z\neq 0$, altrimenti
si potrà scegliere arbitrariamente $u=1$).

\begin{theorem}[argomento]
\index{fase}%
Sia $z\in \CC$ un numero complesso non nullo. Allora esiste un
unico $\theta\in [0,2\pi)$ tale che
\[
z = \abs{z} e^{i\theta}.
\]
Denoteremo tale valore di $\theta$ come l'\emph{argomento}%
\mynote{argomento}%
\index{argomento} 
(o \emph{fase}) di $z$
e scriveremo
\[
  \theta = \arg z.
\]
Se $z=0$ porremo per convenzione $\arg z=0$.
\end{theorem}
%
\begin{proof}
Sia $z\in \CC$, $z\neq 0$ e poniamo $u = z/\abs{z}$.
Posto $u=x+iy$ con $x,y\in \RR$ si ha $\abs{u}^2 = x^2+y^2=1$.

Se $y\ge 0$ se vogliamo che $y=\sin \theta$ con $\theta\in [0,2\pi)$
dovrà necessariamente essere $\theta \in [0,\pi]$
(altrimenti si avrebbe $\sin \theta<0$).
E se vogliamo che sia $x=\cos \theta$ basterà (e si dovrà) scegliere
$\theta = \arccos x$.
Visto che $\sin^2 \theta + \cos^2\theta = 1 = x^2 + y^2$
sapendo che $\cos \theta = x$
si avrà
$\sin^2 \theta = y^2$ cioè $\abs{\sin \theta}=\abs y$.
Ma visto che $y\ge 0$ e $\sin \theta\ge 0$
avremo, come voluto, $\sin \theta = y$.
Se $y<0$ poniamo $\theta = 2\pi - \arccos x$
cosicché si avrà $\theta \in (\pi, 2\pi)$
e, come prima (verificare!),
$x=\cos \theta$, $y=\sin \theta$.

In ogni caso per ogni $z\neq 0$ abbiamo quindi trovato l'unico
$\theta\in [0,2\pi)$ tale che
\[
  u
  = x + i y
  = \cos \theta + i\sin \theta
  = e^{i\theta}.
\]
da cui
\[
  z = \abs{z}\cdot e^{i\theta}.
\]
\end{proof}

Per ogni $z\in \CC$ posto
\[
  \rho = \abs{z}, \qquad \theta = \arg z
\]
si avrà quindi la \emph{rappresentazione esponenziale} o \emph{polare}
\[
  z = \rho e^{i\theta} = \rho \cdot \enclose{\cos \theta
  + i \sin \theta}.
\]
Se $z=x+iy$ è la \emph{rappresentazione cartesiana}, si avranno
le seguenti formule di conversione:
\[
\begin{cases}
  x = \rho \cos \theta\\
  y = \rho \sin \theta
\end{cases}
\qquad
\begin{cases}
 \rho = \sqrt{x^2+y^2}\\
 \theta = \arg z
\end{cases}
\]
con
\[
  \arg z =
  \begin{cases}
%   \arctg \frac y x & \text{se $x>0$,} \\
   \frac \pi 2 - \arctg \frac x y & \text{se $y>0$,} \\
   \frac 3 2 \pi- \arctg \frac x y & \text{se $y<0$,} \\
   \pi & \text{se $y=0$ e $x<0$,} \\
   0 & \text{se $y=0$ e $x\ge 0$.}
   \end{cases}
\]

\section{interpretazione geometrica}
\label{sec:radianti}

Vogliamo ora interpretare geometricamente $\theta = \arg z$ come la misura
di un angolo. Per fare ciò dobbiamo però capire cosa si intende per angolo
e come si misura un angolo.
In questa sezione ragioneremo in maniera intuitiva in quanto le proprietà
formali analitiche delle operazioni sui numeri complessi
sono già state determinate e quello
che vogliamo fare è darne una interpretazione geometrica.
Daremo quindi per scontate le proprietà geometriche del piano euclideo.

Un \emph{angolo},
geometricamente, è la regione piana delimitata da due semirette
uscenti da uno stesso punto. Le due semirette si chiamano \emph{lati}
dell'angolo e il punto in comune si chiama \emph{vertice}.

Due angoli $\alpha,\beta$ si dicono congruenti se è possibile traslare e ruotare uno dei
due angoli (diciamo $\alpha$) in modo che si sovrapponga all'altro.
In particolare sarà sempre possibile trovare una traslazione
che manda il vertice dell'angolo $\alpha$ sul vertice dell'angolo $\beta$
e sarà possibile trovare una rotazione che fa coincidere uno dei due lati
in modo che uno dei due angoli copra interamente l'altro.
Sia $\alpha'$ la roto-traslazione di $\alpha$ così individuata.
Se $\alpha'=\beta$ diremo che i due angoli sono congruenti,
altrimenti se
$\alpha' \supset \beta$ diremo che $\alpha$ è maggiore di $\beta$ e se
invece $\alpha' \subset \beta$ diremo che $\alpha$ è minore di $\beta$.
Con un procedimento simile è in genere possibile sommare due angoli:
si sposta uno dei due tramite una roto-traslazione in modo da far coincidere
il vertice e un lato dei due angoli e in modo
che i due angoli non si sovrappongano
(questo non è sempre possibile perché se gli angoli sono troppo grandi
si sovrapporranno sempre).
La loro unione sarà un angolo
che chiamiamo somma degli angoli dati.

Misurare un angolo significa associare ad ogni angolo un numero (reale positivo)
in modo che
si abbiano le seguenti proprietà: angoli congruenti hanno la stessa misura,
angoli maggiori hanno misure maggiori (proprietà di monotonia)
e la misura della somma di due angoli è la somma delle misure
(additività).
Si può intuire che una volta scelto quale angolo ha misura $1$
(l'unità di misura) la misura di ogni altro angolo sarà univocamente determinata
da queste proprietà. Infatti la misura dei multipli e dei sottomultipli
dell'unità è determinata dalla additività della misura e la misura
degli angoli incommensurabili si potrà ottenere per approssimazione
sfruttando la monotonia.

Se ora identifichiamo il piano euclideo con il piano complesso ogni angolo
potrà essere traslato e ruotato in modo che uno dei due lati vada
a coincidere con la semiretta dei reali positivi e in modo che l'angolo si
estenda al di sopra di tale semiretta e sia delimitato da una seconda
semiretta passante per un punto $u$ a distanza $1$ dall'origine
ovvero con $\abs{u}=1$. Vogliamo giustificare il fatto che $\theta = \arg u$
può essere scelto come misura dell'angolo.
Studiando la monotonia delle funzioni $\cos$ e $\sin$ si può verificare facilmente
che $\theta$ è crescente con l'angolo. Più rilevante è chiedersi se
$\theta$ è additivo.
Siano $u,v\in \CC$ due numeri complessi unitari che rappresentino due diversi
angoli. Sia $\theta = \arg u$ e $\phi=\arg v$ da cui
$u=e^{i\theta}$ e $v=e^{i\phi}$.
Consideriamo la trasformazione $R_\theta(z) = e^{i\theta}\cdot z$.
Si nota che $R_\theta$ è una trasformazione rigida del piano ovvero
una trasformazione che mantiene la distanza tra i punti
(una \emph{isometria})
in quanto essendo $\abs{e^{i\theta}}=1$ si ha
\[
\abs{R_\theta(z)-R_\theta(w)} = \abs{e^{i\theta}z-e^{i\theta}w}
= \abs{e^{i\theta}}\cdot\abs{z-w} = \abs{z-w}.
\]
Inoltre $R_\theta(0)=0$ e $R_\theta(1)=u$ significa che $R_\theta$ non è altro che una
rotazione che tiene fissa l'origine e manda la semiretta dei reali
positivi nella semiretta uscente da $0$ e passante per $u=e^{i\theta}$.
Dunque $R_\theta$ è la trasformazione che rende l'angolo identificato
dal numero complesso unitario $v$ in un angolo adiacente a quello
identificato da $u$. Quindi la somma (geometrica) degli angoli
$u$ e $v$ è identificata dal numero complesso unitario
$R_\theta(v) = u\cdot v$.
Ma, per la proprietà additiva dell'esponenziale complesso,
\[
 u\cdot v = e^{i\theta} \cdot e^{i\phi} = e^{i(\theta+\phi)}
\]
e dunque se $\theta+\phi<2\pi$ si ha
\[
 \arg(u\cdot v) = \theta + \phi = \arg(u) + \arg(v)
\]
che corrisponde alla proprietà additiva della misura degli angoli.

L'angolo unitario identificato dal numero complesso $e^i$
si chiama \emph{radiante}%
\mynote{radiante}%
\index{radiante} ed è l'unità di misura che abbiamo scelto
per gli angoli.
L'angolo retto sarà identificato dal numero complesso $i=e^{i\frac \pi 2}$
e avrà una misura pari a $\frac \pi 2$ radianti. L'angolo piatto sarà identificato
dal numero complesso $-1 = e^{i\pi}$ e avrà una misura di $\pi$ radianti.
Geometricamente la misura degli angoli può essere data dalla lunghezza
dell'arco di raggio unitario identificato dall'angolo. Dunque la definizione
geometrica di $\pi$ (rapporto tra lunghezza della circonferenza e diametro)
corrisponde a richiedere che la semicirconferenza unitaria abbia misura
$\pi$ radianti.
Questo significa che la definizione analitica di $\pi$ che abbiamo
dato (teorema~\ref{th:pi}) si riconcilia con la definizione geometrica.

Possiamo riconciliare definizione analitica e geometrica di $\pi$ anche
con delle osservazioni dirette.

\begin{remark}[lunghezza della circonferenza tramite moto circolare uniforme]
Si considera la curva $t\mapsto e^{it}$
come l'equazione oraria del moto di un punto
che si muove nel piano complesso. Visto che $\abs{e^{it}}=1$
tale punto si muove sulla circonferenza unitaria.
Possiamo determinare la velocità istantanea del punto considerando
la variazione della posizione:
\[
  \frac{e^{i(t+\Delta t)}-e^{it}}{\Delta t}
  = e^{it}\cdot \frac{e^{i\Delta t}-1}{\Delta t}.
\]
Prendendo un incremento temporale $\Delta t = \eps /n$ con $n\to +\infty$
si osserva che il modulo della velocità è dato da
\[
 v = \lim_{n\to +\infty} \abs{e^{it}}\cdot \abs{\frac{e^{i\frac \eps n}-1}{\frac \eps n}}
\]
e visto che $\abs{e^{it}}=1$ ricordando il limite notevole~\eqref{eq:limite_exp_complesso}
(teorema~\ref{th:exp_complesso}) si ottiene che la velocità è pari ad $1$.
Significa che il punto $e^{it}$ si muove sulla circonferenza unitaria
con velocità unitaria e quindi la lunghezza della curva percorsa risulta
numericamente uguale al tempo trascorso.
Visto che per $t$ che varia da $0$ a $2\pi$
il punto compie un giro completo attorno alla circonferenza unitaria
significa che la lunghezza della circonferenza unitaria è $2\pi$.
\end{remark}

\begin{remark}[lunghezza della circonferenza tramite approssimazione con poligonali]
\label{rem:pi}
Possiamo
calcolare la lunghezza della circonferenza unitaria come il limite dei perimetri dei poligoni
di $N$ lati iscritti nella circonferenza.
Fissato $N$
consideriamo per $k=0, \dots, N$ i punti
\[
  u_k = e^{i\frac{2\pi k}{N}}.
\]
Osserviamo che
\[
  \abs{u_{k+1}-u_k}
  = \abs{e^{i\frac{2\pi (k+1)} N}-e^{i\frac{2\pi k} N}}
  = \abs{e^{i\frac{2\pi k}{N}}} \cdot \abs{e^{i\frac{2\pi}{N}}-1}
  = \abs{e^{i\frac{2\pi}{N}}-1}
\]
cioè i punti $u_k$ sono equidistanti tra loro.
Si noti che $u_0=u_N=1$
e quindi i punti $u_1,\dots, u_N$ sono gli $N$ vertici
di un poligono regolare di $N$ lati iscritto nella
circonferenza unitaria. Il perimetro del poligono è quindi
dato da
\[
P_N = N \cdot \abs{e^{i\frac{2\pi}{N}}-1}
\]
e per $N\to +\infty$ si ha,
sempre utilizzando il limite notevole~\eqref{eq:limite_exp_complesso}
\[
P_N = 2 \pi \cdot \abs{\frac{e^{i\frac{2\pi}{N}}-1}{\frac{2\pi}{N}}}
  \to 2\pi.
\]
\end{remark}

\begin{remark}[matrici di rotazione]
Fissato $\theta\in \RR$
consideriamo, come prima, la funzione $R_\theta\colon \CC \to \CC$,
$R_\theta(z) = e^{i\theta}\cdot z$.
Un altro modo per convincerci che $R_\theta$
rappresenta una rotazione di $\theta$ radianti è
quello di guardare la matrice associata.
Se identifichiamo il piano complesso $\CC$ con $\RR^2$ la trasformazione
$R_\theta$ può essere rappresentata
da una matrice $M_\alpha$ che ha come
colonne le coordinate di $R_\theta(1) = e^{i\theta} = \cos \theta + i \sin \theta$
e le coordinate di $R_\theta(i) = e^{i\theta}\cdot i = -\sin \theta + i \cos \theta$:
\[
  M_\alpha =
  \begin{pmatrix}
  \cos \alpha & -\sin \alpha \\
  \sin \alpha & \cos \alpha
  \end{pmatrix}.
\]
\end{remark}

\begin{remark}[interpretazione geometrica del prodotto di numeri complessi]
Possiamo ora dare una interpretazione geometrica del prodotto tra
due numeri complessi $z,w\in \CC$.
Se $z\neq 0$ possiamo scrivere $z = \abs{z} \cdot e^{i\theta}$
con $\theta= \arg z$, cosicché:
\[
  z \cdot w = \abs{z} \cdot R_\theta(w).
\]
Si capisce quindi che il numero complesso $z\cdot w$ si ottiene ruotando
$w$ dell'angolo identificato da $z$ con l'asse dei reali positivi, e quindi
riscalando il punto ottenuto di un fattore $\abs{z}$.
Se poniamo $\psi = \arg w$ possiamo interpretare la moltiplicazione complessa
in coordinate polari:
\[
  z \cdot w = \abs{z}e^{i\theta}\cdot \abs{w} e^{i\psi}
   = \abs{z} \cdot \abs{w} \cdot e^{i(\theta + \psi)}.
\]
Dunque il prodotto di due numeri complessi è quel numero complesso
che ha come modulo il prodotto dei moduli e come argomento
la somma (a meno di multipli di $2\pi$) degli argomenti.
\end{remark}


\begin{remark}[interpretazione geometrica dell'esponenziale complesso]
Ricordiamo che (teorema~\ref{th:limite_notevole_esponenziale_complesso})
\[
  e^{iy} = \lim_{n\to +\infty} \enclose{1+\frac{iy}{n}}^n.
\]

Possiamo osservare che se $y\in \RR$
i punti $(1+iy/n)^k$ per $k=1\dots n$
sono i vertici di una spezzata
formata da $n$ segmenti
di lunghezza
\begin{align*}
 \abs{\enclose{1+\frac{iy}n}^{k+1}\!\!\! - \enclose{1+ \frac{iy}n}^k}
 &= \abs{\enclose{1+\frac{iy}n}^k\cdot \enclose{1+\frac{iy}n -1}}\\
 &= \enclose{\sqrt{1+\frac{y^2}{n^2}}}^{\!\!k} \cdot \frac{\abs y}{n}
 \le \enclose{1+\frac{y^2}{n^2}}^{\frac n 2}\cdot \frac{\abs{y}}{n}.
\end{align*}
In particolare la lunghezza totale della spezzata $\ell_n$ può essere stimata
come segue
\[
  \abs{y}
  \le \ell_n
  \le \enclose{1+\frac{y^2}{n^2}}^{\frac n 2} \cdot \abs{y}
\]
da cui osservando che
\[
 \enclose{1+\frac{y^2}{n^2}}^{\frac n 2} \to 1
\]
e utilizzando il criterio del confronto
si ottiene $\ell_n \to \abs{y}$.

Si osserva anche che i punti di tale spezzata si avvicinano
sempre di più alla circonferenza unitaria, infatti:
\[
  1
  \le \abs{\enclose{1 + \frac i n}^k}
  \le \abs{1+\frac i n}^n
  = \enclose{1 + \frac 1 {n^2}}^{\frac n 2}
  \to 1.
\]

E' dunque sensato pensare che il punto $e^{iy}$ sia il punto
della circonferenza unitaria che identifica un arco di lunghezza $\abs y$
a partire dal punto $1$ sull'asse reale.
Se $y>0$ l'arco è misurato in senso antiorario, altrimenti in senso orario.
Avremo dunque $\arg\enclose{e^{iy}} = y$ essendo $y$ la lunghezza dell'arco
ovvero la misura in radianti dell'angolo corrispondente.
\end{remark}

\section{radici complesse $n$-esime}

Sia $c\in \CC$ un numero
complesso $c\neq 0$.
Ci poniamo il problema di determinare le soluzioni complesse
dell'equazione
\[
  z^n = c.
\]
Tali soluzioni saranno chiamate \emph{radici $n$-esime}%
\mynote{radici $n$-esime}%
\index{radice!$n$-esima} di $c$.

Scriviamo $c$ e $z$ in forma esponenziale:
\[
  c = r e^{i\alpha}, \qquad
  z = \rho e^{i\theta}.
\]
Si avrà allora
\[
  z^n = \rho^n (e^{i\theta})^n = \rho^n e^{i n \theta}.
\]
Affinche sia $z^n = c$ si dovrà avere l'uguaglianza dei moduli, cioè $\rho^n = r$ e l'uguaglianza a meno di multipli interi di $2\pi$ degli argomenti:
$n \theta = \alpha + 2 k \pi$ con $k\in \ZZ$.
Dunque si trova
\[
  \theta = \frac{\alpha}{n} + k\frac{2\pi}{n}
\qquad k \in \ZZ.
\]
Osserviamo ora che per $k=0,\dots, n-1$ il secondo addendo
$k 2\pi /n$ assume $n$ valori distinti compresi in $[0,2\pi)$.
Per gli altri valori di $k$ si ottengono degli angoli che differiscono
da questi di un multiplo di $2\pi$ e quindi non si trovano
altre soluzioni.

Dunque l'equazione $z^n = c$ per $c\neq 0$ ha $n$ soluzioni distinte date
da
\[
z_k = \sqrt[n]{r} \cdot e^{i\alpha/n + 2k\pi i /n},
\qquad k=0,1, \dots, n-1
\]
dove $\alpha = \arg(c)$ e $r = \abs{c}$.
Dal punto di vista geometrico si osserva che
$z_0$ è il numero complesso con modulo la radice $n$-esima del numero
dato $c$ e argomento pari ad un $n$-esimo dell'argomento di $c$.
Tutte le altre soluzioni si trovano sulla circonferenza centrata in $0$
e passante per $z_0$ e risultano essere, insieme ad $z_0$, i vertici
di un $n$-agono regolare.

In particolare nel caso $c=1$ si osserva che le radici $n$-esime dell'unità
si rappresentano geometricamente come i vertici dell'$n$-agono regolare iscritto
nella circonferenza unitaria e con un vertice in $z_0=1$.

\begin{exercise}
Si trovino le soluzioni $z \in \CC$ delle seguenti equazioni.
Scrivere le soluzioni in forma polare e cartesiana.
\begin{gather*}
   z^4 = -4 \\
   z^6 = i\\
   z^3 = -8i \\
   z^4 = z\\
   z^2 + 1 = i\sqrt{3} \\
   (z-i)^4 = 1\\
   1 + z + z^2 + z^3 = 0\\
   z^{14} - z^6 - z^8 + 1 = 0
\end{gather*}
\end{exercise}

\section{ancora polinomi}
\label{ch:ancora_polinomi}
Proseguiamo la trattazione dei polinomi 
che abbiamo iniziato nel capitolo~\ref{ch:polinomi}.

\begin{theorem}[divisione di Euclide]
  \label{th:divisione_polinomi}%
  \index{divisione tra polinomi}%
  \index{polinomio!divisione}%
  \index{polinomio!algoritmo di Euclide}%
  \index{Euclide!algoritmo di divisione}%
  \index{algoritmo!di Euclide}%
  Se $\KK$ è un campo,
  dati due polinomi $P$ e $S$ in $\KK[x]$ con $S\neq 0$ è possibile
  trovare, in modo unico, due polinomi $Q$ (quoziente)
  e $R$ (resto) in $\KK[x]$ con $\deg R < \deg S$
  tali che:
  \[
    P = Q \cdot S + R.
  \]
  \end{theorem}
  %
  \begin{proof}
  \emph{Passo 1:} supponiamo che sia $\deg P < \deg S$.
  In questo caso basta prendere $Q=0$ e $R=P$.
  
  \emph{Passo 2:} supponiamo che sia $\deg P \ge \deg S$.
  Poniamo $N=\deg P$ e $M=\deg S$.
  Sia $a_N\neq 0$ il coefficiente del termine di grado massimo
  di $P$ e $b_M\neq 0$ il coefficiente di grado massimo
  del polinomio $S$.
  E' allora facile verificare che il polinomio
  \[
  \frac{a_N}{b_M} \cdot x^{N-M}\cdot S
  \]
  ha lo stesso grado di $P$ e il suo coefficiente di grado
  massimo è uguale ad $a_N$ in quanto è il prodotto di
  $a_N/b_M$ per $b_M$.
  Dunque il polinomio
  \[
   P_1 = P - \frac{a_N}{b_M} \cdot x^{N-M}\cdot S
  \]
  ha grado strettamente inferiore a $P$.
  Possiamo ora supporre, mediante un ragionamento induttivo
  su $\deg P - \deg S$
  che per il polinomio $P_1$ il risultato del teorema sia
  valido cioè
  che esistano, dei polinomio $Q_1$ e $R$
  con $\deg R < \deg S$ tali che
  \[
    P_1 = Q_1 \cdot S + R.
  \]
  Il caso base del ragionamento induttivo,
  $\deg P = \deg S$, è garantito dal passo 1.
  Avremo allora:
  \begin{align*}
    P &= P_1 + \frac{a_N}{b_M} x^{N-M}\cdot S\\
      &= Q_1 \cdot S + R_1 + \frac{a_N}{b_M} x^{N-M}\cdot S\\
      &= (\frac{a_N}{b_M} x^{N-M} + Q_1) \cdot S + R_1.
  \end{align*}
  Posto quindi
  \[
   Q = \frac{a_N}{b_N} x^{N-M} + Q_1
  \]
  il risultato è dimostrato.
  
  \emph{Passo 3:} dimostriamo che $Q$ e $R$ sono unici. Se infatti avvessimo 
  \[
    P = Q_1 S + R_1 = Q_2 S + R_2   
  \]
  con $\deg R_1<\deg S$ e $\deg R_2 < \deg S$ si avrebbe 
  \[
  (Q_2 - Q_1) S = R_1 - R_2.    
  \]
  Se fosse $Q_2 \neq Q_1$ il polinomio al lato sinistro avrebbe grado non inferiore al grado di 
  $S$ mentre il lato destro ha certamente grado minore di $S$.
  Dunque $Q_2=Q_1$ ma allora il lato sinistro è nullo e quindi anche il lato destro deve 
  esserlo: $R_2=R_1$.
  \end{proof}
  
  La dimostrazione del teorema precedente fornisce anche un
  algoritmo, chiamato \emph{algoritmo di Euclide},
  per eseguire la divisione (con resto) tra polinomi.
  Lo sperimentiamo nel seguente esercizio.
  
  \begin{exercise}
  Sia $P = x^4-3 x^2 + 2x + 1$ e $S = x^2-1$.
  Eseguire la divisione con resto cioè:
  trovare $Q$ ed $R$ con $\deg R < \deg S$ tali che
  \[
  P = Q \cdot S + R.
  \]
  \end{exercise}
  %
  \begin{proof}[Svolgimento.]
  Il rapporto tra i termini di grado massimo di
  $P$ e $S$ è $x^4/x^2 = x^2$.
  Dunque consideriamo come primo monomio $x^2$.
  Si ha
  \begin{align*}
    P_1
    &= P - x^2 \cdot S
    = x^4-3x^2+2x+1 - x^4+x^2 \\
    &= -2x^2+2x+1.
  \end{align*}
  Ripetiamo il procedimento con $P_1$ al posto di $P$.
  Il rapporto tra i termini di grado massimo di $P_1$ e $S$
  è il monomio $-2$. Si ha
  \[
    P_2 = P_1 - (-2) S = -2x^2 + 2x + 1 + 2x^2 - 2 = 2x -1.
  \]
  Visto che $\deg P_2 < \deg S$ la divisione termina e
  si pone $R = 2x-1$.
  Si ha quindi $Q = x^2 - 2$ (la somma dei monomi trovati)
  e risulta:
  \[
    P = (x^2 - 2)\cdot S + 2x -1.
  \]
  \end{proof}
  
  \begin{theorem}[Ruffini]
  \label{th:Ruffini}%
  \index{teorema!di Ruffini}%
  \index{Ruffini}%
  Sia $P\in \KK[x]$ un polinomio non nullo.
  Se $x_0 \in \KK$ è tale che $P(x_0)=0$
  allora esiste un polinomio $Q$ con $\deg Q = (\deg P) - 1$
  tale che
  \[
    P = (x-x_0)\cdot Q.
  \]
  \end{theorem}
  %
  \begin{proof}
  In base al teorema precedente si può fare la divisione tra
  $P$ e $S = x - x_0$ per ottenere un polinomio $Q$ e un resto
  $R$ con $\deg R < 1$ tali che
  \[
    P = (x-x_0)\cdot Q + R.
  \]
  Siccome $\deg R < 1$ si ha in effetti che $R=c\in \KK$
  è un polinomio costante e dunque
  \[
    P = (x-x_0) \cdot Q + c
  \]
  ma valutando $P$ in $x=x_0$ si scopre che
  \[
   P(x_0) = 0\cdot Q(x_0) + c = c
  \]
  dunque $c=P(x_0) = 0$.
  \end{proof}
  
  Possiamo ora chiederci se polinomi diversi corrispondono
  a funzioni polinomiali diverse.
  In generale questo non è vero, se ad esempio prendessimo
  come campo $\KK=\ZZ_2 = \ENCLOSE{0,1}$, un campo finito
  in cui poniamo $1+1=0$ scopriremmo che la funzione polinomiale
  $P(x) = x^2+x$ è identicamente nulla in quanto $0^2+0=0$
  e $1^2+1=1+1=0$ in questo campo.
  Ma nei casi che interessano a noi $\KK=\RR$ o $\KK=\CC$
  questo problema non si presenta e si potrà quindi identificare
  ogni polinomio con la corrispondente funzione polinomiale.
  Per avere questa garanzia ci servono i seguenti risultati.
  
  \begin{theorem}[principio di annullamento dei polinomi]
  \label{th:annullamento_polinomi}%
  \index{principio!di annullamento dei polinomi}%
  Sia $P\in \KK[x]$ un polinomio non nullo di grado $n$.
  Allora la funzione polinomiale associata a $P$
  si annulla in al più $n$ punti distinti di $\KK$.
  
  In particolare se un polinomio si annulla in infiniti
  punti distinti allora tale polinomio è certamente nullo.
  \end{theorem}
  %
  \begin{proof}
  Dimostriamo per induzione su $n$ che se un polinomio
  $P$ di grado non superiore a $n$ si annulla in $n+1$
  punti distinti $x_1, \dots, x_{n+1}\in \KK$ allora
  $P=0$.
  
  Se $n=0$ il polinomio $P$ è costante ma si annulla
  in un punto e quindi è il polinomio nullo.
  Se $n>0$ per il teorema di Ruffini applicato al
  punto $x_{n+1}$ sappiamo che esise un polinomio $Q$
  di grado inferiore ad $n$ per cui si ha
  \[
    P = (x-x_{n+1}) \cdot Q
  \]
  sostituendo $x=x_k$ con $k\le n$ si ha $x_k-x_{n+1}\neq 0$
  e quindi
  \[
    Q(x_k) = \frac{P(x_k)}{x_k-x_{n+1}} = 0.
  \]
  Dunque $Q$ si annulla nei punti $x_1, \dots, x_n$
  e, per ipotesi induttiva, scopriamo che $Q$ deve
  essere nullo. Di conseguenza anche $P$ è nullo.
  \end{proof}
  
  Se $\KK$ è un insieme infinito (come nei casi $\KK=\RR$ o $\KK=\CC$)
  si trova che
  se a due polinomi $P$, $Q$ corrisponde
  la stessa funzione polinomiale:
  \[
    P(x) = Q(x) \qquad \text{per ogni $x\in \KK$}
  \]
  allora $P$ e $Q$ sono lo stesso polinomio $P=Q$
  in quanto la differenza $P-Q$ si annulla in tutti i punti
  di $\KK$ e qualunque sia il suo grado, se $\KK$ è infinito,
  questo ci dice che $P-Q=0$. 
  
  Questo corollario viene spesso enunciato come segue.
  \begin{theorem}[principio di identità dei polinomi]
    Siano $a_k$ e $b_k$ coefficienti nel campo $\KK=\RR$ o $\KK=\CC$.
    Se per ogni $x\in \KK$ si ha 
    \[
       \sum_{k=1}^n a_k x^k = \sum_{k=1}^m b_k x^k
    \]
    allora $m=n$ e $a_k=b_k$ per ogni $k=0, \dots, n$.
  \end{theorem}
  %
  \begin{proof}
  Facendo la differenza dei due lati dell'uguaglianza si ottiene 
  che una espressione polinomiale si annulla identicamente. 
  Allora per il teorema~\ref{th:annullamento_polinomi} tale espressione 
  rappresenta il polinomio nullo e di conseguenza tutti i coefficienti, 
  che sono dati dalla differenza $a_k-b_k$, devono essere nulli.
  \end{proof}
  
\section{il teorema fondamentale dell'algebra}

Il teorema fondamentale dell'algebra afferma che ogni polinomio non costante 
si annulla in almeno un punto del piano complesso.

La formula risolutiva per determinare le radici di un polinomio di secondo grado 
era già nota ai Babilonesi e sebbene i numeri complessi non erano noti 
(e neanche i numeri negativi) la stessa formula si applica 
nel caso generale e fornisce le radici complesse.

Lo studio delle equazioni di grado superiore al secondo si sviluppa invece 
nel 1500 ad opera dei matematici italiani Scipione del Ferro, 
Gerolamo Cardano, Niccolò Tartaglia e Lodovico Ferrari.
Essi determinano le formule risolutive per determinare le radici 
dei polinomi di terzo e quarto grado. 
Nell'applicare la formula risolutiva per l'equazione di terzo grado 
può capitare di dover calcolare la radice quadrata di un numero negativo
anche in situazioni in cui tutte le soluzioni alla fine risultano reali.
Ed è proprio in questo contesto che nascono i numeri complessi, come 
mero artificio matematico utilizzato per portare a termine il conto.

Successivamente Paolo Ruffini (1765-1822) e Niels Abel (1802-1829) dimostrarono che le equazioni 
di grado maggiore del quarto non ammettono una formula risolutiva 
esprimibile mediante radicali. 
Il criterio per determinare se un polinomio
ammette o meno formule risolutive è dovuto al matematico francese Évariste Galois
(1811-1832) che è considerato il fondatore della teoria dei gruppi.

Il teorema fondamentale dell'algebra, e cioè l'esistenza di radici 
complesse per polinomi di grado qualunque, viene dimostrato dal 
matematico tedesco Carl Friedrich Gauss (1777--1855). 
Questo teorema opera una svolta nel pensiero matematico in quanto 
per la prima volta si dà rilevanza ad un risultato astratto di esistenza 
slegato da una formula risolutiva. 
Il teorema non era affatto scontato, basti pensare che sia Leibniz 
che Nikolas Bernoulli erano convinti di aver trovato dei polinomi 
di quarto grado che non possono essere fattorizzati in contrasto 
con il teorema~\ref{th:fattorizzazione_polinomio_reale}.

Per dimostrare il teorema fondamentale dell'algebra dobbiamo estendere il teorema 
di Weierstrass alle funzioni di una variabile complessa.
Nel teorema di Weierstrass reale la funzione per ipotesi è definita su un intervallo
chiuso e limitato. 
Nel piano complesso non esiste il concetto di \emph{intervallo} in quanto non abbiamo 
un ordinamento ma vedremo che comunque il teorema di Weierstrass rimane valido per 
le funzioni continue definite su insiemi chiusi e limitati secondo le seguenti definizioni.

\begin{definition}[chiusura sequenziale]
Un insieme $A\subset \CC$ si dice
essere \emph{sequenzialmente!chiuso}%
\mynote{sequenzialmente!chiuso}%
\index{sequenzialmente!chiuso}%
se presa una qualunque successione
di punti $a_n\in A$ se $a_n \to a$ per qualche $a\in \CC$
allora $a\in A$.
\end{definition}

\begin{definition}[limitatezza]
Un insieme $A\subset \CC$ si dice essere \emph{limitato}%
\mynote{limitato}%
\index{limitato}
se
\[
  \sup \ENCLOSE{ \abs{z}\colon z \in A} < +\infty.
\]

Una successione $z_n\in \CC$ si dice essere limitata se l'insieme
$\ENCLOSE{z_n\colon n\in \NN}$ è limitato ovvero se
\[
  \sup_{n\in \NN} \abs{z_n} < +\infty.
\]
\end{definition}

\begin{comment} 
  %% il concetto di disco non viene piu' utilizzato nella dimostrazione 
  %% del teorema fondamentale dell'algebra.
\begin{definition}[disco]
Dato $R\ge 0$ si può definire il disco complesso di raggio $R$ come
l'insieme $D_R\subset \CC$ definito da
\[
  D_R = \ENCLOSE{z\in \CC\colon \abs{z} \le R}.
\]
Geometricamente si tratta di un cerchio pieno di raggio $R$ centrato in $0$.
\end{definition}

\begin{theorem}[il disco è chiuso e limitato]
Per ogni $R\in [0,+\infty)$ il disco $D_R$ è un sottoinsieme di $C$ non vuoto, chiuso e limitato.
\end{theorem}
%
\begin{proof}
Per ogni $R\ge 0$ si ha $0\in D_R$ e quindi $D_R$ non è mai vuoto.

Che $D_R$ sia limitato è pure ovvio,
in quanto dato $z\in D_R$ si ha per
definizione $\abs{z}\le R$ e dunque $\sup_{z\in D_R} \abs{z} = R < +\infty$.

Per dimostrare che $D_R$ è chiuso consideriamo una qualunque successione $a_n \in D_R$. Sappiamo dunque che $\abs{a_n} \le R$
cioè $R-\abs{a_n} \ge 0$ per ogni $n\in \NN$.
Per la continuità del modulo sappiamo che $R-\abs{a_n}\to R-\abs{a}$
e per il teorema della permanenza del segno possiamo concludere che $R-\abs{a}\ge 0$ cioè che $\abs{a}\le R$ ovvero $a \in D_R$. Come volevamo dimostrare.
\end{proof}
\end{comment}

\begin{theorem}[Bolzano-Weierstrass complesso]
Se $z_n\in \CC$ è una successione limitata allora
è possibile estrarre una sottosuccessione $z_{n_k}$ convergente:
$z_{n_k} \to z$ con $z\in \CC$.
\end{theorem}
%
\begin{proof}
Siano $x_n$ e $y_n$ la parte reale ed immaginaria di $z_n$: $z_n = x_n + i y_n$. Visto che $\abs{x_n} =\sqrt{x_n^2}\le \sqrt{x_n^2+y_n^2} = \abs{z_n}$ e, allo stesso modo $\abs{y_n} \le \abs{z_n}$,
possiamo affermare che entrambe le successioni $x_n$ e $y_n$ sono limitate (ma stavolta in $\RR$).
Dunque possiamo applicare il teorema di Bolzano-Weierstrass (reale) alla successione $x_n$ per trovare una sottosuccessione $x_{n_j}\to x$ convergente. E possiamo applicare di nuovo il teorema di Bolzano-Weierstrass alla sottosuccessione $y_{n_j}$ per trovare una sotto-sottosuccessione $y_{n_{j_k}}\to y$ anch'essa convergente.
Posto $n_k = n_{j_k}$ avremo dunque trovato una sottosuccessione $z_{n_k} = x_{n_k} + i y_{n_k} \to x+iy$ convergente.
\end{proof}

\begin{theorem}[Weierstrass complesso]
Sia $A\subset \CC$ un insieme non vuoto, sequenzialmente chiuso e limitato e sia $f\colon A \to \RR$ una funzione continua.
Allora $f$ ha massimo e minimo su $A$.
\end{theorem}
%
\begin{proof}
Dimostriamo l'esistenza del minimo: per il massimo la dimostrazione è perfettamente analoga.
Sia $m=\inf f(A)$.
Essendo $A$ non vuoto, per il lemma sull'esistenza delle successioni minimizzanti sappiamo esistere una successione $z_n \in A$ tale che $f(z_n) \to m$.
Essendo $A$ limitato possiamo applicare il teorema di Bolzano-Weierstrass per trovare $z\in \CC$ e una sottosuccessione $z_{n_k} \to z$. Essendo $A$ sequenzialmente chiuso possiamo quindi affermare che $z\in A$. Essendo $f$ continua concludiamo che
\[
f(z) = \lim_{k\to+\infty} f(z_{n_k}) = m
\]
e dunque $z$ è un punto di minimo per $f$.
\end{proof}

\begin{theorem}[esistenza del minimo per funzioni coercive]
Sia $f\colon \CC \to \RR$ una funzione continua tale che per ogni
successione $z_n \to \infty$ (ovvero $\abs{z_n}\to +\infty$)
si abbia $f(z_n) \to +\infty$.
Allora $f$ ha minimo.
\end{theorem}
%
\begin{proof}
Consideriamo l'insieme
\[
  A = \ENCLOSE{z \in \CC \colon f(z) \le f(0)}.
\]
Chiaramente $0\in A$ e quindi $A$ non è vuoto.
L'insieme $A$ è anche sequenzialmente chiuso in quanto se $z_k\in A$ allora $f(0) - f(z_k)\ge 0$,
per continuità $f(0)-f(z_k)\to f(0)-f(z)$
e per il teorema della permanenza del segno si ottiene $f(0)-f(z) \ge 0$ cioè $z \in A$.
Dimostriamo ora che $A$ è anche limitato. 
Se non lo fosse esisterebbe, per assurdo, una successione $z_n \in A$ tale che $\abs{z_n}\to +\infty$ cioè $z_n \to \infty$. 
Ma allora, per ipotesi su $f$, si avrebbe $f(z_n)\to +\infty$ che contraddice la condizione $f(z_n) \le f(0)$. 
Essendo $A$ non vuoto, sequenzialmente chiuso e limitato ed essendo $f\colon A \to \RR$ continua, 
possiamo applicare il teorema di Weierstrass complesso per dedurre che $f$ ha minimo su $A$ in un punto $w \in A$. 
Ma essendo $0\in A$ si avrà sicuramente $f(w)\le f(0)$ e per ogni $z\in \CC \setminus A$ si ha invece $f(z) > f(0)$ per come è stato definito $A$. 
Dunque $w$ è minimo di $f$ su tutto $\CC$, non solo su $A$.
\end{proof}

\begin{theorem}[teorema fondamentale dell'algebra]
\label{th:fondamentale_algebra}
\mynote{teorema fondamentale dell'algebra}%
\index{teorema!fondamentale dell'algebra}%
Sia $f(z)$ un polinomio di grado $N\ge 1$ a coefficienti complessi:
\[
  f(z) = \sum_{j=0}^N a_j \cdot z^j
\]
con $a_j\in \CC$ per $j=0,\dots,N$ e $a_N \neq 0$.
Allora esiste $w\in \CC$ tale che $f(w) = 0$.
\end{theorem}
%
\begin{comment}
  %% questa dimostrazione e' stata riscritta 
  %% in modo da risultare piu' diretta.
  %% se vuoi mostrare questa dimostrazione dovrai anche 
  %% mostrare la definizione di disco e la dimostrazione che i dischi sono chiusi e limitati

\begin{proof}
Osserviamo innanzitutto che $\abs{f(z)}$ è coerciva cioè che
se $z_n \to \infty$ allora $\abs{f(z_n)}\to +\infty$.
Infatti si ha
\begin{align*}
  \abs{f(z_n)}
  &= \abs{\sum_{j=0}^N a_j z_n^j}
  = \abs{a_N z_n^N  + \sum_{j=0}^{N-1} a_j z_n^j}\\
  &= \abs{z_n}^N \cdot \abs{a_N + \sum_{j=0}^{N-1} \frac{a_j}{z_n^{N-j}}}
  \to +\infty
\end{align*}
se $z_n \to \infty$.

Sappiamo che tutti i polinomi sono funzioni continue in quanto somme di prodotti di funzioni continue e il modulo è anch'esso una funzione continua dunque $\abs{f(z)}$ è certamente una funzione continua.

Dunque possiamo applicare il teorema di esistenza del minimo per le funzioni coercive: esiste $w\in \CC$ tale che $\abs{f(w)}$ è minimo.

Per concludere il teorema basterà dimostrare che $f(w)=0$.
L'idea che vogliamo sviluppare è che i polinomi complessi se assumono un valore
$f(w)$ in un punto $w\in \CC$ allora assumono anche tutti i valori vicini
ad esso in quanto \emph{localmente} il polinomio assomiglia ad una potenza $z^n$
e l'equazione $z^n=c$ ha sempre soluzione, come abbiamo già visto.
Dunque vicino a $w$ ci saranno dei punti in cui $f$ assume valori che in modulo sono minori a $f(w)$: a meno che non sia proprio $f(w)=0$, nel qual caso ovviamente non è possibile avere numeri con modulo inferiore a $0$.
Per semplificare la notazione andremo a traslare e riscalare il polinomio $f$ in modo che il punto di minimo vada in $0$ e il valore con modulo minimo diventi $1$.

Supponiamo per assurdo che sia $f(w)\neq 0$ e consideriamo il polinomio ausiliario
\[
  g(z) = \frac{f(w+z)}{f(w)}.
\]
Andremo a dimostrare che esiste uno $z\neq 0$ tale che $\abs{g(z)}<1$:
questo ci porterà all'assurdo in quanto si avrebbe
\[
\abs{f(w+z)} = \abs{f(w)} \cdot \abs{g(z)} < \abs{f(w)}
\]
e quindi $w$ non sarebbe un punto di minimo per $\abs{f}$.

Il polinomio $g$ si può scrivere, al solito, come somma di monomi
\[
  g(z) = b_0 + b_1 z + \dots + b_N z^N.
\]
Essendo $g(0)=1$ si ha $b_0=1$. Vicino a $z=0$ il comportamento del polinomio è dominato dai termini di grado più basso.
Sia $k\ge 1$ il primo indice per cui $b_k\neq 0$. Osserviamo che tale $k$ esiste perché se tutti i coefficienti $b_k$ fossero nulli per $k\ge 1$
allora
il polinomio $g$ sarebbe costante e allora anche $f$ sarebbe costante, cosa che abbiamo escluso richiedendo per ipotesi che $f$ abbia grado $N\ge 1$.
Il polinomio $g$
si potrà dunque scrivere nella forma:
\begin{align*}
  g(z) &= 1 + b_k z^k + b_{k+1} z^{k+1}+\dots + b_N z^n \\
       &= 1 + b_k z^k + z^{k+1}\enclose{b_{k+1} + b_{k+2}z + \dots + b_N
       z^{n-k-1}}\\
       &= 1 + b_k z^k + z^{k+1} \cdot q(z)
\end{align*}
dove $q(z)$ è un polinomio di grado $n-k-1$.

Se scriviamo $b_k$ e $z$ in forma esponenziale:
\[
  b_k = r e^{i\alpha}, \qquad
  z = \rho e^{i\theta}
\]
scegliendo $\theta = (\pi - \alpha)/k$ otteniamo che $b_k z^k$ sia un numero reale negativo, in particolare:
\begin{align*}
  \abs{g(\rho e^{i\theta})}
    &= \abs{1 + r e^{i\alpha} \rho^k e^{ik\theta}
    +  \rho^{k+1} e^{i(k+1)\theta} q (\rho e^{i\theta})} \\
    & \le  \abs{1 + r \rho^k e^{i\pi}} + \rho^{k+1} \abs{q(\rho e^{i\theta})} \\
    &= \abs{1 - r \rho^k} + \rho^{k+1} \abs{q(\rho e^{i\theta})}.
\end{align*}

Essendo $q(z)$ una funzione continua sappiamo, per il teorema di Weierstrass complesso, 
che $\abs{q(z)}$ ha massimo $M<+\infty$
su $D_1$.
Ovvero per ogni $z\in \CC$ con $\abs{z} \le 1$ si ha $\abs{q(z)} \le M$.
In particolare nel nostro caso $\abs{z}=\rho$ e quindi se $\rho \le 1$ possiamo affermare che $\abs{q(\rho e^{i\theta})} \le M$.

Dunque, proseguendo la stima fatta in precedenza, si ha, per ogni $\rho \le 1$
\[
\abs{g(\rho e^{i\theta})} \le \abs{1-r \rho^k} + M\cdot \rho^{k+1}.
\]
Se ora imponiamo anche che sia $\rho < 1/\sqrt[k]{r}$ possiamo togliere il valore assoluto e
richiedendo inoltre che sia $\rho < r/M$ (ricordiamo che $r>0$ in quanto $b_k \neq 0$) si ottiene
\[
\abs{g(\rho e^{i\theta})}
\le 1-r \rho^k + M\cdot \rho^{k+1}
< 1 - r \rho^k + r \rho^k = 1.
\]
E' dunque possibile determinare un valore di $\rho$ abbastanza piccolo, ma non nullo,
in modo che posto $z= \rho e^{i\theta}$, si abbia $\abs{g(z)} < 1$ e la dimostrazione è completata.
\end{proof}
\end{comment}
\begin{proof}
  Osserviamo innanzitutto che $\abs{f(z)}$ è coerciva cioè che
  se $z_n \to \infty$ allora $\abs{f(z_n)}\to +\infty$.
  Infatti si ha
  \begin{align*}
    \abs{f(z_n)}
    &= \abs{\sum_{j=0}^N a_j z_n^j}
    = \abs{a_N z_n^N  + \sum_{j=0}^{N-1} a_j z_n^j}\\
    &= \abs{z_n}^N \cdot \abs{a_N + \sum_{j=0}^{N-1} \frac{a_j}{z_n^{N-j}}}
    \to +\infty
  \end{align*}
  se $z_n \to \infty$.
  
  Sappiamo che tutti i polinomi sono funzioni continue in quanto somme di 
  prodotti di funzioni continue e il modulo è anch'esso una funzione continua 
  dunque $\abs{f(z)}$ è certamente una funzione continua.
  
  Dunque possiamo applicare il teorema di esistenza del minimo per le funzioni 
  coercive: esiste $w\in \CC$ tale che $\abs{f(w)}$ è minimo.
  
  Per concludere il teorema basterà dimostrare che $f(w)=0$.
  L'idea che vogliamo sviluppare è che i polinomi complessi se assumono un valore
  $f(w)$ in un punto $w\in \CC$ allora assumono anche tutti i valori vicini
  ad esso in quanto \emph{localmente} il polinomio assomiglia ad una potenza $z^n$
  e l'equazione $z^n=c$ ha sempre soluzione, come abbiamo già visto.
  Dunque vicino a $w$ ci saranno dei punti in cui $f$ assume valori che in modulo 
  sono minori a $f(w)$: a meno che non sia proprio $f(w)=0$, nel qual caso 
  ovviamente non è possibile avere numeri con modulo inferiore a $0$.
  
  Possiamo scrivere il polinomio $f(z)$ nella forma
  \[
    f(z) = \sum_{j=0}^N b_j (z-w)^j
  \]
  in quanto la traslazione di un polinomio è ancora un polinomio dello stesso 
  grado.
  Dimostrare che $f(w)=0$ è quindi equivalente a dimostrare che $b_0=0$.
  Supponiamo allora per assurdo che sia $b_0\neq 0$. 
  L'andamento del polinomio vicino al punto $w$ è dominato dai termini di grado 
  più basso: alcuni di questi potrebbero essere nulli, ma possiamo considerare 
  il primo indice $k>0$ per cui si ha $b_k\neq 0$. Allora potremo scrivere:
  \[
      f(z) = b_0 + b_k(z-w)^k + \sum_{j=k+1}^N b_j (z-w)^j.
  \]
Ora scriviamo $z-w$ e $b_0$ in forma esponenziale:
$z - w = \rho e^{i\theta}$, $b_0 = re^{i \alpha}$ 
e applichiamo la disuguaglianza triangolare
  \begin{align*}
    \abs{f(z)} 
    &= \abs{r e^{i\alpha} + \rho^k e^{i\theta k} + \sum_{j=k+1}^N b_j \rho^j e^{i\theta j}} \\
    &\le \abs{r e^{i\alpha} + \rho^k e^{i\theta k}} + \sum_{j=k+1}^N \abs{b_j} \rho^j.
  \end{align*}
Per raggiungere un assurdo vogliamo dimostrare che esiste $z$ (cioè esistono $\rho$ e $\theta$)
per cui il valore della funzione risulti in modulo minore di $r = \abs{b_0} = \abs{f(w)}$.
Innanzitutto se $\rho$ è sufficientemente piccolo possiamo rendere arbitrariamente piccole 
le potenze $\rho^j$ con $\rho>k$: basta scegliere $\rho\le 1$ 
e $\rho \le 1/(2\sum \abs{b_j})$
cosicché si avrà 
\[
  \sum_{j=k+1}^N \abs{b_j} \rho^j
  \le \rho^{k+1} \sum_{j=k+1}^N \abs{b_j} \le \frac{\rho^{k}}{2}.
\]
Per quanto riguarda il termine
$r e^{i\alpha} + \rho^k e^{i\theta k}$ 
sarà sufficiente scegliere $\theta = (\pi -\alpha) /k$ in modo che i due addendi 
abbiano fase opposta e la somma sia distruttiva:
\[
  \abs{r e^{i\alpha} + \rho^k e^{i\theta k}}
  = \abs{e^{i\alpha}(r + \rho^k r^{i\pi})}
  = r - \rho^k.
\]
In definitiva abbiamo
\[
\abs{f(z)} = \abs{f(w+\rho e^{i\theta}} 
\le r - \rho^k + \frac{\rho^k}{2} 
< r = \abs{f(w)}
\]
che è assurdo in quanto $\abs{f(w)}$ doveva essere il valore minimo.
\end{proof}
  
\section{fattorizzazione dei polinomi}

\begin{theorem}[fattorizzazione dei polinomi complessi]
\index{decomposizione!dei polinomi complessi}%
\index{fattorizzazione!dei polinomi complessi}%
\index{polinomio!fattorizzazione complessa}%
Sia $p(z)$ un polinomio non nullo. Allora posto $n=\deg p$ esistono dei numeri complessi $z_1, z_2, \dots, z_n$ ed un numero complesso $c\neq 0$ tali che
\begin{equation}\label{eq:34985}
  p(z) = c \prod_{k=1}^n (z-z_k).
\end{equation}
Gli $z_k$ sono unici a meno dell'ordine e $c$ pure è univocamente determinato.
\end{theorem}
%
\begin{proof}
Dimostriamo il teorema per induzione su $n=\deg p$. Se $n=0$ il polinomio $p$ è costante: $p(z) = c$. Ricordando che un prodotto di $n=0$ fattori è uguale a $1$ si ottiene quindi il risultato voluto.

Sia ora $p(z)$ un qualunque polinomio di grado $n>0$. Per il teorema fondamentale dell'algebra sappiamo che esiste un numero complesso $z_n$ tale che $p(z_n)=0$. Per il teorema di Ruffini si ha allora
\[
  p(z) = (z-z_n) q(z)
\]
con $q$ un qualche polinomio di grado $n-1$. Per ipotesi induttiva possiamo dunque supporre che esistano $z_1, \dots, z_{n-1}$ e $c$ numeri complessi tali che
\[
   q(z) = c \prod_{k=1}^{n-1} (z-z_k)
\]
e la tesi segue.
\end{proof}

Nella fattorizzazione~\eqref{eq:34985} le \emph{radici} $z_k$ possono
anche ripetersi. Se mettiamo insieme i fattori corrispondenti alla stessa radice
si ottiene una decomposizione della forma
\begin{equation}\label{eq:358925}
P(z) = c \prod_{k=1}^m (z-z_k)^{p_k}
\end{equation}
con $z_1, \dots, z_m$ numeri complessi distinti (le radici del polinomio $P$)
e $p_k$ interi positivi.
L'esponente $p_k$ si chiama
\emph{molteplicità}%
\mynote{molteplicità}%
\index{molteplicità} della radice $z_k$ e risulta
\[
  p_1 + \dots + p_m = n.
\]
Questa uguaglianza si esprime dicendo che un polinomio $P\in \CC[z]$
di grado $n$ ha sempre $n$ radici contate con la loro molteplicità.
Le radici distinte sono invece $m\le n$.

Se invece $P\in \RR[x]$ è un polinomio a coefficienti reali
non è detto che $P$ abbia radici: ad esempio il
polinomio $x^2+1$ non ha radici reali in quanto $x^2+1>0$
come succede in ogni campo ordinato.
Potrà allora essere utile pensare a $P$ come ad un polinomio
in $\CC[x]$ con coefficienti reali.

Se $P\in \RR[x]$ è un polinomio a coefficienti reali
\[
  P = \sum_{k=0}^n a_k x^k, \qquad a_k\in \RR
\]
possiamo pensare a $P$ anche come un polinomio in
$\CC[x]$, visto che $a_k\in \RR\subset \CC$.
Il seguente teorema ci dà un criterio per distinguere
i polinomi a coefficienti reali dentro a $\CC[x]$.

\begin{theorem}
\label{th:caratterizzazione_polinomi_reali}%
Se $P\in \CC[x]$ è un polinomio a coefficienti complessi
\[
  P = \sum_{k=0}^n a_k x^k,\qquad a_k\in \CC
\]
e se esiste un insieme infinito $A\subset \RR$
tale che per ogni $x\in A$ si abbia $P(x)\in \RR$
allora $a_k\in\RR$ per ogni $k=0,\dots,n$ e dunque $P\in \RR[x]$.

In particolare se la funzione polinomiale associata a
$P\in\CC[x]$ manda $\RR$ in $\RR$ allora $P$ è un polinomio
a coefficienti reali.
\end{theorem}
%
\begin{proof}
Consideriamo il polinomio:
\[
  Q = \sum_{k=0}^n (a_k-\bar a_k) x^k.
\]
Allora per ogni $x\in A$ si ha
\[
  Q(x) = \sum_{k=0}^n a_k x^k - \sum_{k=0}^n \bar a_k x^k
     = P(x) - \overline{P(x)} = P(x) - P(x) = 0
\]
in quanto se $x\in \RR$ risulta
$\overline{x^k} = {\bar x}^k = x^k$.
Visto che $A$ è infinito,
per il principio di annullamento dei polinomi
(teorema~\ref{th:annullamento_polinomi})
deduciamo che $Q=0$.
Ma allora tutti i coefficienti di $Q$ devono
essere nulli e cioè $\bar a_k = a_k$.
Ne consegue che $a_k\in \RR$ per ogni $k=0,\dots,n$.
\end{proof}

Rifacendoci alla fattorizzazione dei polinomi a coefficienti
complessi possiamo fattorizzare anche i polinomi a coefficienti
reali se ci accontentiamo di avere fattori quadratici
invece che lineari.

\begin{theorem}[fattorizzazione dei polinomi reali]
  \label{th:fattorizzazione_polinomio_reale}%
Se $P\in \RR[x]$ è un polinomio a coefficienti reali
potremo scrivere
\begin{equation}\label{eq:35549}
  P = a \cdot \prod_{k=1}^n (x-x_k)^{p_k} \cdot \prod_{j=1}^m (x^2 + \alpha_j x + \beta_j)^{q_j}
\end{equation}
dove $a\in \RR$, $x_k\in \RR$, $p_k\in \NN\setminus\ENCLOSE{0}$,
$\alpha_j,\beta_j\in \RR$, $q_j\in \NN\setminus\ENCLOSE{0}$
con $\alpha_j^2 - 4 \beta_j < 0$
e
\[
  \sum_{k=1}^n p_k + 2 \sum_{j=1}^m q_j = \deg P.
\]

La fattorizzazione~\eqref{eq:35549} è unica a meno
dell'ordine dei fattori.

I numeri $x_1,\dots,x_n$ sono tutte le radici reali distinte
del polinomio $P$ con rispettive molteplicità
$p_1,\dots,p_n$ mentre tutte le radici complesse (non reali)
distinte saranno $\mu_1,\dots, \mu_m$
e $\bar \mu_1, \dots, \bar \mu_m$ con
\[
  x^2 + \alpha_j x + \beta_j = (x-\mu)\cdot(x-\bar \mu)
\]
da cui
\[
  \alpha_j = -2\Re \mu_j, \qquad \beta_j=\abs{\mu_j}^2
\]
e $q_j$ saranno le molteplicità delle radici $\mu_j$
 e $\bar \mu_j$.
\end{theorem}
%
\begin{proof}
Osserviamo innanzitutto che se $P$ è a coefficienti reali
si ha
\[
  \overline{P(z)} = P(\bar z), \qquad \forall z\in \CC
\]
dunque se $\mu$ è una radice di $P$ anche $\bar \mu$
è una radice di $P$.
Ora se $\mu$ è una radice non reale di $P$ sappiamo
che in campo complesso $P$ risulta divisibile
per $x-\mu$ (teorema~\ref{th:Ruffini} di Ruffini):
\[
  P = (x-\mu) \cdot P_1.
\]
Ma anche $\bar \mu$ è radice di $P$ ed essendo
$\bar \mu \neq \mu$ si dovrà avere
\[
Q(\bar \mu) = \frac{P(\bar \mu)}{\bar \mu - \mu} = 0
\]
e dunque applicando nuovamente il teorema di Ruffini
\[
 P = (x-\mu)\cdot (x-\bar \mu)\cdot P_2.
\]
Ora osserviamo che
\[
(x-\mu)\cdot(x-\bar \mu) = x^2 - (\mu + \bar \mu) x + \mu \bar \mu
 = x^2 - 2 (\Re \mu)\cdot x + \abs{\mu}^2
\]
è un polinomio a coefficienti reali e
visto che $P_2$ si ottiene dividendo $P$ per tale polinomio,
anche $P_2$ è un polinomio a coefficienti reali.

Ripetendo lo stesso procedimento sul polinomio $P_2$
potremo fattorizzare $P$ diminuendo il grado a passi
di $2$ finché non si esauriscono tutte le radici complesse
non reali accoppiandole a due a due.
Dunque se $\mu$ è una radice complessa
del polinomio reale $P$ la molteplicità di $\mu$ è
uguale alla molteplicità di $\bar \mu$.

Dopodiché
si potrà completare la fattorizzazione dividendo
per i fattori lineari $x-x_k$ corrispondenti
alle radici reali del polinomio $P$.
Si otterrà quindi la fattorizzazione desiderata.
\end{proof}

\include{chapters/chapter-05-derivate}
\include{chapters/chapter-06-integrali}
\include{chapters/chapter-07-spazi}
\include{chapters/chapter-08-ricorrenza}
\include{chapters/chapter-09-edo}

\appendix

\appendix
\chapter{Listati}

Il seguente codice è scritto in \myemph{python}, un linguaggio di programmazione
molto semplice e pulito che permette, tra l'altro, di utilizzare diverse librerie
utili per il calcolo numerico e scientifico.

\lstset{% general command to set parameter(s)
  basicstyle=\tiny, % print whole listing small
  keywordstyle=\color{black}\bfseries\underbar,
  % underlined bold black keywords
  identifierstyle=, % nothing happens
  commentstyle=\color{white}, % white comments
  stringstyle=\ttfamily, % typewriter type for strings
  showstringspaces=false} % no special string spaces

\section{series.py}

Vedi esempio~\ref{ex:52573}.
\myqrcode{https://github.com/paolini/AnalisiUno/blob/master/code/series.py}{github}{series.py}
\label{code:series}
\lstinputlisting{code/series.py}


\section{bisection.py}

Vedi esempio~\ref{ex:75445}.
\myqrcode{https://github.com/paolini/AnalisiUno/blob/master/code/bisection.py}{github}{bisection.py}
\label{code:bisection}
\lstinputlisting{code/bisection.py}

\newpage

\section{compute\_e.py}

Vedi tabella~\ref{fig:cifre_e}.
\myqrcode{https://github.com/paolini/AnalisiUno/blob/master/code/compute_e.py}{github}{compute_e.py}
\label{code:compute_e}
\lstinputlisting{code/compute_e.py}

\section{Mandelbrot.py}

Vedi figura~\ref{fig:mandelbrot}.
\myqrcode{https://github.com/paolini/AnalisiUno/blob/master/code/Mandelbrot.py}{github}{Mandelbrot.py}
\label{code:Mandelbrot}
\lstinputlisting{code/Mandelbrot.py}

\newpage

\section{compute\_pi.py}

Vedi osservazione~\ref{rem:cifre_pi}.
\myqrcode{https://github.com/paolini/AnalisiUno/blob/master/code/compute_pi.py}{github}{compute_pi.py}
\label{code:compute_pi}
\lstinputlisting{code/compute_pi.py}

\section{Koch.py}

Vedi figura~\ref{fig:koch}.
\myqrcode{https://github.com/paolini/AnalisiUno/blob/master/code/Koch.py}{github}{Koch.py}
\label{code:Koch}
\lstinputlisting{code/Koch.py}

\newpage
\section{Fourier.py}

Vedi figura~\ref{fig:fourier}
\myqrcode{https://github.com/paolini/AnalisiUno/blob/master/code/Fourier.py}{github}{Fourier.py}
\label{code:Fourier}
\lstinputlisting{code/Fourier.py}


\backmatter

\nocite{Giusti}
\nocite{Courant}
\nocite{Marcellini}
\nocite{Rudin}
\nocite{PaganiSalsa}
\nocite{appunti_logica}


\bibliography{biblio}{}
\bibliographystyle{plain}

\printindex

\end{document}
